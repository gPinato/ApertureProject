

\begin{longtable}{|c|p{6cm}|c|c|}
\caption{Requisiti}
\label{tab:Requisiti} \\
\toprule
\multicolumn{1}{|c}{\textbf{\underline{Requisito}}} & \multicolumn{1}{|p{6cm}}{\textbf{Descrizione}}   & \multicolumn{1}{|c}{\textbf{Fonte}} & \multicolumn{1}{|c|}{\textbf{Caso d'uso}}\\
\midrule
\endfirsthead
\multicolumn{2}{l}{\footnotesize\itshape\tablename~\thetable: continua dalla pagina precedente} \\
\toprule
\multicolumn{1}{|c}{\textbf{Requisito}} & \multicolumn{1}{|p{6cm}}{\textbf{Descrizione}}   & \multicolumn{1}{|c}{\textbf{Fonte}} & \multicolumn{1}{|c|}{\textbf{Caso d'uso}}\\
\midrule
\endhead
\midrule
\multicolumn{2}{r}{\footnotesize\itshape\tablename~\thetable: continua nella prossima pagina} \\
\endfoot
\bottomrule
\multicolumn{2}{r}{\footnotesize\itshape\tablename~\thetable: si conclude dalla pagina precedente} \\
\endlastfoot



\midrule
RFOb1
& Il software deve essere in grado di caricare una scena 3D descritta da un file con un formato particolare, modificarla secondo le richieste dell'utente e convertirla nel formato desiderato
& Capitolato
& UC1
\\


\midrule
RFOb1.1
& Il software deve permettere all'utente di selezionare un file da cui caricare la scena 3D
& Capitolato
& UC1.1
\\


\midrule
RFOb1.1.1
& Il software deve permettere all'utente di selezionare il formato del file da cui caricare la scena 3D
& Capitolato
& UC1.1.1
\\


\midrule
RFOb1.1.1.1
& Il software deve permettere all'utente di selezionare il formato 3DS
& Capitolato
& UC1.1.1.1
\\


\midrule
RFDe1.1.1.2
& Il software deve permettere all'utente di selezionare il formato Wavefront OBJ/MTL
& Capitolato
& UC1.1.1.2
\\


\midrule
RFDe1.1.1.3
& Il software deve permettere all'utente di selezionare il formato JSON
& Capitolato
& UC1.1.1.3
\\


\midrule
RFDe1.1.1.4
& Il software deve permettere all'utente di selezionare il formato XML
& Interno
& UC1.1.1.4
\\


\midrule
RFOb1.1.2
& Il software deve permettere di selezionare il percorso di un file di cui si conosce il formato
& Interno
& UC1.1.2
\\


\midrule
RFDe1.2
& Il software deve permettere all'utente di modificare la scena 3D
& Capitolato
& UC1.2
\\


\midrule
RFDe1.2.1
& Il software deve permettere all'utente di selezionare una fonte di luce esistente
& Interno
& UC1.2.1
\\


\midrule
RFDe1.2.2
& Il software deve permettere all'utente di selezionare la mesh da modificare
& Capitolato
& UC1.2.2
\\


\midrule
RFDe1.2.3
& Il software deve permettere all'utente di aggiungere una nuova fonte di luce
& Interno
& UC1.2.3
\\


\midrule
RFDe1.2.4
& Il software deve permettere all'utente di modificare la mesh selezionata
& Capitolato
& UC1.2.4
\\


\midrule
RFDe1.2.4.1
& Il software deve permettere all'utente di spostare la mesh selezionata nello spazio tridimensionale di un vettore arbitrario (traslate)
& Capitolato
& UC1.2.4.1
\\


\midrule
RFDe1.2.4.1.1
& Il software deve permettere all'utente di impostare il valore della componente x del vettore di traslazione
& Interno
& UC1.2.4.1.2
\\


\midrule
RFDe1.2.4.1.2
& Il software deve permettere all'utente di impostare il valore della componente y del vettore di traslazione
& Interno
& UC1.2.4.1.2
\\


\midrule
RFDe1.2.4.1.3
& Il software deve permettere all'utente di impostare il valore della componente z del vettore di traslazione
& Interno
& UC1.2.4.1.3
\\


\midrule
RFDe1.2.4.2
& Il software deve offrire la possibilità di ruotare la mesh selezionata di un angolo theta arbitrario rispetto ad un asse di rotazione
& Capitolato
& UC1.2.4.2
\\


\midrule
RFDe1.2.4.2.1
& Il software deve permettere all'utente di selezionare l'asse di rotazione
& Capitolato
& UC1.2.4.2.1
\\


\midrule
RFDe1.2.4.2.1.1
& Il software deve permettere all'utente di selezionare l'asse x come asse di rotazione
& Interno
& UC1.2.4.2.1.1
\\


\midrule
RFDe1.2.4.2.1.2
& Il software deve permettere all'utente di selezionare l'asse y come asse di rotazione
& Interno
& UC1.2.4.2.1.2
\\


\midrule
RFDe1.2.4.2.1.3
& Il software deve permettere all'utente di selezionare l'asse z come asse di rotazione
& Interno
& UC1.2.4.2.1.3
\\


\midrule
RFDe1.2.4.2.2
& Il software deve permettere all'utente di impostare l'ampiezza dell'angolo theta di rotazione nell'intervallo 0-360°
& Capitolato
& UC1.2.4.2.2
\\


\midrule
RFDe1.2.4.3
& Il software deve permettere all'utente di ridimensionare la mesh selezionata
& Capitolato
& UC1.2.4.3
\\


\midrule
RFDe1.2.4.3.1
& Il software deve permettere all'utente di ridimensionare la mesh secondo uno dei suoi assi
& Capitolato
& UC1.2.4.3.1
\\


\midrule
RFDe1.2.4.3.1.1
& Il software deve permettere all'utente di scegliere il valore della componente x
& Capitolato
& UC1.2.4.3.1.1
\\


\midrule
RFDe1.2.4.3.1.2
& Il software deve permettere all'utente di scegliere il valore della componente y
& Capitolato
& UC1.2.4.3.1.2
\\


\midrule
RFDe1.2.4.3.1.3
& Il software deve permettere all'utente di scegliere il valore della componente z
& Capitolato
& UC1.2.4.3.1.3
\\


\midrule
RFDe1.2.4.3.2
& Il software deve permettere all'utente di ridimensionare la mesh selezionata tramite moltiplicatore scalare
& Capitolato
& UC1.2.4.3.2
\\


\midrule
RFDe1.2.4.3.2.1
& Il software deve permettere all'utente di scegliere il valore del moltiplicatore scalare
& Capitolato
& UC1.2.4.3.2.1
\\


\midrule
RFDe1.2.4.4
& Il software deve permettere all'utente di modificare le caratteristiche del materiale della mesh selezionata
& Capitolato
& UC1.2.4.4
\\


\midrule
RFDe1.2.4.4.1
& Il software deve permettere all'utente di modificare la componente riflessiva diffusa del materiale della mesh selezionata
& Capitolato
& UC1.2.4.4.1
\\


\midrule
RFDe1.2.4.4.2
& Il software deve permettere all'utente di modificare la componente emissiva del materiale della mesh selezionata
& Capitolato
& UC1.2.4.4.2
\\


\midrule
RFDe1.2.4.4.3
& Il software deve permettere all'utente di modificare la componente riflessiva speculare del materiale della mesh selezionata
& Capitolato
& UC1.2.4.4.3
\\


\midrule
RFDe1.2.4.4.4
& Il software deve permettere all'utente di modificare la componente riflessiva ambientale del materiale della mesh selezionata
& Capitolato
& UC1.2.4.4.4
\\


\midrule
RFDe1.2.4.4.5
& Il software deve permettere all'utente di impostare l'opacità del materiale della mesh selezionata nell'intervallo 0-1
& Capitolato
& UC1.2.4.4.5
\\


\midrule
RFOp1.2.4.5
& Il software deve permettere all'utente di rimuovere la mesh selezionata
& Interno
& UC1.2.4.5
\\


\midrule
RFOp1.2.5
& Il software deve permettere all'utente di selezionare una camera
& Interno
& UC1.2.5
\\


\midrule
RFOp1.2.6
& Il software deve permettere all'utente di modificare la camera selezionata
& Interno
& UC1.2.6
\\


\midrule
RFOp1.2.6.1
& Il software deve permettere all'utente di modificare la posizione della camera
& Interno
& UC1.2.6.1
\\


\midrule
RFOp1.2.6.1.1
& Il software deve permettere all'utente di traslare la camera
& Interno
& UC1.2.6.1.1
\\


\midrule
RFOp1.2.6.1.1.1
& Il software deve permettere all'utente di avvicinare la camera al centro della scena
& Interno
& UC1.2.6.1.1.1
\\


\midrule
RFOp1.2.6.1.1.2
& Il software deve permettere all'utente di allontanare la camera dal centro della scena
& Interno
& UC1.2.6.1.1.2
\\


\midrule
RFOp1.2.6.1.1.3
& Il software deve permettere all'utente di impostare la traslazione della camera
& Interno
& UC1.2.6.1.1.3
\\


\midrule
RFOp1.2.6.1.2
& Il software deve permettere all'utente di ruotare la camera attorno alla scena 3D
& Capitolato
& UC1.2.6.1.2
\\


\midrule
RFOp1.2.6.2
& Il software deve permettere all'utente di rimuovere la camera selezionata
& Interno
& UC1.2.6.2
\\


\midrule
RFDe1.2.7
& Il software deve permettere all'utente di modificare le caratteristiche di una fonte di luce
& Capitolato
& UC1.2.7
\\


\midrule
RFOp1.2.7.1
& Il software deve permettere all'utente di rimuovere la fonte di luce selezionata
& Interno
& UC1.2.7.1
\\


\midrule
RFDe1.2.7.2
& Il software deve permettere all'utente di modificare la posizione della fonte di luce selezionata
& Capitolato
& UC1.2.7.2
\\


\midrule
RFOp1.2.7.3
& Il software deve permettere all'utente di modificare i \underline{parametri d'attenuazione} della fonte di luce selezionata
& Interno
& UC1.2.7.3
\\


\midrule
RFOp1.2.7.3.1
& Il software deve permettere all'utente di impostare il \underline{parametro d'attenuazione costante} nell'intervallo 0-1
& Interno
& UC1.2.7.3.1
\\


\midrule
RFOp1.2.7.3.2
& Il software deve permettere all'utente di impostare il \underline{parametro d'attenuazione lineare} nell'intervallo 0-1
& Interno
& UC1.2.7.3.2
\\


\midrule
RFOp1.2.7.3.3
& Il software deve permettere all'utente di impostare il \underline{parametro d'attenuazione quadratica} nell'intervallo 0-1
& Interno
& UC1.2.7.3.3
\\


\midrule
RFDe1.2.7.4
& Il software deve permettere all'utente di modificare il colore della fonte di luce selezionata
& Capitolato
& UC1.2.7.4
\\


\midrule
RFDe1.2.7.4.1
& Il software deve permettere all'utente di modificare il colore della componente di fonte di luce selezionata
& Capitolato
& UC1.2.7.4.1
\\


\midrule
RFDe1.2.7.4.1.1
& Il software deve permettere all'utente di selezionare la componente ambient di una fonte di luce
& Capitolato
& UC1.2.7.4.1.1
\\


\midrule
RFDe1.2.7.4.1.2
& Il software deve permettere all'utente di selezionare la componente specular di una fonte di luce
& Capitolato
& UC1.2.7.4.1.2
\\


\midrule
RFDe1.2.7.4.1.3
& Il software deve permettere all'utente di selezionare la componente diffuse di una fonte di luce
& Capitolato
& UC1.2.7.4.1.3
\\


\midrule
RFDe1.2.7.4.2
& Il software deve permettere all'utente di impostare i valori RGB della componente di luce selezionata
& Capitolato
& UC1.2.7.4.2
\\


\midrule
RFDe1.2.7.4.2.1
& Il software deve permettere all'utente di impostare il valore red della componente di luce selezionata nell'intervallo 0-1
& Capitolato
& UC1.2.7.4.2.1
\\


\midrule
RFDe1.2.7.4.2.2
& Il software deve permettere all'utente di impostare il valore green della componente di luce selezionata nell'intervallo 0-1
& Capitolato
& UC1.2.7.4.2.2
\\


\midrule
RFDe1.2.7.4.2.3
& Il software deve permettere all'utente di impostare il valore blue della componente di luce selezionata nell'intervallo 0-1
& Capitolato
& UC1.2.7.4.2.3
\\


\midrule
RFDe1.2.7.5
& Il software deve permettere all'utente di modificare la tipologia di una fonte di luce selezionata
& Capitolato
& UC1.2.7.5
\\


\midrule
RFDe1.2.7.5.1
& Il software deve permettere all'utente di modificare la tipologia di luce corrente in spotlight
& Capitolato
& UC1.2.7.5.1
\\


\midrule
RFDe1.2.7.5.2
& Il software deve permettere all'utente di modificare la tipologia di luce corrente in omni light
& Capitolato
& UC1.2.7.5.2
\\


\midrule
RFDe1.2.7.5.3
& Il software deve permettere all'utente di modificare la tipologia di luce corrente in directional light
& Capitolato
& UC1.2.7.5.3
\\


\midrule
RFOb1.3
& Il software deve permettere all'utente di esportare la scena 3D
& Capitolato
& UC1.3
\\


\midrule
RFOb1.3.1
& Il software deve permettere all'utente di selezionare il formato verso il quale desidera esportare
& Capitolato
& UC1.3.1
\\


\midrule
RFOb1.3.1.1
& Il software deve permettere all'utente di selezionare l'esportazione in formato JSON
& Capitolato
& UC1.3.1.1
\\


\midrule
RFOb1.3.1.1.1
& Il software deve permettere all'utente di selezionare l'esportazione in formato JSON "compatta"
& Capitolato
& UC1.3.1.1.1
\\


\midrule
RFOb1.3.1.1.2
& Il software deve permettere all'utente di selezionare l'esportazione in formato JSON "leggibile"
& Capitolato
& UC1.3.1.1.2
\\


\midrule
RFDe1.3.1.2
& Il software deve permettere all'utente di selezionare l'esportazione in formato XML
& Capitolato
& UC1.3.1.2
\\


\midrule
RFOb1.3.2
& Il software deve permettere all'utente di selezionare la precisione dei valori da esportare
& Capitolato
& UC1.3.2
\\


\midrule
RFOb1.3.2.1
& Il software deve permettere all'utente di selezionare la precisione float
& Capitolato
& UC1.3.2.1
\\


\midrule
RFOb1.3.2.2
& Il software deve permettere all'utente di poter selezionare la precisione double
& Capitolato
& UC1.3.2.2
\\


\midrule
RFDe1.3.3
& Il software deve permettere all'utente di selezionare la directory di destinazione
& Interno
& UC1.3.3
\\


\midrule
RFDe1.3.4
& Il software deve permettere all'utente di nominare il file di esportazione
& Interno
& UC1.3.4
\\


\midrule
RFDe1.4
& Il software deve permettere all'utente di poter visualizzare l'anteprima della scena 3D
& Capitolato
& UC1.4
\\


\midrule
RFDe1.4.1
& Il software deve permettere all'utente di visualizzare un'anteprima semplice
& Interno
& UC1.4.1
\\


\midrule
RFDe1.4.1.1
& Il software deve permettere all'utente di traslare il suo punto d'osservazione
& Interno
& UC1.4.1.1
\\


\midrule
RFDe1.4.1.1.1
& Il software deve permettere all'utente di avvicinare il suo punto di osservazione al centro della scena
& Interno
& UC1.4.1.1.1
\\


\midrule
RFDe1.4.1.1.2
& Il software deve permettere all'utente di allontanare il suo punto di vista dal centro della scena
& Interno
& UC1.4.1.1.2
\\


\midrule
RFDe1.4.1.1.3
& Il software deve permettere all'utente traslare il punto di osservazione
& Interno
& UC1.4.1.1.3
\\


\midrule
RFDe1.4.1.2
& Il software deve permettere all'utente di ruotare il punto d'osservazione attorno alla scena 3D
& Capitolato
& UC1.4.1.2
\\


\midrule
RFOp1.4.2
& Il software deve permettere all'utente di eseguire il rendering della scena 3D
& Interno
& UC1.4.2
\\


\midrule
RFDe1.5
& Il software deve permettere all'utente di agire sulla cronologia delle modifiche
& Capitolato
& UC1.5
\\


\midrule
RFDe1.5.1
& Il software deve permettere all'utente di annullare l'ultima modifica effettuata
& Capitolato
& UC1.5.1
\\


\midrule
RFDe1.5.2
& Il software deve permettere all'utente di ripetere l'ultima modifica annullata
& Capitolato
& UC1.5.2
\\


\midrule
RFOb2
& Il software deve poter fare il \underline{parsing} di un file con un determinato tipo
& Capitolato
& UC1
\\


\midrule
RFOb2.1
& Il software deve poter fare il parsing di un file in formato 3DS
& Capitolato
& UC1
\\


\midrule
RFDe2.2
& Il software deve poter fare il parsing di un file in formato OBJ con MTL associati
& Capitolato
& UC1
\\


\midrule
RFDe2.3
& Il software deve poter fare il parsing di un file in formato JSON
& Capitolato
& UC1
\\


\midrule
RFDe2.4
& Il software deve poter fare il parsing di un file in formato XML
& Capitolato
& UC1
\\


\midrule
RFOb3
& Il software deve garantire che la \underline{conversione} mantenga le caratteristiche della scena 3D
& Capitolato
& UC1
\\


\midrule
RFOb3.1
& Il software deve garantire che la conversione mantenga lo stesso numero di vertici
& Capitolato
& UC1
\\


\midrule
RFOb3.2
& Il software deve garantire che la conversione mantenga le caratteristiche delle \underline{texture} della scena 3D
& Capitolato
& UC1
\\


\midrule
RFOb3.3
& Il software deve garantire che la conversione mantenga le caratteristiche che determinano il colore della texture
& Capitolato
& UC1
\\


\midrule
RFOb4
& Il software deve garantire che la conversione mantenga tutte le peculiarità dell'illuminazione della scena 3D
& Capitolato
& UC1
\\


\midrule
RFOb4.1
& Il software deve garantire che la conversione mantenga la posizione delle luci
& Capitolato
& UC1
\\


\midrule
RFOb4.2
& Il software deve garantire che la conversione mantenga il colore delle luci
& Capitolato
& UC1
\\


\midrule
RFOb4.3
& Il software deve garantire che la conversione mantenga la direzione definita delle luci
& Capitolato
& UC1
\\


\midrule
RFOp5
& Il software deve mantenere le animazioni implementate con keyframes
& Capitolato
& UC1
\\


\midrule
RFOp6
& Il software deve garantire l'invarianza di ogni altra caratteristica del solido in seguito a una delle possibili modifiche
& Capitolato
& UC1
\\


\midrule
RFDe7
& Il software deve memorizzare la cronologia delle modifiche utente
& Capitolato
& UC1
\\


\midrule
RDOb8
& Il sistema deve permettere all'utente di applicare alla scena i limiti di OpenGL ES 2.0
& Capitolato
&
\\


\midrule
RDOb9
& Il software deve funzionare in ambiente Windows 7 Professional 64 bit
& Capitolato
&
\\


\midrule
RFDe10
& Il sistema deve segnalare l'avanzamento dell'operazione di import o export mostrando il progresso
& Interno
&
\\


\midrule
RQOb11
& Il gruppo fornirà documentazione in tecnologia Doxygen di tutte le classi e di tutti i metodi del software
& Interno
&
\\


\midrule
RQOp12
& Il software deve essere estensibile, in particolare riguardo ai tipi di file caricabili ed esportabili
& Interno
&
\\


\midrule
RQOp13
& Se il software si blocca prima che l'utente abbia salvato il file, si deve essere in grado di ripristinare tutte le modifiche effettuate
& Interno
&
\\


\midrule
RQOp14
& Il software sarà disponibile anche in lingua inglese
& Interno
&
\\


\midrule
RQOp15
& Il software deve impedire l'introduzione di dati scorretti e ove necessario segnalare l'errore
& Interno
&
\\


\midrule
RQOp16
& Insieme al software deve essere fornito un manuale utente
& Interno
&
\\


\midrule
RQOp17
& Insieme al software deve essere fornito l'XML Schema di validazione
& Capitolato
&
\\




\end{longtable}

