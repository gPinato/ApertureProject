
\subsection{UC1 Caso d’uso generale}
\textbf{Diagramma associato:}
\ref{UC1} \\ \\
\textbf{Attori Coinvolti:}
Utente. \\ \\
\textbf{Scopo e Descrizione:}
un utente deve poter scegliere un file opportuno da cui caricare la scena 3D descritta. Una volta caricata, deve essere possibile modificarla ed esportarla nel formato desiderato. Il caricamento e la modifica della scena 3D comportano la visualizzazione dell'anteprima della stessa. \\ \\
\textbf{Precondizione:}
il sistema si trova nello stato iniziale. \\ \\
\textbf{Postcondizione:}
il sistema ha caricato in memoria la scena 3D descritta nel file selezionato. \\ \\
\textbf{Scenario Principale:}
\begin{itemize}
\item l’utente seleziona il file da cui caricare la scena 3D
\item  l’utente eventualmente decide che modifiche apportare alla scena 3D
\item  l’utente esporta la scena 3D nel formato desiderato
\item l'utente agisce sulla cronologia delle modifiche
\\ \\ \end{itemize}
\begin{figure}[h!]
\centering
\includegraphics[width=\textwidth]{{{UC1}}}
\caption{UC1 Caso d’uso generale}
\label{UC1}
\end{figure}


\subsection{UC1.1 Sceglie file da importare}
\textbf{Diagramma associato:}
\ref{UC1.1} \\ \\
\textbf{Attori Coinvolti:}
Utente. \\ \\
\textbf{Scopo e Descrizione:}
un utente deve poter selezionare il formato del file dal quale caricare la scena 3D e scegliere il percorso del file medesimo. \\ \\
\textbf{Precondizione:}
il sistema è in attesa che l'utente selezioni una funzionalità. \\ \\
\textbf{Postcondizione:}
il sistema ha caricato la scena 3D descritta nel file scelto, la scena si trova nel suo stato iniziale. \\ \\
\textbf{Scenario Principale:}
\begin{itemize}
\item  l’utente seleziona l’estensione del file da importare
\item l’utente seleziona il percorso del file
\\ \\ \end{itemize}
\begin{figure}[h!]
\centering
\includegraphics[width=\textwidth]{{{UC1.1}}}
\caption{UC1.1 Sceglie file da importare}
\label{UC1.1}
\end{figure}


\subsection{UC1.1.1 Sceglie tipo file}
\textbf{Diagramma associato:}
\ref{UC1.1.1} \\ \\
\textbf{Attori Coinvolti:}
Utente. \\ \\
\textbf{Scopo e Descrizione:}
un utente deve poter selezionare il formato del file dal quale caricare la scena 3D. \\ \\
\textbf{Precondizione:}
il sistema è in attesa di input riguardo all'importazione. \\ \\
\textbf{Postcondizione:}
il sistema conosce il formato del file scelto dall’utente. \\ \\
\textbf{Scenario Principale:}
\begin{itemize}
\item l’utente seleziona il formato 3DS
\item l’utente seleziona il formato Wavefront OBJ/MTL
\item l’utente seleziona il formato JSON
\item l'utente seleziona il formato XML
\\ \\ \end{itemize}
\begin{figure}[h!]
\centering
\includegraphics[width=\textwidth]{{{UC1.1.1}}}
\caption{UC1.1.1 Sceglie tipo file}
\label{UC1.1.1}
\end{figure}


\subsection{UC1.1.1.1 Sceglie formato 3DS}
\textbf{Diagramma associato:}
\ref{UC1.1.1} \\ \\
\textbf{Attori Coinvolti:}
Utente. \\ \\
\textbf{Scopo e Descrizione:}
un utente deve poter importare dal formato 3DS. \\ \\
\textbf{Precondizione:}
il sistema è in attesa di input riguardo al formato del file da importare. \\ \\
\textbf{Postcondizione:}
il sistema sa che si sta per importare dal formato 3DS. \\ \\
\textbf{Scenario Principale:}
\begin{itemize}
\item l'utente seleziona il formato 3DS
\\ \\ \end{itemize}


\subsection{UC1.1.1.2 Sceglie formato Wavefront OBJ/MTL}
\textbf{Diagramma associato:}
\ref{UC1.1.1} \\ \\
\textbf{Attori Coinvolti:}
Utente. \\ \\
\textbf{Scopo e Descrizione:}
un utente deve poter importare dal formato Wavefront OBJ/MTL. \\ \\
\textbf{Precondizione:}
il sistema è in attesa di input riguardo al formato del file da importare. \\ \\
\textbf{Postcondizione:}
il sistema sa che si sta per importare dal formato OBJ/MTL. \\ \\
\textbf{Scenario Principale:}
\begin{itemize}
\item l'utente seleziona il formato Wavefront OBJ/MTL
\\ \\ \end{itemize}


\subsection{UC1.1.1.3 Sceglie formato JSON}
\textbf{Diagramma associato:}
\ref{UC1.1.1} \\ \\
\textbf{Attori Coinvolti:}
Utente. \\ \\
\textbf{Scopo e Descrizione:}
un utente deve poter importare dal formato JSON. \\ \\
\textbf{Precondizione:}
il sistema è in attesa di input riguardo al formato del file da importare. \\ \\
\textbf{Postcondizione:}
il sistema sa che si sta per importare dal formato JSON. \\ \\
\textbf{Scenario Principale:}
\begin{itemize}
\item l'utente seleziona il formato JSON
\\ \\ \end{itemize}


\subsection{UC1.1.1.4 Sceglie formato XML}
\textbf{Diagramma associato:}
\ref{UC1.1.1} \\ \\
\textbf{Attori Coinvolti:}
Utente. \\ \\
\textbf{Scopo e Descrizione:}
un utente deve poter importare dal formato XML. \\ \\
\textbf{Precondizione:}
il sistema è in attesa di input riguardo al formato del file da importare. \\ \\
\textbf{Postcondizione:}
il sistema sa che si sta per importare dal formato XML. \\ \\
\textbf{Scenario Principale:}
\begin{itemize}
\item l'utente seleziona il formato XML
\\ \\ \end{itemize}


\subsection{UC1.1.2 Seleziona percorso file}
\textbf{Diagramma associato:}
\ref{UC1.1} \\ \\
\textbf{Attori Coinvolti:}
Utente. \\ \\
\textbf{Scopo e Descrizione:}
un utente deve poter selezionare il percorso del file da cui caricare la scena 3D. \\ \\
\textbf{Precondizione:}
il sistema è in attesa di input riguardo all'importazione. \\ \\
\textbf{Postcondizione:}
il sistema conosce la posizione del file dal quale caricare la scena 3D. \\ \\
\textbf{Scenario Principale:}
\begin{itemize}
\item l’utente seleziona il percorso del file dal quale caricare la scena 3D
\\ \\ \end{itemize}


\subsection{UC1.2 Modifica scena 3D}
\textbf{Diagramma associato:}
\ref{UC1.2} \\ \\
\textbf{Attori Coinvolti:}
Utente. \\ \\
\textbf{Scopo e Descrizione:}
un utente deve poter modificare la scena 3D. \\ \\
\textbf{Precondizione:}
il sistema ha in memoria la scena 3D caricata in precedenza (UC1.1), ed è in attesa che l'utente selezioni una funzionalità. \\ \\
\textbf{Postcondizione:}
il sistema ha modificato la scena 3D secondo le richieste dell’utente, e la modifica è stata inserita nella cronologia delle modifiche. La scena non è più nel suo stato iniziale. \\ \\
\textbf{Scenario Principale:}
\begin{itemize}
\item l'utente seleziona la \underline{fonte di luce} che intende modificare (UC1.2.1)
\item l'utente seleziona la \underline{mesh} che intende modificare (UC1.2.2)
\item l'utente aggiunge una nuova fonte di luce con valori di default (UC1.2.3)
\item l'utente modifica la mesh (UC1.2.4)
\item l'utente seleziona la \underline{camera} che intende modificare (UC1.2.5)
\item l'utente modifica la camera (UC1.2.6)
\item l'utente modifica la fonte di luce (UC1.2.7)
\\ \\ \end{itemize}
\begin{figure}[h!]
\centering
\includegraphics[width=\textwidth]{{{UC1.2}}}
\caption{UC1.2 Modifica scena 3D}
\label{UC1.2}
\end{figure}


\subsection{UC1.2.1 Seleziona fonte di luce}
\textbf{Diagramma associato:}
\ref{UC1.2} \\ \\
\textbf{Attori Coinvolti:}
Utente. \\ \\
\textbf{Scopo e Descrizione:}
un utente deve poter selezionare la fonte di luce che intende modificare. \\ \\
\textbf{Precondizione:}
il sistema è in attesa di input  per quanto riguarda la modifica della scena 3D. \\ \\
\textbf{Postcondizione:}
il sistema conosce la fonte di luce selezionata dall'utente. \\ \\
\textbf{Scenario Principale:}
\begin{itemize}
\item l'utente seleziona una fonte di luce
\\ \\ \end{itemize}


\subsection{UC1.2.2 Seleziona mesh}
\textbf{Diagramma associato:}
\ref{UC1.2} \\ \\
\textbf{Attori Coinvolti:}
Utente. \\ \\
\textbf{Scopo e Descrizione:}
un utente deve poter selezionare la mesh che intende modificare. \\ \\
\textbf{Precondizione:}
il sistema è in attesa di input  per quanto riguarda la modifica della scena 3D. \\ \\
\textbf{Postcondizione:}
il sistema conosce la mesh che l'utente vuole modificare. \\ \\
\textbf{Scenario Principale:}
\begin{itemize}
\item l'utente seleziona una mesh
\\ \\ \end{itemize}


\subsection{UC1.2.3 Aggiungi fonte di luce}
\textbf{Diagramma associato:}
\ref{UC1.2} \\ \\
\textbf{Attori Coinvolti:}
Utente. \\ \\
\textbf{Scopo e Descrizione:}
un utente deve poter aggiungere una nuova fonte di luce con valori di default. \\ \\
\textbf{Precondizione:}
il sistema è in attesa di input  per quanto riguarda la modifica della scena 3D. \\ \\
\textbf{Postcondizione:}
il sistema aggiunge la nuova fonte di luce con valori di default. \\ \\
\textbf{Scenario Principale:}
\begin{itemize}
\item l'utente aggiunge una nuova fonte di luce
\\ \\ \end{itemize}


\subsection{UC1.2.4 Modifica mesh}
\textbf{Diagramma associato:}
\ref{UC1.2.4} \\ \\
\textbf{Attori Coinvolti:}
Utente. \\ \\
\textbf{Scopo e Descrizione:}
un utente deve poter modificare la mesh selezionata. \\ \\
\textbf{Precondizione:}
il sistema conosce la mesh da modificare (UC1.2.2). \\ \\
\textbf{Postcondizione:}
il sistema ha modificato la mesh selezionata secondo le richieste dell'utente. \\ \\
\textbf{Scenario Principale:}
\begin{itemize}
\item l'utente effettua una \underline{traslazione} (UC1.2.4.1)
\item l'utente effettua una rotazione (UC1.2.4.2)
\item l'utente modifica la dimensione (UC1.2.4.3)
\item l'utente modifica le caratteristiche del materiale (UC1.2.4.4)
\item l'utente rimuove la mesh (UC1.2.4.5)
\\ \\ \end{itemize}
\begin{figure}[h!]
\centering
\includegraphics[width=\textwidth]{{{UC1.2.4}}}
\caption{UC1.2.4 Modifica mesh}
\label{UC1.2.4}
\end{figure}


\subsection{UC1.2.4.1 Effettua traslazione}
\textbf{Diagramma associato:}
\ref{UC1.2.4.1} \\ \\
\textbf{Attori Coinvolti:}
Utente. \\ \\
\textbf{Scopo e Descrizione:}
un utente deve poter scegliere i valori del vettore [x,y,z] di traslazione da applicare alla mesh selezionata. \\ \\
\textbf{Precondizione:}
il sistema è in attesa che l'utente selezioni una funzionalità di modifica della mesh. \\ \\
\textbf{Postcondizione:}
il sistema ha traslato la mesh secondo le richieste dell’utente. \\ \\
\textbf{Scenario Principale:}
\begin{itemize}
\item l’utente sceglie il valore della componente x del \underline{vettore di traslazione} (UC1.2.4.1.1)
\item  l’utente sceglie il valore della componente y del vettore di traslazione (UC1.2.4.1.2)
\item l’utente sceglie il valore della componente z del vettore di traslazione (UC1.2.4.1.3)
\\ \\ \end{itemize}
\begin{figure}[h!]
\centering
\includegraphics[width=\textwidth]{{{UC1.2.4.1}}}
\caption{UC1.2.4.1 Effettua traslazione}
\label{UC1.2.4.1}
\end{figure}


\subsection{UC1.2.4.1.1 Sceglie valore componente x}
\textbf{Diagramma associato:}
\ref{UC1.2.4.1} \\ \\
\textbf{Attori Coinvolti:}
Utente. \\ \\
\textbf{Scopo e Descrizione:}
un utente deve poter impostare il valore della componente x del vettore di traslazione. \\ \\
\textbf{Precondizione:}
il sistema è in attesa di input riguardo alla traslazione. \\ \\
\textbf{Postcondizione:}
il sistema conosce il valore impostato dall'utente della componente x del vettore di traslazione . \\ \\
\textbf{Scenario Principale:}
\begin{itemize}
\item l'utente imposta il valore della componente x
\\ \\ \end{itemize}


\subsection{UC1.2.4.1.2 Sceglie valore componente y}
\textbf{Diagramma associato:}
\ref{UC1.2.4.1} \\ \\
\textbf{Attori Coinvolti:}
Utente. \\ \\
\textbf{Scopo e Descrizione:}
un utente deve poter impostare il valore della componente y del vettore di traslazione. \\ \\
\textbf{Precondizione:}
il sistema è in attesa di input riguardo alla traslazione. \\ \\
\textbf{Postcondizione:}
il sistema conosce il valore impostato dall'utente della componente y del vettore di traslazione. \\ \\
\textbf{Scenario Principale:}
\begin{itemize}
\item l'utente imposta il valore della componente y
\\ \\ \end{itemize}


\subsection{UC1.2.4.1.3 Sceglie valore componente z}
\textbf{Diagramma associato:}
\ref{UC1.2.4.1} \\ \\
\textbf{Attori Coinvolti:}
Utente. \\ \\
\textbf{Scopo e Descrizione:}
un utente deve poter impostare il valore della componente z del vettore di traslazione. \\ \\
\textbf{Precondizione:}
il sistema è in attesa di input riguardo alla traslazione. \\ \\
\textbf{Postcondizione:}
il sistema conosce il valore impostato dall'utente della componente z del vettore di traslazione. \\ \\
\textbf{Scenario Principale:}
\begin{itemize}
\item l'utente imposta il valore della componente z
\\ \\ \end{itemize}


\subsection{UC1.2.4.2 Effettua rotazione}
\textbf{Diagramma associato:}
\ref{UC1.2.4.2} \\ \\
\textbf{Attori Coinvolti:}
Utente. \\ \\
\textbf{Scopo e Descrizione:}
un utente deve poter ruotare la mesh selezionata scegliendo l’asse di rotazione e l’angolo theta di rotazione. \\ \\
\textbf{Precondizione:}
il sistema è in attesa che l'utente selezioni una funzionalità di modifica della mesh. \\ \\
\textbf{Postcondizione:}
il sistema ha ruotato la mesh secondo le richieste dell’utente. \\ \\
\textbf{Scenario Principale:}
\begin{itemize}
\item l’utente sceglie l’asse di rotazione (UC1.2.4.2.1)
\item l’utente sceglie l’angolo theta di rotazione (UC1.2.4.2.2)
\\ \\ \end{itemize}
\begin{figure}[h!]
\centering
\includegraphics[width=\textwidth]{{{UC1.2.4.2}}}
\caption{UC1.2.4.2 Effettua rotazione}
\label{UC1.2.4.2}
\end{figure}


\subsection{UC1.2.4.2.1 Sceglie asse di rotazione}
\textbf{Diagramma associato:}
\ref{UC1.2.4.2.1} \\ \\
\textbf{Attori Coinvolti:}
Utente. \\ \\
\textbf{Scopo e Descrizione:}
un utente deve poter scegliere l'\underline{asse di rotazione} in base al quale ruotare la mesh selezionata. \\ \\
\textbf{Precondizione:}
il sistema è in attesa di input riguardo alla rotazione. \\ \\
\textbf{Postcondizione:}
il sistema conosce l'asse di rotazione selezionato dall'utente. \\ \\
\textbf{Scenario Principale:}
\begin{itemize}
\item l’utente sceglie l'asse x (UC1.2.4.2.1.1)
\item  l’utente sceglie l'asse y (UC1.2.4.2.1.2)
\item l’utente sceglie l'asse z (UC1.2.4.2.1.3)
\\ \\ \end{itemize}
\begin{figure}[h!]
\centering
\includegraphics[width=\textwidth]{{{UC1.2.4.2.1}}}
\caption{UC1.2.4.2.1 Sceglie asse di rotazione}
\label{UC1.2.4.2.1}
\end{figure}


\subsection{UC1.2.4.2.1.1 Sceglie asse x}
\textbf{Diagramma associato:}
\ref{UC1.2.4.2.1} \\ \\
\textbf{Attori Coinvolti:}
Utente. \\ \\
\textbf{Scopo e Descrizione:}
un utente deve poter scegliere l'asse x come asse di rotazione. \\ \\
\textbf{Precondizione:}
il sistema è in attesa di input riguardo l'asse di rotazione. \\ \\
\textbf{Postcondizione:}
il sistema sa che l'utente ha selezionato l'asse x. \\ \\
\textbf{Scenario Principale:}
\begin{itemize}
\item l'utente seleziona l'asse x
\\ \\ \end{itemize}


\subsection{UC1.2.4.2.1.2 Sceglie asse y}
\textbf{Diagramma associato:}
\ref{UC1.2.4.2.1} \\ \\
\textbf{Attori Coinvolti:}
Utente. \\ \\
\textbf{Scopo e Descrizione:}
un utente deve poter scegliere l'asse y come asse di rotazione. \\ \\
\textbf{Precondizione:}
il sistema è in attesa di input riguardo l'asse di rotazione. \\ \\
\textbf{Postcondizione:}
il sistema sa che l'utente ha selezionato l'asse y. \\ \\
\textbf{Scenario Principale:}
\begin{itemize}
\item l'utente seleziona l'asse y
\\ \\ \end{itemize}


\subsection{UC1.2.4.2.1.3 Sceglie asse z}
\textbf{Diagramma associato:}
\ref{UC1.2.4.2.1} \\ \\
\textbf{Attori Coinvolti:}
Utente. \\ \\
\textbf{Scopo e Descrizione:}
un utente deve poter scegliere l'asse z come asse di rotazione. \\ \\
\textbf{Precondizione:}
il sistema è in attesa di input riguardo l'asse di rotazione. \\ \\
\textbf{Postcondizione:}
il sistema sa che l'utente ha selezionato l'asse z. \\ \\
\textbf{Scenario Principale:}
\begin{itemize}
\item l'utente seleziona l'asse z
\\ \\ \end{itemize}


\subsection{UC1.2.4.2.2 Sceglie ampiezza angolo theta}
\textbf{Diagramma associato:}
\ref{UC1.2.4.2} \\ \\
\textbf{Attori Coinvolti:}
Utente. \\ \\
\textbf{Scopo e Descrizione:}
un utente deve poter impostare l'ampiezza dell'\underline{angolo theta di rotazione} secondo il quale routare la mesh selezionata. \\ \\
\textbf{Precondizione:}
il sistema è in attesa di input riguardo alla rotazione. \\ \\
\textbf{Postcondizione:}
il sistema conosce l'ampiezza dell'angolo theta di rotazione impostato dall'utente. \\ \\
\textbf{Scenario Principale:}
\begin{itemize}
\item l'utente imposta l'ampiezza dell'angolo theta di rotazione
\\ \\ \end{itemize}


\subsection{UC1.2.4.3 Modifica dimensione}
\textbf{Diagramma associato:}
\ref{UC1.2.4.3} \\ \\
\textbf{Attori Coinvolti:}
Utente. \\ \\
\textbf{Scopo e Descrizione:}
un utente deve poter modificare la dimensione della mesh selezionata. \\ \\
\textbf{Precondizione:}
il sistema è in attesa che l'utente selezioni una funzionalità di modifica della mesh. \\ \\
\textbf{Postcondizione:}
il sistema ha effettuato il ridimensionamento della mesh selezionata secondo le richieste dell’utente. \\ \\
\textbf{Scenario Principale:}
\begin{itemize}
\item l’utente sceglie il ridimensionamento secondo gli assi (UC1.2.4.3.1)
\item l’utente sceglie il \underline{ridimensionamento scalare} (UC1.2.4.3.2)
\\ \\ \end{itemize}
\begin{figure}[h!]
\centering
\includegraphics[width=\textwidth]{{{UC1.2.4.3}}}
\caption{UC1.2.4.3 Modifica dimensione}
\label{UC1.2.4.3}
\end{figure}


\subsection{UC1.2.4.3.1 Ridimensiona secondo gli assi}
\textbf{Diagramma associato:}
\ref{UC1.2.4.3.1} \\ \\
\textbf{Attori Coinvolti:}
Utente. \\ \\
\textbf{Scopo e Descrizione:}
un utente deve poter ridimensionare la mesh selezionata secondo i suoi assi. \\ \\
\textbf{Precondizione:}
il sistema propone all'utente una scelta sul tipo di ridimensionamento. \\ \\
\textbf{Postcondizione:}
il sistema ha ridimensionato la mesh secondo le impostazioni dell'utente. \\ \\
\textbf{Scenario Principale:}
\begin{itemize}
\item l'utente sceglie il valore della componente x (UC1.2.4.3.1.1)
\item l'utente sceglie il valore della componente y (UC1.2.4.3.1.2)
\item l'utente sceglie il valore della componente z (UC1.2.4.3.1.3)
\\ \\ \end{itemize}
\begin{figure}[h!]
\centering
\includegraphics[width=\textwidth]{{{UC1.2.4.3.1}}}
\caption{UC1.2.4.3.1 Ridimensiona secondo gli assi}
\label{UC1.2.4.3.1}
\end{figure}


\subsection{UC1.2.4.3.1.1 Sceglie valore componente x}
\textbf{Diagramma associato:}
\ref{UC1.2.4.3.1} \\ \\
\textbf{Attori Coinvolti:}
Utente. \\ \\
\textbf{Scopo e Descrizione:}
un utente deve poter impostare il valore della componente x. \\ \\
\textbf{Precondizione:}
il sistema è in attesa di input riguardo al ridimensionamento della mesh. \\ \\
\textbf{Postcondizione:}
il sistema conosce il valore della componente x impostato dall'utente. \\ \\
\textbf{Scenario Principale:}
\begin{itemize}
\item l'utente imposta il valore della componente x
\\ \\ \end{itemize}


\subsection{UC1.2.4.3.1.2 Sceglie valore componente y}
\textbf{Diagramma associato:}
\ref{UC1.2.4.3.1} \\ \\
\textbf{Attori Coinvolti:}
Utente. \\ \\
\textbf{Scopo e Descrizione:}
un utente deve poter impostare il valore della componente y. \\ \\
\textbf{Precondizione:}
il sistema è in attesa di input riguardo al ridimensionamento della mesh. \\ \\
\textbf{Postcondizione:}
il sistema conosce il valore della componente y impostato dall'utente. \\ \\
\textbf{Scenario Principale:}
\begin{itemize}
\item l'utente imposta il valore della componente y
\\ \\ \end{itemize}


\subsection{UC1.2.4.3.1.3 Sceglie valore componente z}
\textbf{Diagramma associato:}
\ref{UC1.2.4.3.1} \\ \\
\textbf{Attori Coinvolti:}
Utente. \\ \\
\textbf{Scopo e Descrizione:}
un utente deve poter impostare il valore della componente z. \\ \\
\textbf{Precondizione:}
il sistema è in attesa di input riguardo al ridimensionamento della mesh. \\ \\
\textbf{Postcondizione:}
il sistema conosce il valore della componente z impostato dall'utente. \\ \\
\textbf{Scenario Principale:}
\begin{itemize}
\item l'utente imposta il valore della componente z
\\ \\ \end{itemize}


\subsection{UC1.2.4.3.2 Ridimensiona scalarmente}
\textbf{Diagramma associato:}
\ref{UC1.2.4.3.2} \\ \\
\textbf{Attori Coinvolti:}
Utente. \\ \\
\textbf{Scopo e Descrizione:}
un utente deve poter ridimensionare la mesh impostando un moltiplicatore scalare. \\ \\
\textbf{Precondizione:}
il sistema propone all'utente una scelta sul tipo di ridimensionamento. \\ \\
\textbf{Postcondizione:}
il sistema ha ridimensionato (scalarmente) la mesh secondo l'input utente. \\ \\
\textbf{Scenario Principale:}
\begin{itemize}
\item l'utente sceglie il ridimensionamento scalare
\\ \\ \end{itemize}
\begin{figure}[h!]
\centering
\includegraphics[width=\textwidth]{{{UC1.2.4.3.2}}}
\caption{UC1.2.4.3.2 Ridimensiona scalarmente}
\label{UC1.2.4.3.2}
\end{figure}


\subsection{UC1.2.4.3.2.1 Sceglie componente scalare}
\textbf{Diagramma associato:}
\ref{UC1.2.4.3.2} \\ \\
\textbf{Attori Coinvolti:}
Utente. \\ \\
\textbf{Scopo e Descrizione:}
un utente deve poter impostare il moltiplicatore scalare relativo al ridimensionamento di una mesh. \\ \\
\textbf{Precondizione:}
il sistema è in attesa di input riguardo al ridimensionamento della mesh. \\ \\
\textbf{Postcondizione:}
il sistema conosce il valore del moltiplicatore scalare impostato dall'utente. \\ \\
\textbf{Scenario Principale:}
\begin{itemize}
\item l'utente sceglie il valore della componente scalare
\\ \\ \end{itemize}


\subsection{UC1.2.4.4 Modifica caratteristiche materiale}
\textbf{Diagramma associato:}
\ref{UC1.2.4.4} \\ \\
\textbf{Attori Coinvolti:}
Utente. \\ \\
\textbf{Scopo e Descrizione:}
un utente deve poter modificare le caratteristiche di uno dei materiali che compongono la mesh selezionata. \\ \\
\textbf{Precondizione:}
il sistema è in attesa che l'utente selezioni una funzionalità di modifica della mesh. \\ \\
\textbf{Postcondizione:}
il sistema ha effettuato le modifiche a uno dei materiali che compongono la mesh selezionata secondo le richieste dell’utente. \\ \\
\textbf{Scenario Principale:}
\begin{itemize}
\item l'utente modifica la componente riflessiva diffusa (UC1.2.4.4.1)
\item l'utente modifica la componente emissiva (UC1.2.4.4.2)
\item l'utente modifica la componente riflessiva speculare (UC1.2.4.4.3)
\item l'utente modifica la componente riflessiva ambientale (UC1.2.4.4.4)
\item l'utente modifica il parametro di \underline{opacità} (UC1.2.4.4.5)
\\ \\ \end{itemize}
\begin{figure}[h!]
\centering
\includegraphics[width=\textwidth]{{{UC1.2.4.4}}}
\caption{UC1.2.4.4 Modifica caratteristiche materiale}
\label{UC1.2.4.4}
\end{figure}


\subsection{UC1.2.4.4.1 Modifica componente riflessiva diffusa}
\textbf{Diagramma associato:}
\ref{UC1.2.4.4} \\ \\
\textbf{Attori Coinvolti:}
Utente. \\ \\
\textbf{Scopo e Descrizione:}
un utente deve poter modificare la \underline{componente riflessiva diffusa di un materiale}. \\ \\
\textbf{Precondizione:}
il sistema conosce il materiale da modificare ed è in attesa di input per quanto riguarda la modifica dello stesso. \\ \\
\textbf{Postcondizione:}
il sistema conosce il nuovo valore della componente riflessiva diffusa impostato dall'utente. \\ \\
\textbf{Scenario Principale:}
\begin{itemize}
\item l'utente imposta il valore della componente riflessiva diffusa del materiale selezionato
\\ \\ \end{itemize}


\subsection{UC1.2.4.4.2 Modifica componente emissiva}
\textbf{Diagramma associato:}
\ref{UC1.2.4.4} \\ \\
\textbf{Attori Coinvolti:}
Utente. \\ \\
\textbf{Scopo e Descrizione:}
un utente deve poter modificare la \underline{componente emissiva di un materiale}. \\ \\
\textbf{Precondizione:}
il sistema conosce il materiale da modificare ed è in attesa di input per quanto riguarda la modifica dello stesso. \\ \\
\textbf{Postcondizione:}
il sistema conosce il nuovo valore della componente emissiva impostato dall'utente. \\ \\
\textbf{Scenario Principale:}
\begin{itemize}
\item l'utente imposta il valore della componente emissiva del materiale selezionato
\\ \\ \end{itemize}


\subsection{UC1.2.4.4.3 Modifica componente riflessiva speculare}
\textbf{Diagramma associato:}
\ref{UC1.2.4.4} \\ \\
\textbf{Attori Coinvolti:}
Utente. \\ \\
\textbf{Scopo e Descrizione:}
un utente deve poter modificare la \underline{componente riflessiva speculare di un materiale}. \\ \\
\textbf{Precondizione:}
il sistema conosce il materiale da modificare ed è in attesa di input per quanto riguarda la modifica dello stesso. \\ \\
\textbf{Postcondizione:}
il sistema conosce il nuovo valore della componente riflessiva speculare impostato dall'utente. \\ \\
\textbf{Scenario Principale:}
\begin{itemize}
\item l'utente imposta il valore della componente riflessiva speculare del materiale selezionato
\\ \\ \end{itemize}


\subsection{UC1.2.4.4.4 Modifica componente riflessiva ambientale}
\textbf{Diagramma associato:}
\ref{UC1.2.4.4} \\ \\
\textbf{Attori Coinvolti:}
Utente. \\ \\
\textbf{Scopo e Descrizione:}
un utente deve poter modificare la \underline{componente riflessiva ambientale di un materiale}. \\ \\
\textbf{Precondizione:}
il sistema conosce il materiale da modificare ed è in attesa di input per quanto riguarda la modifica dello stesso. \\ \\
\textbf{Postcondizione:}
il sistema conosce il nuovo valore della componente riflessiva ambientale impostato dall'utente. \\ \\
\textbf{Scenario Principale:}
\begin{itemize}
\item l'utente imposta il valore della componente riflessiva ambientale del materiale selezionato
\\ \\ \end{itemize}


\subsection{UC1.2.4.4.5 Modifica parametro opacità}
\textbf{Diagramma associato:}
\ref{UC1.2.4.4} \\ \\
\textbf{Attori Coinvolti:}
Utente. \\ \\
\textbf{Scopo e Descrizione:}
un utente deve poter modificare il valore di opacità del materiale selezionato. \\ \\
\textbf{Precondizione:}
il sistema conosce il materiale da modificare ed è in attesa di input per quanto riguarda la modifica dello stesso. \\ \\
\textbf{Postcondizione:}
il sistema conosce il nuovo valore del parametro di opacità impostato dall'utente. \\ \\
\textbf{Scenario Principale:}
\begin{itemize}
\item l'utente imposta il valore del parametro di opacità associato al materiale selezionato
\\ \\ \end{itemize}


\subsection{UC1.2.4.5 Rimuovi mesh}
\textbf{Diagramma associato:}
\ref{UC1.2.4} \\ \\
\textbf{Attori Coinvolti:}
Utente. \\ \\
\textbf{Scopo e Descrizione:}
un utente deve poter rimuovere la mesh selezionata. \\ \\
\textbf{Precondizione:}
il sistema è in attesa che l'utente selezioni una funzionalità di modifica della mesh. \\ \\
\textbf{Postcondizione:}
il sistema ha rimosso la mesh selezionata in precedenza secondo le richieste utente. \\ \\
\textbf{Scenario Principale:}
\begin{itemize}
\item l'utente rimuove la mesh
\\ \\ \end{itemize}


\subsection{UC1.2.5 Seleziona camera}
\textbf{Diagramma associato:}
\ref{UC1.2} \\ \\
\textbf{Attori Coinvolti:}
Utente. \\ \\
\textbf{Scopo e Descrizione:}
un utente deve poter selezionare la camera che intende modificare. \\ \\
\textbf{Precondizione:}
il sistema è in attesa di input  per quanto riguarda la modifica della scena 3D. \\ \\
\textbf{Postcondizione:}
il sistema conosce la camera selezionata dall'utente. \\ \\
\textbf{Scenario Principale:}
\begin{itemize}
\item l'utente seleziona la camera che intende modificare
\\ \\ \end{itemize}


\subsection{UC1.2.6 Modifica camera}
\textbf{Diagramma associato:}
\ref{UC1.2.6} \\ \\
\textbf{Attori Coinvolti:}
Utente. \\ \\
\textbf{Scopo e Descrizione:}
un utente deve poter modificare la camera selezionata. \\ \\
\textbf{Precondizione:}
il sistema conosce la camera da modificare (UC1.2.5). \\ \\
\textbf{Postcondizione:}
il sistema ha modificato la camera secondo le richieste dell'utente. \\ \\
\textbf{Scenario Principale:}
\begin{itemize}
\item l'utente modifica la posizione della camera (UC1.2.6.1)
\item l'utente rimuove la camera (UC1.2.6.2)
\\ \\ \end{itemize}
\begin{figure}[h!]
\centering
\includegraphics[width=\textwidth]{{{UC1.2.6}}}
\caption{UC1.2.6 Modifica camera}
\label{UC1.2.6}
\end{figure}


\subsection{UC1.2.6.1 Modifica posizione camera}
\textbf{Diagramma associato:}
\ref{UC1.2.6.1} \\ \\
\textbf{Attori Coinvolti:}
Utente. \\ \\
\textbf{Scopo e Descrizione:}
un utente deve poter modificare la posizione della camera. \\ \\
\textbf{Precondizione:}
il sistema è in attesa che l'utente selezioni una funzionalità di modifica della camera. \\ \\
\textbf{Postcondizione:}
il sistema ha riposizionato la camera secondo le richieste utente. \\ \\
\textbf{Scenario Principale:}
\begin{itemize}
\item l'utente trasla la camera (UC1.2.6.1.1)
\item l'utente ruota la camera attorno alla scena 3D (UC1.2.6.1.2)
\\ \\ \end{itemize}
\begin{figure}[h!]
\centering
\includegraphics[width=\textwidth]{{{UC1.2.6.1}}}
\caption{UC1.2.6.1 Modifica posizione camera}
\label{UC1.2.6.1}
\end{figure}


\subsection{UC1.2.6.1.1 Trasla camera}
\textbf{Diagramma associato:}
\ref{UC1.2.6.1.1} \\ \\
\textbf{Attori Coinvolti:}
Utente. \\ \\
\textbf{Scopo e Descrizione:}
un utente deve poter traslare la camera. \\ \\
\textbf{Precondizione:}
il sistema è in attesa di input per quanto riguarda la modifica della posizione della camera. \\ \\
\textbf{Postcondizione:}
il sistema ha in memoria la scena 3D ed ha traslato la camera secondo l'input utente. \\ \\
\textbf{Scenario Principale:}
\begin{itemize}
\item l'utente sceglie di quanto avvicinare la camera verso il centro della scena 3D (UC1.2.6.1.1.1)
\item l'utente sceglie di quanto allontanare la camera dal centro della scena 3D (UC1.2.6.1.1.2)
\item l'utente sceglie i valori secondo i quali traslare la camera della scena 3D (UC1.2.6.1.1.3)
\\ \\ \end{itemize}
\begin{figure}[h!]
\centering
\includegraphics[width=\textwidth]{{{UC1.2.6.1.1}}}
\caption{UC1.2.6.1.1 Trasla camera}
\label{UC1.2.6.1.1}
\end{figure}


\subsection{UC1.2.6.1.1.1 Avvicina camera al centro della scena 3D}
\textbf{Diagramma associato:}
\ref{UC1.2.6.1.1} \\ \\
\textbf{Attori Coinvolti:}
Utente. \\ \\
\textbf{Scopo e Descrizione:}
un utente deve poter avvicinare la camera al centro della scena 3D. \\ \\
\textbf{Precondizione:}
il sistema è in attesa di input per quanto riguarda la traslazione della camera. \\ \\
\textbf{Postcondizione:}
il sistema ha avvicinato la camera verso il centro della scena 3D. \\ \\
\textbf{Scenario Principale:}
\begin{itemize}
\item l'utente sceglie di quanto avvicinare la camera verso il centro della scena 3D
\\ \\ \end{itemize}


\subsection{UC1.2.6.1.1.2 Allontana camera dal centro della scena 3D}
\textbf{Diagramma associato:}
\ref{UC1.2.6.1.1} \\ \\
\textbf{Attori Coinvolti:}
Utente. \\ \\
\textbf{Scopo e Descrizione:}
un utente deve poter allontanare la camera dal centro della scena 3D. \\ \\
\textbf{Precondizione:}
il sistema è in attesa di input per quanto riguarda la traslazione della camera. \\ \\
\textbf{Postcondizione:}
il sistema ha allontanato la camera dal centro della scena 3D. \\ \\
\textbf{Scenario Principale:}
\begin{itemize}
\item l'utente sceglie di quanto allontanare la camera dal centro della scena 3D
\\ \\ \end{itemize}


\subsection{UC1.2.6.1.1.3 Imposta traslazione della camera}
\textbf{Diagramma associato:}
\ref{UC1.2.6.1.1} \\ \\
\textbf{Attori Coinvolti:}
Utente. \\ \\
\textbf{Scopo e Descrizione:}
un utente deve poter impostare lo spostamento della camera. \\ \\
\textbf{Precondizione:}
il sistema è in attesa di input per quanto riguarda la traslazione della camera. \\ \\
\textbf{Postcondizione:}
il sistema ha spostato la camera secondo le specifiche richieste dell'utente. \\ \\
\textbf{Scenario Principale:}
\begin{itemize}
\item l'utente sceglie i valori secondo i quali traslare la camera della scena 3D
\\ \\ \end{itemize}


\subsection{UC1.2.6.1.2 Ruota camera attorno alla scena 3D}
\textbf{Diagramma associato:}
\ref{UC1.2.6.1} \\ \\
\textbf{Attori Coinvolti:}
Utente. \\ \\
\textbf{Scopo e Descrizione:}
un utente deve poter ruotare la camera attorno alla scena 3D. \\ \\
\textbf{Precondizione:}
il sistema è in attesa di input per quanto riguarda la modifica della posizione della camera. \\ \\
\textbf{Postcondizione:}
il sistema ha in memoria la scena 3D ed ha ruotato la camera 3D secondo l'input utente. \\ \\
\textbf{Scenario Principale:}
\begin{itemize}
\item l'utente ruota la camera attorno la scena
\\ \\ \end{itemize}


\subsection{UC1.2.6.2 Rimuovi camera}
\textbf{Diagramma associato:}
\ref{UC1.2.6} \\ \\
\textbf{Attori Coinvolti:}
Utente. \\ \\
\textbf{Scopo e Descrizione:}
un utente deve poter rimuovere la camera selezionata. \\ \\
\textbf{Precondizione:}
il sistema è in attesa che l'utente selezioni una funzionalità di modifica della camera. \\ \\
\textbf{Postcondizione:}
il sistema ha rimosso la camera selezionata in precedenza secondo le richieste utente. \\ \\
\textbf{Scenario Principale:}
\begin{itemize}
\item l'utente rimuove la camera
\\ \\ \end{itemize}


\subsection{UC1.2.7 Modifica fonte di luce}
\textbf{Diagramma associato:}
\ref{UC1.2.7} \\ \\
\textbf{Attori Coinvolti:}
Utente. \\ \\
\textbf{Scopo e Descrizione:}
un utente deve poter modificare le caratteristiche di una fonte di luce (quali posizione, intensità, colore e tipologia). \\ \\
\textbf{Precondizione:}
il sistema conosce la fonte di luce da modficare (UC1.2.1). \\ \\
\textbf{Postcondizione:}
il sistema ha modificato le caratteristiche di una fonte di luce secondo le richieste dell’utente. \\ \\
\textbf{Scenario Principale:}
\begin{itemize}
\item l'utente rimuove la fonte di luce (UC1.2.7.1)
\item l’utente modifica la posizione della fonte di luce (UC1.2.7.2)
\item l’utente modifica l’attenuazione della fonte di luce (UC1.2.7.3)
\item l’utente modifica il colore della fonte di luce (UC.1.2.7.4)
\item l’utente modifica la tipologia della fonte di luce (UC1.2.7.5)
\\ \\ \end{itemize}
\begin{figure}[h!]
\centering
\includegraphics[width=\textwidth]{{{UC1.2.7}}}
\caption{UC1.2.7 Modifica fonte di luce}
\label{UC1.2.7}
\end{figure}


\subsection{UC1.2.7.1 Rimuovi luce}
\textbf{Diagramma associato:}
\ref{UC1.2.7} \\ \\
\textbf{Attori Coinvolti:}
Utente. \\ \\
\textbf{Scopo e Descrizione:}
un utente deve poter rimuovere una fonte di luce. \\ \\
\textbf{Precondizione:}
il sistema è in attesa che l'utente selezioni una funzionalità di modifica delle caratteristiche della fonte di luce. \\ \\
\textbf{Postcondizione:}
il sistema ha rimosso la fonte di luce selezionata in precedenza secondo le richieste utente. \\ \\
\textbf{Scenario Principale:}
\begin{itemize}
\item l'utente sceglie di rimuovere la fonte di luce
\\ \\ \end{itemize}


\subsection{UC1.2.7.2 Modifica posizione luce}
\textbf{Diagramma associato:}
\ref{UC1.2.7} \\ \\
\textbf{Attori Coinvolti:}
Utente. \\ \\
\textbf{Scopo e Descrizione:}
un utente deve poter modificare la posizione di una fonte di luce. \\ \\
\textbf{Precondizione:}
il sistema è in attesa che l'utente selezioni una funzionalità di modifica delle caratteristiche della fonte di luce. \\ \\
\textbf{Postcondizione:}
il sistema ha riposizionato la fonte di luce secondo le impostazioni utente. \\ \\
\textbf{Scenario Principale:}
\begin{itemize}
\item l'utente riposiziona la fonte di luce
\\ \\ \end{itemize}


\subsection{UC1.2.7.3 Modifica attenuazione luce}
\textbf{Diagramma associato:}
\ref{UC1.2.7.3} \\ \\
\textbf{Attori Coinvolti:}
Utente. \\ \\
\textbf{Scopo e Descrizione:}
l’utente deve poter modificare i parametri d’attenuazione di una fonte di luce. \\ \\
\textbf{Precondizione:}
il sistema è in attesa che l'utente selezioni una funzionalità di modifica delle caratteristiche della fonte di luce. \\ \\
\textbf{Postcondizione:}
il sistema ha modificato i parametri di attenuazione secondo l'input utente. \\ \\
\textbf{Scenario Principale:}
\begin{itemize}
\item l’utente modifica l’attenuazione costante (UC1.2.7.3.1)
\item l’utente modifica l’attenuazione lineare (UC1.2.7.3.2)
\item  l’utente modifica l’attenuazione quadratica (UC1.2.7.3.3)
\\ \\ \end{itemize}
\begin{figure}[h!]
\centering
\includegraphics[width=\textwidth]{{{UC1.2.7.3}}}
\caption{UC1.2.7.3 Modifica attenuazione luce}
\label{UC1.2.7.3}
\end{figure}


\subsection{UC1.2.7.3.1 Modifica attenuazione costante}
\textbf{Diagramma associato:}
\ref{UC1.2.7.3} \\ \\
\textbf{Attori Coinvolti:}
Utente. \\ \\
\textbf{Scopo e Descrizione:}
l'utente deve poter modificare il parametro di attenuazione costante. \\ \\
\textbf{Precondizione:}
il sistema è in attesa di input per la modifica dei parametri di attenuazione della luce. \\ \\
\textbf{Postcondizione:}
il sistema conosce il valore del parametro di attenuazione costante scelto dall'utente. \\ \\
\textbf{Scenario Principale:}
\begin{itemize}
\item l'utente modifica il valore del parametro di attenuazione costante
\\ \\ \end{itemize}


\subsection{UC1.2.7.3.2 Modifica attenuazione lineare}
\textbf{Diagramma associato:}
\ref{UC1.2.7.3} \\ \\
\textbf{Attori Coinvolti:}
Utente. \\ \\
\textbf{Scopo e Descrizione:}
l'utente deve poter modificare il parametro di attenuazione lineare. \\ \\
\textbf{Precondizione:}
il sistema è in attesa di input per la modifica dei parametri di attenuazione della luce. \\ \\
\textbf{Postcondizione:}
il sistema conosce il valore del parametro di attenuazione lineare scelto dall'utente. \\ \\
\textbf{Scenario Principale:}
\begin{itemize}
\item l'utente modifica il valore del parametro di attenuazione lineare
\\ \\ \end{itemize}


\subsection{UC1.2.7.3.3 Modifica attenuazione quadratica}
\textbf{Diagramma associato:}
\ref{UC1.2.7.3} \\ \\
\textbf{Attori Coinvolti:}
Utente. \\ \\
\textbf{Scopo e Descrizione:}
l'utente deve poter modificare il paramentro di attenuazione quadratica. \\ \\
\textbf{Precondizione:}
il sistema è in attesa di input per la modifica dei parametri di attenuazione della luce. \\ \\
\textbf{Postcondizione:}
il sistema conosce il valore del parametro di attenuazione quadratica scelto dall'utente. \\ \\
\textbf{Scenario Principale:}
\begin{itemize}
\item l'utente modifica il valore del parametro di attenuazione quadratica
\\ \\ \end{itemize}


\subsection{UC1.2.7.4 Modifica colore componenti luce}
\textbf{Diagramma associato:}
\ref{UC1.2.7.4} \\ \\
\textbf{Attori Coinvolti:}
Utente. \\ \\
\textbf{Scopo e Descrizione:}
un utente deve poter modificare il colore di una fonte di luce, alterando i valori \underline{RGB} dei componenti che la compongono. \\ \\
\textbf{Precondizione:}
il sistema è in attesa che l'utente selezioni una funzionalità di modifica delle caratteristiche della fonte di luce. \\ \\
\textbf{Postcondizione:}
il sistema ha modificato il colore della fonte di luce secondo le richieste dell'utente. \\ \\
\textbf{Scenario Principale:}
\begin{itemize}
\item l’utente sceglie la componente di luce che vuole modificare (UC1.2.7.4.1)
\item l'utente imposta i valori RGB (UC1.2.7.4.2)
\\ \\ \end{itemize}
\begin{figure}[h!]
\centering
\includegraphics[width=\textwidth]{{{UC1.2.7.4}}}
\caption{UC1.2.7.4 Modifica colore componenti luce}
\label{UC1.2.7.4}
\end{figure}


\subsection{UC1.2.7.4.1 Seleziona componente luce}
\textbf{Diagramma associato:}
\ref{UC1.2.7.4.1} \\ \\
\textbf{Attori Coinvolti:}
Utente. \\ \\
\textbf{Scopo e Descrizione:}
un utente deve poter selezionare la componente di luce il cui colore vuole modificare. \\ \\
\textbf{Precondizione:}
il sistema è in attesa di input per la modifica del colore della componente della fonte di luce. \\ \\
\textbf{Postcondizione:}
il sistema ha in memoria la scena 3D e conosce la componente della fonte di luce selezionata dall’utente. \\ \\
\textbf{Scenario Principale:}
\begin{itemize}
\item l’utente seleziona la componente ambient (UC1.2.7.4.1.1)
\item l’utente seleziona la componente specular (UC1.2.7.4.1.2)
\item l’utente seleziona la componente diffuse (UC1.2.7.4.1.3)
\\ \\ \end{itemize}
\begin{figure}[h!]
\centering
\includegraphics[width=\textwidth]{{{UC1.2.7.4.1}}}
\caption{UC1.2.7.4.1 Seleziona componente luce}
\label{UC1.2.7.4.1}
\end{figure}


\subsection{UC1.2.7.4.1.1 Seleziona componente ambient}
\textbf{Diagramma associato:}
\ref{UC1.2.7.4.1} \\ \\
\textbf{Attori Coinvolti:}
Utente. \\ \\
\textbf{Scopo e Descrizione:}
un utente deve poter selezionare la componente ambient di una fonte di luce. \\ \\
\textbf{Precondizione:}
il sistema è in attesa che l'utente scelga la componente il cui colore vuole cambiare. \\ \\
\textbf{Postcondizione:}
il sistema sa che l'utente ha selezionato la componente ambient. \\ \\
\textbf{Scenario Principale:}
\begin{itemize}
\item l'utente seleziona la componente ambient
\\ \\ \end{itemize}


\subsection{UC1.2.7.4.1.2 Seleziona componente specular}
\textbf{Diagramma associato:}
\ref{UC1.2.7.4.1} \\ \\
\textbf{Attori Coinvolti:}
Utente. \\ \\
\textbf{Scopo e Descrizione:}
un utente deve poter selezionare la componente specular di una fonte di luce. \\ \\
\textbf{Precondizione:}
il sistema è in attesa che l'utente scelga la componente il cui colore vuole cambiare. \\ \\
\textbf{Postcondizione:}
il sistema sa che l'utente ha selezionato la componente specular. \\ \\
\textbf{Scenario Principale:}
\begin{itemize}
\item l'utente seleziona la componente specular
\\ \\ \end{itemize}


\subsection{UC1.2.7.4.1.3 Seleziona componente diffuse}
\textbf{Diagramma associato:}
\ref{UC1.2.7.4.1} \\ \\
\textbf{Attori Coinvolti:}
Utente. \\ \\
\textbf{Scopo e Descrizione:}
un utente deve poter selezionare la componente diffuse di una fonte di luce. \\ \\
\textbf{Precondizione:}
il sistema è in attesa che l'utente scelga la componente il cui colore vuole cambiare. \\ \\
\textbf{Postcondizione:}
il sistema sa che l'utente ha selezionato la componente diffuse. \\ \\
\textbf{Scenario Principale:}
\begin{itemize}
\item l'utente seleziona la componente diffuse
\\ \\ \end{itemize}


\subsection{UC1.2.7.4.2 Imposta valori RGB}
\textbf{Diagramma associato:}
\ref{UC1.2.7.4.2} \\ \\
\textbf{Attori Coinvolti:}
Utente. \\ \\
\textbf{Scopo e Descrizione:}
un utente deve poter impostare i valori RGB di una componente di luce. \\ \\
\textbf{Precondizione:}
il sistema conosce la componente selezionata (UC1.2.7.4.1), ed è in attesa di input per la modifica del colore. \\ \\
\textbf{Postcondizione:}
il sistema conosce i valori RGB impostati dall'utente. \\ \\
\textbf{Scenario Principale:}
\begin{itemize}
\item l'utente imposta il valore red (UC1.2.7.4.2.1)
\item l'utente imposta il valore green (UC1.2.7.4.2.2)
\item l'utente imposta il valore blue (UC1.2.7.4.2.3)
\\ \\ \end{itemize}
\begin{figure}[h!]
\centering
\includegraphics[width=\textwidth]{{{UC1.2.7.4.2}}}
\caption{UC1.2.7.4.2 Imposta valori RGB}
\label{UC1.2.7.4.2}
\end{figure}


\subsection{UC1.2.7.4.2.1 Imposta valore red}
\textbf{Diagramma associato:}
\ref{UC1.2.7.4.2} \\ \\
\textbf{Attori Coinvolti:}
Utente. \\ \\
\textbf{Scopo e Descrizione:}
un utente deve poter impostare il valore red di una componente di una fonte di luce. \\ \\
\textbf{Precondizione:}
il sistema è in attesa di input per quanto riguarda i valori RGB. \\ \\
\textbf{Postcondizione:}
il sistema conosce il valore red impostato dall'utente. \\ \\
\textbf{Scenario Principale:}
\begin{itemize}
\item l'utente imposta il valore red
\\ \\ \end{itemize}


\subsection{UC1.2.7.4.2.2 Imposta valore green}
\textbf{Diagramma associato:}
\ref{UC1.2.7.4.2} \\ \\
\textbf{Attori Coinvolti:}
Utente. \\ \\
\textbf{Scopo e Descrizione:}
un utente deve poter impostare il valore green di una componente di una fonte di luce. \\ \\
\textbf{Precondizione:}
il sistema è in attesa di input per quanto riguarda i valori RGB. \\ \\
\textbf{Postcondizione:}
il sistema conosce il valore green impostato dall'utente. \\ \\
\textbf{Scenario Principale:}
\begin{itemize}
\item l'utente imposta il valore green
\\ \\ \end{itemize}


\subsection{UC1.2.7.4.2.3 Imposta valore blue}
\textbf{Diagramma associato:}
\ref{UC1.2.7.4.2} \\ \\
\textbf{Attori Coinvolti:}
Utente. \\ \\
\textbf{Scopo e Descrizione:}
un utente deve poter impostare il valore blue di una componente di una fonte di luce. \\ \\
\textbf{Precondizione:}
il sistema è in attesa di input per quanto riguarda i valori RGB. \\ \\
\textbf{Postcondizione:}
il sistema conosce il valore blue impostato dall'utente. \\ \\
\textbf{Scenario Principale:}
\begin{itemize}
\item l'utente imposta il valore blue
\\ \\ \end{itemize}


\subsection{UC1.2.7.5 Modifica tipologia luce}
\textbf{Diagramma associato:}
\ref{UC1.2.7.5} \\ \\
\textbf{Attori Coinvolti:}
Utente. \\ \\
\textbf{Scopo e Descrizione:}
un utente deve poter modificare la tipologia corrente di una fonte di luce. \\ \\
\textbf{Precondizione:}
il sistema è in attesa che l'utente selezioni una funzionalità di modifica delle caratteristiche della fonte di luce. \\ \\
\textbf{Postcondizione:}
il sistema ha modificato la tipologia della fonte di luce corrente secondo le richieste utente. \\ \\
\textbf{Scenario Principale:}
\begin{itemize}
\item l'utente modifica la tipologia di luce corrente in \underline{spotlight} (UC1.2.7.5.1)
\item l'utente modifica la tipologia di luce corrente in \underline{omni light} (UC1.2.7.5.2)
\item l'utente modifica la tipologia di luce corrente in \underline{directional light} (UC1.2.7.5.3)
\\ \\ \end{itemize}
\begin{figure}[h!]
\centering
\includegraphics[width=\textwidth]{{{UC1.2.7.5}}}
\caption{UC1.2.7.5 Modifica tipologia luce}
\label{UC1.2.7.5}
\end{figure}


\subsection{UC1.2.7.5.1 Modifica la tipologia di luce corrente in spotlight}
\textbf{Diagramma associato:}
\ref{UC1.2.7.5} \\ \\
\textbf{Attori Coinvolti:}
Utente. \\ \\
\textbf{Scopo e Descrizione:}
un utente deve poter modificare la tipologia corrente di una fonte di luce in spotlight. \\ \\
\textbf{Precondizione:}
il sistema è in attesa di input per quanto riguarda la modifica della tipologia di luce corrente. \\ \\
\textbf{Postcondizione:}
il sistema conosce la richiesta di modifica tipologia corrente di luce in spotlight. \\ \\
\textbf{Scenario Principale:}
\begin{itemize}
\item l'utente imposta la tipologia di luce corrente in spotlight
\\ \\ \end{itemize}


\subsection{UC1.2.7.5.2 Modifica la tipologia di luce corrente in omni light}
\textbf{Diagramma associato:}
\ref{UC1.2.7.5} \\ \\
\textbf{Attori Coinvolti:}
Utente. \\ \\
\textbf{Scopo e Descrizione:}
un utente deve poter modificare la tipologia corrente di una fonte di luce in omni light. \\ \\
\textbf{Precondizione:}
il sistema è in attesa di input per quanto riguarda la modifica della tipologia di luce corrente. \\ \\
\textbf{Postcondizione:}
il sistema conosce la richiesta di modifica tipologia corrente di luce in omni light. \\ \\
\textbf{Scenario Principale:}
\begin{itemize}
\item l'utente imposta la tipologia di luce corrente in omni light
\\ \\ \end{itemize}


\subsection{UC1.2.7.5.3 Modifica la tipologia di luce corrente in directional light}
\textbf{Diagramma associato:}
\ref{UC1.2.7.5} \\ \\
\textbf{Attori Coinvolti:}
Utente. \\ \\
\textbf{Scopo e Descrizione:}
un utente deve poter modificare la tipologia corrente di una fonte di luce in directional light. \\ \\
\textbf{Precondizione:}
il sistema è in attesa di input per quanto riguarda la modifica della tipologia di luce corrente. \\ \\
\textbf{Postcondizione:}
il sistema conosce la richiesta di modifica tipologia corrente di luce in directional light. \\ \\
\textbf{Scenario Principale:}
\begin{itemize}
\item l'utente imposta la tipologia di luce corrente in directional light
\\ \\ \end{itemize}


\subsection{UC1.3 Esporta modello 3D}
\textbf{Diagramma associato:}
\ref{UC1.3} \\ \\
\textbf{Attori Coinvolti:}
Utente. \\ \\
\textbf{Scopo e Descrizione:}
un utente deve poter scegliere in che formato esportare la scena 3D, scegliere la \underline{directory} di destinazione e il nome del file. La scelta del formato comporta la scelta della precisione. \\ \\
\textbf{Precondizione:}
il sistema ha in memoria la scena 3D caricata in precedenza (UC1.1), il sistema è in attesa che l'utente selezioni una funzionalità. \\ \\
\textbf{Postcondizione:}
il sistema ha esportato la scena 3D secondo le richieste dell’utente. \\ \\
\textbf{Scenario Principale:}
\begin{itemize}
\item l’utente sceglie il formato verso il quale esportare (UC1.3.1)
\item l’utente sceglie la precisione (UC1.3.2)
\item l’utente sceglie la directory di destinazione (UC1.3.3)
\item l’utente sceglie il nome del file (UC1.3.4)
\\ \\ \end{itemize}
\begin{figure}[h!]
\centering
\includegraphics[width=\textwidth]{{{UC1.3}}}
\caption{UC1.3 Esporta modello 3D}
\label{UC1.3}
\end{figure}


\subsection{UC1.3.1 Seleziona formato}
\textbf{Diagramma associato:}
\ref{UC1.3.1} \\ \\
\textbf{Attori Coinvolti:}
Utente. \\ \\
\textbf{Scopo e Descrizione:}
un utente deve poter scegliere il formato verso il quale esportare. \\ \\
\textbf{Precondizione:}
il sistema è in attesa di input per quanto riguarda l'esportazione. \\ \\
\textbf{Postcondizione:}
il sistema conosce il formato scelto dall’utente. \\ \\
\textbf{Scenario Principale:}
\begin{itemize}
\item l’utente sceglie il formato JSON (UC1.3.1.1)
\item l’utente sceglie il formato XML
\\ \\ \end{itemize}
\begin{figure}[h!]
\centering
\includegraphics[width=\textwidth]{{{UC1.3.1}}}
\caption{UC1.3.1 Seleziona formato}
\label{UC1.3.1}
\end{figure}


\subsection{UC1.3.1.1 Seleziona formato JSON}
\textbf{Diagramma associato:}
\ref{UC1.3.1.1} \\ \\
\textbf{Attori Coinvolti:}
Utente. \\ \\
\textbf{Scopo e Descrizione:}
un utente deve poter scegliere se esportare la scena 3D in JSON “compatto” o “leggibile”. \\ \\
\textbf{Precondizione:}
il sistema è in attesa di input per quanto riguarda il formato di esportazione. \\ \\
\textbf{Postcondizione:}
il sistema conosce il tipo di formato JSON scelto dall’utente. \\ \\
\textbf{Scenario Principale:}
\begin{itemize}
\item  l’utente sceglie il formato JSON “compatto”
\item l’utente sceglie il formato JSON ”leggibile”
\\ \\ \end{itemize}
\begin{figure}[h!]
\centering
\includegraphics[width=\textwidth]{{{UC1.3.1.1}}}
\caption{UC1.3.1.1 Seleziona formato JSON}
\label{UC1.3.1.1}
\end{figure}


\subsection{UC1.3.1.1.1 Seleziona formato JSON compatto}
\textbf{Diagramma associato:}
\ref{UC1.3.1.1} \\ \\
\textbf{Attori Coinvolti:}
Utente. \\ \\
\textbf{Scopo e Descrizione:}
un utente deve poter esportare la scena 3D nel formato JSON "compatto". \\ \\
\textbf{Precondizione:}
il sistema è in attesa di input per quanto riguarda il tipo JSON di esportazione. \\ \\
\textbf{Postcondizione:}
il sistema conosce il formato del file di esportazione scelto dall'utente. \\ \\
\textbf{Scenario Principale:}
\begin{itemize}
\item l'utente seleziona il formato JSON "compatto"
\\ \\ \end{itemize}


\subsection{UC1.3.1.1.2 Seleziona formato JSON leggibile}
\textbf{Diagramma associato:}
\ref{UC1.3.1.1} \\ \\
\textbf{Attori Coinvolti:}
Utente. \\ \\
\textbf{Scopo e Descrizione:}
un utente deve poter esportare la scena 3D nel formato JSON "leggibile". \\ \\
\textbf{Precondizione:}
il sistema è in attesa di input per quanto riguarda il tipo JSON di esportazione. \\ \\
\textbf{Postcondizione:}
il sistema conosce il formato del file di esportazione scelto dall'utente. \\ \\
\textbf{Scenario Principale:}
\begin{itemize}
\item l'utente seleziona il formato JSON "leggibile"
\\ \\ \end{itemize}


\subsection{UC1.3.1.2 Seleziona formato XML}
\textbf{Diagramma associato:}
\ref{UC1.3.1} \\ \\
\textbf{Attori Coinvolti:}
Utente. \\ \\
\textbf{Scopo e Descrizione:}
un utente deve poter esportare la scena 3D nel formato XML. \\ \\
\textbf{Precondizione:}
il sistema è in attesa di input per quanto riguarda il formato di esportazione. \\ \\
\textbf{Postcondizione:}
il sistema conosce il formato del file di esportazione scelto dall'utente. \\ \\
\textbf{Scenario Principale:}
\begin{itemize}
\item l'utente seleziona il formato XML
\\ \\ \end{itemize}


\subsection{UC1.3.2 Sceglie precisione}
\textbf{Diagramma associato:}
\ref{UC1.3.2} \\ \\
\textbf{Attori Coinvolti:}
Utente. \\ \\
\textbf{Scopo e Descrizione:}
un utente deve poter scegliere se esportare la scena 3D in singola o doppia precisione. \\ \\
\textbf{Precondizione:}
il sistema conosce il formato di esportazione, ed è in attesa di input per quanto riguarda l'esportazione. \\ \\
\textbf{Postcondizione:}
il sistema conosce la precisione scelta dall’utente. \\ \\
\textbf{Scenario Principale:}
\begin{itemize}
\item l’utente sceglie la precisione float
\item l’utente sceglie la precisione double
\\ \\ \end{itemize}
\begin{figure}[h!]
\centering
\includegraphics[width=\textwidth]{{{UC1.3.2}}}
\caption{UC1.3.2 Sceglie precisione}
\label{UC1.3.2}
\end{figure}


\subsection{UC1.3.2.1 Seleziona precisione float}
\textbf{Diagramma associato:}
\ref{UC1.3.2} \\ \\
\textbf{Attori Coinvolti:}
Utente. \\ \\
\textbf{Scopo e Descrizione:}
un utente deve poter esportare la scena 3D in singola precisione. \\ \\
\textbf{Precondizione:}
il sistema è in attesa di input per quanto riguarda la precisione dell'esportazione. \\ \\
\textbf{Postcondizione:}
il sistema conosce la precisione scelta dall’utente. \\ \\
\textbf{Scenario Principale:}
\begin{itemize}
\item l'utente sceglie la precisione float
\\ \\ \end{itemize}


\subsection{UC1.3.2.2 Seleziona precisione double}
\textbf{Diagramma associato:}
\ref{UC1.3.2} \\ \\
\textbf{Attori Coinvolti:}
Utente. \\ \\
\textbf{Scopo e Descrizione:}
un utente deve poter esportare la scena 3D in doppia precisione. \\ \\
\textbf{Precondizione:}
il sistema ha in memoria la scena 3D caricata in precedenza e conosce il formato verso il quale l’utente vuole esportare (UC1.3.1). \\ \\
\textbf{Postcondizione:}
il sistema conosce la precisione scelta dall’utente. \\ \\
\textbf{Scenario Principale:}
\begin{itemize}
\item l'utente sceglie la precisione double
\\ \\ \end{itemize}


\subsection{UC1.3.3 Sceglie directory destinazione}
\textbf{Diagramma associato:}
\ref{UC1.3} \\ \\
\textbf{Attori Coinvolti:}
Utente. \\ \\
\textbf{Scopo e Descrizione:}
un utente deve poter scegliere la directory di destinazione del file di esportazione. \\ \\
\textbf{Precondizione:}
il sistema conosce il formato di esportazione, ed è in attesa di input per quanto riguarda l'esportazione. \\ \\
\textbf{Postcondizione:}
il sistema conosce la directory di destinazione selezionata dall'utente. \\ \\
\textbf{Scenario Principale:}
\begin{itemize}
\item l'utente seleziona la directory di destinazione
\\ \\ \end{itemize}


\subsection{UC1.3.4 Sceglie nome file}
\textbf{Diagramma associato:}
\ref{UC1.3} \\ \\
\textbf{Attori Coinvolti:}
Utente. \\ \\
\textbf{Scopo e Descrizione:}
un utente deve poter scegliere il nome del file di esportazione. \\ \\
\textbf{Precondizione:}
il sistema conosce il formato di esportazione, è in attesa di input per quanto riguarda l'esportazione, e conosce la directory di destinazione. \\ \\
\textbf{Postcondizione:}
il sistema conosce il nome del file scelto dall'utente. \\ \\
\textbf{Scenario Principale:}
\begin{itemize}
\item l'utente sceglie il nome del file
\\ \\ \end{itemize}


\subsection{UC1.4 Visualizza anteprima scena 3D}
\textbf{Diagramma associato:}
\ref{UC1.4} \\ \\
\textbf{Attori Coinvolti:}
Utente. \\ \\
\textbf{Scopo e Descrizione:}
un utente deve poter visualizzare l’anteprima della scena 3D. \\ \\
\textbf{Precondizione:}
il sistema ha in memoria la scena 3D caricata in precedenza (UC1.1), il sistema è in attesa che l'utente selezioni una funzionalità. \\ \\
\textbf{Postcondizione:}
l sistema sta visualizzando la scena 3D secondo le richieste dell’utente. \\ \\
\textbf{Scenario Principale:}
\begin{itemize}
\item l’utente visualizza l’anteprima semplice (UC1.4.1)
\item  l’utente esegue il rendering
\\ \\ \end{itemize}
\begin{figure}[h!]
\centering
\includegraphics[width=\textwidth]{{{UC1.4}}}
\caption{UC1.4 Visualizza anteprima scena 3D}
\label{UC1.4}
\end{figure}


\subsection{UC1.4.1 Visualizza anteprima semplice}
\textbf{Diagramma associato:}
\ref{UC1.4.1} \\ \\
\textbf{Attori Coinvolti:}
Utente. \\ \\
\textbf{Scopo e Descrizione:}
un utente deve avere la possibilità di spostare il punto d’osservazione della scena 3D. \\ \\
\textbf{Precondizione:}
il sistema è in attesa che l'utente selezioni una funzionalità per l'anteprima 3D. \\ \\
\textbf{Postcondizione:}
il sistema ha in memoria la scena 3D e ha spostato il punto d’osservazione secondo le richieste dell’utente. \\ \\
\textbf{Scenario Principale:}
\begin{itemize}
\item l’utente trasla  il punto d’osservazione
\item l’utente ruota il punto d’osservazione attorno alla scena
\\ \\ \end{itemize}
\begin{figure}[h!]
\centering
\includegraphics[width=\textwidth]{{{UC1.4.1}}}
\caption{UC1.4.1 Visualizza anteprima semplice}
\label{UC1.4.1}
\end{figure}


\subsection{UC1.4.1.1 Trasla punto d'osservazione}
\textbf{Diagramma associato:}
\ref{UC1.4.1.1} \\ \\
\textbf{Attori Coinvolti:}
Utente. \\ \\
\textbf{Scopo e Descrizione:}
un utente deve poter traslare il \underline{punto d'osservazione} della scena 3D. \\ \\
\textbf{Precondizione:}
il sistema sta visualizzando un'anteprima semplice ed è in attesa che l'utente selezioni una funzionalità. \\ \\
\textbf{Postcondizione:}
il sistema ha in memoria la scena 3D ed ha traslato il punto d'osservazione della scena secondo l'input utente. \\ \\
\textbf{Scenario Principale:}
\begin{itemize}
\item l'utente trasla il punto d'osservazione
\\ \\ \end{itemize}
\begin{figure}[h!]
\centering
\includegraphics[width=\textwidth]{{{UC1.4.1.1}}}
\caption{UC1.4.1.1 Trasla punto d'osservazione}
\label{UC1.4.1.1}
\end{figure}


\subsection{UC1.4.1.1.1 Avvicina punto di osservazione al centro della scena 3D}
\textbf{Diagramma associato:}
\ref{UC1.4.1.1} \\ \\
\textbf{Attori Coinvolti:}
Utente. \\ \\
\textbf{Scopo e Descrizione:}
un utente deve poter avvicinare il punto d'osservazione al centro della scena 3D. \\ \\
\textbf{Precondizione:}
il sistema è in attesa che l'utente selezioni una funzionalità per quanto riguarda la traslazione del punto di osservazione. \\ \\
\textbf{Postcondizione:}
il sistema ha avvicinato il punto di osservazione verso il centro della scena 3D. \\ \\
\textbf{Scenario Principale:}
\begin{itemize}
\item l'utente sceglie di quanto avvicinare il punto di osservazione verso il centro della scena 3D
\\ \\ \end{itemize}


\subsection{UC1.4.1.1.2 Allontana punto di osservazione dal centro della scena 3D}
\textbf{Diagramma associato:}
\ref{UC1.4.1.1} \\ \\
\textbf{Attori Coinvolti:}
Utente. \\ \\
\textbf{Scopo e Descrizione:}
un utente deve poter allontanare il punto d'osservazione dal centro della scena 3D. \\ \\
\textbf{Precondizione:}
il sistema è in attesa che l'utente selezioni una funzionalità per quanto riguarda la traslazione del punto di osservazione. \\ \\
\textbf{Postcondizione:}
il sistema ha allontanato il punto di osservazione dal centro della scena 3D. \\ \\
\textbf{Scenario Principale:}
\begin{itemize}
\item l'utente sceglie di quanto allontanare il punto di osservazione dal centro della scena 3D
\\ \\ \end{itemize}


\subsection{UC1.4.1.1.3 Imposta traslazione del punto di osservazione}
\textbf{Diagramma associato:}
\ref{UC1.4.1.1} \\ \\
\textbf{Attori Coinvolti:}
Utente. \\ \\
\textbf{Scopo e Descrizione:}
un utente deve poter impostare lo spostamento del punto di osservazione. \\ \\
\textbf{Precondizione:}
il sistema è in attesa che l'utente selezioni una funzionalità per quanto riguarda la traslazione del punto di osservazione. \\ \\
\textbf{Postcondizione:}
il sistema ha spostato il punto di osservazione secondo le specifiche richieste dell'utente. \\ \\
\textbf{Scenario Principale:}
\begin{itemize}
\item l'utente sceglie i valori secondo i quali traslare il punto di osservazione della scena 3D
\\ \\ \end{itemize}


\subsection{UC1.4.1.2 Ruota il punto d'osservazione attorno alla scena 3D}
\textbf{Diagramma associato:}
\ref{UC1.4.1} \\ \\
\textbf{Attori Coinvolti:}
Utente. \\ \\
\textbf{Scopo e Descrizione:}
un utente deve poter ruotare il punto d'osservazione attorno alla scena 3D. \\ \\
\textbf{Precondizione:}
il sistema sta visualizzando un'anteprima semplice ed è in attesa che l'utente selezioni una funzionalità. \\ \\
\textbf{Postcondizione:}
il sistema ha in memoria la scena 3D ed ha ruotato il punto d'osservazione della scena 3D secondo l'input utente. \\ \\
\textbf{Scenario Principale:}
\begin{itemize}
\item l'utente ruota il punto d'osservazione attorno alla scena
\\ \\ \end{itemize}


\subsection{UC1.4.2 Esegue rendering}
\textbf{Diagramma associato:}
\ref{UC1.4} \\ \\
\textbf{Attori Coinvolti:}
Utente. \\ \\
\textbf{Scopo e Descrizione:}
un utente deve poter eseguire il rendering della scena 3D. \\ \\
\textbf{Precondizione:}
il sistema è in attesa che l'utente selezioni una funzionalità per l'anteprima 3D. \\ \\
\textbf{Postcondizione:}
il sistema ha in memoria la scena 3D e propone all'utente l'immagine risultante dal rendering. \\ \\
\textbf{Scenario Principale:}
\begin{itemize}
\item l'utente esegue il rendering
\\ \\ \end{itemize}


\subsection{UC1.5 Agisce sulla cronologia delle modifiche}
\textbf{Diagramma associato:}
\ref{UC1.5} \\ \\
\textbf{Attori Coinvolti:}
Utente. \\ \\
\textbf{Scopo e Descrizione:}
un utente deve poter agire sulle modifiche effettuate. \\ \\
\textbf{Precondizione:}
il sistema ha in memoria una scena 3D caricata in precedenza e la cronologia delle modifiche utente non è vuota (UC1.2). \\ \\
\textbf{Postcondizione:}
il sistema ha in memoria la scena 3D ed ha agito sulla cronologia delle modifiche secondo le richieste utente. \\ \\
\textbf{Scenario Principale:}
\begin{itemize}
\item l'utente annulla l'ultima modifica effettuata
\item l'utente ripete l'ultima modifica effettuata
\\ \\ \end{itemize}
\begin{figure}[h!]
\centering
\includegraphics[width=\textwidth]{{{UC1.5}}}
\caption{UC1.5 Agisce sulla cronologia delle modifiche}
\label{UC1.5}
\end{figure}


\subsection{UC1.5.1 Annulla ultima modifica}
\textbf{Diagramma associato:}
\ref{UC1.5} \\ \\
\textbf{Attori Coinvolti:}
Utente. \\ \\
\textbf{Scopo e Descrizione:}
un utente deve poter annullare l'ultima modifica effettuata. \\ \\
\textbf{Precondizione:}
il sistema è in attesa di input, lo stato corrente della scena 3D è diverso da quello iniziale (UC1.1). \\ \\
\textbf{Postcondizione:}
il sistema ha annullato l'ultima modifica effettuata, nella cronologia delle modifiche c'è almeno una modifica da ripetere. \\ \\
\textbf{Scenario Principale:}
\begin{itemize}
\item l'utente annulla l'ultima modifica effettuata
\\ \\ \end{itemize}


\subsection{UC1.5.2 Ripeti ultima modifica}
\textbf{Diagramma associato:}
\ref{UC1.5} \\ \\
\textbf{Attori Coinvolti:}
Utente. \\ \\
\textbf{Scopo e Descrizione:}
un utente deve poter ripetere l'ultima modifica annullata. \\ \\
\textbf{Precondizione:}
il sistema è in attesa di input, nella cronologia delle modifiche ci sono delle modifiche da ripetere (UC1.5.1). \\ \\
\textbf{Postcondizione:}
il sistema ha ripetuto l'ultima modifica annullata. \\ \\
\textbf{Scenario Principale:}
\begin{itemize}
\item l'utente ripete l'ultima modifica annullata
\\ \\ \end{itemize}


