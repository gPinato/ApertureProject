\newpage
\section{Pianificazione dei test}
Di seguito verranno visualizzata delle tabelle, strutturate secondo la sezione 5.3.2 delle Norme di Progetto, che riportano tutti i test che si sono pianificati. \\
\subsection{Test di sistema}
Di seguito verrà mostrata una tabella che riporta tutti i test di sistema pianificati, associati ai requisiti descritti nel documento Analisi dei Requisiti.\\
I test sono da intendere solo per requisiti ai quali è stato ragionevole associare un test.
\subsubsection{Descrizione dei test di sistema}
\begin{center}
\begin{longtable}{|c|p{0.5\linewidth}|c|c|}
\toprule
\textbf{Test} & \textbf{Descrizione} & \textbf{Requisito} & \textbf{Stato}\\
\midrule
TS1 & Viene verificato che il sistema MaaP generi correttamente lo scheletro necessario & ROF1 & D.E.\\
\midrule
TS1.1 & Viene verificato che il sistema MaaP installi correttamente le librerie necessarie & ROF1.1 & D.E.\\
\midrule
TS1.2 & Viene verificato che il sistema MaaP generi correttamente i file necessari & ROF1.2 & D.E.\\
\midrule
TS1.3 & Viene verificato che il sistema MaaP generi correttamente le directory necessarie & ROF1.3 & D.E.\\
\midrule
TS1.4 & Viene verificato che il sottosistema di autenticazione sia installato e configurato correttamente & ROF1.4 & D.E.\\
\midrule
TS1.4.1 & Viene verificato che nel database degli utenti sia presente un profilo di amministrazione di default & ROF1.4.1 ROF6 & D.E.\\
\midrule
TS4 & Viene verificato che il sistema crei correttamente le pagine web partendo dal loro file di descrizione & ROF4 & D.E.\\
\midrule
TS5.1 & Viene verificato che la funzione di registrazione possa essere correttamente abilitata/disabilitata & RDF5.1 & D.E.\\
\midrule
TS5.4 & Viene verificato che il sistema possa utilizzare correttamente il database di analisi & ROF5.4 & D.E.\\
\midrule
TS5.5 & Viene verificato che la funzione di creazione indici possa essere correttamnte abilitata/disabilitata & ROF5.5 & D.E.\\
\midrule
TS7 & Viene verificato che il sistema consenta all'utente registrato di potersi autenticare & ROF 7.0 & D.E.\\
\midrule
TS8 & Viene verificato che il sistema consenta all'utente di potersi registrare & RDF 8.0 & D.E.\\
\midrule
TS9 & Viene verificato che il sistema consenta all'utente di recuperare la password  & ROF 9.0 & D.E\\
\midrule
TS10 & Viene verificato che il sistema apra e visualizzi correttamente le Collection e le Collection-Index & ROF10 & D.E.\\
\midrule
TS10.1 & Viene verificato che il sistema visualizzi correttamente le pagine di Document-Show & ROF10.1 & D.E.\\
\midrule
TS10.2.4 & Viene verificato che il sistema disconnetta correttamente un utente alla sua richiesta & ROF10.2.4 & D.E.\\
\midrule
TS10.3.1.1 & Viene verificato che un utente autenticato possa modificare i dati del suo profilo & ROF10.3.1.1 & D.E.\\
\midrule
TS10.3.1.2 & Viene verificato che le modifiche apportate al profilo di un utente business autenticaro siano consistenti & ROF10.3.1.2 & D.E.\\
\midrule
TS10.3.2 & Viene verificato che la creazione di un nuovo utente da parte di un utente business autenticato amministratore avvenga correttamente & ROF10.3.2 & D.E.\\
\midrule
TS10.3.3 & Viene verificata la corretta cancellazione di un utente da parte di un utente business autenticato amministratore & ROF10.3.3 & D.E.\\
\midrule
TS10.4 & Viene verificato che l'utente business autenticato amministratore possa eliminare correttamente un Document & ROF10.4 & D.E.\\
\midrule
TS10.5 & Viene verificato che l’utente business autenticato amministratore possa modificare correttamente un Document & ROF10.5 & D.E.\\
\midrule
TS10.6 & Viene verificata la corretta visualizzazione delle query più utilizzate & ROF10.6 & D.E.\\
\midrule
TS10.7 & Viene verificata la corretta creazione degli indici di analisi & ROF10.7 & D.E.\\
\midrule
TS17 & Viene verificato che le pagine web prodotte dal framework MaaP siano compatibili con la versione 30.0.x o superiore di Google Chrome & ROV17 & D.E.\\
\midrule
TS18 & Viene verificato che le pagine web prodotte dal framework MaaP siano compatibili con la versione 24.x o superiore di Firefox & ROV18 & D.E.\\
\midrule
TS19 & Viene verificato che il sistema accetti solo file di configurazione validi & ROV19 & D.E.\\
\midrule
TS26 & Viene verificato che il sistema di installazione del software funzioni correttamente & ROV26 & D.E.\\
\midrule
TS27 & Viene verificato che il deployment su Heroku avvenga con successo & ROV27 & D.E.\\
%inserire i test
\bottomrule
\caption{Tabella per test di sistema}
\label{tab:changelog}
\end{longtable}
\end{center}
\subsection{Test d'integrazione}
Di seguito verrà mostrata una tabella che riporta tutti i test d'integrazione pianificati, associati alle componenti descritte nella progettazione ad alto livello.\\
\subsubsection{Descrizione dei test d'integrazione}
\begin{center}
\begin{longtable}{|c|p{0.5\linewidth}|c|c|}
\toprule
\textbf{Test} & \textbf{Descrizione} & \textbf{Componenti} & \textbf{Stato}\\
\midrule
TI.MaaP & Test di integrazione finale tra client e server & MaaP & D.E.\\
\midrule
TI.Server & Verifica il corretto funzionamento delle componenti del server, quindi la corretta integrazione
tra ModelServer e Controller del server. In particolare, viene verificato che le funzionalit\`{a} presenti nel
Controller debbano agire consistentemente sui dati del ModelServer. & Server & D.E.\\
\midrule
TI.Server.ModelServer & Verifica che tutte le operazioni di lettura, scrittura e interpretazione dei dati avvengano
correttamente. & Server.ModelServer & D.E.\\
\midrule
TI.Server.ModelServer.DataManager-Database & Verifica che i gestori dei dati siano correttamente integrati ai database di riferimento. & Server.ModelServer.DataManager Server.ModelServer.Database & D.E.\\
\midrule
TI.Server.ModelServer.DataManager & Verifica che le operazioni di recupero, gestione e scrittura dati nei database avvengano correttamente. & Server.ModelServer.DataManager & D.E.\\
\midrule
TI.Client & Verifica che i dati ottenuti e forniti da e verso il client siano corretti e correttamente gestiti all'interno del client stesso. & Client & D.E.\\
\midrule
TI.Client.View-ControllerModelView & Verifica che i dati uscenti ed entranti la View siano gestiti correttamente dal ControllerModelView & Client.View Client.ControllerModelView & D.E.\\
\midrule
TI.Client.ControllerModelView-ModelClient & Viene verificato che le funzionalit\`{a} del ControllerModelView agiscano correttamente sul ModelClient & Client.ControllerModelView Client.ModelClient & D.E.\\
\midrule
TI.Client.ControllerModelView & Viene verificato che le funzioni del ClientController modifichino correttamente i dati presenti nello Scope & Client.ControllerModelView & D.E.\\
\midrule
TI.Client & Test integrazione finale client & Client & D.E\\
%inserire i test
\bottomrule
\caption{Tabella per test d'integrazione}
\label{tab:changelog}
\end{longtable}
\end{center}
\subsection{Tracciamento}
Di seguito verranno riportati in forma tabellare, descritta nella sezione 5.3.2 delle Norme di Progetto, i tracciamenti componente-test d'integrazione e test d'integrazione-componente
\subsubsection{Tracciamento componente-test d'integrazione}
\begin{center}
\begin{longtable}{|c|c|}
\toprule
\textbf{Componente} & \textbf{Test}\\
%inserire i test
\bottomrule
\caption{Tabella tracciamento componente-test d'integrazione}
\label{tab:changelog}
\end{longtable}
\end{center}
\subsubsection{Tracciamento test d'integrazione-componente}
\begin{center}
\begin{longtable}{|c|c|}
\toprule
\textbf{Test} & \textbf{Componente}\\
%inserire i test
\bottomrule
\caption{Tabella tracciamento test d'integrazione-componenti}
\label{tab:changelog}
\end{longtable}
\end{center}

