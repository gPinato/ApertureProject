\newpage
\section{Gestione amministrativa della revisione}
Di seguito verrà descritto come avviene, all'interno del gruppo, la comunicazione per la gestione di anomalie e per il trattamento delle discrepanze.
\label{3.0}
\subsection{Comunicazione e risoluzione di anomalie}
Con anomalia si intende un esito diverso del prodotto rispetto alle aspettative, una violazione delle norme tipografiche di un documento, un valore di qualche indice non valido, ovvero fuori dal range di accettazione.
Se un verificatore scova un'anomalia, di conseguenza aprirà un \gloss{ticket} su RedMine, strumento descritto nelle \emph{Norme\_di\_progetto\_v\versioneNormeDiProgetto{}.pdf}, in sezione 6.1.1.
\label{3.1}
\subsection{Trattamento delle discrepanze}
\label{3.2}
La discrepanza indica una mancata corrispondenza tra il prodotto atteso e il prodotto finito. Essa non ostruisce il funzionamento del software, ma è inesatto rispetto ai requisiti descritti. Per la gestione delle discrepanze si procede nella stessa maniera vista per la gestione delle anomalie.