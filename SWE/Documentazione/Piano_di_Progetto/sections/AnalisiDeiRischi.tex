\section{Analisi dei rischi}
Nello sviluppo del progetto, in particolare durante la fase di analisi, si è dedicato parte del tempo per individuare rischi che potrebbero compromettere la realizzazione del progetto.\\
Per prevedere e controllare eventuali rischi si è seguita questa sequenza di passi:
\begin{enumerate}
\item Identificazione e analisi: si cerca di individuare i rischi che possono comparire durante lo sviluppo del progetto;
\item Controllo: si valutano le contromisure da attuare  per prevenire i rischi;
\item Soluzione: azioni da intraprendere, in caso di rischio già avvenuto, per vanificare l'effetto dello stesso.
\end{enumerate}
Di seguito verranno elencati i rischi emersi durante la fase di analisi. I rischi saranno accompagnati da una breve descrizione, un livello di rischio, le contromisure da adottare e l'impatto che essi avranno.

\subsection{Conoscenze tecnologiche}
\begin{enumerate}
\item \grassetto{Descrizione}: I componenti del gruppo non hanno una buona conoscenza delle tecnologie utilizzate, in quanto queste sono delle novità assolute per alcuni di loro,in particolare:
\begin{itemize}
\item MongoDB;
\item Node.Js;
\item Mongoose;
\item Express.
\end{itemize}
La scarsa conoscenza delle precedenti tecnologie  saranno un ostacolo che troveranno durante lo sviluppo del progetto;
\item \grassetto{Livello di rischio}: Alto;
\item \grassetto{Contromisure}: I componenti del gruppo si impegnano per istruirsi sull'utilizzo delle tecnologie richieste, partecipando a seminari specifici e utilizzando materiale e documentazione reperibile sul Web. Questi apprendimenti saranno fatti in tempi brevi, per non compromettere  e ritardare il proseguimento del progetto;
\item \grassetto{Impatto}: Alto. I componenti incontreranno difficoltà nell'utilizzo delle tecnologie apprese.
\end{enumerate}

\subsection{Problemi hardware}
\begin{enumerate}
\item \grassetto{Descrizione}: Gran parte del lavoro è basato sull'utilizzo di computer personali e di un server per la repository e per il sistema di gestione dei ticket. In seguito ad una loro eventuale rottura si potrebbe perdere parte del lavoro svolto e le conseguenti attività di ripristino porterebbero ad un  ritardo nello svolgimento delle attività ripartite per il prosieguo del progetto;
\item \grassetto{Livello di rischio}: Basso;
\item \grassetto{Contromisure}: Sulla componente server verranno effettuati periodici backup per non perdere il lavoro svolto, ed anche i singoli componenti del gruppo, al termine della giornata lavorativa, effettueranno un backup su sistemi di Cloud o dispositivi hardware esterni;
\item \grassetto{Impatto}: Basso. In caso di rischio avvenuto, la presenza dei numerosi backup faciliterà il ripristino del lavoro svolto.
\end{enumerate}

\subsection{Variabilità requisiti}
\begin{enumerate}
\item \grassetto{Descrizione}: Il gruppo non devono sottovalutare la possibilità che il committente possa cambiare i requisiti in corso d'opera;
\item \grassetto{Livello di rischio}: Medio;
\item \grassetto{Contromisure}: e il rischio si verifica, sarà compito del gruppo adeguarsi ai nuovi requisiti imposti dal committente;
\item \grassetto{Impatto}: Medio. Cambiamento sostanziale dei requisiti porterà ad un impatto alto di tale rischio e al conseguente ritardo nella consegna del prodotto finale.
\end{enumerate}

\subsection{Comprensione requisiti}
\begin{enumerate}
\item \grassetto{Descrizione}: E' possibile che i componenti del gruppo non comprendano in pieno i requisiti e che alcuni aspetti vengano trattati in maniera errata o incompleta;
\item \grassetto{Livello di rischio}: Medio;
\item \grassetto{Contromisure}: Per ridurre al minimo gli effetti che il rischio comporta, ci saranno, durante la fase di Analisi dei Requisiti, degli incontri con il Committente per delle delucidazioni in merito ai requisiti richiesti dal prodotto;
\item \grassetto{Impatto}: Alto. Se si verifica tale rischio è necessario aggiornare con piccole modifiche l'attuale documento di Analisi dei Requisiti, o nel caso peggiore sarà necessario stilare una nuova versione del medesimo documento.
\end{enumerate}

\subsection{Problemi componenti del gruppo}
\begin{enumerate}
\item \grassetto{Descrizione}: Ogni componente del gruppo ha impegni personali e svolge attività extra lavorative ed alcuni di essi svolgono un vero e proprio lavoro e di conseguenza non possono essere spesso disponibili;
\item \grassetto{Livello di rischio}: Medio;
\item \grassetto{Contromisure}: E' compito del Responsabile di Progetto stilare appositi calendari di gruppo per organizzare le giornate lavorative per il progetto, in modo tale da non contrastare gli impegni personali di ciascun componente. Inoltre il carico di lavoro che un componente non svolgerà per impegni personali, dovrà essere ripartito tra gli altri componenti del gruppo;
\item \grassetto{Impatto}: Medio.
\end{enumerate}

\subsection{Preventivi di costi errati}
\begin{enumerate}
\item \grassetto{Descrizione}: I componenti del gruppo, non avendo esperienza in ambito manageriale e finanziario, possono imbattersi in errori nella pianificazione dei tempi e conseguente aumento dei costi;
\item \grassetto{Livello di rischio}: Medio;
\item \grassetto{Contromisure}: Si cercherà di dedicare più tempo alla pianificazione dei tempi e costi, affrontando i vari passi che ne derivano con maggior calma e attenzione;
\item \grassetto{Impatto}: Medio-Alto. Se il rischio si verifica, bisognerà avvisare il Committente e cercare di ridurre i costi per avvicinarsi il più possibile alla soglia prestabilita.
\end{enumerate}