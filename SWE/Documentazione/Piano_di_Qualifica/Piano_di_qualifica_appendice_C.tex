\newpage

\section{Resoconto delle attività di verifica}

\subsection{Resoconto PDCA}

Durante lo sviluppo del progetto si è sempre applicato il ciclo di Deming per cercare di migliorare la qualità dei processi. Questa attività ha portato a molteplici miglioramenti.
Tra i più significativi si riportano:
\begin{itemize}


\item Migliore politica di assegnazione delle attività: 

All'inizio della progettazione, le attività principali e maggiormente propedeutiche erano state assegnate a gruppi di quattro persone, con l'obiettivo di colmare velocemente il monte ore/persona e ridurre i tempi di completamento. Tuttavia si è notato come il lavoro procedesse più lentamente di quanto atteso e fossero presenti dei ritardi rispetto alle consegne. Nelle seguenti pianificazioni si è ridotto il numero di persone assegnate ad una singola attività cercando nel contempo di massimizzare il parallelismo. Nonostante questo abbia leggermente aumentato il tempo di completamento delle singole attività, i benefici portati dal lavoro concorrente di team più piccoli hanno largamente sorpassato gli svantaggi e comportato un maggior avanzamento complessivo del progetto.

\item Migliore sviluppo dei documenti:

All'inizio i documenti erano sviluppati utilizzando editor di testo come OpenOffice e in seguito, una volta completati, trasformati in \LaTeX\ con una formattazione appropiata.
In seguito si è visto come lo sviluppo dei documenti direttamente in \LaTeX, nonostante sia più oneroso in termini di tempo durante la stesura, sia comunque più rapido rispetto al primo metodo e, soprattutto, privo degli errori di trasposizione, riducendo quindi le ore/persona necessarie e gli errori contenuti nei documenti.

\end{itemize}


\subsection{Tracciamento componenti requisiti}
In questa sezione verranno tracciati i componenti con i requisiti. Per semplicità di lettura i requisiti associati ai figli di un componente non sono riportati anche nel padre.
\subsubsection{Tracciamento componenti - requisiti} 

\begin{center}
\begin{longtable}{|p{0.8\linewidth}|c|}
\toprule
\multicolumn{1}{|p{0.8\linewidth}}{\textbf{Componente}} & \multicolumn{1}{|c|}{\textbf{Requisito}}\\
\midrule
\endfirsthead
\multicolumn{2}{l}{\footnotesize\itshape\tablename~\thetable: continua dalla pagina precedente} \\
\toprule
\multicolumn{1}{|p{0.8\linewidth}}{\textbf{Componente}} & \multicolumn{1}{|c|}{\textbf{Requisito}}\\
\midrule
\endhead
\midrule
\multicolumn{2}{r}{\footnotesize\itshape\tablename~\thetable: continua nella prossima pagina} \\
\endfoot
\bottomrule
\caption{Tracciamento componenti - requisiti}
\label{tab:Tracciamento componenti - requisiti}\\
\endlastfoot

\midrule 
MaaPCLI
& \\

\midrule 
MaaPCLI::Installer
& \\

\midrule 
MaaPCLI::CLI
& \\

\midrule 
MaaPCLI::InstanceManager
& \\

\midrule 
MaaPCLI::ProjectFacade
& \\

\midrule 
MaaPCLI::ProjectCreate
& ROF1\\
& ROF1.1\\
& ROF1.2\\
& ROF1.3\\
& ROF1.4\\
& ROF1.4.1\\
& ROF5\\
& RDF5.3\\

\midrule 
MaaPCLI::ProjectClone
& ROF1.6\\

\midrule 
MaaPCLI::ProjectRemove
& ROF1.5\\

\midrule 
MaaP 
& \\

\midrule 
MaaP::Server
& \\

\midrule 
MaaP::Server::ModelServer::DataManager 
& \\

\midrule
MaaP::Server::ModelServer::DataManager::DatabaseAnalysisManager 
& \\

\midrule 
MaaP::Server::ModelServer::DataManager::DatabaseAnalysisManager ::DatabaseAnalysisManager
& ROF10\\
& ROF10.1\\
& ROF10.1.1\\ 
& RDF10.2\\
& RDF10.2.1\\
& RDF10.2.1.1\\
& RDF10.2.1.2\\
& RDF10.2.2\\
& RDF10.2.3\\
& ROF10.4\\
& ROF10.5\\
& ROF10.5.2\\
& ROF10.6\\
& ROF10.7\\
& ROF10.7.1.2\\
& ROF10.7.2.2\\
& ROF10.7.3\\

\midrule
MaaP::Server::ModelServer::DataManager::DatabaseAnalysisManager:: DataRetrieverAnalysis
& ROF10.6\\ 
& ROF10\\
& ROF10.1\\
& ROF10.1.1\\
& RDF10.2\\
& RDF10.2.1\\
& RDF10.2.1.1\\
& RDF10.2.1.2\\
& RDF10.2.2\\
& RDF10.2.3\\
& ROF10.4\\
& ROF10.5\\
& ROF10.5.2\\

\midrule 
MaaP::Server::ModelServer::DataManager::DatabaseUserManager 
& \\

\midrule 
MaaP::Server::ModelServer::DataManager::DatabaseUserManager:: DatabaseUserManager
& ROF10.3\\
& ROF10.3.1\\
& ROF10.3.1.2\\
& ROF10.3.1.4\\
& ROF10.3.1.5\\
& ROF10.3.2\\
& ROF10.3.3\\


\midrule 
MaaP::Server::ModelServer::DataManager::DatabaseUserManager:: DataRetrieverUsers
& ROF10.3\\
& ROF10.3.1\\
& ROF10.3.1.2\\
& ROF10.3.1.4\\
& ROF10.3.1.5\\
& ROF10.3.2\\
& ROF10.3.3\\


\midrule 
MaaP::Server::ModelServer::DataManager::IndexManager
& \\

\midrule 
MaaP::Server::ModelServer::DataManager::IndexManager::IndexManager
& ROF10.7\\
& ROF10.7.1.2\\
& ROF10.7.2.2\\
& ROF10.7.3\\

\midrule 
MaaP::Server::ModelServer::DataManager::JSonComposer
& \\

\midrule 
MaaP::Server::ModelServer::DataManager::IDatabaseManager
& ROF10\\
& ROF10.1\\
& ROF10.1.1\\
& RDF10.2\\
& RDF10.2.1\\
& RDF10.2.1.1\\
& RDF10.2.1.2\\
& RDF10.2.2\\
& RDF10.2.3\\
& ROF10.3\\
& ROF10.3.1\\
& ROF10.3.1.2\\
& ROF10.3.1.4\\
& ROF10.3.1.5\\
& ROF10.3.2\\
& ROF10.3.3\\
& ROF10.4\\
& ROF10.5\\
& ROF10.5.2\\
& ROF10.6\\
& ROF10.7\\
& ROF10.7.1.2\\
& ROF10.7.2.2\\
& ROF10.7.3\\


\midrule 
MaaP::Server::ModelServer::DataManager::IDataRetriever
& ROF10\\
& ROF10.1\\
& ROF10.1.1\\
& RDF10.2\\
& RDF10.2.1\\
& RDF10.2.1.1\\
& RDF10.2.1.2\\
& RDF10.2.2\\
& RDF10.2.3\\
& ROF10.3\\
& ROF10.3.1\\
& ROF10.3.1.2\\
& ROF10.3.1.4\\
& ROF10.3.1.5\\
& ROF10.3.2\\
& ROF10.3.3\\
& ROF10.4\\
& ROF10.5\\
& ROF10.5.2\\
& ROF10.6\\


\midrule 
MaaP::Server::ModelServer::Database
& \\

\midrule 
MaaP::Server::ModelServer::Database::MongooseDBAnalysis
& ROF10.4\\
& ROF10.5\\
& ROF10.5.2\\
& ROF10.6\\
& ROF10.7\\
& ROF10.7.1.2\\
& ROF10.7.2.2\\
& ROF10.7.3\\

\midrule 
MaaP::Server::ModelServer::Database::DBAnalysis
& ROF10.4\\
& ROF10.5\\
& ROF10.5.2\\
& ROF10.6\\
& ROF10.7\\
& ROF10.7.1.2\\
& ROF10.7.2.2\\
& ROF10.7.3\\

\midrule 
MaaP::Server::ModelServer::Database::Mongoose
& \\

\midrule 
MaaP::Server::ModelServer::Database::MongooseDBFramework
& ROF7\\
& ROF7.1\\
& ROF7.2\\
& ROF7.2.1\\
& ROF10.3\\
& ROF10.3.1\\
& ROF10.3.1.2\\
& ROF10.3.1.4\\
& ROF10.3.1.5\\
& ROF10.3.2\\
& ROF10.3.3\\


\midrule 
MaaP::Server::ModelServer::Database::DBFramework
& ROF7\\
& ROF7.1\\
& ROF7.2\\
& ROF7.2.1\\
& ROF10.3\\
& ROF10.3.1\\
& ROF10.3.1.2\\
& ROF10.3.1.4\\
& ROF10.3.1.5\\
& ROF10.3.2\\
& ROF10.3.3\\


\midrule 
MaaP::Server::ModelServer::Database::User
& ROF7\\
& ROF7.1\\
& ROF7.2\\
& ROF7.2.1\\
& ROF10.3\\
& ROF10.3.1\\
& ROF10.3.1.2\\

\midrule 
MaaP::Server::ModelServer::Database::Query
& ROF10.6\\

\midrule 
MaaP::Server::ModelServer::DSL::DSLParser
& ROF4\\

\midrule 
MaaP::Server::ModelServer::DSL::ParserInterface
& ROF4\\

\midrule 
MaaP::Server::ModelServer::DSL::DSLManager
& ROF4\\

\midrule 
MaaP::Server::ModelServer::DSL::DSLDescriptionFile
& ROF3\\

\midrule 
MaaP::Server::ModelServer::DSL::CollectionData
& \\

\midrule 
MaaP::Server::Controller 
& \\

\midrule 
MaaP::Server::Controller::Dispatcher
& \\

\midrule 
MaaP::Server::Controller::FrontController
& ROF7\\
& ROF7.1\\
& ROF7.2\\
& ROF7.2.1\\
& RDF8\\
& RDF8.1\\
& RDF8.2\\
& RDF8.2.1\\
& ROF9\\
& ROF10.2.4\\
& ROF10.3\\
& ROF10.4\\
& ROF10.5\\
& ROF10.6\\
& ROF10.7\\
& ROF10.7.1.2\\
& ROF10.7.2.2\\
& ROF10.7.3\\

\midrule 
MaaP::Server::Controller::IPassport
& ROF7\\
& ROF7.1\\
& ROF7.2\\
& ROF7.2.1\\
& ROF10.2.4\\

\midrule 
MaaP::Server::Controller::PassportAdapter 
& ROF7\\
& ROF7.1\\
& ROF7.2\\
& ROF7.2.1\\
& ROF10.2.4\\

\midrule 
MaaP::Server::Controller::Passport
& ROF7\\
& ROF7.1\\
& ROF7.2\\
& ROF7.2.1\\
& ROF10.2.4\\

\midrule 
MaaP::Client 
& \\

\midrule 
MaaP::Client::View 
& \\

\midrule 
MaaP::Client::View::Template
& \\

\midrule 
MaaP::Client::View::Template::SignIn
& ROF7\\
& ROF7.1\\
& ROF7.2\\
& ROF7.2.1\\

\midrule 
MaaP::Client::View::Template::SingUp
& RDF8\\
& RDF8.1\\
& RDF8.2\\
& RDF8.2.1\\

\midrule 
MaaP::Client::View::Template::AdminMainPageCollection
& ROF10\\
& RDF10.2\\
& RDF10.2.1\\
& RDF10.2.1.1\\
& RDF10.2.1.2\\
& RDF10.2.2\\
& RDF10.2.3\\
& ROF10.2.4\\
& ROF10.2.5\\
& ROF10.4\\
& ROF10.5\\
& ROF10.6\\


\midrule 
MaaP::Client::View::Template::UserMainPageCollection
& ROF10\\
& RDF10.2\\
& RDF10.2.1\\
& RDF10.2.1.1\\
& RDF10.2.1.2\\
& RDF10.2.2\\
& RDF10.2.3\\
& ROF10.2.4\\
& ROF10.2.5\\

\midrule 
MaaP::Client::View::Template::AdminMainPageDocument
& ROF10.1\\
& ROF10.1.1\\
& ROF10.1.2\\
& ROF10.2.4\\
& ROF10.2.5\\
& ROF10.4\\
& ROF10.5\\


\midrule 
MaaP::Client::View::Template::UserMainPageDocument
& ROF10.1\\
& ROF10.1.1\\
& ROF10.2.4\\
& ROF10.2.5\\


\midrule 
MaaP::Client::View::Template::MainPageDocumentEdit
& ROF10.1.3\\
& ROF10.2.4\\
& ROF10.2.5\\
& ROF10.5.1\\
& ROF10.5.2\\
& ROF10.5.3\\


\midrule 
MaaP::Client::View::Template::UserProfileEdit
& ROF10.2.4\\
& ROF10.2.5\\
& ROF10.3.1.1\\
& ROF10.3.1.2\\
& ROF10.3.1.3\\

\midrule 
MaaP::Client::View::Template::UserProfile
& ROF10.2.4\\
& ROF10.2.5\\
& ROF10.3\\
& ROF10.3.1\\

\midrule 
MaaP::Client::View::Template::AdminProfile
& ROF10.2.4\\
& ROF10.2.5\\
& ROF10.3.1.4\\
& ROF10.3.1.5\\
& ROF10.3.2\\
& ROF10.3.3\\


\midrule 
MaaP::Client::View::Template::PasswordRecovery
& ROF9\\
& ROF10.2.4\\

\midrule 
MaaP::Client::View::Template::IndexPage
& ROF10.6\\
& ROF10.7\\
& ROF10.7.1\\
& ROF10.7.1.1\\
& ROF10.7.1.2\\
& ROF10.7.2\\
& ROF10.7.2.1\\
& ROF10.7.2.2\\
& ROF10.7.3\\

\midrule 
MaaP::Client::ControllerModelView
& \\

\midrule 
MaaP::Client::ControllerModelView::ControllerClient
& \\

\midrule 
MaaP::Client::ControllerModelView::ControllerClient::ControllerAutenticazione
& ROF7\\
& ROF7.1\\
& ROF7.2\\
& ROF7.2.1\\
& RDF8\\
& RDF8.1\\
& RDF8.2\\
& RDF8.2.1\\

\midrule 
MaaP::Client::ControllerModelView::ControllerClient::ControllerCollection
& ROF10\\
& RDF10.2\\
& RDF10.2.1\\
& RDF10.2.1.1\\
& RDF10.2.1.2\\
& RDF10.2.2\\
& RDF10.2.3\\
& ROF10.4\\
& ROF10.5\\


\midrule 
MaaP::Client::ControllerModelView::ControllerClient::ControllerDocument
& ROF10.1\\
& ROF10.1.1\\
& ROF10.1.2\\
& ROF10.1.3\\
& ROF10.5.1\\
& ROF10.5.2\\
& ROF10.5.3\\


\midrule 
MaaP::Client::ControllerModelView::ControllerClient::ControllerProfilo
& ROF9\\
& ROF10.3\\
& ROF10.3.1\\
& ROF10.3.1.1\\
& ROF10.3.1.2\\
& ROF10.3.1.3\\
& ROF10.3.1.4\\
& ROF10.3.1.5\\
& ROF10.3.2\\
& ROF10.3.3\\


\midrule 
MaaP::Client::ControllerModelView::ControllerClient::ControllerMenu
& ROF10.2.4\\
& ROF10.2.5\\

\midrule 
MaaP::Client::ControllerModelView::ControllerClient::ControllerIndici
& ROF10.6\\
& ROF10.7\\
& ROF10.7.1\\
& ROF10.7.1.1\\
& ROF10.7.1.2\\
& ROF10.7.2\\
& ROF10.7.2.1\\
& ROF10.7.2.2\\
& ROF10.7.3\\


\midrule 
MaaP::Client::ControllerModelView::Scope
& \\

\midrule 
MaaP::Client::ControllerModelView::Scope::Collection
& ROF10\\
& RDF10.2\\
& RDF10.2.1\\
& RDF10.2.1.1\\
& RDF10.2.1.2\\
& RDF10.2.2\\
& RDF10.2.3\\
& ROF10.4\\
& ROF10.5\\

\midrule 
MaaP::Client::ControllerModelView::Scope::Query
& ROF10.6\\
& ROF10.7.1.1\\
& ROF10.7.1.2\\
& ROF10.7.3\\

\midrule 
MaaP::Client::ControllerModelView::Scope::Document
& ROF10.1\\
& ROF10.1.1\\
& ROF10.5.1\\
& ROF10.5.2\\
& ROF10.5.3\\


\midrule 
MaaP::Client::ControllerModelView::Scope::Profilo
& ROF10.3\\
& ROF10.3.1\\
& ROF10.3.1.1\\
& ROF10.3.1.2\\
& ROF10.3.1.4\\
& ROF10.3.1.5\\
& ROF10.3.2\\
& ROF10.3.3\\


\midrule 
MaaP::Client::ControllerModelView::Scope::Menu
& ROF10.2.4\\

\midrule 
MaaP::Client::ModelClient
& \\

\midrule 
MaaP::Client::ModelClient::Services
& \\

\midrule 
MaaP::Client::ModelClient::Services::HTTP
& \\

\midrule 
MaaP::Client::ModelClient::Model
& \\

\midrule 
MaaP::Client::ModelClient::Model::SessionData
& \\

%FINE TABELLA TRACCIAMENTO COMPONENTI REQUISITI
\end{longtable}
\end{center}

\subsubsection{Tracciamento requisiti - componenti}
\begin{center}
\begin{longtable}{|c|p{0.25\linewidth}|p{0.5\linewidth}|}
\toprule
\multicolumn{1}{|c|}{\textbf{Requisito}}
& \multicolumn{1}{|p{0.25\linewidth}}{\textbf{Descrizione}} 
& \multicolumn{1}{|p{0.5\linewidth}|}{\textbf{Componente}}\\
\midrule
\endfirsthead
\multicolumn{2}{l}{\footnotesize\itshape\tablename~\thetable: continua dalla pagina precedente} \\
\toprule
\multicolumn{1}{|c|}{\textbf{Requisito}}
& \multicolumn{1}{|p{0.25\linewidth}}{\textbf{Descrizione}} 
& \multicolumn{1}{|p{0.5\linewidth}|}{\textbf{Componente}}\\
\midrule
\endhead
\midrule
\multicolumn{2}{r}{\footnotesize\itshape\tablename~\thetable: continua nella prossima pagina} \\
\endfoot
\bottomrule
\caption{Tracciamento requisiti - componenti}
\label{tab:Tracciamento requisiti - componenti}\\
\endlastfoot

\midrule
ROF1
& Il sistema MaaP deve essere in grado di generare lo scheletro del progetto
& MaaPCLI::ProjectCreate\\

\midrule
ROF1.1
& Il sistema MaaP deve installare le librerie necessarie al funzionamento del progetto
& MaaPCLI::ProjectCreate\\

\midrule
ROF1.2
& Il sistema MaaP deve generare i file di configurazione necessari al funzionamento del progetto
& MaaPCLI::ProjectCreate\\

\midrule
ROF1.3
& Il sistema MaaP deve generare le directory necessarie al funzionamento del progetto
& MaaPCLI::ProjectCreate\\

\midrule
ROF1.4
& Il sistema MaaP deve generare il sistema di autenticazione per le pagine web
& MaaPCLI::ProjectCreate\\

\midrule
ROF1.4.1
& Il sistema di autenticazione per le pagine web deve essere generato insieme ad un profilo amministratore di default
& MaaPCLI::ProjectCreate\\

\midrule
ROF1.5
& Il sistema MaaP deve essere in grado eliminare un progetto esistente
& MaaPCLI::ProjectRemove\\

\midrule
ROF1.6
& Il sistema MaaP deve essere in grado di clonare un progetto esistente
& MaaPCLI::ProjectClone\\

\midrule
RFF2
& Il sistema MaaP deve permettere all'utente sviluppatore di utilizzare un editor interno specializzato per la scrittura/modifica dei file di descrizione
& \\

\midrule
RFF2.1
& Il sistema MaaP deve permettere all'utente sviluppatore di utilizzare un editor interno specializzato per la scrittura di un nuovo file di descrizione
& \\

\midrule
RFF2.1.1
& Il sistema MaaP deve permettere all'utente sviluppatore di utilizzare un editor interno specializzato per scrivere il codice del file di descrizione che intende creare
& \\

\midrule
RFF2.1.2
& Il sistema MaaP deve permettere all'utente sviluppatore di utilizzare un editor interno specializzato per salvare il codice scritto in modo permanente
& \\

\midrule
RFF2.2
& Il sistema MaaP deve permettere all'utente sviluppatore di utilizzare un editor interno specializzato per la modifica di un file di descrizione esistente
& \\

\midrule
RFF2.2.1
& Il sistema MaaP deve permettere all'utente sviluppatore di utilizzare un editor interno specializzato per modificare il codice di un file di descrizione esistente
& \\

\midrule
RFF2.2.2
& Il sistema MaaP deve permettere all'utente sviluppatore di utilizzare un editor interno specializzato per salvare il codice modificato in modo permanente
& \\

\midrule
RFF2.2.3
& Il sistema MaaP deve permettere all'utente sviluppatore di utilizzare un editor interno specializzato per annullare le modifiche al codice del file di descrizione modificato
& \\

\midrule
ROF3
& Il sistema MaaP deve permette all'utente sviluppatore di inserire un file di descrizione
& MaaP::Server::ModelServer::DSL:: DSLDescriptionFile\\

\midrule
ROF4
& Il sistema deve permette all'utente sviluppatore di utilizzare un file di descrizione
& MaaP::Server::ModelServer::DSL:: DSLParser\\
& & MaaP::Server::ModelServer::DSL:: ParserInterface\\
& & MaaP::Server::ModelServer::DSL:: DSLManager\\

\midrule
ROF4.1
& Il sistema MaaP deve permettere all'utente sviluppatore di creare la visualizzazione della Collection
& \\

\midrule
ROF4.1.1
& Il sistema MaaP deve permettere all'utente sviluppatore di creare la visualizzazione del menù per le Collection
& \\

\midrule
ROF4.1.1.1
& Il sistema MaaP deve permettere all'utente sviluppatore di definire il nome della voce relativa alla Collection
& \\

\midrule
ROF4.1.1.2
& Il sistema MaaP deve permettere all'utente sviluppatore di definire la posizione di una voce all'interno del menù
& \\

\midrule
ROF4.1.2
& Il sistema MaaP deve permettere all'utente sviluppatore di creare la visualizzazione della pagina Collection-Index
& \\

\midrule
ROF4.1.2.1
& Il sistema MaaP deve permettere all'utente sviluppatore di aggiungere delle chiavi da visualizzare nella pagina Collection-Index
& \\

\midrule
ROF4.1.2.1.1
& Il sistema MaaP deve permettere all'utente sviluppatore di aggiungere definire un'etichetta per la chiave da visualizzare
& \\

\midrule
ROF4.1.2.1.2
& Il sistema MaaP deve permettere all'utente sviluppatore di definire un campo associato alla chiave da visualizzare
& \\

\midrule
ROF4.1.2.1.3
& Il sistema MaaP deve permettere all'utente sviluppatore di definire un campo associato alla chiave da visualizzare, proveniente da un documento esterno
& \\

\midrule
ROF4.1.2.1.4
& Il sistema MaaP deve permettere all'utente sviluppatore di definire un campo associato alla chiave da visualizzare, proveniente dal risultato di una query
& \\

\midrule
ROF4.1.2.1.5
& Il sistema MaaP deve permettere all'utente sviluppatore di definire un campo associato alla chiave da visualizzare come trasformazione
& \\

\midrule
ROF4.1.2.2
& Il sistema MaaP deve permettere all'utente sviluppatore di definire un ordinamento rispetto a una chiave
& \\

\midrule
ROF4.1.2.3
& Il sistema deve permettere all'utente sviluppatore di  definire un numero massimo di Document da visualizzare per la pagina Collection-Index
& \\

\midrule
RFF4.1.2.4
& Il sistema MaaP deve permettere all'utente sviluppatore di aggiungere dei pulsanti all'interno della pagina Collection-Index
& \\

\midrule
ROF4.1.3
& Il sistema MaaP deve permettere all'utente sviluppatore di creare la visualizzazione per la pagine Document-Show
& \\

\midrule
ROF4.1.3.1
& Il sistema MaaP deve permettere all'utente sviluppatore di aggiungere delle chiavi da visualizzare nella pagina Document-Show
& \\

\midrule
ROF4.1.3.1.1
& Il sistema MaaP deve permettere all'utente sviluppatore di aggiungere definire un'etichetta per la chiave da visualizzare
& \\

\midrule
ROF4.1.3.1.2
& Il sistema MaaP deve permettere all'utente sviluppatore di definire un campo associato alla chiave da visualizzare
& \\

\midrule
ROF4.1.3.1.3
& Il sistema MaaP deve permettere all'utente sviluppatore di definire un campo associato alla chiave da visualizzare, proveniente da un documento esterno
& \\

\midrule
ROF4.1.3.1.4
& Il sistema MaaP deve permettere all'utente sviluppatore di definire un campo associato alla chiave da visualizzare, proveniente dal risultato di una query
& \\

\midrule
ROF4.1.3.1.5
& Il sistema MaaP deve permettere all'utente sviluppatore di definire un campo associato alla chiave da visualizzare come trasformazione
& \\

\midrule
RFF4.1.3.2
& Il sistema MaaP deve permettere all'utente sviluppatore di aggiungere un pulsante all'interno della  pagina Document-Show
& \\

\midrule
ROF4.2
& Il sistema MaaP deve permettere all'utente sviluppatore di modificare la visualizzazione della Collection
& \\

\midrule
ROF4.2.1
& Il sistema MaaP deve permettere all'utente sviluppatore di impostare la visualizzazione del menù delle le Collection
& \\

\midrule
ROF4.2.1.1
& Il sistema MaaP deve permettere all'utente sviluppatore di modificare il nome della voce relativa alla Collection
& \\

\midrule
ROF4.2.1.2
& Il sistema MaaP deve permettere all'utente sviluppatore di modificare la posizione di una voce all'interno del menù
& \\

\midrule
ROF4.2.2
& Il sistema MaaP deve permettere all'utente sviluppatore di impostare la visualizzazione della pagina Collection-Index
& \\

\midrule
ROF4.2.2.1
& Il sistema MaaP deve permettere all'utente sviluppatore di aggiungere delle chiavi da visualizzare nella pagina Collection-Index
& \\

\midrule
ROF4.2.2.2
& Il sistema MaaP deve permettere all'utente sviluppatore di eliminare delle chiavi da visualizzare nella pagina Collection-Index
& \\

\midrule
ROF4.2.2.3
& Il sistema MaaP deve permettere all'utente sviluppatore di definire un ordinamento, alfabetico crescente o decrescente, rispetto a una chiave
& \\

\midrule
ROF4.2.2.4
& Il sistema MaaP deve permettere all'utente sviluppatore di eliminare un ordinamento rispetto a una chiave
& \\

\midrule
ROF4.2.2.5
& Il sistema deve permettere all'utente sviluppatore di  definire un numero massimo di Document da visualizzare per la pagina Collection-Index
& \\

\midrule
ROF4.2.2.6
& Il sistema deve permettere all'utente sviluppatore di  eliminare il numero massimo di Document da visualizzare per la pagina Collection-Index
& \\

\midrule
RFF4.2.2.7
& Il sistema MaaP deve permettere all'utente sviluppatore di aggiungere dei pulsanti all'interno della pagina Collection-Index, specificando il nome del pulsante e l'azione che deve eseguire
& \\

\midrule
RFF4.2.2.8
& Il sistema MaaP deve permettere all'utente sviluppatore di eliminare dei pulsanti all'interno della pagina Collection-Index
& \\

\midrule
ROF4.2.3
& Il sistema MaaP deve permettere all'utente sviluppatore di impostare la visualizzazione per la pagine Document-Show
& \\

\midrule
ROF4.2.3.1
& Il sistema MaaP deve permettere all'utente sviluppatore di aggiungere delle chiavi da visualizzare nella pagina Document-Show
& \\

\midrule
ROF4.2.3.2
& Il sistema MaaP deve permettere all'utente sviluppatore di eliminare delle chiavi da visualizzare nella pagina Document-Show
& \\

\midrule
RFF4.2.3.3
& Il sistema MaaP deve permettere all'utente sviluppatore di aggiungere dei pulsanti all'interno della  pagina Document-Show, specificando il nome del pulsante e l'azione che deve eseguire
& \\

\midrule
RFF4.2.3.4
& Il sistema MaaP deve permettere all'utente sviluppatore di eliminare dei pulsanti all'interno della  pagina Document-Show
& \\

\midrule
ROF4.3
& Il sistema MaaP deve permettere all'utente sviluppatore di definire una query personalizzata
& \\

\midrule
ROF4.4
& Il sistema MaaP deve permettere all'utente sviluppatore di eliminare una query personalizzata
& \\

\midrule
ROF5
& Il sistema deve permettere all'utente sviluppatore la modifica dei file di configurazione
& MaaPCLI::ProjectCreate\\

\midrule
RDF5.1
& Il sistema deve permettere all'utente sviluppatore di abilitare la funzionalità di registrazione per l'utente finale nelle pagine web. Nel caso la registrazione sia abilitata l'utente finale può registrarsi al sistema, altrimenti no.
& \\

\midrule
RFF5.2
& Il sistema MaaP deve permettere all'utente sviluppatore di abilitare la funzionalità per la creazione di nuovi Document all'interno della pagina Collection-Index
& \\

\midrule
RDF5.3
& Il sistema MaaP deve permettere all'utente sviluppatore di modificare i template per le pagine web
& MaaPCLI::ProjectCreate\\

\midrule
ROF5.4
& Il sistema deve permettere all'utente sviluppatore di specificare nome, indirizzo e password, relativi al database di analisi con il quale interagire
& \\

\midrule
ROF5.5
& Il sistema deve permettere all'utente sviluppatore di abilitare la funzionalità per la creazione di nuovi indici all'interno della pagina Collection-Index
& \\


%FINE TABELLA REQUISITI MAAP, 
%Requisiti utente business

\midrule
ROF6
& L'utente business, al primo accesso, deve poter usare il profilo amministratore di default
& \\

\midrule
ROF7
& L'utente business deve potersi autenticare inserendo dei dati personali
& MaaP::Server::Controller:: FrontController\\

\midrule
ROF7.1
& L'utente business deve inserire l'email per l'autenticazione
& MaaP::Server::Controller:: FrontController\\

\midrule
ROF7.2
& L'utente business deve inserire la password per l'autenticazione
& MaaP::Server::Controller:: FrontController\\

\midrule
ROF7.2.1
& La password per l'autenticazione deve essere alfanumerica e contenere almeno otto caratteri
& MaaP::Server::Controller:: FrontController\\

\midrule
RDF8
& L'utente business deve potersi registrare inserendo dei dati personali
& MaaP::Server::Controller:: FrontController\\
& & MaaP::Client::View::Template:: SignUp\\
& & MaaP::Client::ControllerModelView:: ControllerClient::ControllerAutenticazione\\

\midrule
RDF8.1
& L'utente business, per registrarsi, deve inserire una email non presente nel sistema
& MaaP::Server::Controller:: FrontController\\
& & MaaP::Client::View::Template:: SignUp\\
& & MaaP::Client::ControllerModelView:: ControllerClient::ControllerAutenticazione\\

\midrule
RDF8.2
& L'utente business, per registrarsi, deve inserire una password
& MaaP::Server::Controller:: FrontController\\
& & MaaP::Client::View::Template:: SignUp\\
& & MaaP::Client::ControllerModelView:: ControllerClient::ControllerAutenticazione\\

\midrule
RDF8.2.1
& La password per la registrazione deve essere alfanumerica e contenere almeno otto caratteri
& MaaP::Server::Controller:: FrontController\\
& & MaaP::Client::View::Template:: SignUp\\
& & MaaP::Client::ControllerModelView:: ControllerClient::ControllerAutenticazione\\

\midrule
ROF9
& L'utente business deve poter recuperare la password
& MaaP::Server::Controller:: FrontController\\
& & MaaP::Client::View::Template:: PasswordRecovery\\
& & MaaP::Client::ControllerModelView:: ControllerClient::ControllerProfilo\\

\midrule
ROF10
& L'utente business autenticato deve poter aprire una Collection e visualizzare la sua pagina Collection-Index
& MaaP::Server::ModelServer::DataManager:: DatabaseAnalysisManager::DatabaseAnalysisManager\\
& & MaaP::Server::ModelServer::DataManager:: DatabaseAnalysisManager::DataRetrieverAnalysis\\
& & MaaP::Server::ModelServer::DataManager:: IDatabaseManager\\
& & MaaP::Server::ModelServer::DataManager:: IDataRetriever\\
& & MaaP::Client::View::Template:: AdminMainPageCollection\\
& & MaaP::Client::View::Template:: UserMainPageCollection\\
& & MaaP::Client::ControllerModelView:: ControllerClient::ControllerCollection\\
& & MaaP::Client::ControllerModelView:: Scope::Collection\\

\midrule
ROF10.1
& L'utente business autenticato deve poter visualizzare una pagina Document-Show
& MaaP::Server::ModelServer::DataManager:: DatabaseAnalysisManager::DatabaseAnalysisManager\\
& & MaaP::Server::ModelServer::DataManager:: DatabaseAnalysisManager::DataRetrieverAnalysis\\
& & MaaP::Server::ModelServer::DataManager:: IDatabaseManager\\
& & MaaP::Server::ModelServer::DataManager:: IDataRetriever\\
& & MaaP::Client::View::Template:: AdminMainPageDocument\\
& & MaaP::Client::View::Template:: UserMainPageDocument\\
& & MaaP::Client::ControllerModelView:: ControllerClient::ControllerDocument\\
& & MaaP::Client::ControllerModelView:: Scope::Document\\

\midrule
ROF10.1.1
& L'utente business autenticato deve poter visualizzare il Document selezionato
& MaaP::Server::ModelServer::DataManager:: DatabaseAnalysisManager::DatabaseAnalysisManager\\
& & MaaP::Server::ModelServer::DataManager:: DatabaseAnalysisManager::DataRetrieverAnalysis\\
& & MaaP::Server::ModelServer::DataManager:: IDatabaseManager\\
& & MaaP::Server::ModelServer::DataManager:: IDataRetriever\\
& & MaaP::Client::View::Template:: AdminMainPageDocument\\
& & MaaP::Client::View::Template:: UserMainPageDocument\\
& & MaaP::Client::ControllerModelView:: ControllerClient::ControllerDocument\\
& & MaaP::Client::ControllerModelView:: Scope::Document\\

\midrule
ROF10.1.2
& L'utente business autenticato amministratore deve poter eliminare il Document che sta visualizzando
& MaaP::Client::View::Template:: AdminMainPageDocument\\
& & MaaP::Client::ControllerModelView:: ControllerClient::ControllerDocument\\

\midrule
ROF10.1.3
& L'utente business autenticato amministratore deve poter modificare il Document che sta visualizzando
& MaaP::Client::View::Template:: MainPageDocumentEdit\\
& & MaaP::Client::ControllerModelView:: ControllerClient::ControllerDocument\\


\midrule
RDF10.2
& L'utente business autenticato deve poter modificare la visualizzazione dei Document
& MaaP::Server::ModelServer::DataManager:: DatabaseAnalysisManager::DatabaseAnalysisManager\\
& & MaaP::Server::ModelServer::DataManager:: DatabaseAnalysisManager::DataRetrieverAnalysis\\
& & MaaP::Server::ModelServer::DataManager:: IDatabaseManager\\
& & MaaP::Server::ModelServer::DataManager:: IDataRetriever\\
& & MaaP::Client::View::Template:: AdminMainPageCollection\\
& & MaaP::Client::View::Template:: UserMainPageCollection\\
& & MaaP::Client::ControllerModelView:: ControllerClient::ControllerCollection\\
& & MaaP::Client::ControllerModelView:: Scope::Collection\\

\midrule
RDF10.2.1
& L'utente business autenticato deve poter selezionare dei criteri per la visualizzazione
& MaaP::Server::ModelServer::DataManager:: DatabaseAnalysisManager::DatabaseAnalysisManager\\
& & MaaP::Server::ModelServer::DataManager:: DatabaseAnalysisManager::DataRetrieverAnalysis\\
& & MaaP::Server::ModelServer::DataManager:: IDatabaseManager\\
& & MaaP::Server::ModelServer::DataManager:: IDataRetriever\\
& & MaaP::Client::View::Template:: AdminMainPageCollection\\
& & MaaP::Client::View::Template:: UserMainPageCollection\\
& & MaaP::Client::ControllerModelView:: ControllerClient::ControllerCollection\\
& & MaaP::Client::ControllerModelView:: Scope::Collection\\

\midrule
RDF10.2.1.1
& L'utente business autenticato deve poter effettuare un ordinamento rispetto a una chiave
& MaaP::Server::ModelServer::DataManager:: DatabaseAnalysisManager::DatabaseAnalysisManager\\
& & MaaP::Server::ModelServer::DataManager:: DatabaseAnalysisManager::DataRetrieverAnalysis\\
& & MaaP::Server::ModelServer::DataManager:: IDatabaseManager\\
& & MaaP::Server::ModelServer::DataManager:: IDataRetriever\\
& & MaaP::Client::View::Template:: AdminMainPageCollection\\
& & MaaP::Client::View::Template:: UserMainPageCollection\\
& & MaaP::Client::ControllerModelView:: ControllerClient::ControllerCollection\\
& & MaaP::Client::ControllerModelView:: Scope::Collection\\

\midrule
RDF10.2.1.2
& L'utente business deve poter selezionare un numero massimo di Document da visualizzare per pagina
& MaaP::Server::ModelServer::DataManager:: DatabaseAnalysisManager::DatabaseAnalysisManager\\
& & MaaP::Server::ModelServer::DataManager:: DatabaseAnalysisManager::DataRetrieverAnalysis\\
& & MaaP::Server::ModelServer::DataManager:: IDatabaseManager\\
& & MaaP::Server::ModelServer::DataManager:: IDataRetriever\\
& & MaaP::Client::View::Template:: AdminMainPageCollection\\
& & MaaP::Client::View::Template:: UserMainPageCollection\\
& & MaaP::Client::ControllerModelView:: ControllerClient::ControllerCollection\\
& & MaaP::Client::ControllerModelView:: Scope::Collection\\

\midrule
RDF10.2.2
& L'utente business autenticato deve poter applicare un filtro alla visualizzazione dei Document
& MaaP::Server::ModelServer::DataManager:: DatabaseAnalysisManager::DatabaseAnalysisManager\\
& & MaaP::Server::ModelServer::DataManager:: DatabaseAnalysisManager::DataRetrieverAnalysis\\
& & MaaP::Server::ModelServer::DataManager:: IDatabaseManager\\
& & MaaP::Server::ModelServer::DataManager:: IDataRetriever\\
& & MaaP::Client::View::Template:: AdminMainPageCollection\\
& & MaaP::Client::View::Template:: UserMainPageCollection\\
& & MaaP::Client::ControllerModelView:: ControllerClient::ControllerCollection\\
& & MaaP::Client::ControllerModelView:: Scope::Collection\\

\midrule
RDF10.2.3
& L'utente business autenticato deve poter annullare il filtro
& MaaP::Server::ModelServer::DataManager:: DatabaseAnalysisManager::DatabaseAnalysisManager\\
& & MaaP::Server::ModelServer::DataManager:: DatabaseAnalysisManager::DataRetrieverAnalysis\\
& & MaaP::Server::ModelServer::DataManager:: IDatabaseManager\\
& & MaaP::Server::ModelServer::DataManager:: IDataRetriever\\
& & MaaP::Client::View::Template:: AdminMainPageCollection\\
& & MaaP::Client::View::Template:: UserMainPageCollection\\
& & MaaP::Client::ControllerModelView:: ControllerClient::ControllerCollection\\
& & MaaP::Client::ControllerModelView:: Scope::Collection\\

\midrule
ROF10.2.4
& L'utente business autenticato deve poter disconnettersi
& MaaP::Server::Controller:: FrontController\\
& & MaaP::Server::Controller:: IPassport\\
& & MaaP::Server::Controller:: PassportAdapter\\
& & MaaP::Server::Controller:: Passport\\
& & MaaP::Client::View::Template:: AdminMainPageCollection\\
& & MaaP::Client::View::Template:: UserMainPageCollection\\
& & MaaP::Client::View::Template:: AdminMainPageDocument\\
& & MaaP::Client::View::Template:: UserMainPageDocument\\
& & MaaP::Client::View::Template:: MainPageDocumentEdit\\
& & MaaP::Client::View::Template:: UserProfileEdit\\
& & MaaP::Client::View::Template:: UserProfile\\
& & MaaP::Client::View::Template:: AdminProfile\\
& & MaaP::Client::View::Template:: PasswordRecovery\\
& & MaaP::Client::ControllerModelView:: ControllerClient::ControllerMenu\\
& & MaaP::Client::ControllerModelView:: Scope::Menu\\

\midrule
ROF10.2.5
& L'utente business autenticato deve poter navigare tra la Collection
& MaaP::Client::View::Template:: AdminMainPageCollection\\
& & MaaP::Client::View::Template:: UserMainPageCollection\\
& & MaaP::Client::View::Template:: AdminMainPageDocument\\
& & MaaP::Client::View::Template:: UserMainPageDocument\\
& & MaaP::Client::View::Template:: MainPageDocumentEdit\\
& & MaaP::Client::View::Template:: UserProfileEdit\\
& & MaaP::Client::View::Template:: UserProfile\\
& & MaaP::Client::View::Template:: AdminProfile\\
& & MaaP::Client::ControllerModelView:: ControllerClient::ControllerMenu\\

\midrule
ROF10.3
& L'utente business autenticato deve poter gestire il proprio profilo
& MaaP::Server::ModelServer::DataManager:: DatabaseUserManager::DatabaseUserManager\\
& & MaaP::Server::ModelServer::DataManager:: DatabaseUserManager::DataRetrieverUsers\\
& & MaaP::Server::ModelServer::DataManager:: IDatabaseManager\\
& & MaaP::Server::ModelServer::DataManager:: IDataRetriever\\
& & MaaP::Server::ModelServer::Database:: MongooseDBFramework\\
& & MaaP::Server::ModelServer::Database:: DBFramework\\
& & MaaP::Server::ModelServer::Database:: User\\
& & MaaP::Server::Controller:: FrontController\\
& & MaaP::Client::View::Template:: UserProfile\\
& & MaaP::Client::ControllerModelView:: ControllerClient::ControllerProfilo\\
& & MaaP::Client::ControllerModelView:: Scope::Profilo\\

\midrule
ROF10.3.1
& L'utente business autenticato deve poter gestire i propri dati
& MaaP::Server::ModelServer::DataManager:: DatabaseUserManager::DatabaseUserManager\\
& & MaaP::Server::ModelServer::DataManager:: DatabaseUserManager::DataRetrieverUsers\\
& & MaaP::Server::ModelServer::DataManager:: IDatabaseManager\\
& & MaaP::Server::ModelServer::DataManager:: IDataRetriever\\
& & MaaP::Server::ModelServer::Database:: MongooseDBFramework\\
& & MaaP::Server::ModelServer::Database:: DBFramework\\
& & MaaP::Server::ModelServer::Database:: User\\
& & MaaP::Client::View::Template:: UserProfile\\
& & MaaP::Client::ControllerModelView:: ControllerClient::ControllerProfilo\\
& & MaaP::Client::ControllerModelView:: Scope::Profilo\\

\midrule
ROF10.3.1.1
& L'utente business autenticato deve poter modificare i propri dati utente
& MaaP::Client::View::Template:: UserProfileEdit\\
& & MaaP::Client::ControllerModelView:: ControllerClient::ControllerProfilo\\
& & MaaP::Client::ControllerModelView:: Scope::Profilo\\

\midrule
ROF10.3.1.2
& L'utente business autenticato deve poter  salvare le modifiche apportate
& MaaP::Server::ModelServer::DataManager:: DatabaseUserManager::DatabaseUserManager\\
& & MaaP::Server::ModelServer::DataManager:: DatabaseUserManager::DataRetrieverUsers\\
& & MaaP::Server::ModelServer::DataManager:: IDatabaseManager\\
& & MaaP::Server::ModelServer::DataManager:: IDataRetriever\\
& & MaaP::Server::ModelServer::Database:: MongooseDBFramework\\
& & MaaP::Server::ModelServer::Database:: DBFramework\\
& & MaaP::Server::ModelServer::Database:: User\\
& & MaaP::Client::View::Template:: UserProfileEdit\\
& & MaaP::Client::ControllerModelView:: ControllerClient::ControllerProfilo\\
& & MaaP::Client::ControllerModelView:: Scope::Profilo\\

\midrule
ROF10.3.1.3
& L'utente business autenticato deve poter annullare le modifiche apportate
& MaaP::Client::View::Template:: UserProfileEdit\\
& & MaaP::Client::ControllerModelView:: ControllerClient::ControllerProfilo\\

\midrule
ROF10.3.1.4
& L'utente business autenticato amministratore deve poter modificare i dati degli utenti business
& MaaP::Server::ModelServer::DataManager:: DatabaseUserManager::DatabaseUserManager\\
& & MaaP::Server::ModelServer::DataManager:: DatabaseUserManager::DataRetrieverUsers\\
& & MaaP::Server::ModelServer::DataManager:: IDatabaseManager\\
& & MaaP::Server::ModelServer::DataManager:: IDataRetriever\\
& & MaaP::Server::ModelServer::Database:: MongooseDBFramework\\
& & MaaP::Server::ModelServer::Database:: DBFramework\\
& & MaaP::Client::View::Template:: AdminProfile\\
& & MaaP::Client::ControllerModelView:: ControllerClient::ControllerProfilo\\
& & MaaP::Client::ControllerModelView:: Scope::Profilo\\

\midrule
ROF10.3.1.5
& L'utente business autenticato deve poter modificare i permessi degli utenti business
& MaaP::Server::ModelServer::DataManager:: DatabaseUserManager::DatabaseUserManager\\
& & MaaP::Server::ModelServer::DataManager:: DatabaseUserManager::DataRetrieverUsers\\
& & MaaP::Server::ModelServer::DataManager:: IDatabaseManager\\
& & MaaP::Server::ModelServer::DataManager:: IDataRetriever\\
& & MaaP::Server::ModelServer::Database:: MongooseDBFramework\\
& & MaaP::Server::ModelServer::Database:: DBFramework\\
& & MaaP::Client::View::Template:: AdminProfile\\
& & MaaP::Client::ControllerModelView:: ControllerClient::ControllerProfilo\\
& & MaaP::Client::ControllerModelView:: Scope::Profilo\\

\midrule
ROF10.3.2
& L'utente business autenticato amministratore deve poter creare un nuovo utente business
& MaaP::Server::ModelServer::DataManager:: DatabaseUserManager::DatabaseUserManager\\
& & MaaP::Server::ModelServer::DataManager:: DatabaseUserManager::DataRetrieverUsers\\
& & MaaP::Server::ModelServer::DataManager:: IDatabaseManager\\
& & MaaP::Server::ModelServer::DataManager:: IDataRetriever\\
& & MaaP::Server::ModelServer::Database:: MongooseDBFramework\\
& & MaaP::Server::ModelServer::Database:: DBFramework\\
& & MaaP::Client::View::Template:: AdminProfile\\
& & MaaP::Client::ControllerModelView:: ControllerClient::ControllerProfilo\\
& & MaaP::Client::ControllerModelView:: Scope::Profilo\\

\midrule
ROF10.3.3
& L'utente business autenticato amministratore deve poter eliminare un utente business
& MaaP::Server::ModelServer::DataManager:: DatabaseUserManager::DatabaseUserManager\\
& & MaaP::Server::ModelServer::DataManager:: DatabaseUserManager::DataRetrieverUsers\\
& & MaaP::Server::ModelServer::DataManager:: IDatabaseManager\\
& & MaaP::Server::ModelServer::DataManager:: IDataRetriever\\
& & MaaP::Server::ModelServer::Database:: MongooseDBFramework\\
& & MaaP::Server::ModelServer::Database:: DBFramework\\
& & MaaP::Client::View::Template:: AdminProfile\\
& & MaaP::Client::ControllerModelView:: ControllerClient::ControllerProfilo\\
& & MaaP::Client::ControllerModelView:: Scope::Profilo\\

\midrule
ROF10.4
& L'utente business autenticato amministratore deve poter cancellare un Document
& MaaP::Server::ModelServer::DataManager:: DatabaseAnalysisManager::DatabaseAnalysisManager\\
& & MaaP::Server::ModelServer::DataManager:: DatabaseAnalysisManager::DataRetrieverAnalysis\\
& & MaaP::Server::ModelServer::DataManager:: IDatabaseManager\\
& & MaaP::Server::ModelServer::DataManager:: IDataRetriever\\
& & MaaP::Server::ModelServer::Database:: MongooseDBAnalysis\\
& & MaaP::Server::ModelServer::Database:: DBAnalysis\\
& & MaaP::Server::Controller:: FrontController\\
& & MaaP::Client::View::Template:: AdminMainPageCollection\\
& & MaaP::Client::View::Template:: AdminMainPageDocument\\
& & MaaP::Client::ControllerModelView:: ControllerClient::ControllerCollection\\
& & MaaP::Client::ControllerModelView:: Scope::Collection\\

\midrule
ROF10.5
& L'utente business autenticato amministratore deve poter modificare un Document
& MaaP::Server::ModelServer::DataManager:: DatabaseAnalysisManager::DatabaseAnalysisManager\\
& & MaaP::Server::ModelServer::DataManager:: DatabaseAnalysisManager::DataRetrieverAnalysis\\
& & MaaP::Server::ModelServer::DataManager:: IDatabaseManager\\
& & MaaP::Server::ModelServer::DataManager:: IDataRetriever\\
& & MaaP::Server::ModelServer::Database:: MongooseDBAnalysis\\
& & MaaP::Server::ModelServer::Database:: DBAnalysis\\
& & MaaP::Server::Controller:: FrontController\\
& & MaaP::Client::View::Template:: AdminMainPageCollection\\
& & MaaP::Client::View::Template:: AdminMainPageDocument\\
& & MaaP::Client::ControllerModelView:: ControllerClient::ControllerCollection\\
& & MaaP::Client::ControllerModelView:: Scope::Collection\\

\midrule
ROF10.5.1
& L'utente business autenticato amministratore deve poter modificare i valori associati alla chiavi
& MaaP::Client::View::Template:: MainPageDocumentEdit\\
& & MaaP::Client::ControllerModelView:: ControllerClient::ControllerDocument\\
& & MaaP::Client::ControllerModelView:: Scope::Document\\

\midrule
ROF10.5.2
& L'utente business autenticato amministratore deve poter salvare le modifiche apportate al Document
& MaaP::Server::ModelServer::DataManager:: DatabaseAnalysisManager::DatabaseAnalysisManager\\
& & MaaP::Server::ModelServer::DataManager:: DatabaseAnalysisManager::DataRetrieverAnalysis\\
& & MaaP::Server::ModelServer::DataManager:: IDatabaseManager\\
& & MaaP::Server::ModelServer::DataManager:: IDataRetriever\\
& & MaaP::Server::ModelServer::Database:: MongooseDBAnalysis\\
& & MaaP::Server::ModelServer::Database:: DBAnalysis\\
& & MaaP::Client::View::Template:: MainPageDocumentEdit\\
& & MaaP::Client::ControllerModelView:: ControllerClient::ControllerDocument\\
& & MaaP::Client::ControllerModelView:: Scope::Document\\

\midrule
ROF10.5.3
& L'utente business autenticato amministratore deve poter annullare le modifiche apportate al Document
& MaaP::Client::View::Template:: MainPageDocumentEdit\\
& & MaaP::Client::ControllerModelView:: ControllerClient::ControllerDocument\\
& & MaaP::Client::ControllerModelView:: Scope::Document\\

\midrule
ROF10.6
& L'utente business autenticato amministratore deve poter visualizzare le query più utilizzate dal sistema MaaP
& MaaP::Server::ModelServer::DataManager:: DatabaseAnalysisManager::DatabaseAnalysisManager\\
& & MaaP::Server::ModelServer::DataManager:: DatabaseAnalysisManager::DataRetrieverAnalysis\\
& & MaaP::Server::ModelServer::DataManager:: IDatabaseManager\\
& & MaaP::Server::ModelServer::DataManager:: IDataRetriever\\
& & MaaP::Server::ModelServer::Database:: MongooseDBAnalysis\\
& & MaaP::Server::ModelServer::Database:: DBAnalysis\\
& & MaaP::Server::ModelServer::Database:: Query\\
& & MaaP::Server::Controller:: FrontController\\
& & MaaP::Client::View::Template:: AdminMainPageCollection\\
& & MaaP::Client::View::Template:: IndexPage\\
& & MaaP::Client::ControllerModelView:: ControllerClient::ControllerIndici\\
& & MaaP::Client::ControllerModelView:: Scope::Query\\

\midrule
ROF10.7
& L'utente business autenticato amministratore deve poter gestire la creazione e l'eliminazione degli indici
& MaaP::Server::ModelServer::DataManager:: DatabaseAnalysisManager::DatabaseAnalysisManager\\
& & MaaP::Server::ModelServer::DataManager:: IndexManager::IndexManager\\
& & MaaP::Server::ModelServer::DataManager:: IDatabaseManager\\
& & MaaP::Server::ModelServer::Database:: MongooseDBAnalysis\\
& & MaaP::Server::ModelServer::Database:: DBAnalysis\\
& & MaaP::Server::ModelServer::Database:: MongooseDBFramework\\
& & MaaP::Server::ModelServer::Database:: DBFramework\\
& & MaaP::Server::ModelServer::Database:: User\\
& & MaaP::Server::Controller:: FrontController\\
& & MaaP::Server::Controller:: IPassport\\
& & MaaP::Server::Controller:: PassportAdapter\\
& & MaaP::Server::Controller:: Passport\\
& & MaaP::Client::View::Template:: SignIn\\
& & MaaP::Client::View::Template:: IndexPage\\
& & MaaP::Client::ControllerModelView:: ControllerClient::ControllerAutenticazione\\
& & MaaP::Client::ControllerModelView:: ControllerClient::ControllerIndici\\

\midrule
ROF10.7.1
& L'utente business autenticato amministratore deve poter creare degli indici
& MaaP::Server::ModelServer::Database:: MongooseDBFramework\\
& & MaaP::Server::ModelServer::Database:: DBFramework\\
& & MaaP::Server::ModelServer::Database:: User\\
& & MaaP::Server::Controller:: IPassport\\
& & MaaP::Server::Controller:: PassportAdapter\\
& & MaaP::Server::Controller:: Passport\\
& & MaaP::Client::View::Template:: SignIn\\
& & MaaP::Client::View::Template:: IndexPage\\
& & MaaP::Client::ControllerModelView:: ControllerClient::ControllerAutenticazione\\
& & MaaP::Client::ControllerModelView:: ControllerClient::ControllerIndici\\

\midrule
ROF10.7.1.1
& L'utente business autenticato amministratore deve poter selezionare una query
& MaaP::Client::View::Template:: IndexPage\\
& & MaaP::Client::ControllerModelView:: ControllerClient::ControllerIndici\\
& & MaaP::Client::ControllerModelView:: Scope::Query\\

\midrule
ROF10.7.1.2
& L'utente business autenticato amministratore deve poter creare l'indice
& MaaP::Server::ModelServer::DataManager:: DatabaseAnalysisManager::DatabaseAnalysisManager\\
& & MaaP::Server::ModelServer::DataManager:: IndexManager::IndexManager\\
& & MaaP::Server::ModelServer::DataManager:: IDatabaseManager\\
& & MaaP::Server::ModelServer::Database:: MongooseDBAnalysis\\
& & MaaP::Server::ModelServer::Database:: DBAnalysis\\
& & MaaP::Server::Controller:: FrontController\\
& & MaaP::Client::View::Template:: IndexPage\\
& & MaaP::Client::ControllerModelView:: ControllerClient::ControllerIndici\\
& & MaaP::Client::ControllerModelView:: Scope::Query\\

\midrule
ROF10.7.2
& L'utente business autenticato amministratore deve poter eliminare degli indici
& MaaP::Server::ModelServer::Database:: MongooseDBFramework\\
& & MaaP::Server::ModelServer::Database:: DBFramework\\
& & MaaP::Server::ModelServer::Database:: User\\
& & MaaP::Server::Controller:: IPassport\\
& & MaaP::Server::Controller:: PassportAdapter\\
& & MaaP::Server::Controller:: Passport\\
& & MaaP::Client::View::Template:: SignIn\\
& & MaaP::Client::View::Template:: IndexPage\\
& & MaaP::Client::ControllerModelView:: ControllerClient::ControllerAutenticazione\\
& & MaaP::Client::ControllerModelView:: ControllerClient::ControllerIndici\\

\midrule
ROF10.7.2.1
& L'utente business autenticato amministratore deve poter selezionare un indice presente nel sistema
& MaaP::Server::ModelServer::Database:: MongooseDBFramework\\
& & MaaP::Server::ModelServer::Database:: DBFramework\\
& & MaaP::Server::ModelServer::Database:: User\\
& & MaaP::Server::Controller:: IPassport\\
& & MaaP::Server::Controller:: PassportAdapter\\
& & MaaP::Server::Controller:: Passport\\
& & MaaP::Client::View::Template:: SignIn\\
& & MaaP::Client::View::Template:: IndexPage\\
& & MaaP::Client::ControllerModelView:: ControllerClient::ControllerAutenticazione\\
& & MaaP::Client::ControllerModelView:: ControllerClient::ControllerIndici\\

\midrule
ROF10.7.2.2
& L'utente business autenticato amministratore deve poter eliminare l'indice selezionato
& MaaP::Server::ModelServer::DataManager:: DatabaseAnalysisManager::DatabaseAnalysisManager\\
& & MaaP::Server::ModelServer::DataManager:: IndexManager::IndexManager\\
& & MaaP::Server::ModelServer::DataManager:: IDatabaseManager\\
& & MaaP::Server::ModelServer::Database:: MongooseDBAnalysis\\
& & MaaP::Server::ModelServer::Database:: DBAnalysis\\
& & MaaP::Server::Controller:: FrontController\\
& & MaaP::Client::View::Template:: IndexPage\\
& & MaaP::Client::ControllerModelView:: ControllerClient::ControllerIndici\\

\midrule
ROF10.7.3
& L'utente business autenticato amministratore deve poter visualizzare gli indici presenti nel sistema
& MaaP::Server::ModelServer::DataManager:: DatabaseAnalysisManager::DatabaseAnalysisManager\\
& & MaaP::Server::ModelServer::DataManager:: IndexManager::IndexManager\\
& & MaaP::Server::ModelServer::DataManager:: IDatabaseManager\\
& & MaaP::Server::ModelServer::Database:: MongooseDBAnalysis\\
& & MaaP::Server::ModelServer::Database:: DBAnalysis\\
& & MaaP::Server::Controller:: FrontController\\
& & MaaP::Client::View::Template:: IndexPage\\
& & MaaP::Client::ControllerModelView:: ControllerClient::ControllerIndici\\
& & MaaP::Client::ControllerModelView:: Scope::Query\\

%FINE TABELLA TRACCIAMENTO REQUISITI COMPONENTI
\bottomrule
\caption{Tracciamento requisiti - componenti}
\end{longtable}
\end{center} 


\subsection{Riassunto delle attività di verifica}
In questa sezione sono descritti i resoconti delle attività di verifica effettuate sui documenti prima di ciascuna revisione.
\subsubsection{Revisione dei Requisiti}
Nel periodo precedente a questa revisione i documenti sono stati controllati dai verificatori seguendo le Norme di Progetto nelle sezione 6.4.1 e 6.4.2; è stata applicata l'analisi statica descritta nella sezione 2.8.1 di questo documento.
Inizialmente è stata applicata la tecnica di Walkthrough, dove sono scovati e successivamente corretti gli errori; ogni volta che si trovava un errore, esso veniva messo nell'apposta lista che serve per l'inspection.
Dopo il walkthrough è stata applicata la tecnica di Inspection, utilizzando l'apposita lista, disponible in appendice delle Norme di Progetto. Inoltre per questo documento sono state calcolate le metriche descritte nella sezione 2.9.2 del documento corrente.
Per quanto riguarda i processi, essi sono stati controllati e verificati secondo le metodologie descritte nelle Norme di Progetto in sezione 5.4.4. Sono state calcolate le metriche per i processi descritti in sezione 2.9.1 di questo documento, e riportati i corrispondenti valori di BV e SV in forma tabellare.
\subsubsection{Revisione di Progettazione}
Nel periodo precedente a questa revisione i documenti sono stati controllati dai verificatori seguendo le Norme di Progetto nelle sezione 6.4.1 e 6.4.2; è stata applicata l'analisi statica descritta nella sezione 2.8.1 di questo documento.
Inizialmente è stata applicata la tecnica di Walkthrough, dove sono scovati e successivamente corretti gli errori; ogni volta che si trovava un errore, esso veniva messo nell'apposta lista che serve per l'inspection.
Dopo il walkthrough è stata applicata la tecnica di Inspection, utilizzando l'apposita lista, disponible in appendice delle Norme di Progetto; è stata posta particolare attenzione al documento Specifica Tecnica. Inoltre per questo documento sono state calcolate le metriche descritte nella sezione 2.9.2 del documento corrente.
Per quanto riguarda i processi, essi sono stati controllati e verificati secondo le metodologie descritte nelle Norme di Progetto in sezione 5.4.4. Sono state calcolate le metriche per i processi descritti in sezione 2.9.1 di questo documento, e riportati i corrispondenti valori di BV e SV in forma tabellare.
\subsection{Dettaglio delle verifiche tramite analisi}
\subsubsection{Analisi dei Requisiti}
\subsubsubsection{Processi}
Di seguito vengono riportati i valori degli indici SV e BV calcolati durante il periodo di tempo dedicato all'Analisi dei Requisiti.
\begin{longtable}{|c|p{3cm}|p{3cm}|}
\toprule
\textbf{Attività} & \textbf{SV} & \textbf{BV} \\

%aggiungere qui una midrule per aggiungere una nuova riga alla tabella

\midrule
\emph{Studio Fattibilità} & 0 & 0 \\
\midrule
\emph{Analisi dei Requisiti} & +50 & +50\\
\midrule
\emph{Glossario} & 0  & 0\\
\midrule
\emph{Norme di Progetto} & 0 & 0\\
\midrule
\emph{Piano di Progetto} & 0 & 0\\
\midrule
\emph{Piano di Qualifica} & -15 & -15\\
\bottomrule
\caption{BV e SV calcolati sui documenti durante l'Analisi}
\label{tab:changelog}
\end{longtable}
\subsubsubsection{Conclusioni}
In questa tabella, i valori positivi indicano un costo risparmiato, viceversa i valori negativi mostrano un costo eccedente.
I valori indicati in tabella sono espressi in euro.
Non avendo previsto degli intervalli di tempo libero tra un'attività e la successiva, abbiamo ottenuto degli SV positivi in Analisi dei Requisiti e negativi in Piano di Qualifica.
Questa è stata una mancanza da parte del team, che vedrà di migliorarsi nelle prossime fasi e di adottare una tattica di pianificazione più flessibile.
I costi aggiuntivi sono comunque in linea con i nostri obiettivi.
\subsubsubsection{Documenti}
Di seguito vengono riportati, per ogni documento, i valori dell'indice di Gulpease calcolati durante il periodo di tempo dedicato all'Analisi dei Requisiti. Un documento è valido solo se rispecchia i range in sezione 2.9.2.1.
\begin{longtable}{|c|p{3cm}|p{3cm}|}
\toprule
\textbf{Documento} & \textbf{Valore indice} & \textbf{Esito} \\

%aggiungere qui una midrule per aggiungere una nuova riga alla tabella

\midrule
\emph{Studio Fattibilità v1.2.0} & 46 & Sufficiente\\
\midrule
\emph{Analisi dei Requisiti v1.2.0} & 52& Superato\\
\midrule
\emph{Glossario v1.2.0} & 46 & Sufficiente\\
\midrule
\emph{Norme di Progetto v1.2.0} & 52 & Superato\\
\midrule
\emph{Piano di Progetto v1.2.0} & 50 & Superato\\
\midrule
\emph{Piano di Qualifica v1.2.0} & 47 & Sufficiente\\
\bottomrule
\caption{Esiti dell'indice di Gulpease calcolato sui documenti durante l'Analisi}
\label{tab:changelog}
\end{longtable}
\subsubsection{Analisi in Dettaglio}
\subsubsubsection{Processi}
Di seguito vengono riportati i valori degli indici SV e BV calcolati durante il periodo di tempo dedicato all'Analisi in Dettaglio.
\begin{longtable}{|c|p{3cm}|p{3cm}|}
\toprule
\textbf{Attività} & \textbf{SV} & \textbf{BV} \\

%aggiungere qui una midrule per aggiungere una nuova riga alla tabella

\midrule
\emph{Studio Fattibilità} & 0 & 0 \\
\midrule
\emph{Analisi dei Requisiti} & 0 & -50\\
\midrule
\emph{Glossario} & 0  & 0\\
\midrule
\emph{Norme di Progetto} & 0 & 0\\
\midrule
\emph{Piano di Progetto} & 0 & 0\\
\midrule
\emph{Piano di Qualifica} & 0 & 0\\
\bottomrule
\caption{BV e SV calcolati sui documenti durante l'Analisi in Dettaglio}
\label{tab:changelog}
\end{longtable}
\subsubsubsection{Conclusioni}
Come si può notare dalla tabella, il BV è negativo, in quanto non sono state pianificate alcune attività correttive, ed è stato messo a budget il costo necessario per effettuare queste attività non previste.\\
Lo SV invece è pari a zero, in quanto l'ampio slack di tempo pianificato è servito a coprire le correzioni non previste e di conseguenza non è stato prodotto niente di più rispetto a quanto pianificato. 
\subsubsubsection{Documenti}
Di seguito vengono riportati, per ogni documento, i valori dell'indice di Gulpease calcolati durante il periodo di tempo dedicato all'Analisi in Dettaglio.

\begin{longtable}{|c|p{3cm}|p{3cm}|}
\toprule
\textbf{Documento} & \textbf{Valore indice} & \textbf{Esito} \\

%aggiungere qui una midrule per aggiungere una nuova riga alla tabella

\midrule
\emph{Analisi dei Requisiti v2.2.0} & 55 & Superato \\
\midrule
\emph{Glossario v2.2.0} & 46 & Sufficiente\\
\midrule
\emph{Norme di Progetto v2.2.0} & 52 & Superato\\
\midrule
\emph{Piano di Progetto v2.2.0} & 48 & Sufficiente \\
\midrule
\emph{Piano di Qualifica v2.2.0} & 47 & Sufficiente \\
\bottomrule
\caption{Esiti dell'indice di Gulpease calcolato sui documenti durante l'Analisi in Dettaglio}
\label{tab:changelog}
\end{longtable}
\subsubsection{Progettazione Architetturale}
\subsubsubsection{Processi}
Di seguito vengono riportati i valori degli indici SV e BV calcolati durante il periodo di tempo dedicato alla Progettazione Architetturale.
\begin{longtable}{|c|p{3cm}|p{3cm}|}
\toprule
\textbf{Attività} & \textbf{SV} & \textbf{BV} \\

%aggiungere qui una midrule per aggiungere una nuova riga alla tabella

\midrule
\emph{Analisi dei Requisiti} & 0 & +35\\
\midrule
\emph{Glossario} & 0  & +45\\
\midrule
\emph{Norme di Progetto} & 20 & +45\\
\midrule
\emph{Piano di Progetto} & 0 & +45 \\
\midrule
\emph{Piano di Qualifica} & 0 & -20\\
\midrule
\emph{Specifica Tecnica} & 0 & -87\\
\bottomrule
\caption{BV e SV calcolati sui documenti durante la Progettazione Architetturale}
\label{tab:changelog}
\end{longtable}

\subsubsubsection{Conclusioni}
Lo SV è positivo, in quanto lo slack dedicato al documento Norme di Progetto ha permesso l'aggiunta di valore non pianificato, come l'aggiunta di sezioni.\\
Il BV è positivo, e nonostante il fatto che si è dedicato più tempo alla progettazione, e quindi dedicandoci più budget; a causa di questo si è riuscito a risparmiare budget per le attività dedicate agli altri documenti, dedicando maggior budget per la Verifica della progettazione, che nella pianificazione non era adeguato, e togliendone da altre attività.
\subsubsubsection{Documenti}
Di seguito vengono riportati, per ogni documento, i valori dell'indice di Gulpease calcolati durante il periodo di tempo dedicato alla Progettazione Architetturale.

\begin{longtable}{|c|p{3cm}|p{3cm}|}
\toprule
\textbf{Documento} & \textbf{Valore indice} & \textbf{Esito} \\

%aggiungere qui una midrule per aggiungere una nuova riga alla tabella

\midrule
\emph{Analisi dei Requisiti v3.2.0} & 58 & Superato \\
\midrule
\emph{Glossario v3.2.0} &  & \\
\midrule
\emph{Norme di Progetto v3.2.0} & 53  & Superato\\
\midrule
\emph{Piano di Progetto v3.2.0} & 49  & Sufficiente\\
\midrule
\emph{Piano di Qualifica v3.2.0} & 48  & Sufficiente\\
\midrule
\emph{Specifica Tecnica v3.2.0} & 41 & Sufficiente\\
\bottomrule
\caption{Esiti dell'indice di Gulpease calcolato sui documenti durante la Progettazione}
\label{tab:changelog}
\end{longtable}

\subsubsubsection{Progettazione}
Viene qui riportata una tabella riassuntiva che riporta il calcolo dei parametri di accoppiamento afferente ed efferente per i componenti individuati nella progettazione architetturale.

\begin{longtable}{|p{11cm}|c|c|}
\toprule
\textbf{Componente} & \textbf{Afferente} & \textbf{Efferente} \\

\midrule 
MaaP::Server
& 1 & 0\\

\midrule 
MaaP::Server::ModelServer
& 3 & 0\\

\midrule 
MaaP::Server::ModelServer::DataManager 
& 1 & 4\\

\midrule
MaaP::Server::ModelServer::DataManager::DatabaseAnalysisManager 
& 1 & 7\\

\midrule 
MaaP::Server::ModelServer::DataManager::DatabaseUserManager 
& 1 & 4\\


\midrule 
MaaP::Server::ModelServer::DataManager::IndexManager
& 1 & 2\\


\midrule 
MaaP::Server::ModelServer::Database
& 4 & 0\\


\midrule 
MaaP::Server::ModelServer::DSL
& 1 & 0\\

\midrule 
MaaP::Server::Controller 
& 1 & 4\\

\midrule 
MaaP::Client 
& 0 & 1\\

\midrule 
MaaP::Client::View 
& 0 & 6\\

\midrule 
MaaP::Client::View::Template
& 0 & 6\\

\midrule 
MaaP::Client::ControllerModelView
& 12 & 2\\

\midrule 
MaaP::Client::ControllerModelView::ControllerClient
& 12 & 7\\

\midrule 
MaaP::Client::ControllerModelView::Scope
& 5 & 0\\

\midrule 
MaaP::Client::ModelClient
& 2 & 1\\

\midrule 
MaaP::Client::ModelClient::Services
& 1 & 1\\

\midrule 
MaaP::Client::ModelClient::Model
& 1 & 0\\


\bottomrule
\caption{Tabella accoppiamento componenti}
\end{longtable}

Come si può vedere dalla tabella, l'accoppiamento afferente risulta generalmente basso ad eccezione del componente ControllerModelView del package Client e relativo ControllerClient i quali hanno un valore relativamente alto. Questo delinea la criticità del componente in oggetto, che quindi andrà trattato con dovute cautele durante la generazione dei test e la loro esecuzione per ottenere un componente stabile più velocemente, prevenendo il rischio di regressione dovuto ad un alto accoppiamento.\\
Per quanto riguarda l'accoppiamento efferente, anch'esso è relativamente basso ad eccezione dei componenti interni del package MaaP::Server::ModelServer::DataManager e del componente ControllerClient che per la loro natura intrinseca hanno un alto livello di accoppiamento dovendo interagire con diverse classi di package esterni.

\subsection{Dettaglio dell'esito delle revisioni}
Per ciascuna revisione alla quale si intende partecipare, il Committente avrà il compito di segnalare eventuali problematiche trovate, dando una valutazione globale dell'andamento del progetto e una descrizione per ciascun documento con correzioni e accorgimenti da apportare.
Di seguito vengono elencate le modifiche apportate ai documenti, come suggerito dal Committente, per ciascuna revisione.
\subsubsection{Revisione dei Requisiti}
\begin{itemize}
\item \grassetto{Studio di Fattibilità:} il documento ha avuto una valutazione positiva, quindi non ci sono stati accorgimenti da apportare;
\item \grassetto{Norme di Progetto:} il documento è stato riorganizzato come suggerito, ovvero per processi, attività procedure e strumenti; è stata migliorata la descrizione della rotazione dei ruoli e il documento è stato incrementato con le parti riguardanti la parte di progettazione;
\item \grassetto{Analisi dei Requisiti:} sono stati corretti degli errori grammaticali, chiariti i significati di alcune parole; i casi d'uso segnalati hanno subito modifiche e aggiustamenti alle pre e post condizioni, mentre altri sono stati descritti più approfonditamente. Sempre dei casi d'uso sono stati tolti o spostati perchè in contrasto tra di loro, mentre per quanto riguarda la suddivisione dei requisiti in funzionali, desiderabili, obbligatori ecc.. sono stati rimossi e spostati perchè non adatti alla categoria in cui si presentavano. Il documento ha avuto una buona valutazione sulla struttura, quindi non si è cambiata.
\item \grassetto{Piano di Progetto:} come suggerito, alcuni contenuti sono stati spostati nell'Appendice del documento; è stato corretto l'utilizzo della parola fase, e usata solo se strettamente necessario e in contesti che la richiedono. La sezione "Preventivo a finire" è stata corretta in "Consuntivo", in quanto si è capito la differenza tra i significati dei due termini. Si è deciso, per i prossimi intervalli di tempo antecedenti le revisioni, di dedicare più tempo all'attività di Verifica, cercando di raggiungere la soglia del 30\% del tempo totale, come suggerito. Il documento inoltre è stato incrementato con le parti relative alla progettazione;
\item \grassetto{Piano di Qualifica:} il documento ha subito profonde modifiche, è stato ristrutturato e riorganizzato. Per fare ciò, è stata seguita la best practice per la struttura dei documenti presente nel sito del Professor Vardanega; il documento ha subito profonde modifiche anche nei contenuti, inoltre è stato incrementato con le parti relative alla progettazione;
\item \grassetto{Glossario:} il documento ha subito una lieve ristrutturazione, è stato tolto l'indice come suggerito; il documento è stato incrementato con l'inserimento di altri termini.
\end{itemize}






