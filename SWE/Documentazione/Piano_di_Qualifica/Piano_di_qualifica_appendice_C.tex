\newpage
\section{Pianificazione dei test}
Di seguito verranno visualizzate delle tabelle, strutturate secondo la sezione 5.3 delle  \emph{Norme\_di\_progetto\_v\versioneNormeDiProgetto{}.pdf}, che riportano tutti i test che si sono pianificati. \\


\subsection{Test di unità}
Di seguito verrà mostrata la tabella che riporta tutti i test di unità pianificati, associati alla componente testata.\\
Per ciascun test, verrà specificato anche il suo stato. Esso può essere di due tipi:
\begin{itemize}
\item \grassetto{D.S.}: ovvero da sviluppare;
\item \grassetto{Superato}: ovvero test che ha avuto esito positivo.
\end{itemize}
\subsubsection{Descrizione dei test di unità}

%QUESTE TABELLE SONO STATE GENERATE MANUALMENTE DA JACK
Sono qui riportate le tabelle per i testi di unit\'{a}  riguardanti il client.
Le funzionalit\'{a} dettagliate sono descritte nella Definizione di Prodotto v.\versioneDefinizioneDiProdotto{}.

\begin{center}
\begin{longtable}{|p{1cm}|p{5cm}|p{6cm}|p{1cm}|}
\toprule
\multicolumn{1}{|p{1cm}}{\textbf{Test}}
& \multicolumn{1}{|p{5cm}}{\textbf{Descrizione}}
& \multicolumn{1}{|p{6cm}}{\textbf{Funzioni testate}}
& \multicolumn{1}{|p{1cm}|}{\textbf{Stato}}\\
\midrule
\endfirsthead
\multicolumn{2}{l}{\footnotesize\itshape\tablename~\thetable: continua dalla pagina precedente} \\
\toprule
\multicolumn{1}{|p{1cm}}{\textbf{Test}}
& \multicolumn{1}{|p{5cm}}{\textbf{Descrizione}}
& \multicolumn{1}{|p{6cm}}{\textbf{Funzioni testate}}
& \multicolumn{1}{|p{1cm}|}{\textbf{Stato}}\\
\midrule
\endhead
\midrule
\multicolumn{2}{r}{\footnotesize\itshape\tablename~\thetable: continua nella prossima pagina} \\
\endfoot
\bottomrule
\caption{Test di unit\'{a} parte client}
\endlastfoot

\midrule
TU1
& Vengono verificate tutte le funzionalità rese disponibili dal Collection Controller.
& init()
& D.E.\\
& &  getData() &\\
& &  numerify(num) &\\
& & previousPage() &\\
& & nextPage() &\\
& & toPage(index) &\\
& & changeSort() &\\
& & columnSort(index) &\\
& & delete\_ document(index) &\\



\midrule
TU2
& Vengono verificate tutte le funzionalità rese disponibili dal Dashboard Controller.
& CollectionListService.get
& D.E.\\


\midrule
TU3
& Vengono verificate tutte le funzionalità rese disponibili dal Document Controller.
& DocumentDataService.query
& D.E.\\
& & delete\_ document(index) &\\




\midrule
TU4
& Vengono verificate tutte le funzionalità rese disponibili dal DocumentEdit Controller.
& DocumentEditService.query
& D.E.\\
& & edit\_ document(index) &\\
& & delete\_ document(index) &\\



\midrule
TU5
& Vengono verificate tutte le funzionalità rese disponibili dal Index Controller.
&  init() & DE\\
& & getData()& \\
& &  numerify(num)& \\
& &  previousPage()& \\
& &  nextPage()& \\
& &  toPage(index)& \\
& &  changeSort()& \\
& &  columnSort(index)& \\
& &  delete\_ document(index)& \\




\midrule
TU6
& Vengono verificate tutte le funzionalità rese disponibili dal Login Controller.
& AuthService.login
& D.E.\\


\midrule
TU7
& Vengono verificate tutte le funzionalità rese disponibili dal NavBar Controller.
& CollectionListService.get
& D.E.\\
& & logout() &\\



\midrule
TU8
& Vengono verificate tutte le funzionalità rese disponibili dal Profile Controller.
& ProfileDataService.query
& D.E.\\
& & delete\_ document(index) &\\


\midrule
TU9
& Vengono verificate tutte le funzionalità rese disponibili dal ProfileEdit Controller.
& ProfileEditService.query
& D.E.\\

& & edit\_ document(index) &\\
& & delete\_ document(index) &\\


\midrule
TU10
& Vengono verificate tutte le funzionalità rese disponibili dal Query Controller.
& QueryService.query
& D.E.\\
& & createIndex(id) &\\
& & numerify(num) &\\
& & previousPage() &\\
& & nextPage() &\\
& & toPage(index) &\\
& & changeSort() &\\
& & columnSort(index) &\\
& & delete\_ document(index) &\\



\midrule
TU11
& Vengono verificate tutte le funzionalità rese disponibili dal Register Controller.
& signupForm
& D.E.\\


\midrule
TU12
& Vengono verificate tutte le funzionalità rese disponibili dal UserCollection Controller.
& init()
& D.E.\\
& & getData() &\\
& & numerify(num) &\\
& & previousPage() &\\
& & nextPage() &\\
& & toPage(index) &\\
& & changeSort() &\\
& & columnSort(index) &\\
& & delete\_ document(index) &\\


\midrule
TU13
& Vengono verificate tutte le funzionalità rese disponibili dal UserController Controller.
& UserDataService.query
& D.E.\\
& & delete\_ document(index) &\\



\midrule
TU14
& Vengono verificate tutte le funzionalità rese disponibili dal UserEdit Controller.
& UserEditService.query
& D.E.\\
& & edit\_ document(index) &\\
& & delete\_ document(index) &\\





\end{longtable}
\end{center}


Test di unità per i servizi del client.
\begin{center}
\begin{longtable}{|p{2cm}|p{7cm}|p{2cm}|}
\toprule
\multicolumn{1}{|p{2cm}}{\textbf{Test}}
& \multicolumn{1}{|p{7cm}}{\textbf{Descrizione}}
& \multicolumn{1}{|p{2cm}|}{\textbf{Stato}}\\
\midrule
\endfirsthead
\multicolumn{2}{l}{\footnotesize\itshape\tablename~\thetable: continua dalla pagina precedente} \\
\toprule
\multicolumn{1}{|p{2cm}}{\textbf{Test}}
& \multicolumn{1}{|p{7cm}}{\textbf{Descrizione}}
& \multicolumn{1}{|p{2cm}|}{\textbf{Stato}}\\
\midrule
\endhead
\midrule
\multicolumn{2}{r}{\footnotesize\itshape\tablename~\thetable: continua nella prossima pagina} \\
\endfoot
\bottomrule
\caption{Test di unit\'{a} dei servizi del client}
\endlastfoot

\midrule
TU15
& Vengono verificate tutte le funzionalità rese disponibili dal servizio AuthService.
& D.E.\\


\midrule
TU16
& Vengono verificate tutte le funzionalità rese disponibili dal servizio CollectionDataService.
& D.E.\\


\midrule
TU17
& Vengono verificate tutte le funzionalità rese disponibili dal servizio CollectionListService.
& D.E.\\


\midrule
TU18
& Vengono verificate tutte le funzionalità rese disponibili dal servizio DocumentDataService.
& D.E.\\


\midrule
TU19
& Vengono verificate tutte le funzionalità rese disponibili dal servizio DocumentEditService.
& D.E.\\



\midrule
TU20
& Vengono verificate tutte le funzionalità rese disponibili dal dal servizio IndexService.
& D.E.\\


\midrule
TU21
& Vengono verificate tutte le funzionalità rese disponibili dal servizio LogoutService.
& D.E.\\


\midrule
TU22
& Vengono verificate tutte le funzionalità rese disponibili dal servizio ProfileDataService.
& D.E.\\

\midrule
TU23
& Vengono verificate tutte le funzionalità rese disponibili dal servizio ProfileEditService.
& D.E.\\


\midrule
TU24
& Vengono verificate tutte le funzionalità rese disponibili dal servizio QueryService.
& D.E.\\


\midrule
TU25
& Vengono verificate tutte le funzionalità rese disponibili dal servizio RegisterService.
& D.E.\\


\midrule
TU26
& Vengono verificate tutte le funzionalità rese disponibili dal servizio UserCollectionService.
& D.E.\\

\midrule
TU27
& Vengono verificate tutte le funzionalità rese disponibili dal servizio UserDataService.
& D.E.\\

\midrule
TU28
& Vengono verificate tutte le funzionalità rese disponibili dal servizio UserEditService.
& D.E.\\



\end{longtable}
\end{center}


Sono qui riportate le tabelle per i test di unit\'{a}  riguardanti il server.
Le funzionalit\'{a} dettagliate sono descritte nella Definizione di Prodotto v.\versioneDefinizioneDiProdotto{}.

\begin{center}
\begin{longtable}{|p{1cm}|p{5cm}|p{6cm}|p{1cm}|}
\toprule
\multicolumn{1}{|p{1cm}}{\textbf{Test}}
& \multicolumn{1}{|p{5cm}}{\textbf{Descrizione}}
& \multicolumn{1}{|p{6cm}}{\textbf{Funzioni testate}}
& \multicolumn{1}{|p{1cm}|}{\textbf{Stato}}\\
\midrule
\endfirsthead
\multicolumn{2}{l}{\footnotesize\itshape\tablename~\thetable: continua dalla pagina precedente} \\
\toprule
\multicolumn{1}{|p{1cm}}{\textbf{Test}}
& \multicolumn{1}{|p{5cm}}{\textbf{Descrizione}}
& \multicolumn{1}{|p{6cm}}{\textbf{Funzioni testate}}
& \multicolumn{1}{|p{1cm}|}{\textbf{Stato}}\\
\midrule
\endhead
\midrule
\multicolumn{2}{r}{\footnotesize\itshape\tablename~\thetable: continua nella prossima pagina} \\
\endfoot
\bottomrule
\caption{Test di unit\'{a} parte server}
\endlastfoot

\midrule
TU40
& Il test verifica che il dispatcher reindirizzi in modo corretto le richieste ricevute dal client.
& maap\_server::controller::dispatcher:: dispatcherInit()
& D.E.\\

\midrule
TU41
& Il test verifica che il dispatcher venga inizializzato in maniera corretta.
& maap\_server::controller::frontController:: initFrontController()
& D.E.\\

\midrule
TU42
& Il test verifica che il front controller venga inizializzato in maniera corretta.
& maap\_server::controller::index:: init()
& D.E.\\

\midrule
TU43
& Il test verifica che il modulo passport, utilizzato per la gestione dell'autenticazione, venga inizializzato in maniera corretta e che l'utente sia correttamente autenticato.
& maap\_server::controller::passport:: initPassport()
& superato.\\
& & maap\_server::controller::passport:: checkAuthenticatedAdmin()
& superato.\\
& & maap\_server::controller::passport:: checkAuthenticated()
& superato.\\
& & maap\_server::controller::passport:: checkNotAuthenticated()
& superato.\\

\midrule
TU44
& Il test verifica che la connessione al database di analisi e al database degli utenti avvenga in maniera corretta; inoltre viene verificato che venga aggiunto un utente di default all'avvio del sistema.
& maap\_server::modelServer::database:: index::addAdminDefault()
& D.E.\\
& & maap\_server::modelServer::database:: index::initDB()
& D.E.\\

\midrule
TU45
& Il test verifica che per ogni collection definita tramite DSL esita lo schema corrispondente, e successivamente lo rende disponibile.
& maap\_server::modelServer::database:: MongooseDBAnalysis::init()
& D.E.\\

\midrule
TU46
& Il test verifica che vengano costruiti in maniera corretta gli schemi del database per l'utente e per le query; successivamente verranno creati i modelli per questi schemi e resi disponibili.
& maap\_server::modelServer::database:: MongooseDBFramework::init()
& D.E.\\

\midrule
TU47
& Il test verifica che la gestione dei dati nel database di analisi avvenga in maniera corretta. In particolare viene controllato che le collection restituite siano quelle presenti nel database, che un document possa venir modificato o rimosso. Si verifica che le query possano essere cancellate o mostrate le più utilizzate; inoltre viene verificato anche il comportamento per la creazione, rimozione o restituzione di indici.
& maap\_server::modelServer:: dataManager:: DatabaseAnalysisManager::DatabaseAnalysisManager:: sendCollectionsList()
& D.E.\\
& & maap\_server::modelServer:: dataManager:: DatabaseAnalysisManager::DatabaseAnalysisManager:: sendCollection()
& D.E.\\
& & maap\_server::modelServer:: dataManager:: DatabaseAnalysisManager::DatabaseAnalysisManager:: sendDocument()
& D.E.\\
& & maap\_server::modelServer:: dataManager:: DatabaseAnalysisManager::DatabaseAnalysisManager:: sendDocumentEdit()
& D.E.\\
& & maap\_server::modelServer:: dataManager:: DatabaseAnalysisManager::DatabaseAnalysisManager:: updateDocument()
& D.E.\\
& & maap\_server::modelServer:: dataManager:: DatabaseAnalysisManager::DatabaseAnalysisManager:: removeDocument()
& D.E.\\
& & maap\_server::modelServer:: dataManager:: DatabaseAnalysisManager::DatabaseAnalysisManager:: resetQueries()
& D.E.\\
& & maap\_server::modelServer:: dataManager:: DatabaseAnalysisManager::DatabaseAnalysisManager:: getTopQueries()
& D.E.\\
& & maap\_server::modelServer:: dataManager:: DatabaseAnalysisManager::DatabaseAnalysisManager:: getIndexesList()
& D.E.\\
& & maap\_server::modelServer:: dataManager:: DatabaseAnalysisManager::DatabaseAnalysisManager:: createIndex()
& D.E.\\
& & maap\_server::modelServer:: dataManager:: DatabaseAnalysisManager::DatabaseAnalysisManager:: deleteIndex()
& D.E.\\

\midrule
TU48
& Il test verifica che dato un nome di una collection venga restituito il modello corrispondente; dopodichè vengono verificati i document restituiti dalla query effettuata sul database in base alle azioni compiute dall'utente. Si controlla che le trasformazioni specificate nel file DSL vengano applicate correttamente.
Inoltre viene verificato che la richiesta di visualizzare una collection con dei document avvenga in maniera corretta, in base alle scelte effettuate dall'utente.
Viene testato anche il comportamento riguardante la modifica di un document e la sua eventuale rimozione.
& maap\_server::modelServer:: dataManager:: DatabaseAnalysisManager::DataRetrieverAnalysis:: getModel()
& D.E.\\
& & maap\_server::modelServer:: dataManager:: DatabaseAnalysisManager::DataRetrieverAnalysis:: getDocuments()
& D.E.\\
& & maap\_server::modelServer:: dataManager:: DatabaseAnalysisManager::DataRetrieverAnalysis:: getCollectionsList()
& D.E.\\
& & maap\_server::modelServer:: dataManager:: DatabaseAnalysisManager::DataRetrieverAnalysis:: applyTrasformations()
& D.E.\\
& & maap\_server::modelServer:: dataManager:: DatabaseAnalysisManager::DataRetrieverAnalysis:: sortDocumentsByLabels()
& D.E.\\
& & maap\_server::modelServer:: dataManager:: DatabaseAnalysisManager::DataRetrieverAnalysis:: getCollectionIndex()
& D.E.\\
& & maap\_server::modelServer:: dataManager:: DatabaseAnalysisManager::DataRetrieverAnalysis:: getDocumentShow()
& D.E.\\
& & maap\_server::modelServer:: dataManager:: DatabaseAnalysisManager::DataRetrieverAnalysis:: getDocumentShowEdit()
& D.E.\\
& & maap\_server::modelServer:: dataManager:: DatabaseAnalysisManager::DataRetrieverAnalysis:: updateDocument()
& D.E.\\
& & maap\_server::modelServer:: dataManager:: DatabaseAnalysisManager::DataRetrieverAnalysis:: removeDocument()
& D.E.\\

\midrule
TU49
& Il test verifica la gestione di utente; in particolare viene verificata la mail dell'utente, la possibilità di effettuare la registrazione o di modificare i dati del profilo. Inoltre viene controllata la visualizzazione degli utenti quando richiesta e la possibilità di rimuovere un utente dal database.
&  maap\_server::modelServer:: dataManager:: DatabaseUserManager::DatabaseUserManager:: checkMail()
& D.E.\\
& & maap\_server::modelServer:: dataManager:: DatabaseUserManager::DatabaseUserManager:: userSignup()
& D.E.\\
& & maap\_server::modelServer:: dataManager:: DatabaseUserManager::DatabaseUserManager:: sendUserProfile()
& D.E.\\
& & maap\_server::modelServer:: dataManager:: DatabaseUserManager::DatabaseUserManager:: sendUserProfileEdit()
& D.E.\\
& & maap\_server::modelServer:: dataManager:: DatabaseUserManager::DatabaseUserManager:: updateUserProfile()
& D.E.\\
& & maap\_server::modelServer:: dataManager:: DatabaseUserManager::DatabaseUserManager:: getUsersList()
& D.E.\\
& & maap\_server::modelServer:: dataManager:: DatabaseUserManager::DatabaseUserManager:: sendUser()
& D.E.\\
& & maap\_server::modelServer:: dataManager:: DatabaseUserManager::DatabaseUserManager:: sendUserEdit()
& D.E.\\
& & maap\_server::modelServer:: dataManager:: DatabaseUserManager::DatabaseUserManager:: updateUser()
& D.E.\\
& & maap\_server::modelServer:: dataManager:: DatabaseUserManager::DatabaseUserManager:: removeUser()
& D.E.\\

\midrule
TU50
& Il test verifica che il recupero dei dati utente dal database avvenga in maniera corretta; in particolare verranno verificate le funzionalità di aggiunta utente, visualizzazione dati profilo, modifica utente e rimozione utente.
&  maap\_server::modelServer:: dataManager:: DatabaseUserManager::DataRetrieverUsers:: addUser()
& D.E.\\
& & maap\_server::modelServer:: dataManager:: DatabaseUserManager::DataRetrieverUsers:: getUserProfile()
& D.E.\\
& & maap\_server::modelServer:: dataManager:: DatabaseUserManager::DataRetrieverUsers::updateUserProfile()
& D.E.\\
& & maap\_server::modelServer:: dataManager:: DatabaseUserManager::DataRetrieverUsers::getUsersList()
& D.E.\\
& & maap\_server::modelServer:: dataManager:: DatabaseUserManager::DataRetrieverUsers::updateUser()
& D.E.\\
& & maap\_server::modelServer:: dataManager:: DatabaseUserManager::DataRetrieverUsers::removeUser()
& D.E.\\

\midrule
TU51
& Il test verifica che le query efettuate sul database avvengano in maniera corretta. Inoltre vengono controllate le funzionalità riguardanti la creazione di indici.
&  maap\_server::modelServer:: dataManager:: IndexManager::IndexManager:: getModel()
& D.E.\\
& & maap\_server::modelServer:: dataManager:: IndexManager::IndexManager:: addQuery()
& D.E.\\
& & maap\_server::modelServer:: dataManager:: IndexManager::IndexManager:: resetQueries()
& D.E.\\
& & maap\_server::modelServer:: dataManager:: IndexManager::IndexManager:: getQueries()
& D.E.\\
& & maap\_server::modelServer:: dataManager:: IndexManager::IndexManager:: getIndex()
& D.E.\\
& & maap\_server::modelServer:: dataManager:: IndexManager::IndexManager:: createIndex()
& D.E.\\
& & maap\_server::modelServer:: dataManager:: IndexManager::IndexManager:: createIndex()
& D.E.\\

\midrule
TU52
& Il test verifica che il JSON creato corrisponde alle richieste ricevute.
&  maap\_server::modelServer:: dataManager:: JSonComposer:: createCollectionsList()
& superato\\
& & maap\_server::modelServer:: dataManager:: JSonComposer:: checkLabels()
& superato\\
& & maap\_server::modelServer:: dataManager:: JSonComposer:: createCollection()
& superato\\
& & maap\_server::modelServer:: dataManager:: JSonComposer:: createDocument()
& superato\\
& & maap\_server::modelServer:: dataManager:: JSonComposer:: createQueriesList()
& superato\\
& & maap\_server::modelServer:: dataManager:: JSonComposer:: createIndexesList()
& superato\\
& & maap\_server::modelServer:: dataManager:: JSonComposer:: createUserProfile()
& superato\\
& & maap\_server::modelServer:: dataManager:: JSonComposer:: createUserProfileEdit()
& superato\\
& & maap\_server::modelServer:: dataManager:: JSonComposer:: createUsersList()
& superato\\
& & maap\_server::modelServer:: dataManager:: JSonComposer:: createUser()
& superato\\
& & maap\_server::modelServer:: dataManager:: JSonComposer:: createUser()
& superato\\

\midrule
TU53
& Il test verifica che esistono dei file DSL presenti in uno spazio apposito, e viene eseguito il parser su di essi.
&  maap\_server::modelServer:: DSL:: DSLManager:: generateFunction()
& superato\\
& & maap\_server::modelServer:: DSL:: DSLManager:: deleteFolderRecursive()
& D.E.\\
& & maap\_server::modelServer:: DSL:: DSLManager:: checkDSL()
& superato\\

\midrule
TU54
& Il test verifica che il parser di un file DSL venga effettuato in maniera corretta; vengono eseguiti i controlli sui campi scritti nel DSL per verificarne il contenuto.
&  maap\_server::modelServer:: DSL:: DSLParser:: addField()
& superato\\
& & maap\_server::modelServer:: DSL:: DSLParser:: checkFieldThrow()
& superato\\
& & maap\_server::modelServer:: DSL:: DSLParser:: checkField()
& superato\\
& & maap\_server::modelServer:: DSL:: DSLParser:: checkFieldContentThrow()
& superato\\
& & maap\_server::modelServer:: DSL:: DSLParser:: checkFieldContent()
& superato\\
& & maap\_server::modelServer:: DSL:: DSLParser:: IntValue()
& superato\\
& & maap\_server::modelServer:: DSL:: DSLParser:: parseDSL()
& superato\\

\midrule
TU55
& Il test verifica che il controllo sui file DSL parta all'avvio del server.
&  maap\_server::modelServer:: DSL:: index:: init()
& superato\\

\midrule
TU56
& Il test verifica che vengono creati gli schemi giusti per le collection scritte nel DSL; inoltre viene effettuato un controllo se gli schemi sono già presenti.
&  maap\_server::modelServer:: DSL:: schemaGenerator:: getPopulatedCollection()
& superato\\
& & maap\_server::modelServer:: DSL:: schemaGenerator:: arrayAddElement()
& superato\\
& & maap\_server::modelServer:: DSL:: schemaGenerator:: generate()
& superato\\

\end{longtable}
\end{center}



\subsubsection{Tracciamento}
Di seguito verranno riportati in forma tabellare i tracciamenti componente-test d'unità sia lato server che lato client.\\


\paragraph{Tracciamento componente server-test di unità}
\begin{center}
\begin{longtable}{|p{12cm}|p{2cm}|}
\toprule
\multicolumn{1}{|p{12cm}}{\textbf{Componente}}
& \multicolumn{1}{|p{2cm}}{\textbf{Test}}\\
\midrule
\endfirsthead
\multicolumn{2}{l}{\footnotesize\itshape\tablename~\thetable: continua dalla pagina precedente} \\
\toprule
\multicolumn{1}{|p{12cm}}{\textbf{Componente}}
& \multicolumn{1}{|p{2cm}}{\textbf{Test}}\\
\midrule
\endhead
\midrule
\multicolumn{2}{r}{\footnotesize\itshape\tablename~\thetable: continua nella prossima pagina} \\
\endfoot
\bottomrule
\caption{Tracciamento componente-test unit\'{a} parte server}
\endlastfoot

\midrule
maap\_server::controller::dispatcher:: dispatcherInit() & TU40\\

\midrule
maap\_server::controller::frontController:: initFrontController() & TU41\\


\midrule
maap\_server::controller::index:: init() & TU42\\

\midrule
maap\_server::controller::passport:: initPassport() & TU43\\

\midrule
maap\_server::controller::passport:: checkAuthenticatedAdmin() & TU43\\

\midrule
maap\_server::controller::passport:: checkAuthenticated() & TU43\\

\midrule
maap\_server::controller::passport:: checkNotAuthenticated() & TU43\\

\midrule
maap\_server::modelServer::database:: index::addAdminDefault() & TU44\\

\midrule
maap\_server::modelServer::database:: index::initDB() & TU44\\

\midrule
maap\_server::modelServer::database:: MongooseDBAnalysis::init() & TU45\\


\midrule
maap\_server::modelServer::database:: MongooseDBFramework::init() & TU46\\


\midrule
maap\_server::modelServer::dataManager:: DatabaseAnalysisManager::DatabaseAnalysisManager:: sendCollectionsList() & TU47\\

\midrule
maap\_server::modelServer::dataManager:: DatabaseAnalysisManager::DatabaseAnalysisManager:: sendCollection() & TU47\\

\midrule
maap\_server::modelServer::dataManager:: DatabaseAnalysisManager::DatabaseAnalysisManager:: sendDocument() & TU47\\

\midrule
maap\_server::modelServer::dataManager:: DatabaseAnalysisManager::DatabaseAnalysisManager:: sendDocumentEdit() & TU47\\

\midrule
maap\_server::modelServer::dataManager:: DatabaseAnalysisManager::DatabaseAnalysisManager:: updateDocument() & TU47\\

\midrule
maap\_server::modelServer::dataManager:: DatabaseAnalysisManager::DatabaseAnalysisManager:: removeDocument() & TU47\\

\midrule
maap\_server::modelServer::dataManager:: DatabaseAnalysisManager::DatabaseAnalysisManager:: resetQueries() & TU47\\

\midrule
maap\_server::modelServer::dataManager:: DatabaseAnalysisManager::DatabaseAnalysisManager:: getTopQueries() & TU47\\

\midrule
maap\_server::modelServer::dataManager:: DatabaseAnalysisManager::DatabaseAnalysisManager:: getIndexesList() & TU47\\

\midrule
maap\_server::modelServer::dataManager:: DatabaseAnalysisManager::DatabaseAnalysisManager:: createIndex() & TU47\\

\midrule
maap\_server::modelServer::dataManager:: DatabaseAnalysisManager::DatabaseAnalysisManager:: deleteIndex() & TU47\\

\midrule
maap\_server::modelServer::dataManager:: DatabaseAnalysisManager::DataRetrieverAnalysis:: getModel() & TU48\\

\midrule
maap\_server::modelServer::dataManager:: DatabaseAnalysisManager::DataRetrieverAnalysis:: getDocuments() & TU48\\

\midrule
maap\_server::modelServer::dataManager:: DatabaseAnalysisManager::DataRetrieverAnalysis:: getCollectionsList() & TU48\\

\midrule
maap\_server::modelServer::dataManager:: DatabaseAnalysisManager::DataRetrieverAnalysis:: applyTrasformations() & TU48\\

\midrule
maap\_server::modelServer::dataManager:: DatabaseAnalysisManager::DataRetrieverAnalysis:: sortDocumentsByLabels() & TU48\\

\midrule
maap\_server::modelServer::dataManager:: DatabaseAnalysisManager::DataRetrieverAnalysis:: getCollectionIndex() & TU48\\

\midrule
maap\_server::modelServer::dataManager:: DatabaseAnalysisManager::DataRetrieverAnalysis:: getDocumentShow() & TU48\\

\midrule
maap\_server::modelServer::dataManager:: DatabaseAnalysisManager::DataRetrieverAnalysis:: getDocumentShowEdit() & TU48\\

\midrule
maap\_server::modelServer::dataManager:: DatabaseAnalysisManager::DataRetrieverAnalysis:: updateDocument() & TU48\\

\midrule
maap\_server::modelServer::dataManager:: DatabaseAnalysisManager::DataRetrieverAnalysis:: removeDocument() & TU48\\


\midrule
maap\_server::modelServer::dataManager:: DatabaseUserManager::DatabaseUserManager:: checkMail() & TU49\\

\midrule
maap\_server::modelServer::dataManager:: DatabaseUserManager::DatabaseUserManager:: userSignup() & TU49\\

\midrule
maap\_server::modelServer::dataManager:: DatabaseUserManager::DatabaseUserManager:: sendUserProfile() & TU49\\

\midrule
maap\_server::modelServer::dataManager:: DatabaseUserManager::DatabaseUserManager:: sendUserProfileEdit() & TU49\\

\midrule
maap\_server::modelServer::dataManager:: DatabaseUserManager::DatabaseUserManager:: updateUserProfile() & TU49\\

\midrule
maap\_server::modelServer::dataManager:: DatabaseUserManager::DatabaseUserManager:: getUsersList() & TU49\\

\midrule
maap\_server::modelServer::dataManager:: DatabaseUserManager::DatabaseUserManager:: sendUser() & TU49\\

\midrule
maap\_server::modelServer::dataManager:: DatabaseUserManager::DatabaseUserManager:: sendUserEdit() & TU49\\

\midrule
maap\_server::modelServer::dataManager:: DatabaseUserManager::DatabaseUserManager:: updateUser() & TU49\\

\midrule
maap\_server::modelServer::dataManager:: DatabaseUserManager::DatabaseUserManager:: removeUser() & TU49\\

\midrule
maap\_server::modelServer::dataManager:: DatabaseUserManager::DataRetrieverUsers:: addUser() & TU50\\

\midrule
maap\_server::modelServer::dataManager:: DatabaseUserManager::DataRetrieverUsers:: getUserProfile() & TU50\\

\midrule
maap\_server::modelServer::dataManager:: DatabaseUserManager::DataRetrieverUsers::updateUserProfile() & TU50\\

\midrule
maap\_server::modelServer::dataManager:: DatabaseUserManager::DataRetrieverUsers::getUsersList() & TU50\\

\midrule
maap\_server::modelServer::dataManager:: DatabaseUserManager::DataRetrieverUsers::updateUser() & TU50\\

\midrule
maap\_server::modelServer::dataManager:: DatabaseUserManager::DataRetrieverUsers::removeUser() & TU50\\


\midrule
maap\_server::modelServer::dataManager:: IndexManager::IndexManager:: getModel() & TU51\\

\midrule
maap\_server::modelServer::dataManager:: IndexManager::IndexManager:: addQuery() & TU51\\

\midrule
maap\_server::modelServer::dataManager:: IndexManager::IndexManager:: resetQueries() & TU51\\

\midrule
maap\_server::modelServer::dataManager:: IndexManager::IndexManager:: getQueries() & TU51\\

\midrule
maap\_server::modelServer::dataManager:: IndexManager::IndexManager:: getIndex() & TU51\\

\midrule
maap\_server::modelServer::dataManager:: IndexManager::IndexManager:: createIndex() & TU51\\

\midrule
maap\_server::modelServer::dataManager:: IndexManager::IndexManager:: createIndex() & TU51\\

\midrule
maap\_server::modelServer::dataManager:: JSonComposer:: createCollectionsList() & TU52\\

\midrule
maap\_server::modelServer::dataManager:: JSonComposer:: checkLabels() & TU52\\

\midrule
maap\_server::modelServer::dataManager:: JSonComposer:: createCollection() & TU52\\

\midrule
maap\_server::modelServer::dataManager:: JSonComposer:: createDocument() & TU52\\

\midrule
maap\_server::modelServer::dataManager:: JSonComposer:: createQueriesList() & TU52\\

\midrule
maap\_server::modelServer::dataManager:: JSonComposer:: createIndexesList() & TU52\\

\midrule
maap\_server::modelServer::dataManager:: JSonComposer:: createUserProfile() & TU52\\

\midrule
maap\_server::modelServer::dataManager:: JSonComposer:: createUserProfileEdit() & TU52\\

\midrule
maap\_server::modelServer::dataManager:: JSonComposer:: createUsersList() & TU52\\

\midrule
maap\_server::modelServer::dataManager:: JSonComposer:: createUser() & TU52\\

\midrule
maap\_server::modelServer::dataManager:: JSonComposer:: createUser() & TU52\\

\midrule
maap\_server::modelServer::DSL::DSLManager:: generateFunction() & TU53\\

\midrule
maap\_server::modelServer::DSL::DSLManager:: deleteFolderRecursive() & TU53\\

\midrule
maap\_server::modelServer::DSL::DSLManager:: checkDSL() & TU53\\

\midrule
maap\_server::modelServer::DSL::DSLParser:: addField() & TU54\\

\midrule
maap\_server::modelServer::DSL::DSLParser:: checkFieldThrow() & TU54\\

\midrule
maap\_server::modelServer::DSL::DSLParser:: checkField() & TU54\\

\midrule
maap\_server::modelServer::DSL::DSLParser:: checkFieldContentThrow() & TU54\\

\midrule
maap\_server::modelServer::DSL::DSLParser:: checkFieldContent() & TU54\\

\midrule
maap\_server::modelServer::DSL::DSLParser:: IntValue() & TU54\\

\midrule
maap\_server::modelServer::DSL::DSLParser:: parseDSL() & TU54\\

\midrule
maap\_server::modelServer::DSL::index:: init() & TU55\\

\midrule
maap\_server::modelServer::DSL::schemaGenerator:: getPopulatedCollection() & TU56\\

\midrule
maap\_server::modelServer::DSL::schemaGenerator:: arrayAddElement() & TU56\\

\midrule
maap\_server::modelServer::DSL::schemaGenerator:: generate() & TU56\\


\end{longtable}
\end{center}



\paragraph{Tracciamento componente client-test di unità}
\begin{center}
\begin{longtable}{|p{12cm}|p{2cm}|}
\toprule
\multicolumn{1}{|p{12cm}}{\textbf{Componente}}
& \multicolumn{1}{|p{2cm}}{\textbf{Test}}\\
\midrule
\endfirsthead
\multicolumn{2}{l}{\footnotesize\itshape\tablename~\thetable: continua dalla pagina precedente} \\
\toprule
\multicolumn{1}{|p{12cm}}{\textbf{Componente}}
& \multicolumn{1}{|p{2cm}}{\textbf{Test}}\\
\midrule
\endhead
\midrule
\multicolumn{2}{r}{\footnotesize\itshape\tablename~\thetable: continua nella prossima pagina} \\
\endfoot
\bottomrule
\caption{Tracciamento componente-test unit\'{a} parte client}
\endlastfoot

\midrule
maap\_client::controller::Collection Controller & TU1\\

\midrule
maap\_client::controller::Dashboard Controller & TU2\\

\midrule
maap\_client::controller::Document Controller & TU3\\

\midrule
maap\_client::controller::DocumentEdit Controller & TU4\\

\midrule
maap\_client::controller::Index Controller & TU5\\

\midrule
maap\_client::controller::Login Controller & TU6\\

\midrule
maap\_client::controller::NavBar Controller & TU7\\

\midrule
maap\_client::controller::Profile Controller & TU8\\

\midrule
maap\_client::controller::ProfileEdit Controller & TU9\\

\midrule
maap\_client::controller::Query Controller & TU10\\

\midrule
maap\_client::controller::Register Controller & TU11\\

\midrule
maap\_client::controller::UserCollection Controller & TU12\\

\midrule
maap\_client::controller::UserController Controller & TU13\\

\midrule
maap\_client::controller::UserEdit Controller & TU14\\


\midrule
maap\_client::services::AuthService & TU15\\

\midrule
maap\_client::services::CollectionDataService & TU16\\

\midrule
maap\_client::services::CollectionListService & TU17\\

\midrule
maap\_client::services::DocumentDataService & TU18\\

\midrule
maap\_client::services::DocumentEditService & TU19\\

\midrule
maap\_client::services::IndexService & TU20\\

\midrule
maap\_client::services::LogoutService & TU21\\

\midrule
maap\_client::services::ProfileDataService & TU22\\

\midrule
maap\_client::services::ProfileEditService & TU23\\

\midrule
maap\_client::services::QueryService & TU24\\

\midrule
maap\_client::services::RegisterService & TU25\\

\midrule
maap\_client::services::UserCollectionService & TU26\\

\midrule
maap\_client::services::UserDataService & TU27\\

\midrule
maap\_client::services::UserEditService & TU28\\


\end{longtable}
\end{center}

\subsection{Test di sistema}
Di seguito verrà mostrata la tabella che riporta tutti i test di sistema pianificati, associati ai requisiti descritti nel documento Analisi dei Requisiti.\\
I test sono da intendere solo per requisiti ai quali è stato ragionevole associare un test.

\subsubsection{Descrizione dei test di sistema}
%la seguente tabella è generata automaticamente dallo scriptTest che
%prende i dati esportati dal database di Access e genera la tabella in latex
%QUESTE TABELLE SONO STATE GENERATE AUTOMATICAMENTE DA TESTSCRIPT [Apr 15 12:17:31 2014]

\begin{center}
\begin{longtable}{|p{2cm}|p{7cm}|p{2cm}|p{2cm}|}
\toprule
\multicolumn{1}{|p{2cm}}{\textbf{Test}}
& \multicolumn{1}{|p{7cm}}{\textbf{Descrizione}}
& \multicolumn{1}{|p{2cm}}{\textbf{Requisito}}
& \multicolumn{1}{|p{2cm}|}{\textbf{Stato}}\\
\midrule
\endfirsthead
\multicolumn{2}{l}{\footnotesize\itshape\tablename~\thetable: continua dalla pagina precedente} \\
\toprule
\multicolumn{1}{|p{2cm}}{\textbf{Test}}
& \multicolumn{1}{|p{7cm}}{\textbf{Descrizione}}
& \multicolumn{1}{|p{2cm}}{\textbf{Requisito}}
& \multicolumn{1}{|p{2cm}|}{\textbf{Stato}}\\
\midrule
\endhead
\midrule
\multicolumn{2}{r}{\footnotesize\itshape\tablename~\thetable: continua nella prossima pagina} \\
\endfoot
\bottomrule
\caption{Test di sistema}
\endlastfoot

\midrule
TS1
& Viene verificato che il sistema MaaP generi correttamente lo scheletro necessario.
& ROF1
& D.E.\\


\midrule
TS1.1
& Viene verificato che il sistema MaaP installi correttamente le librerie necessarie.
& ROF1.1
& D.E.\\


\midrule
TS1.2
& Viene verificato che il sistema MaaP generi correttamente i file necessari.
& ROF1.2
& D.E.\\


\midrule
TS1.3
& Viene verificato che il sistema MaaP generi correttamente le directory necessarie.
& ROF1.3
& D.E.\\


\midrule
TS1.4
& Viene verificato che il sottosistema di autenticazione sia installato e configurato correttamente.
& ROF1.4
& D.E.\\


\midrule
TS1.4.1
& Viene verificato che nel database degli utenti sia presente un profilo di amministrazione di dafault.
& ROF1.4.1
& D.E.\\


\midrule
TS1.4.1
& Viene verificato che nel database degli utenti sia presente un profilo di amministrazione di dafault.
& ROF6
& D.E.\\


\midrule
TS1.5
& Viene verificato che il sistema sia in grado di eliminare un progetto esistente.
& ROF1.5
& D.E.\\


\midrule
TS1.6
& Viene verificato che il sistema sia in grado di clonare un progetto esistente.
& ROF1.6
& D.E.\\


\midrule
TS4
& Viene verificato che il sistema crei correttamente le pagine web partendo dal loro file di descrizione.
& ROF4
& D.E.\\


\midrule
TS5.1
& Viene verificato che la funzione di registrazione possa essere correttamente abilitata/disabilitata.
& RDF5.1
& D.E.\\


\midrule
TS5.4
& Viene verificato che il sistema possa utilizzare correttamente il database di analisi.
& ROF5.4
& D.E.\\


\midrule
TS5.5
& Viene verificato che la funzione di creazione indici possa essere correttamnte abilitata/disabilitata.
& ROF5.5
& D.E.\\


\midrule
TS7
& Viene verificato che il sistema consenta all'utente registrato di potersi autenticare.
& ROF7
& D.E.\\


\midrule
TS8
& Viene verificato che il sistema consenta all'utente di potersi registrare.
& RDF8
& D.E.\\


\midrule
TS9
& Viene verificato che il sistema consenta all'utente di recuperare la password.
& ROF9
& D.E.\\


\midrule
TS10
& Viene verificato che il sistema apra e visualizzi correttamente le Collection e le Collection-Index.
& ROF10
& D.E.\\


\midrule
TS10.1
& Viene verificato che il sistema visualizzi correttamente le pagine Document-Show.
& ROF10.1
& D.E.\\


\midrule
TS10.2.4
& Viene verificato che il sistema disconnetta correttamente un utente alla sua richiesta.
& ROF10.2.4
& D.E.\\


\midrule
TS10.3.1.1
& Viene verificato che un utente autenticato possa modificare i dati del suo profilo.
& ROF10.3.1.1
& D.E.\\


\midrule
TS10.3.1.2
& Viene verificato che le modifiche apportate al profilo di un utente business autenticato siano consistenti.
& ROF10.3.1.2
& D.E.\\


\midrule
TS10.3.2
& Viene verificato che la creazione di un nuovo utente da parte di un utente business autenticato amministratore avvenga correttamente.
& ROF10.3.2
& D.E.\\


\midrule
TS10.3.3
& Viene verificata la corretta cancellazione di un  utente da parte di un utente business autenticato amministratore.
& ROF10.3.3
& D.E.\\


\midrule
TS10.4
& Viene verificato che l'utente business autenticato amministratore possa eliminare correttamente un Document.
& ROF10.4
& D.E.\\


\midrule
TS10.5
& Viene verificato che l'utente business autenticato amministratore possa modificare correttamente un Document.
& ROF10.5
& D.E.\\


\midrule
TS10.6
& Viene verificata la corretta visualizzazione delle query più utilizzate.
& ROF10.6
& D.E.\\


\midrule
TS10.7.1.2
& Viene verificata la corretta creazione degli indici di analisi.
& ROF10.7.1.2
& D.E.\\


\midrule
TS10.7.2
& Viene verificata la corretta eliminazione degli indici di analisi.
& ROF10.7.2
& D.E.\\


\midrule
TS10.7.3
& Viene verificata la corretta visualizzazione degli indici di analisi.
& ROF10.7.3
& D.E.\\


\midrule
TS17
& Viene verificato che le pagine web prodotte dal framework MaaP siano compatibili con la versione 30.0.x o superiore di Google Chrome.
& ROV18
& D.E.\\


\midrule
TS19
& Viene verificato che il sistema accetti solo file di configurazione validi.
& ROV20
& D.E.\\


\midrule
TS26
& Viene verificato che il sistema di installazione del software funzioni correttamente.
& ROV27
& D.E.\\


\midrule
TS27
& Viene verificato che il deployment su Heroku avvenga con successo.
& ROV28
& D.E.\\


\midrule
TS18
& Viene verificato che le pagine web prodotte dal framework MaaP siano compatibili con la versione 24.x o superiore di Firefox.
& ROV19
& D.E.\\


\end{longtable}
\end{center}
%QUESTE TABELLE SONO STATE GENERATE AUTOMATICAMENTE DA TESTSCRIPT [Apr 15 12:17:31 2014]



\subsection{Test d'integrazione}

\subsubsection{Diagramma d'integrazione}
\immagine{./integrazione}{Diagramma d'integrazione}
Di seguito verrà mostrata la tabella che riporta tutti i test d'integrazione pianificati, associati alle componenti descritte nella progettazione ad alto livello.\\

\subsubsection{Descrizione dei test d'integrazione}
%la seguente tabella è generata automaticamente dallo scriptTest che
%prende i dati esportati dal database di Access e genera la tabella in latex
%QUESTE TABELLE SONO STATE GENERATE AUTOMATICAMENTE DA TESTSCRIPT [Apr 15 12:17:31 2014]

\begin{center}
\begin{longtable}{|p{4.5cm}|p{3cm}|p{5.5cm}|c|}
\toprule
\multicolumn{1}{|p{4.5cm}}{\textbf{Test}}
& \multicolumn{1}{|p{3cm}}{\textbf{Descrizione}}
& \multicolumn{1}{|p{5.5cm}}{\textbf{Componente}}
& \multicolumn{1}{|c|}{\textbf{Stato}}\\
\midrule
\endfirsthead
\multicolumn{2}{l}{\footnotesize\itshape\tablename~\thetable: continua dalla pagina precedente} \\
\toprule
\multicolumn{1}{|p{4.5cm}}{\textbf{Test}}
& \multicolumn{1}{|p{3cm}}{\textbf{Descrizione}}
& \multicolumn{1}{|p{5.5cm}}{\textbf{Componente}}
& \multicolumn{1}{|c|}{\textbf{Stato}}\\
\midrule
\endhead
\midrule
\multicolumn{2}{r}{\footnotesize\itshape\tablename~\thetable: continua nella prossima pagina} \\
\endfoot
\bottomrule
\caption{Test d'integrazione}
\endlastfoot

\midrule
TI.MaaP
& Test di integrazione finale tra client e server.
& MaaP
& Superato.\\


\midrule
TI.Server
& Verifica il corretto funzionamento delle componenti del server, quindi la corretta integrazione tra ModelServer e Controller del server. In particolare, viene verificato che le funzionalità presenti nel Controller  debbano agire consistentemente sui dati del ModelServer
quindi la corretta integrazione
tra ModelServer e Controller
del server.
& MaaP::Server
& Superato.\\


\midrule
TI.Server.ModelServer
& Verifica che tutte le operazioni di lettura, scrittura e interpretazione dei dati avvengano correttamente.
& MaaP::Server::ModelServer
& Superato.\\


\midrule
TI.Server.ModelServer .DataManager-Database
& Verifica che i gestori dei dati siano correttamente integrati ai database di riferimento.
& MaaP::Server::ModelServer ::DataManager
& Superato.\\

\midrule
TI.Server.ModelServer .DataManager-Database
& Verifica che i gestori dei dati siano correttamente integrati ai database di riferimento.
& MaaP::Server::ModelServer ::Database
& Superato.\\


\midrule
TI.Server.ModelServer .DataManager
& Verifica che le operazioni di recupero, gestione e scrittura dati nei database avvengano correttamente.
& MaaP::Server::ModelServer ::DataManager
& Superato.\\


\midrule
TI.Client
& Verifica che i dati ottenuti e forniti da e verso il client siano corretti e correttamente gestiti all'interno del client stesso.
& MaaP::Client
& Superato.\\


\midrule
TI.Client.View-ControllerModelView
& Verifica che i dati uscenti ed entranti dalla View siano gestiti correttamente dal ControllerModelView.
& MaaP::Client::View
& Superato.\\


\midrule
TI.Client.View-ControllerModelView
& Verifica che i dati uscenti ed entranti dalla View siano gestiti correttamente dal ControllerModelView.
& MaaP::Client::ControllerModelView
& Superato.\\


\midrule
TI.Client.ControllerModelView-ModelClient
& Viene verificato che le funzionalità del ControllerModelView agiscano correttamente sul ModelClient.
& MaaP::Client::ControllerModelView
& Superato.\\


\midrule
TI.Client.ControllerModelView
& Viene verificato che le funzioni del ClientController modifichino correttamente i dati presenti nello scope.
& MaaP::Client::ModelClient
& Superato.\\


\midrule
TI.Client.ControllerModelView
& Viene verificato che le funzioni del ClientController modifichino correttamente i dati presenti nello scope.
& MaaP::Client::ControllerModelView
& Superato.\\


\end{longtable}
\end{center}
%QUESTE TABELLE SONO STATE GENERATE AUTOMATICAMENTE DA TESTSCRIPT [Apr 15 12:17:31 2014]



\subsubsection{Tracciamento}
Di seguito verranno riportati in forma tabellare i tracciamenti componente-test d'integrazione e test d'integrazione-componente.\\

\paragraph{Tracciamento componente-test d'integrazione}
%la seguente tabella è generata automaticamente dallo scriptTest che
%prende i dati esportati dal database di Access e genera la tabella in latex
%QUESTE TABELLE SONO STATE GENERATE AUTOMATICAMENTE DA TESTSCRIPT [Jul 17 17:11:50 2014]

\begin{center}
\begin{longtable}{|p{7cm}|p{7cm}|}
\toprule
\textbf{Componente} & \textbf{Test}\\

\midrule
MaaP::Client
& TI.Client\\

\midrule
MaaP::Client::ControllerModelView
& TI.Client.ControllerModelView\\

\midrule
MaaP::Client::ModelClient
& TI.Client.ControllerModelView\\

\midrule
MaaP::Client::ControllerModelView
& TI.Client.ControllerModelView-ModelClient\\
& TI.Client.View-ControllerModelView\\

\midrule
MaaP::Client::View
& TI.Client.View-ControllerModelView\\

\midrule
MaaP
& TI.MaaP\\

\midrule
MaaP::Server
& TI.Server\\

\midrule
MaaP::Server::ModelServer
& TI.Server.ModelServer\\
& TI.Server.ModelServer.DataManager\\
& TI.Server.ModelServer.DataManager-Database\\
& TI.Server.ModelServer.DataManager-Database\\

\bottomrule
\caption{Tracciamento componente-test d'integrazione}
\end{longtable}
\end{center}
%QUESTE TABELLE SONO STATE GENERATE AUTOMATICAMENTE DA TESTSCRIPT [Jul 17 17:11:50 2014]



\vspace{5cm}

\paragraph{Tracciamento test d'integrazione-componente}
%la seguente tabella è generata automaticamente dallo scriptTest che
%prende i dati esportati dal database di Access e genera la tabella in latex
\begin{center}
\begin{longtable}{|p{7cm}|p{7cm}|}
\toprule
\textbf{Test} & \textbf{Componente}\\
\midrule
TI.MaaP & MaaP\\
\midrule
TI.Server & Server\\
\midrule
TI.Server.ModelServer & Server::ModelServer\\
\midrule
TI.Server.ModelServer.DataManager-Database & Server::ModelServer::DataManager\\
& Server::ModelServer::Database\\
\midrule
TI.Server.ModelServer.DataManager & Server::ModelServer::DataManager\\
\midrule
TI.Client & Client\\
\midrule
TI.Client.View-ControllerModelView & Client::View\\
& Client::ControllerModelView\\
\midrule
TI.Client.ControllerModelView-ModelClient & Client::ControllerModelView\\
& Client::ModelClient\\
\midrule
TI.Client.ControllerModelView & Client::ControllerModelView\\
%inserire i test
\bottomrule
\caption{Tracciamento test d'integrazione-componente}
\label{tab:changelog}
\end{longtable}
\end{center}

\subsection{Test di validazione}
In questa sezione vengono descritti i test di validazione, utili ad accertarsi che il prodotto realizzato sia conforme alle attese.
Per ogni test viene descritta la serie di passi che un utente deve eseguire per validare la conformità ai requisiti ad esso associati.

\subsubsection{TV 1 - Lato Utente Business Autenticato}

L'utente vuole verificare le funzionalità di visualizzazione di Collection e Document, per tanto gli è richiesto di:


\begin{enumerate}
\item Cliccare sul bottone "Collections";
\item Selezionare la Collection da visualizzare dall'apposito elenco;
\begin{enumerate}
\item Scorrere le pagine della Collection fino a trovare il Document ricercato;
\item Ordinare una \gloss{chiave} in ordine crescente o decrescente;
\end{enumerate}
\item Selezionare il Document ricercato cliccando sul bottone "View";
\item Verificare che il Document venga visualizzato correttamente.
\end{enumerate}

In alternativa può:
\begin{enumerate}
\item Cliccare sul bottone "Dashboard" nel pannello superiore;
\item Inserire una stringa nella barra di ricerca delle collection;
\item  Verificare che la lista delle Collections, le quali soddisfano la ricerca, venga visualizzata correttamente;
\item Cliccare sul bottone con il nome della Collection che vuole visualizzare;
\begin{enumerate}
\item Scorrere le pagine della Collection fino a trovare il Document ricercato;
\item Ordinare una \gloss{chiave} in ordine crescente o decrescente;
\end{enumerate}
\item Selezionare il Document ricercato cliccando sul bottone "View";
\item Verificare che il Document venga visualizzato correttamente.
\end{enumerate}
L'utente vuole verificare la funzionalità di visualizzazione del manuale utente online.
Quindi è richiesto di:
\begin{enumerate}
\item Cliccare sul bottone "Help" del pannello superiore.
\end{enumerate}

\subsubsection{TV 2 - Lato Utente Business}
L'utente vuole verificare il corretto funzionamento delle funzionalità di registrazione, autenticazione e gestione profilo.
Si dovrà quindi seguire la seguente \gloss{procedura} di validazione:

\begin{enumerate}
\item Verificare di potersi registrare;
\begin{enumerate}
\item Cliccare sul link "Or become a member of the awesome MaaPerture community";
\item Compilare il \gloss{form} di registrazione con dati validi;
\item Confermare la registrazione mediante il bottone "SignUp".
\end{enumerate}
\item Verificare di poter recuperare la password in caso di dimenticanza;
\begin{enumerate}
\item Cliccare sul link di recupero password;
\item Inserire la \gloss{email} con cui si è registrati al sistema;
\item Verificare che il sistema invii correttamente la mail per il recupero della password;
\item Verificare che la password fornita nella mail consenta l'accesso al sistema.
\end{enumerate}
\item Verificare di poter effettuare l'autenticazione;
\begin{enumerate}
\item Inserire le credenziali precedentemente registrate;
\item Confermare l'autenticazione mediante il bottone "Login".
\end{enumerate}
\item Verificare che il sistema effettui correttamente l' autenticazione;
\item Verificare di poter modificare i propri dati utente;
\begin{enumerate}
\item Entrare nel proprio profilo utente cliccando il bottone "Profile" nel pannello superiore;
\item Cliccare sul bottone di "Edit";
\item Verificare di poter modificare una o più chiavi del profilo;
\item Verificare di poter salvare le modifiche mediante il bottone "Save";
\item Verificare di poter annullare le modifiche mediante il bottone "Discard".
\end{enumerate}
\item Verificare di poter aggiungere un nuovo utente;
\begin{enumerate}
\item Cliccare sul bottone "Manage Users" nel pannello superiore;
\item Compilare il form con dati validi;
\item Verificare di poter aggiungere l'utente mediante il bottone "Register new user";
\item Verificare che il nuovo utente sia presente nella lista degli utenti. 
\end{enumerate}
\item Verificare di potersi disconnettere correttamente dal sistema;
\begin{enumerate}
\item Cliccare sul bottone "Logout" nel pannello superiore;
\item Verificare di non poter più accedere a pagine riservate ad utenti autenticati.
\end{enumerate}
\item Verificare di poter consultare il manuale utente online;
\begin{enumerate}
\item Cliccare sul bottone link "Help".
\end{enumerate}

\end{enumerate}

\subsubsection{TV 3 - Lato Utente Business Autenticato Amministratore}

L'utente amministratore vuole verificare il corretto funzionamento della gestione dei profili utenti.
Si dovrà quindi seguire la seguente procedura:

\begin{enumerate}
\item Effettuare l'autenticazione con le proprie credenziali;
\begin{enumerate}
\item Al primo accesso assoluto al sistema, utilizzare le credenziali di default fornite dal sistema stesso.
\end{enumerate}
\item Verificare di poter accedere alla sezione per la gestione degli utenti;
\begin{enumerate}
\item Cliccare sul bottone "Manage User"s o in alternativa sul bottone "Collections" selezionando la Collection "Users";
\end{enumerate}
\item Verificare di poter modificare i dati profilo di un utente;
\begin{enumerate}
\item Cliccare sul bottone "Edit";
\item Verificare di poter modificare una o più chiavi del profilo;
\item Verificare di poter salvare le modifiche mediante il bottone "Save";
\item Verificare di poter annullare le modifiche mediante il bottone "Discard".
\end{enumerate}
\item Verificare di poter modificare i permessi associati al profilo utente;
\begin{enumerate}
\item Cliccare sul bottone "Edit";
\item Verificare di poter modificare i permessi disponibili;
\item Verificare di poter salvare le modifiche mediante il bottone "Save";
\item Verificare di poter annullare le modifiche mediante il bottone "Discard".
\end{enumerate}
\item Verificare di poter eliminare correttamente un utente;
\begin{enumerate}
\item Cliccare sul bottone "Delete";
\item Verificare che le credenziali dell'utente cancellato non siano più presenti nel sistema.
\end{enumerate}
\end{enumerate}


\subsubsection{TV 4 - Lato Utente Business Autenticato Amministratore}

L’utente amministratore vuole verificare il corretto funzionamento della gestione dei Document.
Dovrà quindi seguire la seguente procedura:


\begin{enumerate}
\item Dopo essere entrato in una pagina Collection, selezionare un Document mediante il bottone "View";
\item Verificare di poter modificare un Document;
\begin{enumerate}
\item Cliccare sul bottone "Edit";
\item Verificare di poter modificare le chiavi disponibili;
\item Verificare di poter salvare le modifiche mediante il bottone "Save";
\item Verificare di poter annullare le modifiche mediante il bottone "Discard".
\end{enumerate}
\item Verificare di poter cancellare un Document;
\begin{enumerate}
\item Cliccare sul bottone "Delete";
\item Verificare che il Document non sia più presente nel sistema.
\end{enumerate}
\end{enumerate}


\subsubsection{TV 5 - Lato Utente Business Autenticato Amministratore}
L’utente amministratore vuole verificare corretta gestione delle query più utilizzate dal sistema.
Dovrà quindi seguire la seguente procedura:

\begin{enumerate}
\item Entrare nella gestione queries;
\begin{enumerate}
\item Cliccare sul bottone "Collections";
\item Selezionare la Collection "Queries".
\end{enumerate}
\item Verificare che l'elenco delle queries più utilizzate venga visualizzato correttamente;
\item Verificare di poter creare un indice su una query;
\begin{enumerate}
\item Cliccare sul bottone "Create Index";
\item Verificare che venga creato l'indice nel sistema;
\begin{enumerate}
\item Verificare la corretta visualizzazione del \gloss{pop-up} di conferma di avvenuta creazione;
\item Cliccare sul bottone "Ok" all'interno del pop-up;
\item Verificare la corretta visualizzazione dell'elenco degli indici nel sistema.
\end{enumerate}
\end{enumerate}
\item Verificare che si possa visualizzare il comando di creazione indice da riga di comando;
\begin{enumerate}
\item Cliccare sul bottone "Index cmd";
Verificare la corretta visualizzazione del pop-up contenente il comando.
\end{enumerate}

\end{enumerate}
\subsubsection{TV 6 - Lato Utente Business Autenticato Amministratore}

L’utente amministratore vuole verificare il corretto funzionamento della gestione degli indici.
Dovrà quindi seguire la seguente procedura:

\begin{enumerate}
\item Entrare nella gestione degli indici;
\begin{enumerate}
\item Cliccare sul bottone "Collections";
\item Selezionare la Collection "Indexes".
\end{enumerate}
\item Verificare di poter eliminare un indice;
\begin{enumerate}
\item Cliccare sul bottone "Delete Index";
\item Verificare che l'indice non sia più presente.
\end{enumerate}
\end{enumerate}

\subsubsection{TV 7 - Lato Utente Sviluppatore}

L'utente sviluppatore vuole verificare il corretto funzionamento della gestione ed interpretazione del \gloss{DSL} per la creazione di una pagina Collection-Index.
Dovrà quindi seguire la seguente procedura:

\begin{enumerate}
\item Scrivere un file DSL valido;
\begin{enumerate}
\item Specificare le chiavi da visualizzare;
\item Specificare la provenienza dei dati;
\begin{enumerate}
\item Provenienti da query standard;
\item Provenienti da query personalizzate;
\item Provenienti da funzioni personalizzate;
\item Provenienti da riferimenti esterni.
\end{enumerate}
\item Specificare eventuali etichette;
\item Specificare il numero massimo di Document per pagina;
\item Specificare il nome della Collection;
\item Specificare la posizione della Collection nel menù.
\end{enumerate}
\item Caricare il file DSL nell'apposita cartella;
\item Avviare il server MaaP;
\item Effettuare l'autenticazione;
\item Entrare nella Collection la cui pagina è descritta dal file DSL inserito e verificare la bontà della sua interpretazione.
\end{enumerate}


\subsubsection{TV 8 - Lato Utente Sviluppatore}

L’utente sviluppatore vuole verificare il corretto funzionamento della gestione ed interpretazione del DSL per la creazione di una pagina Document-Show.
Dovrà quindi seguire la seguente procedura:

\begin{enumerate}
\item Scrivere un file DSL valido;
\begin{enumerate}
\item Specificare le chiavi da visualizzare;
\item Specificare la provenienza dei dati;
\begin{enumerate}
\item Provenienti da query standard;
\item Provenienti da query personalizzate;
\item Provenienti da funzioni personalizzate;
\item Provenienti da riferimenti esterni.
\end{enumerate}
\item Specificare eventuali etichette per le chiavi;
\end{enumerate}
\item Caricare il file DSL nell'apposita cartella;
\item Avviare il server MaaP;
\item Effettuare l'autenticazione;
\item Entrare nel document la cui pagina è descritta dal file DSL inserito e verificare la bontà della sua interpretazione.
\end{enumerate}


\subsubsection{TV 9 - Lato Utente Sviluppatore}

L'utente sviluppatore vuole verificare il corretto funzionamento delle funzioni di gestione del progetto.
Dovrà quindi seguire la seguente procedura:

\begin{enumerate}
\item Creare un nuovo progetto mediante il comando create da riga di comando;
\item Verificare che il nuovo progetto sia disponibile;
\item Clonare un progetto esistente in una directory diversa utilizzando la combinazione di tasti Ctrl+c per copiare e Ctrl+v per incollare;
\item Verificare che il progetto clonato sia disponibile;
\item Verificare di poter gestire un progetto;
\begin{enumerate}
\item Verificare di poter avviare, fermare o riavviare il server MaaP;
\item Verificare di poter modificare i file di configurazione del progetto;
\begin{enumerate}
\item Abilitare/disabilitare la registrazione;
\item Abilitare/disabilitare la creazione di indici;
\item Impostare la connessione al database di analisi.
\end{enumerate}
\item Verificare di poter modificare o caricare nuovi file DSL;
\item Verificare di poter modificare i \gloss{template} disponibili.
\end{enumerate}
\item Eliminare un progetto il tasto canc;
\item Verificare che il progetto eliminato non sia più disponibile.
\end{enumerate}

\subsubsection{Tracciamento}
Di seguito verranno riportati in forma tabellare i tracciamenti test di validazione - requisiti e l'inverso, ovvero requisiti - test di validazione.\\

\paragraph{Tracciamento Test di Validazione - Requisiti}
%la seguente tabella è generata automaticamente dallo scriptTest che
%prende i dati esportati dal database di Access e genera la tabella in latex
%QUESTE TABELLE SONO STATE GENERATE AUTOMATICAMENTE DA TESTSCRIPT [Apr 15 12:17:32 2014]

\begin{center}
\begin{longtable}{|p{7cm}|p{7cm}|}
\toprule
\multicolumn{1}{|p{7cm}}{\textbf{Test}}
& \multicolumn{1}{|p{7cm}|}{\textbf{Requisiti}} \\
\midrule
\endfirsthead
\multicolumn{2}{l}{\footnotesize\itshape\tablename~\thetable: continua dalla pagina precedente} \\
\toprule
\multicolumn{1}{|p{7cm}}{\textbf{Test}}
& \multicolumn{1}{|p{7cm}|}{\textbf{Requisiti}} \\
\midrule
\endhead
\midrule
\multicolumn{2}{r}{\footnotesize\itshape\tablename~\thetable: continua nella prossima pagina} \\
\endfoot
\bottomrule
\caption{Tracciamento Test di validazione - Requisiti}
\endlastfoot


\midrule
TV1
& RDF10.2\\
& RDF10.2.1\\
& RDF10.2.1.1\\
& RDF10.2.1.2\\
& RDF10.2.2\\
& RDF10.2.3\\
& ROF10\\
& ROF10.1\\
& ROF10.2.5\\

\midrule
TV2
& RDF8\\
& RDF8.1\\
& RDF8.2\\
& RDF8.2.1\\
& ROF10.2.4\\
& ROF10.3\\
& ROF10.3.1\\
& ROF10.3.1.1\\
& ROF10.3.1.2\\
& ROF10.3.1.3\\
& ROF7\\
& ROF7.1\\
& ROF7.2\\
& ROF7.2.1\\
& ROF9\\

\midrule
TV3
& ROF10.3.1.4\\
& ROF10.3.2\\
& ROF10.3.3\\
& ROF6\\

\midrule
TV4
& ROF10.4\\
& ROF10.5\\
& ROF10.5.1\\
& ROF10.5.2\\
& ROF10.5.3\\

\midrule
TV5
& ROF10.6\\
& ROF10.7\\
& ROF10.7.1\\
& ROF10.7.1.1\\
& ROF10.7.1.2\\
& ROF10.7.2\\
& ROF10.7.2.1\\
& ROF10.7.2.2\\
& ROF10.7.3\\

\midrule
TV6
& RFF4.2.2.7\\
& ROF3\\
& ROF4\\
& ROF4.1\\
& ROF4.1.2\\
& ROF4.1.2.1\\
& ROF4.1.2.1.1\\
& ROF4.1.2.1.2\\
& ROF4.1.2.1.3\\
& ROF4.1.2.1.4\\
& ROF4.1.2.1.5\\
& ROF4.1.2.2\\
& ROF4.1.2.3\\
& ROF4.2\\
& ROF4.2.1\\
& ROF4.2.1.1\\
& ROF4.2.1.2\\
& ROF4.2.2\\
& ROF4.2.2.1\\
& ROF4.2.2.2\\
& ROF4.2.2.3\\
& ROF4.2.2.4\\
& ROF4.2.2.5\\
& ROF4.2.2.6\\
& ROF4.3\\
& ROF4.4\\

\midrule
TV7
& ROF4.1.3\\
& ROF4.1.3.1\\
& ROF4.1.3.1.1\\
& ROF4.1.3.1.2\\
& ROF4.1.3.1.3\\
& ROF4.1.3.1.4\\
& ROF4.1.3.1.5\\
& ROF4.2.3\\
& ROF4.2.3.1\\
& ROF4.2.3.2\\
& ROF4.3\\
& ROF4.4\\

\midrule
TV8
& RDF5.1\\
& RDF5.3\\
& ROF5\\
& ROF5.4\\
& ROF5.5\\

\end{longtable}
\end{center}
%QUESTE TABELLE SONO STATE GENERATE AUTOMATICAMENTE DA TESTSCRIPT [Apr 15 12:17:32 2014]



\paragraph{Tracciamento Requisiti - Test di Validazione}
%la seguente tabella è generata automaticamente dallo scriptTest che
%prende i dati esportati dal database di Access e genera la tabella in latex
%QUESTE TABELLE SONO STATE GENERATE AUTOMATICAMENTE DA TESTSCRIPT [Apr 15 12:17:32 2014]

\begin{center}
\begin{longtable}{|p{7cm}|p{7cm}|}
\toprule
\multicolumn{1}{|p{7cm}}{\textbf{Requisiti}}
& \multicolumn{1}{|p{7cm}|}{\textbf{Test}} \\
\midrule
\endfirsthead
\multicolumn{2}{l}{\footnotesize\itshape\tablename~\thetable: continua dalla pagina precedente} \\
\toprule
\multicolumn{1}{|p{7cm}}{\textbf{Requisiti}}
& \multicolumn{1}{|p{7cm}|}{\textbf{Test}} \\
\midrule
\endhead
\midrule
\multicolumn{2}{r}{\footnotesize\itshape\tablename~\thetable: continua nella prossima pagina} \\
\endfoot
\bottomrule
\caption{Tracciamento Requisiti - Test di validazione}
\endlastfoot


\midrule
RDF10.2.3
& TV1\\

\midrule
ROF10
& TV1\\

\midrule
ROF10.2.5
& TV1\\

\midrule
RDF10.2.2
& TV1\\

\midrule
RDF10.2.1.2
& TV1\\

\midrule
RDF10.2.1.1
& TV1\\

\midrule
RDF10.2.1
& TV1\\

\midrule
RDF10.2
& TV1\\

\midrule
ROF10.1
& TV1\\

\midrule
ROF10.3.1
& TV2\\

\midrule
RDF8.2.1
& TV2\\

\midrule
ROF10.3.1.1
& TV2\\

\midrule
ROF10.3
& TV2\\

\midrule
ROF10.2.4
& TV2\\

\midrule
ROF9
& TV2\\

\midrule
ROF10.3.1.3
& TV2\\

\midrule
RDF8.1
& TV2\\

\midrule
RDF8
& TV2\\

\midrule
ROF7.2.1
& TV2\\

\midrule
ROF7.2
& TV2\\

\midrule
ROF7.1
& TV2\\

\midrule
ROF7
& TV2\\

\midrule
RDF8.2
& TV2\\

\midrule
ROF10.3.1.2
& TV2\\

\midrule
ROF6
& TV3\\

\midrule
ROF10.3.1.4
& TV3\\

\midrule
ROF10.3.2
& TV3\\

\midrule
ROF10.3.3
& TV3\\

\midrule
ROF10.4
& TV4\\

\midrule
ROF10.5.3
& TV4\\

\midrule
ROF10.5.2
& TV4\\

\midrule
ROF10.5
& TV4\\

\midrule
ROF10.5.1
& TV4\\

\midrule
ROF10.7.3
& TV5\\

\midrule
ROF10.6
& TV5\\

\midrule
ROF10.7
& TV5\\

\midrule
ROF10.7.1
& TV5\\

\midrule
ROF10.7.1.1
& TV5\\

\midrule
ROF10.7.1.2
& TV5\\

\midrule
ROF10.7.2
& TV5\\

\midrule
ROF10.7.2.1
& TV5\\

\midrule
ROF10.7.2.2
& TV5\\

\midrule
ROF4.2.2.4
& TV6\\

\midrule
ROF4.4
& TV6\\

\midrule
ROF4.3
& TV6\\

\midrule
RFF4.2.2.7
& TV6\\

\midrule
ROF3
& TV6\\

\midrule
ROF4.2.2.5
& TV6\\

\midrule
ROF4
& TV6\\

\midrule
ROF4.2.2.3
& TV6\\

\midrule
ROF4.2.2.2
& TV6\\

\midrule
ROF4.2.2.1
& TV6\\

\midrule
ROF4.2.2
& TV6\\

\midrule
ROF4.2.1.2
& TV6\\

\midrule
ROF4.2.1.1
& TV6\\

\midrule
ROF4.1.2.1.1
& TV6\\

\midrule
ROF4.2.2.6
& TV6\\

\midrule
ROF4.2.1
& TV6\\

\midrule
ROF4.1
& TV6\\

\midrule
ROF4.1.2.1
& TV6\\

\midrule
ROF4.1.2.1.2
& TV6\\

\midrule
ROF4.1.2.1.3
& TV6\\

\midrule
ROF4.1.2.1.4
& TV6\\

\midrule
ROF4.1.2.1.5
& TV6\\

\midrule
ROF4.1.2.2
& TV6\\

\midrule
ROF4.1.2.3
& TV6\\

\midrule
ROF4.2
& TV6\\

\midrule
ROF4.1.2
& TV6\\

\midrule
ROF4.1.3.1
& TV7\\

\midrule
ROF4.2.3
& TV7\\

\midrule
ROF4.4
& TV7\\

\midrule
ROF4.3
& TV7\\

\midrule
ROF4.2.3.2
& TV7\\

\midrule
ROF4.2.3.1
& TV7\\

\midrule
ROF4.1.3.1.5
& TV7\\

\midrule
ROF4.1.3.1.4
& TV7\\

\midrule
ROF4.1.3.1.3
& TV7\\

\midrule
ROF4.1.3.1.1
& TV7\\

\midrule
ROF4.1.3
& TV7\\

\midrule
ROF4.1.3.1.2
& TV7\\

\midrule
ROF5.5
& TV8\\

\midrule
ROF5
& TV8\\

\midrule
RDF5.1
& TV8\\

\midrule
RDF5.3
& TV8\\

\midrule
ROF5.4
& TV8\\

\end{longtable}
\end{center}
%QUESTE TABELLE SONO STATE GENERATE AUTOMATICAMENTE DA TESTSCRIPT [Apr 15 12:17:32 2014]

