%includo il file che contiene la versione dei documenti
\newcommand{\versioneAnalisiDeiRequisiti}{2.2.0}			
\newcommand{\versioneNormeDiProgetto}{2.2.0}			
\newcommand{\versioneGlossario}{2.2.0}			
\newcommand{\versionePianoDiQualifica}{2.2.0}			
\newcommand{\versionePianoDiProgetto}{2.2.0}	
\newcommand{\versioneStudioDiFattibilita}{2.2.0}
\newcommand{\versioneSpecificaTecnica}{2.2.0}


\newcommand{\Versione}{\versionePianoDiQualifica{}}	%Versione Finale
\newcommand{\Data}{2013-11-28}						%Data di creazione
\newcommand{\DataUltimaModifica}{2014-06-26}
\newcommand{\TipoDocumento}{Piano di Qualifica}		%tipo documento


%includo il file header.tex (logo grande in prima pagina piu qualche altra regola)
%questo file contiene impostazioni comuni per tutte i documenti

%definizione packages utilizzati
\documentclass[a4paper]{article}
\usepackage[utf8x]{inputenc}
\usepackage{enumitem}
\usepackage[italian]{babel}
\usepackage{latexsym}
\usepackage{xparse}
\usepackage{float}
\usepackage{subfloat}
\usepackage{subfig}
\usepackage{fancyhdr}
\usepackage{eurofont}
\usepackage{lastpage}
\usepackage{graphicx}
\usepackage{textcomp}
\usepackage{booktabs}
\usepackage{color}
\usepackage{lscape}
\usepackage{hyperref}
\hypersetup{colorlinks=true, linkcolor=black, anchorcolor=red, urlcolor=blue}
\usepackage{longtable}
\usepackage{tabularx}
\usepackage{abstract}
\usepackage{appendix}
\usepackage{multicol}
\usepackage{bmpsize}
\usepackage[all]{hypcap}
\usepackage{titlesec}
\usepackage{indentfirst}
\usepackage{lipsum,titletoc}

%\setcounter{secnumdepth}{4}

%****************INIZIO GESTIONE SUBSECTION MULTIPLE
\makeatletter
\newcommand\level[1]{%
  \ifcase#1\relax\expandafter\chapter\or
    \expandafter\section\or
    \expandafter\subsection\or
    \expandafter\subsubsection\else
    \def\next{\@level{#1}}\expandafter\next
  \fi}
\newcommand{\@level}[1]{%
  \@startsection{level#1}
    {#1}
    {\z@}%
    {-3.25ex\@plus -1ex \@minus -.2ex}%
    {1.5ex \@plus .2ex}%
    {\normalfont\normalsize\bfseries}}

\newdimen\@leveldim
\newdimen\@dotsdim
{\normalfont\normalsize
 \sbox\z@{0}\global\@leveldim=\wd\z@
 \sbox\z@{.}\global\@dotsdim=\wd\z@
}

\newcounter{level4}[subsubsection]
\@namedef{thelevel4}{\thesubsubsection.\arabic{level4}}
\@namedef{level4mark}#1{}
\def\l@section{\@dottedtocline{1}{0pt}{\dimexpr\@leveldim*4+\@dotsdim*1+6pt\relax}}
\def\l@subsection{\@dottedtocline{2}{0pt}{\dimexpr\@leveldim*5+\@dotsdim*2+6pt\relax}}
\def\l@subsubsection{\@dottedtocline{3}{0pt}{\dimexpr\@leveldim*6+\@dotsdim*3+6pt\relax}}
\@namedef{l@level4}{\@dottedtocline{4}{0pt}{\dimexpr\@leveldim*7+\@dotsdim*4+6pt\relax}}

\count@=4
\def\@ncp#1{\number\numexpr\count@+#1\relax}
\loop\ifnum\count@<100
  \begingroup\edef\x{\endgroup
    \noexpand\newcounter{level\@ncp{1}}[level\number\count@]
    \noexpand\@namedef{thelevel\@ncp{1}}{%
      \noexpand\@nameuse{thelevel\@ncp{0}}.\noexpand\arabic{level\@ncp{1}}}
    \noexpand\@namedef{level\@ncp{1}mark}####1{}%
    \noexpand\@namedef{l@level\@ncp{1}}%
      {\noexpand\@dottedtocline{\@ncp{1}}{0pt}{\the\dimexpr\@leveldim*\@ncp{5}+\@dotsdim*\@ncp{0}\relax}}}%
  \x
  \advance\count@\@ne
\repeat
\makeatother
\setcounter{secnumdepth}{100}
\setcounter{tocdepth}{100}
%****************FINE GESTIONE SUBSECTION MULTIPLE

%impostazioni relative alla visualizzazione delle section 
%nell'indice
\titlecontents{section}
[0pt]%left indent
{\bfseries}
{\contentslabel{2.3em}}
{\hspace*{-2.3em}}
{\hfill\contentspage}
[]%separator


\oddsidemargin=.15in
\evensidemargin=.15in
\textwidth=6in
\topmargin=-.5in
\parindent=0in
\headheight=1in
\DeclareMathSizes{10}{10}{10}{10} %per piano qualifica
\pagestyle{fancy}
\lhead{
\bfseries {\Large \TipoDocumento}\\
\bfseries Versione: \Versione\\
}
\chead{}
\lhead{
\includegraphics[scale=0.455]{../Logo&Header/apertureHead.png}
}
\lfoot{\bfseries \TipoDocumento{} v\Versione}
\cfoot{}
\rfoot{\thepage\ of \mypageref{LastPage}}
\newcommand{\mypageref}[1]{
\hypersetup{linkcolor=black}\pageref{#1}\hypersetup{linkcolor=black}}
%\userpackage{lipsum}
\renewcommand{\footrulewidth}{0.4pt}

%definizioni comandi comuni utilizzati
\newcommand{\numref}[1]{\textsl{\nameref{#1} (\ref{#1})}}
\newcommand{\NomeGruppo}{Aperture Software}
\newcommand{\Progetto}{MaaP: MongoDB as an admin Platform}
\newcommand{\Prop}{CoffeeStrap}

%definizione tecnologie
\newcommand{\Node}{Node.js}
\newcommand{\NodeJS}{Node.js}
\newcommand{\Nodejs}{Node.js}

\newcommand{\mongodb}{MongoDB}

%tanti sub quanti ne vogliamo! :)
\newcommand{\subsubsubsection}{\level{4}}
\newcommand{\subsubsubsubsection}{\level{5}}
\newcommand{\subsubsubsubsubsection}{\level{6}}
\newcommand{\subsubsubsubsubsubsection}{\level{7}}
\newcommand{\subsubsubsubsubsubsubsection}{\level{8}}


%definizione comando per parola glossario
\newcommand{\gloss}[1]{\emph{#1}\ped{\emph{\tiny{G}}}}

\newcommand{\grassetto}{\textbf}

%per inserire immagini
\newcommand{\immagine}[2]{ 
\begin{center}
\begin{figure}[H]
\includegraphics[width=\textwidth]{{{#1}}}
\caption{#2}
\label{#1}
\end{figure}
\end{center}
}

\newcommand{\Glossario}{
Al fine di evitare ogni ambiguità nella comprensione del linguaggio utilizzato nel presente documento e, in generale, nella documentazione fornita dal gruppo \NomeGruppo{}, ogni termine tecnico, di difficile comprensione o di necessario approfondimento verrà inserito nel documento \emph{Glossario\_{}v\versioneGlossario{}.pdf}.\\
Saranno in esso definiti e descritti tutti i termini in corsivo e allo stesso tempo marcati da una lettera "G" maiuscola in pedice nella documentazione fornita.
}

\newcommand{\Prodotto}{
Lo scopo del prodotto è produrre un framework per generare interfacce web di amministrazione dei dati di business basati sullo stack \Nodejs{} e \mongodb{}.\\
L'obiettivo è quello di semplificare il lavoro allo sviluppatore che dovrà rispondere in modo rapido e standard alle richieste degli esperti di business.
}

%inizio pagina del documento 
\begin{document}
\thispagestyle{empty}

\begin{center}\centerline{
%inserisco il logo grande della prima pagina
\includegraphics[scale=0.8]{../Logo&Header/logo.png}}

%metto il link dell'email sotto al logo
%{\href{mailto:ApertureSWE@gmail.com}{\color[rgb]{0.39,0.37,0.38}%ApertureSWE@gmail.com}}\\ [3pc]

\vspace{0.5in}

%titolo del progetto
{\Huge {\Progetto}}\\[.5pc]

\underline{\hspace{6in}}\\[8pc]

{\Huge {\TipoDocumento}}\\[1pc]
%{\emph{Versione \Versione}}\\
\end{center}

%\vspace{.05in}

%\vspace{.05in}

%informazioni documento
\begin{center}
%\section{Informazioni documento}
\begin{tabular}{r|l}
%\textbf{Nome} &\TipoDocumento \\
\textbf{Versione} & \Versione{} \\
\textbf{Data creazione} & \Data{} \\
\textbf{Data ultima modifica} & \DataUltimaModifica{} \\
\textbf{Stato del Documento} & Formale \\		%CAMBIARE QUI
\textbf{Uso del Documento} & Esterno \\			%CAMBIARE QUI
\textbf{Redazione} & Alessandro Benetti, Giacomo Pinato, Fabio Miotto,\\			%CAMBIARE QUI
\textbf{Verifica} & Alberto Garbui, Michele Maso\\%ED ANCHE QUI!
\textbf{Approvazione} & Mattia Sorgato\\				%CAMBIARE QUI
\textbf{Distribuzione} & \parbox[t]{4cm}{Prof. Tullio Vardanega \\ Prof. Riccardo Cardin \\ \Prop{} }
\end{tabular}
\end{center}

\vspace{0.05in}

%inizio sommario del documento
\begin{abstract}
\begin{center}
Questo documento ha lo scopo di presentare le strategie adottate dal gruppo \NomeGruppo{} nell'ottica del miglioramento continuo e assicurazione della qualità.
\end{center}
\end{abstract}

%\vspace{.4in}

%seconda pagina, diario delle modifiche
\newpage
Diario delle modifiche
\begin{center}
\begin{longtable}{|c|c|c|p{0.5\linewidth}|}
\toprule
\textbf{Versione} & \textbf{Data} & \textbf{Autore} & \textbf{Modifiche effettuate}\\

\midrule
4.2.0 & 2014-06-26 & Mattia Sorgato (RE) & Approvazione documento.\\

\midrule
4.1.1 & 2014-06-26 & Michele Maso (VR) & Verifica documento.\\

\midrule
4.1.0 & 2014-06-25 & Alberto Garbui (VR) & Verifica documento.\\

\midrule
4.0.5 & 2014-06-24 & Giacomo Pinato (VR) & Resoconto attività di Verifica\\

\midrule
4.0.4 & 2014-05-30 & Fabio Miotto (VR) & Stesura esiti test\\

\midrule
4.0.3 & 2014-05-16 & Alessandro Benetti (PR) & Aggiunti test di unità lato client\\

\midrule
4.0.2 & 2014-05-13 & Alessandro Benetti (PR) & Aggiunti test di unità lato server\\

\midrule
4.0.1 & 2014-04-08 &  Fabio Miotto (AN) & Effettuate correzioni segnalate dal Committente.\\

\midrule
3.2.0 & 2014-03-25 & Fabio Miotto (RE) & Approvazione documento.\\
\midrule
3.1.1 & 2014-03-24 & Alessandro Benetti (VR) & Verifica documento.\\
\midrule
3.1.0 & 2014-03-21 & Fabio Miotto (VR) & Verifica documento.\\
\midrule
3.0.8 & 2014-03-19 & Mattia Sorgato (VR) & Resoconto attività di verifica.\\
\midrule
3.0.7 & 2014-03-19 & Giacomo Pinato (VR) & Resoconto attività PDCA.\\
\midrule
3.0.6 & 2014-03-12 & Giacomo Pinato (PR) & Aggiunta test di validazione e tracciamento.\\
\midrule
3.0.5 & 2014-03-12 & Mattia Sorgato (PR) & Aggiunta test di integrazione e tracciamento.\\
\midrule
3.0.4 & 2014-03-10 & Mattia Sorgato (PR) & Aggiunta test di sistema.\\
\midrule
3.0.3 & 2014-02-28 & Michele Maso (PR) & Incremento documento con progettazione test.\\
\midrule
3.0.2 & 2014-01-17 & Alberto Garbui (AM) & Aggiunta analisi RR documento.\\
\midrule
3.0.1 & 2014-01-16 & Fabio Miotto (AM) & Aggiunta analisi RR documento.\\

\midrule
3.0.1 & 2014-01-14 & Fabio Miotto (AN) & Effettuate correzioni segnalate dal Committente.\\

\midrule
2.2.0 & 2014-01-07 & Alberto Garbui (RE) & Approvazione documento.\\
\midrule
2.1.0 & 2014-01-06 & Fabio Miotto (VR) & Verifica documento.\\
\midrule
2.0.3 & 2014-01-04 & Andrea Perin (VR) & Esito metriche processi.\\
\midrule
2.0.2 & 2014-01-04  & Andrea Perin (VR) & Esito metriche documenti.\\
\midrule
2.0.1 & 2014-01-03  & Michele Maso (AN) & Incremento documento.\\

\midrule
1.2.0 & 2013-12-16 & Giacomo Pinato (RE) & Approvazione documento\\
\midrule
1.1.1 & 2013-12-16 & Alessandro Benetti (VR) & Verifica documento\\
\midrule
1.1.0 & 2013-12-15 & Fabio Miotto (VR) & Verifica documento\\
\midrule
1.0.6 & 2013-12-13 & Giacomo Pinato (RE) & Aggiunto resoconto attività di Verifica e standard di qualità\\
\midrule
1.0.4 & 2013-12-04 & Mattia Sorgato (AM) & Aggiunta metriche\\
\midrule
1.0.3 & 2013-12-03 & Alberto Garbui (AN) & Aggiunta analisi\\
\midrule
1.0.2 & 2013-12-01 & Andrea Perin (RE) & Aggiunta strategie di Verifica\\
\midrule
1.0.1 & 2013-11-28 & Andrea Perin (RE) & Creazione documento\\

\bottomrule
\caption{Registro delle modifiche}
\label{tab:changelog}
\end{longtable}
\end{center}

%terza pagina Indice (viene aggiornato in automatico con due compilazioni)
\newpage
\tableofcontents

%pagine successive hanno la lista di tabelle e lista delle figure
%(vengono aggiornate in automatico)
\newpage
\listoftables
\listoffigures

%qui inizia la prima pagina ufficiale
\newpage
\section{Introduzione}%1.0
\label{1.0}
\subsection{Scopo del documento}%1.1
\label{1.1}
Il Piano di Qualifica ha l'obiettivo di definire le strategie adottate dal gruppo Aperture \gloss{Software} per garantire la qualità del prodotto che verrà sviluppato.
Il presente documento descriverà le qualità desiderate che il software dovrà avere, le metriche utilizzate per rendere il prodotto e i processi quantificabili. Per ottenere obiettivi finali qualitativi è necessario un continuo e costante \gloss{processo} di verifica, per scovare ed eliminare errori in maniera rapida e senza spreco di risorse.

\subsection{Scopo del prodotto}%1.2
\label{1.2}
\Prodotto{}

\subsection{Glossario}%1.3
\label{1.3}
\Glossario{}

\subsection{Riferimenti} %1.4
\label{1.4}
\subsubsection{Normativi}
\label{1.4.1}
\begin{itemize}
\item \grassetto{Norme di Progetto}: \emph{Norme\_di\_progetto\_v\versioneNormeDiProgetto{}.pdf};\\
\item \grassetto{Capitolato d'appalto C1}: MaaP as an admin Platform\\
\url{http://www.math.unipd.it/~tullio/IS-1/2013/Progetto/C1.pdf}.
\end{itemize}
\subsubsection{Informativi}
\label{1.4.2}
\begin{itemize}
\item \grassetto{Piano di Progetto}:  \emph{Piano\_di\_progetto\_v\versionePianoDiProgetto{}.pdf};\\
\item \grassetto{Glossario}: \emph{Glossario\_v\versioneGlossario{}.pdf};\\
\item \grassetto{Slides del corso di Ingegneria del software Modulo A, AA 2013/2014 del prof. Tullio Vardanega}:
\\ \url{http://www.math.unipd.it/~tullio/IS-1/2013/};
\item \grassetto{SWEBOK-Version 3 (2004):} capitolo 11-Software Quality \\ \url{http://www.computer.org/portal/web/swebook/html/ch11};
\item \grassetto{Wikipedia}: \url{http://it.wikipedia.org};
\item \grassetto{Ian Sommerville, Software Engineering, 9 edizione (2011)}:
\begin{itemize}
\item Capitolo 24 - Quality management;
\item Capitolo 26 - Process improvement.
\end{itemize}
\item \grassetto{Standard ISO/IEC TR 15504 Software process assessment}:
\\ \url{http://en.wikipedia.org/wiki/ISO/IEC_15504};
\item \grassetto{Standard ISO/IEC 9126:2001 Software engineering-product quality}:
\\ \url{http://en.wikipedia.org/wiki/ISO_9126};
\item \grassetto{Budget Variance e Schedule Variance - Dati empirici}: \\ \url{http://office.microsoft.com/en-us/project-help/determine-the-right-threshold-for-project-cost-and-schedule-variances-HA010173335.aspx}.
\item \grassetto{Indice Gulpease:}
\begin{itemize}
\item \url{http://it.wikipedia.org/wiki/Indice_Gulpease};
\item \url{http://xoomer.virgilio.it/roberto-ricci/variabilialeatorie/esperimenti/leggibilita.htm}.
\end{itemize}
\item \grassetto{Complessità ciclomatica:}
\begin{itemize}
\item \url{http://it.wikipedia.org/wiki/Complessit%C3%A0_ciclomatica.}
\end{itemize}
\end{itemize}


\newpage
\section{Obiettivi di qualità}%2.0
\label{2.1}
In questa sezione verranno mostrati gli obietti di qualità che il gruppo si è imposto per la realizzazione del prodotto \Progetto{}. \\
Tali obiettivi saranno poi seguiti da:
\begin{itemize}
\item \grassetto{Strategia}: procedimento tramite il quale si raggiunge l'obiettivo prefissato. L'elenco completo delle strategie e la descrizione di ciascuna, sarà possibile consultarlo in sezione 3 di questo documento.
\item \grassetto{Metrica o Misura}: ove possibile, sarà specificata una metrica o una misura per rendere l'obiettivo quantificabile; successivamente in sezione 4 di questo documento sarà fornita una descrizione di ogni metrica e di ogni misura, specificando anche, dove possibile:
\begin{itemize}
\item \grassetto{Obiettivo Sufficiente}: range stabilito come minimo accettabile, sotto il quale ogni unità o documento non verrà accettato come completo;
\item \grassetto{Obiettivo Ottimale}: range consigliato e da usare come riferimento, dal quale è necessario scostarsi il meno possibile.
\end{itemize}
\end{itemize}
\subsection{Qualità di processo} %2.1
\label{2.1.1}
Per garantire la qualità del prodotto finale è necessario migliorare la metodologia che porta alla qualità dei processi che compongono il prodotto. Per fare questo si è deciso di utilizzare lo standard ISO/IEC 15504\footnote{vedi \gloss{Appendice}, sezione A.1} denominato SPICE\footnote{Software Process Improvement and Capability Determination}.
Per applicare il modello appena citato si deve utilizzare il ciclo di Deming\footnote{vedi appendice, sezione A.2} che ha come obiettivo il miglioramento continuo dei processi nel loro \gloss{ciclo di vita}.
\subsubsection{Obiettivo} 
Si è deciso di fissare come obiettivo il raggiungimento del terzo livello dello standard sopracitato, ovvero established. Per cercare di ottenere questo, bisognerà:
\begin{itemize}
\item Pianificare dettagliatamente i processi utilizzando standard di riferimento;
\item Pianificare le risorse da utilizzare.
\end{itemize}
Sono state scelti due indici che si basano sui costi e i tempi spesi per un processo, per renderlo quantificabile. Tali indici sono:
\begin{itemize}
\item \grassetto{Schedule Variance(SV)}: vedi sezione 4.1.1 di questo documento;
\item \grassetto{Budget Variance(BV)}: vedi sezione 4.1.2 di questo documento;
\end{itemize} 
Data la scarsa esperienza di cooperazione e di pianificazione delle attività del gruppo, gli obiettivi di questi indici fanno riferimento a convenzioni comuni.

\subsection{Qualità di prodotto} %2.1
\label{2.1.2}
La qualità di prodotto può essere suddivisa in due categorie. La prima riguarda il software, mentre la seconda riguarda la documentazione. Per ciascuna categoria il gruppo ha individuato degli obiettivi di qualità da raggiungere, illustrati di seguito.
\subsubsection{Obiettivi qualità software}
Per cercare di realizzare e progettare un prodotto software, in accordo con specifiche e standard definiti, ed essere privo di non conformità o difetti, è necessario usare lo standard ISO/IEC 9126\footnote{vedi appendice, sezione A.3}, il quale redige e descrive obiettivi qualitativi e fornisce delle linee guida d'utilizzo di metriche al fine di tracciare il progresso nel miglioramento continuo di processo e prodotto. 
Di seguito verranno elencati gli obiettivi software, basandosi sulle caratteristiche dello standard sopracitato. 
\begin{itemize}
\item \grassetto{Funzionalità}: il prodotto software deve soddisfare tutti i requisiti funzionali obbligatori, tutti i requisiti obbligatori trovati e di conseguenza fornire le funzionalità attese;
\begin{itemize}
\item \grassetto{Strategia adottata}: Prodotto Funzionale(vedi sezione 3.1 di questo documento).
\item \grassetto{Misura o metrica utilizzata}: Soddisfacimento requisiti funzionali obbligatori e requisiti obbligatori(vedi sezione 4.2.1 di questo documento);
\end{itemize}
\item \grassetto{Affidabilità}: il prodotto software dovrà essere in grado di funzionare anche in condizioni non ottimali; ad esempio non devono essere prodotti risultati non corretti e la gestione degli errori deve essere trattata in maniera ottimale; verrà misurata la quantità di esecuzioni andate a buon fine.
\begin{itemize}
\item \grassetto{Strategia adottata}: Prodotto Affidabile(vedi sezione 3.2 di questo documento).
\item \grassetto{Misura o metrica utilizzata}:
\begin{itemize}
\item Complessità Ciclomatica(vedi sezione 4.2.2 di questo documento);
\item Linee di codice per metodo(vedi sezione 4.2.3 di questo documento);
\item Numero di esecuzioni andate a buon fine(vedi sezione 4.2.5 di questo documento).
\end{itemize}
\end{itemize}
\item \grassetto{Usabilità}: essendo legata al soggetto utente, non è facilmente verificabile e misurabile in modo meccanico. Il prodotto dovrà essere comprensibile e facilmente apprendibile;
\begin{itemize}
\item \grassetto{Strategia adottata}: Prodotto Usabile(vedi sezione 3.4 di questo documento).
\item \grassetto{Misura o metrica utilizzata}:
\begin{itemize}
\item Lunghezza manuale utente(vedi sezione 4.2.6 di questo documento);
\end{itemize} 
\end{itemize}
\item \grassetto{Manutenibilità}: deriva da una buona progettazione architetturale, ma soprattutto da un basso accoppiamento delle componenti;
\begin{itemize}
\item \grassetto{Strategia adottata}: Prodotto Manutenibile (vedi sezione 3.5 di questo documento).
\item \grassetto{Misura o metrica utilizzata}:
\begin{itemize}
\item Accoppiamento efferente ed afferente(vedi sezione 4.2.7 e 4.2.8 di questo documento);
\item Complessità ciclomatica(vedi sezione 4.2.2 di questo documento);
\item Lunghezza manuale utente(vedi sezione 4.2.6 di questo documento);
\item Parametri per metodo(vedi sezione 4.2.10 di questo documento);
\item Linee di codice per metodo(vedi sezione 4.2.3 di questo documento).
\end{itemize}
\end{itemize}
\item \grassetto{Portabilità}: il prodotto software deve essere portabile sotto i vincoli imposti nell'Analisi dei Requisiti
\begin{itemize}
\item \grassetto{Strategia adottata}: Prodotto Portabile(vedi sezione 3.6 di questo documento).
\item \grassetto{Misura o metrica utilizzata}:
\begin{itemize}
\item Parametri per metodo(vedi sezione 4.2.10 di questo documento);
\item Linee di codice per metodo(vedi sezione 4.2.3 di questo documento).
\end{itemize}
\end{itemize}
\end{itemize} 

\subsubsection{Obiettivi qualità documentazione}
Il gruppo \NomeGruppo{} si è imposto di produrre documentazione di qualità, nonostante la scarsa inesperienza nel produrre documenti di questo tipo. Proprio per questo motivo è stato introdotto il seguente obiettivo:
\begin{itemize}
\item \grassetto{Leggibilità}: si cercherà di produrre testo caratterizzato da una buona leggibilità.
\begin{itemize}
\item \grassetto{Strategia adottata}: Testo leggibile(vedi sezione 3.6 di questo documento).
\item \grassetto{Misura o metrica utilizzata}: Indice di Gulpease(vedi sezione 4.4.1 di questo documento).
\end{itemize}
\end{itemize} 

\newpage

\section{Strategie}
Verranno di seguito riportate le strategie che il gruppo adotta per il raggiungimento degli obiettivi. Alcune strategie fanno riferimento agli obiettivi riguardanti le metriche individuate, mentre altre strategie fanno riferimento agli obiettivi non quantificabili.
Per quanto riguarda le strategie di verifica della qualità del prodotto, si utilizzerà l'analisi statica e l'analisi dinamica. Entrambe sono descritte in Appendice B, sezione B.5.

\subsection{Prodotto Funzionale}
Per quanto riguarda la funzionalità del prodotto, verrà utilizzata la tabella di tracciamento, presente nella sezione 5 del documento Analisi dei Requisiti, che mostra tutti i requisiti obbligatori funzionali e i requisiti obbligatori. Per il raggiungimento di questo obiettivo il prodotto finale dovrà soddisfare tutti i requisiti mostrati in tabella.

\subsection{Prodotto Affidabile}
Per l'affidabilità del prodotto, si cercherà di scrivere del codice con il minor numero possibile di rami logici. Se questo non risulta possibile, bisogna dividere in più metodi. Prima di effettuare un commit bisogna assicurarsi che quanto scritto fino a quel momento rientri nei range stabiliti delle metriche utilizzate.

\subsection{Prodotto Usabile}
Per quanto riguarda l'usabilità del prodotto, si cercherà di realizzarlo pensando di farlo il più semplice possibile, ponendo maggiore attenzione all'interazione che esso avrà con chi lo usa. Inoltre il manuale utente che verrà rilasciato insieme al prodotto dovrà essere il più semplice possibile e di facile comprensione; esso dovrà avere un buon valore di indice di leggibilità.

\subsection{Prodotto Manutenibile}
Per quanto riguarda la manutenibilità del prodotto, ogni volta che verranno progettate delle componenti si procederà al calcolo dell'accoppiamento afferente ed efferente; dopodiché si procede al calcolo dell'instabilità cercando di rientrare nel valore ottimale corrispondente. Inoltre prima di effettuare un commit, ogni pezzo di codice scritto deve soddisfare le metriche individuate.

\subsection{Prodotto Portabile}
Per la portabilità del prodotto, verrà utilizzata la tabella di tracciamento dei requisiti di vincolo presente nel documento Analisi dei Requisiti. Per assicurarsi maggior portabilità dovranno essere soddisfatte le tipologie di requisiti indicate.

\subsection{Testo leggibile}
Per cercare di ottenere un testo leggibile e che soddisfi l'indice a esso associato, il gruppo si impegna a scrivere testo contenente frasi corte ma significative. Inoltre dovranno essere usate parole di uso comune e non troppo lunghe.

\subsection{Strategie aggiuntive}
Verranno ora elencate le strategie che il gruppo utilizzerà per ottenere più qualità possibile.

\subsubsection{Annidamento accettabile}
Ogni volta che si realizza un metodo, si controlla se il numero di chiamate annidate rientra nei parametri prestabiliti.

\subsubsection{Copertura del codice accettabile}
Ogni volta che si realizza un metodo, esso deve essere il più semplice possibile, in modo tale che non necessiti di testing.

\newpage

\section{Misure e metriche}
In base agli obiettivi di qualità descritti nella sezione 2 di questo documento, il gruppo ha individuato una serie di metriche e misure che permetteranno di rendere gli obiettivi quantificabili e di conseguenza verificabili. Come descritto sempre in tale sezione, ogni metrica e ogni misura sarà accompagnata, dove possibile, da un range di valori. Esse saranno divise per le categorie Processi, Software e Documentazione. 

\subsection{Processi}
\subsubsection{Schedule Variance}
Indica se si è in linea, in anticipo o in ritardo rispetto alla schedulazione delle attività di progetto pianificate nella \gloss{baseline}. È un indicatore di efficacia e se il suo valore è $> 0$ allora il progetto sta avanzando con maggiore velocità rispetto a quanto pianificato. Viceversa se negativo.
Gli obiettivi fissati sono:
\begin{itemize}
\item \grassetto{Obiettivo Sufficiente}: $ [\geq -(costo\:preventivo\:per\:fase * 5\%)]; $
\item \grassetto{Obiettivo Ottimale}: $ [\geq 0]. $
\end{itemize}

\subsubsection{Budget Variance}
Indica se alla data corrente si è speso di più o di meno rispetto a quanto si era pianificato. Se tale valore è $>0$ allora il progetto sta consumando il proprio budget con minor velocità rispetto a quanto pianificato. Viceversa se negativo.
\begin{itemize}
\item \grassetto{Obiettivo Sufficiente}: $[\geq -(costo\:preventivo\:per\:fase * 10\%)];$
\item \grassetto{Obiettivo Ottimale}: $[\geq 0].$
\end{itemize}

\subsection{Software}
\subsubsection{Soddisfacimento requisiti funzionali obbligatori e requisiti obbligatori}
Fornisce un valore che indica quanti requisiti, delle tipologie indicate, sono stati soddisfatti.
Degli obiettivi ragionevoli sono:
\begin{itemize}
\item \grassetto{Obiettivo Sufficiente}:[100\% soddisfacimento requisiti funzionali obbligatori e requisiti obbligatori ]
\item \grassetto{Obiettivo Ottimale}:[100\% soddisfacimento di tutti i requisiti individuati]
\end{itemize}

\subsubsection{Complessità ciclomatica}
\label{4.2.1}
La complessità ciclomatica è una metrica software applicabile singolarmente a funzioni, moduli, metodi e classi di un \gloss{programma}.
Questa metrica è calcolata utilizzando il grafo di controllo di flusso del programma, ovvero i nodi del grafo rappresentano gruppi indivisibili di istruzioni, mentre gli archi connettono due nodi se il secondo gruppo di istruzioni può essere eseguito subito dopo il primo gruppo.
Alti valori di questa metrica implicano una scarsa manutenibilità del software, mentre valori troppo bassi possono indicare un'altrettanta bassa efficienza del software.
Un modulo, con complessità ciclomatica elevata, necessita di più \gloss{testing} rispetto ad un altro modulo con complessità ciclomatica minore.
La complessità è quindi definita come:

$$v(G) = e - n + 2p$$

dove:
\begin{itemize}
\item \grassetto{v(G)} = complessità ciclomatica del grafo G;
\item \grassetto{e} = numero di archi del grafo;
\item \grassetto{n} = numero di nodi del grafo;
\item \grassetto{p} = numero di componenti connesse.
\end{itemize}
Degli obiettivi ragionevoli per questa metrica sono i seguenti:
\begin{itemize}
\item \grassetto{Obiettivo Sufficiente}: [11--15];
\item \grassetto{Obiettivo Ottimale}: [1--10]\footnote{Il valore 10 come massimo è stato calcolato da T.J.McCabe, inventore della metrica.}.
\end{itemize}

\subsubsection{Linee di codice per metodo}
\begin{itemize}
\item \grassetto{Obiettivo Sufficiente}:[11--30]
\item \grassetto{Obiettivo Ottimale}:[10]
\end{itemize}

\subsubsection{Numero di esecuzioni andate a buon fine}
\begin{itemize}
\item \grassetto{Obiettivo Sufficiente}:[98\%]
\item \grassetto{Obiettivo Ottimale}:[100\%]
\end{itemize}

\subsubsection{Lunghezza manuale utente}
Fornisce un valore che indica il numero di pagine che un manuale utente deve avere. I valori per questi obietti verranno forniti nella prossima revisione, in quanto non è stato completato il Manuale Utente per questa revisione.
\begin{itemize}
\item \grassetto{Obiettivo Sufficiente}:[]
\item \grassetto{Obiettivo Ottimale}:[]
\end{itemize}

\subsubsection{Accoppiamento afferente}
Questo valore indica la quantità di classi esterne ad un \gloss{package} che dipendono da classi interne allo stesso.
Un alto valore di accoppiamento in una singola classe del package influisce sull'accoppiamento dell'intero package. Questo fatto non è necessariamente un errore di progettazione, ma il package in esame può rappresentare un punto critico del software. Per contro, un package con basso fattore di accoppiamento può delineare una scarsa utilità del package stesso, che probabilmente andrebbe inglobato con altri package. Esso è legato, insieme all'accoppiamento efferente, ad una metrica importante, ovvero l'instabilità.

\subsubsection{Accoppiamento efferente}
Questo fattore indica l'accoppiamento contrario, ovvero il numero di classi interne al package che dipendono da classi esterne. Più questo indice è basso, più indipendente è il package stesso.

\subsubsection{Instabilità}
\label{4.2.8}
Il fattore di instabilità di un package indica la possibilità di modifica del package senza influire sulla stabilità del software ad esso dipendente.
Questo indice è calcolato con la formula seguente:
$$I = Ce / (Ca + Ce)$$
dove Ca è l'accoppiamento afferente e Ce l'accoppiamento efferente.
Gli obiettivi forniti da \emph{best practice}\footnote{Range ricavati da \url{http://staff.unak.is/andy/StaticAnalysis0809/metrics/i.html}} sono i seguenti:
\begin{itemize}
\item \grassetto{Obiettivo Sufficiente}: [0.4--0.8];
\item \grassetto{Obiettivo Ottimale}: [0.0--0.3].
\end{itemize}

\subsubsection{Parametri per metodo}
Un alto numero di parametri per metodo denota un'eccessiva complessità del metodo stesso, comportandone probabilmente un'ulteriore lunghezza non accettabile. \`{E} buona norma scrivere dei metodi con pochi parametri, al fine di ottenere procedure specifiche e atomiche, di conseguenza facilmente assegnabili e verificabili.
Degli obiettivi validi per questa metrica sono:
\begin{itemize}
\item \grassetto{Obiettivo Sufficiente}: [5--8];
\item \grassetto{Obiettivo Ottimale}: [0--4].
\end{itemize}

\subsection{Metriche aggiuntive}
Per raffinare ulteriormente la qualità, soprattutto per la manutenibilità e l'affidabilità, vengono utilizzate delle ulteriori metriche.

\subsubsection{Livelli di annidamento}
\label{4.2.2}
Il numero di livelli di annidamento dei metodi rappresenta la quantità di richiami di altri metodi all'interno di uno stesso \gloss{metodo}.
Un elevato livello di annidamento definisce un'elevata complessità del codice e di altrettanta comprensione dello stesso.
Gli obiettivi stimati per questa metrica sono:
\begin{itemize}
\item \grassetto{Obiettivo Sufficiente}: [4--6];
\item \grassetto{Obiettivo Ottimale}: [1--3].
\end{itemize}

\subsubsection{Linee di codice per linee di commento}
\label{4.2.5}
Questo numero indica il rapporto tra linee di codice e linee di commento, per avere un fattore di commenti all'interno di un'unità software. In generale, un alto grado di commento del codice porta ad una maggiore manutenibilità ed informazione per uno \gloss{sviluppatore}.
Gli obiettivi stimati sono:
\begin{itemize}
\item \grassetto{Obiettivo Sufficiente}: [$>0.25$];
\item \grassetto{Obiettivo Ottimale}: [0.26--0.30]\footnote{Il valore di 0.30 è stato calcolato dal rapporto 22/78, derivato dalle medie di Ohloh \url{https://www.ohloh.net/p/firefox/factoids\#FactoidCommentsLow}}.
\end{itemize}

\subsubsection{Copertura del codice}
\label{4.2.9}
Questo fattore indica la percentuale di codice coperto durante l'esecuzione dei test. Più alta sarà la percentuale, minore sarà la possibilità di errori riscontrabili nell'esecuzione del software. Per abbassare questo indice sarà sufficiente scrivere metodi semplici che non necessitino di testing. Il valore ideale di 100\% indica che tutte le porzioni di codice sono testate da uno o più test.
Gli obiettivi stimati per questa metrica sono:
\begin{itemize}
\item \grassetto{Obiettivo Sufficiente}: [42\%--65\%];
\item \grassetto{Obiettivo Ottimale}: [66\%--100\%].
\end{itemize}

\subsection{Documentazione}

\subsubsection{Indice Gulpease}
L'indice Gulpease è un indice di leggibilità del testo che basa il suo calcolo su componenti del testo enumerabili meccanicamente, così da rendere automatico il processo di Verifica. Consente di misurare la complessità dello stile di un documento.
L'indice di Gulpease considera due variabili linguistiche: la lunghezza della parola e la lunghezza della frase rispetto al numero di lettere.\\
La formula per il suo calcolo è la seguente:
\\
\begin{center}
\begin{math}
89+\frac{300*(numero\:delle\:frasi)-10*(numero\:delle\:lettere)}{numero\:delle\:parole}
\end{math}
\end{center}
.\\
I risultati sono compresi tra 0 e 100, dove il valore "100" indica la leggibilità più alta e "0" la leggibilità più bassa. In generale risulta che testi con un indice
\begin{itemize}
\item inferiore a 80 sono difficili da leggere per chi ha la licenza elementare;
\item inferiore a 60 sono difficili da leggere per chi ha la licenza media;
\item inferiore a 40 sono difficili da leggere per chi ha un diploma superiore.
\end{itemize}
L'indice prevede un intervallo di valori tra 0 e 100, dove 100 esprime la leggibilità massima.
I nostri obiettivi per l'indice Gulpease sono i seguenti:
\begin{itemize}
\item \grassetto{Obiettivo ottimale}: [51--100];
\item \grassetto{Obiettivo sufficiente}: [40--50].
\end{itemize}

\newpage

\section{Procedure di controllo di qualità}

\subsection{Procedure di controllo di qualità di processo} %2.1
\label{2.2}
Per garantire la qualità dei processi si utilizza il ciclo PDCA\footnote{Alias, Ciclo di Deming, vedi appendice, sezione A.2}.  Questo principio permette un continuo miglioramento della qualità di tutti i processi coinvolti nella realizzazione del prodotto finale.
Per controllare la qualità bisogna che i processi siano pianificati dettagliatamente, che le risorse siano individuate e ripartite in maniera quantificabile e che ci sia un controllo sui processi. Lo sviluppo di quanto scritto prima è descritto dettagliatamente nel \emph{Piano di Progetto \versionePianoDiProgetto{}}.
Inoltre verrà monitorata la qualità dei processi con l'analisi continua della qualità del prodotto.


\subsection{Procedure di controllo di qualità di prodotto} %2.1
\label{2.3}
Per garantire il controllo di qualità si utilizza:
\begin{itemize}
\item \grassetto{Quality Assurance:} tradotta in "assicurazione di qualità", è l'insieme di processi che hanno come fine il miglioramento e il perseguimento della qualità. L'intenzione di un team di lavoro consiste nell'ottenere quella che si dice correction by construction, ovvero "correttezza per costruzione";
\item \grassetto{Strategie proattive:} l'insieme delle strategie proattive, le cui procedure sono descritte nel documento Norme di Progetto, permettono di garantire qualità a tempo zero, limitando \gloss{attività} di verifica che hanno un costo non indifferente.
\item \grassetto{Verifica}: è la valutazione che un prodotto, \gloss{servizio} o sistema, sia conforme a regole, requisiti, specifiche o condizioni imposte. È spesso un processo interno e differisce dalla validazione. In analogia, l'attività di Verifica deve rispondere alla domanda: "did i built the system right?", ovvero "ho costruito il sistema in modo corretto?";
\item \grassetto{Validazione:} è l'assicurazione che un prodotto, servizio o sistema, incontri le necessità che i clienti o gli stakeholder identificano. Spesso comporta l'accettazione e l'idoneità con clienti esterni. In questo caso la domanda è: "did i built the right system?", tradotto in "ho costruito il sistema giusto?".
\end{itemize}

\newpage
\appendix
\section{Standard di qualità} %A.0

\subsection{ISO/IEC 15504} %A.1
ISO/IEC 15504, anche conosciuto come SPICE(Software Process Improvement and Capability Determination, ovvero miglioramento di processi software e determinazione di capacità) è un insieme di documenti di standard tecnici per lo sviluppo software.
Questo documento viene utilizzato nel perseguimento della qualità di processo in quanto stabilisce una struttura per la definizione degli obiettivi per il miglioramento dei processi stessi.
Lo standard dichiara che ogni processo deve essere sottoposto ad un controllo continuo, ripetibile e quantificabile al fine di individuare i punti critici e misurare i miglioramenti.

\immagine{SPY}{Software Process Assessment and Improvement}

Secondo SPICE, un processo può essere classificato in base al suo livello di maturità, in una \gloss{scala} da 1 a 6, con annessi i livelli di capacità ad ogni livello:
\begin{itemize}
\item \grassetto{Incomplete:} I risultati del processo non esistono o non sono appropriati;
\item \grassetto{Performed:} Si ottengono dei risultati, ma in un modo non specificato o non prevedibile;
\begin{itemize}
\item \emph{Process Performance}: capacità del processo di produrre degli output da dagli \gloss{input}.
\end{itemize}
\item  \grassetto{Managed:} l'esecuzione è pianificata e tracciata, il prodotto è conforme a standard e requisiti specifici;
\begin{itemize}
\item \emph{Performance Management:} capacità del processo di produrre un output coerente con gli obiettivi del processo;
\item \emph{Work Product Management:} capacità del processo di creare un risultato documentato, controllato e verificato.
\end{itemize}
\item  \grassetto{Established:} il processo è eseguito e controllato riferendosi a dei buoni principi di ingegneria del software;
\begin{itemize}
\item \emph{Process Definition}: il processo fa riferimento a degli standard di processo per definire i risultati attesi;
\item \emph{Process \gloss{Deployment}}: capacità del processo di utilizzare risorse appropriate per il raggiungimento degli obiettivi.
\end{itemize}
\item  \grassetto{Predictable:}  il processo è eseguito consistentemente con dei limiti di controllo definiti, per raggiungere altrettanto definiti obiettivi di processo;
\begin{itemize}
\item \emph{Process Measurement}: capacità di definizione di obiettivi e metriche di prodotto e di processo, con cui garantire il raggiungimento di obiettivi aziendali;
\item \emph{Process Control}: capacità di controllo tramite metriche di progetto e prodotto definite, per puntare al miglioramento.
\end{itemize}
\item  \grassetto{Optimizing:} l'esecuzione del processo è ottimizzata per soddisfare bisogni correnti e futuri, e il processo soddisfa ripetibilmente i suoi obiettivi prefissati;
\begin{itemize}
\item \emph{Process Innovation}: capacità di gestione di eventuali cambiamenti nel prodotto in modo controllato ed efficace;
\item \emph{Continuous Optimization}: capacità di identificare e applicare modifiche atte al miglioramento dei processi aziendali.
\end{itemize}
\end{itemize}
Per finire, lo standard definisce 4 stadi di misurazione degli attributi di un processo, suddivisi in:
\begin{itemize}
\item \grassetto{N}, non adeguato o non posseduto;
\item \grassetto{P}, parzialmente posseduto;
\item \grassetto{L}, largamente posseduto;
\item \grassetto{F}, completamente posseduto.
\end{itemize}

\subsection{ISO/IEC 9126} %A.2

Con la sigla ISO/IEC 9126 si individua una serie di normative e linee guida, sviluppate dall'\gloss{ISO} (Organizzazione internazionale per la normazione) in collaborazione con l'\gloss{IEC} (Commissione Elettrotecnica Internazionale), preposte a descrivere un modello di qualità del software. Il modello propone un approccio alla qualità in modo tale che le società di software possano migliorare l'organizzazione e i processi e, quindi come conseguenza concreta, la qualità del prodotto sviluppato.

\immagine{ISOIEC}{Modello di qualità ISO/IEC 9126}

Il presente standard definisce 6 caratteristiche di qualità che ogni prodotto software deve perseguire, al fine di garantire la conformità agli standard con efficienza ed efficacia. Le caratteristiche delle qualità in uso esulano dal presente progetto didattico, in quanto non è prevista l'attività di manutenzione conseguente al rilascio del prodotto. Quindi ci soffermiamo all'analisi della qualità interna ed esterna dello standard ISO/IEC 9126.
Le caratteristiche che un prodotto deve avere sono le seguenti:

\begin{itemize}
\item \grassetto{Funzionalità}: capacità di un prodotto software di fornire funzioni che soddisfano esigenze stabilite, necessarie per operare sotto condizioni specifiche;
\begin{itemize}
\item \emph{Appropriatezza}: rappresenta la capacità del prodotto software di fornire un appropriato insieme di funzioni per gli specificati compiti ed obiettivi prefissati all'\gloss{utente};
\item \emph{Accuratezza}: la capacità del prodotto software di fornire i risultati concordati o precisi effetti richiesti;
\item \emph{Interoperabilità}: la capacità del prodotto software di interagire ed operare con uno o più sistemi specificati;
\item \emph{Conformità}: la capacità del prodotto software di aderire agli standard, convenzioni e regolamentazioni rilevanti al settore operativo a cui vengono applicati;
\item \emph{Sicurezza}: la capacità del prodotto software di proteggere informazioni e i dati negando in ogni modo che persone o sistemi non autorizzati possano accedervi o modificarli, e che a persone o sistemi effettivamente autorizzati non sia negato l'accesso ad essi.
\end{itemize}
\item \grassetto{Affidabilità}: capacità del prodotto software di mantenere uno specificato livello di prestazioni quando usato in date condizioni per un dato periodo;
\begin{itemize}
\item \emph{Maturità}: capacità di un prodotto software di evitare che si verificano errori, malfunzionamenti o siano prodotti risultati non corretti;
\item \emph{Tolleranza degli errori}: capacità di mantenere livelli predeterminati di prestazioni anche in presenza di malfunzionamenti o usi scorretti del prodotto;
\item \emph{Recuperabilità}: capacità di un prodotto di ripristinare il livello appropriato di prestazioni e di recupero delle informazioni rilevanti, in seguito a un malfunzionamento. A seguito di un errore, il software può risultare non accessibile per un determinato periodo di tempo, questo arco di tempo è valutato proprio dalla caratteristica di recuperabilità;
\item \emph{Aderenza}: capacità di aderire a standard, regole e convenzioni inerenti all'affidabilità.
\end{itemize}
\item \grassetto{Efficienza}: capacità di fornire appropriate prestazioni relativamente alla quantità di risorse usate;
\begin{itemize}
\item \emph{Comportamento rispetto al tempo}: capacità di fornire adeguati tempi di risposta, elaborazione e velocità di attraversamento, sotto condizioni determinate;
\item \emph{Utilizzo delle risorse}: capacità di utilizzo di quantità e tipo di risorse in maniera adeguata;
\item \emph{Conformità}: capacità di aderire a standard e specifiche sull'efficienza.
\end{itemize}
\item \grassetto{Usabilità}: capacità del prodotto software di essere capito, appreso, usato e benaccetto dall'utente, quando usato sotto condizioni specificate.
\begin{itemize}
\item \emph{Comprensibilità}: esprime la facilità di comprensione dei concetti del prodotto, mettendo in grado l'utente di comprendere se il software è appropriato;
\item \emph{Apprendibilità}: capacità di ridurre l'impegno richiesto agli utenti per imparare ad usare la sua applicazione;
\item \emph{Operabilità}: capacità di mettere in condizione gli utenti di farne uso per i propri scopi e controllarne l'uso;
\item \emph{Attrattiva}: capacità del software di essere piacevole per l'utente che ne fa uso;
\item \emph{Conformità}: capacità del software di aderire a standard o convenzioni relativi all'usabilità.
\end{itemize}
\item \grassetto{Manutenibilità}: capacità del software di essere modificato, includendo correzioni, miglioramenti o adattamenti;
\begin{itemize}
\item \emph{Analizzabilità}: rappresenta la facilità con la quale è possibile analizzare il codice per localizzare un errore nello stesso;
\item \emph{Modificabilità}: capacità del prodotto software di permettere l'implementazione di una specificata modifica (sostituzioni componenti);
\item \emph{Stabilità}: capacità del software di evitare effetti inaspettati derivanti da modifiche errate;
\item \emph{Testabilità}: capacità di essere facilmente testato per validare le modifiche apportate al software.
\end{itemize}
\item \grassetto{Portabilità}: capacità del software di essere trasportato da un ambiente di lavoro ad un altro. (Ambiente che può variare dall'hardware al sistema operativo);
\begin{itemize}
\item \emph{Adattabilità}: capacità del software di essere adattato per differenti ambienti operativi senza dover applicare modifiche diverse da quelle fornite per il software considerato;
\item \emph{Installabilità}: capacità del software di essere installato in uno specificato ambiente;
\item \emph{Conformità}: capacità del prodotto software di aderire a standard e convenzioni relative alla portabilità;
\item \emph{Sostituibilità}: capacità di essere utilizzato al posto di un altro software per svolgere gli stessi compiti nello stesso ambiente.
\end{itemize}
\end{itemize}

\subsubsection{Qualità esterne} %A.2.1

Le metriche esterne, specificate nella norma ISO/IEC 9126-2, misurano i comportamenti del software sulla base dei test, dall'operatività e dall'osservazione durante la sua esecuzione, in funzione degli obiettivi stabiliti in un contesto tecnico rilevante o di \gloss{business}.

\subsubsection{Qualità interne} %A.2.2

La qualità interna, più precisamente le metriche interne, è specificata nella norma ISO/IEC 9126-3 e si applica al software non eseguibile (ad esempio il codice sorgente) durante le fasi di progettazione e codifica. Le misure effettuate permettono di prevedere il livello di qualità esterna ed in uso del prodotto finale, poiché gli attributi interni influiscono su quelli esterni e quelli in uso. Le metriche interne permettono di individuare eventuali problemi che potrebbero influire sulla qualità finale del prodotto prima che sia realizzato il software eseguibile. Esistono metriche che possono simulare il comportamento del prodotto finale tramite simulazioni.

\subsection{Ciclo di Deming (ciclo PDCA)} %A.3

Il ciclo di Deming o Deming Cycle (ciclo di PDCA - plan–do–check–act) è un modello studiato per il miglioramento continuo della qualità in un'ottica a lungo raggio. Serve per promuovere una cultura della qualità che è tesa al miglioramento continuo dei processi e all'utilizzo ottimale delle risorse. Questo strumento parte dall'assunto che per il raggiungimento del massimo della qualità sia necessaria la costante interazione tra ricerca, progettazione, test, produzione e vendita. Per migliorare la qualità e soddisfare il cliente, le quattro fasi devono ruotare costantemente, tenendo come criterio principale la qualità.
La sequenza logica dei quattro punti ripetuti per un miglioramento continuo è la seguente:
\begin{itemize}
\item \grassetto{P} - Plan. Pianificazione.
\item \grassetto{D} - Do. Esecuzione del programma, dapprima in contesti circoscritti.
\item \grassetto{C} - Check. Test e controllo, studio e raccolta dei risultati e dei riscontri.
\item \grassetto{A} - Act. Azione per rendere definitivo e/o migliorare il processo.
\end{itemize}

\immagine{PDCA}{Ciclo PDCA}

\newpage
\section{Pianificazione dei test}
Di seguito verranno visualizzata delle tabelle, strutturate secondo la sezione 5.3.2 delle Norme di Progetto, che riportano tutti i test che si sono pianificati. \\
\subsection{Test di sistema}
Di seguito verrà mostrata una tabella che riporta tutti i test di sistema pianificati, associati ai requisiti descritti nel documento Analisi dei Requisiti.\\
I test sono da intendere solo per requisiti ai quali è stato ragionevole associare un test.
\subsubsection{Descrizione dei test di sistema}
\begin{center}
\begin{longtable}{|c|p{0.5\linewidth}|c|c|}
\toprule
\textbf{Test} & \textbf{Descrizione} & \textbf{Requisito} & \textbf{Stato}\\
\midrule
TS1 & Viene verificato che il sistema MaaP generi correttamente lo scheletro necessario & ROF1 & D.E.\\
\midrule
TS1.1 & Viene verificato che il sistema MaaP installi correttamente le librerie necessarie & ROF1.1 & D.E.\\
\midrule
TS1.2 & Viene verificato che il sistema MaaP generi correttamente i file necessari & ROF1.2 & D.E.\\
\midrule
TS1.3 & Viene verificato che il sistema MaaP generi correttamente le directory necessarie & ROF1.3 & D.E.\\
\midrule
TS1.4 & Viene verificato che il sottosistema di autenticazione sia installato e configurato correttamente & ROF1.4 & D.E.\\
\midrule
TS1.4.1 & Viene verificato che nel database degli utenti sia presente un profilo di amministrazione di default & ROF1.4.1 ROF6 & D.E.\\
\midrule
TS4 & Viene verificato che il sistema crei correttamente le pagine web partendo dal loro file di descrizione & ROF4 & D.E.\\
\midrule
TS5.1 & Viene verificato che la funzione di registrazione possa essere correttamente abilitata/disabilitata & RDF5.1 & D.E.\\
\midrule
TS5.4 & Viene verificato che il sistema possa utilizzare correttamente il database di analisi & ROF5.4 & D.E.\\
\midrule
TS5.5 & Viene verificato che la funzione di creazione indici possa essere correttamnte abilitata/disabilitata & ROF5.5 & D.E.\\
\midrule
TS7 & Viene verificato che il sistema consenta all'utente registrato di potersi autenticare & ROF 7.0 & D.E.\\
\midrule
TS8 & Viene verificato che il sistema consenta all'utente di potersi registrare & RDF 8.0 & D.E.\\
\midrule
TS9 & Viene verificato che il sistema consenta all'utente di recuperare la password  & ROF 9.0 & D.E\\
\midrule
TS10 & Viene verificato che il sistema apra e visualizzi correttamente le Collection e le Collection-Index & ROF10 & D.E.\\
\midrule
TS10.1 & Viene verificato che il sistema visualizzi correttamente le pagine di Document-Show & ROF10.1 & D.E.\\
\midrule
TS10.2.4 & Viene verificato che il sistema disconnetta correttamente un utente alla sua richiesta & ROF10.2.4 & D.E.\\
\midrule
TS10.3.1.1 & Viene verificato che un utente autenticato possa modificare i dati del suo profilo & ROF10.3.1.1 & D.E.\\
\midrule
TS10.3.1.2 & Viene verificato che le modifiche apportate al profilo di un utente business autenticaro siano consistenti & ROF10.3.1.2 & D.E.\\
\midrule
TS10.3.2 & Viene verificato che la creazione di un nuovo utente da parte di un utente business autenticato amministratore avvenga correttamente & ROF10.3.2 & D.E.\\
\midrule
TS10.3.3 & Viene verificata la corretta cancellazione di un utente da parte di un utente business autenticato amministratore & ROF10.3.3 & D.E.\\
\midrule
TS10.4 & Viene verificato che l'utente business autenticato amministratore possa eliminare correttamente un Document & ROF10.4 & D.E.\\
\midrule
TS10.5 & Viene verificato che l’utente business autenticato amministratore possa modificare correttamente un Document & ROF10.5 & D.E.\\
\midrule
TS10.6 & Viene verificata la corretta visualizzazione delle query più utilizzate & ROF10.6 & D.E.\\
\midrule
TS10.7 & Viene verificata la corretta creazione degli indici di analisi & ROF10.7 & D.E.\\
\midrule
TS17 & Viene verificato che le pagine web prodotte dal framework MaaP siano compatibili con la versione 30.0.x o superiore di Google Chrome & ROV17 & D.E.\\
\midrule
TS18 & Viene verificato che le pagine web prodotte dal framework MaaP siano compatibili con la versione 24.x o superiore di Firefox & ROV18 & D.E.\\
\midrule
TS19 & Viene verificato che il sistema accetti solo file di configurazione validi & ROV19 & D.E.\\
\midrule
TS26 & Viene verificato che il sistema di installazione del software funzioni correttamente & ROV26 & D.E.\\
\midrule
TS27 & Viene verificato che il deployment su Heroku avvenga con successo & ROV27 & D.E.\\
%inserire i test
\bottomrule
\caption{Tabella per test di sistema}
\label{tab:changelog}
\end{longtable}
\end{center}
\subsection{Test d'integrazione}
Di seguito verrà mostrata una tabella che riporta tutti i test d'integrazione pianificati, associati alle componenti descritte nella progettazione ad alto livello.\\
\subsubsection{Descrizione dei test d'integrazione}
\begin{center}
\begin{longtable}{|c|p{0.5\linewidth}|c|c|}
\toprule
\textbf{Test} & \textbf{Descrizione} & \textbf{Componente} & \textbf{Stato}\\
\midrule
TI.MaaP & Test integrazione finale client-server & MaaP & D.E\\
\midrule
TI.Server & Test integrazione finale server & Server & D.E\\
\midrule
TI.Model & Test di funzionalità recupero e salvataggio dati & Server.Model & D.E\\
\midrule
TI.DSL & Verifica che le operazioni di elaborazione del DSL vengano eseguite correttamente /* interpretazione, caricamento roots, collections n shit*/ & DSL & D.E\\
\midrule
TI.Databases & Verifica che tutte le operazioni di recupero e scrittura dati e di creazione indici avvengano correttamente & Databases & D.E\\
\midrule
TI.DBAnalysis & Verifica che le operazioni di recupero e scrittura dati sul database di analisi avvengano correttamente & DBAnalysis & D.E\\
\midrule
TI.DBUser & Verifica che le operazioni di recupero e scrittura dati sul database utenti avvengano correttamente & DBUser & D.E\\
\midrule
TI.Client & Test integrazione finale client & Client & D.E\\
%inserire i test
\bottomrule
\caption{Tabella per test d'integrazione}
\label{tab:changelog}
\end{longtable}
\end{center}
\subsection{Tracciamento}
Di seguito verranno riportati in forma tabellare, descritta nella sezione 5.3.2 delle Norme di Progetto, i tracciamenti componente-test d'integrazione e test d'integrazione-componente
\subsubsection{Tracciamento componente-test d'integrazione}
\begin{center}
\begin{longtable}{|c|c|}
\toprule
\textbf{Componente} & \textbf{Test}\\
%inserire i test
\bottomrule
\caption{Tabella tracciamento componente-test d'integrazione}
\label{tab:changelog}
\end{longtable}
\end{center}
\subsubsection{Tracciamento test d'integrazione-componente}
\begin{center}
\begin{longtable}{|c|c|}
\toprule
\textbf{Test} & \textbf{Componente}\\
%inserire i test
\bottomrule
\caption{Tabella tracciamento test d'integrazione-componenti}
\label{tab:changelog}
\end{longtable}
\end{center}


\newpage
\section{Resoconto delle attività di verifica}
\subsection{Tracciamento componenti requisiti}
\subsection{Riassunto delle attività di verifica}
In questa sezione sono descritti i resoconti delle attività di verifica effettuate sui documenti prima di ciascuna revisione.
\subsubsection{Revisione dei Requisiti}
Nel periodo precedente a questa revisione i documenti sono stati controllati dai verificatori seguendo le Norme di Progetto nelle sezione 6.4.1 e 6.4.2; è stata applicata l'analisi statica descritta nella sezione 2.8.1 di questo documento.
Inizialmente è stata applicata la tecnica di Walkthrough, dove sono scovati e successivamente corretti gli errori; ogni volta che si trovava un errore, esso veniva messo nell'apposta lista che serve per l'inspection.
Dopo il walkthrough è stata applicata la tecnica di Inspection, utilizzando l'apposita lista, disponible in appendice delle Norme di Progetto. Inoltre per questo documento sono state calcolate le metriche descritte nella sezione 2.9.2 del documento corrente.
Per quanto riguarda i processi, essi sono stati controllati e verificati secondo le metodologie descritte nelle Norme di Progetto in sezione ???. Sono state calcolate le metriche per i processi descritti in sezione 2.9.1 di questo documento, e riportati i corrispondenti valori di BV e SV in forma tabellare.
\subsubsection{Revisione di Progettazione}
Nel periodo precedente a questa revisione i documenti sono stati controllati dai verificatori seguendo le Norme di Progetto nelle sezione 6.4.1 e 6.4.2; è stata applicata l'analisi statica descritta nella sezione 2.8.1 di questo documento.
Inizialmente è stata applicata la tecnica di Walkthrough, dove sono scovati e successivamente corretti gli errori; ogni volta che si trovava un errore, esso veniva messo nell'apposta lista che serve per l'inspection.
Dopo il walkthrough è stata applicata la tecnica di Inspection, utilizzando l'apposita lista, disponible in appendice delle Norme di Progetto; è stata posta particolare attenzione al documento Specifica Tecnica. Inoltre per questo documento sono state calcolate le metriche descritte nella sezione 2.9.2 del documento corrente.
Per quanto riguarda i processi, essi sono stati controllati e verificati secondo le metodologie descritte nelle Norme di Progetto in sezione ???. Sono state calcolate le metriche per i processi descritti in sezione 2.9.1 di questo documento, e riportati i corrispondenti valori di BV e SV in forma tabellare.
\subsection{Dettaglio delle verifiche tramite analisi}
\subsubsection{Analisi dei Requisiti}
\paragraph{Processi}
Di seguito vengono riportati i valori degli indici SV e BV calcolati durante il periodo di tempo dedicato all'Analisi dei Requisiti.
\begin{longtable}{|c|p{3cm}|p{3cm}|}
\toprule
\textbf{Attività} & \textbf{SV} & \textbf{BV} \\

%aggiungere qui una midrule per aggiungere una nuova riga alla tabella

\midrule
\emph{Studio Fattibilità} & 0 & 0 \\
\midrule
\emph{Analisi dei Requisiti} & +50 & +50\\
\midrule
\emph{Glossario} & 0  & 0\\
\midrule
\emph{Norme di Progetto} & 0 & 0\\
\midrule
\emph{Piano di Progetto} & 0 & \\
\midrule
\emph{Piano di Qualifica} & -15 & -15\\
\bottomrule
\caption{Esiti dell'indice di Gulpease calcolato sui documenti durante l'Analisi}
\label{tab:changelog}
\end{longtable}
In questa tabella, i valori positivi indicano un costo eccedente, viceversa i valori negativi mostrano un costo risparmiato.
I valori indicati in tabella sono espressi in euro.
Non avendo previsto degli intervalli di tempo libero tra un'attività e la successiva, abbiamo ottenuto degli SV positivi in Analisi dei Requisiti e negativi in Piano di Qualifica.
Questa è stata una mancanza da parte del team, che vedrà di migliorarsi nelle prossime fasi e di adottare una tattica di pianificazione più flessibile.
I costi aggiuntivi sono comunque in linea con i nostri obiettivi.
\paragraph{Documenti}
Di seguito vengono riportati, per ogni documento, i valori dell'indice di Gulpease calcolati durante il periodo di tempo dedicato all'Analisi dei Requisiti. Un documento è valido solo se rispecchia i range in sezione 2.9.2.1.
\begin{longtable}{|c|p{3cm}|p{3cm}|}
\toprule
\textbf{Documento} & \textbf{Valore indice} & \textbf{Esito} \\

%aggiungere qui una midrule per aggiungere una nuova riga alla tabella

\midrule
\emph{Studio Fattibilità v1.2.0} & 46 & Sufficiente\\
\midrule
\emph{Analisi dei Requisiti v1.2.0} & 52& Superato\\
\midrule
\emph{Glossario v1.2.0} & 46 & Sufficiente\\
\midrule
\emph{Norme di Progetto v1.2.0} & 52 & Superato\\
\midrule
\emph{Piano di Progetto v1.2.0} & 50 & Superato\\
\midrule
\emph{Piano di Qualifica v1.2.0} & 47 & Sufficiente\\
\bottomrule
\caption{Esiti dell'indice di Gulpease calcolato sui documenti durante l'Analisi}
\label{tab:changelog}
\end{longtable}
\subsubsection{Analisi in Dettaglio}
\paragraph{Processi}
Di seguito vengono riportati i valori degli indici SV e BV calcolati durante il periodo di tempo dedicato all'Analisi in Dettaglio.
\paragraph{Documenti}
Di seguito vengono riportati, per ogni documeno, i valori dell'indice di Gulpease calcolati durante il periodo di tempo dedicato all'Analisi in Dettaglio.

\begin{longtable}{|c|p{3cm}|p{3cm}|}
\toprule
\textbf{Documento} & \textbf{Valore indice} & \textbf{Esito} \\

%aggiungere qui una midrule per aggiungere una nuova riga alla tabella

\midrule
\emph{Studio Fattibilità v1.2.0} &  & \\
\midrule
\emph{Analisi dei Requisiti v1.2.0} & & \\
\midrule
\emph{Glossario v1.2.0} &  &\\
\midrule
\emph{Norme di Progetto v1.2.0} &  & \\
\midrule
\emph{Piano di Progetto v1.2.0} &  & \\
\midrule
\emph{Piano di Qualifica v1.2.0} &  & \\
\bottomrule
\caption{Esiti dell'indice di Gulpease calcolato sui documenti durante l'Analisi}
\label{tab:changelog}
\end{longtable}

\subsection{Dettaglio dell'esito delle revisioni}
Per ciascuna revisione alla quale si intende partecipare, il committente avrà il compito di segnalare eventuali problematiche trovate, dando una valutazione globale dell'andamento del progetto e una descrizione per ciascun documento con correzioni e accorgimenti da apportare.
Di seguito vengono elencate le modifiche apportate ai documenti, come suggerito dal committente, per ciascuna revisione.
\subsubsection{Revisione dei Requisiti}
\begin{itemize}
\item \grassetto{Studio di Fattibilità:} il documento è stato valutato bene, quindi non ci sono stati accorgimenti da apportate;
\item \grassetto{Norme di Progetto:}
\item \grassetto{Analisi dei Requisiti:}
\item \grassetto{Piano di Progetto:}
\item \grassetto{Piano di Qualifica:}
\item \grassetto{Glossario:}
\end{itemize}







\newpage
\section{Gestione amministrativa della revisione}
Di seguito verrà descritto come avviene, all'interno del gruppo, la comunicazione per la gestione di anomalie e per il trattamento delle discrepanze.
\label{3.0}
\subsection{Comunicazione e risoluzione di anomalie}
Con anomalia si intende un esito diverso del prodotto rispetto alle aspettative, una violazione delle norme tipografiche di un documento, un valore di qualche indice non valido, ovvero fuori dal range di accettazione.
Se un verificatore scova un'anomalia, di conseguenza aprirà un \gloss{ticket} su RedMine, strumento descritto nelle \emph{Norme\_di\_progetto\_v\versioneNormeDiProgetto{}.pdf}, in appendice A.1.
\label{3.1}
\subsection{Trattamento delle discrepanze}
\label{3.2}
La discrepanza indica una mancata corrispondenza tra il prodotto atteso e il prodotto finito. Essa non ostruisce il funzionamento del software, ma è inesatto rispetto ai requisiti descritti. Per la gestione delle discrepanze si procede nella stessa maniera vista per la gestione delle anomalie.
\newpage

\section{Resoconto delle attività di verifica}

\subsection{Resoconto PDCA}

Durante lo sviluppo del progetto si è sempre applicato il ciclo di Deming per cercare di migliorare la qualità dei processi. Questa attività ha portato a molteplici miglioramenti.
Tra i più significativi si riportano:
\begin{itemize}


\item Migliore politica di assegnazione delle attività:

All'inizio della progettazione, le attività principali e maggiormente propedeutiche erano state assegnate a gruppi di quattro persone, con l'obiettivo di colmare velocemente il monte ore/persona e ridurre i tempi di completamento. Tuttavia si è notato come il lavoro procedesse più lentamente di quanto atteso e fossero presenti dei ritardi rispetto alle consegne. Nelle seguenti pianificazioni si è ridotto il numero di persone assegnate ad una singola attività cercando nel contempo di massimizzare il parallelismo. Nonostante questo abbia leggermente aumentato il tempo di completamento delle singole attività, i benefici portati dal lavoro concorrente di team più piccoli hanno largamente sorpassato gli svantaggi e comportato un maggior avanzamento complessivo del progetto.

\item Migliore sviluppo dei documenti:

All'inizio i documenti erano sviluppati utilizzando \gloss{editor} di testo come OpenOffice e in seguito, una volta completati, trasformati in \LaTeX\ con una formattazione appropriata.
In seguito si è visto come lo sviluppo dei documenti direttamente in \LaTeX, nonostante sia più oneroso in termini di tempo durante la stesura, sia comunque più rapido rispetto al primo metodo e, soprattutto, privo degli errori di trasposizione, riducendo quindi le ore/persona necessarie e gli errori contenuti nei documenti.

\end{itemize}


\subsection{Riassunto delle attività di verifica}
In questa sezione sono descritti i resoconti delle attività di verifica effettuate sui documenti prima di ciascuna revisione.

\subsubsection{Revisione dei Requisiti}
Nel periodo precedente a questa revisione i documenti sono stati controllati dai Verificatori seguendo le \emph{Norme\_di\_progetto\_v\versioneNormeDiProgetto{}.pdf}; è stata applicata l'analisi statica descritta nella sezione 2.8.1 di questo documento.
Inizialmente è stata applicata la tecnica di Walkthrough, dove sono scovati e successivamente corretti gli errori; ogni volta che si trovava un errore, esso veniva messo nell'apposta lista che serve per l'Inspection.
Dopo il Walkthrough è stata applicata la tecnica di Inspection, utilizzando l'apposita lista, disponibile in appendice delle Norme di Progetto. Inoltre per questo documento sono state calcolate le metriche descritte nella sezione 2.9.2 del documento corrente.
Per quanto riguarda i processi, essi sono stati controllati e verificati secondo le metodologie descritte nelle  \emph{Norme\_di\_progetto\_v\versioneNormeDiProgetto{}.pdf}. Sono state calcolate le metriche per i processi descritti in sezione 2.9.1 di questo documento, e riportati i corrispondenti valori di BV e SV in forma tabellare.

\subsubsection{Revisione di Progettazione}
Nel periodo precedente a questa revisione i documenti sono stati controllati dai verificatori seguendo le \emph{Norme\_di\_progetto\_v\versioneNormeDiProgetto{}.pdf}; è stata applicata l'analisi statica descritta nella sezione 2.8.1 di questo documento.
Inizialmente è stata applicata la tecnica di Walkthrough, dove sono scovati e successivamente corretti gli errori; ogni volta che si trovava un errore, esso veniva messo nell'apposta lista che serve per l'Inspection.
Dopo il Walkthrough è stata applicata la tecnica di Inspection, utilizzando l'apposita lista, disponibile in appendice delle Norme di Progetto; è stata posta particolare attenzione al documento Specifica Tecnica. Inoltre per questo documento sono state calcolate le metriche descritte nella sezione 2.9.2 del documento corrente.
Per quanto riguarda i processi, essi sono stati controllati e verificati secondo le metodologie descritte nelle \emph{Norme\_di\_progetto\_v\versioneNormeDiProgetto{}.pdf}. Sono state calcolate le metriche per i processi descritti in sezione 2.9.1 di questo documento, e riportati i corrispondenti valori di BV e SV in forma tabellare.
\subsubsection{Revisione di Qualifica}
Nel periodo precedente a questa revisione i documenti sono stati controllati dai verificatori seguendo le \emph{Norme\_di\_progetto\_v\versioneNormeDiProgetto{}.pdf}; è stata applicata l'analisi statica descritta nell'appendice B sezione 5.1 di questo documento. Inizialmente è stata applicata la tecnica di Walkthrough, dove sono scovati e successivamente corretti gli errori; ogni volta che si trovava un errore, esso veniva messo nell'apposta lista che serve per l'Inspection. Dopo il Walkthrough è stata applicata la tecnica di Inspection, utilizzando l'apposita lista, disponibile in appendice delle Norme di Progetto. E' stata posta particolare attenzione al documento Specifica Tecnica. Inoltre per questo documento sono state calcolate le metriche descritte nelle sezioni 4.2.6, 4.2.7, 4.2.8 del documento corrente. Per quanto riguarda i processi, essi sono stati controllati e verificati secondo le metodologie descritte nelle\emph{Norme\_di\_progetto\_v\versioneNormeDiProgetto{}.pdf}. Sono state calcolate le metriche per i processi descritti in
sezione 4.1 di questo documento, e riportati i corrispondenti valori di BV e SV in forma tabellare.
\subsubsection{Revisione di Accettazione}
Nel periodo precedente a questa revisione i documenti sono stati controllati dai verificatori seguendo le \emph{Norme\_di\_progetto\_v\versioneNormeDiProgetto{}.pdf}; è stata applicata l'analisi statica descritta nell'appendice B sezione 5.1 di questo documento. Inizialmente è stata applicata la tecnica di Walkthrough, dove sono scovati e successivamente corretti gli errori; ogni volta che si trovava un errore, esso veniva messo nell'apposta lista che serve per l'Inspection. Dopo il Walkthrough è stata applicata la tecnica di Inspection, utilizzando l'apposita lista, disponibile in appendice delle Norme di Progetto. E' stata posta particolare attenzione ai documenti Definizione di Prodotto e Manuale Utente. Per quanto riguarda i processi, essi sono stati controllati e verificati secondo le metodologie descritte nelle\emph{Norme\_di\_progetto\_v\versioneNormeDiProgetto{}.pdf}. Sono state calcolate le metriche per i processi descritti in sezione 4.1 di questo documento, e riportati i corrispondenti valori di BV e SV in forma tabellare.

\subsection{Dettaglio delle verifiche tramite analisi}
\subsubsection{Analisi dei Requisiti}
\paragraph{Processi}
Di seguito vengono riportati i valori degli indici SV e BV calcolati durante il periodo di tempo dedicato all'Analisi dei Requisiti.
\begin{longtable}{|c|p{3cm}|p{3cm}|}
\toprule
\textbf{Attività} & \textbf{SV} & \textbf{BV} \\

%aggiungere qui una midrule per aggiungere una nuova riga alla tabella

\midrule
\emph{Studio Fattibilità} & 0 & 0 \\
\midrule
\emph{Analisi dei Requisiti} & +50 & +50\\
\midrule
\emph{Glossario} & 0  & 0\\
\midrule
\emph{Norme di Progetto} & 0 & 0\\
\midrule
\emph{Piano di Progetto} & 0 & 0\\
\midrule
\emph{Piano di Qualifica} & -15 & -15\\
\bottomrule
\caption{BV e SV calcolati sui documenti durante l'Analisi}
\label{tab:changelog}
\end{longtable}

\paragraph{Conclusioni}
In questa tabella, i valori positivi indicano un costo risparmiato, viceversa i valori negativi mostrano un costo eccedente.
I valori indicati in tabella sono espressi in euro.
Non avendo previsto degli intervalli di tempo libero tra un'attività e la successiva, abbiamo ottenuto degli SV positivi in Analisi dei Requisiti e negativi in Piano di Qualifica.
Questa è stata una mancanza da parte del team, che vedrà di migliorarsi nelle prossime fasi e di adottare una tattica di pianificazione più flessibile.
I costi aggiuntivi sono comunque in linea con i nostri obiettivi.

\paragraph{Documenti}
Di seguito vengono riportati, per ogni documento, i valori dell'indice di Gulpease calcolati durante il periodo di tempo dedicato all'Analisi dei Requisiti. Un documento è valido solo se rispecchia i range in sezione 2.9.2.1.
\begin{longtable}{|c|p{3cm}|p{3cm}|}
\toprule
\textbf{Documento} & \textbf{Valore indice} & \textbf{Esito} \\

%aggiungere qui una midrule per aggiungere una nuova riga alla tabella

\midrule
\emph{Studio Fattibilità v1.2.0} & 46 & Sufficiente\\
\midrule
\emph{Analisi dei Requisiti v1.2.0} & 52 & Superato\\
\midrule
\emph{Glossario v1.2.0} & 46 & Sufficiente\\
\midrule
\emph{Norme di Progetto v1.2.0} & 52 & Superato\\
\midrule
\emph{Piano di Progetto v1.2.0} & 50 & Superato\\
\midrule
\emph{Piano di Qualifica v1.2.0} & 47 & Sufficiente\\
\bottomrule
\caption{Esiti dell'indice di Gulpease calcolato sui documenti durante l'Analisi}
\label{tab:changelog}
\end{longtable}

\subsubsection{Analisi in Dettaglio}
\paragraph{Processi}
Di seguito vengono riportati i valori degli indici SV e BV calcolati durante il periodo di tempo dedicato all'Analisi in Dettaglio.
\begin{longtable}{|c|p{3cm}|p{3cm}|}
\toprule
\textbf{Attività} & \textbf{SV} & \textbf{BV} \\

%aggiungere qui una midrule per aggiungere una nuova riga alla tabella

\midrule
\emph{Studio Fattibilità} & 0 & 0 \\
\midrule
\emph{Analisi dei Requisiti} & 0 & -50\\
\midrule
\emph{Glossario} & 0  & 0\\
\midrule
\emph{Norme di Progetto} & 0 & 0\\
\midrule
\emph{Piano di Progetto} & 0 & 0\\
\midrule
\emph{Piano di Qualifica} & 0 & 0\\
\bottomrule
\caption{BV e SV calcolati sui documenti durante l'Analisi in Dettaglio}
\label{tab:changelog}
\end{longtable}

\paragraph{Conclusioni}
Come si può notare dalla tabella, il BV è negativo, in quanto non sono state pianificate alcune attività correttive, ed è stato messo a budget il costo necessario per effettuare queste attività non previste.\\
Lo SV invece è pari a zero, in quanto l'ampio slack di tempo pianificato è servito a coprire le correzioni non previste e di conseguenza non è stato prodotto niente di più rispetto a quanto pianificato.
\paragraph{Documenti}
Di seguito vengono riportati, per ogni documento, i valori dell'indice di Gulpease calcolati durante il periodo di tempo dedicato all'Analisi in Dettaglio.

\begin{longtable}{|c|p{3cm}|p{3cm}|}
\toprule
\textbf{Documento} & \textbf{Valore indice} & \textbf{Esito} \\

%aggiungere qui una midrule per aggiungere una nuova riga alla tabella

\midrule
\emph{Analisi dei Requisiti v2.2.0} & 55 & Superato \\
\midrule
\emph{Glossario v2.2.0} & 56 & Sufficiente\\
\midrule
\emph{Norme di Progetto v2.2.0} & 52 & Superato\\
\midrule
\emph{Piano di Progetto v2.2.0} & 48 & Sufficiente \\
\midrule
\emph{Piano di Qualifica v2.2.0} & 47 & Sufficiente \\
\bottomrule
\caption{Esiti dell'indice di Gulpease calcolato sui documenti durante l'Analisi in Dettaglio}
\label{tab:changelog}
\end{longtable}

\subsubsection{Progettazione Architetturale}
\paragraph{Processi}
Di seguito vengono riportati i valori degli indici SV e BV calcolati durante il periodo di tempo dedicato alla Progettazione Architetturale.
\begin{longtable}{|c|p{3cm}|p{3cm}|}
\toprule
\textbf{Attività} & \textbf{SV} & \textbf{BV} \\

%aggiungere qui una midrule per aggiungere una nuova riga alla tabella

\midrule
\emph{Analisi dei Requisiti} & 0 & +35\\
\midrule
\emph{Glossario} & 0  & +45\\
\midrule
\emph{Norme di Progetto} & 20 & +45\\
\midrule
\emph{Piano di Progetto} & 0 & +45 \\
\midrule
\emph{Piano di Qualifica} & 0 & -20\\
\midrule
\emph{Specifica Tecnica} & 0 & -87\\
\bottomrule
\caption{BV e SV calcolati sui documenti durante la Progettazione Architetturale}
\label{tab:changelog}
\end{longtable}

\paragraph{Conclusioni}
Lo SV è positivo, in quanto lo slack dedicato al documento Norme di Progetto ha \gloss{permesso} l'aggiunta di valore non pianificato, come l'aggiunta di sezioni.\\
Il BV è positivo, e nonostante il fatto che si è dedicato più tempo alla progettazione, e quindi dedicandoci più budget; a causa di questo si è riuscito a risparmiare budget per le attività dedicate agli altri documenti, dedicando maggior budget per la Verifica della progettazione, che nella pianificazione non era adeguato, e togliendone da altre attività.

\paragraph{Documenti}
Di seguito vengono riportati, per ogni documento, i valori dell'indice di Gulpease calcolati durante il periodo di tempo dedicato alla Progettazione Architetturale.

\begin{longtable}{|c|p{3cm}|p{3cm}|}
\toprule
\textbf{Documento} & \textbf{Valore indice} & \textbf{Esito} \\

%aggiungere qui una midrule per aggiungere una nuova riga alla tabella

\midrule
\emph{Analisi dei Requisiti v3.2.0} & 58 & Superato \\
\midrule
\emph{Glossario v3.2.0} & 56 & Superato \\
\midrule
\emph{Norme di Progetto v3.2.0} & 53  & Superato\\
\midrule
\emph{Piano di Progetto v3.2.0} & 49  & Sufficiente\\
\midrule
\emph{Piano di Qualifica v3.2.0} & 48  & Sufficiente\\
\midrule
\emph{Specifica Tecnica v1.2.0} & 44 & Sufficiente\\
\bottomrule
\caption{Esiti dell'indice di Gulpease calcolato sui documenti durante la Progettazione}
\label{tab:changelog}
\end{longtable}

\paragraph{Progettazione}
Viene qui riportata una tabella riassuntiva che riporta il calcolo dei parametri di accoppiamento afferente ed efferente per i componenti individuati nella progettazione architetturale.

\begin{longtable}{|p{11cm}|c|c|c|}
\toprule
\textbf{Componente} & \textbf{Afferente} & \textbf{Efferente} & \textbf{Instabilità} \\

\midrule
MaaP::Server
& 1 & 0 & 0\\

\midrule
MaaP::Server::ModelServer
& 3 & 0 & 0\\

\midrule
MaaP::Server::ModelServer::DataManager
& 1 & 4 & 0.8\\

\midrule
MaaP::Server::ModelServer::DataManager::DatabaseAnalysisManager
& 1 & 7 & 0.8\\

\midrule
MaaP::Server::ModelServer::DataManager::DatabaseUserManager
& 1 & 4 & 0.8\\


\midrule
MaaP::Server::ModelServer::DataManager::IndexManager
& 1 & 2 & 0.6\\


\midrule
MaaP::Server::ModelServer::Database
& 4 & 0 & 0\\


\midrule
MaaP::Server::ModelServer::DSL
& 1 & 0 & 0\\

\midrule
MaaP::Server::Controller
& 1 & 4 & 0.8\\

\midrule
MaaP::Client
& 0 & 1 & 1\\

\midrule
MaaP::Client::View
& 0 & 14 & 1\\

\midrule
MaaP::Client::ControllerModelView
& 14 & 22 & 0.6\\

\midrule
MaaP::Client::ControllerModelView::ControllerClient
& 14 & 22 & 0.6\\

\midrule
MaaP::Client::ModelClient
& 22 & 1 & 0\\

\midrule
MaaP::Client::ModelClient::Services
& 20 & 1 & 0\\

\midrule
MaaP::Client::ModelClient::Directives
& 2 & 0 & 0\\

\midrule
MaaP::Client::ModelClient::Model
& 0 & 0 & 0\\


\bottomrule
\caption{Tabella accoppiamento componenti}
\end{longtable}

Come si può vedere dalla tabella, l'accoppiamento afferente risulta generalmente basso ad eccezione del componente ControllerModelView del package Client e relativo ControllerClient i quali hanno un valore relativamente alto. Questo delinea la criticità del componente in oggetto, che quindi andrà trattato con dovute cautele durante la generazione dei test e la loro esecuzione per ottenere un componente stabile più velocemente, prevenendo il rischio di regressione dovuto ad un alto accoppiamento.\\
Per quanto riguarda l'accoppiamento efferente, anch'esso è relativamente basso ad eccezione dei componenti interni del package MaaP::Server::ModelServer::DataManager e del componente ControllerClient che per la loro natura intrinseca hanno un alto livello di accoppiamento dovendo interagire con diverse classi di package esterni.

\subsubsection{Progettazione di dettaglio e codifica}
\paragraph{Processi}
Di seguito vengono riportati i valori degli indici SV e BV calcolati durante il periodo di tempo dedicato alla Progettazione di dettaglio e codifica.
\begin{longtable}{|c|p{3cm}|p{3cm}|}
\toprule
\textbf{Attività} & \textbf{SV} & \textbf{BV} \\

%aggiungere qui una midrule per aggiungere una nuova riga alla tabella

\midrule
\emph{Analisi dei Requisiti} & 0 & +25 \\
\midrule
\emph{Glossario} & 0 & -15  \\
\midrule
\emph{Norme di Progetto} & 0 & -30 \\
\midrule
\emph{Piano di Progetto} & 0 & +90 \\
\midrule
\emph{Piano di Qualifica} & 0 & +51\\
\midrule
\emph{Specifica Tecnica} & 0 & -1\\
\midrule
\emph{Definizione di Prodotto} & 0 & -241 \\
\midrule
\emph{Manuale utente} & -105 & +105 \\
\midrule
\emph{Codifica} & 0 & +480 \\
\midrule
\emph{Verifica della Codifica} & 0 & +150 \\
\bottomrule
\caption{BV e SV calcolati sui documenti durante la Progettazione di dettaglio e codifica}
\label{tab:changelog}
\end{longtable}

\paragraph{Conclusioni}
Lo SV è negativo in quanto non è stato completato il Manuale Utente come pianificato.
Il BV è invece positivo a causa del fatto che sono state richieste molte meno ore di Codifica per realizzare quanto pianificato, e di conseguenza sono state tolte ore di Verifica del codice per il motivo precedente.
Inoltre il fatto di non aver completato il Manuale Utente alza il BV, a discapito dello SV; il BV della Definizione di Prodotto è negativo perchè sono state aggiunte ore di Verifica non pianificate; nonostante questo, il BV complessivo è positivo.


\paragraph{Documenti}
Di seguito vengono riportati, per ogni documento, i valori dell'indice di Gulpease calcolati durante il periodo di tempo dedicato alla Progettazione di dettaglio e codifica.

\begin{longtable}{|c|p{3cm}|p{3cm}|}
\toprule
\textbf{Documento} & \textbf{Valore indice} & \textbf{Esito} \\

%aggiungere qui una midrule per aggiungere una nuova riga alla tabella

\midrule
\emph{Analisi dei Requisiti v4.2.0} & 58 & Superato \\
\midrule
\emph{Glossario v4.2.0} & 57 & Superato \\
\midrule
\emph{Norme di Progetto v4.2.0} & 56 & Superato \\
\midrule
\emph{Piano di Progetto v4.2.0} & 51 & Superato\\
\midrule
\emph{Piano di Qualifica v4.2.0} & 45 & Sufficiente \\
\midrule
\emph{Specifica Tecnica v2.2.0} & 52 & Superato\\
\midrule
\emph{Definizione di Prodotto v1.2.0} & 62 & Superato\\
\midrule
\emph{Manuale Utente v1.2.0} & 83 & Superato\\
\bottomrule
\caption{Esiti dell'indice di Gulpease calcolato sui documenti durante la Progettazione di dettaglio e codifica}
\label{tab:changelog}
\end{longtable}

\subsubsection{Verifica e Validazione}
\paragraph{Processi}
Di seguito vengono riportati i valori degli indici SV e BV calcolati durante il periodo di tempo dedicato alla Verifica e Validazione.
\begin{longtable}{|c|p{3cm}|p{3cm}|}
\toprule
\textbf{Attività} & \textbf{SV} & \textbf{BV} \\

%aggiungere qui una midrule per aggiungere una nuova riga alla tabella

\midrule
\emph{Analisi dei Requisiti} & 0 & -130 \\
\midrule
\emph{Glossario} & 0 & 0 \\
\midrule
\emph{Norme di Progetto} & 0 & +20 \\
\midrule
\emph{Piano di Progetto} & 0 & 0 \\
\midrule
\emph{Piano di Qualifica} & 0 & +45 \\
\midrule
\emph{Specifica Tecnica} & 0 & 0 \\
\midrule
\emph{Definizione di Prodotto} & 0 & 0 \\
\midrule
\emph{Manuale utente} & 0 & 0 \\
\midrule
\emph{Codifica} & 0 & -30 \\
\midrule
\emph{Verifica della Codifica} & 0 & -15 \\
\midrule
\emph{Preparazione collaudo} & 0 & 0 \\
\midrule
\emph{Collaudo} & 0 & +30 \\
\bottomrule
\caption{BV e SV calcolati sui documenti durante la Verifica e Validazione}
\label{tab:changelog}
\end{longtable}

\paragraph{Conclusioni}
Lo SV è nullo in quanto è stato realizzato tutto ciò che è stato pianificato.
Il BV dell' Analisi dei Requisiti è invece negativo a causa del fatto che sono state richieste delle ore di analista per aggiungere i requisiti espressi dal Proponente. Di conseguenza sono state richieste delle ore di verifica per verificare le aggiunte.



\paragraph{Documenti}
Di seguito vengono riportati, per ogni documento, i valori dell'indice di Gulpease calcolati durante il periodo di tempo dedicato alla Verifica e Validazione.

\begin{longtable}{|c|p{3cm}|p{3cm}|}
\toprule
\textbf{Documento} & \textbf{Valore indice} & \textbf{Esito} \\

%aggiungere qui una midrule per aggiungere una nuova riga alla tabella

\midrule
\emph{Analisi dei Requisiti v5.2.0} &  &  \\
\midrule
\emph{Glossario v5.2.0} &  &  \\
\midrule
\emph{Norme di Progetto v5.2.0} &  &  \\
\midrule
\emph{Piano di Progetto v5.2.0} &  & \\
\midrule
\emph{Piano di Qualifica v5.2.0} &  &  \\
\midrule
\emph{Specifica Tecnica v3.2.0} &  & \\
\midrule
\emph{Definizione di Prodotto v2.2.0} &  & \\
\midrule
\emph{Manuale Utente v2.2.0} &  & \\
\bottomrule
\caption{Esiti dell'indice di Gulpease calcolato sui documenti durante la Verifica e Validazione}
\label{tab:changelog}
\end{longtable}

\paragraph{Risultati delle misurazioni sul codice}
Di seguito vengono riportati i risultati dei test di analisi statica e dinamica effettuati sul codice.
Per ogni test effettuato si riporta il valore medio e il valore massimo.
\begin{itemize}
\item{\textbf{Complessità ciclomatica}}
\begin{itemize}
\item{\textbf{Valore medio: }}3.7
\end{itemize}
\item{\textbf{Parametri per metodo}}
\begin{itemize}
\item{\textbf{Valore medio: }}1
\end{itemize}
\item{\textbf{Linee di codice per metodo}}
\begin{itemize}
\item{\textbf{Valore medio: }}13
\end{itemize}
\item{\textbf{Linee di codice per linee di commento}}
\begin{itemize}
\item{\textbf{Valore medio: }}
\end{itemize}
\item{\textbf{Instabilità}}
\begin{itemize}
\item{\textbf{Valore medio: }}0.4
\end{itemize}
\end{itemize}
\begin{longtable}{|p{11cm}|c|c|c|}
\toprule
\textbf{Componente} & \textbf{Instabilità} \\

\midrule
MaaP::Server
& 0\\

\midrule
MaaP::Server::ModelServer
& 0\\

\midrule
MaaP::Server::ModelServer::DataManager
& 0.8\\

\midrule
MaaP::Server::ModelServer::DataManager::DatabaseAnalysisManager
& 0.8\\

\midrule
MaaP::Server::ModelServer::DataManager::DatabaseUserManager
& 0.8\\


\midrule
MaaP::Server::ModelServer::DataManager::IndexManager
& 0.6\\


\midrule
MaaP::Server::ModelServer::Database
& 0\\


\midrule
MaaP::Server::ModelServer::DSL
& 0\\

\midrule
MaaP::Server::Controller
& 0.8\\

\midrule
MaaP::Client
& 1\\

\midrule
MaaP::Client::View
& 1\\

\midrule
MaaP::Client::ControllerModelView
& 0.6\\

\midrule
MaaP::Client::ControllerModelView::ControllerClient
& 0.6\\

\midrule
MaaP::Client::ModelClient
& 0\\

\midrule
MaaP::Client::ModelClient::Services
& 0\\

\midrule
MaaP::Client::ModelClient::Directives
& 0\\

\midrule
MaaP::Client::ModelClient::Model
& 0\\


\bottomrule
\caption{Tabella accoppiamento componenti}
\end{longtable}

\subparagraph{Conclusioni}
bla bla bla.....
Per quanto riguarda l'instabilità, la stragrande maggioranza dei package rientra nel range di accettazione, molti dei quali anche nel range ottimale. L'unico package che risulta essere instabile, è la view del client. Questo, ampiamente preventivato in precedenza, è dovuto all'utilizzo di AngularJS per la parte client.

\paragraph{Copertura del codice}









\subsection{Dettaglio dell'esito delle revisioni}
Per ciascuna revisione alla quale si intende partecipare, il Committente avrà il compito di segnalare eventuali problematiche trovate, dando una valutazione globale dell'andamento del progetto e una descrizione per ciascun documento con correzioni e accorgimenti da apportare.
Di seguito vengono elencate le modifiche apportate ai documenti, come suggerito dal Committente, per ciascuna revisione.
\subsubsection{Revisione dei Requisiti}
\begin{itemize}
\item \grassetto{Studio di Fattibilità:} il documento ha avuto una valutazione positiva, quindi non ci sono stati accorgimenti da apportare;
\item \grassetto{Norme di Progetto:} il documento è stato riorganizzato come suggerito, ovvero per processi, attività procedure e strumenti; è stata migliorata la descrizione della rotazione dei ruoli e il documento è stato incrementato con le parti riguardanti la parte di progettazione;
\item \grassetto{Analisi dei Requisiti:} sono stati corretti degli errori grammaticali, chiariti i significati di alcune parole; i casi d'uso segnalati hanno subito modifiche e aggiustamenti alle pre e post condizioni, mentre altri sono stati descritti più approfonditamente. Sempre dei casi d'uso sono stati tolti o spostati perché in contrasto tra di loro, mentre per quanto riguarda la suddivisione dei requisiti in funzionali, desiderabili, obbligatori ecc.. sono stati rimossi e spostati perché non adatti alla categoria in cui si presentavano. Il documento ha avuto una buona valutazione sulla struttura, quindi non si è cambiata.
\item \grassetto{Piano di Progetto:} come suggerito, alcuni contenuti sono stati spostati nell'Appendice del documento; è stato corretto l'utilizzo della parola fase, e usata solo se strettamente necessario e in contesti che la richiedono. La sezione "Preventivo a finire" è stata corretta in "Consuntivo", in quanto si è capito la differenza tra i significati dei due termini. Si è deciso, per i prossimi intervalli di tempo antecedenti le revisioni, di dedicare più tempo all'attività di Verifica, cercando di raggiungere la soglia del 30\% del tempo totale, come suggerito. Il documento inoltre è stato incrementato con le parti relative alla progettazione;
\item \grassetto{Piano di Qualifica:} il documento ha subito profonde modifiche, è stato ristrutturato e riorganizzato. Per fare ciò, è stata seguita la best practice per la struttura dei documenti presente nel sito del Professor Vardanega; il documento ha subito profonde modifiche anche nei contenuti, inoltre è stato incrementato con le parti relative alla progettazione;
\item \grassetto{Glossario:} il documento ha subito una lieve ristrutturazione, è stato tolto l'indice come suggerito; il documento è stato incrementato con l'inserimento di altri termini.
\end{itemize}

\subsubsection{Revisione di Progettazione}
\begin{itemize}
\item \grassetto{Norme di Progetto:} il documento è stato riorganizzato come suggerito ed è stato incrementato con le parti riguardanti la parte di progettazione in dettaglio e codifica;
\item \grassetto{Analisi dei Requisiti:} sono stati corretti alcuni scenari alternativi e inseriti come estensioni, come da suggerimento, e apportate piccole modifiche. Nel complesso il documento ha raggiunto un buon livello di maturità.
\item \grassetto{Piano di Progetto:} come suggerito, il documento è stato riorganizzato e nella forma e nei contenuti. Inoltre è stato incrementato con le parti relative alla progettazione;
\item \grassetto{Piano di Qualifica:} il documento ha subito profonde modifiche, è stato ristrutturato e riorganizzato.Sono state aggiunte le sezioni riguardanti le strategie e gli obiettivi di qualità.
Il documento inoltre ha subito un incremento con l'aggiunta delle parti relative alla progettazione di dettaglio e codifica;
\item \grassetto{Glossario:} il documento ha subito piccole modifiche, ed è stato incrementato con termini nuovi.
\item \grassetto{Specifica tecnica:} il documento ha subito profonde modifiche; come suggerito, sono stati rivisti e rifatti alcuni diagrammi ed è stato inserito il tracciamento mancante.
\end{itemize}

\subsubsection{Revisione di Qualifica}
\begin{itemize}
\item \grassetto{Norme di Progetto:} al documento sono state apportate le modifiche richieste in sede di Revisione di Qualifica;
\item \grassetto{Analisi dei Requisiti:} al documento sono stati aggiunti nuovi requisiti emersi dall'ultimo incontro con il Proponente;
\item \grassetto{Piano di Progetto:} il documento non ha subito modifiche. Inoltre il documento è stato incrementato con il consuntivo della fase Pre RA e il consuntivo finale.
\item \grassetto{Piano di Qualifica:} al documento non sono state apportate modifiche sostanziali.
Il documento inoltre ha subito un incremento con l'aggiunta delle parti relative ai resoconti dell'attività di Verifica;
\item \grassetto{Glossario:} il documento è stato incrementato con termini nuovi;
\item \grassetto{Specifica tecnica:} il documento ha subito le modifiche richieste in sede di Revisione di Qualifica;
\item \grassetto{Definizione di Prodotto:} il documento ha subito le modifiche richieste in sede di Revisione di Qualifica. Inoltre sono stati aggiunti nuovi metodi;
\item \grassetto{Manuale Utente:} il documento ha subito le modifiche richieste in sede di Revisione di Qualifica ed è stato ultimato.
\end{itemize}





%FINE DOCUMENTO NON CANCELLARE
\end{document}
