%includo il file che contiene la versione dei documenti
\newcommand{\versioneAnalisiDeiRequisiti}{2.2.0}			
\newcommand{\versioneNormeDiProgetto}{2.2.0}			
\newcommand{\versioneGlossario}{2.2.0}			
\newcommand{\versionePianoDiQualifica}{2.2.0}			
\newcommand{\versionePianoDiProgetto}{2.2.0}	
\newcommand{\versioneStudioDiFattibilita}{2.2.0}
\newcommand{\versioneSpecificaTecnica}{2.2.0}


\newcommand{\Versione}{\versionePianoDiQualifica{}}	%Versione Finale
\newcommand{\Data}{2013-11-28}						%Data di creazione
\newcommand{\DataUltimaModifica}{2014-03-25}
\newcommand{\TipoDocumento}{Piano di Qualifica}		%tipo documento


%includo il file header.tex (logo grande in prima pagina piu qualche altra regola)
%questo file contiene impostazioni comuni per tutte i documenti

%definizione packages utilizzati
\documentclass[a4paper]{article}
\usepackage[utf8x]{inputenc}
\usepackage{enumitem}
\usepackage[italian]{babel}
\usepackage{latexsym}
\usepackage{xparse}
\usepackage{float}
\usepackage{subfloat}
\usepackage{subfig}
\usepackage{fancyhdr}
\usepackage{eurofont}
\usepackage{lastpage}
\usepackage{graphicx}
\usepackage{textcomp}
\usepackage{booktabs}
\usepackage{color}
\usepackage{lscape}
\usepackage{hyperref}
\hypersetup{colorlinks=true, linkcolor=black, anchorcolor=red, urlcolor=blue}
\usepackage{longtable}
\usepackage{tabularx}
\usepackage{abstract}
\usepackage{appendix}
\usepackage{multicol}
\usepackage{bmpsize}
\usepackage[all]{hypcap}
\usepackage{titlesec}
\usepackage{indentfirst}
\usepackage{lipsum,titletoc}

%\setcounter{secnumdepth}{4}

%****************INIZIO GESTIONE SUBSECTION MULTIPLE
\makeatletter
\newcommand\level[1]{%
  \ifcase#1\relax\expandafter\chapter\or
    \expandafter\section\or
    \expandafter\subsection\or
    \expandafter\subsubsection\else
    \def\next{\@level{#1}}\expandafter\next
  \fi}
\newcommand{\@level}[1]{%
  \@startsection{level#1}
    {#1}
    {\z@}%
    {-3.25ex\@plus -1ex \@minus -.2ex}%
    {1.5ex \@plus .2ex}%
    {\normalfont\normalsize\bfseries}}

\newdimen\@leveldim
\newdimen\@dotsdim
{\normalfont\normalsize
 \sbox\z@{0}\global\@leveldim=\wd\z@
 \sbox\z@{.}\global\@dotsdim=\wd\z@
}

\newcounter{level4}[subsubsection]
\@namedef{thelevel4}{\thesubsubsection.\arabic{level4}}
\@namedef{level4mark}#1{}
\def\l@section{\@dottedtocline{1}{0pt}{\dimexpr\@leveldim*4+\@dotsdim*1+6pt\relax}}
\def\l@subsection{\@dottedtocline{2}{0pt}{\dimexpr\@leveldim*5+\@dotsdim*2+6pt\relax}}
\def\l@subsubsection{\@dottedtocline{3}{0pt}{\dimexpr\@leveldim*6+\@dotsdim*3+6pt\relax}}
\@namedef{l@level4}{\@dottedtocline{4}{0pt}{\dimexpr\@leveldim*7+\@dotsdim*4+6pt\relax}}

\count@=4
\def\@ncp#1{\number\numexpr\count@+#1\relax}
\loop\ifnum\count@<100
  \begingroup\edef\x{\endgroup
    \noexpand\newcounter{level\@ncp{1}}[level\number\count@]
    \noexpand\@namedef{thelevel\@ncp{1}}{%
      \noexpand\@nameuse{thelevel\@ncp{0}}.\noexpand\arabic{level\@ncp{1}}}
    \noexpand\@namedef{level\@ncp{1}mark}####1{}%
    \noexpand\@namedef{l@level\@ncp{1}}%
      {\noexpand\@dottedtocline{\@ncp{1}}{0pt}{\the\dimexpr\@leveldim*\@ncp{5}+\@dotsdim*\@ncp{0}\relax}}}%
  \x
  \advance\count@\@ne
\repeat
\makeatother
\setcounter{secnumdepth}{100}
\setcounter{tocdepth}{100}
%****************FINE GESTIONE SUBSECTION MULTIPLE

%impostazioni relative alla visualizzazione delle section 
%nell'indice
\titlecontents{section}
[0pt]%left indent
{\bfseries}
{\contentslabel{2.3em}}
{\hspace*{-2.3em}}
{\hfill\contentspage}
[]%separator


\oddsidemargin=.15in
\evensidemargin=.15in
\textwidth=6in
\topmargin=-.5in
\parindent=0in
\headheight=1in
\DeclareMathSizes{10}{10}{10}{10} %per piano qualifica
\pagestyle{fancy}
\lhead{
\bfseries {\Large \TipoDocumento}\\
\bfseries Versione: \Versione\\
}
\chead{}
\lhead{
\includegraphics[scale=0.455]{../Logo&Header/apertureHead.png}
}
\lfoot{\bfseries \TipoDocumento{} v\Versione}
\cfoot{}
\rfoot{\thepage\ of \mypageref{LastPage}}
\newcommand{\mypageref}[1]{
\hypersetup{linkcolor=black}\pageref{#1}\hypersetup{linkcolor=black}}
%\userpackage{lipsum}
\renewcommand{\footrulewidth}{0.4pt}

%definizioni comandi comuni utilizzati
\newcommand{\numref}[1]{\textsl{\nameref{#1} (\ref{#1})}}
\newcommand{\NomeGruppo}{Aperture Software}
\newcommand{\Progetto}{MaaP: MongoDB as an admin Platform}
\newcommand{\Prop}{CoffeeStrap}

%definizione tecnologie
\newcommand{\Node}{Node.js}
\newcommand{\NodeJS}{Node.js}
\newcommand{\Nodejs}{Node.js}

\newcommand{\mongodb}{MongoDB}

%tanti sub quanti ne vogliamo! :)
\newcommand{\subsubsubsection}{\level{4}}
\newcommand{\subsubsubsubsection}{\level{5}}
\newcommand{\subsubsubsubsubsection}{\level{6}}
\newcommand{\subsubsubsubsubsubsection}{\level{7}}
\newcommand{\subsubsubsubsubsubsubsection}{\level{8}}


%definizione comando per parola glossario
\newcommand{\gloss}[1]{\emph{#1}\ped{\emph{\tiny{G}}}}

\newcommand{\grassetto}{\textbf}

%per inserire immagini
\newcommand{\immagine}[2]{ 
\begin{center}
\begin{figure}[H]
\includegraphics[width=\textwidth]{{{#1}}}
\caption{#2}
\label{#1}
\end{figure}
\end{center}
}

\newcommand{\Glossario}{
Al fine di evitare ogni ambiguità nella comprensione del linguaggio utilizzato nel presente documento e, in generale, nella documentazione fornita dal gruppo \NomeGruppo{}, ogni termine tecnico, di difficile comprensione o di necessario approfondimento verrà inserito nel documento \emph{Glossario\_{}v\versioneGlossario{}.pdf}.\\
Saranno in esso definiti e descritti tutti i termini in corsivo e allo stesso tempo marcati da una lettera "G" maiuscola in pedice nella documentazione fornita.
}

\newcommand{\Prodotto}{
Lo scopo del prodotto è produrre un framework per generare interfacce web di amministrazione dei dati di business basati sullo stack \Nodejs{} e \mongodb{}.\\
L'obiettivo è quello di semplificare il lavoro allo sviluppatore che dovrà rispondere in modo rapido e standard alle richieste degli esperti di business.
}

%inizio pagina del documento 
\begin{document}
\thispagestyle{empty}

\begin{center}\centerline{
%inserisco il logo grande della prima pagina
\includegraphics[scale=0.8]{../Logo&Header/logo.png}}

%metto il link dell'email sotto al logo
%{\href{mailto:ApertureSWE@gmail.com}{\color[rgb]{0.39,0.37,0.38}%ApertureSWE@gmail.com}}\\ [3pc]

\vspace{0.5in}

%titolo del progetto
{\Huge {\Progetto}}\\[.5pc]

\underline{\hspace{6in}}\\[8pc]

{\Huge {\TipoDocumento}}\\[1pc]
%{\emph{Versione \Versione}}\\
\end{center}

%\vspace{.05in}

%\vspace{.05in}

%informazioni documento
\begin{center}
%\section{Informazioni documento}
\begin{tabular}{r|l}
%\textbf{Nome} &\TipoDocumento \\
\textbf{Versione} & \Versione{} \\
\textbf{Data creazione} & \Data{} \\
\textbf{Data ultima modifica} & \DataUltimaModifica{} \\
\textbf{Stato del Documento} & Formale \\		%CAMBIARE QUI
\textbf{Uso del Documento} & Esterno \\			%CAMBIARE QUI
\textbf{Redazione} & Mattia Sorgato, Giacomo Pinato, Fabio Miotto,\\			%CAMBIARE QUI
& Michele Maso, Alberto Garbui \\
\textbf{Verifica} & Fabio Miotto, Alessandro Benetti \\%ED ANCHE QUI!
\textbf{Approvazione} & Fabio Miotto\\				%CAMBIARE QUI
\textbf{Distribuzione} & \parbox[t]{4cm}{Prof. Tullio Vardanega \\ Prof. Riccardo Cardin \\ \Prop{} }
\end{tabular}
\end{center}

\vspace{0.05in}

%inizio sommario del documento
\begin{abstract}
\begin{center}
Questo documento ha lo scopo di presentare le strategie adottate dal gruppo \NomeGruppo{} nell'ottica del miglioramento continuo e assicurazione della qualità.
\end{center}
\end{abstract}

%\vspace{.4in}

%seconda pagina, diario delle modifiche
\newpage
Diario delle modifiche
\begin{center}
\begin{longtable}{|c|c|c|p{0.5\linewidth}|}
\toprule
\textbf{Versione} & \textbf{Data} & \textbf{Autore} & \textbf{Modifiche effettuate}\\

%aggiungere qui una midrule per aggiungere una nuova riga alla tabella
\midrule
3.2.0 & 2014-03-25 & Fabio Miotto (RE) & Approvazione documento.\\
\midrule
3.1.1 & 2014-03-24 & Alessandro Benetti (VR) & Verifica documento.\\
\midrule
3.1.0 & 2014-03-21 & Fabio Miotto (VR) & Verifica documento.\\
\midrule
3.0.8 & 2014-03-19 & Mattia Sorgato (VR) & Resoconto attività di verifica.\\
\midrule
3.0.7 & 2014-03-19 & Giacomo Pinato (VR) & Resoconto attività PDCA.\\
\midrule
3.0.6 & 2014-03-12 & Giacomo Pinato (PR) & Aggiunta test di validazione e tracciamento.\\
\midrule
3.0.5 & 2014-03-12 & Mattia Sorgato (PR) & Aggiunta test di integrazione e tracciamento.\\
\midrule
3.0.4 & 2014-03-10 & Mattia Sorgato (PR) & Aggiunta test di sistema.\\
\midrule
3.0.3 & 2014-02-28 & Michele Maso (PR) & Incremento documento con progettazione test.\\
\midrule
3.0.2 & 2014-01-17 & Alberto Garbui (AM) & Aggiunta analisi RR documento.\\
\midrule
3.0.1 & 2014-01-16 & Fabio Miotto (AM) & Aggiunta analisi RR documento.\\

\midrule
3.0.1 & 2014-01-14 & Fabio Miotto (AN) & Effettuate correzioni segnalate dal Committente.\\

\midrule
2.2.0 & 2014-01-07 & Alberto Garbui (RE) & Approvazione documento.\\
\midrule
2.1.0 & 2014-01-06 & Fabio Miotto (VR) & Verifica documento.\\
\midrule
2.0.3 & 2014-01-04 & Andrea Perin (VR) & Esito metriche processi.\\
\midrule
2.0.2 & 2014-01-04  & Andrea Perin (VR) & Esito metriche documenti.\\
\midrule
2.0.1 & 2014-01-03  & Michele Maso (AN) & Incremento documento.\\

\midrule
1.2.0 & 2013-12-16 & Giacomo Pinato (RE) & Approvazione documento\\
\midrule
1.1.1 & 2013-12-16 & Alessandro Benetti (VR) & Verifica documento\\
\midrule
1.1.0 & 2013-12-15 & Fabio Miotto (VR) & Verifica documento\\
\midrule
1.0.6 & 2013-12-13 & Giacomo Pinato (RE) & Aggiunto resoconto attività di Verifica e standard di qualità\\
\midrule
1.0.4 & 2013-12-04 & Mattia Sorgato (AM) & Aggiunta metriche\\
\midrule
1.0.3 & 2013-12-03 & Alberto Garbui (AN) & Aggiunta analisi\\
\midrule
1.0.2 & 2013-12-01 & Andrea Perin (RE) & Aggiunta strategie di Verifica\\
\midrule
1.0.1 & 2013-11-28 & Andrea Perin (RE) & Creazione documento\\

\bottomrule
\caption{Registro delle modifiche}
\label{tab:changelog}
\end{longtable}
\end{center}

%terza pagina Indice (viene aggiornato in automatico con due compilazioni)
\newpage
\tableofcontents

%pagine successive hanno la lista di tabelle e lista delle figure
%(vengono aggiornate in automatico)
\newpage
\listoftables
\listoffigures

%qui inizia la prima pagina ufficiale
\newpage
\section{Introduzione}%1.0
\label{1.0}
\subsection{Scopo del documento}%1.1
\label{1.1}
Il Piano di Qualifica ha l'obiettivo di definire le strategie adottate dal gruppo Aperture \gloss{Software} per garantire la qualità del prodotto che verrà sviluppato.
Il presente documento descriverà le qualità desiderate che il software dovrà avere, le metriche utilizzate per rendere il prodotto e i processi quantificabili. Per ottenere obiettivi finali qualitativi è necessario un continuo e costante \gloss{processo} di verifica, per scovare ed eliminare errori in maniera rapida e senza spreco di risorse.

\subsection{Scopo del prodotto}%1.2
\label{1.2}
\Prodotto{}

\subsection{Glossario}%1.3
\label{1.3}
\Glossario{}

\subsection{Riferimenti} %1.4
\label{1.4}
\subsubsection{Normativi}
\label{1.4.1}
\begin{itemize}
\item \grassetto{Norme di Progetto}: \emph{Norme\_di\_progetto\_v\versioneNormeDiProgetto{}.pdf};\\
\item \grassetto{Capitolato d'appalto C1}: MaaP as an admin Platform\\
\url{http://www.math.unipd.it/~tullio/IS-1/2013/Progetto/C1.pdf}.
\end{itemize}
\subsubsection{Informativi}
\label{1.4.2}
\begin{itemize}
\item \grassetto{Piano di Progetto}:  \emph{Piano\_di\_progetto\_v\versionePianoDiProgetto{}.pdf};\\
\item \grassetto{Glossario}: \emph{Glossario\_v\versioneGlossario{}.pdf};\\
\item \grassetto{Slides del corso di Ingegneria del software Modulo A, AA 2013/2014 del prof. Tullio Vardanega}:
\\ \url{http://www.math.unipd.it/~tullio/IS-1/2013/};
\item \grassetto{SWEBOK-Version 3 (2004):} capitolo 11-Software Quality \\ \url{http://www.computer.org/portal/web/swebook/html/ch11};
\item \grassetto{Wikipedia}: \url{http://it.wikipedia.org};
\item \grassetto{Ian Sommerville, Software Engineering, 9 edizione (2011)}:
\begin{itemize}
\item Capitolo 24 - Quality management;
\item Capitolo 26 - Process improvement.
\end{itemize}
\item \grassetto{Standard ISO/IEC TR 15504 Software process assessment}:
\\ \url{http://en.wikipedia.org/wiki/ISO/IEC_15504};
\item \grassetto{Standard ISO/IEC 9126:2001 Software engineering-product quality}:
\\ \url{http://en.wikipedia.org/wiki/ISO_9126};
\item \grassetto{Budget Variance e Schedule Variance - Dati empirici}: \\ \url{http://office.microsoft.com/en-us/project-help/determine-the-right-threshold-for-project-cost-and-schedule-variances-HA010173335.aspx}.
\item \grassetto{Indice Gulpease:}
\begin{itemize}
\item \url{http://it.wikipedia.org/wiki/Indice_Gulpease};
\item \url{http://xoomer.virgilio.it/roberto-ricci/variabilialeatorie/esperimenti/leggibilita.htm}.
\end{itemize}
\item \grassetto{Complessità ciclomatica:}
\begin{itemize}
\item \url{http://it.wikipedia.org/wiki/Complessit%C3%A0_ciclomatica.}
\end{itemize}
\end{itemize}


\newpage
\section{Panoramica della metodologia di verifica}%2.0
\label{2.1}
\subsection{Definizione obiettivi}
\subsubsection{Qualità di processo} %2.1
\label{2.1.1}
Per garantire la qualità del prodotto finale è necessario migliorare la metodologia che porta alla qualità dei processi che compongono il prodotto. Per fare questo si è deciso di utilizzare lo standard ISO/IEC 15504\footnote{vedi \gloss{Appendice}, sezione A.1} denominato SPICE\footnote{Software Process Improvement and Capability Determination}.
Per applicare il modello appena citato si deve utilizzare il ciclo di Deming\footnote{vedi appendice, sezione A.2} che ha come obiettivo il miglioramento continuo dei processi nel loro \gloss{ciclo di vita}.

\subsubsection{Qualità di prodotto} %2.1
\label{2.1.2}
Per cercare di realizzare e progettare un prodotto software, in accordo con specifiche e standard definiti, ed essere privo di non conformità o difetti, è necessario usare lo standard ISO/IEC 9126\footnote{vedi appendice, sezione A.3}, il quale redige e descrive obiettivi qualitativi e fornisce delle linee guida d'utilizzo di metriche al fine di tracciare il progresso nel miglioramento continuo di processo e prodotto.

\subsection{Procedure di controllo di qualità di processo} %2.1
\label{2.2}
Per garantire la qualità dei processi si utilizza il ciclo PDCA\footnote{Alias, Ciclo di Deming, vedi appendice, sezione A.2}.  Questo principio permette un continuo miglioramento della qualità di tutti i processi coinvolti nella realizzazione del prodotto finale.
Per controllare la qualità bisogna che i processi siano pianificati dettagliatamente, che le risorse siano individuate e ripartite in maniera quantificabile e che ci sia un controllo sui processi. Lo sviluppo di quanto scritto prima è descritto dettagliatamente nel \emph{Piano di Progetto \versionePianoDiProgetto{}}.
Inoltre verrà monitorata la qualità dei processi con l'analisi continua della qualità del prodotto.
Per rendere quantificabile la qualità dei processi si utilizzano le metriche descritte nella sezione 2.9.1 di questo documento.

\subsection{Procedure di controllo di qualità di prodotto} %2.1
\label{2.3}
Per garantire il controllo di qualità si utilizza:
\begin{itemize}
\item \grassetto{Quality Assurance:} tradotta in "assicurazione di qualità", è l'insieme di processi che hanno come fine il miglioramento e il perseguimento della qualità. L'intenzione di un team di lavoro consiste nell'ottenere quella che si dice correction by construction, ovvero "correttezza per costruzione";
\item \grassetto{Strategie proattive:} l'insieme delle strategie proattive, le cui procedure sono descritte nel documento Norme di Progetto, permettono di garantire qualità a tempo zero, limitando \gloss{attività} di verifica che hanno un costo non indifferente.
\item \grassetto{Verifica}: è la valutazione che un prodotto, \gloss{servizio} o sistema, sia conforme a regole, requisiti, specifiche o condizioni imposte. È spesso un processo interno e differisce dalla validazione. In analogia, l'attività di Verifica deve rispondere alla domanda: "did i built the system right?", ovvero "ho costruito il sistema in modo corretto?";
\item \grassetto{Validazione:} è l'assicurazione che un prodotto, servizio o sistema, incontri le necessità che i clienti o gli stakeholder identificano. Spesso comporta l'accettazione e l'idoneità con clienti esterni. In questo caso la domanda è: "did i built the right system?", tradotto in "ho costruito il sistema giusto?".
\end{itemize}

\subsection{Organizzazione} %2.1
\label{2.4}
Per ogni processo attuato ci sono delle attività di Verifica e per ogni processo realizzato viene verificata la qualità del processo stesso e la qualità dell'eventuale prodotto ottenuto da esso.
Ogni periodo di tempo antecedente la consegna di revisione descritto nel Piano di Progetto, necessita di attività di Verifica:
\begin{itemize}
\item \grassetto{Analisi:} si seguono i metodi di Verifica descritti nelle Norme di Progetto sui documenti prodotti e i processi attuati. La messa in opera di tali tecniche è descritta nell'appendice sezione C.3.1;
\item\grassetto{Analisi di Dettaglio:} si verificano i processi che determinano l'incremento dei documenti redatti per il precedente periodo di Analisi, e si verificano i prodotti generati dai relativi processi, seguendo le Norme di Progetto. La messa in opera di tale attività è descritta nell'appendice sezione C.3.2;
\item\grassetto{Progettazione Architetturale:}  si verificano i processi che determinano l'incremento dei documenti redatti per il precedente periodo di Analisi in Dettaglio e si verificano i prodotti generati dai relativi processi; inoltre si verificano processi e prodotti per l'attività di Progettazione Architetturale, seguendo le Norme di Progetto. La messa in opera di tale attività è descritta nell'appendice sezione C.3.3.
\end{itemize}
Questa sezione conterrà i riferimenti alle successive attività di Verifica, aggiornati dopo ogni fase di produzione e di \gloss{test} del prodotto.
La stesura dei documenti è l'attività principale e costante nello svolgimento del progetto, mentre il processo di Verifica viene diviso in due attività.
In ogni documento è presente un diario delle modifiche per mantenere una cronologia delle attività svolte e di chi le ha svolte.

\subsection{Pianificazione strategica e temporale} %2.1
\label{2.5}
Avendo scadenze prefissate nel Piano di Progetto, dobbiamo garantire che le attività di Verifica di tutti i documenti e prodotti debbano essere sistematiche, disciplinate e quantificabili. Procedendo in questa maniera si correggono gli errori il prima possibile.
La metodologia da seguire per l'individuazione e correzione degli errori è descritta nelle Norme di Progetto.
Ogni attività di redazione di documenti e di scrittura del \gloss{codice} è stata preceduta da uno studio iniziale sull'impaginazione dei documenti e del contenuto degli stessi. Questo serve per minimizzare la possibilità di incorrere in errori di tipo concettuale e tecnico.

\subsection{Responsabilità} %2.1
\label{2.6}
Per garantire che il processo di Verifica sia disciplinato, sistematico e quantificabile, bisogna attribuire responsabilità a specifici ruoli di progetto. I ruoli sono Responsabile di Progetto e Verificatore. I compiti di ciascun ruolo sono descritti nelle \emph{Norme\_di\_progetto\_v\versioneNormeDiProgetto{}.pdf}, rispettivamente nelle sezioni 4.1 e 4.5.

\subsection{Risorse} %2.1
\label{2.7}
Per raggiungere gli obiettivi di qualità prefissati sono necessarie risorse umane e tecnologiche, suddivise rispettivamente in strumenti software e \gloss{hardware} utilizzati dai componenti del gruppo per effettuare Verifica su processi e prodotti. I ruoli maggiormente coinvolti nella responsabilità delle attività di Verifica e Validazione sono il Responsabile di Progetto e il Verificatore e i rispettivi compiti sono descritti dettagliatamente nelle Norme di Progetto. Per facilitare il lavoro dei Verificatori sono stati usati degli strumenti automatici che eseguono controlli sistematici sui prodotti generati. Questi strumenti sono descritti nelle Norme di Progetto.

\subsection{Tecniche di analisi} %3.0
\label{2.8}

\subsubsection{Analisi statica}
\label{3.1}
Questa tipologia di analisi può essere applicata sia al codice che alla documentazione, dato che prevede l'utilizzo di tecniche generali per ogni tipo di prodotto del team.

\paragraph{Walkthrough}

Questa tecnica utilizza una scansione ampia e non mirata dell'oggetto in verifica, data la mancanza di esperienza \gloss{best practice} del Verificatore.
L'attuazione di questa tecnica di Analisi è quindi molto onerosa, per questo sarà nostro obiettivo renderla più parallelizzabile possibile, così da ridurre i costi di Verifica e per essere più efficace ed efficiente.
Si comincia con una attività preliminare di lettura, seguita da una individuazione degli errori; poi si procede con la correzione degli stessi e con una successiva attività di lettura per controllare le modifiche apportate.
Dopo ogni attività di Verifica tramite Walkthrough, sperabilmente avremo trovato la maggior parte degli errori, fornendoci una visione delle erroneità commesse, di conseguenza potremo raffinare l'Analisi e avvicinarci alla metodologia di Inspection.

\paragraph{Inspection}
\label{3.1.2}
Questa tecnica è un'evoluzione del Walkthrough e applica una ricerca più mirata e specifica. È possibile utilizzare questa tecnica dopo aver acquisito dimestichezza con l'attività di Verifica, stilando una lista di controllo contenente i maggiori errori riscontrati applicando la tecnica di Walkthrough, quindi non sarà possibile utilizzarla fin da subito in quanto la lista inizialmente è vuota. \`{E} obiettivo di una fase di Inspection la ricerca mirata di errori, aumentando l'efficienza della Verifica e riducendo i costi in termini di tempo e risorse.
Durante l'utilizzo della tecnica di Inspection la lista verrà aggiornata; la lista risultante è in appendice A delle \emph{Norme\_di\_progetto\_v\versioneNormeDiProgetto{}.pdf}.

\subsubsection{Analisi dinamica}
\label{3.2}
Questa particolare tipologia di Analisi si applica solamente ai prodotti software sviluppati dal team, mediante l'utilizzo di test progettati e scritti appositamente per verificare la correttezza dei prodotti e la loro effettiva validazione.
L'importanza di un test si attua nella sua automazione, in quanto riduce il tempo dedicato alla Verifica manuale del codice, certamente più onerosa. Per questo motivo la proprietà più importante di un test è la sua ripetibilità.
Per fare in modo che un test abbia questa qualità, è fondamentale definire a priori certe caratteristiche, ovvero:
\begin{itemize}
\item \grassetto{\gloss{Ambiente}}: deve essere specificato l'insieme di componenti hardware e software su cui verrà eseguito il prodotto software, al fine di evitare problemi di incompatibilità e malfunzionamento;
\item \grassetto{Variabili}: si deve conoscere e garantire la corretta struttura delle variabili in ingresso ai test, in modo da prevedere gli \gloss{output} attesi e verificare la loro correttezza;
\item \grassetto{Procedure}: deve essere chiara la sequenza delle operazioni e la metodologia di applicazione dei test.
\end{itemize}
Di seguito analizzeremo i 5 tipi di test attuati nelle varie parti del progetto:
\begin{itemize}
\item \grassetto{Test di unità}: attività di Verifica svolta su ogni singola unità software del sistema, mediante l'utilizzo di \gloss{stub}, \gloss{driver} e \gloss{logger}. Un'unità è la più piccola parte di lavoro che viene assegnata individualmente al \gloss{programmatore}, successivamente sarà prodotta e verificata singolarmente. Mediante tali test viene verificato il corretto funzionamento dei moduli di cui il sistema è composto, in modo da cancellare eventuali errori di implementazione commessi dai programmatori;
\item \grassetto{Test di integrazione}: attività di Verifica che controlla la corretta integrazione di più unità software aggiunte in maniera incrementale, il cui scopo è analizzare che la combinazione delle unità software funzioni come attesa. Grazie a questo test si possono rilevare errori non riscontrabili nei test di unità e comportamenti inaspettati di componenti software già esistenti rilasciati da altri fornitori che non interagiscono correttamente tra di loro. Anche in questa attività, per poter simulare le unità nell'integrazione, vengono create unità fittizie specifiche, come stub e driver, in modo da replicare componenti non ancora sviluppate in modo da non falsare i test;
\item \grassetto{Test di sistema}: questo test si propone di validare il prodotto, una volta che si è stabilita  la sua \gloss{versione} definitiva. Questo test verifica che la copertura dei requisiti obbligatori decisi nel periodo di tempo dedicato all'Analisi in Dettaglio sia completa;
\item \grassetto{Test di regressione}: attività di Verifica che consiste nel ripetere i test già effettuati su una \gloss{componente}, ogni qualvolta quella componente venga modificata o aggiornata; un aiuto lo fornisce il \gloss{tracciamento} delle componenti che permette di scovare e ripetere in modo semplificato i test di unità, di regressione e possibilmente quelli di sistema che sono stati potenzialmente alterati dalla modifica;
\item \grassetto{Test di accettazione}: test finale effettuato dal proponente del software, al cui superamento segue il rilascio del prodotto ultimato.
\end{itemize}

\subsection{Misure e Metriche}
\label{4.0}
Il processo di verifica, per essere chiarificatore, deve essere quantificabile. Le misure apprese dal processo di verifica devono quindi essere basate su metriche prestabilite. Di seguito sono descritte due tipologie di range per le metriche:
\begin{itemize}
\item \grassetto{Sufficiente}: range stabilito come minimo accettabile, sotto il quale ogni unità o documento non verrà accettato come completo;
\item \grassetto{Ottimale}: range consigliato e da usare come riferimento, dal quale è necessario scostarsi il meno possibile.
\end{itemize}

\subsubsection{Metriche per i processi} %2.1
\label{4.3}
Sono stati scelti due indicatori che si basano sui costi e i tempi spesi per un processo come delle metriche ragionevoli per rendere i processi quantificabili. Tali indicatori sono descritti nel Piano di Progetto. Data la scarsa esperienza di cooperazione e di pianificazione delle attività del gruppo, gli obiettivi di questi indici fanno riferimento a convenzioni comuni.

\paragraph{Schedule Variance}
\label{4.3.1}
Indica se si è in linea, in anticipo o in ritardo rispetto alla schedulazione delle attività di progetto pianificate nella \gloss{baseline}. È un indicatore di efficacia e se il suo valore è $> 0$ allora il progetto sta avanzando con maggiore velocità rispetto a quanto pianificato. Viceversa se negativo.
Gli obiettivi fissati sono:
\begin{itemize}
\item \grassetto{Obiettivo Sufficiente}: $ [\geq -(costo\:preventivo\:per\:fase * 5\%)]; $
\item \grassetto{Obiettivo Ottimale}: $ [\geq 0]. $
\end{itemize}

\paragraph{Budget Variance}
\label{4.3.2}
Indica se alla data corrente si è speso di più o di meno rispetto a quanto si era pianificato. Se tale valore è $>0$ allora il progetto sta consumando il proprio budget con minor velocità rispetto a quanto pianificato. Viceversa se negativo.
\begin{itemize}
\item \grassetto{Obiettivo Sufficiente}: $[\geq -(costo\:preventivo\:per\:fase * 10\%)];$
\item \grassetto{Obiettivo Ottimale}: $[\geq 0].$
\end{itemize}

\subsubsection{Metriche per i documenti}
\label{4.1}
Dopo aver vagliato diverse metriche di misurazione nell'ambito dei documenti, abbiamo scelto di utilizzare una \gloss{metrica} di complessità di leggibilità di un testo, tarata sulla lingua italiana. L'eccessiva variabilità nei metodi di analisi della sillabazione dei termini e la conseguente non predicibilità dei test basati sulla sillabazione, hanno portato a scartare diverse metriche internazionali. Inoltre non sono state trovate metriche (oltre all'indice Gulpease) specifiche della lingua italiana e con esito certo.

\paragraph{Indice Gulpease}
L'indice Gulpease è un indice di leggibilità del testo che basa il suo calcolo su componenti del testo enumerabili meccanicamente, così da rendere automatico il processo di Verifica. Consente di misurare la complessità dello stile di un documento.
L'indice di Gulpease considera due variabili linguistiche: la lunghezza della parola e la lunghezza della frase rispetto al numero delle lettere.\\
La formula per il suo calcolo è la seguente:
\\
\begin{center}
\begin{math}
89+\frac{300*(numero\:delle\:frasi)-10*(numero\:delle\:lettere)}{numero\:delle\:parole}
\end{math}
\end{center}
.\\
I risultati sono compresi tra 0 e 100, dove il valore "100" indica la leggibilità più alta e "0" la leggibilità più bassa. In generale risulta che testi con un indice
\begin{itemize}
\item inferiore a 80 sono difficili da leggere per chi ha la licenza elementare;
\item inferiore a 60 sono difficili da leggere per chi ha la licenza media;
\item inferiore a 40 sono difficili da leggere per chi ha un diploma superiore.
\end{itemize}
L'indice prevede un intervallo di valori tra 0 e 100, dove 100 esprime la leggibilità massima.
I nostri obiettivi per l'indice Gulpease sono i seguenti:
\begin{itemize}
\item \grassetto{Obiettivo ottimale}: [51--100];
\item \grassetto{Obiettivo sufficiente}: [40--50].
\end{itemize}

\subsubsection{Metriche software}
\label{2.9.3}
Di seguito verranno riportate delle metriche ritenute le più efficaci, per raggiungere obiettivi di qualità software. Visto che questa sezione riguarda la prossima attività di Codifica, ci potranno essere ulteriori aggiornamenti nelle prossime revisioni.

\paragraph{Complessità ciclomatica}
\label{4.2.1}
La complessità ciclomatica è una metrica software applicabile singolarmente a funzioni, moduli, metodi e classi di un \gloss{programma}.
Questa metrica è calcolata utilizzando il grafo di controllo di flusso del programma, ovvero i nodi del grafo rappresentano gruppi indivisibili di istruzioni, mentre gli archi connettono due nodi se il secondo gruppo di istruzioni può essere eseguito subito dopo il primo gruppo.
Alti valori di questa metrica implicano una scarsa manutenibilità del software, mentre valori troppo bassi possono indicare un'altrettanta bassa efficienza del software.
Un modulo, con complessità ciclomatica elevata, necessita di più \gloss{testing} rispetto ad un altro modulo con complessità ciclomatica minore.
La complessità è quindi definita come:

$$v(G) = e - n + 2p$$

dove:
\begin{itemize}
\item \grassetto{v(G)} = complessità ciclomatica del grafo G;
\item \grassetto{e} = numero di archi del grafo;
\item \grassetto{n} = numero di nodi del grafo;
\item \grassetto{p} = numero di componenti connesse.
\end{itemize}
Degli obiettivi ragionevoli per questa metrica sono i seguenti:
\begin{itemize}
\item \grassetto{Obiettivo Sufficiente}: [11--15];
\item \grassetto{Obiettivo Ottimale}: [1--10]\footnote{Il valore 10 come massimo è stato calcolato da T.J.McCabe, inventore della metrica.}.
\end{itemize}

\paragraph{Livelli di annidamento}
\label{4.2.2}
Il numero di livelli di annidamento dei metodi rappresenta la quantità di richiami di altri metodi all'interno di uno stesso \gloss{metodo}.
Un elevato livello di annidamento definisce un'elevata complessità del codice e di altrettanta comprensione dello stesso.
Gli obiettivi stimati per questa metrica sono:
\begin{itemize}
\item \grassetto{Obiettivo Sufficiente}: [4--6];
\item \grassetto{Obiettivo Ottimale}: [1--3].
\end{itemize}

\paragraph{Attributi per \gloss{classe}}
\label{4.2.3}
Il numero di attributi per classe esprime, appunto, la quantità di proprietà di una classe. Un elevato numero di attributi può denotare un'eccessiva dimensione della classe stessa, che potrebbe piuttosto essere suddivisa in classi più piccole, relazionate tra di loro.
Gli obiettivi per questa metrica sono:
\begin{itemize}
\item \grassetto{Obiettivo Sufficiente}: [9--16];
\item \grassetto{Obiettivo Ottimale}: [3--8].
\end{itemize}

\paragraph{Parametri per metodo}
\label{4.2.4}
Un alto numero di parametri per metodo denota un'eccessiva complessità del metodo stesso, comportandone probabilmente un'ulteriore lunghezza non accettabile. \`{E} buona norma scrivere dei metodi con pochi parametri, al fine di ottenere procedure specifiche e atomiche, di conseguenza facilmente assegnabili e verificabili.
Degli obiettivi validi per questa metrica sono:
\begin{itemize}
\item \grassetto{Obiettivo Sufficiente}: [5--8];
\item \grassetto{Obiettivo Ottimale}: [0--4].
\end{itemize}

\paragraph{Linee di codice per linee di commento}
\label{4.2.5}
Questo numero indica il rapporto tra linee di codice e linee di commento, per avere un fattore di commenti all'interno di un'unità software. In generale, un alto grado di commento del codice porta ad una maggiore manutenibilità ed informazione per uno \gloss{sviluppatore}.
Gli obiettivi stimati sono:
\begin{itemize}
\item \grassetto{Obiettivo Sufficiente}: [$>0.25$];
\item \grassetto{Obiettivo Ottimale}: [0.26--0.30]\footnote{Il valore di 0.30 è stato calcolato dal rapporto 22/78, derivato dalle medie di Ohloh \url{https://www.ohloh.net/p/firefox/factoids\#FactoidCommentsLow}}.
\end{itemize}

\paragraph{Flusso di informazioni}
\label{4.2.6}
Valore che indica il flusso di informazioni passanti per un modulo.
Con:
\begin{itemize}
\item \grassetto{Fan-in}: numero di moduli che passano informazioni al modulo in esame;
\item \grassetto{Fan-out}: numero di moduli a cui il modulo in esame passa informazioni.
\end{itemize}
il valore calcolato è:
\begin{math}(lunghezza\:funzione)2 * Fan-in * Fan-out\end{math}

\paragraph{Accoppiamento}
\label{4.2.7}
Si divide in due categorie:
\begin{itemize}
\item \grassetto{Accoppiamento afferente}: questo valore indica la quantità di classi esterne ad un \gloss{package} che dipendono da classi interne allo stesso.
Un alto valore di accoppiamento in una singola classe del package influisce sull'accoppiamento dell'intero package. Questo fatto non è necessariamente un errore di progettazione, ma il package in esame può rappresentare un punto critico del software. Per contro, un package con basso fattore di accoppiamento può delineare una scarsa utilità del package stesso, che probabilmente andrebbe inglobato con altri package;
\item \grassetto{Accoppiamento efferente}: questo fattore indica l'accoppiamento contrario, ovvero il numero di classi interne al package che dipendono da classi esterne. Più questo indice è basso, più indipendente è il package stesso.
\end{itemize}

\paragraph{Instabilità}
\label{4.2.8}
Il fattore di instabilità di un package indica la possibilità di modifica del package senza influire sulla stabilità del software ad esso dipendente.
Questo indice è calcolato con la formula seguente:
$$I = Ce / (Ca + Ce)$$
dove Ca è l'accoppiamento afferente e Ce l'accoppiamento efferente.
Gli obiettivi forniti da \emph{best practice}\footnote{Range ricavati da \url{http://staff.unak.is/andy/StaticAnalysis0809/metrics/i.html}} sono i seguenti:
\begin{itemize}
\item \grassetto{Obiettivo Sufficiente}: [0.4--0.8];
\item \grassetto{Obiettivo Ottimale}: [0.0--0.3].
\end{itemize}

\paragraph{Copertura del codice}
\label{4.2.9}
Questo fattore indica la percentuale di codice coperto durante l'esecuzione dei test. Più alta sarà la percentuale, minore sarà la possibilità di errori riscontrabili nell'esecuzione del software. Per abbassare questo questo indice sarà sufficiente scrivere metodi semplici che non necessitino di testing. Il valore ideale di 100\% indica che tutte le porzioni di codice sono testate da uno o più test.
Gli obiettivi stimati per questa metrica sono:
\begin{itemize}
\item \grassetto{Obiettivo Sufficiente}: [42\%--65\%];
\item \grassetto{Obiettivo Ottimale}: [66\%--100\%].
\end{itemize}


\newpage


\section{Gestione amministrativa della revisione}
Di seguito verrà descritto come avviene, all'interno del gruppo, la comunicazione per la gestione di anomalie e per il trattamento delle discrepanze.
\label{3.0}
\subsection{Comunicazione e risoluzione di anomalie}
Con anomalia si intende un esito diverso del prodotto rispetto alle aspettative, una violazione delle norme tipografiche di un documento, un valore di qualche indice non valido, ovvero fuori dal range di accettazione.
Se un verificatore scova un'anomalia, di conseguenza aprirà un \gloss{ticket} su RedMine, strumento descritto nelle \emph{Norme\_di\_progetto\_v\versioneNormeDiProgetto{}.pdf}, in sezione 6.1.1.
\label{3.1}
\subsection{Trattamento delle discrepanze}
\label{3.2}
La discrepanza indica una mancata corrispondenza tra il prodotto atteso e il prodotto finito. Essa non ostruisce il funzionamento del software, ma è inesatto rispetto ai requisiti descritti. Per la gestione delle discrepanze si procede nella stessa maniera vista per la gestione delle anomalie.
\subsection{Procedure di controllo di qualità di processo}
Per la gestione dei processi e per il miglioramento degli stessi si utilizzerà il ciclo di Deming o PDCA, descritto in appendice sezione A.
\label{3.3}

\newpage
\appendix
\section{Standard di qualità} %A.0

\subsection{ISO/IEC 15504} %A.1
ISO/IEC 15504, anche conosciuto come SPICE(Software Process Improvement and Capability Determination, ovvero miglioramento di processi software e determinazione di capacità) è un insieme di documenti di standard tecnici per lo sviluppo software.
Questo documento viene utilizzato nel perseguimento della qualità di processo in quanto stabilisce una struttura per la definizione degli obiettivi per il miglioramento dei processi stessi.
Lo standard dichiara che ogni processo deve essere sottoposto ad un controllo continuo, ripetibile e quantificabile al fine di individuare i punti critici e misurare i miglioramenti.

\immagine{SPY}{Software Process Assessment and Improvement}

Secondo SPICE, un processo può essere classificato in base al suo livello di maturità, in una \gloss{scala} da 1 a 6, con annessi i livelli di capacità ad ogni livello:
\begin{itemize}
\item \grassetto{Incomplete:} I risultati del processo non esistono o non sono appropriati;
\item \grassetto{Performed:} Si ottengono dei risultati, ma in un modo non specificato o non prevedibile;
\begin{itemize}
\item \emph{Process Performance}: capacità del processo di produrre degli output da dagli \gloss{input}.
\end{itemize}
\item  \grassetto{Managed:} l'esecuzione è pianificata e tracciata, il prodotto è conforme a standard e requisiti specifici;
\begin{itemize}
\item \emph{Performance Management:} capacità del processo di produrre un output coerente con gli obiettivi del processo;
\item \emph{Work Product Management:} capacità del processo di creare un risultato documentato, controllato e verificato.
\end{itemize}
\item  \grassetto{Established:} il processo è eseguito e controllato riferendosi a dei buoni principi di ingegneria del software;
\begin{itemize}
\item \emph{Process Definition}: il processo fa riferimento a degli standard di processo per definire i risultati attesi;
\item \emph{Process \gloss{Deployment}}: capacità del processo di utilizzare risorse appropriate per il raggiungimento degli obiettivi.
\end{itemize}
\item  \grassetto{Predictable:}  il processo è eseguito consistentemente con dei limiti di controllo definiti, per raggiungere altrettanto definiti obiettivi di processo;
\begin{itemize}
\item \emph{Process Measurement}: capacità di definizione di obiettivi e metriche di prodotto e di processo, con cui garantire il raggiungimento di obiettivi aziendali;
\item \emph{Process Control}: capacità di controllo tramite metriche di progetto e prodotto definite, per puntare al miglioramento.
\end{itemize}
\item  \grassetto{Optimizing:} l'esecuzione del processo è ottimizzata per soddisfare bisogni correnti e futuri, e il processo soddisfa ripetibilmente i suoi obiettivi prefissati;
\begin{itemize}
\item \emph{Process Innovation}: capacità di gestione di eventuali cambiamenti nel prodotto in modo controllato ed efficace;
\item \emph{Continuous Optimization}: capacità di identificare e applicare modifiche atte al miglioramento dei processi aziendali.
\end{itemize}
\end{itemize}
Per finire, lo standard definisce 4 stadi di misurazione degli attributi di un processo, suddivisi in:
\begin{itemize}
\item \grassetto{N}, non adeguato o non posseduto;
\item \grassetto{P}, parzialmente posseduto;
\item \grassetto{L}, largamente posseduto;
\item \grassetto{F}, completamente posseduto.
\end{itemize}

\subsection{ISO/IEC 9126} %A.2

Con la sigla ISO/IEC 9126 si individua una serie di normative e linee guida, sviluppate dall'\gloss{ISO} (Organizzazione internazionale per la normazione) in collaborazione con l'\gloss{IEC} (Commissione Elettrotecnica Internazionale), preposte a descrivere un modello di qualità del software. Il modello propone un approccio alla qualità in modo tale che le società di software possano migliorare l'organizzazione e i processi e, quindi come conseguenza concreta, la qualità del prodotto sviluppato.

\immagine{ISOIEC}{Modello di qualità ISO/IEC 9126}

Il presente standard definisce 6 caratteristiche di qualità che ogni prodotto software deve perseguire, al fine di garantire la conformità agli standard con efficienza ed efficacia. Le caratteristiche delle qualità in uso esulano dal presente progetto didattico, in quanto non è prevista l'attività di manutenzione conseguente al rilascio del prodotto. Quindi ci soffermiamo all'analisi della qualità interna ed esterna dello standard ISO/IEC 9126.
Le caratteristiche che un prodotto deve avere sono le seguenti:

\begin{itemize}
\item \grassetto{Funzionalità}: capacità di un prodotto software di fornire funzioni che soddisfano esigenze stabilite, necessarie per operare sotto condizioni specifiche;
\begin{itemize}
\item \emph{Appropriatezza}: rappresenta la capacità del prodotto software di fornire un appropriato insieme di funzioni per gli specificati compiti ed obiettivi prefissati all'\gloss{utente};
\item \emph{Accuratezza}: la capacità del prodotto software di fornire i risultati concordati o precisi effetti richiesti;
\item \emph{Interoperabilità}: la capacità del prodotto software di interagire ed operare con uno o più sistemi specificati;
\item \emph{Conformità}: la capacità del prodotto software di aderire agli standard, convenzioni e regolamentazioni rilevanti al settore operativo a cui vengono applicati;
\item \emph{Sicurezza}: la capacità del prodotto software di proteggere informazioni e i dati negando in ogni modo che persone o sistemi non autorizzati possano accedervi o modificarli, e che a persone o sistemi effettivamente autorizzati non sia negato l'accesso ad essi.
\end{itemize}
\item \grassetto{Affidabilità}: capacità del prodotto software di mantenere uno specificato livello di prestazioni quando usato in date condizioni per un dato periodo;
\begin{itemize}
\item \emph{Maturità}: capacità di un prodotto software di evitare che si verificano errori, malfunzionamenti o siano prodotti risultati non corretti;
\item \emph{Tolleranza degli errori}: capacità di mantenere livelli predeterminati di prestazioni anche in presenza di malfunzionamenti o usi scorretti del prodotto;
\item \emph{Recuperabilità}: capacità di un prodotto di ripristinare il livello appropriato di prestazioni e di recupero delle informazioni rilevanti, in seguito a un malfunzionamento. A seguito di un errore, il software può risultare non accessibile per un determinato periodo di tempo, questo arco di tempo è valutato proprio dalla caratteristica di recuperabilità;
\item \emph{Aderenza}: capacità di aderire a standard, regole e convenzioni inerenti all'affidabilità.
\end{itemize}
\item \grassetto{Efficienza}: capacità di fornire appropriate prestazioni relativamente alla quantità di risorse usate;
\begin{itemize}
\item \emph{Comportamento rispetto al tempo}: capacità di fornire adeguati tempi di risposta, elaborazione e velocità di attraversamento, sotto condizioni determinate;
\item \emph{Utilizzo delle risorse}: capacità di utilizzo di quantità e tipo di risorse in maniera adeguata;
\item \emph{Conformità}: capacità di aderire a standard e specifiche sull'efficienza.
\end{itemize}
\item \grassetto{Usabilità}: capacità del prodotto software di essere capito, appreso, usato e benaccetto dall'utente, quando usato sotto condizioni specificate.
\begin{itemize}
\item \emph{Comprensibilità}: esprime la facilità di comprensione dei concetti del prodotto, mettendo in grado l'utente di comprendere se il software è appropriato;
\item \emph{Apprendibilità}: capacità di ridurre l'impegno richiesto agli utenti per imparare ad usare la sua applicazione;
\item \emph{Operabilità}: capacità di mettere in condizione gli utenti di farne uso per i propri scopi e controllarne l'uso;
\item \emph{Attrattiva}: capacità del software di essere piacevole per l'utente che ne fa uso;
\item \emph{Conformità}: capacità del software di aderire a standard o convenzioni relativi all'usabilità.
\end{itemize}
\item \grassetto{Manutenibilità}: capacità del software di essere modificato, includendo correzioni, miglioramenti o adattamenti;
\begin{itemize}
\item \emph{Analizzabilità}: rappresenta la facilità con la quale è possibile analizzare il codice per localizzare un errore nello stesso;
\item \emph{Modificabilità}: capacità del prodotto software di permettere l'implementazione di una specificata modifica (sostituzioni componenti);
\item \emph{Stabilità}: capacità del software di evitare effetti inaspettati derivanti da modifiche errate;
\item \emph{Testabilità}: capacità di essere facilmente testato per validare le modifiche apportate al software.
\end{itemize}
\item \grassetto{Portabilità}: capacità del software di essere trasportato da un ambiente di lavoro ad un altro. (Ambiente che può variare dall'hardware al sistema operativo);
\begin{itemize}
\item \emph{Adattabilità}: capacità del software di essere adattato per differenti ambienti operativi senza dover applicare modifiche diverse da quelle fornite per il software considerato;
\item \emph{Installabilità}: capacità del software di essere installato in uno specificato ambiente;
\item \emph{Conformità}: capacità del prodotto software di aderire a standard e convenzioni relative alla portabilità;
\item \emph{Sostituibilità}: capacità di essere utilizzato al posto di un altro software per svolgere gli stessi compiti nello stesso ambiente.
\end{itemize}
\end{itemize}

\subsubsection{Qualità esterne} %A.2.1

Le metriche esterne, specificate nella norma ISO/IEC 9126-2, misurano i comportamenti del software sulla base dei test, dall'operatività e dall'osservazione durante la sua esecuzione, in funzione degli obiettivi stabiliti in un contesto tecnico rilevante o di \gloss{business}.

\subsubsection{Qualità interne} %A.2.2

La qualità interna, più precisamente le metriche interne, è specificata nella norma ISO/IEC 9126-3 e si applica al software non eseguibile (ad esempio il codice sorgente) durante le fasi di progettazione e codifica. Le misure effettuate permettono di prevedere il livello di qualità esterna ed in uso del prodotto finale, poiché gli attributi interni influiscono su quelli esterni e quelli in uso. Le metriche interne permettono di individuare eventuali problemi che potrebbero influire sulla qualità finale del prodotto prima che sia realizzato il software eseguibile. Esistono metriche che possono simulare il comportamento del prodotto finale tramite simulazioni.

\subsection{Ciclo di Deming (ciclo PDCA)} %A.3

Il ciclo di Deming o Deming Cycle (ciclo di PDCA - plan–do–check–act) è un modello studiato per il miglioramento continuo della qualità in un'ottica a lungo raggio. Serve per promuovere una cultura della qualità che è tesa al miglioramento continuo dei processi e all'utilizzo ottimale delle risorse. Questo strumento parte dall'assunto che per il raggiungimento del massimo della qualità sia necessaria la costante interazione tra ricerca, progettazione, test, produzione e vendita. Per migliorare la qualità e soddisfare il cliente, le quattro fasi devono ruotare costantemente, tenendo come criterio principale la qualità.
La sequenza logica dei quattro punti ripetuti per un miglioramento continuo è la seguente:
\begin{itemize}
\item \grassetto{P} - Plan. Pianificazione.
\item \grassetto{D} - Do. Esecuzione del programma, dapprima in contesti circoscritti.
\item \grassetto{C} - Check. Test e controllo, studio e raccolta dei risultati e dei riscontri.
\item \grassetto{A} - Act. Azione per rendere definitivo e/o migliorare il processo.
\end{itemize}

\immagine{PDCA}{Ciclo PDCA}

\newpage
\section{Pianificazione dei test}
Di seguito verranno visualizzata delle tabelle, strutturate secondo la sezione 5.3.2 delle Norme di Progetto, che riportano tutti i test che si sono pianificati. \\
\subsection{Test di sistema}
Di seguito verrà mostrata una tabella che riporta tutti i test di sistema pianificati, associati ai requisiti descritti nel documento Analisi dei Requisiti.\\
I test sono da intendere solo per requisiti ai quali è stato ragionevole associare un test.
\subsubsection{Descrizione dei test di sistema}
\begin{center}
\begin{longtable}{|c|p{0.5\linewidth}|c|c|}
\toprule
\textbf{Test} & \textbf{Descrizione} & \textbf{Requisito} & \textbf{Stato}\\
\midrule
TS1 & Viene verificato che il sistema MaaP generi correttamente lo scheletro necessario & ROF1 & D.E.\\
\midrule
TS1.1 & Viene verificato che il sistema MaaP installi correttamente le librerie necessarie & ROF1.1 & D.E.\\
\midrule
TS1.2 & Viene verificato che il sistema MaaP generi correttamente i file necessari & ROF1.2 & D.E.\\
\midrule
TS1.3 & Viene verificato che il sistema MaaP generi correttamente le directory necessarie & ROF1.3 & D.E.\\
\midrule
TS1.4 & Viene verificato che il sottosistema di autenticazione sia installato e configurato correttamente & ROF1.4 & D.E.\\
\midrule
TS1.4.1 & Viene verificato che nel database degli utenti sia presente un profilo di amministrazione di default & ROF1.4.1 ROF6 & D.E.\\
\midrule
TS4 & Viene verificato che il sistema crei correttamente le pagine web partendo dal loro file di descrizione & ROF4 & D.E.\\
\midrule
TS5.1 & Viene verificato che la funzione di registrazione possa essere correttamente abilitata/disabilitata & RDF5.1 & D.E.\\
\midrule
TS5.4 & Viene verificato che il sistema possa utilizzare correttamente il database di analisi & ROF5.4 & D.E.\\
\midrule
TS5.5 & Viene verificato che la funzione di creazione indici possa essere correttamnte abilitata/disabilitata & ROF5.5 & D.E.\\
\midrule
TS7 & Viene verificato che il sistema consenta all'utente registrato di potersi autenticare & ROF 7.0 & D.E.\\
\midrule
TS8 & Viene verificato che il sistema consenta all'utente di potersi registrare & RDF 8.0 & D.E.\\
\midrule
TS9 & Viene verificato che il sistema consenta all'utente di recuperare la password  & ROF 9.0 & D.E\\
\midrule
TS10 & Viene verificato che il sistema apra e visualizzi correttamente le Collection e le Collection-Index & ROF10 & D.E.\\
\midrule
TS10.1 & Viene verificato che il sistema visualizzi correttamente le pagine di Document-Show & ROF10.1 & D.E.\\
\midrule
TS10.2.4 & Viene verificato che il sistema disconnetta correttamente un utente alla sua richiesta & ROF10.2.4 & D.E.\\
\midrule
TS10.3.1.1 & Viene verificato che un utente autenticato possa modificare i dati del suo profilo & ROF10.3.1.1 & D.E.\\
\midrule
TS10.3.1.2 & Viene verificato che le modifiche apportate al profilo di un utente business autenticaro siano consistenti & ROF10.3.1.2 & D.E.\\
\midrule
TS10.3.2 & Viene verificato che la creazione di un nuovo utente da parte di un utente business autenticato amministratore avvenga correttamente & ROF10.3.2 & D.E.\\
\midrule
TS10.3.3 & Viene verificata la corretta cancellazione di un utente da parte di un utente business autenticato amministratore & ROF10.3.3 & D.E.\\
\midrule
TS10.4 & Viene verificato che l'utente business autenticato amministratore possa eliminare correttamente un Document & ROF10.4 & D.E.\\
\midrule
TS10.5 & Viene verificato che l’utente business autenticato amministratore possa modificare correttamente un Document & ROF10.5 & D.E.\\
\midrule
TS10.6 & Viene verificata la corretta visualizzazione delle query più utilizzate & ROF10.6 & D.E.\\
\midrule
TS10.7 & Viene verificata la corretta creazione degli indici di analisi & ROF10.7 & D.E.\\
\midrule
TS17 & Viene verificato che le pagine web prodotte dal framework MaaP siano compatibili con la versione 30.0.x o superiore di Google Chrome & ROV17 & D.E.\\
\midrule
TS18 & Viene verificato che le pagine web prodotte dal framework MaaP siano compatibili con la versione 24.x o superiore di Firefox & ROV18 & D.E.\\
\midrule
TS19 & Viene verificato che il sistema accetti solo file di configurazione validi & ROV19 & D.E.\\
\midrule
TS26 & Viene verificato che il sistema di installazione del software funzioni correttamente & ROV26 & D.E.\\
\midrule
TS27 & Viene verificato che il deployment su Heroku avvenga con successo & ROV27 & D.E.\\
%inserire i test
\bottomrule
\caption{Tabella per test di sistema}
\label{tab:changelog}
\end{longtable}
\end{center}
\subsection{Test d'integrazione}
Di seguito verrà mostrata una tabella che riporta tutti i test d'integrazione pianificati, associati alle componenti descritte nella progettazione ad alto livello.\\
\subsubsection{Descrizione dei test d'integrazione}
\begin{center}
\begin{longtable}{|c|p{0.5\linewidth}|c|c|}
\toprule
\textbf{Test} & \textbf{Descrizione} & \textbf{Componente} & \textbf{Stato}\\
\midrule
TI.MaaP & Test integrazione finale client-server & MaaP & D.E\\
\midrule
TI.Server & Test integrazione finale server & Server & D.E\\
\midrule
TI.Model & Test di funzionalità recupero e salvataggio dati & Server.Model & D.E\\
\midrule
TI.DSL & Verifica che le operazioni di elaborazione del DSL vengano eseguite correttamente /* interpretazione, caricamento roots, collections n shit*/ & DSL & D.E\\
\midrule
TI.Databases & Verifica che tutte le operazioni di recupero e scrittura dati e di creazione indici avvengano correttamente & Databases & D.E\\
\midrule
TI.DBAnalysis & Verifica che le operazioni di recupero e scrittura dati sul database di analisi avvengano correttamente & DBAnalysis & D.E\\
\midrule
TI.DBUser & Verifica che le operazioni di recupero e scrittura dati sul database utenti avvengano correttamente & DBUser & D.E\\
\midrule
TI.Client & Test integrazione finale client & Client & D.E\\
%inserire i test
\bottomrule
\caption{Tabella per test d'integrazione}
\label{tab:changelog}
\end{longtable}
\end{center}
\subsection{Tracciamento}
Di seguito verranno riportati in forma tabellare, descritta nella sezione 5.3.2 delle Norme di Progetto, i tracciamenti componente-test d'integrazione e test d'integrazione-componente
\subsubsection{Tracciamento componente-test d'integrazione}
\begin{center}
\begin{longtable}{|c|c|}
\toprule
\textbf{Componente} & \textbf{Test}\\
%inserire i test
\bottomrule
\caption{Tabella tracciamento componente-test d'integrazione}
\label{tab:changelog}
\end{longtable}
\end{center}
\subsubsection{Tracciamento test d'integrazione-componente}
\begin{center}
\begin{longtable}{|c|c|}
\toprule
\textbf{Test} & \textbf{Componente}\\
%inserire i test
\bottomrule
\caption{Tabella tracciamento test d'integrazione-componenti}
\label{tab:changelog}
\end{longtable}
\end{center}


\newpage
\section{Resoconto delle attività di verifica}
\subsection{Tracciamento componenti requisiti}
\subsection{Riassunto delle attività di verifica}
In questa sezione sono descritti i resoconti delle attività di verifica effettuate sui documenti prima di ciascuna revisione.
\subsubsection{Revisione dei Requisiti}
Nel periodo precedente a questa revisione i documenti sono stati controllati dai verificatori seguendo le Norme di Progetto nelle sezione 6.4.1 e 6.4.2; è stata applicata l'analisi statica descritta nella sezione 2.8.1 di questo documento.
Inizialmente è stata applicata la tecnica di Walkthrough, dove sono scovati e successivamente corretti gli errori; ogni volta che si trovava un errore, esso veniva messo nell'apposta lista che serve per l'inspection.
Dopo il walkthrough è stata applicata la tecnica di Inspection, utilizzando l'apposita lista, disponible in appendice delle Norme di Progetto. Inoltre per questo documento sono state calcolate le metriche descritte nella sezione 2.9.2 del documento corrente.
Per quanto riguarda i processi, essi sono stati controllati e verificati secondo le metodologie descritte nelle Norme di Progetto in sezione ???. Sono state calcolate le metriche per i processi descritti in sezione 2.9.1 di questo documento, e riportati i corrispondenti valori di BV e SV in forma tabellare.
\subsubsection{Revisione di Progettazione}
Nel periodo precedente a questa revisione i documenti sono stati controllati dai verificatori seguendo le Norme di Progetto nelle sezione 6.4.1 e 6.4.2; è stata applicata l'analisi statica descritta nella sezione 2.8.1 di questo documento.
Inizialmente è stata applicata la tecnica di Walkthrough, dove sono scovati e successivamente corretti gli errori; ogni volta che si trovava un errore, esso veniva messo nell'apposta lista che serve per l'inspection.
Dopo il walkthrough è stata applicata la tecnica di Inspection, utilizzando l'apposita lista, disponible in appendice delle Norme di Progetto; è stata posta particolare attenzione al documento Specifica Tecnica. Inoltre per questo documento sono state calcolate le metriche descritte nella sezione 2.9.2 del documento corrente.
Per quanto riguarda i processi, essi sono stati controllati e verificati secondo le metodologie descritte nelle Norme di Progetto in sezione ???. Sono state calcolate le metriche per i processi descritti in sezione 2.9.1 di questo documento, e riportati i corrispondenti valori di BV e SV in forma tabellare.
\subsection{Dettaglio delle verifiche tramite analisi}
\subsubsection{Analisi dei Requisiti}
\paragraph{Processi}
Di seguito vengono riportati i valori degli indici SV e BV calcolati durante il periodo di tempo dedicato all'Analisi dei Requisiti.
\begin{longtable}{|c|p{3cm}|p{3cm}|}
\toprule
\textbf{Attività} & \textbf{SV} & \textbf{BV} \\

%aggiungere qui una midrule per aggiungere una nuova riga alla tabella

\midrule
\emph{Studio Fattibilità} & 0 & 0 \\
\midrule
\emph{Analisi dei Requisiti} & +50 & +50\\
\midrule
\emph{Glossario} & 0  & 0\\
\midrule
\emph{Norme di Progetto} & 0 & 0\\
\midrule
\emph{Piano di Progetto} & 0 & \\
\midrule
\emph{Piano di Qualifica} & -15 & -15\\
\bottomrule
\caption{Esiti dell'indice di Gulpease calcolato sui documenti durante l'Analisi}
\label{tab:changelog}
\end{longtable}
In questa tabella, i valori positivi indicano un costo eccedente, viceversa i valori negativi mostrano un costo risparmiato.
I valori indicati in tabella sono espressi in euro.
Non avendo previsto degli intervalli di tempo libero tra un'attività e la successiva, abbiamo ottenuto degli SV positivi in Analisi dei Requisiti e negativi in Piano di Qualifica.
Questa è stata una mancanza da parte del team, che vedrà di migliorarsi nelle prossime fasi e di adottare una tattica di pianificazione più flessibile.
I costi aggiuntivi sono comunque in linea con i nostri obiettivi.
\paragraph{Documenti}
Di seguito vengono riportati, per ogni documento, i valori dell'indice di Gulpease calcolati durante il periodo di tempo dedicato all'Analisi dei Requisiti. Un documento è valido solo se rispecchia i range in sezione 2.9.2.1.
\begin{longtable}{|c|p{3cm}|p{3cm}|}
\toprule
\textbf{Documento} & \textbf{Valore indice} & \textbf{Esito} \\

%aggiungere qui una midrule per aggiungere una nuova riga alla tabella

\midrule
\emph{Studio Fattibilità v1.2.0} & 46 & Sufficiente\\
\midrule
\emph{Analisi dei Requisiti v1.2.0} & 52& Superato\\
\midrule
\emph{Glossario v1.2.0} & 46 & Sufficiente\\
\midrule
\emph{Norme di Progetto v1.2.0} & 52 & Superato\\
\midrule
\emph{Piano di Progetto v1.2.0} & 50 & Superato\\
\midrule
\emph{Piano di Qualifica v1.2.0} & 47 & Sufficiente\\
\bottomrule
\caption{Esiti dell'indice di Gulpease calcolato sui documenti durante l'Analisi}
\label{tab:changelog}
\end{longtable}
\subsubsection{Analisi in Dettaglio}
\paragraph{Processi}
Di seguito vengono riportati i valori degli indici SV e BV calcolati durante il periodo di tempo dedicato all'Analisi in Dettaglio.
\paragraph{Documenti}
Di seguito vengono riportati, per ogni documeno, i valori dell'indice di Gulpease calcolati durante il periodo di tempo dedicato all'Analisi in Dettaglio.

\begin{longtable}{|c|p{3cm}|p{3cm}|}
\toprule
\textbf{Documento} & \textbf{Valore indice} & \textbf{Esito} \\

%aggiungere qui una midrule per aggiungere una nuova riga alla tabella

\midrule
\emph{Studio Fattibilità v1.2.0} &  & \\
\midrule
\emph{Analisi dei Requisiti v1.2.0} & & \\
\midrule
\emph{Glossario v1.2.0} &  &\\
\midrule
\emph{Norme di Progetto v1.2.0} &  & \\
\midrule
\emph{Piano di Progetto v1.2.0} &  & \\
\midrule
\emph{Piano di Qualifica v1.2.0} &  & \\
\bottomrule
\caption{Esiti dell'indice di Gulpease calcolato sui documenti durante l'Analisi}
\label{tab:changelog}
\end{longtable}

\subsection{Dettaglio dell'esito delle revisioni}
Per ciascuna revisione alla quale si intende partecipare, il committente avrà il compito di segnalare eventuali problematiche trovate, dando una valutazione globale dell'andamento del progetto e una descrizione per ciascun documento con correzioni e accorgimenti da apportare.
Di seguito vengono elencate le modifiche apportate ai documenti, come suggerito dal committente, per ciascuna revisione.
\subsubsection{Revisione dei Requisiti}
\begin{itemize}
\item \grassetto{Studio di Fattibilità:} il documento è stato valutato bene, quindi non ci sono stati accorgimenti da apportate;
\item \grassetto{Norme di Progetto:}
\item \grassetto{Analisi dei Requisiti:}
\item \grassetto{Piano di Progetto:}
\item \grassetto{Piano di Qualifica:}
\item \grassetto{Glossario:}
\end{itemize}








%FINE DOCUMENTO NON CANCELLARE
\end{document}
