\newpage

\section{Pianificazione dei test}
Di seguito verranno visualizzate delle tabelle, strutturate secondo la sezione 5.3.2 delle Norme di Progetto, che riportano tutti i test che si sono pianificati. \\
\subsection{Test di sistema}
Di seguito verrà mostrata la tabella che riporta tutti i test di sistema pianificati, associati ai requisiti descritti nel documento Analisi dei Requisiti.\\
I test sono da intendere solo per requisiti ai quali è stato ragionevole associare un test.
\subsubsection{Descrizione dei test di sistema}
\begin{center}
\begin{longtable}{|p{2cm}|p{7cm}|p{2cm}|p{2cm}|}
\toprule
\textbf{Test} & \textbf{Descrizione} & \textbf{\gloss{Requisito}} & \textbf{Stato}\\
\midrule
TS1 & Viene verificato che il sistema \gloss{MaaP} generi correttamente lo scheletro necessario & ROF1 & D.E.\\
\midrule
TS1.1 & Viene verificato che il sistema MaaP installi correttamente le librerie necessarie & ROF1.1 & D.E.\\
\midrule
TS1.2 & Viene verificato che il sistema MaaP generi correttamente i file necessari & ROF1.2 & D.E.\\
\midrule
TS1.3 & Viene verificato che il sistema MaaP generi correttamente le directory necessarie & ROF1.3 & D.E.\\
\midrule
TS1.4 & Viene verificato che il \gloss{sottosistema} di autenticazione sia installato e configurato correttamente & ROF1.4 & D.E.\\
\midrule
TS1.4.1 & Viene verificato che nel \gloss{database} degli utenti sia presente un \gloss{profilo} di amministrazione di \gloss{default} & ROF1.4.1 ROF6 & D.E.\\
\midrule
TS1.5 & Viene verificato che il sistema sia in grado di eliminare un progetto esistente & ROF1.5 & D.E.\\
\midrule
TS1.6 & Viene verificato che il sistema sia in grado di clonare un progetto esistente & ROF1.6 & D.E.\\
\midrule
TS4 & Viene verificato che il sistema crei correttamente le \gloss{pagine \gloss{web}} partendo dal loro \gloss{file di descrizione} & ROF4 & D.E.\\
\midrule
TS5.1 & Viene verificato che la funzione di \gloss{registrazione} possa essere correttamente abilitata/disabilitata & RDF5.1 & D.E.\\
\midrule
TS5.4 & Viene verificato che il sistema possa utilizzare correttamente il database di analisi & ROF5.4 & D.E.\\
\midrule
TS5.5 & Viene verificato che la funzione di creazione indici possa essere correttamnte abilitata/disabilitata & ROF5.5 & D.E.\\
\midrule
TS7 & Viene verificato che il sistema consenta all'utente registrato di potersi autenticare & ROF 7.0 & D.E.\\
\midrule
TS8 & Viene verificato che il sistema consenta all'utente di potersi registrare & RDF 8.0 & D.E.\\
\midrule
TS9 & Viene verificato che il sistema consenta all'utente di recuperare la \gloss{password}  & ROF 9.0 & D.E\\
\midrule
TS10 & Viene verificato che il sistema apra e visualizzi correttamente le \gloss{Collection} e le \gloss{Collection-Index} & ROF10 & D.E.\\
\midrule
TS10.1 & Viene verificato che il sistema visualizzi correttamente le pagine \gloss{Document-Show} & ROF10.1 & D.E.\\
\midrule
TS10.2.4 & Viene verificato che il sistema disconnetta correttamente un utente alla sua richiesta & ROF10.2.4 & D.E.\\
\midrule
TS10.3.1.1 & Viene verificato che un utente autenticato possa modificare i dati del suo profilo & ROF10.3.1.1 & D.E.\\
\midrule
TS10.3.1.2 & Viene verificato che le modifiche apportate al profilo di un \gloss{utente business} autenticaro siano consistenti & ROF10.3.1.2 & D.E.\\
\midrule
TS10.3.2 & Viene verificato che la creazione di un nuovo utente da parte di un \gloss{utente business autenticato} \gloss{amministratore} avvenga correttamente & ROF10.3.2 & D.E.\\
\midrule
TS10.3.3 & Viene verificata la corretta cancellazione di un utente da parte di un \gloss{utente business autenticato amministratore} & ROF10.3.3 & D.E.\\
\midrule
TS10.4 & Viene verificato che l'utente business autenticato amministratore possa eliminare correttamente un \gloss{Document} & ROF10.4 & D.E.\\
\midrule
TS10.5 & Viene verificato che l’utente business autenticato amministratore possa modificare correttamente un Document & ROF10.5 & D.E.\\
\midrule
TS10.6 & Viene verificata la corretta visualizzazione delle \gloss{query} più utilizzate & ROF10.6 & D.E.\\
\midrule
TS10.7.1.2 & Viene verificata la corretta creazione degli indici di analisi & ROF10.7.1.2 & D.E.\\
\midrule
TS10.7.2 & Viene verificata la corretta eliminazione degli indici di analisi & ROF10.7.2 & D.E.\\
\midrule
TS10.7.3 & Viene verificata la corretta visualizzazione degli indici di analisi & ROF10.7.3 & D.E.\\
\midrule
TS17 & Viene verificato che le pagine web prodotte dal \gloss{framework} MaaP siano compatibili con la versione 30.0.x o superiore di \gloss{Google} Chrome & ROV17 & D.E.\\
\midrule
TS18 & Viene verificato che le pagine web prodotte dal framework MaaP siano compatibili con la versione 24.x o superiore di \gloss{Firefox} & ROV18 & D.E.\\
\midrule
TS19 & Viene verificato che il sistema accetti solo \gloss{file di configurazione} validi & ROV19 & D.E.\\
\midrule
TS26 & Viene verificato che il sistema di installazione del software funzioni correttamente & ROV26 & D.E.\\
\midrule
TS27 & Viene verificato che il deployment su \gloss{Heroku} avvenga con successo & ROV27 & D.E.\\
%inserire i test
\bottomrule
\caption{Test di sistema}
\label{tab:changelog}
\end{longtable}
\end{center}
\subsection{Test d'integrazione}

\subsubsection{Diagramma d'integrazione}
\immagine{./integrazione}{Diagramma d'integrazione}

Di seguito verrà mostrata la tabella che riporta tutti i test d'integrazione pianificati, associati alle componenti descritte nella progettazione ad alto livello.\\
\subsubsection{Descrizione dei test d'integrazione}
\begin{center}
\begin{longtable}{|p{5cm}|p{5cm}|p{5cm}|p{1cm}|}
\toprule
\textbf{Test} & \textbf{Descrizione} & \textbf{Componente} & \textbf{Stato}\\
\midrule
TI.MaaP & Test di integrazione finale tra \gloss{client} e \gloss{server} & MaaP & D.E.\\
\midrule
TI.Server & Verifica il corretto funzionamento delle componenti del server, quindi la corretta integrazione
tra ModelServer e \gloss{Controller} del server. In particolare, viene verificato che le funzionalit\`{a} presenti nel
Controller debbano agire consistentemente sui dati del ModelServer. & Server & D.E.\\
\midrule
TI.Server.ModelServer & Verifica che tutte le operazioni di lettura, scrittura e interpretazione dei dati avvengano
correttamente. & Server::ModelServer & D.E.\\
\midrule
TI.Server.ModelServer. DataManager-Database & Verifica che i gestori dei dati siano correttamente integrati ai database di riferimento. & Server::ModelServer::DataManager & D.E.\\
& & Server::ModelServer::Database & \\
\midrule
TI.Server.ModelServer. DataManager & Verifica che le operazioni di recupero, gestione e scrittura dati nei database avvengano correttamente. & Server::ModelServer::DataManager & D.E.\\
\midrule
TI.Client & Verifica che i dati ottenuti e forniti da e verso il client siano corretti e correttamente gestiti all'interno del client stesso. & Client & D.E.\\
\midrule
TI.Client.View-ControllerModelView & Verifica che i dati uscenti ed entranti la \gloss{View} siano gestiti correttamente dal ControllerModelView & Client::View & D.E.\\
& & Client::ControllerModelView & \\
\midrule
TI.Client.ControllerModelView-ModelClient & Viene verificato che le funzionalit\`{a} del ControllerModelView agiscano correttamente sul ModelClient & Client::ControllerModelView & D.E.\\
& & Client::ModelClient & \\
\midrule
TI.Client.ControllerModelView & Viene verificato che le funzioni del ClientController modifichino correttamente i dati presenti nello \gloss{Scope} & Client::ControllerModelView & D.E.\\
%inserire i test
\bottomrule
\caption{Test d'integrazione}
\label{tab:changelog}
\end{longtable}
\end{center}
\subsubsection{Tracciamento}
Di seguito verranno riportati in forma tabellare i tracciamenti componente-test d'integrazione e test d'integrazione-componente.\\
\paragraph{Tracciamento componente-test d'integrazione}
\begin{center}
\begin{longtable}{|p{7cm}|p{7cm}|}
\toprule
\textbf{Componente} & \textbf{Test}\\
\midrule
MaaP & TI.MaaP\\
\midrule
Server & TI.Server\\
\midrule
Server::ModelServer & TI.Server.ModelServer\\
\midrule
Server::ModelServer::DataManager & TI.Server.ModelServer.DataManager TI.Server.ModelServer.DataManager-Database\\
\midrule
Server::ModelServer::Database & TI.Server.ModelServer.DataManager-Database\\
\midrule
Client & TI.Client\\
\midrule
Client::View & TI.Client.View-ControllerModelView\\
\midrule
Client::ControllerModelView & TI.Client.ControllerModelView TI.Client.View-ControllerModelView TI.Client.ControllerModelView-ModelClient\\
\midrule
Client::ModelClient & TI.Client.ControllerModelView-ModelClient\\
%inserire i test
\bottomrule
\caption{Tracciamento componente-test d'integrazione}
\label{tab:changelog}
\end{longtable}
\end{center}
\vspace{5cm}
\paragraph{Tracciamento test d'integrazione-componente}
\begin{center}
\begin{longtable}{|p{7cm}|p{7cm}|}
\toprule
\textbf{Test} & \textbf{Componente}\\
\midrule
TI.MaaP & MaaP\\
\midrule
TI.Server & Server\\
\midrule
TI.Server.ModelServer & Server::ModelServer\\
\midrule
TI.Server.ModelServer.DataManager-Database & Server::ModelServer::DataManager\\
& Server::ModelServer::Database\\
\midrule
TI.Server.ModelServer.DataManager & Server::ModelServer::DataManager\\
\midrule
TI.Client & Client\\
\midrule
TI.Client.View-ControllerModelView & Client::View\\
& Client::ControllerModelView\\
\midrule
TI.Client.ControllerModelView-ModelClient & Client::ControllerModelView\\
& Client::ModelClient\\
\midrule
TI.Client.ControllerModelView & Client::ControllerModelView\\
%inserire i test
\bottomrule
\caption{Tracciamento test d'integrazione-componente}
\label{tab:changelog}
\end{longtable}
\end{center}

\subsection{Test di validazione}
In questa sezione vengono descritti i test di validazione, utili ad accertarsi che il prodotto realizzato sia conforme alle attese.
Per ogni test viene descritta la serie di passi che un utente deve eseguire per validare la conformità ai requisiti ad esso associati.

\subsubsection{TV 1 - Lato Utente Business}

L’utente vuole verificare le funzionalità di visualizzazione di Collection e Document, per tanto gli è richiesto di:


\begin{enumerate}
\item Cliccare sul pulsante Collection;
\item Selezionare la Collection da visualizzare dall'apposito elenco;
\begin{enumerate}
\item Scorrere le pagine della Collection fino a trovare il document ricercato;
\item Usare un \gloss{filtro} per ricercare un Document;
\item Annullare il filtro applicato;
\item Ordinare una \gloss{chiave}, se abilitata, in ordine crescente o decrescente;
\item Impostare un numero massimo di Document da visualizzare.
\end{enumerate}
\item Selezionare il Document ricercato cliccando sulla chiave selezionabile;
\item Verificare che il Document venga visualizzato correttamente.
\end{enumerate}

\subsubsection{TV 2 - Lato Utente Business}
L’utente vuole verificare il corretto funzionamento delle funzionalità di registrazione, autenticazione e gestione profilo.
Si dovrà quindi seguire la seguente \gloss{procedura} di validazione:

\begin{enumerate}
\item Verificare di potersi registrare;
\begin{enumerate}
\item Cliccare sul pulsante di registrazione;
\item Compilare il \gloss{form} di registrazione con dati validi;
\item Confermare la registrazione mediante il \gloss{link} contenuto nella mail inviata automaticamente dal sistema.
\end{enumerate}
\item Verificare di poter recuperare la password in caso di dimenticanza;
\begin{enumerate}
\item Cliccare sul pulsante di recupero password;
\item Inserire la \gloss{email} con cui si è registrati al sistema;
\item Verificare che il sistema invii correttamente la mail per il recupero della password;
\item Verificare che la nuova password fornita nella mail consenta l'accesso al sistema.
\end{enumerate}
\item Verificare di poter effettuare l'autenticazione;
\begin{enumerate}
\item Cliccare sul pulsante di autenticazione;
\item Inserire le credenziali precedentemente registrate;
\item Confermare l'autenticazione.
\end{enumerate}
\item Verificare che il sistema effettui correttamente l' autenticazione;
\item Verificare di poter modificare i propri dati utente;
\begin{enumerate}
\item Entrare nel proprio profilo utente;
\item Cliccare sul pulsante di modifica;
\item Verificare di poter modificare una o più chiavi del profilo;
\item Verificare di poter salvare le modifiche;
\item Verificare di poter annullare le modifiche.
\end{enumerate}
\item Verificare di potersi disconnettere correttamente dal sistema;
\begin{enumerate}
\item Cliccare sul pulsante di disconnessione;
\item Verificare di non poter più accedere a pagine riservate ad utenti autenticati.
\end{enumerate}
\end{enumerate}

\subsubsection{TV 3 - Lato Utente Business Amministratore}

L’utente amministratore vuole verificare il corretto funzionamento della gestione dei profili utenti.
Si dovrà quindi seguire la seguente procedura:

\begin{enumerate}
\item Effettuare l'autenticazione con le proprie credenziali;
\begin{enumerate}
\item Al primo accesso assoluto al sistema, utilizzare le credenziali di default fornite dal sistema stesso.
\end{enumerate}
\item Cliccare sul pulsante di gestione profili utenti;
\item Selezionare il profilo utente da modificare;
\begin{enumerate}
\item Scorrere le pagine della Collection fino a trovare il profilo ricercato;
\item Usare un filtro per ricercare un profilo;
\item Annullare il filtro applicato;
\item Ordinare una chiave, se abilitata, in ordine crescente o decrescente;
\item Impostare un numero massimo di profili da visualizzare.
\end{enumerate}
\item Verificare di poter modificare le informazioni del profilo utente;
\begin{enumerate}
\item Cliccare sul pulsante di modifica;
\item Verificare di poter modificare una o più chiavi del profilo;
\item Verificare di poter salvare le modifiche;
\item Verificare di poter annullare le modifiche.
\end{enumerate}
\item Verificare di poter modificare i permessi associati al profilo utente;
\begin{enumerate}
\item Cliccare sul pulsante di modifica;
\item Verificare di poter modificare i permessi disponibili;
\item Verificare di poter salvare le modifiche;
\item Verificare di poter annullare le modifiche.
\end{enumerate}
\item Verificare di poter eliminare correttamente un utente;
\begin{enumerate}
\item Entrare nel profilo utente da eliminare;
\item Cliccare sul pulsante di eliminazione;
\item Confermare l'eliminazione;
\item Verificare che le credenziali dell'utente cancellato non siano più presenti nel sistema.
\end{enumerate}
\item Verificare di poter creare un nuovo profilo utente;
\begin{enumerate}
\item Cliccare sul pulsante di creazione;
\item Completare il form di registrazione con informazioni valide;
\item Verificare che il profilo sia stato creato correttamente.
\end{enumerate}
\end{enumerate}


\subsubsection{TV 4 - Lato Utente Business Amministratore}

L’utente amministratore vuole verificare il corretto funzionamento della gestione dei Document.
Dovrà quindi seguire la seguente procedura:


\begin{enumerate}
\item Dopo essere entrato in una pagina Collection, selezionare un Document;
\item Verificare di poter modificare un Document;
\begin{enumerate}
\item Cliccare sul pulsante di modifica;
\item Verificare di poter modificare le chiavi disponibili;
\item Verificare di poter salvare le modifiche;
\item Verificare di poter annullare le modifiche.
\end{enumerate}
\item Verificare di poter cancellare un Document;
\begin{enumerate}
\item Selezionare il pulsante di cancellazione;
\item Verificare che il Document non sia più presente nel sistema.
\end{enumerate}
\end{enumerate}

\subsubsection{TV 5 - Lato Utente Business Amministratore}

L’utente amministratore vuole verificare il corretto funzionamento della gestione degli indici.
Dovrà quindi seguire la seguente procedura:

\begin{enumerate}
\item Entrare nella gestione degli indici;
\item Verificare di poter creare un indice;
\begin{enumerate}
\item Cliccare sul pulsante di creazione indici;
\item Selezionare una tra le query più utilizzate;
\item Confermare la creazione;
\item Verificar che l'indice sia stato creato correttamente.
\end{enumerate}
\item Verificare di poter eliminare un indice;
\begin{enumerate}
\item Selezionare un indice presente;
\item Cliccare sul pulsante di eliminazione;
\item Verificare che l'indice non sia più presente.
\end{enumerate}
\end{enumerate}

\subsubsection{TV 6 - Lato Utente Sviluppatore}

L’utente sviluppatore vuole verificare il corretto funzionamento della gestione ed interpretazione del \gloss{DSL} per la creazione di una pagina Collection-Index.
Dovrà quindi seguire la seguente procedura:

\begin{enumerate}
\item Scrivere un file DSL valido;
\begin{enumerate}
\item Specificare le chiavi da visualizzare;
\item Specificare la provenienza dei dati;
\begin{enumerate}
\item Provenienti da query standard;
\item Provenienti da query personalizzate;
\item Provenienti da funzioni personalizzate;
\item Provenienti da riferimenti esterni.
\end{enumerate}
\item Specificare eventuali etichette;
\item Specificare eventuali chiavi ordinabili;
\item Specificare il numero massimo di Document per pagina;
\item Specificare il nome della Collection;
\item Specificare la posizione della Collection nel menù.
\end{enumerate}
\item Caricare il file DSL nell'apposita cartella;
\item Avviare il server MaaP;
\item Effettuare l'autenticazione;
\item Entrare nella Collection la cui pagina è descritta dal file DSL inserito e verificare la bontà della sua interpretazione.
\end{enumerate}


\subsubsection{TV 7 - Lato Utente Sviluppatore}

L’utente sviluppatore vuole verificare il corretto funzionamento della gestione ed interpretazione del DSL per la creazione di una pagina Document-Show.
Dovrà quindi seguire la seguente procedura:

\begin{enumerate}
\item Scrivere un file DSL valido;
\begin{enumerate}
\item Specificare le chiavi da visualizzare;
\item Specificare la provenienza dei dati;
\begin{enumerate}
\item Provenienti da query standard;
\item Provenienti da query personalizzate;
\item Provenienti da funzioni personalizzate;
\item Provenienti da riferimenti esterni.
\end{enumerate}
\item Specificare eventuali etichette per le chiavi;
\end{enumerate}
\item Caricare il file DSL nell'apposita cartella;
\item Avviare il server MaaP;
\item Effettuare l'autenticazione;
\item Entrare nel document la cui pagina è descritta dal file DSL inserito e verificare la bontà della sua interpretazione.
\end{enumerate}


\subsubsection{TV 8 - Lato Utente Sviluppatore}

L’utente sviluppatore vuole verificare il corretto funzionamento delle funzioni di gestione del progetto.
Dovrà quindi seguire la seguente procedura:

\begin{enumerate}
\item Creare un nuovo progetto;
\item Verificare che il nuovo progetto sia disponibile;
\item Clonare un progetto;
\item Verificare che il progetto clonato sia identico all’originale;
\item Verificare di poter gestire un progetto;
\begin{enumerate}
\item Verificare di poter avviare, fermare o riavviare il server MaaP;
\item Verificare di poter modificare i file di configurazione del progetto;
\begin{enumerate}
\item Abilitare/disabilitare la registrazione;
\item Abilitare/disabilitare la creazione di indici;
\item Impostare la connessione al database di analisi.
\end{enumerate}
\item Verificare di poter modificare o caricare nuovi file DSL;
\item Verificare di poter modificare i \gloss{template} disponibili.
\end{enumerate}
\item Eliminare un progetto;
\item Verificare che il progetto eliminato non sia più disponibile.
\end{enumerate}

\subsubsection{Tracciamento}
Di seguito verranno riportati in forma tabellare i tracciamenti test di validazione - requisiti.\\
\paragraph{Tracciamento Test di Validazione - Requisiti}
\begin{center}
\begin{longtable}{|p{7cm}|p{7cm}|}
\toprule
\textbf{Test} & \textbf{Requisiti}\\
\midrule
TV 1 & ROF10\\ & ROF10.1\\ & RDF10.2\\ & RDF10.2.1\\ &  RDF10.2.1.1\\ &  RDF10.2.1.2\\ &  RDF10.2.2 \\ & RDF10.2.3\\ &   ROF10.2.5\\
\midrule
TV 2 & ROF7\\ &  ROF7.1\\ &  ROF7.2\\ &  ROF7.2.1\\ &  RDF8\\ &  RDF8.1\\ &  RDF8.2\\ &  RDF8.2.1\\ &  ROF9\\ &  ROF10.2.4\\ &  ROF10.3\\ &  ROF10.3.1\\ &  ROF10.3.1.1\\ &  ROF10.3.1.2\\ &  ROF10.3.1.3\\
\midrule
TV 3 & ROF6\\ &  ROF10.3.1.4\\ &  ROF10.3.1.5\\ &  ROF10.3.2\\ &  ROF10.3.3\\
\midrule
TV 4 &  ROF10.4\\ &  ROF10.5\\ &  ROF10.5.1\\ &  ROF10.5.2\\ &  ROF10.5.3\\
\midrule
TV 5 &  ROF10.6\\ &  ROF10.7\\ &  ROF10.7.1\\ &  ROF10.7.1.1\\ &  ROF10.7.1.2\\ &  ROF10.7.2\\ &  ROF10.7.2.1\\ &  ROF10.7.2.2\\ &  ROF10.7.3\\
\midrule
TV 6 & ROF3\\ &  ROF4\\ &  ROF4.1\\ &  ROF4.1.2\\ &  ROF4.1.2.1\\ &  ROF4.1.2.1.1\\ &  ROF4.1.2.1.2\\ &  ROF4.1.2.1.3\\ &  ROF4.1.2.1.4\\ &  ROF4.1.2.1.5\\ &  ROF4.1.2.2\\ &  ROF4.1.2.3\\ &  ROF4.2\\ &  ROF4.2.1\\ &  ROF4.2.1.1\\ &  ROF4.2.1.2\\ &  ROF4.2.2\\ &  ROF4.2.2.1\\ &  ROF4.2.2.2\\ &  ROF4.2.2.3\\ &  ROF4.2.2.4\\ &  ROF4.2.2.5\\ &  ROF4.2.2.6\\ &  ROF4.2.2.7\\ &  ROF4.3\\ &  ROF4.4\\
\midrule
TV 7 & ROF4.1.3\\ &  ROF4.1.3.1\\ &  ROF4.1.3.1.1\\ &  ROF4.1.3.1.2\\ &  ROF4.1.3.1.3\\ &  ROF4.1.3.1.4\\ &  ROF4.1.3.1.5\\ &  ROF4.2.3\\ &  ROF4.2.3.1\\ &  ROF4.2.3.2\\ &  ROF4.3\\ &  ROF4.4\\
\midrule
TV 8 & ROF5\\ &  ROF5.1\\ &  RDF5.3\\ &  ROF5.4\\ &  ROF5.5\\

%inserire i test
\bottomrule
\caption{Tracciamento Test di Validazione - Requisiti}
\label{tab:changelog}
\end{longtable}
\end{center}

\paragraph{Tracciamento Requisiti - Test di Validazione}
\begin{center}
\begin{longtable}{|p{7cm}|p{7cm}|}
\toprule
\textbf{Requisiti} & \textbf{Test}\\
\midrule
ROF3 & TV6\\
\midrule
ROF4 & TV6\\
\midrule
ROF4.1 & TV6\\
\midrule
ROF4.1.2 & TV6\\
\midrule
ROF4.1.2.1 & TV6\\
\midrule
ROF4.1.2.1.1 & TV6\\
\midrule
ROF4.1.2.1.2 & TV6\\
\midrule
ROF4.1.2.1.3 & TV6\\
\midrule
ROF4.1.2.1.4 & TV6\\
\midrule
ROF4.1.2.1.5 & TV6\\
\midrule
ROF4.1.2.2 & TV6\\
\midrule
ROF4.1.2.3 & TV6\\
\midrule
ROF4.1.3 & TV7\\
\midrule
ROF4.1.3.1 & TV7\\
\midrule
ROF4.1.3.1.1 & TV7\\
\midrule
ROF4.1.3.1.2 & TV7\\
\midrule
ROF4.1.3.1.3 & TV7\\
\midrule
ROF4.1.3.1.4 & TV7\\
\midrule
ROF4.1.3.1.5 & TV7\\
\midrule
ROF4.2 & TV6\\
\midrule
ROF4.2.1 & TV6\\
\midrule
ROF4.2.1.1 & TV6\\
\midrule
ROF4.2.1.2 & TV6\\
\midrule
ROF4.2.2 & TV6\\
\midrule
ROF4.2.2.1 & TV6\\
\midrule
ROF4.2.2.2 & TV6\\
\midrule
ROF4.2.2.3 & TV6\\
\midrule
ROF4.2.2.4 & TV6\\
\midrule
ROF4.2.2.5 & TV6\\
\midrule
ROF4.2.2.6 & TV6\\
\midrule
ROF4.2.2.7 & TV6\\
\midrule
ROF4.2.3 & TV7\\
\midrule
ROF4.2.3.1 & TV7\\
\midrule
ROF4.2.3.2 & TV7\\
\midrule
ROF4.3 & TV6\\ & TV7\\
\midrule
ROF4.4 & TV6\\ & TV7\\
\midrule
ROF5 & TV8\\
\midrule
ROF5.1 & TV8\\
\midrule
RDF5.3 & TV8\\
\midrule
ROF5.4 & TV8\\
\midrule
ROF5.5 & TV8\\
\midrule
ROF6 & TV3\\
\midrule
ROF7 & TV2\\
\midrule
ROF7.1 & TV2\\
\midrule
ROF7.2 & TV2\\
\midrule
ROF7.2.1 & TV2\\
\midrule
RDF8 & TV2\\
\midrule
RDF8.1 & TV2\\
\midrule
RDF8.2 & TV2\\
\midrule
RDF8.2.1 & TV2\\
\midrule
ROF9 & TV2\\
\midrule
ROF10 & TV1\\
\midrule
ROF10.1 & TV1\\
\midrule
RDF10.2 & TV1\\
\midrule
RDF10.2.1 & TV1\\
\midrule
RDF10.2.1.1 & TV1\\
\midrule
RDF10.2.1.2 & TV1\\
\midrule
RDF10.2.2 & TV1\\
\midrule
RDF10.2.3 & TV1\\
\midrule
ROF10.2.4 & TV2\\
\midrule
ROF10.2.5 & TV1\\
\midrule
ROF10.3 & TV2\\
\midrule
ROF10.3.1 & TV2\\
\midrule
ROF10.3.1.1 & TV2\\
\midrule
ROF10.3.1.2 & TV2\\
\midrule
ROF10.3.1.3 & TV2\\
\midrule
ROF10.3.1.4 & TV3\\
\midrule
ROF10.3.1.5 & TV3\\
\midrule
ROF10.3.2 & TV3\\
\midrule
ROF10.3.3 & TV3\\
\midrule
ROF10.4 & TV4\\
\midrule
ROF10.5 & TV4\\
\midrule
ROF10.5.1 & TV4\\
\midrule
ROF10.5.2 & TV4\\
\midrule
ROF10.5.3 & TV4\\
\midrule
ROF10.6 & TV5\\
\midrule
ROF10.7 & TV5\\
\midrule
ROF10.7.1 & TV5\\
\midrule
ROF10.7.1.1 & TV5\\
\midrule
ROF10.7.1.2 & TV5\\
\midrule
ROF10.7.2 & TV5\\
\midrule
ROF10.7.2.1 & TV5\\
\midrule
ROF10.7.2.2 & TV5\\
\midrule
ROF10.7.3 & TV5\\

%inserire i test
\bottomrule
\caption{Tracciamento Requisiti - Test di Validazione}
\label{tab:changelog}
\end{longtable}
\end{center}
