\UCtitle
{Caso d'uso UC2}
{MaaP's Web}
\UCimmagine{UC2}{UC2 - Azioni \gloss{utente business}}
\UC
{UC2}
{Utente business, \gloss{Utente business autenticato}}
{L'utente business attraverso il \gloss{browser} visualizza la pagina di autenticazione dove può accedere, recuperare la \gloss{password} o registrarsi, mentre l'utente business autenticato può aprire una Collection, disconnettersi o gestire il profilo}
{Il sistema ha creato lo scheletro del progetto, la pagina di autenticazione viene proposta all'utente business mediante il browser}
\scenario{
L'utente business può registrarsi (UC2.1);|
L'utente business può autenticarsi (UC2.2);|
L'utente business può recuperare la password (UC2.3);|
L'utente business autenticato può aprire una Collection (UC2.4);|
L'utente business autenticato può disconnettersi dal sistema (UC2.5);|
L'utente business autenticato può gestire il profilo (UC2.6).
}
\post
{Il sistema ha memorizzato le informazioni inserite dall'utente business per l'autenticazione o per il recupero password e le operazioni che l'utente business autenticato può effettuare}



\UCtitle
{Caso d'uso UC2.1}
{Registrazione}
\UCimmagine{UC2.1}{UC2.1 Registrazione}
\UC
{UC2.1}
{Utente business}
{L'utente business può registrarsi}
{Il sistema si trova nello stato iniziale e propone all'utente una schermata per registrarsi}
\scenario
{
L'utente business può inserire l'\gloss{email} (UC2.1.1);|
L'utente business può inserire la password (UC2.1.2).
}
\scenarioAlt
{
L'utente business può visualizzare un messaggio d'errore nel caso in cui i dati inseriti non fossero corretti (UC2.1.3).
}
\post
{Il sistema ha creato un nuovo utente e reputa l'utente business autenticato}

\UCtitle
{Caso d'uso UC2.1.1}
{Inserimento email}
\UC
{UC2.1}
{Utente business}
{L'utente business può inserire l'email per la registrazione}
{Il sistema si trova nello stato iniziale e propone all'utente una casella per inserire l'email}
\scenario
{L'utente business può inserire l'email per la registrazione.}
\post
{Il sistema ha ottenuto la mail inserita dall'utente business}

\UCtitle
{Caso d'uso UC2.1.2}
{Inserimento password}
\UC
{UC2.1}
{Utente business}
{L'utente business può inserire la password per la registrazione}
{Il sistema si trova nello stato iniziale e propone all'utente una casella per inserire la password}
\scenario
{L'utente business può inserire la password per la registrazione.}
\post
{Il sistema ha ottenuto la password inserita dall'utente business}

\UCtitle
{Caso d'uso UC2.1.3}
{Visualizzazione messaggio d'errore}
\UC
{UC2.1}
{Utente business}
{L'utente business può visualizzare un messaggio d'errore}
{Il sistema ha ottenuto dei dati per la registrazione utente non corretti}
\scenario
{L'utente business può visualizzare un messaggio d'errore.}
\post
{Il sistema ha visualizzato un messaggio d'errore all'utente business}

\UCtitle
{Caso d'uso UC2.2}
{Autenticazione}
\UCimmagine{UC2.2}{UC2.2 Autenticazione}
\UC
{UC2.2}
{Utente business}
{Un utente business tenta di autenticarsi}
{Il sistema si trova nello stato iniziale e propone all'utente una schermata per inserire i dati per l'autenticazione}
\scenario
{
L'utente business può inserire l'email (UC2.2.1);|
L'utente business può inserire la password (UC2.2.2).
}
\scenarioAlt
{
L'utente business può visualizzare un messaggio d'errore nel caso in cui i dati inseriti non fossero corretti (UC2.2.3).
}
\post
{Il sistema reputa l'utente business come autenticato}

\UCtitle
{Caso d'uso UC2.2.1}
{Inserimento email}
\UC
{UC2.2}
{Utente business}
{L'utente business può inserire l'email per l'autenticazione}
{Il sistema si trova nello stato iniziale e propone all'utente una casella per inserire l'email}
\scenario
{L'utente business può inserire l'email per l'autenticazione.}
\post
{Il sistema ha ottenuto la mail inserita dall'utente business}

\UCtitle
{Caso d'uso UC2.2.2}
{Inserimento password}
\UC
{UC2.2}
{Utente business}
{L'utente business può inserire la password per l'autenticazione}
{Il sistema si trova nello stato iniziale e propone all'utente una casella per inserire la password}
\scenario
{L'utente business può inserire la password per l'autenticazione.}
\post
{Il sistema ha ottenuto la password inserita dall'utente business}

\UCtitle
{Caso d'uso UC2.2.3}
{Visualizzazione messaggio d'errore}
\UC
{UC2.1}
{Utente business}
{L'utente business può visualizzare un messaggio d'errore}
{Il sistema ha ottenuto dei dati per l'autenticazione utente non corretti}
\scenario
{L'utente business può visualizzare un messaggio d'errore.}
\post
{Il sistema ha visualizzato un messaggio d'errore all'utente business}

\UCtitle
{Caso d'uso UC2.3}
{Recupero password}
\UCimmagine{UC2.3}{UC2.3 Recupero password}
\UC
{UC2.3}
{Utente business}
{Un utente business registrato recupera la password dimenticata}
{Il sistema  reputa l'utente business come registrato}
\scenario
{L'utente business inserisce l'email (UC2.3.1).
}
\post
{Il sistema ha fornito la password all'utente business}

\UCtitle
{Caso d'uso UC2.3.1}
{Inserimento email}
\UC
{UC2.3}
{Utente business}
{L'utente business può inserire l'email per il recupero password}
{Il sistema propone all'utente una casella per inserire l'email}
\scenario
{L'utente business può inserire l'email per il recupero password.}
\post
{Il sistema ha ottenuto la mail inserita dall'utente business}


\UCtitle
{Caso d'uso UC2.4}
{Apertura Collection}
\UCimmagine{UC2.4}{UC2.4 - Apertura Collection}
\UC
{UC2.4}
{Utente business autenticato, Utente business autenticato \gloss{amministratore}}
{L’ utente business autenticato può effettuare diverse operazioni: visualizzare una pagina Document-Show, modificare la visualizzazione dei Document, disconnettersi dal sistema, navigare tra le pagine della pagina Collection-Index correntemente visualizzata,  gestire il profilo, visualizzare le query più utilizate e creare indici su di esse. L'\gloss{utente business autenticato amministratore} potrà inoltre cancellare o modificare un Document}
{Il sistema reputa l'utente business come autenticato e visualizza la Collection}
\scenario{
L'utente business autenticato può visualizzare pagine Document-Show (UC2.4.1);|
L'utente business autenticato può modificare la visualizzazione dei Document (UC2.4.2);|
L'utente business autenticato può navigare tra le pagine della Collection-Index attualmente visualizzata (UC2.4.3);|
L'utente business autenticato può disconnettersi (UC2.5);|
L'utente business autenticato può gestire il profilo (UC2.6);|
L'utente business autenticato amministratore può cancellare un Document (UC2.4.4);|
L'utente business autenticato amministratore può modificare un Document (UC2.4.5);|
L'utente business autenticato amministratore può visualizzare le query più frequenti che il sistema gli propone (UC2.4.6);|
L'utente business autenticato amministratore può creare degli indici sulle query più frequentemente visualizzate (UC2.4.7).
}
\post
{Il sistema ha ottenuto le informazioni sulle operazioni che l'utente business autenticato o l'utente business autenticato amministratore desidera eseguire}



\UCtitle
{Caso d'uso UC2.4.1}
{Visualizzazione pagina Document-Show}
\UCimmagine{UC2.4.1}{UC2.4.1 - Visualizzazione pagina Document-Show}
\UC
{UC2.4}
{Utente business autenticato}
{L'utente business autenticato visualizza la pagina Document-Show relativa alla chiave selezionata}
{Il sistema reputa l'utente business come autenticato e visualizza la Collection scelta}
\scenario{
L'utente business autenticato visualizza la Document-Show relativa alla chiave selezionata (UC2.4.1.1);|
L'utente business autenticato amministratore può eliminare il Document che sta visualizzando (UC2.4.1.2);|
L'utente business autenticato amministratore può modificare il Document che sta visualizzando (UC2.4.1.3).
}
\post
{Viene visualizzata la pagina Document-Show relativa alla chiave selezionata}

\UCtitle
{Caso d'uso UC2.4.1.1}
{Visualizzazione Document}
\UC
{UC2.4.1}
{Utente business autenticato}
{L'utente business autenticato visualizza il Document relativo alla chiave selezionata}
{Il sistema reputa l'utente business come autenticato ed ha ottenuto la chiave relativa al Document da visualizzare}
\scenario{
L'utente business autenticato visualizza il Document relativo alla chiave selezionata.}
\post
{Viene visualizzato il Document relativo alla chiave selezionata}

\UCtitle
{Caso d'uso UC2.4.1.2}
{Eliminazione Document}
\UC
{UC2.4.1}
{Utente business autenticato}
{L'utente business autenticato può eliminare il Document che sta visualizzando}
{Il sistema reputa l'utente business come autenticato e visualizza il Document precedentemente selezionato}
\scenario
{L'utente business autenticato può eliminare il Document che sta visualizzando.}
\post
{Il sistema ha eliminato il Document selezionato}

\UCtitle
{Caso d'uso UC2.4.1.3}
{Modifica Document}
\UC
{UC2.4.1}
{Utente business autenticato}
{L'utente business autenticato può modificare il Document che sta visualizzando}
{Il sistema reputa l'utente business come autenticato e visualizza il Document precedentemente selezionato}
\scenario
{L'utente business autenticato può modificare il Document che sta visualizzando.}
\post
{Il sistema ha modificato il Document visualizzato}

\UCtitle
{Caso d'uso UC2.4.2}
{Modifica visualizzazione dei Document}
\UCimmagine{UC2.4.2}{UC2.4.2 - Modifica visualizzazione dei Document}
\UC
{UC2.4.2}
{Utente business autenticato}
{L'utente business autenticato può cambiare la modalità di visualizzazione dei Document}
{Il sistema reputa l'utente business come autenticato e visualizza la Collection scelta}
\scenario{
L'utente business autenticato può selezionare i criteri per la visualizzazione della lista di Document (UC2.4.2.1);|
L'utente business autenticato può applicare dei filtri (UC2.4.2.2);|
L'utente business autenticato può annullare un \gloss{filtro} applicato (UC2.4.2.3).
}
\post
{Il sistema ha ottenuto le informazioni sulle operazioni che l'utente business autenticato desidera eseguire ed aggiorna la visualizzazione della pagina}

\UCtitle
{Caso d'uso UC2.4.2.1}
{Seleziona criteri per la visualizzazione}
\UCimmagine{UC2.4.2.1}{UC2.4.2.1 - Seleziona criteri per la visualizzazione}
\UC
{UC2.4.2.1}
{Utente business autenticato}
{L'utente business autenticato può selezionare dei criteri per cambiare la visualizzazione della lista di Document: cambiare il numero di Document da visualizzare per pagina o cambiare l'ordinamento dei Document secondo una chiave}
{Il sistema reputa l'utente business come autenticato e visualizza la Collection scelta}
\scenario{
L'utente business autenticato può cambiare l'ordinamento dei Document secondo una chiave (UC2.4.2.1.1);|
L'utente business autenticato può cambiare il numero di Document da visualizzare per pagina (UC2.4.2.1.2).
}
\post
{Il sistema ha ottenuto le informazioni sulle operazioni che l'utente business autenticato desidera eseguire ed aggiorna la visualizzazione della pagina}


\UCtitle
{Caso d'uso UC2.4.2.1.1}
{Cambio ordinamento per chiave}
\UC
{UC2.4.2.1}
{Utente business autenticato}
{L'utente business autenticato può cambiare l'ordinamento dei documenti visualizzati secondo una chiave}
{Il sistema reputa l'utente business come autenticato e visualizza la Collection scelta}
\scenario
{L'utente business autenticato può cambiare l'ordinamento dei documenti visualizzati.}
\post
{Il sistema aggiorna la visualizzazione della pagina secondo i criteri imposti dall'utente business autenticato}


\UCtitle
{Caso d'uso UC2.4.2.1.2}
{Cambio numero di Document da visualizzare per pagina}
\UC
{UC2.4.2.1}
{Utente business autenticato}
{L'utente business autenticato può cambiare il numero di Document da visualizzare per pagina}
{Il sistema reputa l'utente business come autenticato e visualizza la Collection scelta}
\scenario
{L'utente business autenticato può cambiare il numero di Document da visualizzare per pagina.}
\post
{Il sistema aggiorna la visualizzazione del numero di Document per pagina secondo i criteri imposti dall'utente business autenticato}

\UCtitle
{Caso d'uso UC2.4.2.2}
{Applicazione filtro}
\UC
{UC2.4.2}
{Utente business autenticato}
{L'utente business autenticato può applicare un filtro disponibile nella pagina Collection-Index}
{Il sistema reputa l'utente business come autenticato e visualizza la Collection scelta}
\scenario
{L'utente business autenticato può applicare un filtro.}
\post
{Il sistema aggiorna la visualizzazione della pagina secondo i criteri imposti dal filtro applicato}

\UCtitle
{Caso d'uso UC2.4.2.3}
{Annullare filtro applicato}
\UC
{UC2.4.2}
{Utente business autenticato}
{L'utente business autenticato ha la possibilità di rimuovere un filtro precedentemente applicato}
{Il sistema ha applicato un filtro secondo i criteri imposti dall'utente business autenticato}
\scenario
{L'utente business autenticato può rimuovere un filtro.}
\post
{Il sistema aggiorna la visualizzazione della Collection allo stato precedente l'applicazione del filtro}


\UCtitle
{Caso d'uso UC2.4.3}
{Navigazione tra le pagine della Collection-Index}
\UC
{UC2.4}
{Utente business autenticato}
{L'utente business autenticato può navigare tra le pagine della Collection-Index attualmente visualizzata}
{Il sistema reputa l'utente business come autenticato e visualizza la Collection scelta}
\scenario
{L'utente business autenticato può navigare tra le pagine della Collection-Index attualmente visualizzata.}
\post
{Il sistema visualizza un'altra Collection scelta dall'utente business autenticato}

\UCtitle
{Caso d'uso UC2.4.4}
{Cancellazione Document}
\UC
{UC2.4}
{Utente business autenticato amministratore}
{L'utente business autenticato come amministratore può cancellare un Document presente nella Collection attualmente visualizzata}
{Il sistema reputa l'utente business autenticato come amministratore e visualizza la Collection scelta}
\scenario
{L'utente business autenticato come amministratore può cancellare un Document presente nella Collection attualmente visualizzata.}
\post
{Il sistema ha cancellato il Document selezionato dall'utente business autenticato amministratore ed aggiorna la pagina attualmente visualizzata}

\UCtitle
{Caso d'uso UC2.4.5}
{Modifica Document}
\UCimmagine{UC2.4.5}{UC2.4.5 - Modifica Document}
\UC
{UC2.4.5}
{Utente business autenticato amministratore}
{L'utente business autenticato come amministratore può modificare un Document presente nella Collection attualmente visualizzata}
{Il sistema reputa l'utente business autenticato come amministratore e visualizza la Collection scelta}
\scenario{
L'utente business autenticato amministratore può modificare il valore di ciascuna chiave (UC2.4.5.1);|
L'utente business autenticato amministratore può salvare il Document appena modificato (UC2.4.5.2).}
\estensioni
{L'utente business autenticato amministratore può annullare le modifiche fino a quel momento apportate (UC2.4.5.3).}
\post
{Il sistema modifica il Document selezionato secondo le specifiche imposte dall'utente business autenticato amministratore}


\UCtitle
{Caso d'uso UC2.4.5.1}
{Modifica valore della chiave}
\UC
{UC2.4.5}
{Utente business autenticato amministratore}
{L'utente business autenticato come amministratore può modificare il valore di ciascuna chiave del Document tramite form editabile}
{Il sistema reputa l'utente business autenticato come amministratore e visualizza la pagina Document-Show che l'utente business autenticato come amministratore desidera modificare}
\scenario
{L'utente business autenticato come amministratore può modificare il valore di ciascuna chiave del Document.}
\post
{Il sistema ha ottenuto un valore per ogni chiave del Document che l'utente business autenticato come amministratore ha modificato}

\UCtitle
{Caso d'uso UC2.4.5.2}
{Salvataggio Document}
\UC
{UC2.4.5}
{Utente business autenticato amministratore}
{L'utente business autenticato come amministratore può salvare le modifiche fino a quel momento apportate al Document}
{Il sistema ha ottenuto un valore per ogni chiave del Document che l'utente business autenticato come amministratore vuole modificare}
\scenario
{L'utente business autenticato come amministratore può salvare delle modifiche apportate al Document.}
\post
{Il sistema salva le modifiche apportate al Document}

\UCtitle
{Caso d'uso UC2.4.5.3}
{Annulla modifiche Document}
\UC
{UC2.4.5}
{Utente business autenticato amministratore}
{L'utente business autenticato come amministratore ha la possibilità di annullare le modifiche apportate al Document (vedi UC2.4.5)}
{Il sistema ha ottenuto un valore per ogni chiave del Document che si sta modificando}
\scenario
{L'utente business autenticato come amministratore ha la possibilità di annullare le modifiche apportate al Document.}
\post
{Il sistema non considera i valori inseriti per ogni chiave del Document e visualizza la pagina Collection-Index principale}

\UCtitle
{Caso d'uso UC2.4.6}
{Visualizzazione query più utilizzate}
\UC
{UC2.4}
{Utente business autenticato amministratore}
{L'utente business autenticato amministratore può visualizzare le query più utilizzate che il sistema gli offre}
{Il sistema reputa l'utente business autenticato come amministratore}
\scenario
{L'utente business autenticato amministratore può visualizzare le query più utilizzate.}
\post
{Il sistema mostra all'utente business autenticato amministratore le query più utilizzate}

\UCtitle
{Caso d'uso UC2.4.7}
{Gestione indici}
\UCimmagine{UC2.4.7}{UC2.4.7 - Gestione indici}
\UC
{UC2.4.7}
{Utente business autenticato amministratore}
{L'utente business autenticato amministratore può gestire la creazione degli indici sulla base delle query più utilizzate e l'eliminazione degli indici già presenti nel sistema}
{Il sistema reputa l'utente business autenticato come amministratore}
\scenario{
L'utente business autenticato amministratore può creare degli indici (UC2.4.7.1);|
L'utente business autenticato amministratore può eliminare degli indici (UC2.4.7.2);|
L'utente business autenticato amministratore può visualizzare gli indici presenti nel sistema (UC2.4.7.3).
}
\post
{Il sistema ha creato un nuovo indice o eliminato un indice esistente come richiesto dall'utente business autenticato amministratore}

\UCtitle
{Caso d'uso UC2.4.7.1}
{Creazione indici}
\UCimmagine{UC2.4.7.1}{UC2.4.7.1 - Creazione indici}
\UC
{UC2.4.7.1}
{Utente business autenticato amministratore}
{L'utente business autenticato amministratore può selezionare una query tra quelle che il sistema presenta come più utilizzate e può creare l'indice corrispondente.}
{Il sistema reputa l'utente business autenticato come amministratore e visualizza una lista di query più utilizzate}
\scenario{
L'utente business autenticato amministratore può selezionare una query (UC2.4.7.1.1);|
L'utente business autenticato amministratore può creare l'indice (UC2.4.7.1.2).
}
\post
{Il sistema ha creato un nuovo indice relativo alla query selezionata dall'utente business autenticato amministratore}

\UCtitle
{Caso d'uso UC2.4.7.1.1}
{Seleziona query}
\UC
{UC2.4.7.1}
{Utente business autenticato amministratore}
{L'utente business autenticato amministratore può selezionare una query tra quelle che il sistema presenta come più utilizzate.}
{Il sistema reputa l'utente business autenticato come amministratore e visualizza una lista di query più utilizzate}
\scenario
{L'utente business autenticato amministratore può selezionare una query.}
\post
{Il sistema ha ottentuo l'informazione riguardo alla query selezionata dall'utente business autenticato amministratore}

\UCtitle
{Caso d'uso UC2.4.7.1.2}
{Crea indice}
\UC
{UC2.4.7.1}
{Utente business autenticato amministratore}
{L'utente business autenticato amministratore può creare un indice relativo alla query precedentemente selezionata.}
{Il sistema reputa l'utente business autenticato come amministratore e l'utente business autenticato amministratore ha precedentemente selezionato una query}
\scenario
{L'utente business autenticato amministratore può creare un indice.}
\post
{Il sistema ha creato un indice relativo alla query selezionata dall'utente business autenticato amministratore}


\UCtitle
{Caso d'uso UC2.4.7.2}
{Eliminazione indici}
\UCimmagine{UC2.4.7.2}{UC2.4.7.2 - Eliminazione indici}
\UC
{UC2.4.7.2}
{Utente business autenticato amministratore}
{L'utente business autenticato amministratore può selezionare un indice tra quelli presenti nel sistema e può eliminarlo.}
{Il sistema reputa l'utente business autenticato come amministratore e visualizza una lista degli indici presenti nel sistema.}
\scenario{
L'utente business autenticato amministratore può selezionare un indice (UC2.4.7.2.1);|
L'utente business autenticato amministratore può eliminare l'indice (UC2.4.7.2.2).
}
\post
{Il sistema ha eliminato l'indice selezionato dall'utente business autenticato amministratore}

\UCtitle
{Caso d'uso UC2.4.7.2.1}
{Seleziona indice}
\UC
{UC2.4.7.2}
{Utente business autenticato amministratore}
{L'utente business autenticato amministratore può selezionare un indice tra quelli presenti nel sistema}
{Il sistema reputa l'utente business autenticato come amministratore e visualizza una lista degli indici presenti nel sistema.}
\scenario{
L'utente business autenticato amministratore può selezionare un indice.}
\post
{Il sistema ha ottenuto le informazioni riguardo all'indice selezionato dall'utente business autenticato amministratore}

\UCtitle
{Caso d'uso UC2.4.7.2.2}
{Elimina indice}
\UC
{UC2.4.7.2}
{Utente business autenticato amministratore}
{L'utente business autenticato amministratore può eliminare l'indice precedentemente selezionato}
{Il sistema reputa l'utente business autenticato come amministratore e l'utente business autenticato amministratore ha selezionato un indice tra quelli presenti nel sistema.}
\scenario{
L'utente business autenticato amministratore può eliminare l'indice selezionato.}
\post
{Il sistema ha eliminato l'indice selezionato dall'utente business autenticato amministratore}


\UCtitle
{Caso d'uso UC2.4.7.3}
{Visualizzazione indici}
\UC
{UC2.4.7}
{Utente business autenticato amministratore}
{L'utente business autenticato amministratore può visualizzare gli indici presenti nel sistema.}
{Il sistema reputa l'utente business autenticato come amministratore}
\scenario{
L'utente business autenticato amministratore può visualizzare gli indici presenti nel sistema.}
\post
{Il sistema visualizza gli indici attualmente presenti nel sistema}

\UCtitle
{Caso d'uso UC2.5}
{Disconnessione}
\UC
{UC2}
{Utente business autenticato}
{L'utente può uscire dal proprio profilo e disconnettersi dal sistema}
{Il sistema reputa l'utente business come autenticato}
\scenario
{L'utente può disconnettersi dal sistema.}
\post
{Il sistema reputa l'utente business come non autenticato}

\UCtitle
{Caso d'uso UC2.6}
{Gestione utente}
\UCimmagine{UC2.6}{UC2.6 - Gestione utente}
\UC
{UC2.6}
{Utente business autenticato}
{L'utente business autenticato può modificare i dati utente, mentre l'utente business autenticato amministratore può anche eseguire operazioni di creazione e cancellazione di un utente}
{Il sistema reputa l'utente business come autenticato}
\scenario{
L'utente business autenticato può gestire i dati (UC2.6.3);|
L'utente business autenticato amministratore può creare un nuovo utente (UC2.6.1);|
L'utente business autenticato amministratore può eliminare un utente (UC2.6.2).
}
\post
{Il sistema ha ottenuto le informazioni sulle operazioni che l'utente business desidera eseguire}

\UCtitle
{Caso d'uso UC2.6.1}
{Creazione nuovo utente}
\UC
{UC2.6}
{Utente business autenticato amministratore}
{L'utente business autenticato come amministratore può creare un nuovo utente}
{Il sistema reputa l'utente business autenticato come amministratore}
\scenario
{L'utente business autenticato come amministratore può creare un nuovo utente.}
\post
{Il sistema ha creato un nuovo utente}


\UCtitle
{Caso d'uso UC2.6.2}
{Eliminazione utente}
\UC
{UC2.6}
{Utente business autenticato amministratore}
{L'utente business autenticato come amministratore può eliminare un profilo utente preesistente}
{Il sistema reputa l'utente business autenticato come amministratore}
\scenario
{L'utente business autenticato come amministratore può eliminare un profilo utente.}
\post
{Il sistema ha eliminato il profilo utente specificato}

\UCtitle
{Caso d'uso UC2.6.3}
{Gestione dati}
\UCimmagine{UC2.6.3}{UC2.6.3 - Gestione dati}
\UC
{UC2.6.3}
{Utente business autenticato}
{L'utente business autenticato può modificare i dati utente del proprio profilo e salvare le modifiche o annullarle. L'utente business autenticato come amministratore può modificare i dati dei vari profili utenti, modificare i permessi degli utenti, salvare le modifiche o annullarle}
{Il sistema reputa l'utente business come autenticato}
\scenario{
L'utente business autenticato può modificare i dati utente del proprio profilo (UC2.6.3.1);|
L'utente business autenticato può salvare le modifiche (UC2.6.3.2);|
L'utente business autenticato come amministratore può modificare i permessi degli utenti (UC2.6.3.3).
}
\post
{Il sistema modifica i dati utente secondo le specifiche imposte dall'utente business autenticato}

\UCtitle
{Caso d'uso UC2.6.3.1}
{Modifica dati utente}
\UC
{UC2.6.3}
{Utente business autenticato}
{L'utente business autenticato può modificare i dati utente del proprio profilo, ovvero email e password}
{Il sistema reputa l'utente business come autenticato}
\scenario
{L'utente business autenticato può modificare i dati utente del proprio profilo.}
\scenarioAlt
{
L'utente business autenticato può annullare le modifiche apportate (UC2.6.3.4).
}
\post
{Il sistema ha ottenuto tutti i dati modificati del profilo utente}

\UCtitle
{Caso d'uso UC2.6.3.2}
{Salvataggio modifiche}
\UC
{UC2.6.3}
{Utente business autenticato}
{L'utente business autenticato può salvare le modifiche apportate durante la modifica dei dati utente}
{Il sistema ha ottenuto i dati modificati del profilo utente}
\scenario
{L'utente business autenticato può salvare le modifiche apportate durante la modifica dei dati utente.}
\post
{Il sistema aggiorna il profilo utente con i nuovi dati inseriti}

\UCtitle
{Caso d'uso UC2.6.3.3}
{Modifica permessi utenti}
\UC
{UC2.6.3}
{Utente business autenticato amministratore}
{L'utente business autenticato come amministratore può modificare i permessi degli utenti, ovvero elevare un utente a livello amministratore o declassare un utente amministratore ad utente business autenticato}
{Il sistema reputa l'utente business autenticato come amministratore}
\scenario
{L'utente business autenticato come amministratore può modificare i permessi degli utenti.}
\post
{Il sistema ha ottenuto i nuovi permessi da applicare al profilo utente}

\UCtitle
{Caso d'uso UC2.6.3.4}
{Annulla modifiche}
\UC
{UC2.6.3}
{Utente business autenticato}
{L'utente business autenticato può annullare le modifiche apportate}
{Il sistema ha ottenuto i dati modificati del profilo utente}
\scenario
{L'utente business autenticato può annullare le modifiche apportate.}
\post
{Il sistema non considera le modifiche apportate}
