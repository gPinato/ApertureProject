%"variabili globali" che verranno aggiornate in tutte le pagine/footer
% per inserire nuove macro ->  \newcommand{\nomeMacro}{valoreMacro}

\newcommand{\Versione}{1.2.0}						%Versione Finale
\newcommand{\Data}{2014-03-7}						%Data di creazione
\newcommand{\DataUltimaModifica}{2014-03-7}
\newcommand{\TipoDocumento}{Verbale esterno 2014-03-5}		%tipo documento

%includo il file header.tex (logo grande in prima pagina piu qualche altra regola)
%questo file contiene impostazioni comuni per tutte i documenti

%definizione packages utilizzati
\documentclass[a4paper]{article}
\usepackage[utf8x]{inputenc}
\usepackage{enumitem}
\usepackage[italian]{babel}
\usepackage{latexsym}
\usepackage{xparse}
\usepackage{float}
\usepackage{subfloat}
\usepackage{subfig}
\usepackage{fancyhdr}
\usepackage{eurofont}
\usepackage{lastpage}
\usepackage{graphicx}
\usepackage{textcomp}
\usepackage{booktabs}
\usepackage{color}
\usepackage{lscape}
\usepackage{hyperref}
\hypersetup{colorlinks=true, linkcolor=black, anchorcolor=red, urlcolor=blue}
\usepackage{longtable}
\usepackage{tabularx}
\usepackage{abstract}
\usepackage{appendix}
\usepackage{multicol}
\usepackage{bmpsize}
\usepackage[all]{hypcap}
\usepackage{titlesec}
\usepackage{indentfirst}
\usepackage{lipsum,titletoc}

%\setcounter{secnumdepth}{4}

%****************INIZIO GESTIONE SUBSECTION MULTIPLE
\makeatletter
\newcommand\level[1]{%
  \ifcase#1\relax\expandafter\chapter\or
    \expandafter\section\or
    \expandafter\subsection\or
    \expandafter\subsubsection\else
    \def\next{\@level{#1}}\expandafter\next
  \fi}
\newcommand{\@level}[1]{%
  \@startsection{level#1}
    {#1}
    {\z@}%
    {-3.25ex\@plus -1ex \@minus -.2ex}%
    {1.5ex \@plus .2ex}%
    {\normalfont\normalsize\bfseries}}

\newdimen\@leveldim
\newdimen\@dotsdim
{\normalfont\normalsize
 \sbox\z@{0}\global\@leveldim=\wd\z@
 \sbox\z@{.}\global\@dotsdim=\wd\z@
}

\newcounter{level4}[subsubsection]
\@namedef{thelevel4}{\thesubsubsection.\arabic{level4}}
\@namedef{level4mark}#1{}
\def\l@section{\@dottedtocline{1}{0pt}{\dimexpr\@leveldim*4+\@dotsdim*1+6pt\relax}}
\def\l@subsection{\@dottedtocline{2}{0pt}{\dimexpr\@leveldim*5+\@dotsdim*2+6pt\relax}}
\def\l@subsubsection{\@dottedtocline{3}{0pt}{\dimexpr\@leveldim*6+\@dotsdim*3+6pt\relax}}
\@namedef{l@level4}{\@dottedtocline{4}{0pt}{\dimexpr\@leveldim*7+\@dotsdim*4+6pt\relax}}

\count@=4
\def\@ncp#1{\number\numexpr\count@+#1\relax}
\loop\ifnum\count@<100
  \begingroup\edef\x{\endgroup
    \noexpand\newcounter{level\@ncp{1}}[level\number\count@]
    \noexpand\@namedef{thelevel\@ncp{1}}{%
      \noexpand\@nameuse{thelevel\@ncp{0}}.\noexpand\arabic{level\@ncp{1}}}
    \noexpand\@namedef{level\@ncp{1}mark}####1{}%
    \noexpand\@namedef{l@level\@ncp{1}}%
      {\noexpand\@dottedtocline{\@ncp{1}}{0pt}{\the\dimexpr\@leveldim*\@ncp{5}+\@dotsdim*\@ncp{0}\relax}}}%
  \x
  \advance\count@\@ne
\repeat
\makeatother
\setcounter{secnumdepth}{100}
\setcounter{tocdepth}{100}
%****************FINE GESTIONE SUBSECTION MULTIPLE

%impostazioni relative alla visualizzazione delle section 
%nell'indice
\titlecontents{section}
[0pt]%left indent
{\bfseries}
{\contentslabel{2.3em}}
{\hspace*{-2.3em}}
{\hfill\contentspage}
[]%separator


\oddsidemargin=.15in
\evensidemargin=.15in
\textwidth=6in
\topmargin=-.5in
\parindent=0in
\headheight=1in
\DeclareMathSizes{10}{10}{10}{10} %per piano qualifica
\pagestyle{fancy}
\lhead{
\bfseries {\Large \TipoDocumento}\\
\bfseries Versione: \Versione\\
}
\chead{}
\lhead{
\includegraphics[scale=0.455]{../Logo&Header/apertureHead.png}
}
\lfoot{\bfseries \TipoDocumento{} v\Versione}
\cfoot{}
\rfoot{\thepage\ of \mypageref{LastPage}}
\newcommand{\mypageref}[1]{
\hypersetup{linkcolor=black}\pageref{#1}\hypersetup{linkcolor=black}}
%\userpackage{lipsum}
\renewcommand{\footrulewidth}{0.4pt}

%definizioni comandi comuni utilizzati
\newcommand{\numref}[1]{\textsl{\nameref{#1} (\ref{#1})}}
\newcommand{\NomeGruppo}{Aperture Software}
\newcommand{\Progetto}{MaaP: MongoDB as an admin Platform}
\newcommand{\Prop}{CoffeeStrap}

%definizione tecnologie
\newcommand{\Node}{Node.js}
\newcommand{\NodeJS}{Node.js}
\newcommand{\Nodejs}{Node.js}

\newcommand{\mongodb}{MongoDB}

%tanti sub quanti ne vogliamo! :)
\newcommand{\subsubsubsection}{\level{4}}
\newcommand{\subsubsubsubsection}{\level{5}}
\newcommand{\subsubsubsubsubsection}{\level{6}}
\newcommand{\subsubsubsubsubsubsection}{\level{7}}
\newcommand{\subsubsubsubsubsubsubsection}{\level{8}}


%definizione comando per parola glossario
\newcommand{\gloss}[1]{\emph{#1}\ped{\emph{\tiny{G}}}}

\newcommand{\grassetto}{\textbf}

%per inserire immagini
\newcommand{\immagine}[2]{ 
\begin{center}
\begin{figure}[H]
\includegraphics[width=\textwidth]{{{#1}}}
\caption{#2}
\label{#1}
\end{figure}
\end{center}
}

\newcommand{\Glossario}{
Al fine di evitare ogni ambiguità nella comprensione del linguaggio utilizzato nel presente documento e, in generale, nella documentazione fornita dal gruppo \NomeGruppo{}, ogni termine tecnico, di difficile comprensione o di necessario approfondimento verrà inserito nel documento \emph{Glossario\_{}v\versioneGlossario{}.pdf}.\\
Saranno in esso definiti e descritti tutti i termini in corsivo e allo stesso tempo marcati da una lettera "G" maiuscola in pedice nella documentazione fornita.
}

\newcommand{\Prodotto}{
Lo scopo del prodotto è produrre un framework per generare interfacce web di amministrazione dei dati di business basati sullo stack \Nodejs{} e \mongodb{}.\\
L'obiettivo è quello di semplificare il lavoro allo sviluppatore che dovrà rispondere in modo rapido e standard alle richieste degli esperti di business.
}

%inizio pagina del documento 
\begin{document}
\thispagestyle{empty}

\begin{center}\centerline{
%inserisco il logo grande della prima pagina
\includegraphics[scale=0.8]{../Logo&Header/logo.png}}

%metto il link dell'email sotto al logo
%{\href{mailto:ApertureSWE@gmail.com}{\color[rgb]{0.39,0.37,0.38}%ApertureSWE@gmail.com}}\\ [3pc]

\vspace{0.5in}

%titolo del progetto
{\Huge {\Progetto}}\\[.5pc]

\underline{\hspace{6in}}\\[8pc]

{\Huge {\TipoDocumento}}\\[1pc]
%{\emph{Versione \Versione}}\\
\end{center}

%\vspace{.05in}

%\vspace{.05in}

%informazioni documento
\begin{center}
%\section{Informazioni documento}
\begin{tabular}{r|l}
%\textbf{Nome} &\TipoDocumento \\
\textbf{Versione} & \Versione{} \\
\textbf{Data creazione} & \Data{2014-03-7} \\
\textbf{Data ultima modifica} & \DataUltimaModifica{2014-03-7} \\
\textbf{Stato del Documento} & Formale \\		%CAMBIARE QUI
\textbf{Uso del Documento} & Esterno \\			%CAMBIARE QUI
\textbf{Redazione} & Giacomo Pinato\\				%CAMBIARE QUI
\textbf{Verifica} & Fabio Miotto\\		%ED ANCHE QUI!
\textbf{Approvazione} & Alberto Garbui \\		%CAMBIARE QUI
\textbf{Distribuzione} & \parbox[t]{4cm}{\NomeGruppo{} \\Prof. Tullio Vardanega \\ Prof. Riccardo Cardin \\ \Prop{} }\\
\end{tabular}
\end{center}

\vspace{0.05in}

%inizio sommario del documento
\begin{abstract}
\begin{center}
Verbale dell'incontro tra i componenti del gruppo \NomeGruppo{} e il Proponente.
\end{center}
\end{abstract}

%\vspace{.4in}

%seconda pagina, diario delle modifiche
\newpage
Diario delle modifiche
\begin{center}
\begin{longtable}{|c|c|c|p{0.5\linewidth}|}
\toprule
\textbf{Versione} & \textbf{Data} & \textbf{Autore} & \textbf{Modifiche effettuate}\\

%aggiungere qui una midrule per aggiungere una nuova riga alla tabella
\midrule
1.2.0 & 2014-01-21 & Alberto Garbui (RE)  & Approvazione documento\\
\midrule
1.1.0 & 2014-01-21 & Fabio Miotto (VE) & Verifica documento\\
\midrule
1.0.1 & 2014-03-7 & Giacomo Pinato (AN) & Stesura documento\\

\bottomrule
\caption{Registro delle modifiche}
\label{tab:changelog}
\end{longtable}
\end{center}

%terza pagina Indice (viene aggiornato in automatico con due compilazioni)
\newpage
\tableofcontents

%pagine successive hanno la lista di tabelle e lista delle figure
%(vengono aggiornate in automatico)
%\newpage
%\listoftables
%\listoffigures

%qui inizia la prima pagina ufficiale
\newpage
\section{Informazioni sulla riunione}%1.0
\label{1.0}
\begin{itemize}
\item \grassetto{Data}: 2014-03-7;
\item \grassetto{Luogo}: Laboratorio informatico Paolotti;
\item \grassetto{Ora}: 15:00;
\item \grassetto{Durata}: 1h 30m;
\item \grassetto{Partecipanti interni}: \NomeGruppo{}
\begin{itemize}
\item Maso Michele;
\item Miotto Fabio;
\item Garbui Alberto;
\item Sorgato Mattia;
\item Perin Andrea;
\item Pinato Giacomo.
\item \grassetto{Partecipanti esterni}:
\begin{itemize}
\item Alessandro Comecavolosichiama
\end{itemize}
\item \grassetto{Motivazioni riunione}: Validazione della progettazione effettuata e discussione su scelte future;
\item \grassetto{Argomenti trattati}: Scelte progettuali effettuate e da effettuare, HTML5 Mode per Angular, gestione mail del sistema.
\end{itemize}
\end{itemize}

\newpage
\section{Punti all'ordine del giorno}
\label{2}

\begin{itemize}
\item Validazione progettazione effettuata;
\item Scelte progettuali a breve termine;
\item Consigli e spunti da parte del Proponente.
\end{itemize}


\subsection{Validazione progettazione effettuata:}
Il Proponente, dopo aver analizzato la progettazione effettuata fin'ora, ha valutato la stessa coerente con il prodotto finale e con le sue aspettative.
Non ha inoltre evidenziato falle, mancanze o inconsistenze nel lavoro effettuato.

\subsection{Scelte progettuali a breve termine:}
Dopo aver discusso col Proponente in merito a scelte progettuali da effettuare a brevissimo e breve termine, si è concluso che:

\begin{itemize}
\item Tutti i file DSL verranno compilati all'avvio del server invece che alla loro prima richiesta di utilizzo;
\item Le trasmissioni tra client e server saranno pi\'{u} frequenti e meno pesanti dal punto di vista della quantit\'{a} di dati trasmessi. Ovvero si faranno molte richieste per pochi dati piuttosto che poche richieste con molti dati;
\item La gestione delle email del sistema, ad esempio per registrazione e recupero password, avver\'{a} utilizzando nodemailer.
\end{itemize}


\subsection{Consigli e spunti da parte del Proponente:}
Il Proponente ha portato alla nostra attenzione le seguenti problematiche e spunti di riflessione:

\begin{itemize}
\item Le URL di Angular, essendo fittizie, non possono essere inserite direttamente nella barra URL del browser. Essendo una funzionalit\'{a} fortemente desiderabile, ci è stato consigliato di valutare l'utilizzo di HTML5 Mode. Questa funzionalità di Angular permette di ristabilire la coerenza tra URL e informazioni visualizzate sul browser;
\item Con un leggero lavoro di adattamento \'{e} possibile far si che MaaP si interfacci anche con database NoSQL non MongoDB.
\end{itemize}





%FINE DOCUMENTO NON CANCELLARE
\end{document}
