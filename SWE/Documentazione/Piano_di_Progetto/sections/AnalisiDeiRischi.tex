\section{Analisi dei rischi}
Nello sviluppo del progetto si è dedicato parte del tempo per individuare rischi che potrebbero compromettere la realizzazione del progetto.\\
Per prevedere e controllare eventuali rischi si è seguita questa sequenza di passi:
\begin{itemize}
\item \grassetto{Identificazione e analisi:} si cerca di individuare i rischi che possono comparire durante lo sviluppo del progetto e nella realizzazione del prodotto; successivamente con l'analisi si cerca di comprendere eventuali criticità e conseguenze che questi rischi possono portare;
\item \grassetto{Controllo:} si valutano le strategie da attuare  per prevenire i rischi;
\item \grassetto{Soluzione:} azioni da intraprendere, in caso di rischio già avvenuto, per vanificare l'effetto dello stesso.
\item \grassetto{Riscontro:} vengono riportati i riscontri effettivi verificati nel corso del progetto. In questo punto verranno anche riassunte le conclusioni che questa verifica ha portato e le attuazioni sul rischio stesso. 
\end{itemize}
Di seguito verranno elencati i rischi emersi e saranno accompagnati da una breve descrizione, un livello di rischio, le contromisure da adottare e l'impatto che essi avranno.

\subsection{Conoscenze tecnologiche}
\begin{itemize}
\item \grassetto{Descrizione}: i componenti del gruppo non hanno una buona conoscenza delle tecnologie utilizzate, in quanto queste sono delle novità assolute per alcuni di loro, in particolare:
\begin{itemize}
\item \gloss{MongoDB};
\item \gloss{Node.js};
\item Mongoose;
\item \gloss{Express}.
\end{itemize}
La scarsa conoscenza delle precedenti tecnologie saranno un ostacolo che i componenti del gruppo troveranno durante lo sviluppo del progetto.
\item \grassetto{Livello di rischio}: alto;
\item \grassetto{Contromisure}: i componenti del gruppo si impegnano per istruirsi sull'utilizzo delle tecnologie richieste, partecipando a seminari specifici e utilizzando materiale e documentazione reperibile sul \gloss{Web} o che fornisce l'Amministratore. Questi apprendimenti saranno fatti in tempi brevi, per non compromettere e ritardare il proseguimento del progetto; i componenti del gruppo hanno dedicato delle ore pianificate il fine settimana per studiare queste tecnologie necessarie;
\item \grassetto{Impatto}: medio. I componenti del gruppo incontreranno difficoltà nell'utilizzo delle tecnologie apprese.
\item \grassetto{Riscontro}: si è riscontrato che la quantità di materiale informativo/educativo riguardante le tecnologie necessarie è davvero molto alta. La quantità e la qualità di tale materiale ha portato ad una facile educazione dei membri del team, con basso dispendio di tempo ed energie rispetto a quanto previsto. Da queste premesse, è stato deciso di abbassare il livello di impatto da "alto" a "medio". 
\end{itemize}

\subsection{Problemi hardware}
\begin{itemize}
\item \grassetto{Descrizione}: gran parte del lavoro è basato sull'utilizzo di personal computer e di un \gloss{server} per il \gloss{repository} e per il sistema di gestione dei \gloss{ticket}, descritto nella sezione apposita delle Norme di Progetto. In seguito ad una loro eventuale rottura si potrebbe perdere parte del lavoro svolto e le conseguenti attività di ripristino porterebbero ad un ritardo nello svolgimento delle attività ripartite per il prosieguo del progetto;
\item \grassetto{Livello di rischio}: basso;
\item \grassetto{Contromisure}: sulla componente server verranno effettuati periodici backup per non perdere il lavoro svolto, ed anche i singoli componenti del gruppo, al termine della giornata lavorativa, effettueranno un backup su sistemi di \gloss{Cloud} o dispositivi \gloss{hardware} esterni;
\item \grassetto{Impatto}: basso. In caso di rischio avvenuto, la presenza dei numerosi backup faciliterà il compito dell' Amministratore, ovvero il ripristino del lavoro svolto.
\item \grassetto{Riscontro}: in base a quanto premesso, si è notato in effetti che i problemi hardware in genere si sono presentati con scarsa probabilità e non hanno influito sulla condotta del progetto. In questi casi hanno aiutato le strutture messe a disposizione dalla facoltà, le quali ci hanno permesso di lavorare con pari efficacia. Ci è parso logico mantenere i livelli di rischio e impatto inalterati.
\end{itemize}

\subsection{Variabilità requisiti}
\begin{itemize}
\item \grassetto{Descrizione}: il gruppo non deve sottovalutare la possibilità che il Proponente possa cambiare i requisiti in corso d'opera;
\item \grassetto{Livello di rischio}: basso;
\item \grassetto{Contromisure}: se il rischio si verifica, sarà compito del gruppo adeguarsi ai nuovi requisiti imposti dal Proponente;
\item \grassetto{Impatto}: medio-alto. Un cambiamento sostanziale dei requisiti porterà ad un impatto alto di tale rischio e al conseguente ritardo nella consegna del prodotto finale.
\item \grassetto{Riscontro}: il Proponente è stato sempre molto disponibile e flessibile negli incontri fissati, e in nessun caso ha mai avanzato richieste che esulassero da quelle descritte nel capitolato. Le uniche modifiche richieste sono state molto specifiche e comunque in linea con i nostri requisiti. Si è quindi deciso di abbassare il livello di rischio da "medio" a "basso".
\end{itemize}

\subsection{Comprensione requisiti}
\begin{itemize}
\item \grassetto{Descrizione}: è possibile che i componenti del gruppo non comprendano in pieno i requisiti e che alcuni aspetti vengano trattati in maniera errata o incompleta;
\item \grassetto{Livello di rischio}: medio;
\item \grassetto{Contromisure}: per ridurre al minimo gli effetti che il rischio comporta, ci saranno, durante il periodo di tempo dedicato all'Analisi dei Requisiti, degli incontri con il Proponente per delle delucidazioni in merito ai requisiti richiesti dal prodotto;
\item \grassetto{Impatto}: medio-alto. Se si verifica tale rischio è necessario aggiornare con piccole modifiche il documento di Analisi dei Requisiti, o nel caso peggiore sarà necessario stilare una nuova \gloss{versione} del medesimo documento.
\item \grassetto{Riscontro}: durante il progetto si sono verificati casi di occorrenza di questo rischio. In particolare, durante le attività di progettazione, la comprensione dei requisiti è stata a tratti ambigua e ciò ha portato a diverse consultazioni con gli Analisti e a revisioni della progettazione, anche se non particolarmente gravi. L'impatto sulla progettazione non è stato grave come previsto, ma comunque abbastanza alto. Si è deciso quindi di abbassare di mezzo grado l'impatto, da "alto" a "medio-alto".
\end{itemize}

\subsection{Problemi componenti del gruppo}
\begin{itemize}
\item \grassetto{Descrizione}: ogni componente del gruppo ha impegni personali e svolge attività extra accademiche ed alcuni di essi svolgono un vero e proprio lavoro. Di conseguenza non possono essere spesso disponibili;
\item \grassetto{Livello di rischio}: alto;
\item \grassetto{Contromisure}: è compito del Responsabile di Progetto stilare appositi calendari di gruppo per organizzare le giornate lavorative per il progetto, in modo tale da non contrastare gli impegni personali di ciascun componente. Inoltre il carico di lavoro che un componente non svolgerà per impegni personali, dovrà essere ripartito tra gli altri componenti del gruppo;
\item \grassetto{Impatto}: alto.
\item \grassetto{Riscontro}: Sebbene la maggior parte del gruppo si ritrovi sempre negli orari stabiliti, alcuni membri, dati li impegni personali, non sono sempre operativi. Questo rischio è stato riscontrato con alta occorrenza, e ha portato ad un rallentamento nell'avanzamento del progetto. Quindi i livelli di rischio e impatto rimangono inalterati. Sarà compito del Responsabile di Progetto stilare dei piani più stringenti o prendere provvedimenti individuali.
\end{itemize}

\subsection{Preventivi di costi errati}
\begin{itemize}
\item \grassetto{Descrizione}: i componenti del gruppo, non avendo esperienza in ambito manageriale e finanziario, possono imbattersi in errori di pianificazione dei tempi e conseguente aumento dei costi;
\item \grassetto{Livello di rischio}: medio-basso;
\item \grassetto{Contromisure}: si cercherà di dedicare più tempo alla pianificazione dei tempi e costi, affrontando i vari passi che ne derivano con maggior calma e attenzione;
\item \grassetto{Impatto}: medio-alto. Se il rischio si verifica, bisognerà avvisare il Proponente e cercare di ridurre i costi per avvicinarsi il più possibile alla soglia prestabilita.
\item \grassetto{Riscontro}: da quanto rilevato dai consuntivi stilati, i preventivi non si sono discostati dai consuntivi in maniera rilevante. Si è deciso quindi di abbassare il livello di rischio da "medio" a "medio-basso".
\end{itemize}
