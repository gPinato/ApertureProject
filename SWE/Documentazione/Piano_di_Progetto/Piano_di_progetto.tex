%"variabili globali" che verranno aggiornate in tutte le pagine/footer
% per inserire nuove macro ->  \newcommand{\nomeMacro}{valoreMacro}

%includo il file che contiene la versione dei documenti
\newcommand{\versioneAnalisiDeiRequisiti}{2.2.0}			
\newcommand{\versioneNormeDiProgetto}{2.2.0}			
\newcommand{\versioneGlossario}{2.2.0}			
\newcommand{\versionePianoDiQualifica}{2.2.0}			
\newcommand{\versionePianoDiProgetto}{2.2.0}	
\newcommand{\versioneStudioDiFattibilita}{2.2.0}
\newcommand{\versioneSpecificaTecnica}{2.2.0}


\newcommand{\Versione}{\versionePianoDiProgetto{}}	%Versione Finale
\newcommand{\Data}{2013-11-28}						%Data di creazione
\newcommand{\DataUltimaModifica}{2013-12-18}
\newcommand{\TipoDocumento}{Piano di Progetto}		%tipo documento

%includo il file header.tex (logo grande in prima pagina piu qualche altra regola)
%questo file contiene impostazioni comuni per tutte i documenti

%definizione packages utilizzati
\documentclass[a4paper]{article}
\usepackage[utf8x]{inputenc}
\usepackage{enumitem}
\usepackage[italian]{babel}
\usepackage{latexsym}
\usepackage{xparse}
\usepackage{float}
\usepackage{subfloat}
\usepackage{subfig}
\usepackage{fancyhdr}
\usepackage{eurofont}
\usepackage{lastpage}
\usepackage{graphicx}
\usepackage{textcomp}
\usepackage{booktabs}
\usepackage{color}
\usepackage{lscape}
\usepackage{hyperref}
\hypersetup{colorlinks=true, linkcolor=black, anchorcolor=red, urlcolor=blue}
\usepackage{longtable}
\usepackage{tabularx}
\usepackage{abstract}
\usepackage{appendix}
\usepackage{multicol}
\usepackage{bmpsize}
\usepackage[all]{hypcap}
\usepackage{titlesec}
\usepackage{indentfirst}
\usepackage{lipsum,titletoc}

%\setcounter{secnumdepth}{4}

%****************INIZIO GESTIONE SUBSECTION MULTIPLE
\makeatletter
\newcommand\level[1]{%
  \ifcase#1\relax\expandafter\chapter\or
    \expandafter\section\or
    \expandafter\subsection\or
    \expandafter\subsubsection\else
    \def\next{\@level{#1}}\expandafter\next
  \fi}
\newcommand{\@level}[1]{%
  \@startsection{level#1}
    {#1}
    {\z@}%
    {-3.25ex\@plus -1ex \@minus -.2ex}%
    {1.5ex \@plus .2ex}%
    {\normalfont\normalsize\bfseries}}

\newdimen\@leveldim
\newdimen\@dotsdim
{\normalfont\normalsize
 \sbox\z@{0}\global\@leveldim=\wd\z@
 \sbox\z@{.}\global\@dotsdim=\wd\z@
}

\newcounter{level4}[subsubsection]
\@namedef{thelevel4}{\thesubsubsection.\arabic{level4}}
\@namedef{level4mark}#1{}
\def\l@section{\@dottedtocline{1}{0pt}{\dimexpr\@leveldim*4+\@dotsdim*1+6pt\relax}}
\def\l@subsection{\@dottedtocline{2}{0pt}{\dimexpr\@leveldim*5+\@dotsdim*2+6pt\relax}}
\def\l@subsubsection{\@dottedtocline{3}{0pt}{\dimexpr\@leveldim*6+\@dotsdim*3+6pt\relax}}
\@namedef{l@level4}{\@dottedtocline{4}{0pt}{\dimexpr\@leveldim*7+\@dotsdim*4+6pt\relax}}

\count@=4
\def\@ncp#1{\number\numexpr\count@+#1\relax}
\loop\ifnum\count@<100
  \begingroup\edef\x{\endgroup
    \noexpand\newcounter{level\@ncp{1}}[level\number\count@]
    \noexpand\@namedef{thelevel\@ncp{1}}{%
      \noexpand\@nameuse{thelevel\@ncp{0}}.\noexpand\arabic{level\@ncp{1}}}
    \noexpand\@namedef{level\@ncp{1}mark}####1{}%
    \noexpand\@namedef{l@level\@ncp{1}}%
      {\noexpand\@dottedtocline{\@ncp{1}}{0pt}{\the\dimexpr\@leveldim*\@ncp{5}+\@dotsdim*\@ncp{0}\relax}}}%
  \x
  \advance\count@\@ne
\repeat
\makeatother
\setcounter{secnumdepth}{100}
\setcounter{tocdepth}{100}
%****************FINE GESTIONE SUBSECTION MULTIPLE

%impostazioni relative alla visualizzazione delle section 
%nell'indice
\titlecontents{section}
[0pt]%left indent
{\bfseries}
{\contentslabel{2.3em}}
{\hspace*{-2.3em}}
{\hfill\contentspage}
[]%separator


\oddsidemargin=.15in
\evensidemargin=.15in
\textwidth=6in
\topmargin=-.5in
\parindent=0in
\headheight=1in
\DeclareMathSizes{10}{10}{10}{10} %per piano qualifica
\pagestyle{fancy}
\lhead{
\bfseries {\Large \TipoDocumento}\\
\bfseries Versione: \Versione\\
}
\chead{}
\lhead{
\includegraphics[scale=0.455]{../Logo&Header/apertureHead.png}
}
\lfoot{\bfseries \TipoDocumento{} v\Versione}
\cfoot{}
\rfoot{\thepage\ of \mypageref{LastPage}}
\newcommand{\mypageref}[1]{
\hypersetup{linkcolor=black}\pageref{#1}\hypersetup{linkcolor=black}}
%\userpackage{lipsum}
\renewcommand{\footrulewidth}{0.4pt}

%definizioni comandi comuni utilizzati
\newcommand{\numref}[1]{\textsl{\nameref{#1} (\ref{#1})}}
\newcommand{\NomeGruppo}{Aperture Software}
\newcommand{\Progetto}{MaaP: MongoDB as an admin Platform}
\newcommand{\Prop}{CoffeeStrap}

%definizione tecnologie
\newcommand{\Node}{Node.js}
\newcommand{\NodeJS}{Node.js}
\newcommand{\Nodejs}{Node.js}

\newcommand{\mongodb}{MongoDB}

%tanti sub quanti ne vogliamo! :)
\newcommand{\subsubsubsection}{\level{4}}
\newcommand{\subsubsubsubsection}{\level{5}}
\newcommand{\subsubsubsubsubsection}{\level{6}}
\newcommand{\subsubsubsubsubsubsection}{\level{7}}
\newcommand{\subsubsubsubsubsubsubsection}{\level{8}}


%definizione comando per parola glossario
\newcommand{\gloss}[1]{\emph{#1}\ped{\emph{\tiny{G}}}}

\newcommand{\grassetto}{\textbf}

%per inserire immagini
\newcommand{\immagine}[2]{ 
\begin{center}
\begin{figure}[H]
\includegraphics[width=\textwidth]{{{#1}}}
\caption{#2}
\label{#1}
\end{figure}
\end{center}
}

\newcommand{\Glossario}{
Al fine di evitare ogni ambiguità nella comprensione del linguaggio utilizzato nel presente documento e, in generale, nella documentazione fornita dal gruppo \NomeGruppo{}, ogni termine tecnico, di difficile comprensione o di necessario approfondimento verrà inserito nel documento \emph{Glossario\_{}v\versioneGlossario{}.pdf}.\\
Saranno in esso definiti e descritti tutti i termini in corsivo e allo stesso tempo marcati da una lettera "G" maiuscola in pedice nella documentazione fornita.
}

\newcommand{\Prodotto}{
Lo scopo del prodotto è produrre un framework per generare interfacce web di amministrazione dei dati di business basati sullo stack \Nodejs{} e \mongodb{}.\\
L'obiettivo è quello di semplificare il lavoro allo sviluppatore che dovrà rispondere in modo rapido e standard alle richieste degli esperti di business.
}

%inizio pagina del documento 
\begin{document}
\thispagestyle{empty}

\begin{center}\centerline{
%inserisco il logo grande della prima pagina
\includegraphics[scale=0.8]{../Logo&Header/logo.png}}

%metto il link dell'email sotto al logo
%{\href{mailto:ApertureSWE@gmail.com}{\color[rgb]{0.39,0.37,0.38}%ApertureSWE@gmail.com}}\\ [3pc]

\vspace{0.5in}

%titolo del progetto
{\Huge {\Progetto}}\\[.5pc]

\underline{\hspace{6in}}\\[8pc]

{\Huge {\TipoDocumento}}\\[1pc]
%{\emph{Versione \Versione}}\\
\end{center}

%\vspace{.05in}

%\vspace{.05in}

%informazioni documento
\begin{center}
%\section{Informazioni documento}
\begin{tabular}{r|l}
%\textbf{Nome} &\TipoDocumento \\
\textbf{Versione} & \Versione{} \\
\textbf{Data creazione} & \Data{} \\
\textbf{Data ultima modifica} & \DataUltimaModifica{} \\
\textbf{Stato del Documento} & Formale \\		%CAMBIARE QUI
\textbf{Uso del Documento} & Esterno \\			%CAMBIARE QUI
\textbf{Redazione} & Alberto Garbui\\			%CAMBIARE QUI
\textbf{Verifica} & Fabio Miotto, Michele Maso\\%ED ANCHE QUI!
\textbf{Approvazione} & Mattia\\				%CAMBIARE QUI
\textbf{Distribuzione} & \parbox[t]{4cm}{\NomeGruppo{} \\ Prof. Tullio Vardanega \\ Prof. Riccardo Cardin \\ \Prop{} }\\
\end{tabular}
\end{center}

\vspace{0.05in}

%inizio sommario del documento
\begin{abstract}
\begin{center}
Questo documento si propone di presentare la pianificazione del progetto MaaP.
\end{center}
\end{abstract}

%\vspace{.4in}

%seconda pagina, diario delle modifiche
\newpage
Diario delle modifiche
\begin{center}
\begin{longtable}{|c|c|c|p{0.5\linewidth}|}
\toprule
\textbf{Versione} & \textbf{Data} & \textbf{Autore} & \textbf{Modifiche effettuate}\\

%aggiungere qui una midrule per aggiungere una nuova riga alla tabella

\midrule
1.2.0 & 2013-12-18 & Alberto Garbui (AN) & Approvazione documento\\

\midrule
0.2 & 2013-12-05 & Alberto Garbui (AN) & Aggiunto capitolo 1\\
\midrule
0.1 & 2013-12-01 & Alberto Garbui (AN) & Creazione documento\\

\bottomrule
\caption{Registro delle modifiche}
\label{tab:changelog}
\end{longtable}
\end{center}

%terza pagina Indice (viene aggiornato in automatico con due compilazioni)
\newpage
\tableofcontents

%pagine successive hanno la lista di tabelle e lista delle figure
%(vengono aggiornate in automatico)
\newpage
\listoftables
\listoffigures

%qui inizia la prima pagina ufficiale
\newpage
\section{Organigramma}%1.0
\label{1.0}
\subsection{Redazione}%1.1
\label{1.1}

\subsection{Approvazione}
\label{1.2}

\subsection{Accettazione componenti}
\label{1.3}

\subsection{Componenti}
\label{1.4}

\subsection{Rotazione dei ruoli}
\label{1.5}
Durante lo sviluppo del progetto i ruoli che i membri del gruppo andranno a ricoprire saranno turnati a rotazione in modo che ogni membro abbia l'opportunità di adempire a tutti gli incarichi. Indispensabile è che non vi siano periodi in cui una stessa risorsa sia verificatrice di se stessa.

\newpage
\section{Ruoli di progetto}
\label{2.0}
Per la piena riuscita del progetto è indispensabile distinguere i vari ruoli che concorrono alla creazione del prodotto finale e le loro diverse responsabilità e competenze.
Ogni ruolo avrà una specifica area di competenza, degli specifici compiti e oneri e delle particolari autorizzazioni. Ogni componente dovrà limitarsi ai compiti ad esso assegnati e, nel caso qualcosa esuli dal suo campo di pertinenza, lo stesso dovrà rivolgersi al suo collega attualmente occupante il ruolo competente.
Tutti i membri, a rotazione, dovranno occupare come minimo una volta ciascuno dei ruoli descritti sottostante.

\subsection{Responsabile di Progetto}
\label{2.1}
Il Responsabile di Progetto incentra su di sé le responsabilità di scelta ed approvazione dei lavori. Ha inoltre il ruolo di rappresentare il gruppo nei contatti con l'esterno e durante la presentazione dei lavori.\\ \\

Le sue competenze principali comprendono
\begin{itemize}
\item Pianificazione, coordinamento e controllo delle attività;
\item Gestione e controllo delle risorse;
\item Approvazione delle analisi di gestione e rischio;
\item Approvazione dei documenti;
\item Comunicazioni con i committenti/proponenti.
\end{itemize}

Il responsabile ha il compito di assicurarsi che le attività di verifica vengano svolte sistematicamente seguendo le Norme di Progetto, che vengano rispettati i ruoli e le competenze assegnate nel Piano di Progetto e che non vi siano conflitti di interesse tra redattori e verificatori. Ha inoltre l'onere di gestire la creazione e l'assegnazione dei ticket di pianificazione e di assegnare ad un membro del gruppo il ruolo di responsabile di quest'ultimo, nel caso riguardi una sotto-attività.

\subsection{Amministratore}
\label{2.2}
L’Amministratore è il responsabile del controllo, dell’efficienza e dell’operatività dell’ambiente di lavoro e degli strumenti per la condivisone e la sincronizzazione.

\begin{itemize}
\item L'individuazione e la gestione di strumenti per automatizzare quanto più possibile processi o attività;
\item L'individuazione e la gestione di strumenti per il controllo dei processi e delle risorse;
\item L'individuazione e la gestione di strumenti e strategie per il controllo della qualità;
\item Gestione del versionamento.
\end{itemize}

\subsection{Analista}
\label{2.3}
L'analista è il responsabile dell'analisi dei requisiti di progetto. Dopo aver compreso pienamente la natura del problema e tutti i suoi domini, il suo ruolo è delineare vincoli e caratteristiche del prodotto finale, redigendo una specifica di progetto dettagliata, precisa e non ambigua, comprensibile sia dal proponente che dal progettista.

\subsection{Progettista}
\label{2.4}
Il progettista è colui che disegna una soluzione attuabile ed efficace che soddisfi i requisiti dettati dagli analisti. Il suo compito è progettare un'architettura che assicuri una facile manutenibilità del prodotto e una buona scomposizione in moduli indipendenti tra di loro.

\subsection{Verificatore}
\label{2.5}
Il Verificatore è responsabile delle attività di verifica. Ha il compito di assicurare che i  documenti e il codice rispettino gli standard qualitativi precedentemente definiti utilizzando gli strumenti e i metodi proposti dal Piano di Qualifica e attenendosi a quanto descritto nelle Norme di Progetto. 

\subsection{Programmatore}
\label{2.6}
Il Programmatore è responsabile delle attività di codifica e delle componenti di ausilio
necessarie per l'esecuzione delle prove di verifica e validazione.\\
Le responsabilità di  tale ruolo sono:
\begin{itemize}
\item wudheudhe
\end{itemize}


\newpage
\section{Introduzione}
\label{3.0}

\subsection{Scopo del documento}
\label{3.1}

\subsection{Scopo del prodotto}
\label{3.2}

\subsection{Glossario}
\label{3.3}
\Glossario{}

\subsection{Riferimenti}
\label{3.4}

\begin{itemize}
\item Slide dell'insegnamento Ingegneria del Software modulo A:\\
Ingegneria dei requisiti: \url{http://www.math.unipd.it/~tullio/IS-1/2013/Dispense/L06.pdf}
\item Software Engineering - Ian Sommerville - 9th Edition (2010):\\
Chapter 4: Requirements engineering.
\item UMLG Distilled - Martin Fowler - 4a Edizione (2010):\\
Capitolo 9: Casi d'uso.
\item Dall'idea al codice con UMLG 2 - L. Baresi, L. Lavazza, M. Pianciamore- 1a Edizione (2006):\\
Capitolo 3: Analisi dei requisiti.
\item IEEE 830-1998: Recommended Practice for Software Requirements Specifications.\\
\url{http://en.wikipedia.org/wiki/Software_requirements_specification}
\end{itemize}

\subsection{Ciclo di vita}
\label{3.5}


\subsection{Scadenze}
\label{3.6}


\section{Pianificazione}
\label{4.0}



%******************************************************************************************************

\subsection{Analisi}
\label{4.1}

\subsubsection{Diagramma di Gantt delle attività}
\label{4.1.1}
\immagine{analisi}{Diagramma di Gantt, fase di Analisi}

\subsubsection{Diagramma WBS delle attività}
\label{4.1.2}
%\immagine{}{Diagramma di Gantt, fase di Analisi}



\immagine{analisidettaglio}{Diagramma di Gantt, fase di Analisi}

\immagine{progett}{Diagramma di Gantt, fase di Analisi}
\immagine{progcodifica}{Diagramma di Gantt, fase di Analisi}



\immagine{verifica}{Diagramma di Gantt, fase di Analisi}










%parte finale, analisi dei rischi
%\input{./analisiDeiRischi.tex}

%FINE DOCUMENTO NON CANCELLARE
\end{document}