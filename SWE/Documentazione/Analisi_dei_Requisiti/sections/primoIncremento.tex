%tabella requisiti codice requisito, descrizione, fonte, UC di riferimento

\subsection{Primo incremento}
\begin{longtable}{|c|p{6cm}|c|c|}

\label{tab:Requisiti MaaP} \\
\toprule
\multicolumn{1}{|c}{\textbf{Requisito}} & \multicolumn{1}{|p{6cm}}{\textbf{Descrizione}}   & \multicolumn{1}{|c}{\textbf{Fonte}} & \multicolumn{1}{|c|}{\textbf{Caso d'uso}}\\
\midrule
\endfirsthead
\multicolumn{2}{l}{\footnotesize\itshape\tablename~\thetable: continua dalla pagina precedente} \\
\toprule
\multicolumn{1}{|c}{\textbf{Requisito}} & \multicolumn{1}{|p{6cm}}{\textbf{Descrizione}}   & \multicolumn{1}{|c}{\textbf{Fonte}} & \multicolumn{1}{|c|}{\textbf{Caso d'uso}}\\
\midrule
\endhead
\midrule
\multicolumn{2}{r}{\footnotesize\itshape\tablename~\thetable: continua nella prossima pagina} \\
\endfoot
\bottomrule
\multicolumn{2}{r}{\footnotesize\itshape\tablename~\thetable: si conclude dalla pagina precedente} \\
\endlastfoot

% Requisiti utente sviluppatore

\midrule
ROF1
& Il sistema MaaP deve essere in grado di generare lo scheletro del progetto
& Capitolato
& UC1 \\
& & & UC1.1
\\

\midrule
ROF1.1
& Il sistema MaaP deve generare le librerie necessarie al funzionamento del progetto
& Capitolato
&
\\
\midrule
ROF1.2
& Il sistema MaaP deve generare i file di configurazione necessari al funzionamento del progetto
& Capitolato
&
\\
\midrule
ROF1.3
& Il sistema MaaP deve generare le directory di descrizione delle pagine web
& Capitolato
&
\\

\midrule
ROF3
& Il sistema MaaP deve permette all'utente sviluppatore di inserire un file di descrizione
& Interna
& UC1\\
& & & UC1.5
\\

\midrule
ROF4
& Il sistema deve permette all'utente sviluppatore di utilizzare un file di descrizione
& Capitolato
& UC1 \\
& & & UC1.2
\\

\midrule
ROF4.1
& Il sistema MaaP deve permettere all'utente sviluppatore di creare la visualizzazione della Collection
& Capitolato
& UC1.2 \\
& & & UC1.2.1
\\

\midrule
ROF4.1.2
& Il sistema MaaP deve permettere all'utente sviluppatore di creare la visualizzazione della pagina Collection-Index
& Capitolato
& UC1.2.1\\
& & & UC1.2.1.2
\\
\midrule
ROF4.1.2.1
& Il sistema MaaP deve permettere all'utente sviluppatore di aggiungere delle chiavi da visualizzare nella pagina Collection-Index
& Capitolato
& UC1.2.1.2\\
& & & UC1.2.1.2.1
\\

\midrule
ROF4.1.3
& Il sistema MaaP deve permettere all'utente sviluppatore di creare la visualizzazione per la pagine Document-Show
& Capitolato
& UC1.2.1\\
& & & UC1.2.1.3
\\
\midrule
ROF4.1.3.1
& Il sistema MaaP deve permettere all'utente sviluppatore di aggiungere delle chiavi da visualizzare nella pagina Document-Show
& Capitolato
& UC1.2.1.3\\
& & & UC1.2.1.3.1
\\

\midrule
ROF4.2
& Il sistema MaaP deve permettere all'utente sviluppatore di modificare la visualizzazione della Collection
& Capitolato
& UC1.2\\
& & & UC1.2.2
\\

\midrule
ROF4.2.2
& Il sistema MaaP deve permettere all'utente sviluppatore di impostare la visualizzazione della pagina Collection-Index
& Capitolato
& UC1.2.2\\
& & & UC1.2.2.2
\\

\midrule
ROF4.2.2.1
& Il sistema MaaP deve permettere all'utente sviluppatore di aggiungere delle chiavi da visualizzare nella pagina Collection-Index
& Capitolato
& UC1.2.1.2\\
& & & UC1.2.1.2.1
\\

\midrule
ROF4.2.2.2
& Il sistema MaaP deve permettere all'utente sviluppatore di eliminare delle chiavi da visualizzare nella pagina Collection-Index
& Interna
& UC1.2.2.2\\
& & & UC1.2.2.2.1
\\

\midrule
ROF4.2.3
& Il sistema MaaP deve permettere all'utente sviluppatore di impostare la visualizzazione per la pagine Document-Show
& Capitolato
& UC1.2.2\\
& & & UC1.2.2.3
\\
\midrule
ROF4.2.3.1
& Il sistema MaaP deve permettere all'utente sviluppatore di aggiungere delle chiavi da visualizzare nella pagina Document-Show
& Capitolato
& UC1.2.1.3\\
& & & UC1.2.1.3.1
\\
\midrule
ROF4.2.3.2
& Il sistema MaaP deve permettere all'utente sviluppatore di eliminare delle chiavi da visualizzare nella pagina Document-Show
& Capitolato
& UC1.2.2.3\\
& & & UC1.2.2.3.1
\\

\midrule
ROF5.4
& Il sistema deve permettere all'utente sviluppatore di specificare nome, indirizzo e password, relativi al database di analisi con il quale interagire
& Interna
& UC1.3\\
& & & UC1.3.4
\\

\midrule
ROF10
& L'utente business autenticato deve poter aprire una Collection e visualizzare la sua pagina Collection-Index
& Capitolato
& UC2\\
& & & UC2.4
\\

\midrule
ROF10.1
& L'utente business autenticato deve poter visualizzare una pagina Document-Show
& Capitolato
& UC2.4\\
& & & UC2.4.1
\\

\midrule
ROF10.2.5
& L'utente business autenticato deve poter navigare tra la Collection
& Capitolato
& UC2.4\\
& & & UC2.4.3\\

\midrule
ROF12
& Definizione di un linguaggio astratto DSL per la definizione delle pagine che verranno generate
& Capitolato
&
\\

\midrule
ROF12.1
& Il linguaggio definito deve essere testuale
& Capitolato
&
\\

\midrule
ROV17
& Le pagine web prodotte dal framework MaaP devono essere compatibili con la versione 30.0.x di \gloss{Google Chrome} o superiori
& Capitolato
&
\\

\midrule
ROV18
& Le pagine web prodotte dal framework MaaP devono essere compatibili con la versione 24.x o superiore di Firefox
& Capitolato
&
\\

\midrule
ROV19
& Il sistema deve accettare solo file di configurazione che hanno un determinato formato già fissato
& Interna
&
\\

\midrule
ROV20
& Il database degli utenti deve essere realizzato utilizzando MongoDB
& Capitolato
&
\\

\midrule
ROV21
& Il database degli utenti deve essere indipendente dal database di analisi
& Capitolato
&
\\

\midrule
ROV22
& Il database di analisi utilizzato deve essere stato realizzato utilizzando MongoDB
& Capitolato
&
\\

\midrule
ROV23
& L'\gloss{interfaccia}  con il database deve essere realizzata con Mongoose
& Capitolato
&
\\

\midrule
ROV24
& L'\gloss{infrastruttura} delle pagine web generate deve essere realizzata con \gloss{Express}
& Capitolato
&
\\

\midrule
ROV25
& La componente \gloss{server} deve essere realizzata con \gloss{Node.js}
& Capitolato
&
\\

\midrule
ROV26
& Il software sarà fornito di un sistema di installazione per farlo funzionare
& Interna
&
\\
\end{longtable}


