\newpage
\section{Rendiconto delle attività di verifica}
\label{6.1}
\subsection{Revisione dei Requisiti}
\label{6.1.1}
\subsection{Documenti}
Di seguito vengono riportati gli indici di Gulpease calcolati per i vari documenti durante la fase di Analisi. L'esito di ciascun documento sarà superato o bocciato in base all'indice di Gulpease stabilito nel paragrafo 4.1.1.

\begin{longtable}{|c|p{3cm}|p{3cm}|}
\toprule
\textbf{Documento} & \textbf{Valore indice} & \textbf{Esito} \\

%aggiungere qui una midrule per aggiungere una nuova riga alla tabella

\midrule
\emph{Studio Fattibilità v1.2.0} & 46 & Sufficiente\\
\midrule
\emph{Analisi dei Requisiti v1.2.0} & 52& Superato\\
\midrule
\emph{Glossario v1.2.0} & 46 & Sufficiente\\
\midrule
\emph{Norme di Progetto v1.2.0} & 52 & Superato\\
\midrule
\emph{Piano di Progetto v1.2.0} & 50 & Superato\\
\midrule
\emph{Piano di Qualifica v1.2.0} & 47 & Sufficiente\\
\bottomrule
\caption{Esiti dell'indice di Gulpease calcolato sui documenti durante l'Analisi}
\label{tab:changelog}
\end{longtable}

\section{Descrizione}%B.0
\label{3}
\subsection{Indice Gulpease} %B.1
\label{3.1}
L'\grassetto{Indice Gulpease} è un indice di leggibilità di un testo tarato sulla lingua italiana. Rispetto ad altri ha il vantaggio di utilizzare la lunghezza delle parole in lettere anziché in sillabe, semplificandone il calcolo automatico.\\
Definito nel 1988 nell'ambito delle ricerche del GULP (Gruppo Universitario Linguistico Pedagogico) presso il Seminario di Scienze dell'Educazione dell'Università degli studi di Roma "La Sapienza", si basa su rilevazioni raccolte tra il 1986 e il 1987 dalle cattedre di Filosofia del linguaggio e di Pedagogia dell'Istituto di Filosofia.\\
L'indice di Gulpease considera due variabili linguistiche: la lunghezza della parola e la lunghezza della frase rispetto al numero delle lettere.\\
La formula per il suo calcolo è la seguente:
\\
\begin{center}
\begin{math}
89+\frac{300*(numero\:delle\:frasi)-10*(numero\:delle\:lettere)}{numero\:delle\:parole}
\end{math}
\end{center}
.\\
I risultati sono compresi tra 0 e 100, dove il valore "100" indica la leggibilità più alta e "0" la leggibilità più bassa. In generale risulta che testi con un indice
\begin{itemize}
\item inferiore a 80 sono difficili da leggere per chi ha la licenza elementare;
\item inferiore a 60 sono difficili da leggere per chi ha la licenza media;
\item inferiore a 40 sono difficili da leggere per chi ha un diploma superiore.
\end{itemize}

\subsection{Metriche software} %B.3
\subsubsection{Complessità ciclomatica}
La complessità ciclomatica (o complessità condizionale) è una metrica software. Sviluppata da Thomas J. McCabe nel 1976, è utilizzata per misurare la complessità di un programma. Misura direttamente il numero di cammini linearmente indipendenti attraverso il grafo di controllo di flusso.
La complessità ciclomatica è calcolata utilizzando il grafo di controllo di flusso del programma: i nodi del grafo corrispondono a gruppi indivisibili di istruzioni, mentre gli archi connettono due nodi se il secondo gruppo di istruzioni può essere eseguito immediatamente dopo il primo gruppo. La complessità ciclomatica può, inoltre, essere applicata a singole funzioni, moduli, metodi o classi di un programma.
La complessità ciclomatica di una sezione di codice è il numero di cammini linearmente indipendenti attraverso il \gloss{codice sorgente}. Per esempio, se il codice sorgente non contiene punti decisionali come IF o cicli FOR, allora la complessità sarà 1, poiché esiste un solo cammino nel sorgente (e quindi nel grafo). Se il codice ha un singolo IF contenente una singola condizione, allora ci saranno due cammini possibili: il primo se l'IF viene valutato a TRUE e un secondo se l'IF viene valutato a FALSE.

%\immagine{Ciclomatica}{Grafo di controllo di flusso.}

Matematicamente, la complessità ciclomatica di un programma è definita in riferimento ad un grafo diretto contenente i blocchi base di un programma con un arco tra due blocchi se il controllo può passare dal primo al secondo (il "grafo di controllo di flusso"). La complessità è quindi definita come:

$$v(G) = e - n + 2p$$

dove:
\begin{itemize}
\item \grassetto{v(G)} = complessità ciclomatica del grafo G;
\item \grassetto{e} = numero di archi del grafo;
\item \grassetto{n} = numero di nodi del grafo;
\item \grassetto{p} = numero di componenti connesse.
\end{itemize}
