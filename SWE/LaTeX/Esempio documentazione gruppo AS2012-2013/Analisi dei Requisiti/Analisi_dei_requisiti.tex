\newcommand{\Versione}{4.0}%Versione Finale
\newcommand{\Data}{2013-01-21}%Data di creazione
\newcommand{\TipoDocumento}{Analisi dei Requisiti}

\newcommand{\json}{JSON}
\newcommand{\treds}{3DS}
\newcommand{\xml}{XML}
\newcommand{\obj}{OBJ}
\newcommand{\mtl}{MTL}

%questo file contiene impostazioni comuni per tutte i documenti

%definizione packages utilizzati
\documentclass[a4paper]{article}
\usepackage[utf8x]{inputenc}
\usepackage{enumitem}
\usepackage[italian]{babel}
\usepackage{latexsym}
\usepackage{xparse}
\usepackage{float}
\usepackage{subfloat}
\usepackage{subfig}
\usepackage{fancyhdr}
\usepackage{eurofont}
\usepackage{lastpage}
\usepackage{graphicx}
\usepackage{textcomp}
\usepackage{booktabs}
\usepackage{color}
\usepackage{lscape}
\usepackage{hyperref}
\hypersetup{colorlinks=true, linkcolor=black, anchorcolor=red, urlcolor=blue}
\usepackage{longtable}
\usepackage{tabularx}
\usepackage{abstract}
\usepackage{appendix}
\usepackage{multicol}
\usepackage{bmpsize}
\usepackage[all]{hypcap}
\usepackage{titlesec}
\usepackage{indentfirst}
\usepackage{lipsum,titletoc}

%\setcounter{secnumdepth}{4}

%****************INIZIO GESTIONE SUBSECTION MULTIPLE
\makeatletter
\newcommand\level[1]{%
  \ifcase#1\relax\expandafter\chapter\or
    \expandafter\section\or
    \expandafter\subsection\or
    \expandafter\subsubsection\else
    \def\next{\@level{#1}}\expandafter\next
  \fi}
\newcommand{\@level}[1]{%
  \@startsection{level#1}
    {#1}
    {\z@}%
    {-3.25ex\@plus -1ex \@minus -.2ex}%
    {1.5ex \@plus .2ex}%
    {\normalfont\normalsize\bfseries}}

\newdimen\@leveldim
\newdimen\@dotsdim
{\normalfont\normalsize
 \sbox\z@{0}\global\@leveldim=\wd\z@
 \sbox\z@{.}\global\@dotsdim=\wd\z@
}

\newcounter{level4}[subsubsection]
\@namedef{thelevel4}{\thesubsubsection.\arabic{level4}}
\@namedef{level4mark}#1{}
\def\l@section{\@dottedtocline{1}{0pt}{\dimexpr\@leveldim*4+\@dotsdim*1+6pt\relax}}
\def\l@subsection{\@dottedtocline{2}{0pt}{\dimexpr\@leveldim*5+\@dotsdim*2+6pt\relax}}
\def\l@subsubsection{\@dottedtocline{3}{0pt}{\dimexpr\@leveldim*6+\@dotsdim*3+6pt\relax}}
\@namedef{l@level4}{\@dottedtocline{4}{0pt}{\dimexpr\@leveldim*7+\@dotsdim*4+6pt\relax}}

\count@=4
\def\@ncp#1{\number\numexpr\count@+#1\relax}
\loop\ifnum\count@<100
  \begingroup\edef\x{\endgroup
    \noexpand\newcounter{level\@ncp{1}}[level\number\count@]
    \noexpand\@namedef{thelevel\@ncp{1}}{%
      \noexpand\@nameuse{thelevel\@ncp{0}}.\noexpand\arabic{level\@ncp{1}}}
    \noexpand\@namedef{level\@ncp{1}mark}####1{}%
    \noexpand\@namedef{l@level\@ncp{1}}%
      {\noexpand\@dottedtocline{\@ncp{1}}{0pt}{\the\dimexpr\@leveldim*\@ncp{5}+\@dotsdim*\@ncp{0}\relax}}}%
  \x
  \advance\count@\@ne
\repeat
\makeatother
\setcounter{secnumdepth}{100}
\setcounter{tocdepth}{100}
%****************FINE GESTIONE SUBSECTION MULTIPLE

%impostazioni relative alla visualizzazione delle section 
%nell'indice
\titlecontents{section}
[0pt]%left indent
{\bfseries}
{\contentslabel{2.3em}}
{\hspace*{-2.3em}}
{\hfill\contentspage}
[]%separator


\oddsidemargin=.15in
\evensidemargin=.15in
\textwidth=6in
\topmargin=-.5in
\parindent=0in
\headheight=1in
\DeclareMathSizes{10}{10}{10}{10} %per piano qualifica
\pagestyle{fancy}
\lhead{
\bfseries {\Large \TipoDocumento}\\
\bfseries Versione: \Versione\\
}
\chead{}
\lhead{
\includegraphics[scale=0.455]{../Logo&Header/apertureHead.png}
}
\lfoot{\bfseries \TipoDocumento{} v\Versione}
\cfoot{}
\rfoot{\thepage\ of \mypageref{LastPage}}
\newcommand{\mypageref}[1]{
\hypersetup{linkcolor=black}\pageref{#1}\hypersetup{linkcolor=black}}
%\userpackage{lipsum}
\renewcommand{\footrulewidth}{0.4pt}

%definizioni comandi comuni utilizzati
\newcommand{\numref}[1]{\textsl{\nameref{#1} (\ref{#1})}}
\newcommand{\NomeGruppo}{Aperture Software}
\newcommand{\Progetto}{MaaP: MongoDB as an admin Platform}
\newcommand{\Prop}{CoffeeStrap}

%definizione tecnologie
\newcommand{\Node}{Node.js}
\newcommand{\NodeJS}{Node.js}
\newcommand{\Nodejs}{Node.js}

\newcommand{\mongodb}{MongoDB}

%tanti sub quanti ne vogliamo! :)
\newcommand{\subsubsubsection}{\level{4}}
\newcommand{\subsubsubsubsection}{\level{5}}
\newcommand{\subsubsubsubsubsection}{\level{6}}
\newcommand{\subsubsubsubsubsubsection}{\level{7}}
\newcommand{\subsubsubsubsubsubsubsection}{\level{8}}


%definizione comando per parola glossario
\newcommand{\gloss}[1]{\emph{#1}\ped{\emph{\tiny{G}}}}

\newcommand{\grassetto}{\textbf}

%per inserire immagini
\newcommand{\immagine}[2]{ 
\begin{center}
\begin{figure}[H]
\includegraphics[width=\textwidth]{{{#1}}}
\caption{#2}
\label{#1}
\end{figure}
\end{center}
}

\newcommand{\Glossario}{
Al fine di evitare ogni ambiguità nella comprensione del linguaggio utilizzato nel presente documento e, in generale, nella documentazione fornita dal gruppo \NomeGruppo{}, ogni termine tecnico, di difficile comprensione o di necessario approfondimento verrà inserito nel documento \emph{Glossario\_{}v\versioneGlossario{}.pdf}.\\
Saranno in esso definiti e descritti tutti i termini in corsivo e allo stesso tempo marcati da una lettera "G" maiuscola in pedice nella documentazione fornita.
}

\newcommand{\Prodotto}{
Lo scopo del prodotto è produrre un framework per generare interfacce web di amministrazione dei dati di business basati sullo stack \Nodejs{} e \mongodb{}.\\
L'obiettivo è quello di semplificare il lavoro allo sviluppatore che dovrà rispondere in modo rapido e standard alle richieste degli esperti di business.
}

%inizio pagina del documento 
\begin{document}
\thispagestyle{empty}

\begin{center}\centerline{
%inserisco il logo grande della prima pagina
\includegraphics[scale=0.8]{../Logo&Header/logo.png}}

%metto il link dell'email sotto al logo
%{\href{mailto:ApertureSWE@gmail.com}{\color[rgb]{0.39,0.37,0.38}%ApertureSWE@gmail.com}}\\ [3pc]

\vspace{0.5in}

%titolo del progetto
{\Huge {\Progetto}}\\[.5pc]

\underline{\hspace{6in}}\\[8pc]

{\Huge {\TipoDocumento}}\\[1pc]
%{\emph{Versione \Versione}}\\
\end{center}

%\vspace{.05in}

\begin{abstract}
\begin{center}
Questo documento si propone di presentare l'analisi dei requisiti che il prodotto \textbf{3DMob} dovrà rispettare, individuati a partire dal capitolato d'appalto del Proponente \Prop{}
\end{center}
\end{abstract}
\begin{center}
\vspace{.0in}
\section*{\\Informazioni documento}
\begin{tabular}{r|l}
\textbf{Nome} &\TipoDocumento \\%CAMBIARE QUI
\textbf{Versione} & \Versione\\
\textbf{Data creazione} & 2013-01-21\\
\textbf{Data ultima modifica} & 2013-07-28 \\%CAMBIARE QUI
\textbf{Stato del Documento} & Formale \\%CAMBIARE QUI
\textbf{Uso del Documento} & Esterno \\%CAMBIARE QUI
\textbf{Redazione} & Pietro Giacomazzi\\
\textbf{Verifica} & Alexandru Prigoreanu, Alessio Fasolo\\
\textbf{Approvazione} & Enrico Bonetti Vieno\\%CAMBIARE QUI
\textbf{Distribuzione} & \parbox[t]{4cm}{\NomeGruppo{} \\ Prof. Tullio Vardanega \\ Prof. Riccardo Cardin \\ \Prop{} }\\
\end{tabular}
\end{center}
\vspace{.4in}

%TESTO DEL SOMMARIO
%\abstract{

%}
%
\newpage
Diario delle modifiche

\begin{center}
\begin{longtable}{|c|c|c|p{0.5\linewidth}|}
\toprule
\textbf{Versione} & \textbf{Data} & \textbf{Autore} & \textbf{Modifiche effettuate}\\
\midrule
4.0 & 2013-07-28 & Enrico Bonetti Vieno (RE) & Approvazione finale documento\\
\midrule
3.9 & 2013-07-27 & Alexandru Prigoreanu (VE) & Verifica documento \\
\midrule
3.8 & 2013-07-26 & Alessio Fasolo (VE) & Controllo ortografico \\
\midrule
3.1 & 2013-07-22 & Pietro Giacomazzi (PR) & Aggiunto requisito RFDe2.4\\
\midrule
3.0 & 2013-06-28 & Alessio Fasolo (RE) & Approvazione finale documento\\
\midrule
2.9 & 2013-06-27 & Enrico Brunelli (VE) & Verifica documento \\
\midrule
2.5 & 2013-06-25 & Enrico Bonetti Vieno (VE) & Controllo ortografico \\
\midrule
2.2 & 2013-06-19 & Pietro Giacomazzi (PR) & Ricollocamento dei requisiti RFDe1.2.5, RFDe1.2.6 e relativi sottorequisiti da desiderabili a opzionali\\
\midrule
2.1 & 2013-06-18 & Pietro Giacomazzi (PR) & Aggiunta caso d'uso UC1.1.1.4 e requisito associato\\
\midrule
2.0 & 2013-03-05 & Nicola Genesin (RE) & Approvazione finale documento\\
\midrule
1.9 & 2013-03-03 & Alexandru Prigoreanu (VE) & Controllo ortografico e contenuti\\
\midrule
1.5 & 2013-03-01 & Enrico Bonetti Vieno (PR) & Apportate ulteriori correzioni a varie sezioni\\
\midrule
1.4 & 2013-02-25 &  Pietro Giacomazzi (AN) & Riorganizzati e corretti alcuni casi d'uso e approfonditi alcuni requisiti\\
\midrule
1.2 & 2013-02-23 &  Pietro Giacomazzi (AN) & Indicate le versioni degli ambienti di esecuzione e delle librerie\\
\midrule
1.1 & 2013-02-21 & Giulio Lovisotto (AN) & Modifica sezione \numref{2.1.1} \\
\midrule
1.0 & 2013-02-07 & Enrico Bonetti Vieno (RE) & Approvazione finale documento\\
\midrule
0.9 & 2013-02-07 & Nicola Genesin (VE) &  Verifica documento\\
\midrule
0.8 & 2013-01-03 & Giulio Lovisotto (AN) & Sottolineatura termini glossario\\
\midrule
0.7 & 2013-01-30 & Pietro Giacomazzi(AN) & Redatta sezione \numref{4.0}\\
\midrule
0.6 & 2013-01-29 & Enrico Brunelli (AN) & Redatta sezione \numref{3.0} \\
\midrule
0.5 & 2013-01-27 & Enrico Bonetti Vieno (AN) & Redatta sezione \numref{2.4} e \numref{2.5}\\
\midrule
0.4 & 2013-01-24 & Alessio Fasolo (AN) & Redatta sezione \numref{2.2} e \numref{2.3}  \\
\midrule
0.3 & 2013-01-23 & Alexander Pigoreanu (AN) &  Redatta sezione \numref{2.1}\\
\midrule
0.2 & 2013-01-22 & Alexander Pigoreanu (AN) & Redatta sezione \numref{1.0} \\
\midrule
0.1 & 2013-01-21 & Alessio Fasolo (AN) & Prima redazione documento\\

\bottomrule
\caption{Registro delle modifiche}
\label{tab:changelog}
\end{longtable}
\end{center}



\newpage
\tableofcontents

\newpage

\listoftables
\listoffigures

\newpage
\section{Introduzione}%1.0
\label{1.0}
\subsection{Scopo del documento}%1.1
\label{1.1}
Il presente documento ha lo scopo di illustrare ed analizzare i \underline{requisiti} del prodotto \textbf{3DMob} introdotti nel capitolato d'appalto C2 dal Proponente \Prop{}

\subsection{Scopo del prodotto} %1.2
\label{1.2}
\Prodotto{}
\subsection{Glossario}%1.3
\label{1.3}
\Glossario{}

\subsection{Riferimenti} %1.41
\label{1.4}
\subsubsection{Normativi} %1.4.1
\label{1.4.1}
\begin{itemize}
\item Capitolato d'appalto C2 - \Progetto{} \\
\url{http://www.math.unipd.it/~tullio/IS-1/2012/Progetto/C2.pdf}
\item Norme di Progetto: Norme\_{}di\_{}Progetto\_{}v3.0.pdf  (allegato alla presente documentazione)\\
\end{itemize}

\subsubsection{Informativi} %1.4.2
\label{1.4.2}
\begin{itemize}
\item Software Engineering - Ian Sommerville - 9th ed. (2011)
\item Glossario: Glossario\_{}v3.0.pdf (allegato alla presente documentazione)
\item SWEBOK - Chapter 2: Software Requirements: \url{http://www.computer.org/portal/web/swebok/html/ch2}
\item Introducing JSON - \url{http://www.json.org}
\item OpenGL ES - \url{http://www.learnopengles.com}
\item Precisione C++ - \url{http://www.cplusplus.com/doc/tutorial/variables/}
\item 3DS - \url{http://www.spacesimulator.net/wiki/index.php?title=Tutorials:3ds_Loader}
\item Verbale proponente: Verbale\_{}Proponente\_{}2013\_{}30\_{}01.pdf (allegato alla presente documentazione)
\end{itemize}

\newpage


\section{Descrizione generale}%2.0
\label{2.0}
\subsection{Contesto d'uso del prodotto} %2.1
\label{2.1}
\subsubsection{Modalità d'uso} %2.1.1
\label{2.1.1}
Il prodotto 3DMob è un convertitore di modelli 3D. Prevede l'importazione di un file \treds{} (oppure {}\obj + \mtl). L'utente avrà la possibilità di modificare le caratteristiche della scena 3D, seguendo un elenco di possibili azioni.	Sarà presente una finestra dedicata alla visualizzazione dell'anteprima dell'oggetto, per facilitare la modifica dello stesso. Infine l'utente potrà scegliere l'esportazione della scena 3D in un file \json{} (oppure \xml).
\subsubsection{Piattaforma d'esecuzione ed uso} %2.1.2
\label{2.1.2}
Il prodotto sarà fruibile da qualsiasi piattaforma di esecuzione che disponga di ambiente \underline{Windows} 7 \underline{x86} e verrà fornito dei file di \underline{libreria} necessari per la sua esecuzione. Non verrà richiesto nessun tipo di installazione.
\subsection{Funzioni del prodotto} %2.2
\label{2.2}
Il software, mediante un'\underline{interfaccia} grafica il più possibile semplice ed intuitiva, permetterà di:
\begin{itemize}
\item caricare un file in formato \treds{} (oppure{} \obj{} + \mtl)
\item modificare il file precedentemente caricato
\item visualizzare un'anteprima del file
\item esportare il file in formato \json{} (oppure {}\xml)
\end{itemize}
Durante la modifica sarà disponibile una lista di funzioni utilizzabili:
\begin{itemize}
\item \emph{Translate}, ovvero spostare l'intero oggetto lungo assi x, y e/o z di un valore arbitrario
\item \emph{Rotate}, ovvero ruotare l'intero oggetto di un angolo arbitrario rispetto ad uno degli assi x, y o z
\item \emph{Scale}, ovvero ingrandire o ridurre le dimensioni dell'oggetto
\item \emph{Modifica delle caratteristiche del \underline{materiale} (o di uno dei materiali)} di cui è composto il solido, ovvero modificare i valori che esprimono il modo in cui il solido reagisce alle diverse tipologie di luci
\item \emph{Modifica delle caratteristiche dell'illuminazione}, come posizione, intensità, colore e tipologia, e aggiunta di ulteriori fonti di luce non originariamente presenti nel file in input
\end{itemize}
Le modifiche, una volta effettuate, potranno essere accettate o meno prima dell'esportazione: questa possibilità permetterà di annullare cambiamenti insoddisfacenti od errati. \\
Durante la fase di esportazione sarà possibile scegliere:
\begin{itemize}
\item una esportazione compatta, ossia che non utilizza formattazione del testo ed usa parole chiave
\item una esportazione leggibile, formattata correttamente in modo che possa essere letta agilmente con un editor di testo qualsiasi
\end{itemize}
Inoltre sarà possibile scegliere la \underline{precisione} del file:
\begin{itemize}
\item \emph{float} (singola precisione, minore dimensione del file \json{} risultante)
\item \emph{double} (doppia precisione, maggiore dimensione del \json {} risultante)
\end{itemize}
Le  precisioni a cui si fa riferimento sono quelle utilizzate dal linguaggio \underline{C++}.

\subsection{Caratteristiche degli utenti}
\label{2.3}
È prevista una sola tipologia di utente che utilizzerà il prodotto. Un utente ipotetico possiede conoscenze standard sugli \underline{oggetti 3D} e sui formati dei file con cui andrà ad interagire. Nello specifico, i fruitori del prodotto sono studenti di grafica 3D o persone che si occupano di \underline{rendering} ed animazioni di oggetti tridimensionali, interessati a questo particolare applicativo di esportazione. Grazie alla presenza dell'interfaccia grafica e del manuale utente sarà reso intuitivo l'utilizzo del software.

\subsection{Vincoli generali}
\label{2.4}
Di seguito sono riportate le restrizioni sul prodotto 3DMob:
\begin{itemize}
\item deve funzionare in ambiente Windows 7 x86
\item il file \json{} deve essere usabile immediatamente dalle librerie \underline{OpenGL ES} 2.0, senza dover apportare ulteriori modifiche
\end{itemize}
Inoltre il software dovrà garantire le seguenti caratteristiche:
\begin{itemize}
\item esportazione dei materiali
\item mantenimento delle caratteristiche del solido dopo l'esportazione
\item possibilità di visualizzare l'anteprima delle modifiche
\item possibilità di visualizzare l'anteprima della \underline{rotazione}
\item modifica delle caratteristiche fondamentali
\item importazione del formato \obj
\item mantenimento della trasparenza dell'oggetto dopo l'esportazione
\item lettura di file \json
\item mantenimento della gestione delle animazioni attraverso keyframes, dopo l'esportazione
\item esportazione compatta o leggibile
\end{itemize}
\subsection{Assunzioni e dipendenze}
\label{2.5}
Il prodotto assume che nel computer in cui verrà avviato sia presente il sistema operativo Windows 7 x86.
\newpage





\section{Casi d'uso} %3

\label{3.0}
Di seguito sono presentati i casi d'uso identificati a partire dal capitolato d'appalto 3DMob.
Ogni caso d'uso figlio eredita le precondizioni dei casi d'uso di livello superiore che fanno parte della sua gerarchia al fine di snellire l'esposizione delle precondizioni.
\\ \\
\textbf{Il resto della sezione è stato generato automaticamente attraverso il sistema di tracciamento dei requisiti.}
\\ \\


\subsection{UC1 Caso d’uso generale}
\textbf{Diagramma associato:}
\ref{UC1} \\ \\
\textbf{Attori Coinvolti:}
Utente. \\ \\
\textbf{Scopo e Descrizione:}
un utente deve poter scegliere un file opportuno da cui caricare la scena 3D descritta. Una volta caricata, deve essere possibile modificarla ed esportarla nel formato desiderato. Il caricamento e la modifica della scena 3D comportano la visualizzazione dell'anteprima della stessa. \\ \\
\textbf{Precondizione:}
il sistema si trova nello stato iniziale. \\ \\
\textbf{Postcondizione:}
il sistema ha caricato in memoria la scena 3D descritta nel file selezionato. \\ \\
\textbf{Scenario Principale:}
\begin{itemize}
\item l’utente seleziona il file da cui caricare la scena 3D
\item  l’utente eventualmente decide che modifiche apportare alla scena 3D
\item  l’utente esporta la scena 3D nel formato desiderato
\item l'utente agisce sulla cronologia delle modifiche
\\ \\ \end{itemize}
\begin{figure}[h!]
\centering
\includegraphics[width=\textwidth]{{{UC1}}}
\caption{UC1 Caso d’uso generale}
\label{UC1}
\end{figure}


\subsection{UC1.1 Sceglie file da importare}
\textbf{Diagramma associato:}
\ref{UC1.1} \\ \\
\textbf{Attori Coinvolti:}
Utente. \\ \\
\textbf{Scopo e Descrizione:}
un utente deve poter selezionare il formato del file dal quale caricare la scena 3D e scegliere il percorso del file medesimo. \\ \\
\textbf{Precondizione:}
il sistema è in attesa che l'utente selezioni una funzionalità. \\ \\
\textbf{Postcondizione:}
il sistema ha caricato la scena 3D descritta nel file scelto, la scena si trova nel suo stato iniziale. \\ \\
\textbf{Scenario Principale:}
\begin{itemize}
\item  l’utente seleziona l’estensione del file da importare
\item l’utente seleziona il percorso del file
\\ \\ \end{itemize}
\begin{figure}[h!]
\centering
\includegraphics[width=\textwidth]{{{UC1.1}}}
\caption{UC1.1 Sceglie file da importare}
\label{UC1.1}
\end{figure}


\subsection{UC1.1.1 Sceglie tipo file}
\textbf{Diagramma associato:}
\ref{UC1.1.1} \\ \\
\textbf{Attori Coinvolti:}
Utente. \\ \\
\textbf{Scopo e Descrizione:}
un utente deve poter selezionare il formato del file dal quale caricare la scena 3D. \\ \\
\textbf{Precondizione:}
il sistema è in attesa di input riguardo all'importazione. \\ \\
\textbf{Postcondizione:}
il sistema conosce il formato del file scelto dall’utente. \\ \\
\textbf{Scenario Principale:}
\begin{itemize}
\item l’utente seleziona il formato 3DS
\item l’utente seleziona il formato Wavefront OBJ/MTL
\item l’utente seleziona il formato JSON
\item l'utente seleziona il formato XML
\\ \\ \end{itemize}
\begin{figure}[h!]
\centering
\includegraphics[width=\textwidth]{{{UC1.1.1}}}
\caption{UC1.1.1 Sceglie tipo file}
\label{UC1.1.1}
\end{figure}


\subsection{UC1.1.1.1 Sceglie formato 3DS}
\textbf{Diagramma associato:}
\ref{UC1.1.1} \\ \\
\textbf{Attori Coinvolti:}
Utente. \\ \\
\textbf{Scopo e Descrizione:}
un utente deve poter importare dal formato 3DS. \\ \\
\textbf{Precondizione:}
il sistema è in attesa di input riguardo al formato del file da importare. \\ \\
\textbf{Postcondizione:}
il sistema sa che si sta per importare dal formato 3DS. \\ \\
\textbf{Scenario Principale:}
\begin{itemize}
\item l'utente seleziona il formato 3DS
\\ \\ \end{itemize}


\subsection{UC1.1.1.2 Sceglie formato Wavefront OBJ/MTL}
\textbf{Diagramma associato:}
\ref{UC1.1.1} \\ \\
\textbf{Attori Coinvolti:}
Utente. \\ \\
\textbf{Scopo e Descrizione:}
un utente deve poter importare dal formato Wavefront OBJ/MTL. \\ \\
\textbf{Precondizione:}
il sistema è in attesa di input riguardo al formato del file da importare. \\ \\
\textbf{Postcondizione:}
il sistema sa che si sta per importare dal formato OBJ/MTL. \\ \\
\textbf{Scenario Principale:}
\begin{itemize}
\item l'utente seleziona il formato Wavefront OBJ/MTL
\\ \\ \end{itemize}


\subsection{UC1.1.1.3 Sceglie formato JSON}
\textbf{Diagramma associato:}
\ref{UC1.1.1} \\ \\
\textbf{Attori Coinvolti:}
Utente. \\ \\
\textbf{Scopo e Descrizione:}
un utente deve poter importare dal formato JSON. \\ \\
\textbf{Precondizione:}
il sistema è in attesa di input riguardo al formato del file da importare. \\ \\
\textbf{Postcondizione:}
il sistema sa che si sta per importare dal formato JSON. \\ \\
\textbf{Scenario Principale:}
\begin{itemize}
\item l'utente seleziona il formato JSON
\\ \\ \end{itemize}


\subsection{UC1.1.1.4 Sceglie formato XML}
\textbf{Diagramma associato:}
\ref{UC1.1.1} \\ \\
\textbf{Attori Coinvolti:}
Utente. \\ \\
\textbf{Scopo e Descrizione:}
un utente deve poter importare dal formato XML. \\ \\
\textbf{Precondizione:}
il sistema è in attesa di input riguardo al formato del file da importare. \\ \\
\textbf{Postcondizione:}
il sistema sa che si sta per importare dal formato XML. \\ \\
\textbf{Scenario Principale:}
\begin{itemize}
\item l'utente seleziona il formato XML
\\ \\ \end{itemize}


\subsection{UC1.1.2 Seleziona percorso file}
\textbf{Diagramma associato:}
\ref{UC1.1} \\ \\
\textbf{Attori Coinvolti:}
Utente. \\ \\
\textbf{Scopo e Descrizione:}
un utente deve poter selezionare il percorso del file da cui caricare la scena 3D. \\ \\
\textbf{Precondizione:}
il sistema è in attesa di input riguardo all'importazione. \\ \\
\textbf{Postcondizione:}
il sistema conosce la posizione del file dal quale caricare la scena 3D. \\ \\
\textbf{Scenario Principale:}
\begin{itemize}
\item l’utente seleziona il percorso del file dal quale caricare la scena 3D
\\ \\ \end{itemize}


\subsection{UC1.2 Modifica scena 3D}
\textbf{Diagramma associato:}
\ref{UC1.2} \\ \\
\textbf{Attori Coinvolti:}
Utente. \\ \\
\textbf{Scopo e Descrizione:}
un utente deve poter modificare la scena 3D. \\ \\
\textbf{Precondizione:}
il sistema ha in memoria la scena 3D caricata in precedenza (UC1.1), ed è in attesa che l'utente selezioni una funzionalità. \\ \\
\textbf{Postcondizione:}
il sistema ha modificato la scena 3D secondo le richieste dell’utente, e la modifica è stata inserita nella cronologia delle modifiche. La scena non è più nel suo stato iniziale. \\ \\
\textbf{Scenario Principale:}
\begin{itemize}
\item l'utente seleziona la \underline{fonte di luce} che intende modificare (UC1.2.1)
\item l'utente seleziona la \underline{mesh} che intende modificare (UC1.2.2)
\item l'utente aggiunge una nuova fonte di luce con valori di default (UC1.2.3)
\item l'utente modifica la mesh (UC1.2.4)
\item l'utente seleziona la \underline{camera} che intende modificare (UC1.2.5)
\item l'utente modifica la camera (UC1.2.6)
\item l'utente modifica la fonte di luce (UC1.2.7)
\\ \\ \end{itemize}
\begin{figure}[h!]
\centering
\includegraphics[width=\textwidth]{{{UC1.2}}}
\caption{UC1.2 Modifica scena 3D}
\label{UC1.2}
\end{figure}


\subsection{UC1.2.1 Seleziona fonte di luce}
\textbf{Diagramma associato:}
\ref{UC1.2} \\ \\
\textbf{Attori Coinvolti:}
Utente. \\ \\
\textbf{Scopo e Descrizione:}
un utente deve poter selezionare la fonte di luce che intende modificare. \\ \\
\textbf{Precondizione:}
il sistema è in attesa di input  per quanto riguarda la modifica della scena 3D. \\ \\
\textbf{Postcondizione:}
il sistema conosce la fonte di luce selezionata dall'utente. \\ \\
\textbf{Scenario Principale:}
\begin{itemize}
\item l'utente seleziona una fonte di luce
\\ \\ \end{itemize}


\subsection{UC1.2.2 Seleziona mesh}
\textbf{Diagramma associato:}
\ref{UC1.2} \\ \\
\textbf{Attori Coinvolti:}
Utente. \\ \\
\textbf{Scopo e Descrizione:}
un utente deve poter selezionare la mesh che intende modificare. \\ \\
\textbf{Precondizione:}
il sistema è in attesa di input  per quanto riguarda la modifica della scena 3D. \\ \\
\textbf{Postcondizione:}
il sistema conosce la mesh che l'utente vuole modificare. \\ \\
\textbf{Scenario Principale:}
\begin{itemize}
\item l'utente seleziona una mesh
\\ \\ \end{itemize}


\subsection{UC1.2.3 Aggiungi fonte di luce}
\textbf{Diagramma associato:}
\ref{UC1.2} \\ \\
\textbf{Attori Coinvolti:}
Utente. \\ \\
\textbf{Scopo e Descrizione:}
un utente deve poter aggiungere una nuova fonte di luce con valori di default. \\ \\
\textbf{Precondizione:}
il sistema è in attesa di input  per quanto riguarda la modifica della scena 3D. \\ \\
\textbf{Postcondizione:}
il sistema aggiunge la nuova fonte di luce con valori di default. \\ \\
\textbf{Scenario Principale:}
\begin{itemize}
\item l'utente aggiunge una nuova fonte di luce
\\ \\ \end{itemize}


\subsection{UC1.2.4 Modifica mesh}
\textbf{Diagramma associato:}
\ref{UC1.2.4} \\ \\
\textbf{Attori Coinvolti:}
Utente. \\ \\
\textbf{Scopo e Descrizione:}
un utente deve poter modificare la mesh selezionata. \\ \\
\textbf{Precondizione:}
il sistema conosce la mesh da modificare (UC1.2.2). \\ \\
\textbf{Postcondizione:}
il sistema ha modificato la mesh selezionata secondo le richieste dell'utente. \\ \\
\textbf{Scenario Principale:}
\begin{itemize}
\item l'utente effettua una \underline{traslazione} (UC1.2.4.1)
\item l'utente effettua una rotazione (UC1.2.4.2)
\item l'utente modifica la dimensione (UC1.2.4.3)
\item l'utente modifica le caratteristiche del materiale (UC1.2.4.4)
\item l'utente rimuove la mesh (UC1.2.4.5)
\\ \\ \end{itemize}
\begin{figure}[h!]
\centering
\includegraphics[width=\textwidth]{{{UC1.2.4}}}
\caption{UC1.2.4 Modifica mesh}
\label{UC1.2.4}
\end{figure}


\subsection{UC1.2.4.1 Effettua traslazione}
\textbf{Diagramma associato:}
\ref{UC1.2.4.1} \\ \\
\textbf{Attori Coinvolti:}
Utente. \\ \\
\textbf{Scopo e Descrizione:}
un utente deve poter scegliere i valori del vettore [x,y,z] di traslazione da applicare alla mesh selezionata. \\ \\
\textbf{Precondizione:}
il sistema è in attesa che l'utente selezioni una funzionalità di modifica della mesh. \\ \\
\textbf{Postcondizione:}
il sistema ha traslato la mesh secondo le richieste dell’utente. \\ \\
\textbf{Scenario Principale:}
\begin{itemize}
\item l’utente sceglie il valore della componente x del \underline{vettore di traslazione} (UC1.2.4.1.1)
\item  l’utente sceglie il valore della componente y del vettore di traslazione (UC1.2.4.1.2)
\item l’utente sceglie il valore della componente z del vettore di traslazione (UC1.2.4.1.3)
\\ \\ \end{itemize}
\begin{figure}[h!]
\centering
\includegraphics[width=\textwidth]{{{UC1.2.4.1}}}
\caption{UC1.2.4.1 Effettua traslazione}
\label{UC1.2.4.1}
\end{figure}


\subsection{UC1.2.4.1.1 Sceglie valore componente x}
\textbf{Diagramma associato:}
\ref{UC1.2.4.1} \\ \\
\textbf{Attori Coinvolti:}
Utente. \\ \\
\textbf{Scopo e Descrizione:}
un utente deve poter impostare il valore della componente x del vettore di traslazione. \\ \\
\textbf{Precondizione:}
il sistema è in attesa di input riguardo alla traslazione. \\ \\
\textbf{Postcondizione:}
il sistema conosce il valore impostato dall'utente della componente x del vettore di traslazione . \\ \\
\textbf{Scenario Principale:}
\begin{itemize}
\item l'utente imposta il valore della componente x
\\ \\ \end{itemize}


\subsection{UC1.2.4.1.2 Sceglie valore componente y}
\textbf{Diagramma associato:}
\ref{UC1.2.4.1} \\ \\
\textbf{Attori Coinvolti:}
Utente. \\ \\
\textbf{Scopo e Descrizione:}
un utente deve poter impostare il valore della componente y del vettore di traslazione. \\ \\
\textbf{Precondizione:}
il sistema è in attesa di input riguardo alla traslazione. \\ \\
\textbf{Postcondizione:}
il sistema conosce il valore impostato dall'utente della componente y del vettore di traslazione. \\ \\
\textbf{Scenario Principale:}
\begin{itemize}
\item l'utente imposta il valore della componente y
\\ \\ \end{itemize}


\subsection{UC1.2.4.1.3 Sceglie valore componente z}
\textbf{Diagramma associato:}
\ref{UC1.2.4.1} \\ \\
\textbf{Attori Coinvolti:}
Utente. \\ \\
\textbf{Scopo e Descrizione:}
un utente deve poter impostare il valore della componente z del vettore di traslazione. \\ \\
\textbf{Precondizione:}
il sistema è in attesa di input riguardo alla traslazione. \\ \\
\textbf{Postcondizione:}
il sistema conosce il valore impostato dall'utente della componente z del vettore di traslazione. \\ \\
\textbf{Scenario Principale:}
\begin{itemize}
\item l'utente imposta il valore della componente z
\\ \\ \end{itemize}


\subsection{UC1.2.4.2 Effettua rotazione}
\textbf{Diagramma associato:}
\ref{UC1.2.4.2} \\ \\
\textbf{Attori Coinvolti:}
Utente. \\ \\
\textbf{Scopo e Descrizione:}
un utente deve poter ruotare la mesh selezionata scegliendo l’asse di rotazione e l’angolo theta di rotazione. \\ \\
\textbf{Precondizione:}
il sistema è in attesa che l'utente selezioni una funzionalità di modifica della mesh. \\ \\
\textbf{Postcondizione:}
il sistema ha ruotato la mesh secondo le richieste dell’utente. \\ \\
\textbf{Scenario Principale:}
\begin{itemize}
\item l’utente sceglie l’asse di rotazione (UC1.2.4.2.1)
\item l’utente sceglie l’angolo theta di rotazione (UC1.2.4.2.2)
\\ \\ \end{itemize}
\begin{figure}[h!]
\centering
\includegraphics[width=\textwidth]{{{UC1.2.4.2}}}
\caption{UC1.2.4.2 Effettua rotazione}
\label{UC1.2.4.2}
\end{figure}


\subsection{UC1.2.4.2.1 Sceglie asse di rotazione}
\textbf{Diagramma associato:}
\ref{UC1.2.4.2.1} \\ \\
\textbf{Attori Coinvolti:}
Utente. \\ \\
\textbf{Scopo e Descrizione:}
un utente deve poter scegliere l'\underline{asse di rotazione} in base al quale ruotare la mesh selezionata. \\ \\
\textbf{Precondizione:}
il sistema è in attesa di input riguardo alla rotazione. \\ \\
\textbf{Postcondizione:}
il sistema conosce l'asse di rotazione selezionato dall'utente. \\ \\
\textbf{Scenario Principale:}
\begin{itemize}
\item l’utente sceglie l'asse x (UC1.2.4.2.1.1)
\item  l’utente sceglie l'asse y (UC1.2.4.2.1.2)
\item l’utente sceglie l'asse z (UC1.2.4.2.1.3)
\\ \\ \end{itemize}
\begin{figure}[h!]
\centering
\includegraphics[width=\textwidth]{{{UC1.2.4.2.1}}}
\caption{UC1.2.4.2.1 Sceglie asse di rotazione}
\label{UC1.2.4.2.1}
\end{figure}


\subsection{UC1.2.4.2.1.1 Sceglie asse x}
\textbf{Diagramma associato:}
\ref{UC1.2.4.2.1} \\ \\
\textbf{Attori Coinvolti:}
Utente. \\ \\
\textbf{Scopo e Descrizione:}
un utente deve poter scegliere l'asse x come asse di rotazione. \\ \\
\textbf{Precondizione:}
il sistema è in attesa di input riguardo l'asse di rotazione. \\ \\
\textbf{Postcondizione:}
il sistema sa che l'utente ha selezionato l'asse x. \\ \\
\textbf{Scenario Principale:}
\begin{itemize}
\item l'utente seleziona l'asse x
\\ \\ \end{itemize}


\subsection{UC1.2.4.2.1.2 Sceglie asse y}
\textbf{Diagramma associato:}
\ref{UC1.2.4.2.1} \\ \\
\textbf{Attori Coinvolti:}
Utente. \\ \\
\textbf{Scopo e Descrizione:}
un utente deve poter scegliere l'asse y come asse di rotazione. \\ \\
\textbf{Precondizione:}
il sistema è in attesa di input riguardo l'asse di rotazione. \\ \\
\textbf{Postcondizione:}
il sistema sa che l'utente ha selezionato l'asse y. \\ \\
\textbf{Scenario Principale:}
\begin{itemize}
\item l'utente seleziona l'asse y
\\ \\ \end{itemize}


\subsection{UC1.2.4.2.1.3 Sceglie asse z}
\textbf{Diagramma associato:}
\ref{UC1.2.4.2.1} \\ \\
\textbf{Attori Coinvolti:}
Utente. \\ \\
\textbf{Scopo e Descrizione:}
un utente deve poter scegliere l'asse z come asse di rotazione. \\ \\
\textbf{Precondizione:}
il sistema è in attesa di input riguardo l'asse di rotazione. \\ \\
\textbf{Postcondizione:}
il sistema sa che l'utente ha selezionato l'asse z. \\ \\
\textbf{Scenario Principale:}
\begin{itemize}
\item l'utente seleziona l'asse z
\\ \\ \end{itemize}


\subsection{UC1.2.4.2.2 Sceglie ampiezza angolo theta}
\textbf{Diagramma associato:}
\ref{UC1.2.4.2} \\ \\
\textbf{Attori Coinvolti:}
Utente. \\ \\
\textbf{Scopo e Descrizione:}
un utente deve poter impostare l'ampiezza dell'\underline{angolo theta di rotazione} secondo il quale routare la mesh selezionata. \\ \\
\textbf{Precondizione:}
il sistema è in attesa di input riguardo alla rotazione. \\ \\
\textbf{Postcondizione:}
il sistema conosce l'ampiezza dell'angolo theta di rotazione impostato dall'utente. \\ \\
\textbf{Scenario Principale:}
\begin{itemize}
\item l'utente imposta l'ampiezza dell'angolo theta di rotazione
\\ \\ \end{itemize}


\subsection{UC1.2.4.3 Modifica dimensione}
\textbf{Diagramma associato:}
\ref{UC1.2.4.3} \\ \\
\textbf{Attori Coinvolti:}
Utente. \\ \\
\textbf{Scopo e Descrizione:}
un utente deve poter modificare la dimensione della mesh selezionata. \\ \\
\textbf{Precondizione:}
il sistema è in attesa che l'utente selezioni una funzionalità di modifica della mesh. \\ \\
\textbf{Postcondizione:}
il sistema ha effettuato il ridimensionamento della mesh selezionata secondo le richieste dell’utente. \\ \\
\textbf{Scenario Principale:}
\begin{itemize}
\item l’utente sceglie il ridimensionamento secondo gli assi (UC1.2.4.3.1)
\item l’utente sceglie il \underline{ridimensionamento scalare} (UC1.2.4.3.2)
\\ \\ \end{itemize}
\begin{figure}[h!]
\centering
\includegraphics[width=\textwidth]{{{UC1.2.4.3}}}
\caption{UC1.2.4.3 Modifica dimensione}
\label{UC1.2.4.3}
\end{figure}


\subsection{UC1.2.4.3.1 Ridimensiona secondo gli assi}
\textbf{Diagramma associato:}
\ref{UC1.2.4.3.1} \\ \\
\textbf{Attori Coinvolti:}
Utente. \\ \\
\textbf{Scopo e Descrizione:}
un utente deve poter ridimensionare la mesh selezionata secondo i suoi assi. \\ \\
\textbf{Precondizione:}
il sistema propone all'utente una scelta sul tipo di ridimensionamento. \\ \\
\textbf{Postcondizione:}
il sistema ha ridimensionato la mesh secondo le impostazioni dell'utente. \\ \\
\textbf{Scenario Principale:}
\begin{itemize}
\item l'utente sceglie il valore della componente x (UC1.2.4.3.1.1)
\item l'utente sceglie il valore della componente y (UC1.2.4.3.1.2)
\item l'utente sceglie il valore della componente z (UC1.2.4.3.1.3)
\\ \\ \end{itemize}
\begin{figure}[h!]
\centering
\includegraphics[width=\textwidth]{{{UC1.2.4.3.1}}}
\caption{UC1.2.4.3.1 Ridimensiona secondo gli assi}
\label{UC1.2.4.3.1}
\end{figure}


\subsection{UC1.2.4.3.1.1 Sceglie valore componente x}
\textbf{Diagramma associato:}
\ref{UC1.2.4.3.1} \\ \\
\textbf{Attori Coinvolti:}
Utente. \\ \\
\textbf{Scopo e Descrizione:}
un utente deve poter impostare il valore della componente x. \\ \\
\textbf{Precondizione:}
il sistema è in attesa di input riguardo al ridimensionamento della mesh. \\ \\
\textbf{Postcondizione:}
il sistema conosce il valore della componente x impostato dall'utente. \\ \\
\textbf{Scenario Principale:}
\begin{itemize}
\item l'utente imposta il valore della componente x
\\ \\ \end{itemize}


\subsection{UC1.2.4.3.1.2 Sceglie valore componente y}
\textbf{Diagramma associato:}
\ref{UC1.2.4.3.1} \\ \\
\textbf{Attori Coinvolti:}
Utente. \\ \\
\textbf{Scopo e Descrizione:}
un utente deve poter impostare il valore della componente y. \\ \\
\textbf{Precondizione:}
il sistema è in attesa di input riguardo al ridimensionamento della mesh. \\ \\
\textbf{Postcondizione:}
il sistema conosce il valore della componente y impostato dall'utente. \\ \\
\textbf{Scenario Principale:}
\begin{itemize}
\item l'utente imposta il valore della componente y
\\ \\ \end{itemize}


\subsection{UC1.2.4.3.1.3 Sceglie valore componente z}
\textbf{Diagramma associato:}
\ref{UC1.2.4.3.1} \\ \\
\textbf{Attori Coinvolti:}
Utente. \\ \\
\textbf{Scopo e Descrizione:}
un utente deve poter impostare il valore della componente z. \\ \\
\textbf{Precondizione:}
il sistema è in attesa di input riguardo al ridimensionamento della mesh. \\ \\
\textbf{Postcondizione:}
il sistema conosce il valore della componente z impostato dall'utente. \\ \\
\textbf{Scenario Principale:}
\begin{itemize}
\item l'utente imposta il valore della componente z
\\ \\ \end{itemize}


\subsection{UC1.2.4.3.2 Ridimensiona scalarmente}
\textbf{Diagramma associato:}
\ref{UC1.2.4.3.2} \\ \\
\textbf{Attori Coinvolti:}
Utente. \\ \\
\textbf{Scopo e Descrizione:}
un utente deve poter ridimensionare la mesh impostando un moltiplicatore scalare. \\ \\
\textbf{Precondizione:}
il sistema propone all'utente una scelta sul tipo di ridimensionamento. \\ \\
\textbf{Postcondizione:}
il sistema ha ridimensionato (scalarmente) la mesh secondo l'input utente. \\ \\
\textbf{Scenario Principale:}
\begin{itemize}
\item l'utente sceglie il ridimensionamento scalare
\\ \\ \end{itemize}
\begin{figure}[h!]
\centering
\includegraphics[width=\textwidth]{{{UC1.2.4.3.2}}}
\caption{UC1.2.4.3.2 Ridimensiona scalarmente}
\label{UC1.2.4.3.2}
\end{figure}


\subsection{UC1.2.4.3.2.1 Sceglie componente scalare}
\textbf{Diagramma associato:}
\ref{UC1.2.4.3.2} \\ \\
\textbf{Attori Coinvolti:}
Utente. \\ \\
\textbf{Scopo e Descrizione:}
un utente deve poter impostare il moltiplicatore scalare relativo al ridimensionamento di una mesh. \\ \\
\textbf{Precondizione:}
il sistema è in attesa di input riguardo al ridimensionamento della mesh. \\ \\
\textbf{Postcondizione:}
il sistema conosce il valore del moltiplicatore scalare impostato dall'utente. \\ \\
\textbf{Scenario Principale:}
\begin{itemize}
\item l'utente sceglie il valore della componente scalare
\\ \\ \end{itemize}


\subsection{UC1.2.4.4 Modifica caratteristiche materiale}
\textbf{Diagramma associato:}
\ref{UC1.2.4.4} \\ \\
\textbf{Attori Coinvolti:}
Utente. \\ \\
\textbf{Scopo e Descrizione:}
un utente deve poter modificare le caratteristiche di uno dei materiali che compongono la mesh selezionata. \\ \\
\textbf{Precondizione:}
il sistema è in attesa che l'utente selezioni una funzionalità di modifica della mesh. \\ \\
\textbf{Postcondizione:}
il sistema ha effettuato le modifiche a uno dei materiali che compongono la mesh selezionata secondo le richieste dell’utente. \\ \\
\textbf{Scenario Principale:}
\begin{itemize}
\item l'utente modifica la componente riflessiva diffusa (UC1.2.4.4.1)
\item l'utente modifica la componente emissiva (UC1.2.4.4.2)
\item l'utente modifica la componente riflessiva speculare (UC1.2.4.4.3)
\item l'utente modifica la componente riflessiva ambientale (UC1.2.4.4.4)
\item l'utente modifica il parametro di \underline{opacità} (UC1.2.4.4.5)
\\ \\ \end{itemize}
\begin{figure}[h!]
\centering
\includegraphics[width=\textwidth]{{{UC1.2.4.4}}}
\caption{UC1.2.4.4 Modifica caratteristiche materiale}
\label{UC1.2.4.4}
\end{figure}


\subsection{UC1.2.4.4.1 Modifica componente riflessiva diffusa}
\textbf{Diagramma associato:}
\ref{UC1.2.4.4} \\ \\
\textbf{Attori Coinvolti:}
Utente. \\ \\
\textbf{Scopo e Descrizione:}
un utente deve poter modificare la \underline{componente riflessiva diffusa di un materiale}. \\ \\
\textbf{Precondizione:}
il sistema conosce il materiale da modificare ed è in attesa di input per quanto riguarda la modifica dello stesso. \\ \\
\textbf{Postcondizione:}
il sistema conosce il nuovo valore della componente riflessiva diffusa impostato dall'utente. \\ \\
\textbf{Scenario Principale:}
\begin{itemize}
\item l'utente imposta il valore della componente riflessiva diffusa del materiale selezionato
\\ \\ \end{itemize}


\subsection{UC1.2.4.4.2 Modifica componente emissiva}
\textbf{Diagramma associato:}
\ref{UC1.2.4.4} \\ \\
\textbf{Attori Coinvolti:}
Utente. \\ \\
\textbf{Scopo e Descrizione:}
un utente deve poter modificare la \underline{componente emissiva di un materiale}. \\ \\
\textbf{Precondizione:}
il sistema conosce il materiale da modificare ed è in attesa di input per quanto riguarda la modifica dello stesso. \\ \\
\textbf{Postcondizione:}
il sistema conosce il nuovo valore della componente emissiva impostato dall'utente. \\ \\
\textbf{Scenario Principale:}
\begin{itemize}
\item l'utente imposta il valore della componente emissiva del materiale selezionato
\\ \\ \end{itemize}


\subsection{UC1.2.4.4.3 Modifica componente riflessiva speculare}
\textbf{Diagramma associato:}
\ref{UC1.2.4.4} \\ \\
\textbf{Attori Coinvolti:}
Utente. \\ \\
\textbf{Scopo e Descrizione:}
un utente deve poter modificare la \underline{componente riflessiva speculare di un materiale}. \\ \\
\textbf{Precondizione:}
il sistema conosce il materiale da modificare ed è in attesa di input per quanto riguarda la modifica dello stesso. \\ \\
\textbf{Postcondizione:}
il sistema conosce il nuovo valore della componente riflessiva speculare impostato dall'utente. \\ \\
\textbf{Scenario Principale:}
\begin{itemize}
\item l'utente imposta il valore della componente riflessiva speculare del materiale selezionato
\\ \\ \end{itemize}


\subsection{UC1.2.4.4.4 Modifica componente riflessiva ambientale}
\textbf{Diagramma associato:}
\ref{UC1.2.4.4} \\ \\
\textbf{Attori Coinvolti:}
Utente. \\ \\
\textbf{Scopo e Descrizione:}
un utente deve poter modificare la \underline{componente riflessiva ambientale di un materiale}. \\ \\
\textbf{Precondizione:}
il sistema conosce il materiale da modificare ed è in attesa di input per quanto riguarda la modifica dello stesso. \\ \\
\textbf{Postcondizione:}
il sistema conosce il nuovo valore della componente riflessiva ambientale impostato dall'utente. \\ \\
\textbf{Scenario Principale:}
\begin{itemize}
\item l'utente imposta il valore della componente riflessiva ambientale del materiale selezionato
\\ \\ \end{itemize}


\subsection{UC1.2.4.4.5 Modifica parametro opacità}
\textbf{Diagramma associato:}
\ref{UC1.2.4.4} \\ \\
\textbf{Attori Coinvolti:}
Utente. \\ \\
\textbf{Scopo e Descrizione:}
un utente deve poter modificare il valore di opacità del materiale selezionato. \\ \\
\textbf{Precondizione:}
il sistema conosce il materiale da modificare ed è in attesa di input per quanto riguarda la modifica dello stesso. \\ \\
\textbf{Postcondizione:}
il sistema conosce il nuovo valore del parametro di opacità impostato dall'utente. \\ \\
\textbf{Scenario Principale:}
\begin{itemize}
\item l'utente imposta il valore del parametro di opacità associato al materiale selezionato
\\ \\ \end{itemize}


\subsection{UC1.2.4.5 Rimuovi mesh}
\textbf{Diagramma associato:}
\ref{UC1.2.4} \\ \\
\textbf{Attori Coinvolti:}
Utente. \\ \\
\textbf{Scopo e Descrizione:}
un utente deve poter rimuovere la mesh selezionata. \\ \\
\textbf{Precondizione:}
il sistema è in attesa che l'utente selezioni una funzionalità di modifica della mesh. \\ \\
\textbf{Postcondizione:}
il sistema ha rimosso la mesh selezionata in precedenza secondo le richieste utente. \\ \\
\textbf{Scenario Principale:}
\begin{itemize}
\item l'utente rimuove la mesh
\\ \\ \end{itemize}


\subsection{UC1.2.5 Seleziona camera}
\textbf{Diagramma associato:}
\ref{UC1.2} \\ \\
\textbf{Attori Coinvolti:}
Utente. \\ \\
\textbf{Scopo e Descrizione:}
un utente deve poter selezionare la camera che intende modificare. \\ \\
\textbf{Precondizione:}
il sistema è in attesa di input  per quanto riguarda la modifica della scena 3D. \\ \\
\textbf{Postcondizione:}
il sistema conosce la camera selezionata dall'utente. \\ \\
\textbf{Scenario Principale:}
\begin{itemize}
\item l'utente seleziona la camera che intende modificare
\\ \\ \end{itemize}


\subsection{UC1.2.6 Modifica camera}
\textbf{Diagramma associato:}
\ref{UC1.2.6} \\ \\
\textbf{Attori Coinvolti:}
Utente. \\ \\
\textbf{Scopo e Descrizione:}
un utente deve poter modificare la camera selezionata. \\ \\
\textbf{Precondizione:}
il sistema conosce la camera da modificare (UC1.2.5). \\ \\
\textbf{Postcondizione:}
il sistema ha modificato la camera secondo le richieste dell'utente. \\ \\
\textbf{Scenario Principale:}
\begin{itemize}
\item l'utente modifica la posizione della camera (UC1.2.6.1)
\item l'utente rimuove la camera (UC1.2.6.2)
\\ \\ \end{itemize}
\begin{figure}[h!]
\centering
\includegraphics[width=\textwidth]{{{UC1.2.6}}}
\caption{UC1.2.6 Modifica camera}
\label{UC1.2.6}
\end{figure}


\subsection{UC1.2.6.1 Modifica posizione camera}
\textbf{Diagramma associato:}
\ref{UC1.2.6.1} \\ \\
\textbf{Attori Coinvolti:}
Utente. \\ \\
\textbf{Scopo e Descrizione:}
un utente deve poter modificare la posizione della camera. \\ \\
\textbf{Precondizione:}
il sistema è in attesa che l'utente selezioni una funzionalità di modifica della camera. \\ \\
\textbf{Postcondizione:}
il sistema ha riposizionato la camera secondo le richieste utente. \\ \\
\textbf{Scenario Principale:}
\begin{itemize}
\item l'utente trasla la camera (UC1.2.6.1.1)
\item l'utente ruota la camera attorno alla scena 3D (UC1.2.6.1.2)
\\ \\ \end{itemize}
\begin{figure}[h!]
\centering
\includegraphics[width=\textwidth]{{{UC1.2.6.1}}}
\caption{UC1.2.6.1 Modifica posizione camera}
\label{UC1.2.6.1}
\end{figure}


\subsection{UC1.2.6.1.1 Trasla camera}
\textbf{Diagramma associato:}
\ref{UC1.2.6.1.1} \\ \\
\textbf{Attori Coinvolti:}
Utente. \\ \\
\textbf{Scopo e Descrizione:}
un utente deve poter traslare la camera. \\ \\
\textbf{Precondizione:}
il sistema è in attesa di input per quanto riguarda la modifica della posizione della camera. \\ \\
\textbf{Postcondizione:}
il sistema ha in memoria la scena 3D ed ha traslato la camera secondo l'input utente. \\ \\
\textbf{Scenario Principale:}
\begin{itemize}
\item l'utente sceglie di quanto avvicinare la camera verso il centro della scena 3D (UC1.2.6.1.1.1)
\item l'utente sceglie di quanto allontanare la camera dal centro della scena 3D (UC1.2.6.1.1.2)
\item l'utente sceglie i valori secondo i quali traslare la camera della scena 3D (UC1.2.6.1.1.3)
\\ \\ \end{itemize}
\begin{figure}[h!]
\centering
\includegraphics[width=\textwidth]{{{UC1.2.6.1.1}}}
\caption{UC1.2.6.1.1 Trasla camera}
\label{UC1.2.6.1.1}
\end{figure}


\subsection{UC1.2.6.1.1.1 Avvicina camera al centro della scena 3D}
\textbf{Diagramma associato:}
\ref{UC1.2.6.1.1} \\ \\
\textbf{Attori Coinvolti:}
Utente. \\ \\
\textbf{Scopo e Descrizione:}
un utente deve poter avvicinare la camera al centro della scena 3D. \\ \\
\textbf{Precondizione:}
il sistema è in attesa di input per quanto riguarda la traslazione della camera. \\ \\
\textbf{Postcondizione:}
il sistema ha avvicinato la camera verso il centro della scena 3D. \\ \\
\textbf{Scenario Principale:}
\begin{itemize}
\item l'utente sceglie di quanto avvicinare la camera verso il centro della scena 3D
\\ \\ \end{itemize}


\subsection{UC1.2.6.1.1.2 Allontana camera dal centro della scena 3D}
\textbf{Diagramma associato:}
\ref{UC1.2.6.1.1} \\ \\
\textbf{Attori Coinvolti:}
Utente. \\ \\
\textbf{Scopo e Descrizione:}
un utente deve poter allontanare la camera dal centro della scena 3D. \\ \\
\textbf{Precondizione:}
il sistema è in attesa di input per quanto riguarda la traslazione della camera. \\ \\
\textbf{Postcondizione:}
il sistema ha allontanato la camera dal centro della scena 3D. \\ \\
\textbf{Scenario Principale:}
\begin{itemize}
\item l'utente sceglie di quanto allontanare la camera dal centro della scena 3D
\\ \\ \end{itemize}


\subsection{UC1.2.6.1.1.3 Imposta traslazione della camera}
\textbf{Diagramma associato:}
\ref{UC1.2.6.1.1} \\ \\
\textbf{Attori Coinvolti:}
Utente. \\ \\
\textbf{Scopo e Descrizione:}
un utente deve poter impostare lo spostamento della camera. \\ \\
\textbf{Precondizione:}
il sistema è in attesa di input per quanto riguarda la traslazione della camera. \\ \\
\textbf{Postcondizione:}
il sistema ha spostato la camera secondo le specifiche richieste dell'utente. \\ \\
\textbf{Scenario Principale:}
\begin{itemize}
\item l'utente sceglie i valori secondo i quali traslare la camera della scena 3D
\\ \\ \end{itemize}


\subsection{UC1.2.6.1.2 Ruota camera attorno alla scena 3D}
\textbf{Diagramma associato:}
\ref{UC1.2.6.1} \\ \\
\textbf{Attori Coinvolti:}
Utente. \\ \\
\textbf{Scopo e Descrizione:}
un utente deve poter ruotare la camera attorno alla scena 3D. \\ \\
\textbf{Precondizione:}
il sistema è in attesa di input per quanto riguarda la modifica della posizione della camera. \\ \\
\textbf{Postcondizione:}
il sistema ha in memoria la scena 3D ed ha ruotato la camera 3D secondo l'input utente. \\ \\
\textbf{Scenario Principale:}
\begin{itemize}
\item l'utente ruota la camera attorno la scena
\\ \\ \end{itemize}


\subsection{UC1.2.6.2 Rimuovi camera}
\textbf{Diagramma associato:}
\ref{UC1.2.6} \\ \\
\textbf{Attori Coinvolti:}
Utente. \\ \\
\textbf{Scopo e Descrizione:}
un utente deve poter rimuovere la camera selezionata. \\ \\
\textbf{Precondizione:}
il sistema è in attesa che l'utente selezioni una funzionalità di modifica della camera. \\ \\
\textbf{Postcondizione:}
il sistema ha rimosso la camera selezionata in precedenza secondo le richieste utente. \\ \\
\textbf{Scenario Principale:}
\begin{itemize}
\item l'utente rimuove la camera
\\ \\ \end{itemize}


\subsection{UC1.2.7 Modifica fonte di luce}
\textbf{Diagramma associato:}
\ref{UC1.2.7} \\ \\
\textbf{Attori Coinvolti:}
Utente. \\ \\
\textbf{Scopo e Descrizione:}
un utente deve poter modificare le caratteristiche di una fonte di luce (quali posizione, intensità, colore e tipologia). \\ \\
\textbf{Precondizione:}
il sistema conosce la fonte di luce da modficare (UC1.2.1). \\ \\
\textbf{Postcondizione:}
il sistema ha modificato le caratteristiche di una fonte di luce secondo le richieste dell’utente. \\ \\
\textbf{Scenario Principale:}
\begin{itemize}
\item l'utente rimuove la fonte di luce (UC1.2.7.1)
\item l’utente modifica la posizione della fonte di luce (UC1.2.7.2)
\item l’utente modifica l’attenuazione della fonte di luce (UC1.2.7.3)
\item l’utente modifica il colore della fonte di luce (UC.1.2.7.4)
\item l’utente modifica la tipologia della fonte di luce (UC1.2.7.5)
\\ \\ \end{itemize}
\begin{figure}[h!]
\centering
\includegraphics[width=\textwidth]{{{UC1.2.7}}}
\caption{UC1.2.7 Modifica fonte di luce}
\label{UC1.2.7}
\end{figure}


\subsection{UC1.2.7.1 Rimuovi luce}
\textbf{Diagramma associato:}
\ref{UC1.2.7} \\ \\
\textbf{Attori Coinvolti:}
Utente. \\ \\
\textbf{Scopo e Descrizione:}
un utente deve poter rimuovere una fonte di luce. \\ \\
\textbf{Precondizione:}
il sistema è in attesa che l'utente selezioni una funzionalità di modifica delle caratteristiche della fonte di luce. \\ \\
\textbf{Postcondizione:}
il sistema ha rimosso la fonte di luce selezionata in precedenza secondo le richieste utente. \\ \\
\textbf{Scenario Principale:}
\begin{itemize}
\item l'utente sceglie di rimuovere la fonte di luce
\\ \\ \end{itemize}


\subsection{UC1.2.7.2 Modifica posizione luce}
\textbf{Diagramma associato:}
\ref{UC1.2.7} \\ \\
\textbf{Attori Coinvolti:}
Utente. \\ \\
\textbf{Scopo e Descrizione:}
un utente deve poter modificare la posizione di una fonte di luce. \\ \\
\textbf{Precondizione:}
il sistema è in attesa che l'utente selezioni una funzionalità di modifica delle caratteristiche della fonte di luce. \\ \\
\textbf{Postcondizione:}
il sistema ha riposizionato la fonte di luce secondo le impostazioni utente. \\ \\
\textbf{Scenario Principale:}
\begin{itemize}
\item l'utente riposiziona la fonte di luce
\\ \\ \end{itemize}


\subsection{UC1.2.7.3 Modifica attenuazione luce}
\textbf{Diagramma associato:}
\ref{UC1.2.7.3} \\ \\
\textbf{Attori Coinvolti:}
Utente. \\ \\
\textbf{Scopo e Descrizione:}
l’utente deve poter modificare i parametri d’attenuazione di una fonte di luce. \\ \\
\textbf{Precondizione:}
il sistema è in attesa che l'utente selezioni una funzionalità di modifica delle caratteristiche della fonte di luce. \\ \\
\textbf{Postcondizione:}
il sistema ha modificato i parametri di attenuazione secondo l'input utente. \\ \\
\textbf{Scenario Principale:}
\begin{itemize}
\item l’utente modifica l’attenuazione costante (UC1.2.7.3.1)
\item l’utente modifica l’attenuazione lineare (UC1.2.7.3.2)
\item  l’utente modifica l’attenuazione quadratica (UC1.2.7.3.3)
\\ \\ \end{itemize}
\begin{figure}[h!]
\centering
\includegraphics[width=\textwidth]{{{UC1.2.7.3}}}
\caption{UC1.2.7.3 Modifica attenuazione luce}
\label{UC1.2.7.3}
\end{figure}


\subsection{UC1.2.7.3.1 Modifica attenuazione costante}
\textbf{Diagramma associato:}
\ref{UC1.2.7.3} \\ \\
\textbf{Attori Coinvolti:}
Utente. \\ \\
\textbf{Scopo e Descrizione:}
l'utente deve poter modificare il parametro di attenuazione costante. \\ \\
\textbf{Precondizione:}
il sistema è in attesa di input per la modifica dei parametri di attenuazione della luce. \\ \\
\textbf{Postcondizione:}
il sistema conosce il valore del parametro di attenuazione costante scelto dall'utente. \\ \\
\textbf{Scenario Principale:}
\begin{itemize}
\item l'utente modifica il valore del parametro di attenuazione costante
\\ \\ \end{itemize}


\subsection{UC1.2.7.3.2 Modifica attenuazione lineare}
\textbf{Diagramma associato:}
\ref{UC1.2.7.3} \\ \\
\textbf{Attori Coinvolti:}
Utente. \\ \\
\textbf{Scopo e Descrizione:}
l'utente deve poter modificare il parametro di attenuazione lineare. \\ \\
\textbf{Precondizione:}
il sistema è in attesa di input per la modifica dei parametri di attenuazione della luce. \\ \\
\textbf{Postcondizione:}
il sistema conosce il valore del parametro di attenuazione lineare scelto dall'utente. \\ \\
\textbf{Scenario Principale:}
\begin{itemize}
\item l'utente modifica il valore del parametro di attenuazione lineare
\\ \\ \end{itemize}


\subsection{UC1.2.7.3.3 Modifica attenuazione quadratica}
\textbf{Diagramma associato:}
\ref{UC1.2.7.3} \\ \\
\textbf{Attori Coinvolti:}
Utente. \\ \\
\textbf{Scopo e Descrizione:}
l'utente deve poter modificare il paramentro di attenuazione quadratica. \\ \\
\textbf{Precondizione:}
il sistema è in attesa di input per la modifica dei parametri di attenuazione della luce. \\ \\
\textbf{Postcondizione:}
il sistema conosce il valore del parametro di attenuazione quadratica scelto dall'utente. \\ \\
\textbf{Scenario Principale:}
\begin{itemize}
\item l'utente modifica il valore del parametro di attenuazione quadratica
\\ \\ \end{itemize}


\subsection{UC1.2.7.4 Modifica colore componenti luce}
\textbf{Diagramma associato:}
\ref{UC1.2.7.4} \\ \\
\textbf{Attori Coinvolti:}
Utente. \\ \\
\textbf{Scopo e Descrizione:}
un utente deve poter modificare il colore di una fonte di luce, alterando i valori \underline{RGB} dei componenti che la compongono. \\ \\
\textbf{Precondizione:}
il sistema è in attesa che l'utente selezioni una funzionalità di modifica delle caratteristiche della fonte di luce. \\ \\
\textbf{Postcondizione:}
il sistema ha modificato il colore della fonte di luce secondo le richieste dell'utente. \\ \\
\textbf{Scenario Principale:}
\begin{itemize}
\item l’utente sceglie la componente di luce che vuole modificare (UC1.2.7.4.1)
\item l'utente imposta i valori RGB (UC1.2.7.4.2)
\\ \\ \end{itemize}
\begin{figure}[h!]
\centering
\includegraphics[width=\textwidth]{{{UC1.2.7.4}}}
\caption{UC1.2.7.4 Modifica colore componenti luce}
\label{UC1.2.7.4}
\end{figure}


\subsection{UC1.2.7.4.1 Seleziona componente luce}
\textbf{Diagramma associato:}
\ref{UC1.2.7.4.1} \\ \\
\textbf{Attori Coinvolti:}
Utente. \\ \\
\textbf{Scopo e Descrizione:}
un utente deve poter selezionare la componente di luce il cui colore vuole modificare. \\ \\
\textbf{Precondizione:}
il sistema è in attesa di input per la modifica del colore della componente della fonte di luce. \\ \\
\textbf{Postcondizione:}
il sistema ha in memoria la scena 3D e conosce la componente della fonte di luce selezionata dall’utente. \\ \\
\textbf{Scenario Principale:}
\begin{itemize}
\item l’utente seleziona la componente ambient (UC1.2.7.4.1.1)
\item l’utente seleziona la componente specular (UC1.2.7.4.1.2)
\item l’utente seleziona la componente diffuse (UC1.2.7.4.1.3)
\\ \\ \end{itemize}
\begin{figure}[h!]
\centering
\includegraphics[width=\textwidth]{{{UC1.2.7.4.1}}}
\caption{UC1.2.7.4.1 Seleziona componente luce}
\label{UC1.2.7.4.1}
\end{figure}


\subsection{UC1.2.7.4.1.1 Seleziona componente ambient}
\textbf{Diagramma associato:}
\ref{UC1.2.7.4.1} \\ \\
\textbf{Attori Coinvolti:}
Utente. \\ \\
\textbf{Scopo e Descrizione:}
un utente deve poter selezionare la componente ambient di una fonte di luce. \\ \\
\textbf{Precondizione:}
il sistema è in attesa che l'utente scelga la componente il cui colore vuole cambiare. \\ \\
\textbf{Postcondizione:}
il sistema sa che l'utente ha selezionato la componente ambient. \\ \\
\textbf{Scenario Principale:}
\begin{itemize}
\item l'utente seleziona la componente ambient
\\ \\ \end{itemize}


\subsection{UC1.2.7.4.1.2 Seleziona componente specular}
\textbf{Diagramma associato:}
\ref{UC1.2.7.4.1} \\ \\
\textbf{Attori Coinvolti:}
Utente. \\ \\
\textbf{Scopo e Descrizione:}
un utente deve poter selezionare la componente specular di una fonte di luce. \\ \\
\textbf{Precondizione:}
il sistema è in attesa che l'utente scelga la componente il cui colore vuole cambiare. \\ \\
\textbf{Postcondizione:}
il sistema sa che l'utente ha selezionato la componente specular. \\ \\
\textbf{Scenario Principale:}
\begin{itemize}
\item l'utente seleziona la componente specular
\\ \\ \end{itemize}


\subsection{UC1.2.7.4.1.3 Seleziona componente diffuse}
\textbf{Diagramma associato:}
\ref{UC1.2.7.4.1} \\ \\
\textbf{Attori Coinvolti:}
Utente. \\ \\
\textbf{Scopo e Descrizione:}
un utente deve poter selezionare la componente diffuse di una fonte di luce. \\ \\
\textbf{Precondizione:}
il sistema è in attesa che l'utente scelga la componente il cui colore vuole cambiare. \\ \\
\textbf{Postcondizione:}
il sistema sa che l'utente ha selezionato la componente diffuse. \\ \\
\textbf{Scenario Principale:}
\begin{itemize}
\item l'utente seleziona la componente diffuse
\\ \\ \end{itemize}


\subsection{UC1.2.7.4.2 Imposta valori RGB}
\textbf{Diagramma associato:}
\ref{UC1.2.7.4.2} \\ \\
\textbf{Attori Coinvolti:}
Utente. \\ \\
\textbf{Scopo e Descrizione:}
un utente deve poter impostare i valori RGB di una componente di luce. \\ \\
\textbf{Precondizione:}
il sistema conosce la componente selezionata (UC1.2.7.4.1), ed è in attesa di input per la modifica del colore. \\ \\
\textbf{Postcondizione:}
il sistema conosce i valori RGB impostati dall'utente. \\ \\
\textbf{Scenario Principale:}
\begin{itemize}
\item l'utente imposta il valore red (UC1.2.7.4.2.1)
\item l'utente imposta il valore green (UC1.2.7.4.2.2)
\item l'utente imposta il valore blue (UC1.2.7.4.2.3)
\\ \\ \end{itemize}
\begin{figure}[h!]
\centering
\includegraphics[width=\textwidth]{{{UC1.2.7.4.2}}}
\caption{UC1.2.7.4.2 Imposta valori RGB}
\label{UC1.2.7.4.2}
\end{figure}


\subsection{UC1.2.7.4.2.1 Imposta valore red}
\textbf{Diagramma associato:}
\ref{UC1.2.7.4.2} \\ \\
\textbf{Attori Coinvolti:}
Utente. \\ \\
\textbf{Scopo e Descrizione:}
un utente deve poter impostare il valore red di una componente di una fonte di luce. \\ \\
\textbf{Precondizione:}
il sistema è in attesa di input per quanto riguarda i valori RGB. \\ \\
\textbf{Postcondizione:}
il sistema conosce il valore red impostato dall'utente. \\ \\
\textbf{Scenario Principale:}
\begin{itemize}
\item l'utente imposta il valore red
\\ \\ \end{itemize}


\subsection{UC1.2.7.4.2.2 Imposta valore green}
\textbf{Diagramma associato:}
\ref{UC1.2.7.4.2} \\ \\
\textbf{Attori Coinvolti:}
Utente. \\ \\
\textbf{Scopo e Descrizione:}
un utente deve poter impostare il valore green di una componente di una fonte di luce. \\ \\
\textbf{Precondizione:}
il sistema è in attesa di input per quanto riguarda i valori RGB. \\ \\
\textbf{Postcondizione:}
il sistema conosce il valore green impostato dall'utente. \\ \\
\textbf{Scenario Principale:}
\begin{itemize}
\item l'utente imposta il valore green
\\ \\ \end{itemize}


\subsection{UC1.2.7.4.2.3 Imposta valore blue}
\textbf{Diagramma associato:}
\ref{UC1.2.7.4.2} \\ \\
\textbf{Attori Coinvolti:}
Utente. \\ \\
\textbf{Scopo e Descrizione:}
un utente deve poter impostare il valore blue di una componente di una fonte di luce. \\ \\
\textbf{Precondizione:}
il sistema è in attesa di input per quanto riguarda i valori RGB. \\ \\
\textbf{Postcondizione:}
il sistema conosce il valore blue impostato dall'utente. \\ \\
\textbf{Scenario Principale:}
\begin{itemize}
\item l'utente imposta il valore blue
\\ \\ \end{itemize}


\subsection{UC1.2.7.5 Modifica tipologia luce}
\textbf{Diagramma associato:}
\ref{UC1.2.7.5} \\ \\
\textbf{Attori Coinvolti:}
Utente. \\ \\
\textbf{Scopo e Descrizione:}
un utente deve poter modificare la tipologia corrente di una fonte di luce. \\ \\
\textbf{Precondizione:}
il sistema è in attesa che l'utente selezioni una funzionalità di modifica delle caratteristiche della fonte di luce. \\ \\
\textbf{Postcondizione:}
il sistema ha modificato la tipologia della fonte di luce corrente secondo le richieste utente. \\ \\
\textbf{Scenario Principale:}
\begin{itemize}
\item l'utente modifica la tipologia di luce corrente in \underline{spotlight} (UC1.2.7.5.1)
\item l'utente modifica la tipologia di luce corrente in \underline{omni light} (UC1.2.7.5.2)
\item l'utente modifica la tipologia di luce corrente in \underline{directional light} (UC1.2.7.5.3)
\\ \\ \end{itemize}
\begin{figure}[h!]
\centering
\includegraphics[width=\textwidth]{{{UC1.2.7.5}}}
\caption{UC1.2.7.5 Modifica tipologia luce}
\label{UC1.2.7.5}
\end{figure}


\subsection{UC1.2.7.5.1 Modifica la tipologia di luce corrente in spotlight}
\textbf{Diagramma associato:}
\ref{UC1.2.7.5} \\ \\
\textbf{Attori Coinvolti:}
Utente. \\ \\
\textbf{Scopo e Descrizione:}
un utente deve poter modificare la tipologia corrente di una fonte di luce in spotlight. \\ \\
\textbf{Precondizione:}
il sistema è in attesa di input per quanto riguarda la modifica della tipologia di luce corrente. \\ \\
\textbf{Postcondizione:}
il sistema conosce la richiesta di modifica tipologia corrente di luce in spotlight. \\ \\
\textbf{Scenario Principale:}
\begin{itemize}
\item l'utente imposta la tipologia di luce corrente in spotlight
\\ \\ \end{itemize}


\subsection{UC1.2.7.5.2 Modifica la tipologia di luce corrente in omni light}
\textbf{Diagramma associato:}
\ref{UC1.2.7.5} \\ \\
\textbf{Attori Coinvolti:}
Utente. \\ \\
\textbf{Scopo e Descrizione:}
un utente deve poter modificare la tipologia corrente di una fonte di luce in omni light. \\ \\
\textbf{Precondizione:}
il sistema è in attesa di input per quanto riguarda la modifica della tipologia di luce corrente. \\ \\
\textbf{Postcondizione:}
il sistema conosce la richiesta di modifica tipologia corrente di luce in omni light. \\ \\
\textbf{Scenario Principale:}
\begin{itemize}
\item l'utente imposta la tipologia di luce corrente in omni light
\\ \\ \end{itemize}


\subsection{UC1.2.7.5.3 Modifica la tipologia di luce corrente in directional light}
\textbf{Diagramma associato:}
\ref{UC1.2.7.5} \\ \\
\textbf{Attori Coinvolti:}
Utente. \\ \\
\textbf{Scopo e Descrizione:}
un utente deve poter modificare la tipologia corrente di una fonte di luce in directional light. \\ \\
\textbf{Precondizione:}
il sistema è in attesa di input per quanto riguarda la modifica della tipologia di luce corrente. \\ \\
\textbf{Postcondizione:}
il sistema conosce la richiesta di modifica tipologia corrente di luce in directional light. \\ \\
\textbf{Scenario Principale:}
\begin{itemize}
\item l'utente imposta la tipologia di luce corrente in directional light
\\ \\ \end{itemize}


\subsection{UC1.3 Esporta modello 3D}
\textbf{Diagramma associato:}
\ref{UC1.3} \\ \\
\textbf{Attori Coinvolti:}
Utente. \\ \\
\textbf{Scopo e Descrizione:}
un utente deve poter scegliere in che formato esportare la scena 3D, scegliere la \underline{directory} di destinazione e il nome del file. La scelta del formato comporta la scelta della precisione. \\ \\
\textbf{Precondizione:}
il sistema ha in memoria la scena 3D caricata in precedenza (UC1.1), il sistema è in attesa che l'utente selezioni una funzionalità. \\ \\
\textbf{Postcondizione:}
il sistema ha esportato la scena 3D secondo le richieste dell’utente. \\ \\
\textbf{Scenario Principale:}
\begin{itemize}
\item l’utente sceglie il formato verso il quale esportare (UC1.3.1)
\item l’utente sceglie la precisione (UC1.3.2)
\item l’utente sceglie la directory di destinazione (UC1.3.3)
\item l’utente sceglie il nome del file (UC1.3.4)
\\ \\ \end{itemize}
\begin{figure}[h!]
\centering
\includegraphics[width=\textwidth]{{{UC1.3}}}
\caption{UC1.3 Esporta modello 3D}
\label{UC1.3}
\end{figure}


\subsection{UC1.3.1 Seleziona formato}
\textbf{Diagramma associato:}
\ref{UC1.3.1} \\ \\
\textbf{Attori Coinvolti:}
Utente. \\ \\
\textbf{Scopo e Descrizione:}
un utente deve poter scegliere il formato verso il quale esportare. \\ \\
\textbf{Precondizione:}
il sistema è in attesa di input per quanto riguarda l'esportazione. \\ \\
\textbf{Postcondizione:}
il sistema conosce il formato scelto dall’utente. \\ \\
\textbf{Scenario Principale:}
\begin{itemize}
\item l’utente sceglie il formato JSON (UC1.3.1.1)
\item l’utente sceglie il formato XML
\\ \\ \end{itemize}
\begin{figure}[h!]
\centering
\includegraphics[width=\textwidth]{{{UC1.3.1}}}
\caption{UC1.3.1 Seleziona formato}
\label{UC1.3.1}
\end{figure}


\subsection{UC1.3.1.1 Seleziona formato JSON}
\textbf{Diagramma associato:}
\ref{UC1.3.1.1} \\ \\
\textbf{Attori Coinvolti:}
Utente. \\ \\
\textbf{Scopo e Descrizione:}
un utente deve poter scegliere se esportare la scena 3D in JSON “compatto” o “leggibile”. \\ \\
\textbf{Precondizione:}
il sistema è in attesa di input per quanto riguarda il formato di esportazione. \\ \\
\textbf{Postcondizione:}
il sistema conosce il tipo di formato JSON scelto dall’utente. \\ \\
\textbf{Scenario Principale:}
\begin{itemize}
\item  l’utente sceglie il formato JSON “compatto”
\item l’utente sceglie il formato JSON ”leggibile”
\\ \\ \end{itemize}
\begin{figure}[h!]
\centering
\includegraphics[width=\textwidth]{{{UC1.3.1.1}}}
\caption{UC1.3.1.1 Seleziona formato JSON}
\label{UC1.3.1.1}
\end{figure}


\subsection{UC1.3.1.1.1 Seleziona formato JSON compatto}
\textbf{Diagramma associato:}
\ref{UC1.3.1.1} \\ \\
\textbf{Attori Coinvolti:}
Utente. \\ \\
\textbf{Scopo e Descrizione:}
un utente deve poter esportare la scena 3D nel formato JSON "compatto". \\ \\
\textbf{Precondizione:}
il sistema è in attesa di input per quanto riguarda il tipo JSON di esportazione. \\ \\
\textbf{Postcondizione:}
il sistema conosce il formato del file di esportazione scelto dall'utente. \\ \\
\textbf{Scenario Principale:}
\begin{itemize}
\item l'utente seleziona il formato JSON "compatto"
\\ \\ \end{itemize}


\subsection{UC1.3.1.1.2 Seleziona formato JSON leggibile}
\textbf{Diagramma associato:}
\ref{UC1.3.1.1} \\ \\
\textbf{Attori Coinvolti:}
Utente. \\ \\
\textbf{Scopo e Descrizione:}
un utente deve poter esportare la scena 3D nel formato JSON "leggibile". \\ \\
\textbf{Precondizione:}
il sistema è in attesa di input per quanto riguarda il tipo JSON di esportazione. \\ \\
\textbf{Postcondizione:}
il sistema conosce il formato del file di esportazione scelto dall'utente. \\ \\
\textbf{Scenario Principale:}
\begin{itemize}
\item l'utente seleziona il formato JSON "leggibile"
\\ \\ \end{itemize}


\subsection{UC1.3.1.2 Seleziona formato XML}
\textbf{Diagramma associato:}
\ref{UC1.3.1} \\ \\
\textbf{Attori Coinvolti:}
Utente. \\ \\
\textbf{Scopo e Descrizione:}
un utente deve poter esportare la scena 3D nel formato XML. \\ \\
\textbf{Precondizione:}
il sistema è in attesa di input per quanto riguarda il formato di esportazione. \\ \\
\textbf{Postcondizione:}
il sistema conosce il formato del file di esportazione scelto dall'utente. \\ \\
\textbf{Scenario Principale:}
\begin{itemize}
\item l'utente seleziona il formato XML
\\ \\ \end{itemize}


\subsection{UC1.3.2 Sceglie precisione}
\textbf{Diagramma associato:}
\ref{UC1.3.2} \\ \\
\textbf{Attori Coinvolti:}
Utente. \\ \\
\textbf{Scopo e Descrizione:}
un utente deve poter scegliere se esportare la scena 3D in singola o doppia precisione. \\ \\
\textbf{Precondizione:}
il sistema conosce il formato di esportazione, ed è in attesa di input per quanto riguarda l'esportazione. \\ \\
\textbf{Postcondizione:}
il sistema conosce la precisione scelta dall’utente. \\ \\
\textbf{Scenario Principale:}
\begin{itemize}
\item l’utente sceglie la precisione float
\item l’utente sceglie la precisione double
\\ \\ \end{itemize}
\begin{figure}[h!]
\centering
\includegraphics[width=\textwidth]{{{UC1.3.2}}}
\caption{UC1.3.2 Sceglie precisione}
\label{UC1.3.2}
\end{figure}


\subsection{UC1.3.2.1 Seleziona precisione float}
\textbf{Diagramma associato:}
\ref{UC1.3.2} \\ \\
\textbf{Attori Coinvolti:}
Utente. \\ \\
\textbf{Scopo e Descrizione:}
un utente deve poter esportare la scena 3D in singola precisione. \\ \\
\textbf{Precondizione:}
il sistema è in attesa di input per quanto riguarda la precisione dell'esportazione. \\ \\
\textbf{Postcondizione:}
il sistema conosce la precisione scelta dall’utente. \\ \\
\textbf{Scenario Principale:}
\begin{itemize}
\item l'utente sceglie la precisione float
\\ \\ \end{itemize}


\subsection{UC1.3.2.2 Seleziona precisione double}
\textbf{Diagramma associato:}
\ref{UC1.3.2} \\ \\
\textbf{Attori Coinvolti:}
Utente. \\ \\
\textbf{Scopo e Descrizione:}
un utente deve poter esportare la scena 3D in doppia precisione. \\ \\
\textbf{Precondizione:}
il sistema ha in memoria la scena 3D caricata in precedenza e conosce il formato verso il quale l’utente vuole esportare (UC1.3.1). \\ \\
\textbf{Postcondizione:}
il sistema conosce la precisione scelta dall’utente. \\ \\
\textbf{Scenario Principale:}
\begin{itemize}
\item l'utente sceglie la precisione double
\\ \\ \end{itemize}


\subsection{UC1.3.3 Sceglie directory destinazione}
\textbf{Diagramma associato:}
\ref{UC1.3} \\ \\
\textbf{Attori Coinvolti:}
Utente. \\ \\
\textbf{Scopo e Descrizione:}
un utente deve poter scegliere la directory di destinazione del file di esportazione. \\ \\
\textbf{Precondizione:}
il sistema conosce il formato di esportazione, ed è in attesa di input per quanto riguarda l'esportazione. \\ \\
\textbf{Postcondizione:}
il sistema conosce la directory di destinazione selezionata dall'utente. \\ \\
\textbf{Scenario Principale:}
\begin{itemize}
\item l'utente seleziona la directory di destinazione
\\ \\ \end{itemize}


\subsection{UC1.3.4 Sceglie nome file}
\textbf{Diagramma associato:}
\ref{UC1.3} \\ \\
\textbf{Attori Coinvolti:}
Utente. \\ \\
\textbf{Scopo e Descrizione:}
un utente deve poter scegliere il nome del file di esportazione. \\ \\
\textbf{Precondizione:}
il sistema conosce il formato di esportazione, è in attesa di input per quanto riguarda l'esportazione, e conosce la directory di destinazione. \\ \\
\textbf{Postcondizione:}
il sistema conosce il nome del file scelto dall'utente. \\ \\
\textbf{Scenario Principale:}
\begin{itemize}
\item l'utente sceglie il nome del file
\\ \\ \end{itemize}


\subsection{UC1.4 Visualizza anteprima scena 3D}
\textbf{Diagramma associato:}
\ref{UC1.4} \\ \\
\textbf{Attori Coinvolti:}
Utente. \\ \\
\textbf{Scopo e Descrizione:}
un utente deve poter visualizzare l’anteprima della scena 3D. \\ \\
\textbf{Precondizione:}
il sistema ha in memoria la scena 3D caricata in precedenza (UC1.1), il sistema è in attesa che l'utente selezioni una funzionalità. \\ \\
\textbf{Postcondizione:}
l sistema sta visualizzando la scena 3D secondo le richieste dell’utente. \\ \\
\textbf{Scenario Principale:}
\begin{itemize}
\item l’utente visualizza l’anteprima semplice (UC1.4.1)
\item  l’utente esegue il rendering
\\ \\ \end{itemize}
\begin{figure}[h!]
\centering
\includegraphics[width=\textwidth]{{{UC1.4}}}
\caption{UC1.4 Visualizza anteprima scena 3D}
\label{UC1.4}
\end{figure}


\subsection{UC1.4.1 Visualizza anteprima semplice}
\textbf{Diagramma associato:}
\ref{UC1.4.1} \\ \\
\textbf{Attori Coinvolti:}
Utente. \\ \\
\textbf{Scopo e Descrizione:}
un utente deve avere la possibilità di spostare il punto d’osservazione della scena 3D. \\ \\
\textbf{Precondizione:}
il sistema è in attesa che l'utente selezioni una funzionalità per l'anteprima 3D. \\ \\
\textbf{Postcondizione:}
il sistema ha in memoria la scena 3D e ha spostato il punto d’osservazione secondo le richieste dell’utente. \\ \\
\textbf{Scenario Principale:}
\begin{itemize}
\item l’utente trasla  il punto d’osservazione
\item l’utente ruota il punto d’osservazione attorno alla scena
\\ \\ \end{itemize}
\begin{figure}[h!]
\centering
\includegraphics[width=\textwidth]{{{UC1.4.1}}}
\caption{UC1.4.1 Visualizza anteprima semplice}
\label{UC1.4.1}
\end{figure}


\subsection{UC1.4.1.1 Trasla punto d'osservazione}
\textbf{Diagramma associato:}
\ref{UC1.4.1.1} \\ \\
\textbf{Attori Coinvolti:}
Utente. \\ \\
\textbf{Scopo e Descrizione:}
un utente deve poter traslare il \underline{punto d'osservazione} della scena 3D. \\ \\
\textbf{Precondizione:}
il sistema sta visualizzando un'anteprima semplice ed è in attesa che l'utente selezioni una funzionalità. \\ \\
\textbf{Postcondizione:}
il sistema ha in memoria la scena 3D ed ha traslato il punto d'osservazione della scena secondo l'input utente. \\ \\
\textbf{Scenario Principale:}
\begin{itemize}
\item l'utente trasla il punto d'osservazione
\\ \\ \end{itemize}
\begin{figure}[h!]
\centering
\includegraphics[width=\textwidth]{{{UC1.4.1.1}}}
\caption{UC1.4.1.1 Trasla punto d'osservazione}
\label{UC1.4.1.1}
\end{figure}


\subsection{UC1.4.1.1.1 Avvicina punto di osservazione al centro della scena 3D}
\textbf{Diagramma associato:}
\ref{UC1.4.1.1} \\ \\
\textbf{Attori Coinvolti:}
Utente. \\ \\
\textbf{Scopo e Descrizione:}
un utente deve poter avvicinare il punto d'osservazione al centro della scena 3D. \\ \\
\textbf{Precondizione:}
il sistema è in attesa che l'utente selezioni una funzionalità per quanto riguarda la traslazione del punto di osservazione. \\ \\
\textbf{Postcondizione:}
il sistema ha avvicinato il punto di osservazione verso il centro della scena 3D. \\ \\
\textbf{Scenario Principale:}
\begin{itemize}
\item l'utente sceglie di quanto avvicinare il punto di osservazione verso il centro della scena 3D
\\ \\ \end{itemize}


\subsection{UC1.4.1.1.2 Allontana punto di osservazione dal centro della scena 3D}
\textbf{Diagramma associato:}
\ref{UC1.4.1.1} \\ \\
\textbf{Attori Coinvolti:}
Utente. \\ \\
\textbf{Scopo e Descrizione:}
un utente deve poter allontanare il punto d'osservazione dal centro della scena 3D. \\ \\
\textbf{Precondizione:}
il sistema è in attesa che l'utente selezioni una funzionalità per quanto riguarda la traslazione del punto di osservazione. \\ \\
\textbf{Postcondizione:}
il sistema ha allontanato il punto di osservazione dal centro della scena 3D. \\ \\
\textbf{Scenario Principale:}
\begin{itemize}
\item l'utente sceglie di quanto allontanare il punto di osservazione dal centro della scena 3D
\\ \\ \end{itemize}


\subsection{UC1.4.1.1.3 Imposta traslazione del punto di osservazione}
\textbf{Diagramma associato:}
\ref{UC1.4.1.1} \\ \\
\textbf{Attori Coinvolti:}
Utente. \\ \\
\textbf{Scopo e Descrizione:}
un utente deve poter impostare lo spostamento del punto di osservazione. \\ \\
\textbf{Precondizione:}
il sistema è in attesa che l'utente selezioni una funzionalità per quanto riguarda la traslazione del punto di osservazione. \\ \\
\textbf{Postcondizione:}
il sistema ha spostato il punto di osservazione secondo le specifiche richieste dell'utente. \\ \\
\textbf{Scenario Principale:}
\begin{itemize}
\item l'utente sceglie i valori secondo i quali traslare il punto di osservazione della scena 3D
\\ \\ \end{itemize}


\subsection{UC1.4.1.2 Ruota il punto d'osservazione attorno alla scena 3D}
\textbf{Diagramma associato:}
\ref{UC1.4.1} \\ \\
\textbf{Attori Coinvolti:}
Utente. \\ \\
\textbf{Scopo e Descrizione:}
un utente deve poter ruotare il punto d'osservazione attorno alla scena 3D. \\ \\
\textbf{Precondizione:}
il sistema sta visualizzando un'anteprima semplice ed è in attesa che l'utente selezioni una funzionalità. \\ \\
\textbf{Postcondizione:}
il sistema ha in memoria la scena 3D ed ha ruotato il punto d'osservazione della scena 3D secondo l'input utente. \\ \\
\textbf{Scenario Principale:}
\begin{itemize}
\item l'utente ruota il punto d'osservazione attorno alla scena
\\ \\ \end{itemize}


\subsection{UC1.4.2 Esegue rendering}
\textbf{Diagramma associato:}
\ref{UC1.4} \\ \\
\textbf{Attori Coinvolti:}
Utente. \\ \\
\textbf{Scopo e Descrizione:}
un utente deve poter eseguire il rendering della scena 3D. \\ \\
\textbf{Precondizione:}
il sistema è in attesa che l'utente selezioni una funzionalità per l'anteprima 3D. \\ \\
\textbf{Postcondizione:}
il sistema ha in memoria la scena 3D e propone all'utente l'immagine risultante dal rendering. \\ \\
\textbf{Scenario Principale:}
\begin{itemize}
\item l'utente esegue il rendering
\\ \\ \end{itemize}


\subsection{UC1.5 Agisce sulla cronologia delle modifiche}
\textbf{Diagramma associato:}
\ref{UC1.5} \\ \\
\textbf{Attori Coinvolti:}
Utente. \\ \\
\textbf{Scopo e Descrizione:}
un utente deve poter agire sulle modifiche effettuate. \\ \\
\textbf{Precondizione:}
il sistema ha in memoria una scena 3D caricata in precedenza e la cronologia delle modifiche utente non è vuota (UC1.2). \\ \\
\textbf{Postcondizione:}
il sistema ha in memoria la scena 3D ed ha agito sulla cronologia delle modifiche secondo le richieste utente. \\ \\
\textbf{Scenario Principale:}
\begin{itemize}
\item l'utente annulla l'ultima modifica effettuata
\item l'utente ripete l'ultima modifica effettuata
\\ \\ \end{itemize}
\begin{figure}[h!]
\centering
\includegraphics[width=\textwidth]{{{UC1.5}}}
\caption{UC1.5 Agisce sulla cronologia delle modifiche}
\label{UC1.5}
\end{figure}


\subsection{UC1.5.1 Annulla ultima modifica}
\textbf{Diagramma associato:}
\ref{UC1.5} \\ \\
\textbf{Attori Coinvolti:}
Utente. \\ \\
\textbf{Scopo e Descrizione:}
un utente deve poter annullare l'ultima modifica effettuata. \\ \\
\textbf{Precondizione:}
il sistema è in attesa di input, lo stato corrente della scena 3D è diverso da quello iniziale (UC1.1). \\ \\
\textbf{Postcondizione:}
il sistema ha annullato l'ultima modifica effettuata, nella cronologia delle modifiche c'è almeno una modifica da ripetere. \\ \\
\textbf{Scenario Principale:}
\begin{itemize}
\item l'utente annulla l'ultima modifica effettuata
\\ \\ \end{itemize}


\subsection{UC1.5.2 Ripeti ultima modifica}
\textbf{Diagramma associato:}
\ref{UC1.5} \\ \\
\textbf{Attori Coinvolti:}
Utente. \\ \\
\textbf{Scopo e Descrizione:}
un utente deve poter ripetere l'ultima modifica annullata. \\ \\
\textbf{Precondizione:}
il sistema è in attesa di input, nella cronologia delle modifiche ci sono delle modifiche da ripetere (UC1.5.1). \\ \\
\textbf{Postcondizione:}
il sistema ha ripetuto l'ultima modifica annullata. \\ \\
\textbf{Scenario Principale:}
\begin{itemize}
\item l'utente ripete l'ultima modifica annullata
\\ \\ \end{itemize}








\newpage
\section{Requisiti}
\label{4.0}
Di seguito viene presentata la tabella del tracciamento dei requisiti. Le regole di composizione degli identificativi dei requisiti sono riportate nel documento di Norme di Progetto, precisamente alla sezione sette.
%%DENTRO QUESTO FILE C'è ANCHE LA intestazione TABELLA
%tabella requisiti codice requisito, descrizione, fonte, UC di riferimento

\subsection{Requisiti funzionali MaaP}
\begin{longtable}{|c|p{6cm}|c|c|}
\caption{Requisiti funzionali MaaP}
\label{tab:Requisiti MaaP} \\
\toprule
\multicolumn{1}{|c}{\textbf{Requisito}} & \multicolumn{1}{|p{6cm}}{\textbf{Descrizione}}   & \multicolumn{1}{|c}{\textbf{Fonte}} & \multicolumn{1}{|c|}{\textbf{Caso d'uso}}\\
\midrule
\endfirsthead
\multicolumn{2}{l}{\footnotesize\itshape\tablename~\thetable: continua dalla pagina precedente} \\
\toprule
\multicolumn{1}{|c}{\textbf{Requisito}} & \multicolumn{1}{|p{6cm}}{\textbf{Descrizione}}   & \multicolumn{1}{|c}{\textbf{Fonte}} & \multicolumn{1}{|c|}{\textbf{Caso d'uso}}\\
\midrule
\endhead
\midrule
\multicolumn{2}{r}{\footnotesize\itshape\tablename~\thetable: continua nella prossima pagina} \\
\endfoot
\bottomrule
\multicolumn{2}{r}{\footnotesize\itshape\tablename~\thetable: si conclude dalla pagina precedente} \\
\endlastfoot

% Requisiti utente sviluppatore


\midrule
ROF1
& Il sistema MaaP deve essere in grado di generare lo scheletro del progetto
& Capitolato
& UC1\\
& & & UC1.1
\\

\midrule
ROF1.1
& Il sistema MaaP deve installare le librerie necessarie al funzionamento del progetto
& Capitolato
&
\\
\midrule
ROF1.2
& Il sistema MaaP deve generare i file di configurazione necessari al funzionamento del progetto
& Capitolato
&
\\
\midrule
ROF1.3
& Il sistema MaaP deve generare le directory necessarie al funzionamento del progetto
& Capitolato
&
\\
\midrule
ROF1.4
& Il sistema MaaP deve generare il sistema di autenticazione per le pagine web
& Capitolato
&
\\
\midrule
ROF1.4.1
& Il sistema di autenticazione per le pagine web deve essere generato insieme ad un profilo amministratore di \gloss{default}
& Verbale\_2013\_12\_05
&
\\

\midrule
ROF1.5
& Il sistema MaaP deve essere in grado eliminare un progetto esistente
& Interna
& UC1\\
& & & UC1.6
\\

\midrule
ROF1.6
& Il sistema MaaP deve essere in grado di clonare un progetto esistente
& Interna
& UC1\\
& & & UC1.7
\\

\midrule
RFF2
& Il sistema MaaP deve permettere all'utente sviluppatore di utilizzare un editor interno specializzato per la scrittura/modifica dei file di descrizione
& Capitolato
& UC1\\
& & & UC1.4
\\

\midrule
RFF2.1
& Il sistema MaaP deve permettere all'utente sviluppatore di utilizzare un editor interno specializzato per la scrittura di un nuovo file di descrizione
& Interna
& UC1.4 \\
& & & UC1.4.1
\\

\midrule
RFF2.1.1
& Il sistema MaaP deve permettere all'utente sviluppatore di utilizzare un editor interno specializzato per scrivere il codice del file di descrizione che intende creare
& Interna
& UC1.4.1 \\
& & & UC1.4.1.1
\\

\midrule
RFF2.1.2
& Il sistema MaaP deve permettere all'utente sviluppatore di utilizzare un editor interno specializzato per salvare il codice scritto in modo permanente
& Interna
& UC1.4.1 \\
& & & UC1.4.1.2
\\

\midrule
RFF2.2
& Il sistema MaaP deve permettere all'utente sviluppatore di utilizzare un editor interno specializzato per la modifica di un file di descrizione esistente
& Interna
& UC1.4 \\
& & & UC1.4.2
\\

\midrule
RFF2.2.1
& Il sistema MaaP deve permettere all'utente sviluppatore di utilizzare un editor interno specializzato per modificare il codice di un file di descrizione esistente
& Interna
& UC1.4.2 \\
& & & UC1.4.2.1
\\

\midrule
RFF2.2.2
& Il sistema MaaP deve permettere all'utente sviluppatore di utilizzare un editor interno specializzato per salvare il codice modificato in modo permanente
& Interna
& UC1.4.2 \\
& & & UC1.4.2.2
\\

\midrule
RFF2.2.3
& Il sistema MaaP deve permettere all'utente sviluppatore di utilizzare un editor interno specializzato per annullare le modifiche al codice del file di descrizione modificato
& Interna
& UC1.4.2 \\
& & & UC1.4.2.3
\\

\midrule
ROF3
& Il sistema MaaP deve permette all'utente sviluppatore di inserire un file di descrizione
& Interna
& UC1\\
& & & UC1.5\\

\midrule
ROF4
& Il sistema deve permette all'utente sviluppatore di utilizzare un file di descrizione
& Capitolato
& UC1\\
& & & UC1.2
\\
\midrule
ROF4.1
& Il sistema MaaP deve permettere all'utente sviluppatore di creare la visualizzazione della Collection
& Capitolato
& UC1.2 \\
& & & UC1.2.1
\\
\midrule
ROF4.1.1
& Il sistema MaaP deve permettere all'utente sviluppatore di creare la visualizzazione del menù per le Collection
& Capitolato
& UC1.2.1\\
& & & UC1.2.1.1
\\
\midrule
ROF4.1.1.1
& Il sistema MaaP deve permettere all'utente sviluppatore di definire il nome della voce relativa alla Collection
& Capitolato
& UC1.2.1.1\\
& & & UC1.2.1.1.1
\\
\midrule
ROF4.1.1.2
& Il sistema MaaP deve permettere all'utente sviluppatore di definire la posizione di una voce all'interno del menù
& Capitolato
& UC1.2.1.1\\
& & & 1.2.1.1.2
\\
\midrule
ROF4.1.2
& Il sistema MaaP deve permettere all'utente sviluppatore di creare la visualizzazione della pagina Collection-Index
& Capitolato
& UC1.2.1\\
& & & UC1.2.1.2
\\
\midrule
ROF4.1.2.1
& Il sistema MaaP deve permettere all'utente sviluppatore di aggiungere delle chiavi da visualizzare nella pagina Collection-Index
& Capitolato
& UC1.2.1.2\\
& & & UC1.2.1.2.1
\\
\midrule
ROF4.1.2.1.1
& Il sistema MaaP deve permettere all'utente sviluppatore di aggiungere definire un'etichetta per la chiave da visualizzare
& Capitolato
& UC1.2.1.2.1\\
& & & UC1.2.1.2.1.1
\\
\midrule
ROF4.1.2.1.2
& Il sistema MaaP deve permettere all'utente sviluppatore di definire un campo associato alla chiave da visualizzare
& Capitolato
& UC1.2.1.2.1\\
& & & UC1.2.1.2.1.2
\\
\midrule
ROF4.1.2.1.3
& Il sistema MaaP deve permettere all'utente sviluppatore di definire un campo associato alla chiave da visualizzare, proveniente da un documento esterno
& Capitolato
& UC1.2.1.2.1\\
& & & UC1.2.1.2.1.3
\\
\midrule
ROF4.1.2.1.4
& Il sistema MaaP deve permettere all'utente sviluppatore di definire un campo associato alla chiave da visualizzare, proveniente dal risultato di una query
& Capitolato
& UC1.2.1.2.1\\
& & & UC1.2.1.2.1.4
\\
\midrule
ROF4.1.2.1.5
& Il sistema MaaP deve permettere all'utente sviluppatore di definire un campo associato alla chiave da visualizzare come trasformazione
& Capitolato
& UC1.2.1.2.1\\
& & & UC1.2.1.2.1.5
\\

\midrule
ROF4.1.2.2
& Il sistema MaaP deve permettere all'utente sviluppatore di definire un ordinamento rispetto a una chiave
& Capitolato
& UC1.2.1.2\\
& & & UC1.2.1.2.2
\\
\midrule
ROF4.1.2.3
& Il sistema deve permettere all'utente sviluppatore di  definire un numero massimo di Document da visualizzare per la pagina Collection-Index
& Capitolato
& UC1.2.1.2\\
& & & UC1.2.1.2.3
\\
\midrule
RFF4.1.2.4
& Il sistema MaaP deve permettere all'utente sviluppatore di aggiungere dei pulsanti all'interno della pagina Collection-Index
& Capitolato
& UC1.2.1.2\\
& & & UC1.2.1.2.4
\\
\midrule
ROF4.1.3
& Il sistema MaaP deve permettere all'utente sviluppatore di creare la visualizzazione per la pagine Document-Show
& Capitolato
& UC1.2.1\\
& & & UC1.2.1.3
\\
\midrule
ROF4.1.3.1
& Il sistema MaaP deve permettere all'utente sviluppatore di aggiungere delle chiavi da visualizzare nella pagina Document-Show
& Capitolato
& UC1.2.1.3\\
& & & UC1.2.1.3.1
\\

\midrule
ROF4.1.3.1.1
& Il sistema MaaP deve permettere all'utente sviluppatore di aggiungere definire un'etichetta per la chiave da visualizzare
& Capitolato
& UC1.2.1.3.1\\
& & & UC1.2.1.3.1.1
\\
\midrule
ROF4.1.3.1.2
& Il sistema MaaP deve permettere all'utente sviluppatore di definire un campo associato alla chiave da visualizzare
& Capitolato
& UC1.2.1.3.1\\
& & & UC1.2.1.3.1.2
\\
\midrule
ROF4.1.3.1.3
& Il sistema MaaP deve permettere all'utente sviluppatore di definire un campo associato alla chiave da visualizzare, proveniente da un documento esterno
& Capitolato
& UC1.2.1.3.1\\
& & & UC1.2.1.3.1.3
\\
\midrule
ROF4.1.3.1.4
& Il sistema MaaP deve permettere all'utente sviluppatore di definire un campo associato alla chiave da visualizzare, proveniente dal risultato di una query
& Capitolato
& UC1.2.1.3.1\\
& & & UC1.2.1.3.1.4
\\
\midrule
ROF4.1.3.1.5
& Il sistema MaaP deve permettere all'utente sviluppatore di definire un campo associato alla chiave da visualizzare come trasformazione
& Capitolato
& UC1.2.1.3.1\\
& & & UC1.2.1.3.1.5
\\

\midrule
RFF4.1.3.2
& Il sistema MaaP deve permettere all'utente sviluppatore di aggiungere un pulsante all'interno della  pagina Document-Show
& Capitolato
& UC1.2.1.3\\
& & & UC1.2.1.3.2
\\
\midrule
ROF4.2
& Il sistema MaaP deve permettere all'utente sviluppatore di modificare la visualizzazione della Collection
& Capitolato
& UC1.2\\
& & & UC1.2.2
\\


\midrule
ROF4.2.1
& Il sistema MaaP deve permettere all'utente sviluppatore di impostare la visualizzazione del menù delle le Collection
& Capitolato
& UC1.2.2\\
& & & UC1.2.2.1
\\
\midrule
ROF4.2.1.1
& Il sistema MaaP deve permettere all'utente sviluppatore di modificare il nome della voce relativa alla Collection
& Capitolato
& UC1.2.2.1\\
& & & UC1.2.2.1.1
\\
\midrule
ROF4.2.1.2
& Il sistema MaaP deve permettere all'utente sviluppatore di modificare la posizione di una voce all'interno del menù
& Capitolato
& UC1.2.2.1\\
& & & UC1.2.2.1.2
\\
\midrule
ROF4.2.2
& Il sistema MaaP deve permettere all'utente sviluppatore di impostare la visualizzazione della pagina Collection-Index
& Capitolato
& UC1.2.2\\
& & & UC1.2.2.2
\\

\midrule
ROF4.2.2.1
& Il sistema MaaP deve permettere all'utente sviluppatore di aggiungere delle chiavi da visualizzare nella pagina Collection-Index
& Capitolato
& UC1.2.1.2\\
& & & UC1.2.1.2.1
\\
\midrule
ROF4.2.2.2
& Il sistema MaaP deve permettere all'utente sviluppatore di eliminare delle chiavi da visualizzare nella pagina Collection-Index
& Interna
& UC1.2.2.2\\
& & & UC1.2.2.2.1
\\
\midrule
ROF4.2.2.3
& Il sistema MaaP deve permettere all'utente sviluppatore di definire un ordinamento, alfabetico crescente o decrescente, rispetto a una chiave
& Capitolato
& UC1.2.1.2\\
& & & UC1.2.1.2.2
\\
\midrule
ROF4.2.2.4
& Il sistema MaaP deve permettere all'utente sviluppatore di eliminare un ordinamento rispetto a una chiave
& Interna
& UC1.2.2.2\\
& & & UC1.2.2.2.2
\\
\midrule
ROF4.2.2.5
& Il sistema deve permettere all'utente sviluppatore di  definire un numero massimo di Document da visualizzare per la pagina Collection-Index
& Capitolato
& UC1.2.1.2\\
& & & UC1.2.1.2.3
\\
\midrule
ROF4.2.2.6
& Il sistema deve permettere all'utente sviluppatore di  eliminare il numero massimo di Document da visualizzare per la pagina Collection-Index
& Capitolato
& UC1.2.2.2\\
& & & UC1.2.2.2.4
\\
\midrule
RFF4.2.2.7
& Il sistema MaaP deve permettere all'utente sviluppatore di aggiungere dei pulsanti all'interno della pagina Collection-Index, specificando il nome del pulsante e l'azione che deve eseguire
& Capitolato
& UC1.2.1.2\\
& & & UC1.2.1.2.4
\\
\midrule
RFF4.2.2.8
& Il sistema MaaP deve permettere all'utente sviluppatore di eliminare dei pulsanti all'interno della pagina Collection-Index
& Capitolato
& UC1.2.2.2\\
& & & UC1.2.2.2.5
\\
\midrule
ROF4.2.3
& Il sistema MaaP deve permettere all'utente sviluppatore di impostare la visualizzazione per la pagine Document-Show
& Capitolato
& UC1.2.2\\
& & & UC1.2.2.3
\\
\midrule
ROF4.2.3.1
& Il sistema MaaP deve permettere all'utente sviluppatore di aggiungere delle chiavi da visualizzare nella pagina Document-Show
& Capitolato
& UC1.2.1.3\\
& & & UC1.2.1.3.1
\\
\midrule
ROF4.2.3.2
& Il sistema MaaP deve permettere all'utente sviluppatore di eliminare delle chiavi da visualizzare nella pagina Document-Show
& Capitolato
& UC1.2.2.3\\
& & & UC1.2.2.3.1
\\

\midrule
RFF4.2.3.3
& Il sistema MaaP deve permettere all'utente sviluppatore di aggiungere dei pulsanti all'interno della  pagina Document-Show, specificando il nome del pulsante e l'azione che deve eseguire
& Capitolato
& UC1.2.1.3\\
& & & UC1.2.1.3.2
\\
\midrule
RFF4.2.3.4
& Il sistema MaaP deve permettere all'utente sviluppatore di eliminare dei pulsanti all'interno della  pagina Document-Show
& Capitolato
& UC1.2.2.3\\
& & & UC1.2.2.3.2
\\

\midrule
ROF4.3
& Il sistema MaaP deve permettere all'utente sviluppatore di definire una query personalizzata
& Capitolato
& UC1.2.3
\\

\midrule
ROF4.4
& Il sistema MaaP deve permettere all'utente sviluppatore di eliminare una query personalizzata
& Capitolato
& UC1.2.4
\\

\midrule
ROF5
& Il sistema deve permettere all'utente sviluppatore la modifica dei file di configurazione
& Interna
& UC1.3
\\

\midrule
RDF5.1
& Il sistema deve permettere all'utente sviluppatore di abilitare la funzionalità di registrazione per l'utente finale nelle pagine web. Nel caso la registrazione sia abilitata l'utente finale può registrarsi al sistema, altrimenti no.
& Verbale\_2013\_12\_05
& UC1.3\\
& & & UC1.3.1
\\

\midrule
RFF5.2
& Il sistema MaaP deve permettere all'utente sviluppatore di abilitare la funzionalità per la creazione di nuovi Document all'interno della pagina Collection-Index
& Capitolato
& UC1.3\\
& &  & UC1.3.2
\\

\midrule
RDF5.3
& Il sistema MaaP deve permettere all'utente sviluppatore di modificare i template per le pagine web
& Interna
& UC1.3\\
& & & UC1.3.3
\\

\midrule
ROF5.4
& Il sistema deve permettere all'utente sviluppatore di specificare nome, indirizzo e password, relativi al database di analisi con il quale interagire
& Interna
& UC1.3\\
& & & UC1.3.4
\\

\midrule
ROF5.5
& Il sistema deve permettere all'utente sviluppatore di abilitare la funzionalità per la creazione di nuovi indici all'interno della pagina Collection-Index
& Capitolato
& UC1.3\\
& & Verbale\_2013\_12\_05 & UC1.3.5
\\


%FINE TABELLA REQUISITI MAAP, NON CANCELLARE
\end{longtable}

\newpage
\subsection{Requisiti funzionali MaaP's Web}
\begin{longtable}{|c|p{6cm}|c|c|}
\caption{Requisiti funzionali MaaP's Web}
\label{tab:Requisiti MaaP's Web} \\
\toprule
\multicolumn{1}{|c}{\textbf{Requisito}} & \multicolumn{1}{|p{6cm}}{\textbf{Descrizione}}   & \multicolumn{1}{|c}{\textbf{Fonte}} & \multicolumn{1}{|c|}{\textbf{Caso d'uso}}\\
\midrule
\endfirsthead
\multicolumn{2}{l}{\footnotesize\itshape\tablename~\thetable: continua dalla pagina precedente} \\
\toprule
\multicolumn{1}{|c}{\textbf{Requisito}} & \multicolumn{1}{|p{6cm}}{\textbf{Descrizione}}   & \multicolumn{1}{|c}{\textbf{Fonte}} & \multicolumn{1}{|c|}{\textbf{Caso d'uso}}\\
\midrule
\endhead
\midrule
\multicolumn{2}{r}{\footnotesize\itshape\tablename~\thetable: continua nella prossima pagina} \\
\endfoot
\bottomrule
\multicolumn{2}{r}{\footnotesize\itshape\tablename~\thetable: si conclude dalla pagina precedente} \\
\endlastfoot

%Requisiti utente business
\midrule
ROF6
& L'utente business, al primo accesso, deve poter usare il profilo amministratore di default
& Verbale\_2013\_12\_05
&
\\

\midrule
ROF7
& L'utente business deve potersi autenticare inserendo dei dati personali
& Capitolato
& UC2\\
& & & UC2.2
\\

\midrule
ROF7.1
& L'utente business deve inserire l'email per l'autenticazione
& Capitolato
& UC2.2\\
& & & UC2.2.1
\\

\midrule
ROF7.2
& L'utente business deve inserire la password per l'autenticazione
& Capitolato
& UC2.2\\
& & & UC2.2.2
\\

\midrule
ROF7.2.1
& La password per l'autenticazione deve essere alfanumerica e contenere almeno otto caratteri
& Interna
&
\\

\midrule
RDF8
& L'utente business deve potersi registrare inserendo dei dati personali
& Verbale\_2013\_12\_05
& UC2\\
& & & UC2.1
\\

\midrule
RDF8.1
& L'utente business, per registrarsi, deve inserire una email non presente nel sistema
& Capitolato
& UC2.1\\
& & & UC2.1.1
\\

\midrule
RDF8.2
& L'utente business, per registrarsi, deve inserire una password
& Capitolato
& UC2.1\\
& & & UC2.1.2
\\

\midrule
RDF8.2.1
& La password per la registrazione deve essere alfanumerica e contenere almeno otto caratteri
& Interna
&
\\

\midrule
ROF9
& L'utente business deve poter recuperare la password
& Capitolato
& UC2\\
& & & UC2.3
\\

\midrule
ROF10
& L'utente business autenticato deve poter aprire una Collection e visualizzare la sua pagina Collection-Index
& Capitolato
& UC2\\
& & & UC2.4
\\

\midrule
ROF10.1
& L'utente business autenticato deve poter visualizzare una pagina Document-Show
& Capitolato
& UC2.4\\
& & & UC2.4.1
\\

\midrule
ROF10.1.1
& L'utente business autenticato deve poter visualizzare il Document selezionato
& Capitolato
& UC2.4.1\\
& & & UC2.4.1.1
\\

\midrule
ROF10.1.2
& L'utente business autenticato deve poter eliminare il Document che sta visualizzando
& Interna
& UC2.4.1\\
& & & UC2.4.1.2
\\

\midrule
ROF10.1.3
& L'utente business autenticato deve poter modificare il Document che sta visualizzando
& Interna
& UC2.4.1\\
& & & UC2.4.1.3
\\

\midrule
RDF10.2
& L'utente business autenticato deve poter modificare la visualizzazione dei Document
& Interna
& UC2.4\\
& & & UC2.4.2
\\

\midrule
RDF10.2.1
& L'utente business autenticato deve poter selezionare dei criteri per la visualizzazione
& Interna
& UC2.4.2\\
& & & UC2.4.2.1
\\

\midrule
RDF10.2.1.1
& L'utente business autenticato deve poter effettuare un ordinamento rispetto a una chiave
& Interna
& UC2.4.2.1\\
& & & UC2.4.2.1.1
\\

\midrule
RDF10.2.1.2
& L'utente business deve poter selezionare un numero massimo di Document da visualizzare per pagina
& Interna
& UC2.4.2.1\\
& & & UC2.4.2.1.2
\\


\midrule
RDF10.2.2
& L'utente business autenticato deve poter applicare un filtro alla visualizzazione dei Document
& Interna
& UC2.4.2\\
& & Verbale\_2013\_12\_05 & UC2.4.2.2
\\

\midrule
RDF10.2.3
& L'utente business autenticato deve poter annullare il filtro
& Interna
& UC2.4.2\\
& & Verbale\_2013\_12\_05 & UC2.4.2.3
\\

\midrule
ROF10.2.4
& L'utente business autenticato deve poter disconnettersi
& Interna
& UC2.4\\
& & & UC2.5
\\

\midrule
ROF10.2.5
& L'utente business autenticato deve poter navigare tra la Collection
& Capitolato
& UC2.4\\
& & & UC2.4.3\\

\midrule
ROF10.3
& L'utente business autenticato deve poter gestire il proprio profilo
& Interna
& UC2\\
& & & UC2.6
\\

\midrule
ROF10.3.1
& L'utente business autenticato deve poter gestire i propri dati
& Capitolato
& UC2.6\\
& & & UC2.6.3
\\

\midrule
ROF10.3.1.1
& L'utente business autenticato deve poter modificare i propri dati utente
& Capitolato
& UC2.6.3\\
& & & UC2.6.3.1
\\

\midrule
ROF10.3.1.2
& L'utente business autenticato deve poter  salvare le modifiche apportate
& Interna
& UC2.6.3\\
& & & UC2.6.3.2
\\

\midrule
ROF10.3.1.3
& L'utente business autenticato deve poter annullare le modifiche apportate
& Interna
& UC2.6.3\\
& & & UC2.6.3.5
\\

\midrule
ROF10.3.1.4
& L'utente business autenticato amministratore deve poter modificare i dati degli utenti business
& Interna
& UC2.6.3\\
& & & UC2.6.3.3
\\

\midrule
ROF10.3.1.5
& L'utente business autenticato deve poter modificare i permessi degli utenti business
& Capitolato
& UC2.6.3\\
& & & UC2.6.3.4
\\

\midrule
ROF10.3.2
& L'utente business autenticato amministratore deve poter creare un nuovo utente business
& Capitolato
& UC2.6\\
& & & UC2.6.1
\\

\midrule
ROF10.3.3
& L'utente business autenticato amministratore deve poter eliminare un utente business
& Capitolato
& UC2.6\\
& & & UC2.6.2
\\

\midrule
ROF10.4
& L'utente business autenticato amministratore deve poter cancellare un Document
& Interna
& UC2.4\\
& & & UC2.4.4
\\

\midrule
ROF10.5
& L'utente business autenticato amministratore deve poter modificare un Document
& Capitolato
& UC2.4\\
& & & UC2.4.5
\\

\midrule
ROF10.5.1
& L'utente business autenticato amministratore deve poter modificare i valori associati alla chiavi
& Interna
& UC2.4.5\\
& & & UC2.4.5.1
\\

\midrule
ROF10.5.2
& L'utente business autenticato amministratore deve poter salvare le modifiche apportate al Document
& Interna
& UC2.4.5\\
& & & UC2.4.5.2
\\

\midrule
ROF10.5.3
& L'utente business autenticato amministratore deve poter annullare le modifiche apportate al Document
& Interna
& UC2.4.5\\
& & & UC2.4.5.3
\\

\midrule
ROF10.6
& L'utente business autenticato amministratore deve poter visualizzare le query più utilizzate dal sistema MaaP
& Capitolato
& UC2.4\\
& & & UC2.4.6
\\

\midrule
ROF10.7
& L'utente business autenticato amministratore deve poter creare degli indici
& Capitolato
& UC2.4\\
& & & UC2.4.7
\\

%FINE TABELLA REQUISITI MAAPSWEB, NON CANCELLARE
\end{longtable}

\newpage
\subsection{Requisiti funzionali MaaS}
\begin{longtable}{|c|p{6cm}|c|c|}
\caption{Requisiti funzionali MaaS}
\label{tab:Requisiti MaaS} \\
\toprule
\multicolumn{1}{|c}{\textbf{Requisito}} & \multicolumn{1}{|p{6cm}}{\textbf{Descrizione}}   & \multicolumn{1}{|c}{\textbf{Fonte}} & \multicolumn{1}{|c|}{\textbf{Caso d'uso}}\\
\midrule
\endfirsthead
\multicolumn{2}{l}{\footnotesize\itshape\tablename~\thetable: continua dalla pagina precedente} \\
\toprule
\multicolumn{1}{|c}{\textbf{Requisito}} & \multicolumn{1}{|p{6cm}}{\textbf{Descrizione}}   & \multicolumn{1}{|c}{\textbf{Fonte}} & \multicolumn{1}{|c|}{\textbf{Caso d'uso}}\\
\midrule
\endhead
\midrule
\multicolumn{2}{r}{\footnotesize\itshape\tablename~\thetable: continua nella prossima pagina} \\
\endfoot
\bottomrule
\multicolumn{2}{r}{\footnotesize\itshape\tablename~\thetable: si conclude dalla pagina precedente} \\
\endlastfoot

%REQUISITI MAAS
\midrule
RFF11.1
& Il sistema MaaS deve permettere all'utente di autenticarsi al sistema
& Interna
& UC3\\
& & & UC3.6\\

\midrule
RFF11.1.1
& Il sistema MaaS deve permettere all'utente di inserire il nome utente
& Interna
& UC3.6\\
& & & UC3.6.1\\

\midrule
RFF11.1.2
& Il sistema MaaS deve permettere all'utente di inserire la password
& Interna
& UC3.6\\
& & & UC3.6.2\\

\midrule
RFF11.2
& Il sistema MaaS deve permettere all'utente di registrarsi al sistema
& Interna
& UC3\\
& & & UC3.7\\

\midrule
RFF11.2.1
& Il sistema MaaS deve permettere all'utente di inserire la propria email
& Interna
& UC3.7\\
& & & UC3.7.1\\

\midrule
RFF11.3
& Il sistema MaaS deve permettere all'utente di recuperare la password
& Interna
& UC3\\
& & & UC3.8\\

\midrule
RFF11.4
& Il sistema MaaS deve permettere all'utente di visualizzare le pagine web create
& Capitolato
& UC3\\
& & & UC3.5\\

\midrule
RFF11.5
& Il sistema MaaS deve permettere all'utente autenticato di creare lo scheletro del progetto
& Interna
& UC3\\
& & & UC3.1\\

\midrule
RFF11.5.1
& Il sistema MaaS deve permettere all'utente autenticato di inserire il nome del progetto
& Interna
& UC3.1\\
& & & UC3.1.1\\

\midrule
RFF11.6
& Il sistema MaaS deve permettere all'utente autenticato di gestire le pagine web
& Capitolato
& UC3\\
& & & UC3.2\\

\midrule
RFF11.6.1
& Il sistema MaaS deve permettere all'utente autenticato di creare un file di descrizione
& Capitolato
& UC3.2\\
& & & UC3.2.1\\

\midrule
RFF11.6.1.1
& Il sistema MaaS deve permettere all'utente autenticato la scrittura di un file di descrizione tramite editor di testo
& Capitolato
& UC3.2.1\\
& & & UC3.2.1.1\\

\midrule
RFF11.6.1.2
& Il sistema MaaS deve permettere all'utente autenticato di salvare il file di descrizione
& Interna
& UC3.2.1\\
& & & UC3.2.1.2\\


\midrule
RFF11.6.2
& Il sistema MaaS deve permettere all'utente autenticato di eseguire l'upload di un file di descrizione creato precedentemente con il sistema MaaP
& Capitolato
& UC3.2\\
& & & UC3.2.2\\

\midrule
RFF11.6.2.1
& Il sistema MaaS deve permettere all'utente autenticato di navigare all'interno del file system
& Interna
& UC3.2.2\\
& & & UC3.2.2.1\\

\midrule
RFF11.6.2.2
& Il sistema MaaS deve permettere all'utente autenticato di selezionare un file di descrizione
& Interna
& UC3.2.2\\
& & & UC3.2.2.2\\

\midrule
RFF11.6.2.3
& Il sistema MaaS deve permettere all'utente autenticato di confermare l'upload del file selezionato
& Interna
& UC3.2.2\\
& & & UC3.2.2.3\\


\midrule
RFF11.6.3
& Il sistema MaaS deve permettere all'utente autenticato di modificare un file di descrizione esistente
& Interna
& UC3.2\\
& & & UC3.2.3\\

\midrule
RFF11.6.3.1
& Il sistema MaaS deve permettere all'utente autenticato di modificare il codice del file di descrizione selezionato
& Interna
& UC3.2.3\\
& & & UC3.2.3.1\\

\midrule
RFF11.6.3.2
& Il sistema MaaS deve permettere all'utente autenticato di salvare le modifiche apportate al file di descrizione
& Interna
& UC3.2.3\\
& & & UC3.2.3.2\\

\midrule
RFF11.6.3.3
& Il sistema MaaS deve permettere all'utente autenticato di annullare le modifiche apportate al file di descrizione
& Interna
& UC3.2.3\\
& & & UC3.2.3.3\\

\midrule
RFF11.6.4
& Il sistema MaaS deve permettere all'utente autenticato di modificare il file di configurazione
& Capitolato
& UC3.2\\
& & & UC3.2.4\\

\midrule
RFF11.6.4.1
& Il sistema MaaS deve permettere all'utente autenticato di modificare il codice del file di configurazione selezionato
& Interna
& UC3.2.4\\
& & & UC3.2.4.1\\

\midrule
RFF11.6.4.2
& Il sistema MaaS deve permettere all'utente autenticato di salvare le modifiche apportate al file di configurazione
& Interna
& UC3.2.4\\
& & & UC3.2.4.2\\

\midrule
RFF11.6.4.3
& Il sistema MaaS deve permettere all'utente autenticato di annullare le modifiche apportate al file di configurazione
& Interna
& UC3.2.4\\
& & & UC3.2.4.3\\

\midrule
RFF11.6.5
& Il sistema MaaS deve permettere all'utente autenticato di visualizzare tutti i file di descrizione presenti
& Capitolato
& UC3.2\\
& & & UC3.2.5\\

\midrule
RFF11.7
& Il sistema MaaS deve permettere all'utente autenticato di gestire il proprio profilo utente
& Interna
& UC3\\
& & & UC3.3\\

\midrule
RFF11.7.1
& Il sistema MaaS deve permettere all'utente autenticato di generare una nuova password
& Interna
& UC3.3\\
& & & UC3.3.1\\

\midrule
RFF11.8
& Il sistema MaaS deve permettere all'utente autenticato di disconnettersi dal sistema
& Interna
& UC3\\
& & & UC3.4\\

\midrule
RFF11.9
& Il sistema MaaS deve permettere all'utente autenticato di visualizzare le pagine web create
& Capitolato
& UC3\\
& & & UC3.5\\

\midrule
ROF12
& Definizione di un linguaggio astratto DSL per la definizione delle pagine che verranno generate
& Capitolato
&
\\

\midrule
ROF12.1
& Il linguaggio definito deve essere testuale
& Capitolato
&
\\

%FINE TABELLA REQUISITI MAAS, NON CANCELLARE
\end{longtable}
\newpage
\subsection{Requisiti di qualità}
\begin{longtable}{|c|p{6cm}|c|c|}
\caption{Requisiti di qualità}
\label{tab:Requisiti di qualita} \\
\toprule
\multicolumn{1}{|c}{\textbf{Requisito}} & \multicolumn{1}{|p{6cm}}{\textbf{Descrizione}}   & \multicolumn{1}{|c}{\textbf{Fonte}} & \multicolumn{1}{|c|}{\textbf{Caso d'uso}}\\
\midrule
\endfirsthead
\multicolumn{2}{l}{\footnotesize\itshape\tablename~\thetable: continua dalla pagina precedente} \\
\toprule
\multicolumn{1}{|c}{\textbf{Requisito}} & \multicolumn{1}{|p{6cm}}{\textbf{Descrizione}}   & \multicolumn{1}{|c}{\textbf{Fonte}} & \multicolumn{1}{|c|}{\textbf{Caso d'uso}}\\
\midrule
\endhead
\midrule
\multicolumn{2}{r}{\footnotesize\itshape\tablename~\thetable: continua nella prossima pagina} \\
\endfoot
\bottomrule
\multicolumn{2}{r}{\footnotesize\itshape\tablename~\thetable: si conclude dalla pagina precedente} \\
\endlastfoot

% Requisiti di qualita

\midrule
ROQ13
& Devono essere rispettate tutte le norme riportate nel documento Norme di Progetto
& Interna
&
\\

\midrule
ROQ14
& Il \gloss{software} deve essere dotato di un manuale utente che ne descriva l'utilizzo
& Interna
&
\\

\midrule
ROQ15
& Il progetto deve essere pubblicato su \gloss{GitHub} e utilizzare le sue distribuzioni per segnalare eventuali correzioni o errori
& Capitolato
&
\\

%FINE TABELLA REQUISITI QUALITA', NON CANCELLARE
\end{longtable}

\newpage
\subsection{Requisiti di vincolo}
\begin{longtable}{|c|p{6cm}|c|c|}
\caption{Requisiti di vincolo}
\label{tab:Requisiti di vincolo} \\
\toprule
\multicolumn{1}{|c}{\textbf{Requisito}} & \multicolumn{1}{|p{6cm}}{\textbf{Descrizione}}   & \multicolumn{1}{|c}{\textbf{Fonte}} & \multicolumn{1}{|c|}{\textbf{Caso d'uso}}\\
\midrule
\endfirsthead
\multicolumn{2}{l}{\footnotesize\itshape\tablename~\thetable: continua dalla pagina precedente} \\
\toprule
\multicolumn{1}{|c}{\textbf{Requisito}} & \multicolumn{1}{|p{6cm}}{\textbf{Descrizione}}   & \multicolumn{1}{|c}{\textbf{Fonte}} & \multicolumn{1}{|c|}{\textbf{Caso d'uso}}\\
\midrule
\endhead
\midrule
\multicolumn{2}{r}{\footnotesize\itshape\tablename~\thetable: continua nella prossima pagina} \\
\endfoot
\bottomrule
\multicolumn{2}{r}{\footnotesize\itshape\tablename~\thetable: si conclude dalla pagina precedente} \\
\endlastfoot

% Requisiti di vincolo

\midrule
ROV16
& Il database degli utenti deve essere \gloss{criptato}
& Interna
&
\\

\midrule
ROV17
& Le pagine web prodotte dal framework MaaP devono essere compatibili con la versione 30.0.x di \gloss{Google Chrome} o superiori
& Capitolato
&
\\

\midrule
ROV18
& Le pagine web prodotte dal framework MaaP devono essere compatibili con la versione 24.x o superiore di Firefox
& Capitolato
&
\\

\midrule
ROV19
& Il sistema deve accettare solo file di configurazione che hanno un determinato formato già fissato
& Interna
&
\\

\midrule
ROV20
& Il database degli utenti deve essere realizzato utilizzando MongoDB
& Capitolato
&
\\

\midrule
ROV21
& Il database degli utenti deve essere indipendente dal database di analisi
& Capitolato
&
\\

\midrule
ROV22
& Il database di analisi utilizzato deve essere stato realizzato utilizzando MongoDB
& Capitolato
&
\\

\midrule
ROV23
& L'\gloss{interfaccia}  con il database deve essere realizzata con Mongoose
& Capitolato
&
\\

\midrule
ROV24
& L'\gloss{infrastruttura} delle pagine web generate deve essere realizzata con \gloss{Express}
& Capitolato
&
\\

\midrule
ROV25
& La componente \gloss{server} deve essere realizzata con \gloss{Node.js}
& Capitolato
&
\\

\midrule
ROV26
& Il software sarà fornito di un sistema di installazione per farlo funzionare
& Interna
&
\\

\midrule
ROV27
& Deve essere possibile effettuare il \gloss{deployment} su \gloss{Heroku}
& Capitolato
&
\\

%FINE TABELLA REQUISITI DI VINCOLO, NON CANCELLARE
\end{longtable}


%AJK-FINEREQUISITI NON CANCELLARE QUESTO COMMENTO XKE SERVE ALLO SCRIPT REQUISITI :)

\subsection{Requisiti accettati}
Tutti i requisiti obbligatori e desiderabili saranno implementati. A causa di tempo e risorse limitate
i requisiti opzionali non potranno essere soddisfatti.




\subsection{Note sui requisiti}
Dal capitolato è emerso un quantitativo di requisiti chiaramente obbligatori relativamente basso. Data la quantità di risorse a disposizione per lo sviluppo è stato deciso di classificare come desiderabili quei requisiti che, pur non essendo obbligatori, saranno comunque soddisfatti.
\newpage
\section{Tracciamento requisiti - casi d'uso}
\label{5.0}

%%DENTRO QUESTO FILE C'è ANCHE LA intestazione TABELLA


\begin{longtable}{|c|p{7cm}|c|}
\caption{Tracciamento dei requisiti}
\label{tab:Tracciamento requisiti} \\
\toprule
\multicolumn{1}{|c}{\textbf{Caso d'uso}} & \multicolumn{1}{|p{7cm}}{\textbf{Descrizione}}  & \multicolumn{1}{|c|}{\textbf{Requisiti}} \\
\midrule
\endfirsthead
\multicolumn{2}{l}{\footnotesize\itshape\tablename~\thetable: continua dalla pagina precedente} \\
\toprule
\multicolumn{1}{|c}{\textbf{Caso d'uso}} & \multicolumn{1}{|p{7cm}}{\textbf{Descrizione}}  & \multicolumn{1}{|c|}{\textbf{Requisiti}} \\
\midrule
\endhead
\midrule
\multicolumn{2}{r}{\footnotesize\itshape\tablename~\thetable: continua nella prossima pagina} \\
\endfoot
\bottomrule
\multicolumn{2}{r}{\footnotesize\itshape\tablename~\thetable: si conclude dalla pagina precedente} \\
\endlastfoot


\midrule
UC1
& un utente deve poter scegliere un file opportuno da cui caricare la scena 3D descritta. Una volta caricata, deve essere possibile modificarla ed esportarla nel formato desiderato. Il caricamento e la modifica della scena 3D comportano la visualizzazione dell'anteprima della stessa
& RFOb1 \\
& & RFOb2 \\
& & RFOb2.1 \\
& & RFDe2.2 \\
& & RFDe2.3 \\
& & RFDe2.4 \\
& & RFOb3 \\
& & RFOb3.1 \\
& & RFOb3.2 \\
& & RFOb3.3 \\
& & RFOb4 \\
& & RFOb4.1 \\
& & RFOb4.2 \\
& & RFOb4.3 \\
& & RFOp5 \\
& & RFOp6 \\
& & RFDe7 \\


\midrule
UC1.1
& un utente deve poter selezionare il formato del file dal quale caricare la scena 3D e scegliere il percorso del file medesimo
& RFOb1.1 \\


\midrule
UC1.1.1
& un utente deve poter selezionare il formato del file dal quale caricare la scena 3D
& RFOb1.1.1 \\


\midrule
UC1.1.1.1
& un utente deve poter importare dal formato 3DS
& RFOb1.1.1.1 \\


\midrule
UC1.1.1.2
& un utente deve poter importare dal formato Wavefront OBJ/MTL
& RFDe1.1.1.2 \\


\midrule
UC1.1.1.3
& un utente deve poter importare dal formato JSON
& RFDe1.1.1.3 \\


\midrule
UC1.1.1.4
& un utente deve poter importare dal formato XML
& RFDe1.1.1.4 \\


\midrule
UC1.1.2
& un utente deve poter selezionare il percorso del file da cui caricare la scena 3D
& RFOb1.1.2 \\


\midrule
UC1.2
& un utente deve poter modificare la scena 3D
& RFDe1.2 \\


\midrule
UC1.2.1
& un utente deve poter selezionare la fonte di luce che intende modificare
& RFDe1.2.1 \\


\midrule
UC1.2.2
& un utente deve poter selezionare la mesh che intende modificare
& RFDe1.2.2 \\


\midrule
UC1.2.3
& un utente deve poter aggiungere una nuova fonte di luce con valori di default
& RFDe1.2.3 \\


\midrule
UC1.2.4
& un utente deve poter modificare la mesh selezionata
& RFDe1.2.4 \\


\midrule
UC1.2.4.1
& un utente deve poter scegliere i valori del vettore [x,y,z] di traslazione da applicare alla mesh selezionata
& RFDe1.2.4.1 \\


\midrule
UC1.2.4.1.2
& un utente deve poter impostare il valore della componente y del vettore di traslazione
& RFDe1.2.4.1.1 \\
& & RFDe1.2.4.1.2 \\


\midrule
UC1.2.4.1.3
& un utente deve poter impostare il valore della componente z del vettore di traslazione
& RFDe1.2.4.1.3 \\


\midrule
UC1.2.4.2
& un utente deve poter ruotare la mesh selezionata scegliendo l’asse di rotazione e l’angolo theta di rotazione
& RFDe1.2.4.2 \\


\midrule
UC1.2.4.2.1
& un utente deve poter scegliere l'asse di rotazione in base al quale ruotare la mesh selezionata
& RFDe1.2.4.2.1 \\


\midrule
UC1.2.4.2.1.1
& un utente deve poter scegliere l'asse x come asse di rotazione
& RFDe1.2.4.2.1.1 \\


\midrule
UC1.2.4.2.1.2
& un utente deve poter scegliere l'asse y come asse di rotazione
& RFDe1.2.4.2.1.2 \\


\midrule
UC1.2.4.2.1.3
& un utente deve poter scegliere l'asse z come asse di rotazione
& RFDe1.2.4.2.1.3 \\


\midrule
UC1.2.4.2.2
& un utente deve poter impostare l'ampiezza dell'angolo theta di rotazione secondo il quale routare la mesh selezionata
& RFDe1.2.4.2.2 \\


\midrule
UC1.2.4.3
& un utente deve poter modificare la dimensione della mesh selezionata
& RFDe1.2.4.3 \\


\midrule
UC1.2.4.3.1
& un utente deve poter ridimensionare la mesh selezionata secondo i suoi assi
& RFDe1.2.4.3.1 \\


\midrule
UC1.2.4.3.1.1
& un utente deve poter impostare il valore della componente x
& RFDe1.2.4.3.1.1 \\


\midrule
UC1.2.4.3.1.2
& un utente deve poter impostare il valore della componente y
& RFDe1.2.4.3.1.2 \\


\midrule
UC1.2.4.3.1.3
& un utente deve poter impostare il valore della componente z
& RFDe1.2.4.3.1.3 \\


\midrule
UC1.2.4.3.2
& un utente deve poter ridimensionare la mesh impostando un moltiplicatore scalare
& RFDe1.2.4.3.2 \\


\midrule
UC1.2.4.3.2.1
& un utente deve poter impostare il moltiplicatore scalare relativo al ridimensionamento di una mesh
& RFDe1.2.4.3.2.1 \\


\midrule
UC1.2.4.4
& un utente deve poter modificare le caratteristiche di uno dei materiali che compongono la mesh selezionata
& RFDe1.2.4.4 \\


\midrule
UC1.2.4.4.1
& un utente deve poter modificare la componente riflessiva diffusa di un materiale
& RFDe1.2.4.4.1 \\


\midrule
UC1.2.4.4.2
& un utente deve poter modificare la componente emissiva di un materiale
& RFDe1.2.4.4.2 \\


\midrule
UC1.2.4.4.3
& un utente deve poter modificare la componente riflessiva speculare di un materiale
& RFDe1.2.4.4.3 \\


\midrule
UC1.2.4.4.4
& un utente deve poter modificare la componente riflessiva ambientale di un materiale
& RFDe1.2.4.4.4 \\


\midrule
UC1.2.4.4.5
& un utente deve poter modificare il valore di opacità del materiale selezionato
& RFDe1.2.4.4.5 \\


\midrule
UC1.2.4.5
& un utente deve poter rimuovere la mesh selezionata
& RFOp1.2.4.5 \\


\midrule
UC1.2.5
& un utente deve poter selezionare la camera che intende modificare
& RFOp1.2.5 \\


\midrule
UC1.2.6
& un utente deve poter modificare la camera selezionata
& RFOp1.2.6 \\


\midrule
UC1.2.6.1
& un utente deve poter modificare la posizione della camera
& RFOp1.2.6.1 \\


\midrule
UC1.2.6.1.1
& un utente deve poter traslare la camera
& RFOp1.2.6.1.1 \\


\midrule
UC1.2.6.1.1.1
& un utente deve poter avvicinare la camera al centro della scena 3D
& RFOp1.2.6.1.1.1 \\


\midrule
UC1.2.6.1.1.2
& un utente deve poter allontanare la camera dal centro della scena 3D
& RFOp1.2.6.1.1.2 \\


\midrule
UC1.2.6.1.1.3
& un utente deve poter impostare lo spostamento della camera
& RFOp1.2.6.1.1.3 \\


\midrule
UC1.2.6.1.2
& un utente deve poter ruotare la camera attorno alla scena 3D
& RFOp1.2.6.1.2 \\


\midrule
UC1.2.6.2
& un utente deve poter rimuovere la camera selezionata
& RFOp1.2.6.2 \\


\midrule
UC1.2.7
& un utente deve poter modificare le caratteristiche di una fonte di luce (quali posizione, intensità, colore e tipologia)
& RFDe1.2.7 \\


\midrule
UC1.2.7.1
& un utente deve poter rimuovere una fonte di luce
& RFOp1.2.7.1 \\


\midrule
UC1.2.7.2
& un utente deve poter modificare la posizione di una fonte di luce
& RFDe1.2.7.2 \\


\midrule
UC1.2.7.3
& l’utente deve poter modificare i parametri d’attenuazione di una fonte di luce
& RFOp1.2.7.3 \\


\midrule
UC1.2.7.3.1
& l'utente deve poter modificare il parametro di attenuazione costante
& RFOp1.2.7.3.1 \\


\midrule
UC1.2.7.3.2
& l'utente deve poter modificare il parametro di attenuazione lineare
& RFOp1.2.7.3.2 \\


\midrule
UC1.2.7.3.3
& l'utente deve poter modificare il paramentro di attenuazione quadratica
& RFOp1.2.7.3.3 \\


\midrule
UC1.2.7.4
& un utente deve poter modificare il colore di una fonte di luce, alterando i valori RGB dei componenti che la compongono
& RFDe1.2.7.4 \\


\midrule
UC1.2.7.4.1
& un utente deve poter selezionare la componente di luce il cui colore vuole modificare
& RFDe1.2.7.4.1 \\


\midrule
UC1.2.7.4.1.1
& un utente deve poter selezionare la componente ambient di una fonte di luce
& RFDe1.2.7.4.1.1 \\


\midrule
UC1.2.7.4.1.2
& un utente deve poter selezionare la componente specular di una fonte di luce
& RFDe1.2.7.4.1.2 \\


\midrule
UC1.2.7.4.1.3
& un utente deve poter selezionare la componente diffuse di una fonte di luce
& RFDe1.2.7.4.1.3 \\


\midrule
UC1.2.7.4.2
& un utente deve poter impostare i valori RGB di una componente di luce
& RFDe1.2.7.4.2 \\


\midrule
UC1.2.7.4.2.1
& un utente deve poter impostare il valore red di una componente di una fonte di luce
& RFDe1.2.7.4.2.1 \\


\midrule
UC1.2.7.4.2.2
& un utente deve poter impostare il valore green di una componente di una fonte di luce
& RFDe1.2.7.4.2.2 \\


\midrule
UC1.2.7.4.2.3
& un utente deve poter impostare il valore blue di una componente di una fonte di luce
& RFDe1.2.7.4.2.3 \\


\midrule
UC1.2.7.5
& un utente deve poter modificare la tipologia corrente di una fonte di luce
& RFDe1.2.7.5 \\


\midrule
UC1.2.7.5.1
& un utente deve poter modificare la tipologia corrente di una fonte di luce in spotlight
& RFDe1.2.7.5.1 \\


\midrule
UC1.2.7.5.2
& un utente deve poter modificare la tipologia corrente di una fonte di luce in omni light
& RFDe1.2.7.5.2 \\


\midrule
UC1.2.7.5.3
& un utente deve poter modificare la tipologia corrente di una fonte di luce in directional light
& RFDe1.2.7.5.3 \\


\midrule
UC1.3
& un utente deve poter scegliere in che formato esportare la scena 3D, scegliere la directory di destinazione e il nome del file. La scelta del formato comporta la scelta della precisione
& RFOb1.3 \\


\midrule
UC1.3.1
& un utente deve poter scegliere il formato verso il quale esportare
& RFOb1.3.1 \\


\midrule
UC1.3.1.1
& un utente deve poter scegliere se esportare la scena 3D in JSON “compatto” o “leggibile”
& RFOb1.3.1.1 \\


\midrule
UC1.3.1.1.1
& un utente deve poter esportare la scena 3D nel formato JSON "compatto"
& RFOb1.3.1.1.1 \\


\midrule
UC1.3.1.1.2
& un utente deve poter esportare la scena 3D nel formato JSON "leggibile"
& RFOb1.3.1.1.2 \\


\midrule
UC1.3.1.2
& un utente deve poter esportare la scena 3D nel formato XML
& RFDe1.3.1.2 \\


\midrule
UC1.3.2
& un utente deve poter scegliere se esportare la scena 3D in singola o doppia precisione
& RFOb1.3.2 \\


\midrule
UC1.3.2.1
& un utente deve poter esportare la scena 3D in singola precisione
& RFOb1.3.2.1 \\


\midrule
UC1.3.2.2
& un utente deve poter esportare la scena 3D in doppia precisione
& RFOb1.3.2.2 \\


\midrule
UC1.3.3
& un utente deve poter scegliere la directory di destinazione del file di esportazione
& RFDe1.3.3 \\


\midrule
UC1.3.4
& un utente deve poter scegliere il nome del file di esportazione
& RFDe1.3.4 \\


\midrule
UC1.4
& un utente deve poter visualizzare l’anteprima della scena 3D
& RFDe1.4 \\


\midrule
UC1.4.1
& un utente deve avere la possibilità di spostare il punto d’osservazione della scena 3D
& RFDe1.4.1 \\


\midrule
UC1.4.1.1
& un utente deve poter traslare il punto d'osservazione della scena 3D
& RFDe1.4.1.1 \\


\midrule
UC1.4.1.1.1
& un utente deve poter avvicinare il punto d'osservazione al centro della scena 3D
& RFDe1.4.1.1.1 \\


\midrule
UC1.4.1.1.2
& un utente deve poter allontanare il punto d'osservazione dal centro della scena 3D
& RFDe1.4.1.1.2 \\


\midrule
UC1.4.1.1.3
& un utente deve poter impostare lo spostamento del punto di osservazione
& RFDe1.4.1.1.3 \\


\midrule
UC1.4.1.2
& un utente deve poter ruotare il punto d'osservazione attorno alla scena 3D
& RFDe1.4.1.2 \\


\midrule
UC1.4.2
& un utente deve poter eseguire il rendering della scena 3D
& RFOp1.4.2 \\


\midrule
UC1.5
& un utente deve poter agire sulle modifiche effettuate
& RFDe1.5 \\


\midrule
UC1.5.1
& un utente deve poter annullare l'ultima modifica effettuata
& RFDe1.5.1 \\


\midrule
UC1.5.2
& un utente deve poter ripetere l'ultima modifica annullata
& RFDe1.5.2 \\




\end{longtable}




\end{document}
