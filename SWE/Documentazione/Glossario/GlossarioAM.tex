%\newcommand {\gloss} [1]{#1}
%nuova pagina con i termini che iniziano con A
\Inizio{}


\paginaGlossario{A}


\elemento{Account} 
{L'insieme di funzionalità, strumenti e contenuti messi a disposizione da un \gloss{sito web} o da qualsiasi altro tipo di applicazione ad un \gloss{utente}, per usufruire di determinati \gloss{servizi} che il sistema offre.}

\elemento{Adapter}
{\gloss{Design pattern} strutturale che ha il compito di convertire l'\gloss{interfaccia} di una \gloss{classe} in un'altra.
Il fine dell'adapter è quindi di fornire una soluzione astratta al problema dell'interoperabilità tra interfacce differenti.}

\elemento{Agente di provider} 
{Un agente che che ha la capacità e l'autorizzazione di eseguire azioni associate con un \gloss{servizio web} dedicato al suo possessore, ovvero l' entità di \gloss{provider}.}

\elemento{Algoritmo}
{Procedimento che risolve un determinato problema attraverso un numero finito di passi.} 

\elemento{Alfanumerico}
{Carattere singolo composto solamente da lettere latine maiuscole o minuscole dalla a alla z e dalle cifre  0 al 9.}

\elemento{Ambiente}
{Consiste sia del sistema \gloss{hardware} che di quello \gloss{software} sui quali è stato pianificato l'utilizzo del prodotto software sviluppato.}

\elemento{Ambiguo}
{Termine usato per indicare una cosa che non è chiara, ovvero presenta un doppio senso o un fraintendimento.}

\elemento{Amministratore} 
{Classe di utenza privilegiata rispetto ad \gloss{utente} normale, ovvero ha funzionalità in più che gestiscono un sistema.}

\elemento{AngularJS}
{\gloss{Framework} \gloss{open-source} scritto in \gloss{JavaScript} e mantenuto da \glossary{Google}, adatto a sviluppare \gloss{applicazioni web}. Il suo obiettivo principale è fornire alle applicazioni web le funzionalità di \gloss{MVC}, in modo da rendere sia lo sviluppo che il test più semplici.}

\elemento{Applicazioni client-server}
{Applicazioni nelle quali esistono due componenti: un \gloss{client} che fa una richiesta al \gloss{server} e un server che deve rispondere alla richiesta del client.}

\elemento{Applicazione web} 
{Applicazione fruibile via \gloss{web}, solitamente con architettura \gloss{client-server}.}

\elemento{Architettura}
{Ha molteplici definizioni, tra cui decomposizione del sistema in componenti, organizzazione di tali componenti mediante definizione di ruoli, responsabilità e interazioni. }
 
\elemento{Array} 
{Struttura dati complessa, statica e omogenea, usata in molti \gloss{linguaggi di programmazione}; è detto anche "vettore".
 Si può immaginare un array come una sorta di casellario, le cui caselle sono dette celle dell'array stesso. Ciascuna delle celle si comporta come una \gloss{variabile} tradizionale che rappresenta un elemento dell'array; tutte le celle sono variabili di uno stesso tipo preesistente, detto \gloss{tipo base} dell'array.}

\elemento{Astrazione}
{Applicazione del metodo logico di astrazione nella strutturazione della descrizione dei sistemi informatici complessi, per facilitarne la progettazione e manutenzione o la stessa comprensione.}

\elemento{Attività}
{Un'attività è una parte di un processo che non include decisioni e che quindi non è utile scomporre ulteriormente (sebbene la scomposizione sia di per sé possibile). Le attività, quindi, possono sostanziarsi in operazioni su oggetti fisici o informativi oppure in una decisione assunta da un attore coinvolto nel processo.}

\elemento{Attore}
{Ruolo coperto da un certo insieme di entità interagenti con il sistema.}

\elemento{Attributo}
{Descrizione di un campo dati di una \gloss{classe}.}


\paginaGlossario{B}


\elemento{Base di dati} 
{Detta anche banca dati, indica un archivio di dati, o un insieme di archivi, in cui le informazioni in esso contenute sono strutturate e collegate tra loro. I contenuti sono legati tra loro secondo un particolare modello logico e in modo tale da consentire la gestione/organizzazione efficiente dei dati stessi. Inoltre consente l'interfacciamento con le richieste dell'\gloss{utente} attraverso le cosiddette \gloss{query} (di ricerca o interrogazione, inserimento, cancellazione, aggiornamento ecc.) grazie a particolari applicazioni \gloss{software} dedicate, basate su un'architettura di tipo \gloss{client-server}.}

\elemento{Baseline}
{Durante la pianificazione, rappresenta la suddivisione iniziale delle attività nel tempo, a cui si fa riferimento per l'avanzamento del processo.}

\elemento{Best practice} 
{Tradotto con "buona prassi"; in genere identifica le esperienze più significative, o comunque quelle che hanno permesso di ottenere migliori risultati, relativamente a svariati contesti.}

\elemento{Branch}
{Ramo di sviluppo software parallelo di quello principale.}

\elemento{Browser} 
{\gloss{Programma} che consente di usufruire dei \gloss{servizi} di connettività in \gloss{Internet}, o di una \gloss{rete} di computer, e di navigare sul \gloss{Web}.}

\elemento{Business}
{Letteralmente tradotto in "affari"; identifica un'attività economica.}

\elemento{Business logic} 
{Si riferisce a tutta quella logica applicativa che rende operativa un'applicazione. Termine largamente utilizzato nella progettazione del \gloss{software} per individuare un componente di una \gloss{architettura}. Nelle \gloss{applicazioni web} viene eseguita da un \gloss{server} su richiesta di un \gloss{client} attraverso il \gloss{browser} e interfacciandosi con la parte dati che può essere un \gloss{database}.}

\elemento{Bytecode} 
{\gloss{Codice} di \gloss{programmazione} che, una volta effettuata la \gloss{compilazione}, è eseguibile attraverso una \gloss{macchina virtuale} invece che dal \gloss{processore} di un computer.}


\paginaGlossario{C}


\elemento{Camel case}
{Modo di scrivere parole composte o frasi unendo tutte le parole tra di loro, ma lasciando le loro iniziali maiuscole.}

\elemento{Chat}
{Sistema di comunicazione in tempo reale che permette a più \gloss{utenti} di scambiarsi brevi messaggi scritti, emulando una conversazione.}

\elemento{Chiave}
{Identificatore di un elemento del \gloss{database} di tipo \gloss{NoSQL}.}

\elemento{Ciclo di vita}
{Insieme di stati assunti da un prodotto Software. Questi stati sono : Concezione, sviluppo, utilizzo e ritiro.}

\elemento{Classe}
{Nella \gloss{programmazione} \gloss{orientata agli oggetti} una classe è un costrutto di un \gloss{linguaggio di programmazione} usato come modello per creare \gloss{oggetti}. Il modello comprende \gloss{attributi} e \gloss{metodi} che saranno condivisi da tutti gli oggetti creati (\gloss{istanze}) a partire dalla classe. Un "oggetto" è, di fatto, l'istanza di una classe.}

\elemento{Client} 
{Componente che accede ai \gloss{servizi} o alle risorse forniti da un'altra componente, detta \gloss{server}.} 

\elemento{Cloud} 
{Insieme di tecnologie che permettono di archiviare e/o elaborare dati mediante l'utilizzo di risorse \gloss{hardware} e \gloss{software} \gloss{distribuite}.}

\elemento{Cluster}
{Insieme di computer interconnessi tra loro che lavorano parallelamente. In questo modo, si ottiene una grande capacità di calcolo e di memorizzazione.}

\elemento{Codice} 
{Vedi codice sorgente.}

\elemento{Codice oggetto}
{Traduzione del \gloss{codice sorgente} in \gloss{linguaggio macchina}, comprensibile dal \gloss{processore}.}

\elemento{Codice sorgente}
{Linee di codice che compongono un \gloss{programma} scritto in un \gloss{linguaggio di programmazione}. Abbreviato anche come "sorgente" o solamente "codice".}

\elemento{Collection} 
{Raccolta di uno o più \gloss{Document} in un \gloss{database}.}

\elemento{Collection-Index} 
{Pagina generata dal \gloss{framework} \gloss{MaaP} che mostra un elenco di \gloss{Document} con delle specifiche coppie chiave-valore. Inoltre è presente un menù grazie al quale è possibile spostarsi tra le varie pagine \gloss{Collection Index}, o sfruttare altre funzionalità messe a disposizione dallo \gloss{sviluppatore}.}

\elemento{Compatibilità}
{In informatica indica la proprietà di due o più computer che, scambiandosi dei dati, accettano gli stessi formati di dati senza fare modifiche ad essi.}

\elemento{Compilazione} 
{Processo di traduzione che porta alla creazione di un \gloss{codice oggetto} partendo da un \gloss{codice sorgente}.}

\elemento{Componente}
{Elemento costitutivo di un sistema. Nell'ambito informatico, identifica un elemento singolo all'interno di un'architettura software.}

\elemento{Consuntivo}
{Costituisce il rendiconto finale di un periodo di attività.}

\elemento{Controller} 
{Tipo di componente del framework MVC. Un controller accetta input e li converte in comandi per il \gloss{Model} o la \gloss{View}.} 
 
\elemento{Criptato} 
{Aggettivo proprio di un insieme di dati, modificato in modo da non risultare comprensibile a terzi.}


\paginaGlossario{D}


\elemento{Dashboard}
{Pagina principale dove l'\gloss{utente} può aver accesso alle varie funzionalità in modo chiaro e diretto.}

\elemento{Database} 
{Vedi \gloss{basi di dati}.}

\elemento{Database administration} 
{Funzione di gestione e manutenzione di un \gloss{database}.}

\elemento{Debugger} 
{\gloss{Software} specificatamente progettato per l'analisi e l'eliminazione dei bug, ovvero errori di programmazione interni al \gloss{codice} di altri \gloss{programmi}.}

\elemento{Default} 
{Valore o azione standard che viene assegnato/a a priori, per poi essere (opzionalmente) modificato in futuro.}

\elemento{Dependency Injection}
{\gloss{Design Pattern} il cui scopo è quello di semplificare lo sviluppo e migliorare la testabilità di \gloss{software} di grandi dimensioni. Inoltre consente di standardizzare e centralizzare il modo in cui gli oggetti sono costruiti nelle applicazioni.}
 
\elemento{Deployment} 
{Traducibile letteralmente in "spiegamento". \`{E} l'insieme di attività necessarie a rendere un sistema \gloss{software} disponibile all'uso.}

\elemento{Design Pattern}
{Soluzione generale riusabile a problemi che occorrono comunemente, in un contesto di progettazione software.}

\elemento{Diagramma delle classi}
{Descrizione del tipo degli oggetti che compongono un sistema.}

\elemento{Diagramma dei package}
{Diagramma che descrive i \gloss{package} utilizzati nel sistema.}

\elemento{Diagramma di attività}
{Diagramma \gloss{UML} usato per descrivere il comportamento dinamico di un sistema.}

\elemento{Diagramma di Gantt}
{Strumento di supporto alla gestione dei progetti. \`{E} costruito partendo da un asse orizzontale, a rappresentazione dell'arco temporale totale del progetto, suddiviso in fasi incrementali (ad esempio, giorni, settimane, mesi), e da un asse verticale, a rappresentazione delle mansioni o attività che costituiscono il progetto. Un diagramma di \gloss{Gantt} permette dunque la rappresentazione grafica di un calendario di attività, utile al fine di pianificare, coordinare e tracciare specifiche attività in un progetto dando una chiara illustrazione dello stato d'avanzamento del progetto rappresentato.} 

\elemento{Diagramma di sequenza}
{Diagramma che descrive la collaborazione di un gruppo di oggetti che devono implementare collettivamente un comportamento.}

\elemento{Directives}
{Costrutti del linguaggio che specificano come un compilatore (o assemblatore o interprete) dovrebbe processare il suo input.} 

\elemento{Disaccoppiamento}
{Si intende il grado con cui ciascuna componente del programma non si affida su ciascuna delle altre componenti, non dipendendo, appunto, da esso.}

\elemento{Distribuito}
{Insieme di entità autonome (componenti \gloss{software} e \gloss{hardware}) fisicamente separate che comunicano e coordinano tra loro le azioni attraverso scambio di messaggi.}

\elemento{Distribuzione} 
{Collezione di \gloss{programmi} relativi ad uno o più campi di applicazione, selezionati e rilasciati come un unico pacchetto. Ad esempio il \gloss{sistema operativo} \gloss{Linux} offre diverse distribuzioni, come ad esempio Ubuntu, Debian, Fedora, eccetera.}

\elemento{Document}
{\gloss{Istanza}, o \gloss{record}, di un \gloss{database} in \gloss{MongoDB}. Esso è costituito da un insieme di \gloss{chiavi} associate al rispettivo valore.}

\elemento{Document-Show} 
{Pagina che, in seguito ad una selezione di una \gloss{chiave} di un \gloss{Document} nella pagina \gloss{Collection-Index}, mostra il Document per intero con tutte le chiavi e i relativi valori.}

\elemento{DOM}
{Acronimo di "Document Object Model", è una forma di rappresentazione dei documenti strutturati come modello orientato agli oggetti.
DOM è lo standard ufficiale del \gloss{W3C} per la rappresentazione di documenti strutturati in maniera da essere neutrali sia per la lingua che per la piattaforma.}

\elemento{Driver}
{Insieme di procedure che permette ad un \gloss{sistema operativo} di controllare un dispositivo \gloss{hardware} o \gloss{software}.}

\elemento{DSL} 
{Acronimo di "Domain Specific Language", tradotto in "linguaggio specifico di dominio". \`{E} un \gloss{linguaggio di programmazione} altamente contestualizzato, cioè associato ad un dominio specifico.} 


\paginaGlossario{E}

\elemento{Eclipse}
{Ambiente di sviluppo integrato multi-linguaggio e multipiattaforma. Ideato da un consorzio di grandi società chiamato Eclipse Foundation sullo stile dell'open source.}

\elemento{Editor}
{\gloss{Programma} di composizione di testi, il suo scopo è facilitare la scrittura di un testo. \`{E} generalmente incluso in ogni \gloss{sistema operativo}.}

\elemento{Email} 
{\gloss{Servizio} \gloss{Internet} grazie al quale ogni \gloss{utente} abilitato può inviare e ricevere dei messaggi. Questo servizio utilizza un computer o qualsiasi altro dispositivo elettronico connesso a \gloss{Internet}, attraverso un proprio \gloss{account} di posta registrato presso un \gloss{provider} del servizio.}

\elemento{Event-driven}
{\gloss{Paradigma di programmazione} in cui il flusso del programma è determinato da eventi come azioni degli utenti o messaggi da altri programmi. Questo è il paradigma dominante usato nelle interfacce utente grafiche, come applicazioni web Javascript, focalizzate su azioni di risposta a input di utenti.}


\elemento{Express} 
{\gloss{Framework} per \gloss{applicazioni web} scritto in \gloss{Node.js} minimo e flessibile, che fornisce un robusto set di funzionalità per la costruzione di singole e applicazioni web ibride multi-pagina.}


\paginaGlossario{F}


\elemento{Facade}
{\gloss{Design pattern} strutturale che permette di fornire un'interfaccia unica semplice per un sottosistema complesso.}


\elemento{Facebook}
{Servizio di \gloss{social network} lanciato nel febbraio 2004, posseduto e gestito dalla corporation Facebook, Inc.}

\elemento{File di configurazione}
{File che permette di impostare parametri necessari al funzionamento dell'applicazione nella loro versione predefinita; ad esempio la cofigurazione del \gloss{database} per il sistema di \gloss{autenticazione}, impostazioni delle \gloss{e-mail}, configurazione di \gloss{defult} del \gloss{server}, ecc.}

\elemento{File di descrizione}
{File scritto con \gloss{linguaggio} \gloss{DSL} dallo \gloss{sviluppatore}, serve per generare pagine di tipo \gloss{Collection-Index} e \gloss{Document-Show}.}

\elemento{File system}
{Meccanismo con il quale i file sono posizionati e organizzati su un dispositivo di archiviazione.}

\elemento{Filtro} 
{Componente che ha il compito di selezionare una fonte in ingresso secondo dei criteri, in modo da avere un risultato finale utile e più specifio per l'\gloss{utente}.}

\elemento{Firefox}
{\gloss{Browser} \gloss{open source} multipiattaforma. Secondo alcune statistiche è il secondo browser più popolare al mondo.}

\elemento{Form}
{\gloss{Interfaccia} di una applicazione che consente ad un \gloss{utente} di inserire e inviare ad un \gloss{server} dei dati liberamente inseriti dall'utente stesso.}

\elemento{Framework}
{Struttura di supporto su cui un \gloss{software} può essere progettato e realizzato. Alla base di un framework sono sempre presenti delle \gloss{librerie di codice} utilizzabili con uno o più \gloss{linguaggi di programmazione}; esse sono spesso corredate da una serie di strumenti di supporto allo sviluppo software, come ad esempio un \gloss{IDE} o un \gloss{debugger}, o altri strumenti ideati per aumentare la velocità di sviluppo del prodotto finito.
Lo scopo di un framework è quello di far risparmiare allo \gloss{sviluppatore} la riscrittura di \gloss{codice} già scritto precedentemente per fini simili. La necessità di questo strumento si è venuta a creare quando le \gloss{interfacce} utente sono diventate sempre più complesse ed è aumentata la complessità del software con funzionalità secondarie simili.}

\elemento{Funzione-software}
{Particolare costrutto sintattico, scritto in qualche \gloss{linguaggio di programmazione}, che permette di raggruppare, all'interno di un \gloss{programma}, una sequenza di istruzioni in un unico blocco. Il suo scopo è di compiere una determinata operazione, azione o elaborazione sui dati del programma stesso, in modo tale che a partire da determinati \gloss{input} restituisca determinati \gloss{output}.}


\paginaGlossario{G}


\elemento{Gantt} 
{Henry Laurence Gantt (1861-1919), l'inventore dei \gloss{diagrammi di Gantt}, che vengono usati nella gestione e nella pianificazione delle attività.}

\elemento{Git} 
{Sistema di controllo di \gloss{versione} \gloss{distribuito} gratuito e \gloss{open source}, designato alla gestione di progetti \gloss{software}.}

\elemento{GitHub} 
{\gloss{Servizio} di \gloss{web} \gloss{hosting} orientato allo sviluppo \gloss{software} e basato sul sistema di controllo di \gloss{versione} \gloss{Git}. GitHub offre servizi di \gloss{repository} online sia gratuiti che a pagamento.}

\elemento{Gmail} 
{\gloss{Servizio} gratuito di posta elettronica gestito e offerto da \gloss{Google Inc.}.}

\elemento{Google} 
{Motore di ricerca per \gloss{Internet} che oltre alla funzione di effettuare ricerche offre molti altri \gloss{servizi}, ad esempio gestione \gloss{e-mail}, gestione calendari, eccetera.} 

\elemento{Google Chrome} 
{\gloss{Browser} gratuito creato da \gloss{Google Inc.}.}

\elemento{Google Drive} 
{\gloss{Servizio web} offerto e gestito da \gloss{Google Inc.} che permette l'archiviazione, la sincronizzazione, la condivisione e la modifica collaborativa di documenti.}

\elemento{Google Inc.}
{Azienda statunitense che offre \gloss{servizi} online, principalmente nota per il motore di ricerca \gloss{Google} e per servizi come \gloss{Google Drive}.}

\elemento{Google Mail}
{Vedi \gloss{gmail}.}


\paginaGlossario{H}


\elemento{Hardware} 
{Componenti fisiche che compongono un computer.}

\elemento{Header}
{Blocco di informazioni aggiuntive posizionato all'inizio di un pezzo di codice per essere immagazzinato, trasmesso o descritto.}

\elemento{Heroku} 
{Piattaforma \gloss{cloud} atta a fornire \gloss{servizi} tramite il \gloss{deployment} di applicazioni scritte in \gloss{linguaggi} quali \gloss{Node.js}, \gloss{Ruby}, \gloss{Scala}, e molti altri.}

\elemento{Host} 
{Ogni terminale collegato ad una \gloss{rete}, o più in particolare ad \gloss{Internet}.}

\elemento{Hosting} 
{Tradotto letteralmente in "ospitare". \gloss{Servizio} di \gloss{rete} che consiste nell'allocare su un \gloss{server web} le \gloss{pagine web} di un \gloss{sito web}, rendendolo così accessibile dalla rete \gloss{Internet} e ai suoi \gloss{utenti}.}

\elemento{HTML}
{Linguaggio di markup solitamente usato per la formattazione di documenti ipertestuali disponibili nel \gloss{World Wide Web} sotto forma di \gloss{pagine web}.}

\elemento{HTML5}
{Quinta versione di HTML che introduce diverse novità, ad esempio tag di video audio, canvas, integrazione di grafica vettoriale scalabile e tag per formule matematiche.}


\paginaGlossario{I}


\elemento{IDE} 
{Acronimo di "Integrated Developement Enviroment", tradotto in "ambiente di sviluppo integrato". \gloss{Software} che, durante la \gloss{programmazione}, aiuta i \gloss{programmatori} nello sviluppo del \gloss{codice sorgente} di un \gloss{programma}.}

\elemento{IEC} 
{Acronimo di "International Electrotechnical Commission", tradotto in "Commissione Elettrotecnica Internazionale". \`{E} un'organizzazione internazionale per la definizione di standard in materia di elettricità, elettronica e tecnologie correlate. Molti dei suoi standard sono definiti in collaborazione con l'\gloss{ISO}.}

\elemento{Incapsulamento}
{Tecnica di nascondere il funzionamento interno di una parte di un programma. In questo modo si proteggono le altre parti del programma dai cambiamenti che si produrrebbero in esse nel caso che questo funzionamento fosse difettoso, oppure si decidesse di implementarlo in modo diverso.}

\elemento{Incapsulare}
{L'atto di ottenimento dell'\gloss{incapsulamento}.}

\elemento{Infrastruttura} 
{Insieme di risorse \gloss{hardware} e \gloss{software}.}

\elemento{Input}
{Tradotto letteralmente in "immettere". Sequenza di dati o informazioni, immessi per mezzo di una \gloss{periferica}, detta appunto di \gloss{input}, e successivamente elaborati.}

\elemento{Interfaccia} 
{Dispositivo fisico o virtuale che permette la comunicazione e l'interazione tra due entità. Ad esempio in un'\gloss{interfaccia} grafica di un \gloss{programma} possono venire inseriti dei valori e successivamente visualizzare un eventuale risultato o una risposta da parte del programma stesso.}

\elemento{Internet} 
{Contrazione della locuzione inglese "interconnected networks", ovvero "reti interconnesse".
\`{E} una \gloss{rete} mondiale di reti di computer ad accesso pubblico, attualmente rappresentante il principale mezzo di comunicazione di massa, che offre all'\gloss{utente} una vasta serie di contenuti potenzialmente informativi e \gloss{servizi}.}

\elemento{Internet Protocol}
{\gloss{Protocollo di comunicazione} di \gloss{rete} appartenente all'insieme di protocolli \gloss{Internet} TCP/IP su cui è basato il funzionamento della rete Internet.}

\elemento{Interprete} 
{\gloss{Programma} in grado di eseguire altri programmi a partire direttamente dal relativo \gloss{codice sorgente}. Un interprete ha lo scopo di eseguire un programma in un \gloss{linguaggio di alto livello}, senza la previa \gloss{compilazione} dello stesso (\gloss{codice oggetto}) cioè di eseguire le istruzioni nel \gloss{linguaggio} usato, traducendole di volta in volta in istruzioni in \gloss{linguaggio macchina}.}

\elemento{IP} 
{Acronimo di "Internet Protocol Address".L'IP è un etichetta numerica che identifica univocamente un dispositivo (\gloss{host}) collegato a una \gloss{rete} informatica che utilizza l' Internet Protocol come \gloss{protocollo di comunicazione}.}

\elemento{Ipertesto} 
{Insieme di documenti messi in relazione tra loro per mezzo di parole chiave.} 

\elemento{ISO} 
{Acronimo di "International Organization for Standardization", tradotto in "Organizzazione Internazionale per la Normazione". Come dice il nome, è un'organizzazione internazionale atta alla realizzazione di specifiche standard per la realizzazione di prodotti, \gloss{servizi} e pratiche corrette per aiutare ogni impresa a lavorare in modo efficace ed efficiente.}

\elemento{Istanza}
{Un particolare \gloss{oggetto} di una particolare \gloss{classe}.}


\paginaGlossario{J}


\elemento{Java} 
{\gloss{Linguaggio di programmazione} \gloss{orientato agli oggetti}, creato dalla \gloss{Sun Microsystems}.}

\elemento{JavaScript} 
{\gloss{Linguaggio di scripting} \gloss{orientato agli oggetti}, comunemente usato nella creazione di \gloss{siti web}. In applicazioni di tipo \gloss{client-server}, se il codice JavaScript è sul lato \gloss{client}, viene eseguito localmente, così da non sovraccaricare la parte \gloss{server}.}

\elemento{JSON} 
{Acronimo di "\gloss{JavaScript} Object Notation", tradotto in "notazione oggetto JavaScript". Formato standard aperto che usa testo leggibile dall'uomo per trasmettere dati di oggetti che consistono in coppie chiave-valore. Sono usati principalmente per trasmettere dati tra server e web application.}

\elemento{JVM}
{Acronimo di "\gloss{Java} Virtual Machine", tradotto in "\gloss{macchina virtuale} Java". \`{E} il componente della \gloss{piattaforma Java} che esegue i \gloss{programmi} tradotti in \gloss{bytecode} dopo una prima \gloss{compilazione}.}


\paginaGlossario{L}


\elemento{LaTeX} 
{\gloss{Linguaggio} usato per la scrittura di testi basato sul \gloss{programma} \gloss{TeX}.}

\elemento{Layout} 
{Identifica l'impaginazione e la struttura grafica di una \gloss{pagina web} o di un documento.}

\elemento{Libreria di codice} 
{Insieme di funzioni o strutture dati predisposte per essere collegate ad un \gloss{programma} \gloss{software} attraverso opportuno collegamento.}

\elemento{Linguaggio di alto livello} 
{\gloss{Linguaggio di programmazione} più astratto del \gloss{linguaggio macchina}, direttamente eseguibile da un computer, ma più vicino o familiare alla logica del nostro \gloss{linguaggio} naturale. I \gloss{programmi} ad alto livello possono essere ricondotti a programmi in linguaggio macchina in modo automatico, ovvero da un altro programma, detto \gloss{interprete}.}

\elemento{Linguaggio di markup}
{Insieme di regole che descrivono i meccanismi di rappresentazione (strutturali, semantici o presentazionali) di un testo che, utilizzando convenzioni standardizzate, sono utilizzabili su più supporti.}

\elemento{Linguaggio di programmazione} 
{\gloss{Linguaggio formale}, ben definito e composto da una \gloss{sintassi} e una \gloss{semantica}.}

\elemento{Linguaggio di scripting} 
{\gloss{Linguaggio interpretato}, destinato in genere a compiti di automazione del \gloss{sistema operativo}, oppure viene usato all'interno delle \gloss{pagine web} per gestire il comportamento delle pagine stesse in base all' \gloss{input} dell'\gloss{utente}.}

\elemento{Linguaggio formale} 
{Notazione o formalismo con \gloss{sintassi} e \gloss{semantica} definite in modo preciso (spesso matematico/formale) e, in molti casi, tali da consentire qualche forma di elaborazione automatica del \gloss{linguaggio} stesso.}

\elemento{Link} 
{Abbreviazione di "hyperlink". \`{E} un collegamento ipertestuale in grado di rinviare a un contenuto informativo presente in un dominio fisicamente o virtualmente separato. Il link è di solito associato ad una o più parole chiave, evidenziate visivamente da una diversa colorazione o sottolineatura.}

\elemento{Linux} 
{\gloss{Sistema operativo} della \gloss{Linux Foundation}.}

\elemento{Linux Foundation} 
{Associazione senza fini di lucro, specializzata nel campo dell'informatica \gloss{open source}.}

\elemento{Linguaggio interpretato} 
{Un \gloss{linguaggio} informatico è per definizione diverso dal \gloss{linguaggio macchina}. Bisogna quindi tradurlo per renderlo leggibile dal punto di vista del \gloss{processore}. Un \gloss{programma} scritto in un \gloss{linguaggio interpretato} ha bisogno di un programma ausiliario (l'\gloss{interprete}) per tradurre man mano le istruzioni del programma in linguaggio macchina.}

\elemento{Linguaggio macchina} 
{Definito come il \gloss{linguaggio} utilizzato dal \gloss{processore}, tramite sequenze di 0 e 1. Ogni sequenza ordinata di 0 e 1, raggruppata in gruppi di una certa dimensione, identifica una precisa istruzione per il processore.}

\elemento{Logger}
{Componente non intrusivo di registrazione dei dati di esecuzione per analisi dei risultati.}


\paginaGlossario{M}


\elemento{MaaP}
{"MongoDB as an admin Platform". \gloss{Framework} che genera \gloss{interfacce} \gloss{web} di amministrazione dei dati di \gloss{business} basati sulle tecnologie\gloss{Node.js} e \gloss{MongoDB}.}

\elemento{Macchina virtuale} 
{Implementazione \gloss{software} di un \gloss{ambiente} di elaborazione in cui un \gloss{sistema operativo} o un \gloss{programma} possono essere installati ed eseguiti.}

\elemento{MaaP's web}
{Insieme delle \gloss{pagine web} prodotte dal \gloss{framework} \gloss{MaaP}.}

\elemento{Mailing-list}
{Collezione di nomi e indirizzi usati da un'organizzazione o un individuo per inviare materiale a più destinatari.}

\elemento{Manutenibile}
{Caratteristica di un soggetto sottoposto a manutenzione; essa è espressa come una misura della capacità di un soggetto di rimanere funzionante o essere riportato rapidamente in condizioni operative, a fronte di guasti o di manutenzione, impiegando le procedure e le risorse prescritte. }

\elemento{MEAN}
{Acronimo di MongoDB, Express, AngularJS e Node.js. Insieme di framework Javascript che semplificano e accelerano lo sviluppo di applicazioni web, utilizzando le tecnologie descritte.}

\elemento{Metodo}
{Detto anche funzione-software membro, è un termine che viene usato principalmente nel contesto della \gloss{programmazione} \gloss{orientata agli oggetti} per indicare un sottoprogramma associato in modo esclusivo ad una \gloss{classe} e che rappresenta (in genere) un'operazione eseguibile sugli oggetti-software e \gloss{istanze} di quella classe.}

\elemento{Metrica}
{Insieme di regole per fissare le entità da misurare, gli attributi rilevanti, l'unità di misura, la procedura per assegnare e interpretare i valori.}

\elemento{Middleware}
{Insieme di programmi informatici che fungono da intermediari tra diverse applicazioni e componenti software. Sono spesso utilizzati come supporto per sistemi distribuiti complessi.}

\elemento{Milestone}
{Viene utilizzato nella pianificazione e gestione di progetti per indicare il raggiungimento di obiettivi definiti in fase di definizione del progetto stesso. Molto spesso le milestone sono rappresentate da eventi, come ad esempio scadenze per la consegna di documenti, e indicano importanti traguardi intermedi durante lo svolgimento del progetto.}

\elemento{Model}
{Componente del design pattern MVC. Contiene la descrizione del modello dei dati del software, le loro funzioni e la \gloss{logica di business}.}

\elemento{MongoDB}
{Sistema gestionale di \gloss{basi di dati}, non \gloss{relazionale}, \gloss{orientato ai documenti} e di tipo \gloss{NoSQL}. Il \gloss{linguaggio di programmazione} utilizzato per la gestione dei dati è \gloss{JavaScript}, in particolare la sua notazione \gloss{JSON}.}

\elemento{MVC}
{Acronimo di "Model View Controller". \`{E} un pattern architetturale molto diffuso nello sviluppo di sistemi software, in particolare nell'ambito della \gloss{programmazione orientata agli oggetti}, in grado di separare la logica di presentazione dei dati dalla \gloss{logica di business}}

\elemento{MVVM}
{Acronimo di "Model View ViewModel". \`{E} un pattern architetturale e una specifica implementazione focalizzata allo sviluppo di piattaforme di interfacce utente che supporta la programmazione event-driven. La view utlizza il \gloss{Two Way Data Binding}.}
