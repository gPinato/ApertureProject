%tabella requisiti codice requisito, descrizione, fonte, UC di riferimento

\subsection{Requisiti funzionali MaaP}
\begin{longtable}{|c|p{6cm}|c|c|}
\caption{Requisiti funzionali MaaP}
\label{tab:Requisiti MaaP} \\
\toprule
\multicolumn{1}{|c}{\textbf{Requisito}} & \multicolumn{1}{|p{6cm}}{\textbf{Descrizione}}   & \multicolumn{1}{|c}{\textbf{Fonte}} & \multicolumn{1}{|c|}{\textbf{Caso d'uso}}\\
\midrule
\endfirsthead
\multicolumn{2}{l}{\footnotesize\itshape\tablename~\thetable: continua dalla pagina precedente} \\
\toprule
\multicolumn{1}{|c}{\textbf{Requisito}} & \multicolumn{1}{|p{6cm}}{\textbf{Descrizione}}   & \multicolumn{1}{|c}{\textbf{Fonte}} & \multicolumn{1}{|c|}{\textbf{Caso d'uso}}\\
\midrule
\endhead
\midrule
\multicolumn{2}{r}{\footnotesize\itshape\tablename~\thetable: continua nella prossima pagina} \\
\endfoot
\bottomrule
\multicolumn{2}{r}{\footnotesize\itshape\tablename~\thetable: si conclude dalla pagina precedente} \\
\endlastfoot

% Requisiti utente sviluppatore


\midrule
ROF1
& Il sistema MaaP deve essere in grado di generare lo scheletro del progetto
& Capitolato
& UC1\\
& & & UC1.1
\\

\midrule
ROF1.1
& Il sistema MaaP deve installare le librerie necessarie al funzionamento del progetto
& Capitolato
&
\\
\midrule
ROF1.2
& Il sistema MaaP deve generare i file di configurazione necessari al funzionamento del progetto
& Capitolato
&
\\
\midrule
ROF1.3
& Il sistema MaaP deve generare le directory necessarie al funzionamento del progetto
& Capitolato
&
\\
\midrule
ROF1.4
& Il sistema MaaP deve generare il sistema di autenticazione per le pagine web
& Capitolato
&
\\
\midrule
ROF1.4.1
& Il sistema di autenticazione per le pagine web deve essere generato insieme ad un profilo amministratore di \gloss{default}
& Verbale\_2013\_12\_05
&
\\

\midrule
ROF1.5
& Il sistema MaaP deve essere in grado eliminare un progetto esistente
& Interna
& UC1\\
& & & UC1.6
\\

\midrule
ROF1.6
& Il sistema MaaP deve essere in grado di clonare un progetto esistente
& Interna
& UC1\\
& & & UC1.7
\\

\midrule
RFF2
& Il sistema MaaP deve permettere all'utente sviluppatore di utilizzare un editor interno specializzato per la scrittura/modifica dei file di descrizione
& Capitolato
& UC1\\
& & & UC1.4
\\

\midrule
RFF2.1
& Il sistema MaaP deve permettere all'utente sviluppatore di utilizzare un editor interno specializzato per la scrittura di un nuovo file di descrizione
& Interna
& UC1.4 \\
& & & UC1.4.1
\\

\midrule
RFF2.1.1
& Il sistema MaaP deve permettere all'utente sviluppatore di utilizzare un editor interno specializzato per scrivere il codice del file di descrizione che intende creare
& Interna
& UC1.4.1 \\
& & & UC1.4.1.1
\\

\midrule
RFF2.1.2
& Il sistema MaaP deve permettere all'utente sviluppatore di utilizzare un editor interno specializzato per salvare il codice scritto in modo permanente
& Interna
& UC1.4.1 \\
& & & UC1.4.1.2
\\

\midrule
RFF2.2
& Il sistema MaaP deve permettere all'utente sviluppatore di utilizzare un editor interno specializzato per la modifica di un file di descrizione esistente
& Interna
& UC1.4 \\
& & & UC1.4.2
\\

\midrule
RFF2.2.1
& Il sistema MaaP deve permettere all'utente sviluppatore di utilizzare un editor interno specializzato per modificare il codice di un file di descrizione esistente
& Interna
& UC1.4.2 \\
& & & UC1.4.2.1
\\

\midrule
RFF2.2.2
& Il sistema MaaP deve permettere all'utente sviluppatore di utilizzare un editor interno specializzato per salvare il codice modificato in modo permanente
& Interna
& UC1.4.2 \\
& & & UC1.4.2.2
\\

\midrule
RFF2.2.3
& Il sistema MaaP deve permettere all'utente sviluppatore di utilizzare un editor interno specializzato per annullare le modifiche al codice del file di descrizione modificato
& Interna
& UC1.4.2 \\
& & & UC1.4.2.3
\\

\midrule
ROF3
& Il sistema MaaP deve permette all'utente sviluppatore di inserire un file di descrizione
& Interna
& UC1\\
& & & UC1.5\\

\midrule
ROF4
& Il sistema deve permette all'utente sviluppatore di utilizzare un file di descrizione
& Capitolato
& UC1\\
& & & UC1.2
\\
\midrule
ROF4.1
& Il sistema MaaP deve permettere all'utente sviluppatore di creare la visualizzazione della Collection
& Capitolato
& UC1.2 \\
& & & UC1.2.1
\\
\midrule
ROF4.1.1
& Il sistema MaaP deve permettere all'utente sviluppatore di creare la visualizzazione del menù per le Collection
& Capitolato
& UC1.2.1\\
& & & UC1.2.1.1
\\
\midrule
ROF4.1.1.1
& Il sistema MaaP deve permettere all'utente sviluppatore di definire il nome della voce relativa alla Collection
& Capitolato
& UC1.2.1.1\\
& & & UC1.2.1.1.1
\\
\midrule
ROF4.1.1.2
& Il sistema MaaP deve permettere all'utente sviluppatore di definire la posizione di una voce all'interno del menù
& Capitolato
& UC1.2.1.1\\
& & & 1.2.1.1.2
\\
\midrule
ROF4.1.2
& Il sistema MaaP deve permettere all'utente sviluppatore di creare la visualizzazione della pagina Collection-Index
& Capitolato
& UC1.2.1\\
& & & UC1.2.1.2
\\
\midrule
ROF4.1.2.1
& Il sistema MaaP deve permettere all'utente sviluppatore di aggiungere delle chiavi da visualizzare nella pagina Collection-Index
& Capitolato
& UC1.2.1.2\\
& & & UC1.2.1.2.1
\\
\midrule
ROF4.1.2.1.1
& Il sistema MaaP deve permettere all'utente sviluppatore di aggiungere definire un'etichetta per la chiave da visualizzare
& Capitolato
& UC1.2.1.2.1\\
& & & UC1.2.1.2.1.1
\\
\midrule
ROF4.1.2.1.2
& Il sistema MaaP deve permettere all'utente sviluppatore di definire un campo associato alla chiave da visualizzare
& Capitolato
& UC1.2.1.2.1\\
& & & UC1.2.1.2.1.2
\\
\midrule
ROF4.1.2.1.3
& Il sistema MaaP deve permettere all'utente sviluppatore di definire un campo associato alla chiave da visualizzare, proveniente da un documento esterno
& Capitolato
& UC1.2.1.2.1\\
& & & UC1.2.1.2.1.3
\\
\midrule
ROF4.1.2.1.4
& Il sistema MaaP deve permettere all'utente sviluppatore di definire un campo associato alla chiave da visualizzare, proveniente dal risultato di una query
& Capitolato
& UC1.2.1.2.1\\
& & & UC1.2.1.2.1.4
\\
\midrule
ROF4.1.2.1.5
& Il sistema MaaP deve permettere all'utente sviluppatore di definire un campo associato alla chiave da visualizzare come trasformazione
& Capitolato
& UC1.2.1.2.1\\
& & & UC1.2.1.2.1.5
\\

\midrule
ROF4.1.2.2
& Il sistema MaaP deve permettere all'utente sviluppatore di definire un ordinamento rispetto a una chiave
& Capitolato
& UC1.2.1.2\\
& & & UC1.2.1.2.2
\\
\midrule
ROF4.1.2.3
& Il sistema deve permettere all'utente sviluppatore di  definire un numero massimo di Document da visualizzare per la pagina Collection-Index
& Capitolato
& UC1.2.1.2\\
& & & UC1.2.1.2.3
\\
\midrule
RFF4.1.2.4
& Il sistema MaaP deve permettere all'utente sviluppatore di aggiungere dei pulsanti all'interno della pagina Collection-Index
& Capitolato
& UC1.2.1.2\\
& & & UC1.2.1.2.4
\\
\midrule
ROF4.1.3
& Il sistema MaaP deve permettere all'utente sviluppatore di creare la visualizzazione per la pagine Document-Show
& Capitolato
& UC1.2.1\\
& & & UC1.2.1.3
\\
\midrule
ROF4.1.3.1
& Il sistema MaaP deve permettere all'utente sviluppatore di aggiungere delle chiavi da visualizzare nella pagina Document-Show
& Capitolato
& UC1.2.1.3\\
& & & UC1.2.1.3.1
\\

\midrule
ROF4.1.3.1.1
& Il sistema MaaP deve permettere all'utente sviluppatore di aggiungere definire un'etichetta per la chiave da visualizzare
& Capitolato
& UC1.2.1.3.1\\
& & & UC1.2.1.3.1.1
\\
\midrule
ROF4.1.3.1.2
& Il sistema MaaP deve permettere all'utente sviluppatore di definire un campo associato alla chiave da visualizzare
& Capitolato
& UC1.2.1.3.1\\
& & & UC1.2.1.3.1.2
\\
\midrule
ROF4.1.3.1.3
& Il sistema MaaP deve permettere all'utente sviluppatore di definire un campo associato alla chiave da visualizzare, proveniente da un documento esterno
& Capitolato
& UC1.2.1.3.1\\
& & & UC1.2.1.3.1.3
\\
\midrule
ROF4.1.3.1.4
& Il sistema MaaP deve permettere all'utente sviluppatore di definire un campo associato alla chiave da visualizzare, proveniente dal risultato di una query
& Capitolato
& UC1.2.1.3.1\\
& & & UC1.2.1.3.1.4
\\
\midrule
ROF4.1.3.1.5
& Il sistema MaaP deve permettere all'utente sviluppatore di definire un campo associato alla chiave da visualizzare come trasformazione
& Capitolato
& UC1.2.1.3.1\\
& & & UC1.2.1.3.1.5
\\

\midrule
RFF4.1.3.2
& Il sistema MaaP deve permettere all'utente sviluppatore di aggiungere un pulsante all'interno della  pagina Document-Show
& Capitolato
& UC1.2.1.3\\
& & & UC1.2.1.3.2
\\
\midrule
ROF4.2
& Il sistema MaaP deve permettere all'utente sviluppatore di modificare la visualizzazione della Collection
& Capitolato
& UC1.2\\
& & & UC1.2.2
\\


\midrule
ROF4.2.1
& Il sistema MaaP deve permettere all'utente sviluppatore di impostare la visualizzazione del menù delle Collection
& Capitolato
& UC1.2.2\\
& & & UC1.2.2.1
\\
\midrule
ROF4.2.1.1
& Il sistema MaaP deve permettere all'utente sviluppatore di modificare il nome della voce relativa alla Collection
& Capitolato
& UC1.2.2.1\\
& & & UC1.2.2.1.1
\\
\midrule
ROF4.2.1.2
& Il sistema MaaP deve permettere all'utente sviluppatore di modificare la posizione di una voce all'interno del menù
& Capitolato
& UC1.2.2.1\\
& & & UC1.2.2.1.2
\\
\midrule
ROF4.2.2
& Il sistema MaaP deve permettere all'utente sviluppatore di impostare la visualizzazione della pagina Collection-Index
& Capitolato
& UC1.2.2\\
& & & UC1.2.2.2
\\

\midrule
ROF4.2.2.1
& Il sistema MaaP deve permettere all'utente sviluppatore di aggiungere delle chiavi da visualizzare nella pagina Collection-Index
& Capitolato
& UC1.2.1.2\\
& & & UC1.2.1.2.1
\\
\midrule
ROF4.2.2.2
& Il sistema MaaP deve permettere all'utente sviluppatore di eliminare delle chiavi da visualizzare nella pagina Collection-Index
& Interna
& UC1.2.2.2\\
& & & UC1.2.2.2.1
\\
\midrule
ROF4.2.2.3
& Il sistema MaaP deve permettere all'utente sviluppatore di definire un ordinamento, alfabetico crescente o decrescente, rispetto a una chiave
& Capitolato
& UC1.2.1.2\\
& & & UC1.2.1.2.2
\\
\midrule
ROF4.2.2.4
& Il sistema MaaP deve permettere all'utente sviluppatore di eliminare un ordinamento rispetto a una chiave
& Interna
& UC1.2.2.2\\
& & & UC1.2.2.2.2
\\
\midrule
ROF4.2.2.5
& Il sistema deve permettere all'utente sviluppatore di  definire un numero massimo di Document da visualizzare per la pagina Collection-Index
& Capitolato
& UC1.2.1.2\\
& & & UC1.2.1.2.3
\\
\midrule
ROF4.2.2.6
& Il sistema deve permettere all'utente sviluppatore di  eliminare il numero massimo di Document da visualizzare per la pagina Collection-Index
& Capitolato
& UC1.2.2.2\\
& & & UC1.2.2.2.4
\\
\midrule
RFF4.2.2.7
& Il sistema MaaP deve permettere all'utente sviluppatore di aggiungere dei pulsanti all'interno della pagina Collection-Index, specificando il nome del pulsante e l'azione che deve eseguire
& Capitolato
& UC1.2.1.2\\
& & & UC1.2.1.2.4
\\
\midrule
RFF4.2.2.8
& Il sistema MaaP deve permettere all'utente sviluppatore di eliminare dei pulsanti all'interno della pagina Collection-Index
& Capitolato
& UC1.2.2.2\\
& & & UC1.2.2.2.5
\\
\midrule
ROF4.2.3
& Il sistema MaaP deve permettere all'utente sviluppatore di impostare la visualizzazione per la pagine Document-Show
& Capitolato
& UC1.2.2\\
& & & UC1.2.2.3
\\
\midrule
ROF4.2.3.1
& Il sistema MaaP deve permettere all'utente sviluppatore di aggiungere delle chiavi da visualizzare nella pagina Document-Show
& Capitolato
& UC1.2.1.3\\
& & & UC1.2.1.3.1
\\
\midrule
ROF4.2.3.2
& Il sistema MaaP deve permettere all'utente sviluppatore di eliminare delle chiavi da visualizzare nella pagina Document-Show
& Capitolato
& UC1.2.2.3\\
& & & UC1.2.2.3.1
\\

\midrule
RFF4.2.3.3
& Il sistema MaaP deve permettere all'utente sviluppatore di aggiungere dei pulsanti all'interno della  pagina Document-Show, specificando il nome del pulsante e l'azione che deve eseguire
& Capitolato
& UC1.2.1.3\\
& & & UC1.2.1.3.2
\\
\midrule
RFF4.2.3.4
& Il sistema MaaP deve permettere all'utente sviluppatore di eliminare dei pulsanti all'interno della  pagina Document-Show
& Capitolato
& UC1.2.2.3\\
& & & UC1.2.2.3.2
\\

\midrule
ROF4.3
& Il sistema MaaP deve permettere all'utente sviluppatore di definire una query personalizzata
& Capitolato
& UC1.2.3
\\

\midrule
ROF4.4
& Il sistema MaaP deve permettere all'utente sviluppatore di eliminare una query personalizzata
& Capitolato
& UC1.2.4
\\

\midrule
ROF5
& Il sistema deve permettere all'utente sviluppatore la modifica dei file di configurazione
& Interna
& UC1.3
\\

\midrule
RDF5.1
& Il sistema deve permettere all'utente sviluppatore di abilitare la funzionalità di registrazione per l'utente finale nelle pagine web. Nel caso la registrazione sia abilitata l'utente finale può registrarsi al sistema, altrimenti no.
& Verbale\_2013\_12\_05
& UC1.3\\
& & & UC1.3.1
\\

\midrule
RFF5.2
& Il sistema MaaP deve permettere all'utente sviluppatore di abilitare la funzionalità per la creazione di nuovi Document all'interno della pagina Collection-Index
& Capitolato
& UC1.3\\
& &  & UC1.3.2
\\

\midrule
RDF5.3
& Il sistema MaaP deve permettere all'utente sviluppatore di modificare i template per le pagine web
& Interna
& UC1.3\\
& & & UC1.3.3
\\

\midrule
ROF5.4
& Il sistema deve permettere all'utente sviluppatore di specificare nome, indirizzo e password, relativi al database di analisi con il quale interagire
& Interna
& UC1.3\\
& & & UC1.3.4
\\

\midrule
ROF5.5
& Il sistema deve permettere all'utente sviluppatore di abilitare la funzionalità per la creazione di nuovi indici all'interno della pagina Collection-Index
& Capitolato
& UC1.3\\
& & Verbale\_2013\_12\_05 & UC1.3.5
\\


%FINE TABELLA REQUISITI MAAP, NON CANCELLARE
\end{longtable}

\newpage
\subsection{Requisiti funzionali MaaP's Web}
\begin{longtable}{|c|p{6cm}|c|c|}
\caption{Requisiti funzionali MaaP's Web}
\label{tab:Requisiti MaaP's Web} \\
\toprule
\multicolumn{1}{|c}{\textbf{Requisito}} & \multicolumn{1}{|p{6cm}}{\textbf{Descrizione}}   & \multicolumn{1}{|c}{\textbf{Fonte}} & \multicolumn{1}{|c|}{\textbf{Caso d'uso}}\\
\midrule
\endfirsthead
\multicolumn{2}{l}{\footnotesize\itshape\tablename~\thetable: continua dalla pagina precedente} \\
\toprule
\multicolumn{1}{|c}{\textbf{Requisito}} & \multicolumn{1}{|p{6cm}}{\textbf{Descrizione}}   & \multicolumn{1}{|c}{\textbf{Fonte}} & \multicolumn{1}{|c|}{\textbf{Caso d'uso}}\\
\midrule
\endhead
\midrule
\multicolumn{2}{r}{\footnotesize\itshape\tablename~\thetable: continua nella prossima pagina} \\
\endfoot
\bottomrule
\multicolumn{2}{r}{\footnotesize\itshape\tablename~\thetable: si conclude dalla pagina precedente} \\
\endlastfoot

%Requisiti utente business
\midrule
ROF6
& L'utente business, al primo accesso, deve poter usare il profilo amministratore di default
& Verbale\_2013\_12\_05
&
\\

\midrule
ROF7
& L'utente business deve potersi autenticare inserendo dei dati personali
& Capitolato
& UC2\\
& & & UC2.2
\\

\midrule
ROF7.1
& L'utente business deve inserire l'email per l'autenticazione
& Capitolato
& UC2.2\\
& & & UC2.2.1
\\

\midrule
ROF7.2
& L'utente business deve inserire la password per l'autenticazione
& Capitolato
& UC2.2\\
& & & UC2.2.2
\\

\midrule
ROF7.2.1
& La password per l'autenticazione deve essere alfanumerica e contenere almeno otto caratteri
& Interna
&
\\

\midrule
RDF8
& L'utente business deve potersi registrare inserendo dei dati personali
& Verbale\_2013\_12\_05
& UC2\\
& & & UC2.1
\\

\midrule
RDF8.1
& L'utente business, per registrarsi, deve inserire una email non presente nel sistema
& Capitolato
& UC2.1\\
& & & UC2.1.1
\\

\midrule
RDF8.2
& L'utente business, per registrarsi, deve inserire una password
& Capitolato
& UC2.1\\
& & & UC2.1.2
\\

\midrule
RDF8.2.1
& La password per la registrazione deve essere alfanumerica e contenere almeno otto caratteri
& Interna
&
\\

\midrule
ROF9
& L'utente business deve poter recuperare la password
& Capitolato
& UC2\\
& & & UC2.3
\\

\midrule
ROF10
& L'utente business autenticato deve poter aprire una Collection e visualizzare la sua pagina Collection-Index
& Capitolato
& UC2\\
& & & UC2.4
\\

\midrule
ROF10.1
& L'utente business autenticato deve poter visualizzare una pagina Document-Show
& Capitolato
& UC2.4\\
& & & UC2.4.1
\\

\midrule
ROF10.1.1
& L'utente business autenticato deve poter visualizzare il Document selezionato
& Capitolato
& UC2.4.1\\
& & & UC2.4.1.1
\\

\midrule
ROF10.1.2
& L'utente business autenticato amministratore deve poter eliminare il Document che sta visualizzando
& Interna
& UC2.4.1\\
& & & UC2.4.1.2
\\

\midrule
ROF10.1.3
& L'utente business autenticato amministratore deve poter modificare il Document che sta visualizzando
& Interna
& UC2.4.1\\
& & & UC2.4.1.3
\\

\midrule
RDF10.2
& L'utente business autenticato deve poter modificare la visualizzazione dei Document
& Interna
& UC2.4\\
& & & UC2.4.2
\\

\midrule
RDF10.2.1
& L'utente business autenticato deve poter selezionare dei criteri per la visualizzazione
& Interna
& UC2.4.2\\
& & & UC2.4.2.1
\\

\midrule
RDF10.2.1.1
& L'utente business autenticato deve poter effettuare un ordinamento rispetto a una chiave
& Interna
& UC2.4.2.1\\
& & & UC2.4.2.1.1
\\

\midrule
RDF10.2.1.2
& L'utente business deve poter selezionare un numero massimo di Document da visualizzare per pagina
& Interna
& UC2.4.2.1\\
& & & UC2.4.2.1.2
\\


\midrule
RDF10.2.2
& L'utente business autenticato deve poter applicare un filtro alla visualizzazione dei Document
& Interna
& UC2.4.2\\
& & Verbale\_2013\_12\_05 & UC2.4.2.2
\\

\midrule
RDF10.2.3
& L'utente business autenticato deve poter annullare il filtro
& Interna
& UC2.4.2\\
& & Verbale\_2013\_12\_05 & UC2.4.2.3
\\

\midrule
ROF10.2.4
& L'utente business autenticato deve poter disconnettersi
& Interna
& UC2.4\\
& & & UC2.5
\\

\midrule
ROF10.2.5
& L'utente business autenticato deve poter navigare tra la Collection
& Capitolato
& UC2.4\\
& & & UC2.4.3\\

\midrule
ROF10.3
& L'utente business autenticato deve poter gestire il proprio profilo
& Interna
& UC2\\
& & & UC2.6
\\

\midrule
ROF10.3.1
& L'utente business autenticato deve poter gestire i propri dati
& Capitolato
& UC2.6\\
& & & UC2.6.3
\\

\midrule
ROF10.3.1.1
& L'utente business autenticato deve poter modificare i propri dati utente
& Capitolato
& UC2.6.3\\
& & & UC2.6.3.1
\\

\midrule
ROF10.3.1.2
& L'utente business autenticato deve poter  salvare le modifiche apportate
& Interna
& UC2.6.3\\
& & & UC2.6.3.2
\\

\midrule
ROF10.3.1.3
& L'utente business autenticato deve poter annullare le modifiche apportate
& Interna
& UC2.6.3\\
& & & UC2.6.3.5
\\

\midrule
ROF10.3.1.4
& L'utente business autenticato amministratore deve poter modificare i dati degli utenti business
& Interna
& UC2.6.3\\
& & & UC2.6.3.3
\\

\midrule
ROF10.3.1.5
& L'utente business autenticato amministratore deve poter modificare i permessi degli utenti business
& Capitolato
& UC2.6.3\\
& & & UC2.6.3.4
\\

\midrule
ROF10.3.2
& L'utente business autenticato amministratore deve poter creare un nuovo utente business
& Capitolato
& UC2.6\\
& & & UC2.6.1
\\

\midrule
ROF10.3.3
& L'utente business autenticato amministratore deve poter eliminare un utente business
& Capitolato
& UC2.6\\
& & & UC2.6.2
\\

\midrule
ROF10.4
& L'utente business autenticato amministratore deve poter cancellare un Document
& Interna
& UC2.4\\
& & & UC2.4.4
\\

\midrule
ROF10.5
& L'utente business autenticato amministratore deve poter modificare un Document
& Capitolato
& UC2.4\\
& & & UC2.4.5
\\

\midrule
ROF10.5.1
& L'utente business autenticato amministratore deve poter modificare i valori associati alla chiavi
& Interna
& UC2.4.5\\
& & & UC2.4.5.1
\\

\midrule
ROF10.5.2
& L'utente business autenticato amministratore deve poter salvare le modifiche apportate al Document
& Interna
& UC2.4.5\\
& & & UC2.4.5.2
\\

\midrule
ROF10.5.3
& L'utente business autenticato amministratore deve poter annullare le modifiche apportate al Document
& Interna
& UC2.4.5\\
& & & UC2.4.5.3
\\

\midrule
ROF10.6
& L'utente business autenticato amministratore deve poter visualizzare le query più utilizzate dal sistema MaaP
& Capitolato
& UC2.4\\
& & & UC2.4.6
\\

\midrule
ROF10.7
& L'utente business autenticato amministratore deve poter gestire la creazione e l'eliminazione degli indici
& Capitolato
& UC2.4\\
& & & UC2.4.7
\\

\midrule
ROF10.7.1
& L'utente business autenticato amministratore deve poter gestire la creazione degli indici
& Capitolato
& UC2.4.7.1\\

\midrule
ROF10.7.1.1
& L'utente business autenticato amministratore deve poter selezionare una query
& Interna
& UC2.4.7.1\\
& & & UC2.4.7.1.1
\\

\midrule
ROF10.7.1.2
& L'utente business autenticato amministratore deve poter creare l'indice
& Capitolato
& UC2.4.7.1\\
& & & UC2.4.7.1.2
\\


\midrule
ROF10.7.2
& L'utente business autenticato amministratore deve poter eliminare degli indici
& Interna
& UC2.4.7.2\\

\midrule
ROF10.7.2.1
& L'utente business autenticato amministratore deve poter selezionare un indice presente nel sistema
& Interna
& UC2.4.7.2\\
& & & UC2.4.7.2.1
\\

\midrule
ROF10.7.2.2
& L'utente business autenticato amministratore deve poter eliminare l'indice selezionato
& Interna
& UC2.4.7.2\\
& & & UC2.4.7.2.2
\\

\midrule
ROF10.7.3
& L'utente business autenticato amministratore deve poter visualizzare gli indici presenti nel sistema
& Interna
& UC2.4.7\\
& & & UC2.4.7.3
\\


%FINE TABELLA REQUISITI MAAPSWEB, NON CANCELLARE
\end{longtable}

\newpage
\subsection{Requisiti funzionali MaaS}
\begin{longtable}{|c|p{6cm}|c|c|}
\caption{Requisiti funzionali MaaS}
\label{tab:Requisiti MaaS} \\
\toprule
\multicolumn{1}{|c}{\textbf{Requisito}} & \multicolumn{1}{|p{6cm}}{\textbf{Descrizione}}   & \multicolumn{1}{|c}{\textbf{Fonte}} & \multicolumn{1}{|c|}{\textbf{Caso d'uso}}\\
\midrule
\endfirsthead
\multicolumn{2}{l}{\footnotesize\itshape\tablename~\thetable: continua dalla pagina precedente} \\
\toprule
\multicolumn{1}{|c}{\textbf{Requisito}} & \multicolumn{1}{|p{6cm}}{\textbf{Descrizione}}   & \multicolumn{1}{|c}{\textbf{Fonte}} & \multicolumn{1}{|c|}{\textbf{Caso d'uso}}\\
\midrule
\endhead
\midrule
\multicolumn{2}{r}{\footnotesize\itshape\tablename~\thetable: continua nella prossima pagina} \\
\endfoot
\bottomrule
\multicolumn{2}{r}{\footnotesize\itshape\tablename~\thetable: si conclude dalla pagina precedente} \\
\endlastfoot

%REQUISITI MAAS
\midrule
RFF11.1
& Il sistema MaaS deve permettere all'utente di autenticarsi al sistema
& Interna
& UC3\\
& & & UC3.6\\

\midrule
RFF11.1.1
& Il sistema MaaS deve permettere all'utente di inserire il nome utente
& Interna
& UC3.6\\
& & & UC3.6.1\\

\midrule
RFF11.1.2
& Il sistema MaaS deve permettere all'utente di inserire la password
& Interna
& UC3.6\\
& & & UC3.6.2\\

\midrule
RFF11.2
& Il sistema MaaS deve permettere all'utente di registrarsi al sistema
& Interna
& UC3\\
& & & UC3.7\\

\midrule
RFF11.2.1
& Il sistema MaaS deve permettere all'utente di inserire la propria email
& Interna
& UC3.7\\
& & & UC3.7.1\\

\midrule
RFF11.3
& Il sistema MaaS deve permettere all'utente di recuperare la password
& Interna
& UC3\\
& & & UC3.8\\

\midrule
RFF11.4
& Il sistema MaaS deve permettere all'utente di visualizzare le pagine web create
& Capitolato
& UC3\\
& & & UC3.5\\

\midrule
RFF11.5
& Il sistema MaaS deve permettere all'utente autenticato di creare lo scheletro del progetto
& Interna
& UC3\\
& & & UC3.1\\

\midrule
RFF11.5.1
& Il sistema MaaS deve permettere all'utente autenticato di inserire il nome del progetto
& Interna
& UC3.1\\
& & & UC3.1.1\\

\midrule
RFF11.6
& Il sistema MaaS deve permettere all'utente autenticato di gestire le pagine web
& Capitolato
& UC3\\
& & & UC3.2\\

\midrule
RFF11.6.1
& Il sistema MaaS deve permettere all'utente autenticato di creare un file di descrizione
& Capitolato
& UC3.2\\
& & & UC3.2.1\\

\midrule
RFF11.6.1.1
& Il sistema MaaS deve permettere all'utente autenticato la scrittura di un file di descrizione tramite editor di testo
& Capitolato
& UC3.2.1\\
& & & UC3.2.1.1\\

\midrule
RFF11.6.1.2
& Il sistema MaaS deve permettere all'utente autenticato di salvare il file di descrizione
& Interna
& UC3.2.1\\
& & & UC3.2.1.2\\


\midrule
RFF11.6.2
& Il sistema MaaS deve permettere all'utente autenticato di eseguire l'upload di un file di descrizione creato precedentemente con il sistema MaaP
& Capitolato
& UC3.2\\
& & & UC3.2.2\\

\midrule
RFF11.6.2.1
& Il sistema MaaS deve permettere all'utente autenticato di navigare all'interno del \gloss{file system}
& Interna
& UC3.2.2\\
& & & UC3.2.2.1\\

\midrule
RFF11.6.2.2
& Il sistema MaaS deve permettere all'utente autenticato di selezionare un file di descrizione
& Interna
& UC3.2.2\\
& & & UC3.2.2.2\\

\midrule
RFF11.6.2.3
& Il sistema MaaS deve permettere all'utente autenticato di confermare l'upload del file selezionato
& Interna
& UC3.2.2\\
& & & UC3.2.2.3\\


\midrule
RFF11.6.3
& Il sistema MaaS deve permettere all'utente autenticato di modificare un file di descrizione esistente
& Interna
& UC3.2\\
& & & UC3.2.3\\

\midrule
RFF11.6.3.1
& Il sistema MaaS deve permettere all'utente autenticato di modificare il codice del file di descrizione selezionato
& Interna
& UC3.2.3\\
& & & UC3.2.3.1\\

\midrule
RFF11.6.3.2
& Il sistema MaaS deve permettere all'utente autenticato di salvare le modifiche apportate al file di descrizione
& Interna
& UC3.2.3\\
& & & UC3.2.3.2\\

\midrule
RFF11.6.3.3
& Il sistema MaaS deve permettere all'utente autenticato di annullare le modifiche apportate al file di descrizione
& Interna
& UC3.2.3\\
& & & UC3.2.3.3\\

\midrule
RFF11.6.4
& Il sistema MaaS deve permettere all'utente autenticato di modificare il file di configurazione
& Capitolato
& UC3.2\\
& & & UC3.2.4\\

\midrule
RFF11.6.4.1
& Il sistema MaaS deve permettere all'utente autenticato di modificare il codice del file di configurazione selezionato
& Interna
& UC3.2.4\\
& & & UC3.2.4.1\\

\midrule
RFF11.6.4.2
& Il sistema MaaS deve permettere all'utente autenticato di salvare le modifiche apportate al file di configurazione
& Interna
& UC3.2.4\\
& & & UC3.2.4.2\\

\midrule
RFF11.6.4.3
& Il sistema MaaS deve permettere all'utente autenticato di annullare le modifiche apportate al file di configurazione
& Interna
& UC3.2.4\\
& & & UC3.2.4.3\\

\midrule
RFF11.6.5
& Il sistema MaaS deve permettere all'utente autenticato di visualizzare tutti i file di descrizione presenti
& Capitolato
& UC3.2\\
& & & UC3.2.5\\

\midrule
RFF11.7
& Il sistema MaaS deve permettere all'utente autenticato di gestire il proprio profilo utente
& Interna
& UC3\\
& & & UC3.3\\

\midrule
RFF11.7.1
& Il sistema MaaS deve permettere all'utente autenticato di generare una nuova password
& Interna
& UC3.3\\
& & & UC3.3.1\\

\midrule
RFF11.8
& Il sistema MaaS deve permettere all'utente autenticato di disconnettersi dal sistema
& Interna
& UC3\\
& & & UC3.4\\

\midrule
RFF11.9
& Il sistema MaaS deve permettere all'utente autenticato di visualizzare le pagine web create
& Capitolato
& UC3\\
& & & UC3.5\\

\midrule
ROF12
& Definizione di un linguaggio astratto \gloss{DSL} per la definizione delle pagine che verranno generate
& Capitolato
&
\\

\midrule
ROF12.1
& Il linguaggio definito deve essere testuale
& Capitolato
&
\\

%FINE TABELLA REQUISITI MAAS, NON CANCELLARE
\end{longtable}
\newpage
\subsection{Requisiti di qualità}
\begin{longtable}{|c|p{6cm}|c|c|}
\caption{Requisiti di qualità}
\label{tab:Requisiti di qualita} \\
\toprule
\multicolumn{1}{|c}{\textbf{Requisito}} & \multicolumn{1}{|p{6cm}}{\textbf{Descrizione}}   & \multicolumn{1}{|c}{\textbf{Fonte}} & \multicolumn{1}{|c|}{\textbf{Caso d'uso}}\\
\midrule
\endfirsthead
\multicolumn{2}{l}{\footnotesize\itshape\tablename~\thetable: continua dalla pagina precedente} \\
\toprule
\multicolumn{1}{|c}{\textbf{Requisito}} & \multicolumn{1}{|p{6cm}}{\textbf{Descrizione}}   & \multicolumn{1}{|c}{\textbf{Fonte}} & \multicolumn{1}{|c|}{\textbf{Caso d'uso}}\\
\midrule
\endhead
\midrule
\multicolumn{2}{r}{\footnotesize\itshape\tablename~\thetable: continua nella prossima pagina} \\
\endfoot
\bottomrule
\multicolumn{2}{r}{\footnotesize\itshape\tablename~\thetable: si conclude dalla pagina precedente} \\
\endlastfoot

% Requisiti di qualita

\midrule
ROQ13
& Devono essere rispettate tutte le norme riportate nel documento Norme di Progetto
& Interna
&
\\

\midrule
ROQ14
& Il \gloss{software} deve essere dotato di un manuale utente che ne descriva l'utilizzo
& Interna
&
\\

\midrule
ROQ15
& Il progetto deve essere pubblicato su \gloss{GitHub} e utilizzare le sue distribuzioni per segnalare eventuali correzioni o errori
& Capitolato
&
\\

%FINE TABELLA REQUISITI QUALITA', NON CANCELLARE
\end{longtable}

\newpage
\subsection{Requisiti di vincolo}
\begin{longtable}{|c|p{6cm}|c|c|}
\caption{Requisiti di vincolo}
\label{tab:Requisiti di vincolo} \\
\toprule
\multicolumn{1}{|c}{\textbf{Requisito}} & \multicolumn{1}{|p{6cm}}{\textbf{Descrizione}}   & \multicolumn{1}{|c}{\textbf{Fonte}} & \multicolumn{1}{|c|}{\textbf{Caso d'uso}}\\
\midrule
\endfirsthead
\multicolumn{2}{l}{\footnotesize\itshape\tablename~\thetable: continua dalla pagina precedente} \\
\toprule
\multicolumn{1}{|c}{\textbf{Requisito}} & \multicolumn{1}{|p{6cm}}{\textbf{Descrizione}}   & \multicolumn{1}{|c}{\textbf{Fonte}} & \multicolumn{1}{|c|}{\textbf{Caso d'uso}}\\
\midrule
\endhead
\midrule
\multicolumn{2}{r}{\footnotesize\itshape\tablename~\thetable: continua nella prossima pagina} \\
\endfoot
\bottomrule
\multicolumn{2}{r}{\footnotesize\itshape\tablename~\thetable: si conclude dalla pagina precedente} \\
\endlastfoot

% Requisiti di vincolo

\midrule
ROV16
& Il database degli utenti deve essere \gloss{criptato}
& Interna
&
\\

\midrule
ROV17
& Le pagine web prodotte dal framework MaaP devono essere compatibili con la versione 30.0.x di \gloss{Google Chrome} o superiori
& Capitolato
&
\\

\midrule
ROV18
& Le pagine web prodotte dal framework MaaP devono essere compatibili con la versione 24.x o superiore di Firefox
& Capitolato
&
\\

\midrule
ROV19
& Il sistema deve accettare solo file di configurazione che hanno un determinato formato già fissato
& Interna
&
\\

\midrule
ROV20
& Il database degli utenti deve essere realizzato utilizzando MongoDB
& Capitolato
&
\\

\midrule
ROV21
& Il database degli utenti deve essere indipendente dal database di analisi
& Capitolato
&
\\

\midrule
ROV22
& Il database di analisi utilizzato deve essere stato realizzato utilizzando MongoDB
& Capitolato
&
\\

\midrule
ROV23
& L'\gloss{interfaccia}  con il database deve essere realizzata con Mongoose
& Capitolato
&
\\

\midrule
ROV24
& L'\gloss{infrastruttura} delle pagine web generate deve essere realizzata con \gloss{Express}
& Capitolato
&
\\

\midrule
ROV25
& La \gloss{componente} \gloss{server} deve essere realizzata con \gloss{Node.js}
& Capitolato
&
\\

\midrule
ROV26
& Il software sarà fornito di un sistema di installazione per farlo funzionare
& Interna
&
\\

\midrule
ROV27
& Deve essere possibile effettuare il \gloss{deployment} su \gloss{Heroku}
& Capitolato
&
\\

%FINE TABELLA REQUISITI DI VINCOLO, NON CANCELLARE
\end{longtable}


%AJK-FINEREQUISITI NON CANCELLARE QUESTO COMMENTO XKE SERVE ALLO SCRIPT REQUISITI :)

\subsection{Requisiti accettati}
Tutti i requisiti obbligatori e desiderabili saranno implementati. A causa di tempo e risorse limitate
i requisiti opzionali non potranno essere soddisfatti.
