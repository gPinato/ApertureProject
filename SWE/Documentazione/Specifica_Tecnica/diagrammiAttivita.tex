%TEMPLATE PER DIAGRAMMA ATTIVITA' o sequenza

\subsection{Utente Business}
\immagine{./Diagrammi/Activity/attUtenteBusiness}{Diagramma attività: Utente Business}
Il diagramma precedente illustra le funzionalità disponibili all'utente business.\\
Quest'ultimo, dopo aver effettuato il login, può scegliere se navigare il menu delle Collection o accedere direttamente ad una di esse, o ad un Document, mediante un link diretto.
Nel caso decidesse di ricercare manualmente un documento sono disponibili filtri e ordinamenti.
Può infine modificare i propri dati personali dal suo profilo.

\subsection{Utente Business Amministratore}
\immagine{./Diagrammi/Activity/attAmministratore}{Diagramma attività: Utente Business Amministratore}
Il diagramma precedente illustra le funzionalità disponibili all'utente business amministratore.\\
Quest'ultimo ha a disposizione anche tutte le funzionalità di un normale utente business, tuttavia queste ultime sono state omesse dal diagramma per evitare ridondanza e semplificare la lettura.
L'utente amministratore può gestire i profili di tutti gli utenti, gestire gli indici disponibili e modificare o cancellare i Document presenti nel database.

\subsection{Utente Business Amministratore - Gestione indici}
\immagine{./Diagrammi/Activity/GestioneIndici}{Diagramma attività: Gestione Indici}
Il diagramma precedente illustra in dettaglio la gestione degli indici da parte degli Utenti Amministratori.\\
La gestione degli indici offre due possibilità: creazione e cancellazione.
Per creare un indice, l'amministratore seleziona una query tra l'elenco delle più utilizzate e ne fa un indice.
Per l'eliminazione, l'amministratore seleziona un indice esistente e lo cancella dal sistema.

\subsection{Utente Business Amministratore - Gestione Document esterna}
\immagine{./Diagrammi/Activity/attAmministratoreGDIndex}{Diagramma attività: Gestione Document esterna}
Il diagramma precedente illustra in dettaglio la gestione dei Document da parte degli Utenti Amministratori.\\
L'amministratore può modificare o cancellare un Document mediante i pulsanti di scelta rapida posizionati accanto al nome del Document oppure aprire il Document per visualizzarlo e modificarlo/cancellarlo dall'interno.


\subsection{Utente Business Amministratore - Gestione Document interna}
\immagine{./Diagrammi/Activity/attAmministratoreGDIndexsub}{Diagramma attività: Gestione Document interna}
Il diagramma precedente illustra in dettaglio la gestione dei Document da parte degli Utenti Amministratori.\\
In questo diagramma viene descritto il caso in cui il documento viene modificato, dopo essere stato aperto, dalla sua schermata di visualizzazione.

\subsection{Utente Business Amministratore - Gestione utenti}
\immagine{./Diagrammi/Activity/attAmministratoreGU}{Diagramma attività: Gestione utenti}
Il diagramma precedente illustra la gestione degli utenti da parte degli Utenti Amministratori.\\
L'amministratore può ordinale l'elenco utenti per una chiave o utilizzando un filtro. Dopo aver selezionato un profilo utente, è libero di modificarne i dati personali, i permessi o di cancellarlo dal sistema.

\subsection{Utente Sviluppatore - Gestione progetto}
\immagine{./Diagrammi/Activity/attSviluppatore}{Diagramma attività: Gestione progetto}
Il diagramma precedente illustra la gestione del progetto da parte di Utenti Sviluppatori.\\
L'Utente Sviluppatore è libero di creare, clonare o cancellare un progetto. Una volta selezionato un progetto da gestire, è libero di avviare o fermare il server MaaP, modificare i template per le pagine web e modificare od inserire nuovi file DSL, oltre che a modificare altre impostazioni minori del server.
