%\newcommand {\gloss} [1]{#1}
%nuova pagina con i termini che iniziano con A
\paginaGlossario{A}
\elemento
{Account} 
{L'insieme di funzionalità, strumenti e contenuti messi a disposizione da un sito web o da qualsiasi altro tipo di applicazione ad un utente, per usufruire di determinati servizi che il sistema offre.}
\elemento
{Agente di provider} 
{Un \gloss{agente} che che ha la capacità e l'autorizzazione di eseguire azioni associate con un \gloss{servizio (web)} dedicato al suo possessore, ovvero l'\gloss{entità di provider}.}
\elemento
{Alfanumerico}
{Carattere singolo composto solamente da lettere latine maiuscole o minuscole dalla a alla z e dai numeri dallo 0 al 9.}
\elemento
{Ambiente}
{Consiste sia del sistema hardware che di quello software sui quali è stato pianificato l'utilizzo del prodotto software sviluppato.}
\elemento
{Ambiguo}
{Termine usato per indicare una cosa che non è chiara, ovvero presenta un doppio senso o un fraintendimento.}

\elemento
{Amministratore} 
{Classe di utenza privilegiata rispetto ad utente normale, ovvero ha funzionalità in più che gestiscono un sistema.}

\elemento
{Applicazioni client-server}
{Applicazioni nelle quali esistono due componenti: un \gloss{client} che fa una richiesta al \gloss{server} e un \gloss{server} appunto che deve rispondere alla richiesta del \gloss{client}.}
\elemento
{Applicazione web} 
{Applicazione fruibile via \gloss{web}, solitamente con architettura \gloss{client-server}.}
\elemento
{Array} 
{Detto anche "vettore", è una struttura dati complessa, statica e omogenea, usata in molti linguaggi di programmazione. Si può immaginare un array come una sorta di casellario, le cui caselle sono dette celle dell'array stesso. Ciascuna delle \gloss{celle} si comporta come una variabile tradizionale che rappresenta un elemento dell'array; tutte le celle sono variabili di uno stesso tipo preesistente, detto \gloss{tipo base} dell'array.}

\elemento
{Attributo}
{Descrizione di un campo dati di una classe.}

\paginaGlossario{B}
\elemento
{Base di dati} 
{Detta anche banca dati, indica un archivio di dati, o un insieme di archivi, in cui le informazioni in esso contenute sono strutturate e collegate tra loro secondo un particolare modello logico e in modo tale da consentire la gestione/organizzazione efficiente dei dati stessi e l'interfacciamento con le richieste dell'\gloss{utente} attraverso le cosiddette \gloss{query} (di ricerca o interrogazione, inserimento, cancellazione, aggiornamento ecc.) grazie a particolari applicazioni \gloss{software} dedicate, basate su un'architettura di tipo \gloss{client-server}.}

\elemento
{Baseline}
{Durante la pianificazione rappresenta la suddivisione iniziale delle attività nel tempo a cui si fa riferimento per l'avanzamento del processo.}

\elemento
{Best practice} 
{Tradotto in "buona prassi", in genere identifica le esperienze più significative, o comunque quelle che hanno permesso di ottenere migliori risultati, relativamente a svariati contesti.}

\elemento
{Branch}
{E' considerato un ramo di sviluppo parallelo del filone principale.}

\elemento
{Browser} 
{Programma che consente di usufruire dei servizi di connettività in \gloss{Internet}, o di una \gloss{rete} di computer, e di navigare sul \gloss{Web}.}
\elemento
{Business}
{In italiano è tradotto con affari; identifica un'attività economica.}
\elemento
{Business logic} 
{Si riferisce a tutta quella logica applicativa che rende operativa un'applicazione. E' un termine largamente utilizzato nella progettazione del \gloss{software} per individuare un componente \gloss{software} di una architettura \gloss{software}. Nelle \gloss{applicazioni web} viene eseguita da un \gloss{server} su richiesta di un \gloss{client} attraverso il \gloss{browser web} e interfacciandosi con la parte dati che può essere un \gloss{database}.}



\elemento
{Bytecode} 
{Codice di \gloss{programmazione} che, una volta \gloss{compilato}, è eseguibile attraverso una \gloss{macchina virtuale} invece che dal \gloss{processore} di un computer.}

\paginaGlossario{C}

\elemento
{Chat}
{Insieme di servizi che hanno due elementi in comune: il primo è che il dialogo avviene in tempo reale, mentre il secondo è che ci si può mettere in contatto con perfetti sconosciuti.}

\elemento
{Chiave}
{E' un identificatore di un elemento del \gloss{database} di tipo \gloss{NoSQL}.}

\elemento
{Classe (informatca)}
{Nella \gloss{programmazione} \gloss{orientata agli oggetti} una classe è un costrutto di un \gloss{linguaggio di programmazione} usato come modello per creare \gloss{oggetti}. Il modello comprende \gloss{attributi} e \gloss{metodi} che saranno condivisi da tutti gli oggetti creati (\gloss{istanze}) a partire dalla classe. Un "oggetto" è, di fatto, l'\gloss{istanza} di una classe.}

\elemento
{Client} 
{Componente che accede ai \gloss{servizi} o alle risorse di un'altra componente detta \gloss{server}.} 

\elemento
{Cloud} 
{Insieme di tecnologie che permettono di archiviare e/o elaborare dati mediante l'utilizzo di risorse \gloss{hardware} e \gloss{software distribuite}.}

\elemento
{Codice} 
{Vedi codice sorgente.}
\elemento
{Codice oggetto}
{Traduzione del \gloss{codice sorgente} in \gloss{linguaggio macchina}, comprensibile dal \gloss{processore}.}
\elemento
{Codice sorgente}
{Linee di codice che compongono un \gloss{programma} scritto in un \gloss{linguaggio di programmazione}. Abbreviato anche come "sorgente" o solamente "codice".}
\elemento
{Collection} 
{Raccolta di Document in database come MongoDB.}


\elemento
{Collection-Index} 
{Pagina generata dal \gloss{framework} \gloss{MaaP} che mostra un elenco di \gloss{Document} con delle specifiche coppie chiavi-valori; inoltre è presente un menù dove è possibile spostarsi tra le varie pagine \gloss{Collection Index} o sfruttare altre funzionalità messe a disposizione dallo sviluppatore.}


\elemento
{Compatibilità}
{In informatica indica la proprietà di due o più computer che, scambiandosi dei dati, accettano gli stessi programmi senza fare modifiche ad essi.}
\elemento
{Compilazione} 
{Processo di traduzione che porta alla creazione di un \gloss{codice oggetto} partendo da un \gloss{codice sorgente}.}
\elemento
{Criptato} 
{Questo termine viene utilizzato per riferirsi a quella metodologia che fa si che l'invio dei dati sia reso non comprensibile ad eventuali intercettazioni.}

\paginaGlossario{D}
\elemento
{Dashboard}
{Pagina principale dove l'utente può aver accesso alle varie funzionalità in modo chiaro e diretto.}
\elemento
{Database} 
{Vedi basi di dati.}
\elemento
{Database administration} 
{E' la funzione di gestione e manutenzione dei sistemi di gestione dei database; molte aziende importanti necessitano continuamente di una gestione delle \gloss{basi di dati}.}
\elemento
{Debugger} 
{Software specificatamente progettato per l'analisi e l'eliminazione dei bug, ovvero errori di programmazione interni al codice di altri programmi.}
\elemento
{Deployment} 
{Traducibile letteralmente in "spiegamento". E' l'insieme di attività necessarie a rendere un sistema \gloss{software} disponibile all'uso.}
\elemento
{Default} 
{Valore o un'azione standard che caratterizza qualsiasi aspetto globale a meno di un cambiamento improvviso.}
\elemento
{Diagramma di Gantt}
{Strumento di supporto alla gestione dei progetti. E' costruito partendo da un asse orizzontale, a rappresentazione dell'arco temporale totale del progetto, suddiviso in fasi incrementali (ad esempio, giorni, settimane, mesi), e da un asse verticale, a rappresentazione delle mansioni o attività che costituiscono il progetto. Un diagramma di Gantt permette dunque la rappresentazione grafica di un calendario di attività, utile al fine di pianificare, coordinare e tracciare specifiche attività in un progetto dando una chiara illustrazione dello stato d'avanzamento del progetto rappresentato.} 

\elemento
{Distribuito (software)}
{Insieme di entità autonome (componenti \gloss{software} e \gloss{hardware}) spazialmente separate che comunicano e coordinano tra loro le loro azioni attraverso scambio di messaggi.}

\elemento
{Distribuzione} 
{Collezione di programmi relativi ad uno o più campi di applicazione, selezionati e rilasciati come un unico pacchetto. Ad esempio il \gloss{sistema operativo Linux} offre più diverse distribuzioni, come ad esempio Ubuntu, Debian, Fedora, eccetera.}

\elemento
{Document}
{Istanza, o record, di un database in \gloss{MongoDB}. Esso è costituito da un insieme di chiavi con il rispettivo valore}
\elemento
{Document-Show} 
{Pagina che, in seguito ad una selezione di una chiave di un Document nella pagina Collection index, mostra il Document per intero con tutte le chiavi e i relativi valori.}

\elemento
{Driver}
{Insieme di procedure che permette ad un \gloss{sistema operativo} di pilotare un dispositivo \gloss{hardware} o \gloss{software}.}

\elemento
{DSL} 
{Acronimo di "Domain Specific Language", "linguaggio specifico di dominio". E' un \gloss{linguaggio di programmazione} altamente contestualizzato, cioè associato ad un dominio specifico.} 


\paginaGlossario{E}
\elemento
{Editor}
{\gloss{Programma} di composizione di testi, il suo scopo è facilitare la scrittura di un testo. E' generalmente incluso in ogni \gloss{sistema operativo}.}
\elemento
{Email} 
{\gloss{Servizio internet} grazie al quale ogni \gloss{utente} abilitato può inviare e ricevere dei messaggi utilizzando un computer o qualsiasi altro dispositivo elettronico connesso a \gloss{Internet}, attraverso un proprio \gloss{account} di posta registrato presso un \gloss{provider} del {servizio}.}
\elemento
{Express} 
{\gloss{Applicazione web} scritta in \gloss{node.js} minima e flessibile, che fornisce un robusto set di funzionalità per la costruzione di singole e multipagine e \gloss{applicazioni web} ibride.}

\paginaGlossario{F}
\elemento
{File di descrizione}
{File scritto con \gloss{linguaggio} \gloss{DSL} dallo \gloss{sviluppatore}, e serve per generare pagine di tipo\gloss{Collection-Index} e \gloss{Document-Show}.}
\elemento
{File di configurazione}
{E' un file che permettere di impostare parametri necessari al funzionamento dell'applicazione nella loro versione predefinita; ad esempio \gloss{database} per sistema di \gloss{autenticazione}, \gloss{e-mail}, \gloss{server}, ecc.}
\elemento
{File system}
{Meccanismo con il quale i file sono posizionati e organizzati su un dispositivo di archiviazione.}
\elemento
{Filtro} 
{Componente che ha il compito di selezionare una fonte in ingresso secondo dei criteri, in modo da avere un risultato finale utile all'\gloss{utente}.}
\elemento
{Firefox}
{E' un browser open-source multipiattaforma e secondo alcune statistiche è il secondo browser più popolare.}
\elemento
{Form}
{Termine usato per indicare l'\gloss{interfaccia} di una applicazione che consente ad un \gloss{utente} di inserire e inviare ad un \gloss{server} dei dati liberamente digitati dallo stesso \gloss{utente}.}
\elemento
{Framework}
{Struttura di supporto su cui un \gloss{software} può essere progettato e realizzato. Alla base di un \gloss{framework} sono sempre presenti delle \gloss{librerie di codice} utilizzabili con uno o più \gloss{linguaggi di programmazione}; esse sono spesso corredate da una serie di strumenti di supporto allo sviluppo \gloss{software}, come ad esempio un \gloss{IDE} o un \gloss{debugger}, o altri strumenti ideati per aumentare la velocità di sviluppo del prodotto finito.
Lo scopo di un framework è quello di far risparmiare allo\gloss{sviluppatore} la riscrittura di \gloss{codice} già scritto precedentemente per fini simili. La necessità di questo strumento si è venuta a creare quando le \gloss{interfacce} utente sono diventate sempre più complesse ed è cominciata ad aumentare la quantità \gloss{software} con funzionalità secondarie simili.}
\elemento
{Funzione (software)}
{Particolare costrutto \gloss{sintattico}, in qualche \gloss{linguaggio di programmazione}, che permette di raggruppare, all'interno di un \gloss{programma}, una sequenza di istruzioni in un unico blocco di istruzioni espletando così una determinata e in generale più complessa operazione, azione o elaborazione sui dati del \gloss{programma} stesso in modo tale che a partire da determinati \gloss{input} restituisca determinati \gloss{output}.}

\paginaGlossario{G}
\elemento
{Gantt} 
{L'inventore dei \gloss{diagrammi di Gantt}, che vengono usati nella gestione e nella pianificazione delle attività.}
\elemento
{Git} 
{Sistema di controllo di \gloss{versione distribuito} gratuito e \gloss{open source} designato alla gestione di progetti \gloss{software}}
\elemento
{GitHub} 
{\gloss{Servizio} di \gloss{web hosting} orientato allo sviluppo \gloss{software} e basato sul sistema di controllo di \gloss{versione} di \gloss{Git}. \gloss{GitHub} offre \gloss{servizi} di \gloss{repository} online sia gratuiti che a pagamento.}
\elemento
{Gmail} 
{\gloss{Servizio} gratuito di posta elettronica gestito e offerto da \gloss{Google}.}

\elemento
{Google} 
{Motore di ricerca per \gloss{internet} che oltre alla funzione di effettuare ricerche offre molti altri \gloss{servizi}, per esempio gestione \gloss{e-mail}, gestione calendari ecc..} 
\elemento
{Google Chrome} 
{\gloss{Browser} \gloss{web} gratuito creato da \gloss{Google}.}
\elemento
{Google Drive} 
{\gloss{Servizio web} offerto e gestito da \gloss{Google} che permette l'archiviazione, la sincronizzazione, la condivisione e la modifica collaborativa di documenti.}

\paginaGlossario{H}
\elemento
{Hardware} 
{Componenti fisiche che compongono un computer.}
\elemento
{Heroku} 
{\gloss{Piattaforma} \gloss{cloud} per per fornire \gloss{servizi} tramite il \gloss{deployment} di applicazioni scritte in \gloss{linguaggi} quali \gloss{Node.js}, \gloss{Ruby}, \gloss{Scala}, e molti altri.}
\elemento
{Host} 
{Ogni terminale collegato ad una \gloss{rete} o più in particolare ad \gloss{Internet}.}
\elemento
{Hosting} 
{Tradotto letteralmente in "ospitare". \gloss{Servizio} di \gloss{rete} che consiste nell'allocare su un \gloss{server web} le \gloss{pagine web} di un \gloss{sito web}, rendendolo così accessibile dalla \gloss{rete} \gloss{Internet} e ai suoi \gloss{utenti}.}

\paginaGlossario{I}
\elemento
{IDE} 
{Acronimo di "Integrated Developement Enviroment", tradotto in "ambiente di sviluppo integrato". \gloss{Software} che, durante la programmazione, aiuta i \gloss{programmatori} nello sviluppo del \gloss{codice sorgente} di un \gloss{programma}.}
\elemento
{IEC} 
{Acronimo di "International Electrotechnical Commission", tradotto in "Commissione Elettrotecnica Internazionale". E' un'organizzazione internazionale per la definizione di standard in materia di elettricità, elettronica e tecnologie correlate. Molti dei suoi standard sono definiti in collaborazione con l'\gloss{ISO}.}
\elemento
{Input}
{Tradotto letteralmente in "immettere". Sequenza di dati o informazioni, immessi per mezzo di una \gloss{periferica}, detta appunto di \gloss{input}, e successivamente elaborati.}
\elemento
{Infrastruttura} 
{Insieme di risorse \gloss{hardware} e risorse \gloss{software}.}
\elemento
{Interfaccia} 
{Dispositivo fisico o virtuale che permette la comunicazione e l'interazione tra due entità. Ad esempio in un'\gloss{interfaccia} grafica di un \gloss{programma} posso inserire dei valori  e visualizzare un eventuale risultato o una risposta da parte del \gloss{programma} stesso.}
\elemento
{Internet} 
{E' una contrazione della locuzione inglese "interconnected networks", ovvero "reti interconnesse".
 Ed è una \gloss{rete} mondiale di \gloss{reti} di computer ad accesso pubblico, attualmente rappresentante il principale mezzo di comunicazione di massa, che offre all'\gloss{utente} una vasta serie di contenuti potenzialmente informativi e \gloss{servizi}.}
\elemento
{Internet Protocol}
{\gloss{Protocollo di comunicazione} di \gloss{rete} appartenente all'insieme di protocolli \gloss{Internet} \gloss{TCP/IP} su cui è basato il funzionamento della \gloss{rete} \gloss{Internet}.}
\elemento
{Interprete} 
{\gloss{Programma} In grado di eseguire altri \gloss{programmi} a partire direttamente dal relativo \gloss{codice sorgente}. Un \gloss{interprete} ha lo scopo di eseguire un \gloss{programma} in un \gloss{linguaggio di alto livello}, senza la previa \gloss{compilazione} dello stesso (\gloss{codice oggetto}) ciò di eseguire le istruzioni nel \gloss{linguaggio} usato, traducendole di volta in volta in istruzioni in \gloss{linguaggio macchina}.}

\elemento
{Istanza}
{Un particolare \gloss{oggetto} di una particolare \gloss{classe}.}

\elemento
{IP} 
{Acronimo e contrazione di "Internet Protocol Address". Etichetta numerica che identifica univocamente un dispositivo (\gloss{host}) collegato a una \gloss{rete} informatica che utilizza l'\gloss{Internet Protocol} come \gloss{protocollo di comunicazione}.}
\elemento
{Ipertesto} 
{Insieme di documenti messi in relazione tra loro per mezzo di parole chiave.} 
\elemento
{ISO} 
{Acronimo di "International Organization for Standardization", tradotto in "Organizzazione Internazionale per la Normazione". Come dice il nome, è un'organizzazione internazionale atta alla realizzazione di specifiche standard per la realizzazione di prodotti, \gloss{servizi} e pratiche corrette per aiutare ogni impresa a lavorare in modo \gloss{efficace} ed \gloss{efficiente}.}

\paginaGlossario{J}
\elemento
{Java} 
{\gloss{Linguaggio di programmazione orientato agli oggetti}, creato dalla \gloss{Sun Microsystems}.}

\elemento
{JavaScript} 
{\gloss{Linguaggio di scripting orientato agli oggetti} comunemente usato nella creazione di {siti web}. In \gloss{applicazioni di tipo client-server} se il codice \gloss{JavaScript} è sul lato \gloss{client} allora viene eseguito sul \gloss{client} e non sul \gloss{server}, cosi da non sovraccaricare la parte \gloss{server}.}
\elemento
{JSON} 
{Acronimo di "JavaScript Object Notation", tradotto in "notazione oggetto \gloss{JavaScript}". E' un formato di memorizzazione di dati, di piccolo peso e intuitivo.}
\elemento
{JVM}
{Acronimo di "Java Virtual Machine", tradotto in "\gloss{macchina virtuale} Java". E' il componente della \gloss{piattaforma Java} che esegue i \gloss{programmi} tradotti in \gloss{bytecode} dopo una prima \gloss{compilazione}.}

\paginaGlossario{L}
\elemento
{LaTeX} 
{\gloss{Linguaggio} usato per la preparazione di testi basato sul \gloss{programma} \gloss{TeX}}
\elemento
{Layout} 
{Identifica l'impaginazione e la struttura grafica di una \gloss{pagina web} o di un documento.}
\elemento
{Libreria di codice} 
{Insieme di funzioni o strutture dati predisposte per essere collegate ad un \gloss{programma} \gloss{software} attraverso opportuno collegamento.}
\elemento
{Linguaggio di alto livello} 
{\gloss{Linguaggio di programmazione} più astratto del \gloss{linguaggio macchina}, direttamente eseguibile da un computer, ma più vicino o familiare alla logica del nostro \gloss{linguaggio} naturale. I \gloss{programmi} ad alto livello possono essere ricondotti a \gloss{programmi} in \gloss{linguaggio macchina} in modo automatico, ovvero da un altro \gloss{programma}, detto \gloss{interprete}.}
\elemento
{Linguaggio di programmazione} 
{E' un \gloss{linguaggio formale}, ben definito e composto da una \gloss{sintassi} e una \gloss{semantica}.}
\elemento
{Linguaggio di scripting} 
{\gloss{Linguaggio interpretato}, destinato in genere a compiti di automazione del \gloss{sistema operativo}, oppure viene usato all'interno delle \gloss{pagine web} per gestire il comportamento delle \gloss{pagine web} stesse in base all' \gloss{input} dell'\gloss{utente}.}
\elemento
{Linguaggio formale} 
{Notazione o formalismo con \gloss{sintassi} e \gloss{semantica} definite in modo preciso (spesso matematico/formale) e, in molti casi, tali da consentire qualche forma di elaborazione automatica del \gloss{linguaggio} stesso.}
\elemento
{Link} 
{Abbreviazione di "hyperlink". E' un \gloss{collegamento ipertestuale} in grado di rinviare a un contenuto informativo presente in un dominio fisicamente o virtualmente separato. Il \gloss{link} è di solito associato ad una o più parole chiave, evidenziate visivamente da una diversa colorazione o sottolineatura.}
\elemento
{Linux} 
{\gloss{Sistema operativo} della \gloss{Linux Foundation}.}
\elemento
{Linux Foundation} 
{Associazione senza fini di lucro, specializzata nel campo dell'informatica \gloss{open source}.}
\elemento
{Linguaggio interpretato} 
{Un \gloss{linguaggio} informatico è per definizione diverso dal \gloss{linguaggio macchina}. Bisogna quindi tradurlo per renderlo leggibile dal punto di vista del \gloss{processore}. Un \gloss{programma} scritto in un \gloss{linguaggio interpretato} ha bisogno di un \gloss{programma} ausiliario (l'\gloss{interprete}) per tradurre man mano le istruzioni del \gloss{programma} in \gloss{linguaggio macchina}.}
\elemento
{Linguaggio macchina} 
{Definito come il \gloss{linguaggio} utilizzato dal \gloss{processore}, tramite sequenze di 0 e 1. Ogni sequenza ordinata di 0 e 1, raggruppata in gruppi di una certa dimensione, identifica una precisa istruzione per il \gloss{processore}.}
\elemento{Logger}
{Componente non intrusivo di registrazione dei dati di esecuzione per analisi dei risultati.}

\paginaGlossario{M}
\elemento
{MaaP}
{Framework che genera interfacce web di amministrazione dei dati di business basati sulle tecnologie Node.js e MongoDB. }

\elemento
{Macchina virtuale} 
{Implementazione \gloss{software} di un \gloss{ambiente} di elaborazione in cui un \gloss{sistema operativo} o un \gloss{programma} possono essere installati ed eseguiti.}
\elemento
{MaaP's web}
{Insieme delle pagine web prodotte dal framework MaaP.}
\elemento
{Metodo}
{Detto anche funzione membro, è un termine che viene usato principalmente nel contesto della programmazione orientata agli oggetti. Indica un sottoprogramma associato in modo esclusivo ad una classe e che rappresenta (in genere) un'operazione eseguibile sugli oggetti e istanze di quella classe.}
\elemento
{Milestone}
{Viene utilizzato nella pianificazione e gestione di progetti per indicare il raggiungimento di obiettivi definiti in fase di definizione del progetto stesso. Molto spesso le milestone sono rappresentate da eventi, come ad esempio scadenze per la consegna di documenti, e indicano importanti traguardi intermedi durante lo svolgimento del progetto.}
\elemento
{MongoDB}
{Sistema gestionale di basi di dati, non relazionale, orientato ai documenti e di tipo NoSQL. Il linguaggio di programmazione utilizzato per la gestione dei dati è JavaScript, in particolare la sua notazione JSON.}
