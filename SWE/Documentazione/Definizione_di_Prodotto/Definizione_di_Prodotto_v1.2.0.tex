
%includo il file che contiene la versione dei documenti
\newcommand{\versioneAnalisiDeiRequisiti}{2.2.0}			
\newcommand{\versioneNormeDiProgetto}{2.2.0}			
\newcommand{\versioneGlossario}{2.2.0}			
\newcommand{\versionePianoDiQualifica}{2.2.0}			
\newcommand{\versionePianoDiProgetto}{2.2.0}	
\newcommand{\versioneStudioDiFattibilita}{2.2.0}
\newcommand{\versioneSpecificaTecnica}{2.2.0}


\newcommand{\Versione}{\versioneDefinizioneDiProdotto{}} %Versione Finale
\newcommand{\Data}{2014-04-21}				           	 %Data di creazione
\newcommand{\DataUltimaModifica}{2014-05-19}
\newcommand{\TipoDocumento}{Definizione di Prodotto}	 %tipo documento

%includo il file header.tex (logo grande in prima pagina piu qualche altra regola)
%questo file contiene impostazioni comuni per tutte i documenti

%definizione packages utilizzati
\documentclass[a4paper]{article}
\usepackage[utf8x]{inputenc}
\usepackage{enumitem}
\usepackage[italian]{babel}
\usepackage{latexsym}
\usepackage{xparse}
\usepackage{float}
\usepackage{subfloat}
\usepackage{subfig}
\usepackage{fancyhdr}
\usepackage{eurofont}
\usepackage{lastpage}
\usepackage{graphicx}
\usepackage{textcomp}
\usepackage{booktabs}
\usepackage{color}
\usepackage{lscape}
\usepackage{hyperref}
\hypersetup{colorlinks=true, linkcolor=black, anchorcolor=red, urlcolor=blue}
\usepackage{longtable}
\usepackage{tabularx}
\usepackage{abstract}
\usepackage{appendix}
\usepackage{multicol}
\usepackage{bmpsize}
\usepackage[all]{hypcap}
\usepackage{titlesec}
\usepackage{indentfirst}
\usepackage{lipsum,titletoc}

%\setcounter{secnumdepth}{4}

%****************INIZIO GESTIONE SUBSECTION MULTIPLE
\makeatletter
\newcommand\level[1]{%
  \ifcase#1\relax\expandafter\chapter\or
    \expandafter\section\or
    \expandafter\subsection\or
    \expandafter\subsubsection\else
    \def\next{\@level{#1}}\expandafter\next
  \fi}
\newcommand{\@level}[1]{%
  \@startsection{level#1}
    {#1}
    {\z@}%
    {-3.25ex\@plus -1ex \@minus -.2ex}%
    {1.5ex \@plus .2ex}%
    {\normalfont\normalsize\bfseries}}

\newdimen\@leveldim
\newdimen\@dotsdim
{\normalfont\normalsize
 \sbox\z@{0}\global\@leveldim=\wd\z@
 \sbox\z@{.}\global\@dotsdim=\wd\z@
}

\newcounter{level4}[subsubsection]
\@namedef{thelevel4}{\thesubsubsection.\arabic{level4}}
\@namedef{level4mark}#1{}
\def\l@section{\@dottedtocline{1}{0pt}{\dimexpr\@leveldim*4+\@dotsdim*1+6pt\relax}}
\def\l@subsection{\@dottedtocline{2}{0pt}{\dimexpr\@leveldim*5+\@dotsdim*2+6pt\relax}}
\def\l@subsubsection{\@dottedtocline{3}{0pt}{\dimexpr\@leveldim*6+\@dotsdim*3+6pt\relax}}
\@namedef{l@level4}{\@dottedtocline{4}{0pt}{\dimexpr\@leveldim*7+\@dotsdim*4+6pt\relax}}

\count@=4
\def\@ncp#1{\number\numexpr\count@+#1\relax}
\loop\ifnum\count@<100
  \begingroup\edef\x{\endgroup
    \noexpand\newcounter{level\@ncp{1}}[level\number\count@]
    \noexpand\@namedef{thelevel\@ncp{1}}{%
      \noexpand\@nameuse{thelevel\@ncp{0}}.\noexpand\arabic{level\@ncp{1}}}
    \noexpand\@namedef{level\@ncp{1}mark}####1{}%
    \noexpand\@namedef{l@level\@ncp{1}}%
      {\noexpand\@dottedtocline{\@ncp{1}}{0pt}{\the\dimexpr\@leveldim*\@ncp{5}+\@dotsdim*\@ncp{0}\relax}}}%
  \x
  \advance\count@\@ne
\repeat
\makeatother
\setcounter{secnumdepth}{100}
\setcounter{tocdepth}{100}
%****************FINE GESTIONE SUBSECTION MULTIPLE

%impostazioni relative alla visualizzazione delle section 
%nell'indice
\titlecontents{section}
[0pt]%left indent
{\bfseries}
{\contentslabel{2.3em}}
{\hspace*{-2.3em}}
{\hfill\contentspage}
[]%separator


\oddsidemargin=.15in
\evensidemargin=.15in
\textwidth=6in
\topmargin=-.5in
\parindent=0in
\headheight=1in
\DeclareMathSizes{10}{10}{10}{10} %per piano qualifica
\pagestyle{fancy}
\lhead{
\bfseries {\Large \TipoDocumento}\\
\bfseries Versione: \Versione\\
}
\chead{}
\lhead{
\includegraphics[scale=0.455]{../Logo&Header/apertureHead.png}
}
\lfoot{\bfseries \TipoDocumento{} v\Versione}
\cfoot{}
\rfoot{\thepage\ of \mypageref{LastPage}}
\newcommand{\mypageref}[1]{
\hypersetup{linkcolor=black}\pageref{#1}\hypersetup{linkcolor=black}}
%\userpackage{lipsum}
\renewcommand{\footrulewidth}{0.4pt}

%definizioni comandi comuni utilizzati
\newcommand{\numref}[1]{\textsl{\nameref{#1} (\ref{#1})}}
\newcommand{\NomeGruppo}{Aperture Software}
\newcommand{\Progetto}{MaaP: MongoDB as an admin Platform}
\newcommand{\Prop}{CoffeeStrap}

%definizione tecnologie
\newcommand{\Node}{Node.js}
\newcommand{\NodeJS}{Node.js}
\newcommand{\Nodejs}{Node.js}

\newcommand{\mongodb}{MongoDB}

%tanti sub quanti ne vogliamo! :)
\newcommand{\subsubsubsection}{\level{4}}
\newcommand{\subsubsubsubsection}{\level{5}}
\newcommand{\subsubsubsubsubsection}{\level{6}}
\newcommand{\subsubsubsubsubsubsection}{\level{7}}
\newcommand{\subsubsubsubsubsubsubsection}{\level{8}}


%definizione comando per parola glossario
\newcommand{\gloss}[1]{\emph{#1}\ped{\emph{\tiny{G}}}}

\newcommand{\grassetto}{\textbf}

%per inserire immagini
\newcommand{\immagine}[2]{ 
\begin{center}
\begin{figure}[H]
\includegraphics[width=\textwidth]{{{#1}}}
\caption{#2}
\label{#1}
\end{figure}
\end{center}
}

\newcommand{\Glossario}{
Al fine di evitare ogni ambiguità nella comprensione del linguaggio utilizzato nel presente documento e, in generale, nella documentazione fornita dal gruppo \NomeGruppo{}, ogni termine tecnico, di difficile comprensione o di necessario approfondimento verrà inserito nel documento \emph{Glossario\_{}v\versioneGlossario{}.pdf}.\\
Saranno in esso definiti e descritti tutti i termini in corsivo e allo stesso tempo marcati da una lettera "G" maiuscola in pedice nella documentazione fornita.
}

\newcommand{\Prodotto}{
Lo scopo del prodotto è produrre un framework per generare interfacce web di amministrazione dei dati di business basati sullo stack \Nodejs{} e \mongodb{}.\\
L'obiettivo è quello di semplificare il lavoro allo sviluppatore che dovrà rispondere in modo rapido e standard alle richieste degli esperti di business.
}

%inizio pagina del documento 
\begin{document}
\thispagestyle{empty}

\begin{center}\centerline{
%inserisco il logo grande della prima pagina
\includegraphics[scale=0.8]{../Logo&Header/logo.png}}

%metto il link dell'email sotto al logo
%{\href{mailto:ApertureSWE@gmail.com}{\color[rgb]{0.39,0.37,0.38}%ApertureSWE@gmail.com}}\\ [3pc]

\vspace{0.5in}

%titolo del progetto
{\Huge {\Progetto}}\\[.5pc]

\underline{\hspace{6in}}\\[8pc]

{\Huge {\TipoDocumento}}\\[1pc]
%{\emph{Versione \Versione}}\\
\end{center}

%\vspace{.05in}
%\vspace{.05in}

%informazioni documento
\begin{center}
%\section{Informazioni documento}
\begin{tabular}{r|l}
%\textbf{Nome} &\TipoDocumento \\
\textbf{Versione} & \Versione{} \\
\textbf{Data creazione} & \Data{} \\
\textbf{Data ultima modifica} & \DataUltimaModifica{} \\
\textbf{Stato del Documento} & Formale \\		          %CAMBIARE QUI
\textbf{Uso del Documento} & Esterno \\			          %CAMBIARE QUI
\textbf{Redazione} &  Pinato Giacomo, Alberto Garbui\\	  %CAMBIARE QUI
& Alessandro Benetti, Andrea Perin\\
\textbf{Verifica} & Fabio Miotto, Michele Maso\\  %ED ANCHE QUI!
\textbf{Approvazione} & Mattia Sorgato\\				      %CAMBIARE QUI
\textbf{Distribuzione} & \parbox[t]{4cm}{\NomeGruppo{}\\Prof. Vardanega Tullio\\Prof. Cardin Riccardo\\ \Prop{} }\\
\end{tabular}
\end{center}

\vspace{0.05in}

%inizio sommario del documento
\begin{abstract}
\begin{center}
Architettura di dettaglio dell'applicazione \Progetto{}.
\end{center}
\end{abstract}

%\vspace{.4in}

%seconda pagina, diario delle modifiche
\newpage
Diario delle modifiche
\begin{center}
\begin{longtable}{|c|c|c|p{0.5\linewidth}|}
\toprule
\textbf{Versione} & \textbf{Data} & \textbf{Autore} & \textbf{Modifiche effettuate}\\

%aggiungere qui una midrule per aggiungere una nuova riga alla tabella
\midrule
1.2.0 & 2014-05-19 & Mattia Sorgato (RE) & Approvazione documento.\\
\midrule
1.1.1 & 2014-05-19 & Fabio Miotto (VR) & Verifica documento.\\
\midrule
1.1.0 & 2014-05-16 & Michele Maso (VR) & Verifica documento.\\
\midrule

1.0.4 & 2014-05-08 & Alessandro Benetti (PR) & Stesura Datamanager.\\
\midrule
1.0.3 & 2014-05-01 & Alberto Garbui (PR) & Stesura controller del Server\\
\midrule
1.0.2 & 2014-04-29 & Andrea Perin (PR) & Stesura dei controller\\
\midrule
1.0.2 & 2014-04-23 & Giacomo Pinato (PR) & Stesura dei servizi e dei controller del Client\\
\midrule
1.0.1 & 2014-04-21 & Alessandro Benetti (PR) & Prima stesura del documento.\\

\bottomrule
\caption{Registro delle modifiche}
\label{tab:changelog}

\end{longtable}
\end{center}

%terza pagina Indice (viene aggiornato in automatico con due compilazioni)
\newpage
\tableofcontents

%pagine successive hanno la lista di tabelle e lista delle figure
%(vengono aggiornate in automatico)
\newpage
%\listoftables	%tabelle
\listoffigures %elenco delle immagini

%qui inizia la prima pagina ufficiale
\newpage
\section{Introduzione}
\subsection{Scopo del documento}
Il seguente documento ha lo scopo di definire nel dettaglio la struttura del sistema \gloss{MaaP}, approfondendo quanto già riportato nella Specifica Tecnica. Tale documento fornisce una struttura dettagliata e completa che viene utilizzata dai \emph{Programmatori} per le attività di codifica. 

\subsection{Scopo del prodotto}
\Prodotto{}

\subsection{Glossario}
\Glossario{}

\subsection{Notazioni}
Per maggiore chiarezza, il nome ogni modulo verrà riportato in grassetto e con il simbolo "@" prima del nome.\\
Ogni riferimento ad un metodo, invece, sarà identificato dalla formattazione in grassetto del nome del metodo stesso. 
I parametri preceduti dal simbolo "\$", nel lato Client identificano i servizi messi a disposizione da AngularJS. Infine, lato Server, i moduli che non hanno alcun simbolo a precedere il loro nome, identificano quei moduli di NodeJS che sono stati importati da fonti esterne.

\subsection{Riferimenti}

\subsubsection{Normativi}
\begin{itemize}
\item \grassetto{Analisi dei requisiti}: \emph{Analisi\_{}dei\_{}Requisiti\_{}v\versioneAnalisiDeiRequisiti{}.pdf};
\item \grassetto{Norme di Progetto}: \emph{Norme\_{}di\_{}Progetto\_{}v\versioneNormeDiProgetto{}.pdf};
\item \grassetto{Specifica Tecnica}: \emph{Specifica\_{}Tecnica\_{}v\versioneNormeDiProgetto{}.pdf};
\end{itemize}

\subsubsection{Informativi}
\begin{itemize}
\item \grassetto{AngularJS API}: \url{https://docs.angularjs.org/api};
\item \grassetto{MongooseJS API}: \url{http://mongoosejs.com/docs/api.html};
\item \grassetto{MongoDB API per Node.js}: \url{http://mongodb.github.io/node-mongodb-native/};
\item \grassetto{Node.js API}: \url{http://nodejs.org/api/}.
\end{itemize}

\newpage
\section{Standard di progetto}
\subsection{Standard di progettazione architetturale}
Gli standard di progettazione architetturale sono definiti nella \emph{Specifica\_{}Tecnica\_{}v\versioneSpecificaTecnica{}.pdf}.

\subsection{Standard di documentazione del codice}
Gli standard di documentazione del codice sono specificati nel documento di 
\emph{Norme\_{}di\_{}Progetto\_{}v\versioneNormeDiProgetto{}.pdf}, nella sezione 4.3.5.1.2.	%Cambia qui, altrimenti, bad juju at you.

\subsection{Standard di programmazione}
Gli standard di programmazione sono specificati nel documento di 
\emph{Norme\_{}di\_{}Progetto\_{}v\versioneNormeDiProgetto{}.pdf}, nella sezione 4.3.5.	%Cambia anche qui, con la sezione giusta.
\subsubsection{Tipi in JavaScript}
Essendo JavaScript un linguaggio non tipato, una variabile può assumere diversi tipi durante l'esecuzione
di un codice. Infatti è possibile cambiare il tipo di un riferimento semplicemente assegnando un valore diverso ad
una data variabile. \`{E} quindi ambiguo definire all'interno di un frammento di codice il tipo statico che una variabile
possiede. Lo stesso vale per i valori ritornati da una funzione JavaScript, i quali possono essere addirittura funzioni.



%Prefaction: ricordarsi di mettere negli Standard di Programmazione che, per Angular, le variabili nelle
%QueryString sono precedute da i ":".


\subsection{Strumenti di lavoro}
Gli strumenti di lavoro sono trattati nelle sezioni 4.3.5.2.2 e 4.3.5.3.2 del documento di \\ \emph{Norme\_{}di\_{}Progetto\_{}v\versioneNormeDiProgetto{}.pdf}.


\newpage


\section{Namespace MaaP}
%qui ci va la figura della vista generale dei package
Namespace generale del progetto. In accordo con il design pattern architetturale Client-Server che 
abbiamo adottato, le interazioni che il Client ha con il Server sono di tipo REST-like.

\subsection{::Installer}
Package che gestisce l'installazione del framework MaaP. 


\subsubsection{@maaperture}

\begin{description}
 \item[Descrizione] \hfill \\
Modulo di installazione del framework Maap. Si occupa di gestire le richiesti provenienti da linea di comando gestendo i vari comandi.
 \item[Dipendendenze] \hfill
 \begin{itemize}
 	\item fs;
 	\item commander;
 	\item ncp.
 \end{itemize}
 \item[Metodi]
  \begin{mldescription}
   \mlitem{printHelpTitle()} \hfill
   \begin{description}
    \item[Descrizione] \hfill \\
    Si occupa di stampare nel log il nome, la versione e la descrizione del framework Maap.
   \end{description}
  
   \mlitem{changeFileRow(filePath, string2find, newString)} \hfill
      \begin{description}
      	    \item[Parametri] \hfill 
      	     \begin{description}
      	      \item[filePath] \hfill \\
      	      Path del file da modificare.
      	      \item[string2find] \hfill \\
      	      Stringa con la quale inizia la riga da modificare.
      	      \item[newString] \hfill \\
      	      Stringa con i nuovi caratteri da aggiungere.
      	     \end{description}
      	    
       \item[Descrizione] \hfill \\
      Si occupa di cambiare una riga del file specificato sostituendola con quella passata in ingresso.
      \end{description}

   \mlitem{setProjectName(destination, project\_name)} \hfill
   \begin{description}
   \item[Parametri] \hfill 
         	     \begin{description}
         	      \item[destination] \hfill \\
         	      Path di dove sarà situato il progetto.
         	      \item[project\_name] \hfill \\
         	      Nome del progetto.
         	      
         	     \end{description}
    \item[Descrizione] \hfill \\
    Si occupa di sostituire nei file di configurazione del progetto il nome del progetto scelto.
   \end{description}

   \mlitem{initProject(project\_name, output\_path)} \hfill
   \begin{description}
  
    \item[Descrizione] \hfill \\
	    Si occupa di copiare ricorsivamente i file di default del progetto Maap all'interno del percorso specificato in output\_path settando il nome di progetto specificato con project\_name.
   \end{description}
   \end{mldescription}
   \item[Comandi]
     \begin{mldescription}
      \mlitem{maaperture} \hfill
      \begin{description}
       \item[Descrizione] \hfill \\
       Si occupa di stampare la versione del framework, e un piccolo help sull'uso del framework.
      \end{description}
      \mlitem{create} \hfill
            \begin{description}
            \item[Parametri] \hfill 
                         	     \begin{description}
                         	      \item[name] \hfill \\
                         	      Nome del progetto.
                         	      \item[output] \hfill \\
                         	      Percorso di installazione del progetto.
                         	      
                         	     \end{description}
             \item[Descrizione] \hfill \\
             Si occupa di creare il progetto nella cartella specificate con il nome scelto
            \end{description}
      \mlitem{--help} \hfill
                  \begin{description}
                   \item[Descrizione] \hfill \\
                   Si occupa di stampare degli esempi d'uso dei comandi.
                  \end{description}
      \mlitem{*} \hfill
      \begin{description}
      \item[Descrizione] \hfill \\
      Si occupa di invocare printHelpTitle().
      \end{description}
      \end{mldescription}
\end{description}

\section{Specifica componenti MaaP::Server}

Package Server del Design Pattern Client-Server. 

\subsection{::Controller}
Package che gestisce il dialogo con il Client di MaaP. 

\subsubsection{@dispatcher}

\begin{description}
 \item[Descrizione] \hfill \\
Modulo di inizializzazione della componente Server utilizzata per la trasmissione dei dati al Client. \\
Il modulo \grassetto{@dispatcher} gestisce tutte le richieste effettuate dal Client al Server, effettuando le dovute redirezioni e le chiamate per ottenere i dati dai database nel Server. Le richieste che pervengono al 
\grassetto{@dispatcher} sono di tipo REST, a cui il modulo stesso risponde con il JSON della risorsa richiesta. 
Questo avviene solamente se l'utente che ha richiesto la risorsa è registrato ed autenticato, prima di 
effettuare la richiesta, e che abbia i dovuti permessi per effettuare la richiesta. Infatti, prima di eseguire ogni richiesta di accesso ai dati del Server, il \grassetto{@dispatcher} esegue una chiamata a \grassetto{@passport} per verificare l'autenticazione della richiesta HTTP e per distinguere gli utenti standard dagli utenti amministratori. \\
Per ultima cosa, il \grassetto{@dispatcher} gestisce eventuali richieste di risorse inesistenti o errate reindirizzando il chiamante alla pagina iniziale del Client AngularJS.
 \item[Dipendendenze] \hfill
 \begin{itemize}
 \item @passport;
 \item path;
 \item @DatabaseAnalysisManager;
 \item @DatabaseUserManager;
 \item @IndexManager.
 \end{itemize}
 
 \item[Metodi] \hfill
 \begin{description}
 \item[Gestione Collection e Document]
 \begin{mldescription}
  \mlitem{get('/api/collection/list', passport.checkAuthenticated, \\datamanager.sendCollectionsList)} \hfill
  \begin{description}
   \item[Parametri] \hfill 
    \begin{description}
     \item['/api/collection/list'] \hfill \\
     Stringa contenente l'indirizzo della API da richiamare.
     \item[passport.checkAuthenticated] \hfill \\
     Funzione di \grassetto{@passport} che verifica che l'utente sia autenticato.
     \item[datamanager.sendCollectionsList] \hfill \\
     Callback da richiamare al termine dell'autenticazione.
    \end{description}
   \item[Descrizione] \hfill \\
   Intercetta la richiesta di generare l'elenco delle Collection per visualizzarlo, verificando che essa sia autenticata e richiama il metodo \textit{sendCollectionsList} del \grassetto{@datamanager}.
  \end{description}
  
  \mlitem{get('/api/collection/:col\_id', passport.checkAuthenticated, \\datamanager.sendCollection)}\hfill 
  \begin{description}
   \item[Parametri] \hfill 
    \begin{description}
     \item['/api/collection/:col\_id'] \hfill \\
     Stringa contenente l'indirizzo della API da richiamare.
     \item[passport.checkAuthenticated] \hfill \\
     Funzione di \grassetto{@passport} che verifica che l'utente sia autenticato.
     \item[datamanager.sendCollection] \hfill \\
     Callback da richiamare al termine dell'autenticazione.
    \end{description}
   \item[Descrizione] \hfill \\
  Intercetta la richiesta di generare l'elenco dei Document all'interno di una Collection specificata con \textit{:col\_id}. Verifica che essa sia autenticata e richiama il metodo \textit{sendDocument} del \grassetto{@datamanager}.
  \end{description}
  
  \mlitem{get('/api/collection/:col\_id/:doc\_id', passport.checkAuthenticated,\\datamanager.sendDocument)} \hfill 
  \begin{description}
   \item[Parametri] \hfill 
    \begin{description}
     \item['/api/collection/:col\_id/:doc\_id'] \hfill \\
     Stringa contenente l'indirizzo della API da richiamare.
     \item[passport.checkAuthenticated] \hfill \\
     Funzione di \grassetto{@passport} che verifica che l'utente sia autenticato.
     \item[datamanager.sendDocument] \hfill \\
     Callback da richiamare al termine dell'autenticazione.
     \end{description}
   \item[Descrizione] \hfill \\ 
  Intercetta la richiesta di generare i dati di un Document specificato con \textit{:doc\_id} all'interno della Collection specificata con :col\_id per visualizzarli. Verifica che essa sia autenticata e richiama il metodo \textit{sendDocument} del \grassetto{@datamanager}.
  \end{description}
  
  \mlitem{get('/api/collection/:col\_id/:doc\_id/edit', passport.checkAuthenticatedAdmin, \\datamanager.sendDocumentEdit)} \hfill 
  \begin{description}
   \item[Parametri] \hfill 
    \begin{description}
     \item['/api/collection/:col\_id/:doc\_id/edit'] \hfill \\
     Stringa contenente l'indirizzo della API da richiamare.
     \item[passport.checkAuthenticatedAdmin] \hfill \\
     Funzione di \grassetto{@passport} che verifica che l'utente sia autenticato e che il suo livello di accesso sia "Amministratore".
     \item[datamanager.sendDocumentEdit] \hfill \\
     Callback da richiamare al termine dell'autenticazione.
     \end{description}
   \item[Descrizione] \hfill \\ 
  Intercetta la richiesta di generare i dati di un Document specificato con \textit{:doc\_id} all'interno della Collection specificata con :col\_id per modificarli. Verifica che essa sia autenticata e richiama il metodo \textit{sendDocumentEdit} del \grassetto{@datamanager}.
  \end{description}
  
  \mlitem{put('/api/collection/:col\_id/:doc\_id/edit', passport.checkAuthenticatedAdmin, \\datamanager.updateDocument)} \hfill 
  \begin{description}
   \item[Parametri] \hfill 
    \begin{description}
     \item['/api/collection/:col\_id/:doc\_id/edit'] \hfill \\
     Stringa contenente l'indirizzo della API da richiamare.
     \item[passport.checkAuthenticatedAdmin] \hfill \\
     Funzione di \grassetto{@passport} che verifica che l'utente sia autenticato e che il suo livello di accesso sia "Amministratore".
     \item[datamanager.updateDocument] \hfill \\
     Callback da richiamare al termine dell'autenticazione.
     \end{description}
   \item[Descrizione] \hfill \\ 
  Intercetta la richiesta di modificare i dati di un Document specificato con \textit{:doc\_id} all'interno della Collection specificata con :col\_id. Verifica che essa sia autenticata come amministratore e richiama il metodo \textit{updateDocument} del \grassetto{@datamanager}. I dati associati alla richiesta \textit{PUT} contengono le informazioni da modificare nel Database del Document con id \textit{doc\_id} della Collection \textit{col\_id}. 
  \end{description}
  
  \mlitem{delete('/api/collection/:col\_id/:doc\_id/edit', passport.checkAuthenticatedAdmin, \\datamanager.removeDocument)} \hfill 
  \begin{description}
   \item[Parametri] \hfill 
    \begin{description}
     \item['/api/collection/:col\_id/:doc\_id/edit'] \hfill \\
     Stringa contenente l'indirizzo della API da richiamare.
     \item[passport.checkAuthenticatedAdmin] \hfill \\
     Funzione di \grassetto{@passport} che verifica che l'utente sia autenticato e che il suo livello di accesso sia "Amministratore".
     \item[datamanager.removeDocument] \hfill \\
     Callback da richiamare al termine dell'autenticazione.
     \end{description}
   \item[Descrizione] \hfill \\ 
  Intercetta la richiesta di cancellare un Document specificato con \textit{:doc\_id} all'interno della Collection specificata con :col\_id. Verifica che essa sia autenticata come amministratore e richiama il metodo \textit{removeDocument} del \grassetto{@datamanager}.
  \end{description}
 
 \end{mldescription}
 
 \item[Gestione Profilo Utente]
 \begin{mldescription}
   \mlitem{get('/api/profile', passport.checkAuthenticated, \\usermanager.sendUserProfile)}\hfill 
   \begin{description}
    \item[Parametri] \hfill
     \begin{description}
      \item['/api/profile'] \hfill \\
      Stringa contenente l'indirizzo della API da richiamare.
      \item[passport.checkAuthenticated] \hfill \\
      Funzione di \grassetto{@passport} che verifica che l'utente sia autenticato.
      \item[usermanager.sendUserProfile] \hfill \\
      Callback da richiamare al termine dell'autenticazione.
     \end{description}
    \item[Descrizione] \hfill \\
    Intercetta la richiesta di generare i dati del profilo utente per visualizzarli, verificando che essa sia autenticata e richiama il metodo \textit{sendUserProfile} del \grassetto{@usermanager}.
    \end{description}
    
   \mlitem{get('/api/profile/edit', passport.checkAuthenticated, \\usermanager.sendUserProfileEdit)} \hfill 
   \begin{description}
    \item[Parametri] \hfill
     \begin{description}
      \item['/api/profile/edit'] \hfill \\
      Stringa contenente l'indirizzo della API da richiamare.
      \item[passport.checkAuthenticated] \hfill \\
      Funzione di \grassetto{@passport} che verifica che l'utente sia autenticato.
      \item[usermanager.sendUserProfileEdit] \hfill \\
      Callback da richiamare al termine dell'autenticazione.
     \end{description}
    \item[Descrizione] \hfill \\
   Intercetta la richiesta di generare i dati del profilo utente per modificarli, verificando che essa sia autenticata e richiama il metodo \textit{sendUserProfileEdit} del \grassetto{@usermanager}.
   \end{description} 
   
   \mlitem{put('/api/profile/edit', passport.checkAuthenticated, \\usermanager.updateUserProfile)} \hfill 
   \begin{description}
    \item[Parametri] \hfill
     \begin{description}
      \item['/api/profile/edit'] \hfill \\
      Stringa contenente l'indirizzo della API da richiamare.
      \item[passport.checkAuthenticated] \hfill \\
      Funzione di \grassetto{@passport} che verifica che l'utente sia autenticato.
      \item[usermanager.updateUserProfile] \hfill \\
      Callback da richiamare al termine dell'autenticazione.
     \end{description}
    \item[Descrizione] \hfill \\ 
   Intercetta la richiesta di modificare i dati del profilo utente, verificando che essa sia autenticata e richiama il metodo \textit{updateUserProfile} del \grassetto{@usermanager}. I dati associati alla richiesta \textit{PUT} contengono le informazioni da modificare nel Database dell'utente correntemente autenticato. 
   \end{description}
 
 \end{mldescription}  
 
 \item[Gestione Utenti] 
  \begin{mldescription}
    \mlitem{get('/api/users/list', passport.checkAuthenticatedAdmin,\\usermanager.getUsersList)} \hfill 
    \begin{description}
    \item[Parametri] \hfill
     \begin{description}
      \item['/api/users/list'] \hfill \\
      Stringa contenente l'indirizzo della API da richiamare.
      \item[passport.checkAuthenticatedAdmin] \hfill \\
      Funzione di \grassetto{@passport} che verifica che l'utente sia autenticato e che il suo livello di accesso sia "Amministratore".
      \item[usermanager.getUsersList] \hfill \\
      Callback da richiamare al termine dell'autenticazione.
     \end{description}
    \item[Descrizione] \hfill \\
    Intercetta la richiesta di generare l'elenco degli Utenti per visualizzarlo, verificando che essa sia autenticata come ammistratore e richiama il metodo \textit{getUsersListionsList} del \grassetto{@usermanager}.
    \end{description} 
    
    \mlitem{get('/api/users/:user\_id', passport.checkAuthenticatedAdmin,\\usermanager.sendUser)} \hfill
    \begin{description}
    \item[Parametri] \hfill
     \begin{description}
      \item['/api/users/:user\_id'] \hfill \\
      Stringa contenente l'indirizzo della API da richiamare.
      \item[passport.checkAuthenticatedAdmin] \hfill \\
      Funzione di \grassetto{@passport} che verifica che l'utente sia autenticato e che il suo livello di accesso sia "Amministratore".
      \item[usermanager.sendUser] \hfill \\
      Callback da richiamare al termine dell'autenticazione.
     \end{description}
    \item[Descrizione] \hfill \\
    Intercetta la richiesta di generare i dati di un Utente specificato con :user\_id per visualizzarli. Verifica che essa sia autenticata come amministratore e richiama il metodo \textit{sendUser} del \grassetto{@usermanager}.
    \end{description} 
    
    \mlitem{get('/api/users/:user\_id/edit', passport.checkAuthenticatedAdmin, \\usermanager.sendUserEdit)} \hfill 
    \begin{description}
    \item[Parametri] \hfill
     \begin{description}
      \item['/api/users/:user\_id/edit'] \hfill \\
      Stringa contenente l'indirizzo della API da richiamare.
      \item[passport.checkAuthenticatedAdmin] \hfill \\
      Funzione di \grassetto{@passport} che verifica che l'utente sia autenticato e che il suo livello di accesso sia "Amministratore".
      \item[usermanager.sendUserEdit] \hfill \\
      Callback da richiamare al termine dell'autenticazione.
     \end{description}
    \item[Descrizione] \hfill \\
    Intercetta la richiesta di generare i dati di un Utente specificato per con :user\_id per modificarli. Verifica che essa sia autenticata come amministratore e richiama il metodo \textit{sendUserEdit} del \grassetto{@usermanager}.
    \end{description} 
    
    \mlitem{put('/api/users/:user\_id/edit', passport.checkAuthenticatedAdmin,\\usermanager.updateUser)} \hfill 
    \begin{description}
    \item[Parametri] \hfill
     \begin{description}
      \item['/api/users/:user\_id/edit'] \hfill \\
      Stringa contenente l'indirizzo della API da richiamare.
      \item[passport.checkAuthenticatedAdmin] \hfill \\
      Funzione di \grassetto{@passport} che verifica che l'utente sia autenticato e che il suo livello di accesso sia "Amministratore".
      \item[usermanager.updateUser] \hfill \\
      Callback da richiamare al termine dell'autenticazione.
     \end{description}
    \item[Descrizione] \hfill \\
    Intercetta la richiesta di modificare i dati di un Utente specificato per con :user\_id. Verifica che essa sia autenticata come amministratore e richiama il metodo \textit{updateUser} dello \grassetto{@usermanager}. I dati associati alla richiesta \textit{PUT} contengono le informazioni da modificare nel Database degli utenti con id \textit{user\_}. 
    \end{description}
    
    \mlitem{delete('/api/users/:user\_id/edit', passport.checkAuthenticatedAdmin,\\usermanager.removeUser)} \hfill 
    \begin{description}
    \item[Parametri] \hfill
     \begin{description}
      \item['/api/users/:user\_id/edit'] \hfill \\
      Stringa contenente l'indirizzo della API da richiamare.
      \item[passport.checkAuthenticatedAdmin] \hfill \\
      Funzione di \grassetto{@passport} che verifica che l'utente sia autenticato e che il suo livello di accesso sia "Amministratore".
      \item[usermanager.removeUser] \hfill \\
      Callback da richiamare al termine dell'autenticazione.
     \end{description}
    \item[Descrizione] \hfill \\
    Intercetta la richiesta di eliminare un Utente specificato per con :user\_id. Verifica che essa sia autenticata come amministratore e richiama il metodo \textit{removeUser} del \grassetto{@usermanager}. 
  \end{description}
  
  \end{mldescription}    
 
 \item[Gestione Query più utilizzate]
 \begin{mldescription}
    \mlitem{get('/api/queries/list', passport.checkAuthenticatedAdmin,\\datamanager.getTopQueries)} \hfill 
    \begin{description}
    \item[Parametri] \hfill
     \begin{description}
      \item['/api/queries/list'] \hfill \\
      Stringa contenente l'indirizzo della API da richiamare.
      \item[passport.checkAuthenticatedAdmin] \hfill \\
      Funzione di \grassetto{@passport} che verifica che l'utente sia autenticato e che il suo livello di accesso sia "Amministratore".
      \item[datamanager.getTopQueries] \hfill \\
      Callback da richiamare al termine dell'autenticazione.
     \end{description}
    \item[Descrizione] \hfill \\
    Intercetta la richiesta di generare l'elenco delle query più utilizzate. Verifica che essa sia autenticata come amministratore e richiama il metodo \textit{getTopQueries} del \grassetto{@datamanager}.
    \end{description}
    
    \mlitem{delete('/api/queries/list', passport.checkAuthenticatedAdmin,\\datamanager.resetQueries)} \hfill 
    \begin{description}
    \item[Parametri] \hfill
     \begin{description}
      \item['/api/queries/list'] \hfill \\
      Stringa contenente l'indirizzo della API da richiamare.
      \item[passport.checkAuthenticatedAdmin] \hfill \\
      Funzione di \grassetto{@passport} che verifica che l'utente sia autenticato e che il suo livello di accesso sia "Amministratore".
      \item[datamanager.resetQueries] \hfill \\
      Callback da richiamare al termine dell'autenticazione.
     \end{description}
    \item[Descrizione] \hfill \\
    Intercetta la richiesta di eliminare tutte le query più utilizzate. Verifica che essa sia autenticata come amministratore e richiama il metodo \textit{resetQueries} del \grassetto{@datamanager}.
    \end{description}
 
 \end{mldescription}  
 
 \item[Gestione Indici nel Database di analisi]
  \begin{mldescription}
    \mlitem{get('/api/indexes', passport.checkAuthenticatedAdmin,\\datamanager.getIndexesList)} \hfill 
    \begin{description}
    \item[Parametri] \hfill
     \begin{description}
      \item['/api/indexes'] \hfill \\
      Stringa contenente l'indirizzo della API da richiamare.
      \item[passport.checkAuthenticatedAdmin] \hfill \\
      Funzione di \grassetto{@passport} che verifica che l'utente sia autenticato e che il suo livello di accesso sia "Amministratore".
      \item[datamanager.getIndexesList] \hfill \\
      Callback da richiamare al termine dell'autenticazione.
     \end{description}
    \item[Descrizione] \hfill \\
    Intercetta la richiesta di generare l'elenco degli indici esistenti sul database di analisi. Verifica che essa sia autenticata come amministratore e richiama il metodo \textit{getIndexesList} del \grassetto{@datamanager}.
    \end{description}
    
    \mlitem{put('/api/indexes', passport.checkAuthenticatedAdmin,\\datamanager.createIndex)} \hfill 
    \begin{description}
    \item[Parametri] \hfill
     \begin{description}
      \item['/api/indexes'] \hfill \\
      Stringa contenente l'indirizzo della API da richiamare.
      \item[passport.checkAuthenticatedAdmin] \hfill \\
      Funzione di \grassetto{@passport} che verifica che l'utente sia autenticato e che il suo livello di accesso sia "Amministratore".
      \item[datamanager.createIndex] \hfill \\
      Callback da richiamare al termine dell'autenticazione.
     \end{description}
    \item[Descrizione] \hfill \\
    Intercetta la richiesta di inserire un indice nel database di analisi. Verifica che essa sia autenticata come amministratore e richiama il metodo \textit{createIndex} del \grassetto{@datamanager}. I dati associati alla richiesta \textit{PUT} contengono le informazioni del nuovo indice da inserire nel Database nella sezione degli indici. 
    \end{description}
    
    \mlitem{delete('/api/indexes/:index\_name', passport.checkAuthenticatedAdmin,\\datamanager.deleteIndex)} \hfill 
 \begin{description}
    \item[Parametri] \hfill
     \begin{description}
      \item['/api/indexes/:index\_name'] \hfill \\
      Stringa contenente l'indirizzo della API da richiamare.
      \item[passport.checkAuthenticatedAdmin] \hfill \\
      Funzione di \grassetto{@passport} che verifica che l'utente sia autenticato e che il suo livello di accesso sia "Amministratore".
      \item[datamanager.deleteIndex] \hfill \\
      Callback da richiamare al termine dell'autenticazione.
     \end{description}
    \item[Descrizione] \hfill \\
    Intercetta la richiesta di cancellare un indice dal database di analisi. Verifica che essa sia autenticata come amministratore e richiama il metodo \textit{deleteIndex} del \grassetto{@datamanager}.
   \end{description}
 
 \end{mldescription} 
 
 \item[Gestione Login]
 \begin{mldescription}
 \mlitem{post('/api/check/email', passport.checkNotAuthenticated, usermanager.checkMail)}
 \begin{description}
    \item[Parametri] \hfill
     \begin{description}
      \item['/api/indexes/:index\_name'] \hfill \\
      Stringa contenente l'indirizzo della API da richiamare.
      \item[passport.checkNotAuthenticated]
      Funzione di \grassetto{@passport} che verifica che l'utente non sia autenticato.
      \item[usermanager.checkMail] \hfill \\
      Callback da richiamare al termine dell'autenticazione.
     \end{description}
    \item[Descrizione]
    Gestisce la richiesta di controllo della presenza di una particolare email già registrata al sistema. La condizione per richiedere questa informazione è che l'utente non sia autenticato, da qui la chiamata alla funzione \textit{passport.checkNotAuthenticated}. Nel caso la chiamata alla precedente funzione abbia successo, viene richiamato il metodo \textit{checkMail} di \grassetto{@DatabaseAnalysisManager}.
 \end{description}
 
 \mlitem{post('/api/signup', passport.checkNotAuthenticated, usermanager.userSignup)}
 \begin{description}
    \item[Parametri] \hfill
     \begin{description}
      \item['/api/signup'] \hfill \\
      Stringa contenente l'indirizzo della API da richiamare.
      \item[passport.checkNotAuthenticated]
      Funzione di \grassetto{@passport} che verifica che l'utente non sia autenticato.
      \item[usermanager.userSignup] \hfill \\
      Callback da richiamare al termine dell'autenticazione.
     \end{description}
    \item[Descrizione]
    Gestisce la richiesta di registrazione da parte di un utente non autenticato. Se l'utente associato alla richiesta non è autenticato, viene richiamato il metodo \textit{userSignup} di \grassetto{@DatabaseAnalysisManager}.
 \end{description}
 
 \mlitem{get('/loggedin', function(req, res) \{ res.send(req.isAuthenticated() ? req.user : '0'); \})}
 \begin{description}
  \item[Parametri]
   \begin{mldescription}
    \mlitem{'/loggedin'} \hfill \\
    Stringa contenente l'indirizzo della API da richiamare.
    \mlitem{function(req, res) \{ res.send(req.isAuthenticated() ? req.user : '0'); \}} \hfill \\
    Funzione anonima che richiama il metodo isAuthenticated sulla richiesta HTTP. Se il metodo ritorna valore booleano \textbf{true}, viene ritornato al  il proprio campo \textit{user}, altrimenti ritorna 0.
   \end{mldescription}
  \item[Descrizione] \hfill \\
 Gestisce la richiesta di controllo dell'avvenuta autenticazione di un utente. Viene richiamato sulla richiesta HTTP il metodo locale \textit{isAuthenticated()}. Se l'utente è autenticato, gli vengono inviate le proprie informazioni associate, altrimenti viene ritornato 0.
 \end{description}
 
 \mlitem{post('/api/login', passport.checkNotAuthenticated, passport.authenticate, function(req, res) \{ res.send(req.user); \})}
 \begin{description}
  \item[Parametri]
   \begin{mldescription}
    \mlitem{'/api/login'} \hfill \\
    Stringa contenente l'indirizzo della API da richiamare.
    \mlitem{passport.checkNotAuthenticated} \hfill \\
    Funzione di \grassetto{@passport} che verifica che l'utente non sia autenticato.
    \mlitem{passport.authenticate} \hfill \\
    Funzione di \grassetto{@passport} che esegue l'autenticazione.
    \mlitem{function(req, res) \{ res.send(req.user); \}} \hfill \\
    Funzione anonima che manda alla richiesta HTTP la risposta contenente il suo campo \textit{user}.
   \end{mldescription}
  \item[Descrizione] \hfill \\
 Gestisce le richieste di autenticazione al Server. L'utente che richiede questa operazione non deve essere autenticato ma deve essere presente nel database degli utenti presente nel Server. Nel caso di risposta affermativa ad entrambe le precedenti condizioni, viene creata una nuova sessione per la richiesta HTTP.
 \end{description}
 
 \mlitem{get('/api/logout', passport.checkAuthenticated, function(req, res) \{ req.logout(); res.send(200); \})}
 \begin{description}
  \item[Parametri]
   \begin{mldescription}
    \mlitem{'/api/logout'} \hfill \\
    Stringa contenente l'indirizzo della API da richiamare.
    \mlitem{passport.checkAuthenticated} \hfill \\
    Funzione di \grassetto{@passport} che verifica che l'utente sia autenticato.
    \mlitem{function(req, res) \{ req.logout(); res.send(200); \}} \hfill \\
    Funzione anonima che effettua la disconnessione della richiesta HTTP \textit{req} e manda al Client il codice \textit{http} 200.
   \end{mldescription}
  \item[Descrizione] \hfill \\
 Gestisce le richieste di disconnessione dal sistema. Per effettuare questa operazione, l'utente deve essere autenticato. In caso affermativo, viene richiesto il \textit{logout()} della richiesta HTTP.
 \end{description}
 
 \end{mldescription}

\item[Gestione richieste illecite]
 \begin{mldescription}
 \mlitem{get('*', function(req, res) \{ res.sendfile(path.join(config.static\_assets.dir, 'index.html')); \})}
 \begin{description}
  \item[Parametri]
   \begin{mldescription}
    \mlitem{'*'} \hfill \\
    Stringa contentente l'indirizzo delle API da richiamare. In questo caso, l'idirizzo è una stringa qualsiasi.
    \mlitem{function(req, res) \{ res.sendfile(path.join(config.static\_assets.dir, 'index.html')); \}} \hfill \\
    Funzione anonima che reindirizza la richiesta HTTP alla pagina iniziale della view del Client.
   \end{mldescription}
  \item[Descrizione]
 La presente funzione gestisce ogni richiesta di pagina o file che il Server non possiede. L'argomento "*" della funzione \textit{get} consente di gestire tutte le richieste che non sono state servite dalle precedenti funzioni del \grassetto{@dispatcher}. Questa funzione viene richiamata soltanto quando non le funzioni precedenti non sono state chiamate.
 \end{description} 
 \end{mldescription}

\item[Inizializzazione] \hfill
 \begin{description}
  \item[dispatcherInit(app)] \hfill 
  \begin{description}
   \item[Parametro] \hfill 
    \begin{description}
     \item[app] \hfill \\
     Applicazione di \grassetto{Express}.
    \end{description}
   \item[Descrizione] \hfill \\
   Si occupa dell'inizializzazione del \grassetto{@dispatcher}. Viene esportato come \textit{init()}.
  \end{description}
 \end{description}
\end{description}
\end{description}



\subsubsection{@frontController}
\begin{description}
 \item[Descrizione] \hfill \\
 Il \grassetto{@frontController} è l'interfaccia di connessione tra il Server ed il Client. Si occupa dell'inizializzazione del \grassetto{@dispatcher} reindirizzando ad esso le richieste del Client.
 \item[Dipendendenze] \hfill
 \begin{itemize}
 \item @dispatcher.
 \end{itemize}
 \item[Metodi] \hfill
  \begin{description}
    \item[initFrontController(app)] \hfill 
    \begin{description}
       		\item[Parametro] \hfill
       			\begin{description}
       				\item[app] \hfill \\
       				Oggetto contenente l'applicazione di \grassetto{Express}.
       			\end{description}
       		\item[Descrizione] \hfill \\
       		\`{E} la funzione di inizializzazione del \grassetto{@frontController} ed indica di reindirizzare le richieste del Client al \grassetto{@dispatcher} per gestirle. Viene esportato come \textit{init()}.
   \end{description}
  \end{description}
\end{description}

\subsubsection{@index}
\begin{description}
 \item[Descrizione] \hfill \\
  Controller che si occupa dell'inizializzazione del \textit{@frontController}.
 \item[Dipendendenze] \hfill
 \begin{itemize}
 \item @frontController.
 \end{itemize}
 \item[Metodi] \hfill
  \begin{description}
    \item[init(app)]\hfill 
    \begin{description}
           		\item[Parametro] \hfill
           			\begin{description}
           				\item[app] \hfill \\
           				Oggetto contenente l'applicazione di \grassetto{Express}.
           			\end{description}
           		\item[Descrizione] \hfill \\
           		\`{E} la funzione di inizializzazione dell'\grassetto{@index}, si occupa di inizializzare il \grassetto{@frontController}.
       \end{description}
  	
  \end{description}
\end{description}

\subsubsection{@passport}
\begin{description}
 \item[Descrizione] \hfill \\
Controller che si occupa dell'inizializzazione e gestione del modulo passort per le autenticazioni.
\item[Dipendendenze] \hfill
  \begin{itemize}
   \item passport;
   \item passport-local.Strategy;
   \item @MongooseDBFramework.
  \end{itemize}
   \item[Metodi] \hfill
    \begin{description}
     \item[authenticate()] \hfill 
      \begin{description}
        \item[Descrizione] \hfill \\
          Imposta il metodo di autenticazione, nel nostro caso come local.
      \end{description}
     
     \item[initPassport(app)] \hfill 
     \begin{description}
          \item[Parametro] \hfill
          \begin{description}
           \item[app] \hfill \\
           Oggetto contenente l'applicazione di \grassetto{Express}.
          \end{description}
          \item[Descrizione] \hfill \\
          Inizializza il controller \grassetto{@passport}, inizializzando e avviando la sessione del modulo \grassetto{@passport} e definisce le strategie di autenticazione.
            \end{description}
            
     \item[checkAuthenticatedAdmin(req, res, next)] \hfill 
          \begin{description}
            \item[Parametri] \hfill
            \begin{description}
             \item[req] \hfill \\
	         Oggetto contenente i parametri e le informazioni della richiesta HTTP del Client.
	         \item[res] \hfill \\
	         Oggetto duale a \textit{req} su cui invocare i metodi per inviare informazioni al Client richiedente.
             \item[next] \hfill \\
             Funzione da invocare nel caso di successo.
            \end{description}
            \item[Descrizione] \hfill \\
            Verifica che la richiesta proveniente dal client sia autenticata e che il livello dell'utente sia 1. 
         \end{description}
     \item[checkAuthenticated(req, res, next)] \hfill 
     \begin{description}
      \item[Parametri] \hfill
      \begin{description}
       \item[req] \hfill \\
	    Oggetto contenente i parametri e le informazioni della richiesta HTTP del Client.
	    \item[res] \hfill \\
	    Oggetto duale a \textit{req} su cui invocare i metodi per inviare informazioni al Client richiedente.
        La risposta al Client.
       \item[next] \hfill \\
        Funzione da invocare nel caso di successo.
      \end{description}
      \item[Descrizione] \hfill \\
      Verifica che la richiesta proveniente dal client sia autenticata.
     \end{description}
     
     \item[checkNotAuthenticated(req, res, next)] \hfill 
     \begin{description}
      \item[Parametri] \hfill
      \begin{description}
       \item[req] \hfill \\
	    Oggetto contenente i parametri e le informazioni della richiesta HTTP del Client.
	   \item[res] \hfill \\
	    Oggetto duale a \textit{req} su cui invocare i metodi per inviare informazioni al Client richiedente.
       \item[next] \hfill \\
        Funzione da invocare nel caso di successo.
      \end{description}
      \item[Descrizione] \hfill \\
      Verifica che la richiesta proveniente dal client non sia autenticata.
     \end{description}
  \end{description}
\end{description}

\subsection{::ModelServer}
Package che racchiude i database, le loro operazioni di accesso e i file DSL con i relativi interpreti.

\subsubsection{::Database}
\paragraph{@index}
\begin{description}
 \item[Descrizione] \hfill \\
 Si occupa dell'inizializzazione dei database degli utenti e di analisi. Crea l'amministratore di default di \textit{MaaP}.
 \item[Dipendendenze] \hfill
   \begin{itemize}
   \item mongoose;
   \item @MongooseDBAnalysis;
   \item @MongooseDBFramework;
   \item @DataRetrieverUsers.
   \end{itemize}
 \item[Metodi] \hfill
 \begin{description}
 \item[initDB(app)] \hfill 
 \begin{description}
    		\item[Parametro] \hfill
    			\begin{description}
    				\item[app] \hfill \\
    				Oggetto contenente l'applicazione di \grassetto{Express}.
    			\end{description}
    		\item[Descrizione] \hfill \\
    		Si occupa dell'inizializzazione dei database. Effettua la connessione al database utenti e a quello di analisi. Inizializza lo \grassetto{@userDBManager}, invoca \textit{addAdminDefault} e inizializza il \grassetto{@dataDBManager}. Viene esportata come \textit{init()}.
    	\end{description}
  \item[addAdminDefault(config, userDB))] \hfill 
  \begin{description}
      		\item[Parametri] \hfill
      			\begin{description}
      				\item[config] \hfill \\
      				Oggetto contenente l'applicazione di \grassetto{Express}.
      				\item[userDB] \hfill \\
      				Connessione a database degli utenti.
      			\end{description}
      		\item[Descrizione] \hfill \\
      		Si occupa dell'aggiunta dell'amministratore di default nel database degli utenti.
      	\end{description}
  
 \end{description}
\end{description}

\paragraph{@MongooseDBAnalysis}
\begin{description}
 \item[Descrizione] \hfill \\
 Si occupa dell'inizializzazione dei modelli di \grassetto{Mongoose}.
 \item[Dipendendenze] \hfill
    \begin{itemize}
    \item fs;
    \item path;
    \item mongoose.
    \end{itemize}
\item[Metodi] \hfill
 \begin{description}
 \item[init(app)] \hfill 
 \begin{description}
     		\item[Parametro] \hfill
     			\begin{description}
     				\item[app] \hfill \\
     				Oggetto contenente l'applicazione di \grassetto{Express}.
     			\end{description}
     		\item[Descrizione] \hfill \\
     		 Legge la cartella contenente le \textit{collectionData} dei DSL e per ogni file inizializza il modello di \grassetto{Mongoose} come elemento dell'Array \textit{modelArray} che poi esporta come model.
     	\end{description}

 \end{description}
\end{description}

\paragraph{@MongooseDBFramework}
\begin{description}
 \item[Descrizione] \hfill \\
 Si occupa dell'inizializzazione dei modelli di \grassetto{Mongoose}.
  \item[Dipendendenze] \hfill
  \begin{itemize}
  \item mongoose.
  \end{itemize}
\item[Metodi] \hfill
 \begin{description}
 \item[init(app)] \hfill 
\begin{description}
     		\item[Parametro] \hfill
     			\begin{description}
     				\item[app] \hfill \\
     				Oggetto contenente l'applicazione di \grassetto{Express}.
     			\end{description}
     		\item[Descrizione] \hfill \\
     		  Si occupa di creare gli schemi di \grassetto{Mongoose} per la Collection utenti e la Collection query, esporta tali modelli come users e query e la connessione al database come connection.
     	\end{description}
 \end{description}
\end{description}

\subsubsection{::DataManager}
Package che gestisce gli accessi ai database.

\paragraph{::DatabaseAnalysisManager} \hfill \\
Package che gestisce le query ai database di analisi. 

\subparagraph{@DatabaseAnalysisManager}
\begin{description}
 \item[Descrizione] \hfill \\
Modulo del ModelServer che si occupa di interpretare i dati richiesti e spediti da e al database di analisi. 
Prepara le query da effettuare e contatta il \grassetto{@DataRetrieverAnalysis} per ottenere i dati voluti.
 \item[Dipendendenze] \hfill
 \begin{itemize}
  \item path;
  \item @DataRetrieverAnalysis;
  \item @IndexManager;
  \item @JSonComposer.
 \end{itemize}
 
 %aggiungere tutti i req e res 
 
 \item[Metodi] \hfill
 \begin{description}
 \item[sendCollectionList(req, res)] \hfill 
 \begin{description}
 \item[Parametri] \hfill
  \begin{description}
   \item[req] \hfill \\
   Oggetto contenente i parametri e le informazioni della richiesta HTTP del Client.
   \item[res] \hfill \\
   Oggetto duale a \textit{req} su cui invocare i metodi per inviare informazioni al Client richiedente.
  \end{description}
 \item[Descrizione] \hfill \\
 Richiede al \grassetto{@DataRetrieverAnalysis} l'elenco delle Collection e richiama il metodo di creazione dell'oggetto Collection del \grassetto{@JSonComposer}. Infine ritorna l'oggetto CollectionList ottenuto.
 \end{description}
 
 \item[sendCollection(req, res)] \hfill 
 \begin{description}
 \item[Parametri] \hfill
  \begin{description}
   \item[req] \hfill \\
   Oggetto contenente i parametri e le informazioni della richiesta HTTP del Client.
   \item[res] \hfill \\
   Oggetto duale a \textit{req} su cui invocare i metodi per inviare informazioni al Client richiedente.
  \end{description}
 \item[Descrizione] \hfill \\

 Esegue una chiamata asincrona al \grassetto{@DataRetrieverAnalysis} per ottenere una particolare Collection. Nel caso la Collection esista, viene richiamato il metodo \textit{createCollection} di \grassetto{@JSonComposer} per inviare al richiedente l'oggetto Collection desiderato. In caso la Collection non esista nel Server, viene ritornato il codice di errore \textit{http} 404 al Client.
 \end{description}
 
 \item[sendDocument(req, res)] \hfill 
 \begin{description}
 \item[Parametri] \hfill
  \begin{description}
   \item[req] \hfill \\
   Oggetto contenente i parametri e le informazioni della richiesta HTTP del Client.
   \item[res] \hfill \\
   Oggetto duale a \textit{req} su cui invocare i metodi per inviare informazioni al Client richiedente.
  \end{description}
 \item[Descrizione] \hfill \\
 Esegue una chiamata asincrona al \grassetto{@DataRetrieverAnalysis} per ottenere un particolare Document. Nel caso il Document esista, viene richiamato il metodo \textit{createDocument} di \grassetto{@JSonComposer} per inviare al richiedente l'oggetto Document desiderato. In caso il Document non esista nel Server, viene ritornato il codice di errore \textit{http} 404 al Client.
 \end{description}
 
 \item[sendDocumentEdit(req, res)] \hfill 
 \begin{description}
 \item[Parametri] \hfill
  \begin{description}
   \item[req] \hfill \\
   Oggetto contenente i parametri e le informazioni della richiesta HTTP del Client.
   \item[res] \hfill \\
   Oggetto duale a \textit{req} su cui invocare i metodi per inviare informazioni al Client richiedente.
  \end{description}
 \item[Descrizione] \hfill \\
 Esegue una chiamata asincrona al \grassetto{@DataRetrieverAnalysis} per ottenere un particolare Document modificabile. Nel caso il Document esista, viene richiamato il metodo \textit{createDocument} di \grassetto{@JSonComposer} per inviare al richiedente l'oggetto Document desiderato. In caso il Document non esista nel Server, viene ritornato il codice di errore \textit{http} 404 al Client.
 \end{description}
 
 \item[updateDocument(req, res)] \hfill 
 \begin{description}
 \item[Parametri] \hfill
  \begin{description}
   \item[req] \hfill \\
   Oggetto contenente i parametri e le informazioni della richiesta HTTP del Client.
   \item[res] \hfill \\
   Oggetto duale a \textit{req} su cui invocare i metodi per inviare informazioni al Client richiedente.
  \end{description}
 \item[Descrizione] \hfill \\
 Esegue una chiamata asincrona al metodo \textit{updateDocument} di \grassetto{@DataRetrieverAnalysis} per modificare un Document esistente nel Server. Nel caso la modifica effettuata dal \grassetto{@DataRetrieverAnalysis} vada a buon fine, viene ritornato al Client un codice \textit{http} di successo "200", altrimenti viene mandato il codice di errore \textit{http} 404.
 \end{description}
 
 \item[removeDocument(req, res)] \hfill 
 \begin{description}
 \item[Parametri] \hfill
  \begin{description}
   \item[req] \hfill \\
   Oggetto contenente i parametri e le informazioni della richiesta HTTP del Client.
   \item[res] \hfill \\
   Oggetto duale a \textit{req} su cui invocare i metodi per inviare informazioni al Client richiedente.
  \end{description}
 \item[Descrizione] \hfill \\
 Esegue una chiamata asincrona al metodo \textit{removeDocument} di \grassetto{@DataRetrieverAnalysis} per modificare un Document esistente nel Server. Nel caso la rimozione effettuata dal \grassetto{@DataRetrieverAnalysis} vada a buon fine, viene ritornato al Client un codice \textit{http} di successo "200", altrimenti viene mandato il codice di errore \textit{http} 404.
 \end{description}
 
 \end{description}
 
\end{description} %forse è in più.


\subparagraph{@DataRetrieverAnalysis}
\begin{description}
\item[Descrizione] \hfill \\
Modulo che dialoga con il database di analisi per mezzo dell'infrastruttura \textit{Mongoose}. Utilizza i file derivati dall'interpretazione del 
linguaggio DSL per richiedere e inviare i dati che l'utente sviluppatore decide di visualizzare e modificare.
\item[Dipendenze] \hfill 
\begin{itemize}
\item @IndexManager.
\end{itemize}

\item[Metodi]
\begin{mldescription}
	\mlitem{getModel(collection\_name)} \hfill 
		\begin{description}
			\item[Parametro] \hfill
				\begin{description}
					\item[collection\_name] \hfill \\
					Stringa contenente il nome della Collection da ricercare nel database di analisi.
				\end{description}
			\item[Dipendenza] \hfill
				\begin{itemize}
					\item @MongooseDBAnalysis.
				\end{itemize}
			\item[Descrizione] \hfill \\
			Ricerca nei modelli del database di analisi la Collection di nome \textit{collection\_name}.
Se essa è presente, ne ritorna il modello, altrimenti ritorna l'intero "-1".
		\end{description}

	\mlitem{getDocuments(model, querySettings, populate, callback)} \hfill
		\begin{description}
			\item[Parametri] \hfill
			\begin{description}
				\item[model] \hfill \\
				Modello della Collection di cui importare i Document.
				\item[querySettings] \hfill \\
				Oggetto contenente i parametri della query di ricerca dei Document.
				\item[populate] \hfill \\
				Oggetto contenente le funzioni populate specificate nel DSL della Collection.
				\item[callback] \hfill \\
				Funzione da richiamare al termine della funzione \textit{getDocuments}.
			\end{description}
			\item[Descrizione] \hfill \\
			Viene effettuata una query sul modello \textit{model} passato come argomento alla funzione. Si utilizzano come opzioni di ricerca i campi dell'oggetto \textit{querySettings}. Viene in questo modo creato un oggetto locale che verrà passato all fine dell'esecuzione della funzione alla \textit{callback} come argomento. Se non pervengono risultati dalla query, l'oggetto passato sarà vuoto. Nel caso il risultato della query non sia nullo, viene poi effettuata l'operazione di popolamento specificata dal parametro \textit{populate}. Infine viene richiamata la \textit{callback} passando come argomento l'oggetto locale contenente il risulatato della query e del populate.
		\end{description}			
	 
	\mlitem{getCollectionsList()} \hfill 
		\begin{description}
			\item[Dipendenza] \hfill
				\begin{itemize}
					\item DSL/collectionData/collectionsList.json.
				\end{itemize}
			\item[Descrizione] \hfill \\
			Ritorna la lista delle Collection visibili dopo l'interpretazione del DSL. 
		\end{description}
		
	\mlitem{applyTransformations(type, documentsArray, dslArray)} \hfill 
		\begin{description}
			\item[Parametri] \hfill
				\begin{description}
					\item[type] \hfill \\
						Stringa contenente il tipo di visualizzazione dei Document presenti in 										\textit{documentsArray}.
					\item[documentsArray] \hfill \\
						Array di Document a cui applicare le trasformazioni.
					\item[dslArray] \hfill \\
						Array di DSL legati alla Collection contenente i Document di \textit{documentsArray}.
				\end{description}
			\item[Dipendenza] \hfill 
				\begin{itemize}
					\item file dinamico contenente le trasformazioni di una Collection.
				\end{itemize}
			\item[Descrizione] \hfill \\	
			Scorre l'Array di DSL fino a trovare una trasformazione da applicare a uno o più dati dei Document. Se esiste almeno una trasformazione da applicare, la applica ad ogni Document contenuto nel \textit{documentsArray}.							
		\end{description}
		
	\mlitem{sortDocumentsByLabels(documents, keys)} \hfill 
		\begin{description}
			\item[Parametri] \hfill
				\begin{description}
					\item[documents] \hfill \\
					Array di Document.
					\item[keys] \hfill \\
					Array di chiavi di un Document.
				\end{description}
			\item[Descrizione] \hfill \\
			Ordina le coppie chiave-valore dei Document presenti nell'Array \textit{documents} secondo 
			l'ordine stabilito nell'array di chiavi \textit{keys}.
		\end{description}			
	
	
	
	
	
	%AGGIUNGI COUNTDOCUMENTS	
	
	
	
	% E FINDDOCUMENTS
	
	
	
	% PIU' GETDOCUMENTSFORINDEX
	
	
	
	
	
	\mlitem{getCollectionIndex(collection\_name, column, order, page, callback)} \hfill 
		\begin{description}
			\item[Parametri] \hfill
				\begin{description}
					\item[collection\_name] \hfill \\
					Stringa contenente il nome della Collection.
					\item[column] \hfill \\
						Stringa contenente il nome della colonna da utilizzare per ordinare i Document.
					\item[order] \hfill \\
						Stringa contenente l'ordinamento da applicare ai Document, ascendente o discendente.
					\item[page] \hfill \\
						Numero intero che definisce il numero della pagina di Document da visualizzare.
					\item[callback] \hfill \\
					Funzione da richiamare al termine della funzione \textit{getCollectionIndex}.
				\end{description}
			\item[Dipendenza] \hfill
				\begin{itemize}
					\item Oggetto dinamico derivante dall'analisi del file DSL associato alla Collection di nome \textit{collection\_name}.
				\end{itemize}
			\item[Descrizione] \hfill \\
Viene richiesto il modello della Collection di nome \textit{collection\_name}. Successivamente vengono costruiti gli Array delle etichette e dei dati della Collection. Il nome delle varie etichette viene stabilito dal file DSL associato alla Collection, altrimenti viene preso il nome di default della colonna. A questo punto viene eseguito un controllo per stabilire se il campo {\_id} dei vari Document della Collection deve essere visibile nella visualizzazione. Viene eseguita una query nel database di analisi per ottenere tutti i Document che rispettano i parametri. Viene poi eseguita un'ulteriore query per creare localmente l'oggetto Collection con il numero di pagine specificato nel file DSL della Collection e per ottenere i Document identificati dalla pagina di visualizzazione richiesta dal Client. Se le query hanno portato ad un risultato positivo, viene richiamata la funzione \textit{callback} con argomento il JSON creato localmente. Altrimenti la \textit{callback} viene richiamata con argomento vuoto.
			
		\end{description}
	
	\mlitem{getDocumentShow(collection\_name, document\_id, callback)} \hfill
		\begin{description}
			\item[Parametri] \hfill
				\begin{description}
					\item[collection\_name] \hfill \\
						Stringa contenente il nome della Collection di cui fa parte il Document identificato 							con \textit{document\_id}.
					\item[document\_id] \hfill \\
						Id che identifica un singolo Document all'interno della Collection di nome 									\textit{collection\_name}.
					\item[callback] \hfill \\
						Funzione da richiamare al termine della funzione \textit{getDocumentShow}.
				\end{description}
			\item[Dipendenza] \hfill
				\begin{itemize}
					\item JSON generato dall'inerpretazione del DSL della Collection di nome 										\textit{collection\_name}.
				\end{itemize}
			\item[Descrizione] \hfill \\
			Richiede il modello della Collection \textit{collection\_name} e il file DSL associato. Viene controllato se l'output del DSL applica delle modifiche alla visualizzazione standard. Se ciò avviene, vengono impostate le etichette del JSON da inviare al Client come specificato nel file DSL. Inoltre vengono interpretate eventuali istruzioni di \textit{populate} da applicare prima di effettuare la query. Infine, nel caso la query abbia successo, viene richiamata la funzione \textit{callback} con argomento il JSON ottenuto dalla query. Se invece la query non è andata a buon fine, viene chiamata la \textit{callback} con argomento vuoto.
		\end{description}
	
	\mlitem{getDocumentShowEdit(collection\_name, document\_id, callback)} \hfill
		\begin{description}
			\item[Parametri] \hfill
				\begin{description}
					\item[collection\_name] \hfill \\
						Stringa contenente il nome della Collection di cui fa parte il Document identificato 							con \textit{document\_id}.
					\item[document\_id] \hfill \\
						Id che identifica un singolo Document all'interno della Collection di nome 									\textit{collection\_name}.
					\item[callback] \hfill \\
						Funzione da richiamare al termine della funzione \textit{getDocumentShowEdit}.
				\end{description}
			\item[Dipendenza] \hfill
				\begin{itemize}
					\item JSON generato dall'inerpretazione del DSL della Collection di nome 										\textit{collection\_name}.
				\end{itemize}
			\item[Descrizione] \hfill \\
			Richiede il modello della Collection \textit{collection\_name} e il file DSL associato. Viene controllato se l'output del DSL applica delle modifiche alla visualizzazione standard. Se ciò avviene, vengono impostate le etichette del JSON da inviare al Client come specificato nel file DSL. I campi compositi del database di analisi vengono processati e viene inviato solamente il campo \textit{id} dei campi compositi. Inoltre vengono interpretate eventuali istruzioni di \textit{populate} da applicare prima di effettuare la query. Infine, nel caso la query abbia successo, viene richiamata la funzione \textit{callback} con argomento il JSON ottenuto dalla query. Se invece la query non è andata a buon fine, viene chiamata la \textit{callback} con argomento vuoto.
		\end{description}
	
	\mlitem{updateDocument(collection\_name, document\_id, newDocumentData, callback)}
		\begin{description}
			\item[Parametri] \hfill
				\begin{description}
					\item[collection\_name] \hfill \\
					Stringa contenente il nome della Collection contenente il Document da modificare.
					\item[document\_id] \hfill \\
					Id del Document da modificare.
					\item[newDocumentData] \hfill \\
					JSON contenente i campi da modificare nel Document desiderato.					
					\item[callback] \hfill \\
					Funzione da richiamare al termine della modifica del Document.
				\end{description}
			\item[Descrizione] \hfill \\
			Richiede il modello della Collection nominata \textit{collection\_name}. Successivamente richiama il metodo \textit{update} sul modello della Collection scelta per aggiornare il Document identificato con \textit{document\_id} con le nuove informazioni di \textit{newDocumentData}. Al termine di questa operazione, viene chiamata la \textit{callback}.
		\end{description}
	
	\mlitem{removeDocument(collection\_name, document\_id, callback)} \hfill 
		\begin{description}
			\item[Parametri] \hfill
				\begin{description}
					\item[collection\_name] \hfill \\
					Stringa contenente il nome della Collection contenente il Document da eliminare.
					\item[document\_id] \hfill \\
					Id del Document da eliminare.
					\item[callback] \hfill \\
					Funzione da richiamare al termine dell'eliminazione del Document.
				\end{description}
			\item[Descrizione] \hfill \\
Richiede il modello della Collection nominata \textit{collection\_name}. Successivamente richiama
il metodo \textit{remove} sul modello della Collection scelta per rimuovere il Document identificato con \textit{document\_id}. Al termine di questa 	operazione, viene chiamata la \textit{callback}.
		\end{description}
		
\end{mldescription}

\end{description}

\paragraph{::DatabaseUserManager} \hfill \\
Package che gestisce gli accessi al database utenti.

\subparagraph{@DatabaseUserManager}
\begin{description}
 \item[Descrizione] \hfill \\
Modulo del ModelServer che si occupa di interpretare i dati richiesti e spediti da e al database utenti. 
Prepara le query da effettuare e contatta il \grassetto{@DataRetrieverUsers} per ottenere i dati voluti. 
Inoltre dialoga direttamente con \grassetto{@MongooseDBFramework} per l'esecuzione di test durante la registrazione. 
 \item[Dipendendenze] \hfill
 \begin{itemize}
  \item path;
  \item @DataRetrieverUsers;
  \item @MongooseDBFramework;
  \item @JSonComposer.
 \end{itemize}
  
 \item[Metodi] \hfill
  \begin{description}
    \item[checkMail(req, res)] \hfill 
      \begin{description}
	   \item[Parametri] \hfill
	  \begin{description}
	    \item[req] \hfill \\
	    Oggetto contenente i parametri e le informazioni della richiesta HTTP del Client.
	    \item[res] \hfill \\
	    Oggetto duale a \textit{req} su cui invocare i metodi per inviare informazioni al Client richiedente.
	  \end{description}
	\item[Descrizione] \hfill \\
	Effettua un controllo di presenza nel database utenti della email specificata dal Client. In caso di presenza, viene inviato al client un errore \textit{http} 400. In caso contrario, viene segnalato all'utente la mancanza 	di un indirizzo email già registrato e viene inviato il codice \textit{http} 304.
      \end{description}
    \item[userSignup(req, res)] \hfill
      \begin{description}
	\item[Parametri] \hfill
	  \begin{description}
	    \item[req] \hfill \\
	    Oggetto contenente i parametri e le informazioni della richiesta HTTP del Client.
	    \item[res] \hfill \\
	    Oggetto duale a \textit{req} su cui invocare i metodi per inviare informazioni al Client richiedente.
	  \end{description}
	\item[Descrizione] \hfill \\
	Richiama il metodo \textit{addUser} di \grassetto{@DataRetrieverUsers} per registrare un nuovo utente al sistema. Vengono passati al metodo \textit{addUser} le credenziali contenute nell'oggetto \textit{req}. In caso di avvenuto inserimento, viene inviato al Client il codice di conferma \textit{http} 200. In caso contrario, viene inviato il 	codice di errore \textit{http} 404.
      \end{description}
    
    \item[sendUserProfile(req, res)] \hfill
      \begin{description}
	\item[Parametri] \hfill
	  \begin{description}
	    \item[req] \hfill \\
	    Oggetto contenente i parametri e le informazioni della richiesta HTTP del Client.
	    \item[res] \hfill \\
	    Oggetto duale a \textit{req} su cui invocare i metodi per inviare informazioni al Client richiedente.
	  \end{description}
	\item[Descrizione] \hfill \\
	Richiama il metodo \textit{getUserProfile} di \grassetto{@DataRetrieverUsers} per ottenere l'oggetto utente sulla base dei parametri inviati da \textit{req}. Successivamente invia a \textit{res} il JSON dell'utente 	creato tramite il metodo \textit{createUserProfile} di \grassetto{@JsonComposer}.
      \end{description}
      
    \item[sendUserProfileEdit(req, res)] \hfill
      \begin{description}
	\item[Parametri] \hfill
	  \begin{description}
	    \item[req] \hfill \\
	    Oggetto contenente i parametri e le informazioni della richiesta HTTP del Client.
	    \item[res] \hfill \\
	    Oggetto duale a \textit{req} su cui invocare i metodi per inviare informazioni al Client richiedente.
	  \end{description}
	\item[Descrizione]
	Richiama il metodo \textit{getUserProfile} di \grassetto{@DataRetrieverUsers} per ottenere l'oggetto utente sulla base dei parametri inviati da \textit{req}. Successivamente invia a \textit{res} il JSON dell'utente 	creato tramite il metodo \textit{createUserProfileEdit} di \grassetto{@JsonComposer}.
      \end{description}
      
    \item[updateUserProfile(req, res)] \hfill
      \begin{description}
	\item[Parametri] \hfill
	  \begin{description}
	    \item[req] \hfill \\
	    Oggetto contenente i parametri e le informazioni della richiesta HTTP del Client.
	    \item[res] \hfill \\
	    Oggetto duale a \textit{req} su cui invocare i metodi per inviare informazioni al Client richiedente.
	  \end{description}
	\item[Descrizione] \hfill \\
	Richiama il metodo \textit{updateUserProfile} di \grassetto{@DataRetrieverUsers} passando come parametro il richiedente \textit{req} del Client. Ritorna al Client un codice \textit{http} 200 in caso di riuscita, un codice 400 altrimenti.

      \end{description}
      
    \item[getUsersList(req, res)] \hfill
      \begin{description}
	\item[Parametri] \hfill
	  \begin{description}
	    \item[req] \hfill \\
	    Oggetto contenente i parametri e le informazioni della richiesta HTTP del Client.
	    \item[res] \hfill \\
	    Oggetto duale a \textit{req} su cui invocare i metodi per inviare informazioni al Client richiedente.
	  \end{description}
	\item[Descrizione] \hfill \\
	Richiama il metodo \textit{getUsersList} di \grassetto{@DataRetrieverUsers} utilizzando i parametri di visualizzazione di \textit{req} o assegnando dei valori di default. Ottenuti i dati, risponde a \textit{res} inviando il risultato della chiamata del metodo \textit{createUsersList} di \grassetto{@JsonComposer}.
      \end{description}
      
    \item[sendUser(req, res)] \hfill
      \begin{description}
	\item[Parametri] \hfill
	  \begin{description}
	    \item[req] \hfill \\
	    Oggetto contenente i parametri e le informazioni della richiesta HTTP del Client.
	    \item[res] \hfill \\
	    Oggetto duale a \textit{req} su cui invocare i metodi per inviare informazioni al Client richiedente.
	  \end{description}
	\item[Descrizione] \hfill \\
	Richiama il metodo \textit{getUserProfile} di \grassetto{@DataRetrieverUsers} passando come parametro il campo \textit{user\_id} di \textit{req}. Risponde a \textit{res} inviando il risultato della chiamata del metodo \textit{createUser} di \grassetto{@JsonComposer}.
      \end{description}
      
    \item[sendUserEdit(req, res)] \hfill
      \begin{description}
	\item[Parametri] \hfill
	  \begin{description}
	    \item[req] \hfill \\
	    Oggetto contenente i parametri e le informazioni della richiesta HTTP del Client.
	    \item[res] \hfill \\
	    Oggetto duale a \textit{req} su cui invocare i metodi per inviare informazioni al Client richiedente.
	  \end{description}
	\item[Descrizione] \hfill \\
	Richiama il metodo \textit{getUserProfile} di \grassetto{@DataRetrieverUsers} passando come parametro il campo \textit{user\_id} di \textit{req}. Risponde a \textit{res} inviando il risultato della chiamata del metodo \textit{createUser} di \grassetto{@JsonComposer}.
      \end{description}
	
    \item[updateUser(req, res)] \hfill
    \begin{description}
	\item[Parametri] \hfill
	  \begin{description}
	    \item[req] \hfill \\
	    Oggetto contenente i parametri e le informazioni della richiesta HTTP del Client.
	    \item[res] \hfill \\
	    Oggetto duale a \textit{req} su cui invocare i metodi per inviare informazioni al Client richiedente.
	  \end{description}
	\item[Descrizione] \hfill \\
	Richiama il metodo \textit{updateUser} di \grassetto{@DataRetrieverUsers} passando come parametro il richiedente \textit{req}. Ritorna al Client un codice \textit{http} 200 in caso di riuscita, un codice 400 altrimenti.

      \end{description}
      
    \item[removeUser(req, res)] \hfill
    \begin{description}
	\item[Parametri] \hfill
	  \begin{description}
	    \item[req] \hfill \\
	    Oggetto contenente i parametri e le informazioni della richiesta HTTP del Client.
	    \item[res] \hfill \\
	    Oggetto duale a \textit{req} su cui invocare i metodi per inviare informazioni al Client richiedente.
	  \end{description}
	\item[Descrizione] \hfill \\
	Richiama il metodo \textit{removeUser} di \grassetto{@DataRetrieverUsers} passando come parametro il campo \textit{email} del richiedente \textit{req}. Ritorna al Client un codice \textit{http} 200 in caso di riuscita, un codice 400 altrimenti.
      \end{description}
      
 \end{description}
 
\end{description}


\subparagraph{@DataRetrieverUsers}
\begin{description}
 \item[Descrizione] \hfill \\
 Modulo che dialoga direttamente con \grassetto{@MongooseDBFramework} per effettuare le query di visualizzazione e modifica dei dati contenuti nel database degli utenti.
 \item[Dipendenze] \hfill 
 	\begin{itemize}
 		\item @MongooseDBFramework.
 	\end{itemize}
 \item[Metodi]
 	\begin{mldescription}
 		\mlitem{addUser(email, password, level, callback)} \hfill
 			\begin{description}
 				\item[Parametri] \hfill
 					\begin{description}
 						\item[email] \hfill \\
 							Stringa contenente l'indirizzo email del nuovo utente.
 						\item[password] \hfill \\
 							Stringa contenente la password del nuovo utente.
 						\item[level] \hfill \\
 							Numero intero che identifica il livello di permesso di un utente. Il numero 0 identifica un utente standard, 1 individua un utente admin. 
 						\item[callback] \hfill \\
 							Funzione da richiamare alla fine dell'inserimento.
 					\end{description}
 				\item[Descrizione] \hfill \\
 				La funzione inserisce un nuovo utente nel database utenti del sistema. Alla fine 	dell'inserimento, se non sono sollevati errori richiama la funzione \textit{callback} assegnando come parametro \textbf{true}, altrimenti assegna il parametro \textbf{false}.
 			\end{description}
 			
 		\mlitem{getUserProfile(user\_id, callback)} \hfill
 			\begin{description}
 				\item[Parametri] \hfill
 					\begin{description}
 						\item[user\_id] \hfill \\
 							Id univoco di un utente registrato al sistema.
 						\item[callback] \hfill \\
 							Funzione da richiamare alla fine dell'ottenimento del profilo utente.
 					\end{description}
 				\item[Descrizione] \hfill \\
 					Viene ricercato nel database utenti un utente registrato con id uguale a 	\textit{user\_id}. Nel caso venga trovato, viene richiamata la funzione \textit{callback}	passando come parametro l'utente trovato, altrimenti viene chiamata con un oggetto vuoto.
 			\end{description}
 			
 		\mlitem{updateUserProfile(req, callback)} \hfill
 			\begin{description}
 				\item[Parametri] \hfill
 					\begin{description}
 						\item[req] \hfill \\
 							Oggetto contenente i parametri e le informazioni della richiesta HTTP del Client.
 						\item[callback] \hfill \\
 							Funzione da richiamare alla fine dell'aggiornamento del profilo utente.
 					\end{description}
 				\item[Descrizione] \hfill \\
 					Vengono estratti i dati da aggiornare dall'oggetto \textit{req} della richiesta HTTP. Successivamente viene effettuata una query di aggiornamento sul database degli utenti con i dati forniti da \textit{req}. In caso l'aggiornamento dei dati avvenga con successo, 	viene chiamata la funzione \textit{callback} con argomento \textbf{true}, altrimenti viene chiamata con argomento \textbf{false}.  
 			\end{description}
 			
 		\mlitem{getUsersList(column, order, page, perpage, callback)} \hfill
 			\begin{description}
 				\item[Parametri] \hfill
 					\begin{description}
 						\item[column] \hfill \\
 							Stringa contenente il nome della colonna su cui ordinare gli utenti.
 						\item[order] \hfill \\
 							Stringa che specifica l'ordinamento ascendente o discendente degli utenti.
 						\item[page] \hfill \\
 							Numero intero che specifica il numero della pagina di utenti da visualizzare.
 						\item[perpage] \hfill \\
 							Numero intero che specifica la quantità di utenti da visualizzare per pagina.
 						\item[callback] \hfill \\
 							Funzione da richiamare alla fine dell'ottenimento dei profili utente.
 					\end{description}
 				\item[Descrizione] \hfill \\
 				Viene innanzitutto importata l'intera lista degli utenti del sistema. Successivamente, se il risultato della query è vuoto, viene richiamata la funzione \textit{callback} come argomento un oggetto vuoto. Altrimenti, viene processata l'intera lista al fine di invocare la \textit{callback} specificando come argomento l'insieme di Document desiderato. I Document saranno quelli che corrispondono alla pagina \textit{page} correntemente richiesta con l'ordinamento \textit{order} desiderato sulla colonna scelta \textit{column}.
 			\end{description}
 			
 		\mlitem{updateUser(req, callback)} \hfill
 			\begin{description}
 				\item[Parametri] \hfill
 					\begin{description}
 						\item[req] \hfill \\
 							Oggetto contenente i parametri e le informazioni della richiesta HTTP del Client.
 						\item[callback] \hfill \\
 							Funzione da richiamare alla fine dell'aggiornamento del profilo utente.
 					\end{description}
 				\item[Descrizione] \hfill \\
 				Vengono estratti i dati da aggiornare dalla richiesta HTTP \textit{req} inerenti all'utente da aggiornare. Successivamente viene effettuata la query di aggiornamento sul modello del database utenti estratto. Infine viene richiamata la funzione \textit{callback} con argomento \textbf{true} se l'aggiornamento è andato a buon fine, l'argomento è \textbf{false} altrimenti.
 			\end{description}
 			
 		\mlitem{removeUser(email, callback)} \hfill
 			\begin{description}
 				\item[Parametri] \hfill
 					\begin{description}
 						\item[email] \hfill \\
 							Stringa contenente l'indirizzo email dell'utente che si desidera eliminare dal database.
 						\item[callback] \hfill \\
 							Funzione da richiamare alla fine dell'eliminazione del profilo utente.
 					\end{description}
 				\item[Descrizione] \hfill \\
 				Viene estratto il modello degli utenti su cui verrà invocato il metodo \textit{remove} per eliminare l'utente con campo email uguale al parametro \textit{email} passato alla funzione. Al termine della query di eliminazione, viene richiamata la funzione \textit{callback} con argomento \textbf{true} se l'aggiornamento è andato a buon fine, l'argomento è \textbf{false} altrimenti.
 			\end{description}
 			
 	\end{mldescription}
\end{description}


\paragraph{::IndexManager} \hfill \\
Package realizzato per creare e gestire gli indici utilizzati nel database di analisi.
\subparagraph{@IndexManager} \hfill
\begin{description}
 \item[Descrizione] \hfill \\
 Modulo del ModelServer che si occupa di creare e gestire gli indici utilizzati nel database di analisi. 
 \item[Metodi]
  \begin{mldescription}
   \mlitem{getModel(collection\_name)}
	\begin{description}
	 \item[Parametro] \hfill
	  \begin{description}
	   \item[collection\_name] \hfill \\
	   Stringa contenente il nome della Collection di cui ottenere il modello.
	  \end{description}
	 \item[Dipendenza] \hfill
	  \begin{itemize}
	   \item @MongooseDBAnalysis.
	  \end{itemize}
	 \item[Descrizione] \hfill \\
	 Richiede al modulo \grassetto{@MongooseDBAnalysis} il modello della Collection di nome \textit{collection\_name}. Se la Collection è presente, la funzione \textit{getModel} ritorna il suo modello, altrimenti ritorna -1.
	\end{description}	    
    
   \mlitem{addQuery(collection\_name, select)}
    \begin{description}
	 \item[Parametri] \hfill
	  \begin{description}
	   \item[collection\_name] \hfill \\
	   Stringa contenente il nome della Collection su cui è stata effettuata la query.
	   \item[select] \hfill \\
	   Array contenente i campi visualizzati della Collection.
	  \end{description}
	 \item[Dipendenza] \hfill
	  \begin{itemize}
	   \item @MongooseDBFramework.query.
	  \end{itemize}
	 \item[Descrizione] \hfill \\
	 Viene richiesto l'insieme delle query effettuate sul database di analisi precedentemente alla query corrente. Successivamente, se la query è stata eseguita per la prima volta, viene inserita nell'insieme delle query effettuate. Se invece la query è stata già eseguita, il suo contatore viene incrementato.
	\end{description}	    
    
   \mlitem{resetQueries(callback)}
    \begin{description}
	 \item[Parametro] \hfill
	  \begin{description}
	   \item[callback] \hfill \\
	   Funzione da richiamare alla fine della funzione \textit{resetQuery}.
	  \end{description}
	 \item[Dipendenze] \hfill
	  \begin{itemize}
	   \item @MongooseDBFramework.query;
	   \item @MongooseDBFramework.connection.
	  \end{itemize}
	 \item[Descrizione] \hfill \\
	 Richiede il modello della Collection delle query, su cui poi viene applicato il metodo \textit{dropCollection} che resetta la Collection delle query nel database.
	\end{description}	    
    
   \mlitem{getQueries(page, perpage, n\_elements, callback)}
    \begin{description}
	 \item[Parametri] \hfill
	  \begin{description}
	   \item[page] \hfill \\
	   Numero intero che indica la pagina delle query da visualizzare.
	   \item[perpage] \hfill \\
	   Numero intero che indica il numero di query da visualizzare per pagina.
	   \item[n\_elements] \hfill \\
	   Numero intero che indica il numero di query totali da visualizzare.
	   \item[callback] \hfill \\
	   Funzione da richiamare alla fine della funzione \textit{getQueries}.
	  \end{description}
	 \item[Dipendenze] \hfill
	  \begin{itemize}
	   \item @MongooseDBFramework;	   
	   \item @MongooseDBFramework.query.
	  \end{itemize}
	 \item[Descrizione] \hfill \\
	 Richiede l'intera Collection delle query al database di analisi. Calcola il sottoinsieme di query da visualizzare tenendo conto dell'indicatore della pagina da visualizzare e dagli elementi da visualizzare per pagina. Esegue quindi una nuova query per ottenere l'insieme delle query che rispettano suddetti parametri. Se avviene un errore o se non esistono query che rispettano i parametri scelti, viene passato alla \textit{callback} un oggetto vuoto. Se invece la ricerca ha avuto esito positivo, viene richiamata la \textit{callback} passando come parametro l'insieme delle query che soddisfano i requisiti. 
	\end{description}	    
    
   %\mlitem{getIndex(callback)}
    %\begin{description}
	 %\item[Parametri]			MOMENTANEAMENTE FUORI USO
	 %\item[Dipendenza]
	 %\item[Descrizione]
	%\end{description}	    
    
   \mlitem{createIndex(query\_id,  name\_index, callback)}
    \begin{description}
	 \item[Parametri] \hfill
	  \begin{description}
	   \item[query\_id] \hfill \\
	   Id identificativo della query da associare all'indice da creare.
	   \item[name\_index] \hfill \\
	   Stringa contenente il nome dell'indice da creare.
	   \item[callback] \hfill \\
	   Funzione da richiamare alla fine della creazione dell'indice.
	  \end{description}
	 \item[Dipendenze] \hfill
	  \begin{itemize}
	   \item @MongooseDBFramework;	   
	   \item @MongooseDBFramework.query.
	   \item schema della Collection su cui la query di cui si vuole creare l'indice esegue le richieste.
	  \end{itemize}
	 \item[Descrizione] \hfill \\
	 Viene richiesto il modello della query, grazie a cui viene cercata nel database di analisi la query con campo "\_id" uguale al parametro \textit{\_id}. Se la query non esiste o si riscontrano problemi di accesso al database, viene chiamata la funzione \textit{callback} con argomento \textbf{false}. Altrimenti viene caricato lo schema della Collection su cui la query viene eseguita. Sulla collection viene creato l'indice voluto. Se l'operazione di creazione non ha successo, viene viene chiamata la funzione \textit{callback} con argomento \textbf{false}, altrimenti viene chiamata la \textit{callback} con argomento \textbf{true}.
	 
	\end{description}	    
    
  \end{mldescription}
\end{description}

\paragraph{@JsonComposer}
\begin{description}
 \item[Descrizione] \hfill \\
 Modulo che si occupa della preparazione e della formattazione degli oggetti JSON inviati dal Server al Client.
 
 \item[Metodi]
 \begin{mldescription}
  \mlitem{createCollectionsList(collectionsList)} \hfill
   \begin{description}
    \item[Parametro] \hfill
     \begin{description}
      \item[collectionsList] \hfill \\
      Array di Collection.
     \end{description}
    \item[Descrizione] \hfill \\
    La funzione riceve come parametro un Array di Collection, crea e ritorna l'oggetto JSON rappresentante l'insieme delle Collection.
   \end{description}
  
  \mlitem{checkLabels(labelsArray)} \hfill
   \begin{description}
    \item[Parametro] \hfill
     \begin{description}
      \item[labelsArray] \hfill \\
      Array di etichette di una Collection.
     \end{description}
    \item[Descrizione]
    Scorre l'Array di label e ritorna \textit{true} se fra essi è presente il campo \textit{\_id} o il flag \textit{\_\_IDLABEL2SHOW\_\_}, ritorna \textit{false} altrimenti.
   \end{description}
   
  \mlitem{createCollection(labels, data, config)} \hfill
   \begin{description}
    \item[Parametri] \hfill
     \begin{description}
      \item[labels] \hfill \\
      Array contenente l'elenco delle etichette della Collection da creare.
      \item[data] \hfill \\
      Array contenente i dati sensibili della Collection.      
      \item[config] \hfill \\
      Oggetto contenente impostazioni di configurazione.      
     \end{description}
    \item[Descrizione] \hfill \\
    Scorre l'intero Array di dati della Collection e salva i suoi elementi in un Array locale. Esegue un controllo sull'Array di label per identificare la presenza dell'etichetta \textit{\_\_IDLABEL2SHOW\_\_}, la quale definisce di far visualizzare il campo \textit{\_id} nel Client. Dopo aver formattato correttamente gli elementi della Collection, ritorna il JSON corrispondente, aggiungendo l'oggetto \textit{config} contenente le configurazioni dell'oggetto utili al Client.
   \end{description}
   
   \mlitem{createDocument(labels, data)} \hfill
   \begin{description}
    \item[Parametri] \hfill
     \begin{description}
      \item[labels] \hfill \\
      Array contenente l'elenco delle etichette del Document da creare.
      \item[data] \hfill \\
      Array contenente i dati sensibili del Document da creare.
     \end{description}
    \item[Descrizione] \hfill \\
    Scorre l'intero Array di dati del Document e salva i suoi elementi in un Array locale. Esegue un controllo sull'Array di label per identificare la presenza dell'etichetta \textit{\_\_IDLABEL2SHOW\_\_}, la quale definisce di far visualizzare il campo \textit{\_id} nel Client. Dopo aver formattato correttamente gli elementi del Document, ritorna il JSON corrispondente.
   \end{description}
   
   \mlitem{createQueriesList(queries, config)} \hfill
   \begin{description}
    \item[Parametri] \hfill
     \begin{description}
      \item[queries] \hfill \\
      Array contenente l'insieme delle query da creare.
      \item[config] \hfill \\
      Oggetto contenente impostazioni di configurazione.
     \end{description}
    \item[Descrizione] \hfill \\
    Viene creato localmente l'Array contenente le etichette dell'oggetto "lista di query". Viene successivamente scorso l'Array delle query e vengono copiati localmente i loro campi sensibili, per poi essere formattati correttamente. Infine viene realizzato e ritornato il JSON corrispondente, aggiungendo l'oggetto contenente le impostazioni di configurazione della lista di query.
   \end{description}
   
   \mlitem{createIndexesList(indexes)} \hfill
   \begin{description}
    \item[Parametro] \hfill
     \begin{description}
      \item[indexes] \hfill \\
      Array contenente l'insieme di indici da creare.
     \end{description}
    \item[Descrizione] \hfill \\
    Viene creato localmente l'Array contenente le etichette dell'oggetto "lista di indici". Viene successivamente scorso l'Array degli indici e vengono copiati localmente i loro campi sensibili, per poi essere formattati correttamente. Infine viene realizzato e ritornato il JSON corrispondente.
   \end{description}
   
   \mlitem{createUserProfile(user)} \hfill
   \begin{description}
    \item[Parametro] \hfill
     \begin{description}
      \item[user] \hfill \\
      Oggetto contenente i campi email e livello di accesso di un utente.
     \end{description}
    \item[Descrizione] \hfill \\
    Viene interpretato il campo numerico \textit{level} e convertito nella stringa corrispondente al livello di accesso dell'utente da creare. Viene infine ritornato il JSON corrispondente.
   \end{description}
   
   \mlitem{createUserProfileEdit(user)} \hfill
   \begin{description}
    \item[Parametro] \hfill
     \begin{description}
      \item[user] \hfill \\
      Oggetto contenente il campo email di un utente.
     \end{description}
    \item[Descrizione] \hfill \\
    Viene ritornato il JSON di un utente modificabile formattato correttamente.
   \end{description}
   
   \mlitem{createUsersList(users, config)} \hfill
   \begin{description}
    \item[Parametri]
     \begin{description}
      \item[users] \hfill \\
      Array di oggetti "utente".
      \item[config] \hfill \\
      Oggetto contenente impostazioni di configurazione.
     \end{description}
    \item[Descrizione] \hfill \\
    Viene creato localmente l'Array delle etichette della lista utenti. Successivamente viene copiato in un Array locale ogni utente e il campo \textit{level} di ognuno viene interpretato e convertito nella stringa corrispondente. Infine viene assemblato il JSON della lista utenti e viene aggiunto l'oggetto contenente le impostazioni di configurazione della lista.
   \end{description}
   
   \mlitem{createUser(user)} \hfill
   \begin{description}
    \item[Parametro] \hfill \\
     \begin{description}
      \item[user] \hfill \\
      Oggetto contenente i campi id, email e livello di accesso di un utente.
     \end{description}
    \item[Descrizione] \hfill \\
    Viene creato localmente lo scheletro dell'oggetto utente. Viene poi interpretato l'intero contenuto nel campo \textit{level} dell'oggetto \textit{user}. Viene poi ritornato il JSON dell'utente formattato correttamente.
   \end{description}
   
 \end{mldescription}
 
\end{description}

\subsubsection{::DSL}
\subparagraph{@DSLManager}
\begin{description}
 \item[Descrizione] \hfill \\
  Modulo del \grassetto{ModelServer} che si occupa controllare la presenza dei file DSL nell'apposita cartella definita nel file di configurazione e invoca il parser dei DSL.
 \item[Dipendendenze] \hfill 
 \begin{itemize}
 \item{fs};
  \item{path};
  \item{@DSLParser};
  \item{@schemaGenerator}.
 \end{itemize} 
 \item[Metodi]
 \begin{mldescription}
 \mlitem{generateFunction(transformation)} \hfill 
  	\begin{description}
   		\item[Parametro] \hfill
   			\begin{description}
   				\item[transformation] \hfill \\
   				Oggetto che contiene il nome della trasformazione e il codice della trasformazione da eseguire.
   			\end{description}
   		\item[Descrizione]
   		Si occupa di generare le funzioni JavaScript per gestire le trasformazioni dei dati definite nel DSL.
   	\end{description}
 \mlitem{deleteFolderRecursive(path)} \hfill 
   	\begin{description}
    		\item[Parametro] \hfill
    			\begin{description}
    				\item[path] \hfill \\
    				Percorso della cartella da svuotare.
    			\end{description}
    		\item[Descrizione] \hfill \\
    		Si occupa di cancellare i file e le cartelle all'interno della cartella \textit{path} specificata.
    	\end{description}
 \mlitem{checkDSL(app)} \hfill 
 	\begin{description}
    		\item[Parametro] \hfill
    			\begin{description}
    				\item[app] \hfill \\
    				Oggetto contenente l'applicazione di \grassetto{Express}.
    			\end{description}
    		\item[Descrizione] \hfill \\
    		Si occupa di verificare la correttezza del DSL scritto, verificando che tale file contenga un codice corretto sintatticamente e attraverso l'uso del \grassetto{@DSLparser} esegui il parsing vero e proprio. Controlla che il risulatato del parsing sia in formato JSON, lo salva su file e utilizza il metodo \textit{generateFunction} per creare le funzioni di trasformazione. Se non esiste, si occupa di generare lo schema \grassetto{Mongoose} della Collection specificata nel file DSL.
    \end{description}
 \end{mldescription}
 
\end{description}
\subparagraph{@DSLParser}
\begin{description}
 \item[Descrizione] \hfill \\
  Modulo del ModelServer che si occupa effettuare il parsing di un DSL.
 \item[Dipendendenze] \hfill
 \begin{itemize}
  \item{@JavascriptParser}.
 \end{itemize}
  
 \item[Metodi]
 \begin{mldescription}
 \mlitem{addField(collection, field)} \hfill 
 \begin{description}
     		\item[Parametri] \hfill
     			\begin{description}
     				\item[collection] \hfill \\
     				Array della struttura del JSON.
     				\item[field] \hfill \\
     				Campo da aggiungere alla struttura del JSON.
     			\end{description}
     		\item[Descrizione] \hfill \\
     		Inizializza il campo field nell'Array della struttura del JSON, dove tale campo è definito nella struttura di default del JSON, con il valore di defalt.
 \end{description}
 \mlitem{checkFieldThrow(collection, field, root)}\hfill 
 \begin{description}
     		\item[Parametri] \hfill
     			\begin{description}
     				\item[collection] \hfill \\
     				Array della struttura del JSON.
     				\item[field] \hfill \\
     				Campo da verificare nella struttura del JSON.
     				\item[root] \hfill \\
     				\`{E} il nome della sezione in cui deve essere definito il campo.
     			\end{description}
     		\item[Descrizione] \hfill \\
     		Chiama il metodo \textit{checkField} e se tale metodo ritorna \textbf{false} allora lancia un eccezione. Questo metodo serve per verificare che il campo sia definito. 
  \end{description}
 \mlitem{checkField(collection, field)} \hfill 
 \begin{description}
     		\item[Parametri] \hfill
     			\begin{description}
     				\item[collection] \hfill \\
     			     Array della struttura del JSON.
     			    \item[field] \hfill \\
     			     Campo da verificare nella struttura del JSON.
     			\end{description}
     		\item[Descrizione] \hfill \\
     		Verifica se un campo è definito, se lo è ritorna \textbf{true}, altrimenti false.
  \end{description}
 \mlitem{checkFieldContentThrow(collection, field, root)} \hfill
 \begin{description}
     		\item[Parametri] \hfill
     			\begin{description}
     				\item[collection] \hfill \\
     				 Array della struttura del JSON.
     				\item[field] \hfill \\
     				 Campo da verificare nella struttura del JSON.
     				\item[root] \hfill \\
     				 \`{E} il nome della sezione in cui deve essere definito il campo.
     			\end{description}
     		\item[Descrizione] \hfill \\
     		Serve per verificare che il campo sia definito e non vuoto. Utilizza il metodo \textit{checkFieldContent} e se tale metodo ritorna false lancia un eccezione.
  \end{description}
 \mlitem{checkFieldContent(collection, field)} \hfill 
 \begin{description}
     		\item[Parametri] \hfill
     			\begin{description}
     				\item[collection] \hfill \\
     				 Array della struttura del JSON.
       			    \item[field] \hfill \\
       			     Campo da verificare nella struttura del JSON.
     			\end{description}
     		\item[Descrizione] \hfill \\
     		Verifica se un campo è definito e non vuoto, se lo è ritorna true, altrimenti false.
  \end{description}
  
 \mlitem{IntValue(value, field)} \hfill
 \begin{description}
     		\item[Parametri] \hfill
     			\begin{description}
     				\item[value] \hfill \\
     				 Valore del campo.
     				\item[field] \hfill \\
     				 Campo da verificare nella struttura del JSON.
     			\end{description}
     		\item[Descrizione] \hfill \\
     		Verifica che il valore sia numerico, se non lo è lancia un eccezione.
   \end{description}
 \mlitem{parseDSL(DSLstring)} \hfill
 \begin{description}
     		\item[Parametro] \hfill
     			\begin{description}
     				\item[DSLstring] \hfill \\
     				Contenuto del file maap di configurazione.
     			\end{description}
     		\item[Descrizione] \hfill \\
     		Si occupa di effettuare il parsing del DSL trasformandolo in JSON. Per effettuare il parsing verifica che i campi obbligatori siano definiti e non vuoti, che i campi il cui contenuto deve essere un numero lo sia e che la sintassi delle funzioni di trasformazione dei campi sia valida utilizzando il parser \grassetto{@JavascriptParser}.
   \end{description}
 \end{mldescription}
 
\end{description}
\subparagraph{@index}
\begin{description}
 \item[Descrizione] \hfill
  Modulo che si occupa dell'inizializzazione del \grassetto{@DSLManager}.
 \item[Dipendendenze] \hfill
 \begin{itemize}
  \item{@DSLManager}.
 \end{itemize}
  
 \item[Metodi]
 \begin{mldescription}
\mlitem{init(app)} \hfill
 \begin{description}
     		\item[Parametro] \hfill
     			\begin{description}
     				\item[app] \hfill \\
     				Oggetto contenente l'applicazione di \grassetto{Express}.
     			\end{description}
     		\item[Descrizione] \hfill \\
     		Si occupa di invocare il metodo \textit{checkDSL} di \grassetto{@DSLManager}.
   \end{description}
 \end{mldescription}
 
\end{description}
\subparagraph{@JavascriptParser}
\begin{description}
 \item[Descrizione] \hfill \\
  Modulo che effettua il parsing del codice JavaScript. Tale modulo non è stato scritto da noi ma generato con PEG.js. Si consulti il manuale di PEG.js per ulteriori dettagli.
  
 
\end{description}
\subparagraph{@schemaGenerator}
\begin{description}
 \item[Descrizione] \hfill \\
  Modulo che si occupa dell'inizializzazione del \grassetto{@DSLManager}.
 \item[Dipendendenza] \hfill
 \begin{itemize}
  \item{fs}.
 \end{itemize}
  
 \item[Metodi]
 \begin{mldescription}
 \mlitem{getPopulatedCollection(populateArray, key)} \hfill
 \begin{description}
      		\item[Parametri] \hfill
      			\begin{description}
      				\item[populateArray] \hfill \\
      				Array di oggetti contenenti la Collection su cui effettuare il populate e la chiave di riferimento.
      				\item[key] \hfill \\
      				Chiave da ricercare.
      			\end{description}
      		\item[Descrizione] \hfill \\
      		Verifica che sia presente all'interno di un array di oggetti un elemento con chiave uguale al parametro \textit{key}, se la trova ritorna la Collection corrispondente.
    \end{description}
 \mlitem{arrayAddElement(element, array)} \hfill
 \begin{description}
      		\item[Parametri] \hfill
      			\begin{description}
      				\item[element] \hfill \\
      				Oggetto da aggiungere all'array.
      			   	\item[array] \hfill \\
      			   	Array su cui aggiungere l'elemento.
      			\end{description}
      		\item[Descrizione] \hfill \\
      		Verifica che l'oggetto non sia presente all'interno dell'Array e se non lo trova lo aggiunge.
    \end{description}
 \mlitem{generate(config, dslJson)} \hfill
 \begin{description}
      		\item[Parametri] \hfill
      			\begin{description}
      				\item[config] \hfill \\
      				Contiene la configurazione dell'ambiente \grassetto{MaaP}.
      				\item[dslJson] \hfill \\
      				Contiene il JSON del parsing del DSL.
      			\end{description}
      		\item[Descrizione] \hfill \\
      		Si occupa di generare lo schema della Collection per mongoose partendo dai tipi di dati specificati dall'utente nel DSL.
      		
      		
      		
    \end{description}
 \end{mldescription}
 
\end{description}
\section{Specifica componenti Maap::Client}

\subsection{::View}
Package contenente i vari Template utilizzati dal Client AngularJS per visualizzare e gestire i dati delle varie Collection.

\subsubsection{@collection.html}
\begin{description}
	\item[Descrizione] \hfill \\
	Pagina contenente i documenti di una determinata Collection.
	\item[Utilizzo] \hfill \\
	Questa view viene utilizzata per visualizzare e gestire i Document di una data Collection, ordinabili su ogni colonna visualizzata. Mediante un click sul bottone di visualizzazione di un Document, è possibile visualizzare il Document desiderato.
	\item[Attributi] \hfill
	\begin{description}
		\item[\$scope.current\_sorted\_column] \hfill \\
		Indice della colonna correntemente utilizzata per l'ordinamento.
		\item[\$scope.column\_original\_name] \hfill \\
		Array contenente le etichette originali delle colonne della Collection.
		\item[\$scope.current\_sort] \hfill \\
		Stringa contenente l'ordinamento corrente della colonna identificata da \textit{\$scope.current\_sorted\_column}.
		\item[\$scope.current\_page] \hfill \\
		Intero che rappresenta la pagina correntemente visualizzata.
		\item[\$scope.canEdit] \hfill \\
		Valore booleano che identifica se l'utente correntemente autenticato possa modificare o meno i Document visualizzati.
		\item[\$scope.current\_collection] \hfill \\
		Stringa contenente il nome della Collection correntemente visualizzata.
		\item[\$scope.rows] \hfill \\
		Array contenente il contenuto di ogni Document della Collection, salvato per riga.
		\item[\$scope.pages] \hfill \\
		Intero contenente il numero di pagine di Document nella view corrente.
	\end{description}
 	
\end{description}

\subsubsection{@dashboard.html}
\begin{description}
	\item[Descrizione] \hfill \\
	Pagina principale visualizzata dopo il login.
	\item[Utilizzo] \hfill \\
	Questa view viene utilizzata per far visualizzare all'utente autenticato la lista delle Collection tra cui egli può scegliere.
	\item[Attributi] \hfill
	\begin{description}
		\item[\$scope.labels] \hfill \\
		Array contenente i nomi delle Collection tra cui scegliere.
		\item[\$scope.values] \hfill \\
		Array contenente gli identificativi delle Collection che verranno utilizzati per il reindirizzamento alla scelta della Collection desiderata.
		
	\end{description}
 	
\end{description}

\subsubsection{@document.html}
\begin{description}
	\item[Descrizione] \hfill \\
	Pagina per la visualizzazione di un singolo Document.
	\item[Utilizzo] \hfill \\
	In questa pagina di visualizzazione vengono forniti i dati di un singolo Document all'utente autenticato. Se l'utente in questione possiede i privilegi di amministratore, può anche decidere di modificare o eliminare un Document. 
	\item[Attributi] \hfill
	\begin{description}
		\item[\$scope.current\_collection] \hfill \\
		Stringa contenente il nome della Collection che racchiude in Document correntemente visualizzato.
		\item[\$scope.current\_document] \hfill \\
		Stringa contenente l'identificativo del Document correntemente visualizzato.
		\item[\$scope.labels] \hfill \\
		Array contenente i nomi delle colonne dei dati del Document visualizzato.
		\item[\$scope.values] \hfill \\
		Array contenente i dati sensibili del Document da visualizzare nella pagina.
		\item[\$scope.canEdit] \hfill \\
		Valore booleano che identifica se l'utente correntemente autenticato possa modificare o meno il Document visualizzato.
	\end{description}
\end{description}

\subsubsection{@documentEdit.html}
\begin{description}
	\item[Descrizione] \hfill \\
	Pagina per la modifica di un singolo Document.
	\item[Utilizzo] \hfill \\
	La pagina è composta principalmente di una TextArea in cui viene visualizzato il JSON del Document che si è deciso di modificare. In questa TextArea è possibile modificare l'intero Document, per poi salvare i cambiamenti o annullarli.
	\item[Attributi] \hfill
	\begin{description}
		\item[\$scope.current\_collection] \hfill \\
		Stringa contenente il nome della Collection che racchiude in Document correntemente visualizzato.
		\item[\$scope.current\_document] \hfill \\
		Stringa contenente l'identificativo del Document correntemente visualizzato.
		\item[\$scope.canEdit] \hfill \\
		Valore booleano che identifica se l'utente correntemente autenticato possa modificare o meno il Document visualizzato. Nel controller di questa view, esso è impostato di default a \textbf{true}.
		\item[\$scope.original\_data] \hfill \\
		Oggetto contenente i dati originali del Document corrente.
	\end{description}
\end{description}

\subsubsection{@help.html}
\begin{description}
	\item[Descrizione] \hfill \\
	Pagina contenente la sezione di aiuto per l'utente.
	\item[Utilizzo] \hfill \\
 	Questa pagina funge da manuale per l'utente, spiegando le azioni che si possono compiere all'interno di MaaP. Queste spiegazioni vanno dalle semplici istruzioni di registrazione e login fino all'utilizzo delle pagine di visualizzazione e modifica dei Document. 
\end{description}

\subsubsection{@indexCollection.html}
\begin{description}
	\item[Descrizione] \hfill \\
	Pagina per la visualizzazione degli indici creati, riservata all'amministratore.
	\item[Utilizzo] \hfill \\
	Questa pagina mostra gli indici creati sul Database di analisi, correlati alla loro Collection e i campi utilizzati. \`{E} eventualmente possibile eliminare indici attualmente presenti.
	\item[Attributi] \hfill
 	\begin{description}
 		\item[\$scope.current\_sorted\_column] \hfill \\
 		Indice della colonna correntemente utilizzata per l'ordinamento.
 		\item[\$scope.column\_original\_name] \hfill \\
		Array contenente le etichette originali delle colonne della gestione degli indici.
		\item[\$scope.current\_sort] \hfill \\
		Stringa contenente l'ordinamento corrente della colonna identificata da \textit{\$scope.current\_sorted\_column}.
		\item[\$scope.current\_page] \hfill \\
		Intero che rappresenta la pagina correntemente visualizzata.
		\item[\$scope.rows] \hfill \\
		Array contenente il contenuto di ogni indice della Collection, salvato per riga.
 	\end{description}
\end{description}

\subsubsection{@login.html}
\begin{description}
	\item[Descrizione] \hfill \\
	Pagina adibita al login utente.
	\item[Utilizzo] \hfill \\
	Questa pagina viene utilizzata per l'inserimento delle credenziali dell'utente. Contiene collegamenti alla pagina di recupero password, alla registrazione e alla sezione di aiuto.
	\item[Attributi] \hfill
 	\begin{description}
 		\item[\$scope.credentials] \hfill \\
 		Oggetto contenente il campo \textit{username} e \textit{password} inseriti dall'utente.
 	\end{description}
\end{description}

%\subsubsection{@logout.html}
%\begin{description}
%	\item[Descrizione] \hfill \\
%	Pagina di gestione 
%	\item[Utilizzo] \hfill \\
%	
%	\item[Attributi] \hfill
 	
%\end{description}

\subsubsection{@Navbar.html}
\begin{description}
	\item[Descrizione] \hfill \\
	Barra di navigazione superiore, inclusa nella maggior parte delle pagine.
	\item[Utilizzo] \hfill \\
	Sezione di pagina contenente i pulsanti primari di navigazione di \textit{Maap}. Sono presenti i pulsanti di scelta delle Collection e della gestione del profilo utente. Se l'utente autenticato ha privilegi di amministratore, può consultare anche le Collection degli utenti ed accedere all'area di gestione dei profili utente registrati al sistema.
	\item[Attributi] \hfill
 	\begin{description}
 		\item[\$scope.isAdmin] \hfill \\
 		Valore booleano che definisce se l'utente autenticato abbia o meno i privilegi di amministrazione.
 		\item[\$scope.labels] \hfill \\
 		Array contenente i nomi delle Collection presenti nel sistema.
 		\item[\$scope.values] \hfill \\
 		Array contenente i nomi delle Collection necessari per eseguire la navigazione ad una data Collection tramite \textit{query string}.
 	\end{description}
\end{description}

\subsubsection{@pwdrecovery.html}
\begin{description}
	\item[Descrizione] \hfill \\
	Pagina utilizzata per il recupero password di un utente iscritto al sistema.
	\item[Utilizzo] \hfill \\
	Questa pagina fornisce un form di input all'utente che desidera recuperare la password dimenticata. L'utente dovrà inserire la mail con cui si è iscritto al sistema. Il sistema stesso, successivamente, invierà una mail all'indirizzo specificato con la nuova password.
	\item[Attributi] \hfill
 	\begin{description}
 		\item[\$scope.credentials] \hfill \\
 		Oggetto contenente l'indirizzo email specificato dall'utente che desidera recuperare la password.
 	\end{description}
\end{description}

\subsubsection{@queryCollection.html}
\begin{description}
	\item[Descrizione] \hfill \\
	Pagina per la visualizzazione delle query più utilizzate. Questa pagina è riservata agli utenti che possiedono privilegi di amministrazione.
	\item[Utilizzo] \hfill \\
	In questa pagina è possibile visualizzare le query più utilizzate nel sistema. Esse sono correlate da un numero che esprime la quantità di volte che la query è stata utilizzata. Successivamente è possibile creare un indice su una query visualizzata. In alternativa, sempre in questa pagina, è possibile visualizzare il comando da utilizzare nel terminale per creare l'indice manualmente.
	\item[Attributi] \hfill
 	\begin{description}
 		\item[\$scope.current\_sorted\_column] \hfill \\
 		Indice della colonna correntemente utilizzata per l'ordinamento.
 		\item[\$scope.column\_original\_name] \hfill \\
		Array contenente le etichette originali delle colonne della gestione delle query.
		\item[\$scope.current\_sort] \hfill \\
		Stringa contenente l'ordinamento corrente della colonna identificata da \textit{\$scope.current\_sorted\_column}.
		\item[\$scope.current\_page] \hfill \\
		Intero che rappresenta la pagina correntemente visualizzata.
		\item[\$scope.rows] \hfill \\
		Array contenente il contenuto di ogni query. Ovvero, comprende: la Collection di apparteneneza, i campi su cui viene effettuata la query ed il numero di volte che essa è stata effettuata.
 	\end{description}
\end{description}

\subsubsection{@register.html}
\begin{description}
	\item[Descrizione] \hfill \\
	Pagina utilizzata per la registrazione di un nuovo utente.
	\item[Utilizzo] \hfill \\
	Questa pagina fornisce un form di inserimento in cui l'utente deve inserire la propria mail e la password scelta. La mail deve essere valida e non deve essere presente nel Server un utente con la medesima mail.
	\item[Attributi] \hfill
 	\begin{description}
 		\item[\$scope.credentials] \hfill \\
 		Oggetto contenente le credenziali immesse dall'utente, ovvero la sua mail, la password desiderata e la conferma della password.
 		\item[\$scope.submitted] \hfill \\
 		Valore booleano che definisce se il bottone di conferma della registrazione sia abilitato o meno. Questo si verifica solamente quando l'intero form è stato riempito con campi validi.
 	\end{description}
\end{description}

\subsubsection{@userCollection.html}
\begin{description}
	\item[Descrizione] \hfill \\
	Pagina per la gestione della Collection degli utenti di Maap. Questa pagina è visibile solamente per un utente che possiede i privilegi di amministrazione.
	\item[Utilizzo] \hfill \\
	In questa pagina vengono visualizzati tutti gli utenti registrati al sistema. Da qui l'utente amministratore può decidere di visualizzare un profilo utente, modificarlo o eliminarlo. Nella finestra a destra della Collection, è inoltre possibile refistrare un nuovo utente.
	\item[Attributi] \hfill
	\begin{description}
		\item[\$scope.credentials] \hfill \\
		Oggetto contenente le credenziali immesse dall'utente amministratore alla creazione di un nuovo utente, ovvero la mail, la password desiderata e la conferma della password.
		\item[\$scope.current\_sorted\_column] \hfill \\
 		Indice della colonna correntemente utilizzata per l'ordinamento.
 		\item[\$scope.column\_original\_name] \hfill \\
		Array contenente le etichette originali delle colonne dei campi utente.
		\item[\$scope.current\_sort] \hfill \\
		Stringa contenente l'ordinamento corrente della colonna identificata da \textit{\$scope.current\_sorted\_column}.
		\item[\$scope.current\_page] \hfill \\
		Intero che rappresenta la pagina correntemente visualizzata.
		\item[\$scope.rows] \hfill \\
		Array contenente il contenuto di ogni profilo utente, memorizzato per riga.
	\end{description}
\end{description}

\subsubsection{@userDocument.html}
\begin{description}
	\item[Descrizione] \hfill \\
	Pagina utilizzata per la visualizzazione dei dati di un singolo utente, amministratore o non.
	\item[Utilizzo] \hfill \\
	Questa pagina presenta in un'unica visualizzazione le informazioni legate ad un utente registrato al sistema. A destra della visualizzazione è presente un pulsante che porta alla pagina di modifica dei propri dati.
	\item[Attributi] \hfill
 	\begin{description}
 		\item[\$scope.current\_document] \hfill \\
		Stringa contenente l'identificativo del profilo utente correntemente visualizzato.
		\item[\$scope.original\_data] \hfill \\
		Array contenente i campi dati del profilo visualizzato.
		\item[\$scope.original\_keys] \hfill \\
		Array contenente le etichette originali dei campi dati utente.
 	\end{description}
\end{description}

\subsubsection{@userEdit.html}
\begin{description}
	\item[Descrizione] \hfill \\
	Pagina per la modifica del profilo di un utente, amministratore e non.
	\item[Utilizzo] \hfill \\
	In questa pagina vengono visualizzati i campi di un utente, in caselle di testo modificabili. L'utente può quindi modificare i suoi dati e salvarli o annullarli.
	\item[Attributi] \hfill
 	\begin{description}
 		\item[\$scope.current\_document] \hfill \\
		Stringa contenente l'identificativo del profilo utente correntemente visualizzato.
		\item[\$scope.newPassword1] \hfill \\
		Stringa contenente la nuova password inserita.
		\item[\$scope.newPassword2] \hfill \\
		Stringa contenente la conferma della nuova password inserita.
		\item[\$scope.original\_data] \hfill \\
		Array contenente i campi dati del profilo visualizzato.
		\item[\$scope.original\_keys] \hfill \\
		Array contenente le etichette originali dei campi dati utente.
		\item[\$scope.admin] \hfill \\
		Valore booleano che specifica se l'utente sia amministratore o meno.
	\end{description}
\end{description}

\subsubsection{@userProfile.html}
\begin{description}
	\item[Descrizione] \hfill \\
	Pagina per la visualizzazione del proprio profilo utente.
	\item[Utilizzo]
	Questa pagina mostra i dati del proprio profilo utente. Da questa visualizzazione è possibile passare alla pagina di modifica dei propri dati.
	\item[Attributi] \hfill
 	\begin{description}
 		\item[\$scope.original\_data] \hfill \\
		Array contenente i campi dati del proprio profilo.
		\item[\$scope.original\_keys] \hfill \\
		Array contenente le etichette originali dei propri campi dati utente.
 	\end{description}
\end{description}


\subsection{::ControllerModelView}
Package che gestisce il dialogo tra la View ed il Model del Client.

\subsubsection{::ControllerClient}
Package che racchiude i Controller del Client. Il primo metodo descritto di ogni Controller è la funzione base di recupero dei dati dal Server attraverso la chiamata del metodo di fetch del proprio Service.

\paragraph{@CollectionController}
\begin{description}
 \item[Descrizione] \hfill \\
 Modulo che descrive il Controller della Collection view.
 
 \item[Utilizzo] \hfill \\
 Controller utilizzato da AngularJS per fornire e gestire i dati della Collection view.
 Fornisce l'inizializzazione di base della pagina ed esegue la mediazione tra la Collection view
 e le richieste di visualizzazione e modifica dei dati di una Collection.
 
 \item[Dipendendenze iniettate al controller] \hfill
 \begin{itemize}
  \item \$scope;
  \item \$route;
  \item \$location;
  \item \$routeParams;
  \item @DocumentEditService;
  \item @CollectionDataService;
 \end{itemize}
 
 \item[Attributi] \hfill
 \begin{itemize}
 \item \$scope.current\textunderscore sorted \textunderscore column;
 \item \$scope.column \textunderscore original \textunderscore name;
 \item \$scope.current \textunderscore sort;
 \item \$scope.current \textunderscore page;
 \item \$scope.current\textunderscore collection;
 \item \$scope.rows;
 \end{itemize}
 
 \item[Funzionalità] \hfill
 \begin{description}
  \item[init()] \hfill
  	\begin{description}
  		\item[Descrizione] \hfill \\
  Funzione di inizializzazione del \grassetto{@CollectionController}. Imposta i valori iniziali delle variabili dello
  \$scope associate alla Collection view ed esegue la funzione \textit{getData()}.
  	\end{description}
  	
  \item[getData()] \hfill 
  \begin{description}
  		\item[Descrizione] \hfill \\
  Richiama la richiesta RESTful \textit{query} sulla \$resource fornita dal \grassetto{@CollectionDataService}.
  Con questa funzione vengono prelevati i dati legati alla Collection da visualizzare e vengono
  passati alla Collection view tramite il binding con lo \$scope. 
  
  \end{description}
  
  \item[\$scope.numerify(num)] \hfill 
  \begin{description}
  	\item[Parametro] \hfill
  	\begin{description}
  		\item[num] \hfill \\
  		Intero rappresentante il numero di pagine da visualizzare.
	\end{description}  		
  	\item[Descrizione] \hfill \\
  Funzione ausiliaria che crea un array di dimensione \textit{num} che verrà poi usato dalla directive \textit{ng-repeat} di AngularJS per la creazione del numero di pagine disponibili.
  \end{description}
  
  \item[\$scope.previousPage()] \hfill 
  \begin{description}
  		\item[Descrizione] \hfill \\
  Aggiunge allo \$scope la funzionalità di passaggio dalla pagina di visualizzazione corrente alla precedente. Diminuisce di 1 l'indicatore della pagina corrente e esegue una nuova richiesta al server per i dati corrispondenti alla pagina precedente con \textit{getData()}.
  \end{description}
  
  \item[\$scope.nextPage()] \hfill
  \begin{description}
  		\item[Descrizione] \hfill \\
  Aggiunge allo \$scope la funzionalità di passaggio dalla pagina di visualizzazione corrente alla successiva. Aumenta di 1 l'indicatore della pagina corrente e esegue una nuova richiesta al server per i dati corrispondenti alla pagina successiva con \textit{getData()}.
  \end{description}
  
  \item[\$scope.toPage(index)] \hfill
  \begin{description}
  	\item[Parametro] \hfill
  	\begin{description}
  		\item[index] \hfill \\
  		Numero intero che definisce il numero della pagina da visualizzare.
  	\end{description}
  	\item[Descrizione] \hfill \\
  Aggiunge allo \$scope la funzionalità di passaggio alla pagina di visualizzazione specificata dal parametro \textit{index}. Aggiorna l'indicatore della pagina corrente settandolo all'indice passato per parametro alla funzione. Successivamente esegue una richiesta al server per ottenere i dati per la nuova visualizzazione.
  \end{description}
  
  \item[changeSort()] \hfill 
  \begin{description}
  	\item[Descrizione] \hfill \\
  Cambia la variabile \$current\_sort dello \$scope che regola l'ordine di visualizzazione dei Document, da discendente ad ascendente e viceversa.
  \end{description}
  
  \item[\$scope.columnSort(index)] \hfill
  \begin{description}
  	\item[Parametro] \hfill
  	\begin{description}
  		\item[index] \hfill \\
  		Numero intero che definisce il numero della colonna a cui applicare l'ordinamento.
  	\end{description}
  	\item[Descrizione] \hfill \\
  Funzione che cambia l'ordinamento di visualizzazione di una colonna della Collection view. \\
  Se la colonna in questione è una colonna diversa da quella che precedentemente ordinata, la variabile \textit{current\_sorted\_column} dello \$scope viene aggiornata, altrimenti la funzione cambia solamente l'ordinamento della colonna.
  \end{description}
  
  \item[\$scope.deleteDocument(index)] \hfill 
  \begin{description}
  	\item[Parametro] \hfill
  	\begin{description}
  		\item[index] \hfill \\
  		Numero intero che definisce il numero del Document da eliminare.
  	\end{description}
  	\item[Descizione] \hfill \\
  Funzione che elimina il Document identificato dall'indice \textit{index}. Effettua una richiesta REST di rimozione alla risorsa pubblicizzata da \grassetto{@DocumentEditService} fornendo come argomenti l'identificativo della Collection correntemente visualizzata e l'indice del Document da eliminare. \\
  Se l'eliminazione ha successo viene lanciato l'aggiornamento della vista corrente con il Document eliminato, altrimenti si richiama la callback di errore.
  \end{description}
  
 \end{description}
\end{description}

\paragraph{@DashboardController}
\begin{description}
 \item[Descrizione] \hfill \\
 Modulo che descrive il Controller della Dashboard.
 
 \item[Utilizzo] \hfill \\
 Controller utilizzato da AngularJS per fornire e gestire i dati della dashboard.
Fornisce un immediato elenco delle collection disponibili.
 
 \item[Dipendendenze iniettate al controller] \hfill
 \begin{itemize}
  \item \$scope;
  \item @CollectionListService;
 \end{itemize}
 
 \item[Funzionalità]
 \begin{mldescription}
  \mlitem{CollectionListService.get()}
  \begin{description}
  	\item[Descrizione] \hfill \\
    Funzione che richiede al server l'elenco delle collection disponibli e le inserisce nello scope.
  \end{description}
 \end{mldescription}
\end{description}

\paragraph{@DocumentController}
\begin{description}
 \item[Descrizione] \hfill \\
 Modulo che descrive il Controller della pagina di visualizzazione di un documento.
 
 \item[Utilizzo] \hfill \\
 Controller utilizzato da AngularJS per fornire e gestire i dati durante la visualizzazione di un document.
 
 \item[Dipendendenze iniettate al controller] \hfill
 \begin{itemize}
  \item \$scope;
  \item \$location
  \item @DocumentDataService;
  \item @DocumentEditService;
  \item \$routeParams.
 \end{itemize}
 
 
 \item[Attributi] \hfill
 \begin{itemize}
 \item  \$scope.current\textunderscore collection;
 \item  \$scope.current\textunderscore document;
 \item  \$scope.values;
 \end{itemize}
 
 \item[Funzionalità]
 \begin{mldescription}
  \mlitem{DocumentDataService.query()}
  \begin{description}
  	\item[Parametri]
  		\begin{mldescription}
  			\mlitem{\$scope.current\textunderscore collection;}
            \mlitem{\$scope.current\textunderscore document.}
        
        Parametri che identificano la collection e il document da visualizzare.
  		\end{mldescription}
  	\begin{description}
  	\item[Descrizione] \hfill \\
	Funzione che richiede al server il document da visualizzare.
	\end{description}
  \end{description}

  \mlitem{delete\_document()}
  \begin{description}
  	\item[Parametri]
  		\begin{mldescription}
  			\mlitem{\$scope.current\textunderscore collection;}
            \mlitem{\$scope.current\textunderscore document.}
        Parametri che identificano la collection e il document da cancellare.
  		\end{mldescription}
 	\begin{description} 	
  	\item[Descrizione] \hfill \\
    Funzione che richiede al server la cancellazione di un document.
    \end{description}

  \end{description}
 \end{mldescription}
\end{description}

\paragraph{@DocumentEditController}
\begin{description}
 \item[Descrizione] \hfill \\
 Modulo che descrive il Controller della pagina di modifica di un documento.
 
 \item[Utilizzo] \hfill \\
 Controller utilizzato da AngularJS per fornire e gestire i dati durante la modifica di un document.
 
 \item[Dipendendenze iniettate al controller] \hfill
 \begin{itemize}
  \item \$scope;
  \item \$location
  \item @DocumentEditService;
  \item \$routeParams.
  
 \end{itemize}
 
 \item[Attributi] \hfill
 \begin{itemize}
 \item  \$scope.current\textunderscore collection;
    \item  \$scope.current\textunderscore document;
    \item  \$scope.original\textunderscore data;
    \item  \$scope.original\textunderscore keys;
 \end{itemize}
 
 \item[Funzionalità] \hfill
 \begin{description}
  \item[DocumentEditService.query()] \hfill
   \begin{description}
  	\item[Parametri]
  		\begin{mldescription}
  			\mlitem{\$scope.current\textunderscore collection;}
            \mlitem{\$scope.current\textunderscore document.}
       
        Parametri che identificano la collection e il document da modificare.
  		\end{mldescription}
  	\begin{description}
  	\item[Descrizione] \hfill \\
	Funzione che richiede al server il document da modificare.
	\end{description}
  \end{description}
 
 
   \item[edit\_document()] \hfill 
   \begin{description}
  	\item[Parametri]
  		\begin{mldescription}
  			\mlitem{\$scope.current\textunderscore collection;}
            \mlitem{\$scope.current\textunderscore document.}
        
        Parametri che identificano la collection e il document da sovrascrivere.
  		\end{mldescription}
  	\begin{description}
  	\item[Descrizione] \hfill \\
	Funzione che richiede al server di sovrascrivere il document identificato dai parametri con quello modificato dall'utente.
	\end{description}
  \end{description}
  
  
  \item[delete\textunderscore document()] \hfill 
   \begin{description}
  	\item[Parametri]
  		\begin{mldescription}
  			\mlitem{\$scope.current\textunderscore collection;}
            \mlitem{\$scope.current\textunderscore document.}
      
        Parametri che identificano la collection e il document da cancellare.
  		\end{mldescription}
  	\begin{description}
  	\item[Descrizione] \hfill \\
    Funzione che richiede al server la cancellazione di un document.
	\end{description}
  \end{description}
 \end{description}
\end{description}

\paragraph{@IndexController}
\begin{description}
 \item[Descrizione] \hfill \\
 Modulo che descrive il Controller della pagina di gestione degli indici.
 
 \item[Utilizzo] \hfill \\
 Controller utilizzato da AngularJS per fornire e gestire i dati degli indici.

 
 \item[Dipendendenze iniettate al controller] \hfill
 \begin{itemize}
  \item \$scope;
  \item \$route;
  \item \$location;
  \item @IndexService;
 \end{itemize}
 
 \item[Attributi] \hfill
 \begin{itemize}
 \item \$scope.rows;
 \end{itemize}

 \item[Funzionalità] \hfill 
 \begin{description}
  \item[init()] \hfill 
  	\begin{description}
  		\item[Descrizione] \hfill \\
  		Funzione di inizializzazione dell'\grassetto{@IndexController}. Imposta i valori iniziali delle variabili dello \$scope ed esegue la funzione \textit{getData()}.
  	\end{description}
  
  \item[getData()] \hfill 
  \begin{description}
  	\item[Descrizione] \hfill \\
  	Richiama la richiesta RESTful \textit{query} sulla \$resource fornita da \grassetto{@IndexService}.
  Con questa funzione vengono prelevati i dati legati agli indici da visualizzare e vengono passati alla  view tramite il binding con lo \$scope.
  \end{description}
  
  \item[\$scope.numerify(num)] \hfill 
  \begin{description}
  	\item[Parametro] \hfill
  	\begin{description}
  		\item[num] \hfill \\
  		Numero intero che specifica il numero di pagine da visualizzare.
  	\end{description}
  	\item[Descrizione] \hfill \\
  	Funzione ausiliaria che crea un array di dimensione \textit{num} che verrà poi usato dalla directive \textit{ng-repeat} di AngularJS per la creazione del numero di pagine disponibili.
  \end{description}
  
  \item[\$scope.previousPage()] \hfill 
  \begin{description}
  	\item[Descrizione] \hfill \\
  	Aggiunge allo \$scope la funzionalità di passaggio alla pagina di visualizzazione precedente alla corrente. Diminuisce di 1 l'indicatore della pagina corrente e esegue una nuova richiesta al server per i dati corrispondenti alla pagina precedente con \textit{getData()}.
  \end{description}
  
  \item[\$scope.nextPage()] \hfill
  \begin{description}
  	\item[Descrizione] \hfill \\
  Aggiunge allo \$scope la funzionalità di passaggio alla pagina di visualizzazione successiva alla corrente. Aumenta di 1 l'indicatore della pagina corrente e esegue una nuova richiesta al server per i dati corrispondenti alla pagina successiva con \textit{getData()}.
  \end{description}
  
  \item[\$scope.toPage(index)] \hfill
  \begin{description}
  	\item[Parametro] \hfill
  	\begin{description}
  		\item[index] \hfill \\
  		Numero intero che specifica il numero di pagine da visualizzare.
  	\end{description}
  	\item[Descrizione] \hfill \\
  	Aggiunge allo \$scope la funzionalità di passaggio alla pagina di visualizzazione specificata dal parametro \textit{index}.
  Aggiorna l'indicatore della pagina corrente settandolo all'indice passato per parametro alla funzione.
  Successivamente esegue una richiesta al server per ottenere i dati per la nuova visualizzazione.
  \end{description}
  
  \item[changeSort()] \hfill 
  \begin{description}
  	\item[Descrizione] \hfill \\
  Cambia la variabile \$current\_sort dello \$scope che regola l'ordine di visualizzazione dei Document, da discendente ad ascendente e viceversa.
  \end{description}
  
  \item[\$scope.columnSort(index)] \hfill
  \begin{description}
  	\item[Parametro] \hfill
  	\begin{description}
  		\item[index] \hfill \\
  		Numero intero che specifica il numero di pagine da visualizzare.
  	\end{description}
  	\item[Descrizione] \hfill \\
  	  Funzione che cambia l'ordinamento di visualizzazione di una colonna della Collection view. \\
  Se la colonna in questione è una colonna diversa da quella che precedentemente ordinata,
  la variabile \textit{current\_sorted\_column} dello \$scope viene aggiornata, altrimenti la funzione cambia solamente
  l'ordinamento della colonna.
  \end{description}

  
  \item[\$scope.deleteDocument(index)] \hfill
  \begin{description}
  	\item[Parametro] \hfill
  	\begin{description}
  		\item[index] \hfill \\
  		Numero intero che specifica il numero di pagine da visualizzare.
  	\end{description}
  	\item[Descrizione] \hfill \\
  	  Funzione che elimina l'indice identificato dall'indice \textit{index}. Effettua una richiesta REST di rimozione alla risorsa
  gestita da \grassetto{@IndexService} fornendo come argomenti l'indice passato come argomento. \\
  Se l'eliminazione ha successo viene lanciato l'aggiornamento della vista corrente con il Document eliminato, altrimenti si richiama la callback di errore.
  \end{description}
  
  
 \end{description}
\end{description}


\paragraph{@LoginController}
\begin{description}
 \item[Descrizione] \hfill \\
 Modulo che descrive il Controller della pagina di login.
 
 \item[Utilizzo] \hfill \\
 Controller utilizzato da AngularJS per fornire e gestire i dati durante l'autenticazione.
 
 \item[Dipendendenze iniettate al controller] \hfill
 \begin{itemize}
  \item \$scope;
  \item \$route;
  \item \$cookieStore;
  \item \$location;
  \item @AuthService;
  
 \end{itemize}
 
 \item[Attributi] \hfill
 \begin{itemize}
 \item \$scope.credentials;
 \end{itemize}
 
 \item[Funzionalità] \hfill
 \begin{description}
  \item[\$scope.login()] \hfill
  \begin{description}
  	\item[Parametro] \hfill
  	\begin{description}
  		\item[\$scope.credentials] \hfill \\
  		Credenziali di accesso inserite dall'utente.
  	\end{description}
  	\item[Descrizione] \hfill \\
  	Funzione che trasmette al server le credenziali inserite dall'utente e verifica se sono corrette.
  	In caso positivo autentica l'utente, altrimenti mostra un errore.
  \end{description}
 
  
 \end{description}
\end{description}

\paragraph{@NavBarController}
\begin{description}
 \item[Descrizione] \hfill \\
 Modulo che descrive il Controller della barra di navigazione superiore.
 
 \item[Utilizzo] \hfill \\
 Viene utilizzato per gestire i link e le informazioni contenute sulla barra di navigazione presente in tutte le pagine protette da autenticazione.
 Contiene i link alle varie collection e i pulsanti per la gestione del profilo utente e per il logout.
 
 \item[Dipendendenze iniettate al controller] \hfill 
 \begin{itemize}
  \item \$scope;
  \item \$route;
  \item \$cookieStore;
  \item \$location;
  \item @LogoutService;
  \item @CollectionListService.
  
 \end{itemize}
 
 \item[Attributi] \hfill 
 \begin{itemize}
 \item \$scope.isAdmin;
 \item \$scope.singup\_enabled.

 \end{itemize}
 
 \item[Funzionalità] \hfill 
 \begin{description}
   
  	\item[logout()] \hfill \\
  	\begin{description}
  	\item[Descrizione] \hfill \\
  Richiama la richiesta RESTful \textit{logout} sulla \$resource fornita da \grassetto{@LogoutService}. Con questa funzione viene distrutta la sessione riguardante l'utente presente sul server e si viene poi disconnessi.
  	\end{description}
   
 \end{description}
\end{description}

\paragraph{@ProfileController}
\begin{description}
 \item[Descrizione] \hfill \\
 Modulo che descrive il Controller della pagina di gestione del profilo.
 
 \item[Utilizzo] \hfill \\
 Controller utilizzato da AngularJS che permette all'utente di gestire il suo profilo, visualizzandone le informazioni contentute.
 
 \item[Dipendendenze iniettate al controller] \hfill 
 \begin{itemize}
  \item \$scope;
  \item \$location;
  \item @ProfileDataService;
  \item @ProfileEditService;
  
 \end{itemize}
 
 \item[Attributi] \hfill 
 \begin{itemize}
 \item \$scope.original\textunderscore data;
 \item \$scope.original\textunderscore keys;

 \end{itemize}
 
 \item[Funzionalità] \hfill 
 \begin{description}
  \item[ProfileDataService.query()] \hfill 
   \begin{description}
  	\item[Descrizione] \hfill \\
	Funzione che richiede al server le informazioni riguardanti il profilo dell'utente autenticato.
	\end{description}
	
	 \item[\$scope.deleteDocument()] \hfill
	 
	 \begin{description}
	 	\item[Descrizione] \hfill \\	 
  Funzione che elimina dal server il profilo dell'utente autenticato, cancellando l'utente dal sistema.
  	 \end{description}

  \end{description}
  
  
\end{description}

\paragraph{@ProfileEditController}
\begin{description}
 \item[Descrizione] \hfill \\
 Modulo che descrive il Controller della pagina di modifica di un documento.
 
 \item[Utilizzo] \hfill \\
 Controller utilizzato da AngularJS che permette all'utente di gestire il suo profilo, visualizzando e modificando le informazioni contentute.\\ 
 \item[Dipendendenze iniettate al controller] \hfill
 \begin{itemize}
  \item \$scope;
  \item \$location
  \item @ProfileEditService;
  
 \end{itemize}
 
 \item[Attributi] \hfill
 \begin{itemize}
 \item \$scope.oldPassword;
 \item \$scope.newPassword1;
 \item \$scope.newPassowrd2;
 \item \$scope.original\textunderscore data;
 \item \$scope.original\textunderscore keys;
 \item \$scope.valid.
 \end{itemize}
 
 \item[Funzionalità] \hfill
 \begin{description}
  \item[ProfileEditService.query()] \hfill
  \begin{description}
  	\item[Descrizione] \hfill \\
   Richiama la richiesta RESTful \textit{query} sulla \$resource fornita da \grassetto{@ProfileEditService}.
  Con questa funzione vengono prelevati i dati legati al profilo utente e caricati sullo scope.
	 
  \end{description}

  \item[edit\textunderscore document()] \hfill 
  \begin{description}
  	\item[Parametro] \hfill
  		\begin{description}
  			\item[json\_data] \hfill \\
  			JSON contenente le nuove informazioni da modificare nel database utenti presente nel Server.
       \end{description}
  	\item[Descrizione] \hfill \\
Funzione che invia al server il profilo modificato nel client.
  Assembla un JSON a partire da original\textunderscore keys e original\textunderscore data. Quest'ultima variabile nel frattempo è stata modificata con i nuovi valori del documento.
    \end{description}
  
  
\item[\$scope.deleteDocument()] \hfill
	 
	 \begin{description}
	 	\item[Descrizione] \hfill \\	 
  Funzione che elimina dal server il profilo dell'utente autenticato, cancellando l'utente dal sistema.
  	 \end{description}
 \end{description}
\end{description}

\paragraph{@QueryController}
\begin{description}
 \item[Descrizione] \hfill \\
 Modulo che descrive il Controller della pagina di gestione delle query.
 
 \item[Utilizzo] \hfill \\
 Controller utilizzato da AngularJS per fornire e gestire i dati della pagina di gestione delle query.
 
 
 \item[Dipendendenze iniettate al controller] \hfill
 \begin{itemize}
  \item \$scope;
  \item \$route;
  \item \$location;
  \item @QueryService;
  \item @IndexService;
 \end{itemize}
 
 \item[Attributi] \hfill
 \begin{itemize}
 \item \$scope.current\textunderscore sorted\textunderscore column;
 \item \$scope.column\textunderscore original\textunderscore name;
 \item \$scope.current\textunderscore sort;
 \item \$scope.current\textunderscore page;
 \item \$scope.current\textunderscore collection;
 \item \$scope.rows;
 \end{itemize}
 
 \item[Funzionalità] \hfill
 \begin{description}
  \item[init()] \hfill \\
  \begin{description}
  	\item[Descrizione] \hfill \\
  Funzione di inizializzazione del \grassetto{@QueryController}. Imposta i valori iniziali delle variabili ed esegue la funzione \textit{getData()}.
  \end{description}
  
  \item[getData()] \hfill \\
  \begin{description}
  	\item[Descrizione] \hfill \\
  Richiama la richiesta RESTful \textit{query} sulla \$resource fornita dal QueryService.
  Con questa funzione vengono prelevati i dati legati alla Collection da visualizzare e vengono
  passati alla Collection view tramite il binding con lo \$scope. \\
  Se non vengono visualizzati risultati, ovvero se la Collection non esiste, avviene un redirect alla pagina
  che mostra l'errore 404.
  \end{description}
  
    \item[\$scope.createIndex(id)] \hfill
    \begin{description}
  	\item[Parametro] \hfill
  		\begin{description}
  			\item[id] \hfill \\
  			Id della query su cui creare l'indice.
       \end{description}
  	\item[Descrizione] \hfill \\
  Funzione ausiliaria che crea un indice a partire da una delle query presenti nella collection delle query.

    \end{description}
  
  \item[\$scope.numerify(num)] \hfill
   \begin{description}
  	\item[Parametro] \hfill
  		\begin{description}
  			\item[num] \hfill \\
  			Intero rappresentante il numero di pagine disponibili.
       \end{description}
  	\item[Descrizione] \hfill \\
 Funzione ausiliaria che crea un Array di dimensione \textit{num} che verrà poi usato dalla direttiva \textit{ng-repeat} di angular per la creazione del numero di pagine disponibili.
    \end{description}
 
  
  \item[\$scope.previousPage()] \hfill
  \begin{description}
  	\item[Descrizione] \hfill \\  
  Aggiunge allo \$scope la funzionalità di passaggio alla pagina di visualizzazione precedente alla corrente.
  Diminuisce di 1 l'indicatore della pagina corrente e esegue una nuova richiesta al server per i dati corrispondenti alla pagina precedente con \textit{getData()}.
    \end{description}

  
  \item[\$scope.nextPage()] \hfill
  \begin{description}
  	\item[Descrizione] \hfill \\  
  Aggiunge allo \$scope la funzionalità di passaggio alla pagina di visualizzazione successiva alla corrente.
  Aumenta di 1 l'indicatore della pagina corrente e esegue una nuova richiesta al server per i dati corrispondenti alla pagina successiva con \textit{getData()}.
    \end{description}

  
  \item[\$scope.toPage(index)] \hfill
  \begin{description}
  	\item[Parametro] \hfill
  		\begin{description}
  			\item[index] \hfill \\
  			Intero rappresentante il numero della pagina desiderata.
       \end{description}
  	\item[Descrizione] \hfill \\
   Aggiunge allo \$scope la funzionalità di passaggio alla pagina di visualizzazione specificata dal parametro \textit{index}.
  Aggiorna l'indicatore della pagina corrente settandolo all'indice passato per parametro alla funzione.
  Successivamente esegue una richiesta al server per ottenere i dati per la nuova visualizzazione.
    \end{description}

  
  \item[changeSort()] \hfill
  \begin{description}
  	\item[Descrizione] \hfill \\  
  Cambia la variabile \$current\_sort dello \$scope che regola l'ordine di visualizzazione dei Document, da discendente ad ascendente e viceversa.
    \end{description}

  
  \item[\$scope.columnSort(index)] \hfill
   \begin{description}
  	\item[Parametro] \hfill
  		\begin{description}
  			\item[index] \hfill \\
  			Numero intero che rappresenta l'indice della colonna da ordinare.
       \end{description}
  	\item[Descrizione] \hfill \\
   Funzione che cambia l'ordinamento di visualizzazione di una colonna. \\
  Se la colonna in questione è una colonna diversa da quella che precedentemente ordinata,
  la variabile \textit{current\_sorted\_column} dello \$scope viene aggiornata, altrimenti la funzione cambia solamente l'ordinamento della colonna.
    \end{description}
    
  
  
  \item[\$scope.deleteDocument(index)] \hfill
  \begin{description}
  	\item[Parametro] \hfill
  		\begin{description}
  			\item[index] \hfill \\
  			Indice che rappresenta il Document da eliminare.
       \end{description}
  	\item[Descrizione] \hfill \\
     Funzione che elimina il Document identificato dall'indice \textit{index}. Effettua una richiesta REST di rimozione alla risorsa gestita da \grassetto{@QueryService}.
  Se l'eliminazione ha successo viene lanciato l'aggiornamento della vista corrente con il Document eliminato, altrimenti si richiama la \textit{callback} di errore.
    \end{description}
    

  
 \end{description}
\end{description}

\paragraph{@RegisterController}
\begin{description}
 \item[Descrizione] \hfill \\
 Modulo che descrive il Controller della pagina di registrazione.
 
 \item[Utilizzo] \hfill \\
 Controller utilizzato da AngularJS per fornire e gestire i dati durante la registrazione.
 
 \item[Dipendendenze iniettate al controller] \hfill
 \begin{itemize}
  \item \$scope;
  \item \$location;
  \item @RegisterService;
  
 \end{itemize}
 
 \item[Attributi] \hfill
 \begin{itemize}
 \item  \$scope.credentials;
 \end{itemize}
 
 \item[Funzionalità] \hfill
 \begin{description}
  \item[signupForm()] \hfill
  \begin{description}
   	\item[Descrizione] \hfill \\
  Nel caso in cui il form di registrazione sia compilato con dati validi, richiama la richiesta RESTful \textit{query} sulla \$resource 
  fornita da \grassetto{@RegisterService}.
  Con questa funzione viene creato un nuovo utente sul server in base ai dati inseriti.
   \end{description}

  
 \end{description}
\end{description}


\paragraph{@UsersCollectionController}
\begin{description}
 \item[Descrizione] \hfill \\
 Modulo che descrive il Controller della Collection degli user.
 
 \item[Utilizzo] \hfill \\
 Controller utilizzato da AngularJS per fornire e gestire i dati della Collection degli user.
 
 \item[Dipendendenze iniettate al controller] \hfill
 \begin{itemize}
  \item \$scope;
  \item \$route;
  \item \$location;
  \item @UserCollectionService;
  \item @UserEditService;
 \end{itemize}
 
 \item[Attributi] \hfill
 \begin{itemize}
 \item \$scope.current\textunderscore sorted\textunderscore column;
 \item \$scope.column\textunderscore original\textunderscore name;
 \item \$scope.current\textunderscore sort;
 \item \$scope.current\textunderscore page;
 \item \$scope.current\textunderscore collection;
 \item \$scope.rows;
 \end{itemize}
 
 \item[Funzionalità] \hfill
 \begin{description}
  \item[init()] \hfill
  \begin{description}
   	\item[Descrizione] \hfill \\
  Funzione di inizializzazione del \grassetto{@CollectionController}. Imposta i valori iniziali delle variabili dello
  \$scope associate alla Collection view ed esegue la funzione \textit{getData()}.
    \end{description}

  \item[getData()] \hfill
    \begin{description}
  	\item[Descrizione] \hfill \\
  Richiama la richiesta RESTful \textit{query} sulla \$resource fornita da \grassetto{@UserCollectionService}.
  Con questa funzione vengono prelevati i dati legati alla Collection da visualizzare e vengono
  passati alla Collection view tramite il binding con lo \$scope. \\
  
      \end{description}

  \item[\$scope.numerify(num)] \hfill
  \begin{description}
  	\item[Parametro] \hfill
  		\begin{description}
  			\item[num] \hfill \\
  			Numero intero che rappresenta il numero di pagine disponibili.
       \end{description}
  	\item[Descrizione] \hfill \\
     Funzione ausiliaria che crea un Array di dimensione \textit{num} che verrà poi usato dalla directive   \textit{ng-repeat} di angular per la creazione del numero di pagine disponibili.
    \end{description}
  
  
  \item[\$scope.previousPage()] \hfill
    \begin{description}
  	\item[Descrizione] \hfill \\
  Aggiunge allo \$scope la funzionalità di passaggio alla pagina di visualizzazione precedente alla corrente.
  Diminuisce di 1 l'indicatore della pagina corrente e esegue una nuova richiesta al server per i dati corrispondenti alla pagina precedente con \textit{getData()}.
      \end{description}

  \item[\$scope.nextPage()] \hfill
    \begin{description}
  	\item[Descrizione] \hfill \\
  Aggiunge allo \$scope la funzionalità di passaggio alla pagina di visualizzazione successiva alla corrente.
  Aumenta di 1 l'indicatore della pagina corrente e esegue una nuova richiesta al server per i dati corrispondenti alla pagina successiva con \textit{getData()}.
      \end{description}

  \item[\$scope.toPage(index)] \hfill
  \begin{description}
  	\item[Parametro] \hfill
  		\begin{description}
  			\item[index] \hfill \\
  			Numero intero che rappresenta la pagina desiderata.
       \end{description}
  	\item[Descrizione] \hfill \\
       Aggiunge la funzionalità di passaggio alla pagina di visualizzazione specificata dal parametro \textit{index}. Aggiorna l'indicatore della pagina corrente settandolo all'indice passato per parametro alla funzione. Successivamente esegue una richiesta al server per ottenere i dati per la nuova visualizzazione.
    \end{description}
    

  \item[changeSort()] \hfill
    \begin{description}
  	\item[Descrizione] \hfill \\
  Cambia la variabile \$current\textunderscore sort dello \$scope che regola l'ordine di visualizzazione dei Document, da discendente ad ascendente e viceversa.
      \end{description}

  \item[\$scope.columnSort(index)] \hfill
  \begin{description}
  	\item[Parametro] \hfill
  		\begin{description}
  			\item[index] \hfill \\
  			Numero intero che rappresenta la colonna da ordinare.
       \end{description}
  	\item[Descrizione] \hfill \\
        Funzione che cambia l'ordinamento di visualizzazione di una colonna. \\
  Se la colonna in questione è una colonna diversa da quella che precedentemente ordinata, la variabile \textit{current\_sorted\_column} dello \$scope viene aggiornata, altrimenti la funzione cambia solamente
  l'ordinamento della colonna.
    \end{description}

  
  \item[\$scope.deleteDocument(index)] \hfill
  \begin{description}
  	\item[Parametro] \hfill
  		\begin{description}
  			\item[index] \hfill \\
  			Indice che rappresenta il Document da eliminare.
       \end{description}
  	\item[Descrizione] \hfill \\
      Funzione che elimina il profilo identificato dall'indice \textit{index}. Effettua una richiesta REST di rimozione alla risorsa gestita da \grassetto{@UserEditService} fornendo come argomenti l'identificativo della Collection correntemente visualizzata e l'indice del Document da eliminare. \\
  Se l'eliminazione ha successo viene lanciato l'aggiornamento della vista corrente con il Document eliminato, altrimenti si richiama la callback di errore.
    \end{description}
  
  
 \end{description}
\end{description}

\paragraph{@UsersController}
\begin{description}
 \item[Descrizione] \hfill \\
 Modulo che descrive il Controller della pagina di visualizzazione di un profilo utente da parte di un amministratore.
 
 \item[Utilizzo] \hfill \\
 Controller utilizzato da AngularJS per fornire e gestire i dati durante la visualizzazione di un document.
 
 \item[Dipendendenze iniettate al controller] \hfill
 \begin{itemize}
  \item \$scope;
  \item \$location;
  \item \$routeParams;
  \item @UserDataService;
  \item @UserEditService.
  
 \end{itemize}
 
 \item[Attributi] \hfill
 \begin{itemize}
  \item \$scope.current\textunderscore document;
  \item \$scope.values;
 \end{itemize}
 
 \item[Funzionalità] \hfill
 \begin{description}
  \item[UserDataService.query()] \hfill
  \begin{description}
  	\item[Parametro] \hfill
  		\begin{description}
  			\item[\$scope.current\textunderscore document] \hfill \\
  			Indicatore del Document corrente.
       \end{description}
  	\item[Descrizione] \hfill \\  
  Richiama la richiesta RESTful \textit{query} sulla \$resource fornita da \grassetto{@UserDataService}.
  Con questa funzione vengono prelevati i dati legati al document da visualizzare e caricati sullo scope.
 
    \end{description}
 
  
  \item[\$scope.deleteDocument(index)] \hfill
  \begin{description}
  	\item[Parametro] \hfill
  		\begin{description}
  			\item[index] \hfill \\
  			Indice del profilo da eliminare.
       \end{description}
  	\item[Descrizione] \hfill \\
   Funzione che elimina il profilo identificato dall'indice \textit{index}. Effettua una richiesta REST di rimozione alla risorsa gestita da \grassetto{@UserEditService} fornendo come argomenti l'identificativo della Collection correntemente visualizzata e l'indice del Document da eliminare. \\
  Se l'eliminazione ha successo viene lanciato l'aggiornamento della vista corrente con il Document eliminato, altrimenti si richiama la callback di errore.
    \end{description}
 
 \end{description}
\end{description}

\paragraph{@UsersEditController}
\begin{description}
 \item[Descrizione] \hfill \\
 Modulo che descrive il Controller della pagina di modifica di un profilo.
 
 \item[Utilizzo] \hfill \\
 Controller utilizzato da AngularJS per fornire e gestire i dati durante la modifica di un profilo.
 
 \item[Dipendendenze iniettate al controller] \hfill
 \begin{itemize}
  \item \$scope;
  \item \$location
  \item @UserEditService;
  \item \$routeParams.
  
 \end{itemize}
 
 \item[Attributi] \hfill
 \begin{itemize}
    \item  \$scope.current\textunderscore document;
    \item  \$scope.original\textunderscore data;
    \item  \$scope.original\textunderscore keys;
 \end{itemize}
 
 \item[Funzionalità] \hfill
 \begin{description}
  \item[UserEditService.query()] \hfill
  \begin{description}
  	\item[Parametro] \hfill 
  		\begin{description}
  			\item[\$routeParams.user\_id] \hfill \\
  			Campo di \$routeParams contenente il codice identificativo dell'utente di cui richiedere il Document.
       \end{description}
  	\item[Descrizione] \hfill \\
    Richiama la richiesta RESTful \textit{query} sulla \$resource fornita da \grassetto{@UserEditService}.
  Con questa funzione vengono prelevati i dati legati alla Collection da visualizzare e vengono
  passati alla Collection view tramite il binding con lo \$scope. 
  
    \end{description}
 

  \item[edit\textunderscore document()] \hfill
  \begin{description}
   	\item[Descrizione] \hfill \\
  Funzione che invia al server il documento modificato nel client.
  Assembla un JSON a partire da original\textunderscore keys e original\textunderscore data. Quest'ultima variabile nel frattempo è stata modificata con i nuovi valori del documento.
    \end{description}

  \item[delete\textunderscore document()] \hfill
    \begin{description}
  	\item[Descrizione] \hfill \\ 
 Funzione che elimina il profilo identificato dall'indice \textit{index}. Effettua una richiesta REST di rimozione alla risorsa gestita da \grassetto{@UserEditService} fornendo come argomenti l'identificativo della Collection correntemente visualizzata e l'indice del Document da eliminare. \\
  Se l'eliminazione ha successo viene lanciato l'aggiornamento della vista corrente con il Document eliminato, altrimenti si richiama la callback di errore.
      \end{description}

 \end{description}
\end{description}

\subsubsection{::Directives}

\paragraph{@Passwrod\_check}
\begin{description}
\item [Dipendenze iniettate] \hfill
\begin{itemize}
	\item scope;
	\item elem;
	\item attrs;	
	\item ctrl.
\end{itemize}
 \item[Descrizione] \hfill \\
 Directive utilizzata per verificare che il contenuto di due campi di un form sia uguale.
 Viene utilizzata per verificare che l'utente inserisca correttamente entrambe le password durante la registrazione al fine di evitare richieste inutili al server e per prevenire errori piuttosto che correggerli.
 Se le due password non sono uguali il form non viene validato e non può essere inviato al server.

\end{description}


\paragraph{@Unique}
\begin{description}
\item [Dipendenze iniettate] \hfill
\begin{itemize}
	\item \$http;
	\item \$timeout.
\end{itemize}
 \item[Descrizione] \hfill \\
 Directive utilizzata per verificare che la mail inserita dall'utente durante la registrazione non sia già utilizzata.
  Se la mail è gia in utilizzo non viene validato il form e non si può procedere alla registrazione.

\end{description}

\subsection{::ModelClient}

\subsubsection{::Services}
I Servizi del Client sono i tramite tra il Client ed il Server. I componenti di questa sezione sono gli esecutori materiali delle richieste REST per ottenere le informazioni dei dati presenti nei database di analisi e utenti presenti nel Server. Essi gestiscono delle risorse su cui i Controller associati manipolano le informazioni richieste dal Client.

\paragraph{@AuthService}
\begin{description}
 \item[Descrizione] \hfill \\
 Modulo che espone un oggetto \$resource legato all'autenticazaione utente.
 \item[Utilizzo] \hfill \\
 Viene utilizzato per astrarre il processo di verifica delle credenziali eseguito dal server. Questo modulo esegue una richiesta alle API del server per stabilire se le credenziali inserite dall'utente sono valide all'autenticazione. Il modulo ritorna una \$resource per rendere più semplice ed immediato il dialogo tra il Client e il sistema di autenticazione. \\
 La \$resource esportata da questo modulo viene utilizzata nel \grassetto{@LoginCtrl} per verificare le credenziali inserite lato Client.
 \item[Funzionalità] \hfill \\
 \emph{Dipendenze iniettate:}
 \begin{itemize}
  \item \$resource.
 \end{itemize}
 L'istruzione di ritorno crea una nuova \$resource che espone il metodo \textit{login} che verrà invocato dal \grassetto{@LoginCtrl}.
\end{description}

\paragraph{@CollectionDataService}
\begin{description}
 \item[Descrizione] \hfill \\
 Modulo che espone un oggetto \$resource legato ad una particolare Collection. 
 
 \item[Utilizzo] \hfill \\
 Crea ed espone una \$resource legata ad una particolare Collection, su cui il \grassetto{@CollectionController} invocherà delle richieste RESTful aggiungendo i parametri necessari. La risorsa fornita conterrà i Document ad essa correlati e quest'ultimi verranno visualizzati in base alle opzioni della richiesta RESTful.
 
 \item[Funzionalità] \hfill \\
 %qui ci andrebbe tipo uno screenshot del codice, devo ancora capire come esporre le informazioni del sorgente
 \emph{Dipendenze iniettate:}
 \begin{itemize}
  \item \$resource.
 \end{itemize}
 L'istruzione di ritorno crea una nuova \$resource che modella il comportamento della Collection che verrà individuata tramite l'attributo \textit{col\_id}.
\end{description}

\paragraph{@CollectionListService}
\begin{description}
 \item[Descrizione] \hfill \\
 Modulo che espone un oggetto \$resource legato alla lista delle Collection disponibili.
 
 \item[Utilizzo] \hfill \\
 Viene creata una \$resource che modella ad alto livello la lista delle Collection disponibili, per essere utilizzata nel \grassetto{@DashboardController}.
 
 \item[Funzionalità] \hfill \\
 \emph{Dipendenze iniettate:}
 \begin{itemize}
  \item \$resource.
 \end{itemize}
 L'istruzione di ritorno crea una \$resource che contiene la lista delle Collection disponibili, con 
 cui interagire tramite richieste RESTful.
 
\end{description}

\paragraph{@DocumentDataService}
\begin{description}
 \item[Descrizione] \hfill \\
 Modulo che espone un oggetto \$resource legato ad un particolare Document.
 
 \item[Utilizzo] \hfill \\
 Questo servizio ritorna un oggetto \$resource legato ad un particolare Document, che verrà  utilizzato dal \grassetto{@DocumentController} tramite richieste RESTful per visualizzarne gli attributi.
 
 \item[Funzionalità] \hfill \\
 \emph{Dipendenze iniettate:}
 \begin{itemize}
  \item \$resource.
 \end{itemize}
 L'oggetto ritornato da questa Factory lega la \$resource ritornata al Document \textit{doc\_id} della Collection \textit{col\_id}. Inoltre, stabilisce che le richieste \textit{query} vengano effettuate tramite metodo GET. 
 
\end{description}

\paragraph{@DocumentEditservice}
\begin{description}
 \item[Descrizione] \hfill \\
 Modulo che espone un oggetto \$resource legato ad un particolare Document modificabile.
 
 \item[Utilizzo] \hfill \\
 Questo servizio ritorna un oggetto \$resource legato ad un particolare Document, che verrà  utilizzato dal \grassetto{@DocumentEditController} tramite 
 richieste RESTful, sia per visualizzarne  le informazioni, sia per modificarne le proprietà.
 
 \item[Funzionalità] \hfill \\
 \emph{Dipendendenze iniettate:}
 \begin{itemize}
  \item \$resource.
 \end{itemize}
 L'oggetto ritornato da questa Factory lega la \$resource ritornata al Document \textit{doc\_id} della Collection
 \textit{col\_id}. Inoltre, oltre alla visualizzazione, descrive che le richieste di \textit{update} vengono effettuate
 con richieste di \textit{PUT}, mentre le richieste di \textit{remove} useranno il metodo http \textit{DELETE}.
 
\end{description}

\paragraph{@IndexService}
\begin{description}
 \item[Descrizione] \hfill \\
 Modulo che gestisce una \$resource legata ad un indice.
 \item[Utilizzo] \hfill \\
 Questo servizio ritorna una \$resource legata ad un indice fornito dal server, per essere utilizzata dall'\grassetto{@IndexCtrl} per gestire i vari 
 indici visualizzati nel Client. Il presente modulo viene inoltre iniettato nel \grassetto{@QueryCtrl} per creare un nuovo indice a partire da una 
 query data. Questa risorsa esportata è dotata di tre metodi di accesso e modifica dei dati dell'indice, ovvero
 \textit{query}, \textit{insert} e \textit{remove}.
 \item[Funzionalità] \hfill \\
 \emph{Dipendendenze iniettate:}
 \begin{itemize}
  \item \$resource.
 \end{itemize}
 La \$resource esportata fornisce all'\grassetto{@IndexCtrl} del Client una modo diretto per interagire con un indice fornito dal server. Questa risorsa dispone di un metodo \textit{query} per eseguire una GET RESTful per ottenere una lista degli indici presenti sul Server.
 Il metodo \textit{insert} crea un nuovo indice a partire da una delle query più utilizzate. \textit{remove} invece rimuove un indice dalla lista degli indici presenti sul server.
\end{description}

\paragraph{@LogoutService}
\begin{description}
 \item[Descrizione] \hfill \\
 Modulo che fornisce la \$resource legata al servizio di logout di un utente autenticato del Client.
 \item[Utilizzo] \hfill \\
  Questo servizio fornisce un'astrazione del metodo di logout eseguito dal server. La \$resource esportata dal servizio viene utilizzata
  nel \grassetto{@NavBarCtrl} per effettuare il logout di un autente autenticato, tramite il metodo esposto \textit{logout}. 
 \item[Funzionalità] \hfill \\
 \emph{Dipendendenze iniettate:}
 \begin{itemize}
  \item \$resource.
 \end{itemize}
 La risorsa ritornata dal modulo astrae la connessione con il server quando si effettua il logout dal sistema. La funzione \textit{logout} 
 associata alla \$resource ritornata esegue una richiesta RESTful GET al Server per disconnettere l'utente che ha richiesto l'operazione.
\end{description}

\paragraph{@ProfileDataService}
\begin{description}
 \item[Descrizione] \hfill \\
 Il servizio ritorna una \$resource legata ad un particolare profilo utente presente nel Server.
 \item[Utilizzo] \hfill \\
 La risorsa ritornata da questo modulo viene utilizzata dal \grassetto{@ProfileCtrl} per dialogare con le informazioni associate ad un singolo utente.
 Il metodo \textit{query} associato alla risorsa serve per ottenere i dati associati al profilo utente associato.
 \item[Funzionalità] \hfill \\
 \emph{Dipendendenze iniettate:}
 \begin{itemize}
  \item \$resource.
 \end{itemize}
 La funzione \textit{query} esegue una richiesta GET RESTful al server per ottenere le informazioni di un utente. Tali informazioni vengono 
 poi passate al \grassetto{@ProfileCtrl} per essere fornite in visualizzazione al Client che le ha richieste.
\end{description}

\paragraph{@ProfileEditService}
\begin{description}
 \item[Descrizione] \hfill \\
 Il servizio ritorna una \$resource modificabile legata ad un particolare profilo utente presente nel Server.
 \item[Utilizzo] \hfill \\
 La risorsa ritornata da questo modulo viene utilizzata dal \grassetto{@ProfileEditCtrl} per dialogare con le informazioni associate ad un singolo utente.
 Alla risorsa ritornata vengono associati tre metodi: \textit{query}, \textit{update} e \textit{remove}.
 \item[Funzionalità] \hfill \\
 \emph{Dipendendenze iniettate:}
 \begin{itemize}
  \item \$resource.
 \end{itemize}
 La funzione \textit{query} esegue una richiesta GET RESTful al server per ottenere le informazioni di un utente. \textit{update} invia 
 nuove informazioni al Server che andranno a sovrascrivere quelle già presenti per aggiornarle. Il metodo \textit{remove} invece richiede al 
 Server di eseguire l'eliminazione di un dato profilo utente. Lato Server viene effettuato un controllo per verificare se l'utente che 
 ha richiesto l'eliminazione ha le credenziali per eliminare u utente. Nel caso in cui il controllo abbia esito positivo, il profilo
 viene eliminato dal Server.
\end{description}

\paragraph{@QueryService}
\begin{description}
 \item[Descrizione] \hfill \\
 Modulo che esporta una \$resource legata ad una query effettuabile sul database di analisi.
 \item[Utilizzo] \hfill \\
 La risorsa ritornata da questo servizio viene utilizzata nel \grassetto{@QueryCtrl}. Il suddetto controller utilizza la risorsa per richiedere l'esecuzione 
 di una query al Server inviando come argomenti i dati presenti nello \$scope in quel momento. \\
 Il modulo associa alla risorsa ritornata le funzioni \textit{query} e \textit{remove}.
 \item[Funzionalità] \hfill \\
 \emph{Dipendendenze iniettate:}
 \begin{itemize}
  \item \$resource.
 \end{itemize}
 La funzione \textit{query} esegue una richiesta \textit{GET REST} al Server per richiedere l'esecuzione di una query sui dati del database di analisi. 
 I parametri su cui effettuare la query sono inviati dal Client sulla base delle condizioni immesse dall'utente e memorizzate nello \$scope.
 La funzione \textit{remove}, invece, richiede al Server l'eliminazione del Document individuato dall'indice \textit{index} presente nello \$scope. 
\end{description}

\paragraph{@RegisterService}
\begin{description}
 \item[Descrizione] \hfill \\
 Servizio che ritorna una \$resource legata ad un profilo utente intento ad effettuare la registrazione. 
 \item[Utilizzo] \hfill \\
 La risorsa ritornata da questo servizio viene utilizzata come tramite tra il Client e il Server per gestire la registrazione di un nuovo utente. 
 La risorsa viene utilizzata dal \grassetto{@RegisterCtrl} per richiedere la registrazione di un nuovo utente. Viene infatti esposto il metodo \textit{register} 
 che consente di inviare una richiesta di registrazione al Server con le credenziali inserite nell'apposito form del Client. 
 \item[Funzionalità] \hfill \\
 \emph{Dipendendenze iniettate:}
 \begin{itemize}
  \item \$resource.
 \end{itemize}
 La funzione \textit{register} esportata con la risorsa esegue una richiesta \textit{POST REST} al Server inviando con essa le credenziali dell'utente 
 attualmente presenti nello \$scope.
\end{description}

\paragraph{@UserCollectionService}
\begin{description}
 \item[Descrizione] \hfill \\
 Servizio che fornisce una \$resource legata alla lista degli utenti ordinari non amministratori.  
 \item[Utilizzo] \hfill \\
 La risorsa messa a disposizione dallo \grassetto{@UserCollectionService} contiene la lista di tutti gli utenti 
 del sistema. Questa \$resource viene utilizzata dallo \grassetto{@UsersCollectionCtrl} per visualizzare 
 correttamente tutti gli utenti del sistema ed organizzarli nell'apposita vista. Il modulo esporta 
 con la \$resource il metodo \textit{query}. 
 \item[Funzionalità] \hfill \\
 \emph{Dipendendenze iniettate:}
 \begin{itemize}
  \item \$resource.
 \end{itemize}
 La funzione \textit{query} effettua una richiesta \textit{REST GET} al Server per ottenere la lista degli 
 utenti.
\end{description}

\paragraph{@UserDataService}
\begin{description}
 \item[Descrizione] \hfill \\
 Servizio che fornisce una \$resource legata ad un particolare utente non amministratore registrato al sistema.
 \item[Utilizzo] \hfill \\
 Questa risorsa viene utilizzata nello \grassetto{@UserCtrl} per richiedere al Server le informazioni associate ad un utente. Ogni utente è identificato dal proprio \textit{id} e questo campo viene passato dal Client per individuare un singolo utente, se presente, nel Server. La risorsa viene esportata 
con il metodo \textit{query}, per ottenere i dati voluti.
 \item[Funzionalità] \hfill \\
 \emph{Dipendendenze iniettate:}
 \begin{itemize}
  \item \$resource.
 \end{itemize}
 Il metodo \textit{query} esegue una richiesta \textit{GET REST} al Server richiedendo le informazioni di un utente specifico, individuato da un identificativo univoco, inviato dal Client come argomento della richiesta.
\end{description}

\paragraph{@UserEditService}
\begin{description}
 \item[Descrizione] \hfill \\
 Servizio che fornisce una \$resource legata ad un particolare utente con campi modificabili registrato al sistema.
 \item[Utilizzo] \hfill \\
  Questa risorsa viene utilizzata nello \grassetto{@UserEditCtrl} per richiedere al Server le informazioni associate ad un utente. L'argomento della richiesta al 
  Server è l dell'utente. Le funzioni esportate con la \$resource sono: \textit{query}, \textit{update} e 
  \textit{remove}.
 \item[Funzionalità] \hfill \\
 \emph{Dipendendenze iniettate:}
 \begin{itemize}
  \item \$resource.
 \end{itemize}
  L'oggetto ritornato da questo servizio lega la \$resource ritornata all'utente con il campo 
  \textit{id} uguale a quello utilizzato per la ricerca. Inoltre, oltre alla visualizzazione espressa tramite il metodo \textit{query} realizzato con una richiesta \textit{GET} RESTful, descrive che le richieste di \textit{update} vengano effettuate
 con richieste di \textit{PUT}, mentre le richieste di \textit{remove} useranno il metodo http \textit{DELETE}.
\end{description}


\newpage


\section{Diagrammi di sequenza}
\label{sequenza}
\subsection{Modifica della View con successo}
\immagine{./SequenzaSuccesso}{Diagramma sequenza: Modifica View con successo}
Il diagramma precedente illustra la sequenza di operazioni che avviene alla richiesta da parte di un utente di visualizzare un document.
Dopo aver cliccato il bottone di visualizzazione del document avviene una serie di richieste a cascata fino ad arrivare al server. Quest'ultimo verifica l'autenticità della richiesta e risponde con il documento cercato.
Questo documento viene ritornato dal servizio al controller di gestione dei documenti che esegue la callback "success" e prepara il json ad essere visualizzato.
La view viene poi aggiornata con i nuovi dati.

\subsection{Modifica della View con insuccesso}
\immagine{./SequenzaErrore}{Diagramma sequenza: Modifica View con insuccesso}
Questo diagramma illustra la stessa operazione del precedente con esito negativo. Il server rifiuta la richiesta e risponde con un codice di errore.
Il controller esegue la callback "error" e la view verrà aggiornata con un messaggio di errore.
%FINE DOCUMENTO NON CANCELLARE
\end{document}
