
%includo il file che contiene la versione dei documenti
\newcommand{\versioneAnalisiDeiRequisiti}{2.2.0}			
\newcommand{\versioneNormeDiProgetto}{2.2.0}			
\newcommand{\versioneGlossario}{2.2.0}			
\newcommand{\versionePianoDiQualifica}{2.2.0}			
\newcommand{\versionePianoDiProgetto}{2.2.0}	
\newcommand{\versioneStudioDiFattibilita}{2.2.0}
\newcommand{\versioneSpecificaTecnica}{2.2.0}


\newcommand{\Versione}{\versioneSpecificaTecnica{}}	  %Versione Finale
\newcommand{\Data}{xxxx-xx-xx}				           %Data di creazione
\newcommand{\DataUltimaModifica}{xxxx-xx-xx}
\newcommand{\TipoDocumento}{Specifica Tecnica}	       %tipo documento

%includo il file header.tex (logo grande in prima pagina piu qualche altra regola)
%questo file contiene impostazioni comuni per tutte i documenti

%definizione packages utilizzati
\documentclass[a4paper]{article}
\usepackage[utf8x]{inputenc}
\usepackage{enumitem}
\usepackage[italian]{babel}
\usepackage{latexsym}
\usepackage{xparse}
\usepackage{float}
\usepackage{subfloat}
\usepackage{subfig}
\usepackage{fancyhdr}
\usepackage{eurofont}
\usepackage{lastpage}
\usepackage{graphicx}
\usepackage{textcomp}
\usepackage{booktabs}
\usepackage{color}
\usepackage{lscape}
\usepackage{hyperref}
\hypersetup{colorlinks=true, linkcolor=black, anchorcolor=red, urlcolor=blue}
\usepackage{longtable}
\usepackage{tabularx}
\usepackage{abstract}
\usepackage{appendix}
\usepackage{multicol}
\usepackage{bmpsize}
\usepackage[all]{hypcap}
\usepackage{titlesec}
\usepackage{indentfirst}
\usepackage{lipsum,titletoc}

%\setcounter{secnumdepth}{4}

%****************INIZIO GESTIONE SUBSECTION MULTIPLE
\makeatletter
\newcommand\level[1]{%
  \ifcase#1\relax\expandafter\chapter\or
    \expandafter\section\or
    \expandafter\subsection\or
    \expandafter\subsubsection\else
    \def\next{\@level{#1}}\expandafter\next
  \fi}
\newcommand{\@level}[1]{%
  \@startsection{level#1}
    {#1}
    {\z@}%
    {-3.25ex\@plus -1ex \@minus -.2ex}%
    {1.5ex \@plus .2ex}%
    {\normalfont\normalsize\bfseries}}

\newdimen\@leveldim
\newdimen\@dotsdim
{\normalfont\normalsize
 \sbox\z@{0}\global\@leveldim=\wd\z@
 \sbox\z@{.}\global\@dotsdim=\wd\z@
}

\newcounter{level4}[subsubsection]
\@namedef{thelevel4}{\thesubsubsection.\arabic{level4}}
\@namedef{level4mark}#1{}
\def\l@section{\@dottedtocline{1}{0pt}{\dimexpr\@leveldim*4+\@dotsdim*1+6pt\relax}}
\def\l@subsection{\@dottedtocline{2}{0pt}{\dimexpr\@leveldim*5+\@dotsdim*2+6pt\relax}}
\def\l@subsubsection{\@dottedtocline{3}{0pt}{\dimexpr\@leveldim*6+\@dotsdim*3+6pt\relax}}
\@namedef{l@level4}{\@dottedtocline{4}{0pt}{\dimexpr\@leveldim*7+\@dotsdim*4+6pt\relax}}

\count@=4
\def\@ncp#1{\number\numexpr\count@+#1\relax}
\loop\ifnum\count@<100
  \begingroup\edef\x{\endgroup
    \noexpand\newcounter{level\@ncp{1}}[level\number\count@]
    \noexpand\@namedef{thelevel\@ncp{1}}{%
      \noexpand\@nameuse{thelevel\@ncp{0}}.\noexpand\arabic{level\@ncp{1}}}
    \noexpand\@namedef{level\@ncp{1}mark}####1{}%
    \noexpand\@namedef{l@level\@ncp{1}}%
      {\noexpand\@dottedtocline{\@ncp{1}}{0pt}{\the\dimexpr\@leveldim*\@ncp{5}+\@dotsdim*\@ncp{0}\relax}}}%
  \x
  \advance\count@\@ne
\repeat
\makeatother
\setcounter{secnumdepth}{100}
\setcounter{tocdepth}{100}
%****************FINE GESTIONE SUBSECTION MULTIPLE

%impostazioni relative alla visualizzazione delle section 
%nell'indice
\titlecontents{section}
[0pt]%left indent
{\bfseries}
{\contentslabel{2.3em}}
{\hspace*{-2.3em}}
{\hfill\contentspage}
[]%separator


\oddsidemargin=.15in
\evensidemargin=.15in
\textwidth=6in
\topmargin=-.5in
\parindent=0in
\headheight=1in
\DeclareMathSizes{10}{10}{10}{10} %per piano qualifica
\pagestyle{fancy}
\lhead{
\bfseries {\Large \TipoDocumento}\\
\bfseries Versione: \Versione\\
}
\chead{}
\lhead{
\includegraphics[scale=0.455]{../Logo&Header/apertureHead.png}
}
\lfoot{\bfseries \TipoDocumento{} v\Versione}
\cfoot{}
\rfoot{\thepage\ of \mypageref{LastPage}}
\newcommand{\mypageref}[1]{
\hypersetup{linkcolor=black}\pageref{#1}\hypersetup{linkcolor=black}}
%\userpackage{lipsum}
\renewcommand{\footrulewidth}{0.4pt}

%definizioni comandi comuni utilizzati
\newcommand{\numref}[1]{\textsl{\nameref{#1} (\ref{#1})}}
\newcommand{\NomeGruppo}{Aperture Software}
\newcommand{\Progetto}{MaaP: MongoDB as an admin Platform}
\newcommand{\Prop}{CoffeeStrap}

%definizione tecnologie
\newcommand{\Node}{Node.js}
\newcommand{\NodeJS}{Node.js}
\newcommand{\Nodejs}{Node.js}

\newcommand{\mongodb}{MongoDB}

%tanti sub quanti ne vogliamo! :)
\newcommand{\subsubsubsection}{\level{4}}
\newcommand{\subsubsubsubsection}{\level{5}}
\newcommand{\subsubsubsubsubsection}{\level{6}}
\newcommand{\subsubsubsubsubsubsection}{\level{7}}
\newcommand{\subsubsubsubsubsubsubsection}{\level{8}}


%definizione comando per parola glossario
\newcommand{\gloss}[1]{\emph{#1}\ped{\emph{\tiny{G}}}}

\newcommand{\grassetto}{\textbf}

%per inserire immagini
\newcommand{\immagine}[2]{ 
\begin{center}
\begin{figure}[H]
\includegraphics[width=\textwidth]{{{#1}}}
\caption{#2}
\label{#1}
\end{figure}
\end{center}
}

\newcommand{\Glossario}{
Al fine di evitare ogni ambiguità nella comprensione del linguaggio utilizzato nel presente documento e, in generale, nella documentazione fornita dal gruppo \NomeGruppo{}, ogni termine tecnico, di difficile comprensione o di necessario approfondimento verrà inserito nel documento \emph{Glossario\_{}v\versioneGlossario{}.pdf}.\\
Saranno in esso definiti e descritti tutti i termini in corsivo e allo stesso tempo marcati da una lettera "G" maiuscola in pedice nella documentazione fornita.
}

\newcommand{\Prodotto}{
Lo scopo del prodotto è produrre un framework per generare interfacce web di amministrazione dei dati di business basati sullo stack \Nodejs{} e \mongodb{}.\\
L'obiettivo è quello di semplificare il lavoro allo sviluppatore che dovrà rispondere in modo rapido e standard alle richieste degli esperti di business.
}

%inizio pagina del documento 
\begin{document}
\thispagestyle{empty}

\begin{center}\centerline{
%inserisco il logo grande della prima pagina
\includegraphics[scale=0.8]{../Logo&Header/logo.png}}

%metto il link dell'email sotto al logo
%{\href{mailto:ApertureSWE@gmail.com}{\color[rgb]{0.39,0.37,0.38}%ApertureSWE@gmail.com}}\\ [3pc]

\vspace{0.5in}

%titolo del progetto
{\Huge {\Progetto}}\\[.5pc]

\underline{\hspace{6in}}\\[8pc]

{\Huge {\TipoDocumento}}\\[1pc]
%{\emph{Versione \Versione}}\\
\end{center}

%\vspace{.05in}

%\vspace{.05in}

%informazioni documento
\begin{center}
%\section{Informazioni documento}
\begin{tabular}{r|l}
%\textbf{Nome} &\TipoDocumento \\
\textbf{Versione} & \Versione{} \\
\textbf{Data creazione} & \Data{2014-01-24} \\
\textbf{Data ultima modifica} & \DataUltimaModifica{} \\
\textbf{Stato del Documento} & Formale \\		          %CAMBIARE QUI
\textbf{Uso del Documento} & Esterno \\			          %CAMBIARE QUI
\textbf{Redazione} &  Pinato Giacomo,nome2,...\\		        %CAMBIARE QUI
\textbf{Verifica} & nome1,nome2,...\\			        %ED ANCHE QUI!
\textbf{Approvazione} & nome1,nome2,...\\				 %CAMBIARE QUI
\textbf{Distribuzione} & \parbox[t]{4cm}{\NomeGruppo{}}\\
\end{tabular}
\end{center}

\vspace{0.05in}

%inizio sommario del documento
\begin{abstract}
\begin{center}
Questo documento si propone di presentare la Specifica tecnica e architetturale per la Realizzazione del prodotto \textbf{MaaP}.
\end{center}
\end{abstract}

%\vspace{.4in}

%seconda pagina, diario delle modifiche
\newpage
Diario delle modifiche
\begin{center}
\begin{longtable}{|c|c|c|p{0.5\linewidth}|}
\toprule
\textbf{Versione} & \textbf{Data} & \textbf{Autore} & \textbf{Modifiche effettuate}\\

%aggiungere qui una midrule per aggiungere una nuova riga alla tabella
\midrule
1.2.0 & 2014-02-xx & ... (RE) & Approvazione documento\\
\midrule
1.1.0 & 2014-02-xx & ... (VE) & Verifica documento\\
\midrule
1.0.0 & 2014-03-05 & Giacomo Pinato (PR) & Componenti e Classi\\
\midrule
1.0.0 & 2014-01-25 & Giacomo Pinato (PR) & Tecnologie Utilizzate\\
\midrule
1.0.0 & 2014-01-25 & Fabio Miotto (PR) & Tecnologie Utilizzate\\
\midrule
1.0.0 & 2014-01-24 & Giacomo Pinato (PR) & Prima stesura del documento\\

\bottomrule
\caption{Registro delle modifiche}
\label{tab:changelog}

\end{longtable}
\end{center}

%terza pagina Indice (viene aggiornato in automatico con due compilazioni)
\newpage
\tableofcontents

%pagine successive hanno la lista di tabelle e lista delle figure
%(vengono aggiornate in automatico)
%\newpage
%\listoftables
%\listoffigures

%qui inizia la prima pagina ufficiale
\newpage
\section{Introduzione}
\label{1.0}
\subsection{Scopo del documento}
\label{1.1}
Il presente documento ha lo scopo di definire la progettazione ad alto livello del progetto \textbf{MaaP}, a partire dai requisiti individuati durante l'Analisi. Verrà presentata l'architettura generale secondo la quale saranno organizzate le varie componenti software, i \gloss{Design Pattern} e le tecnologie utilizzate per poi descrivere più dettagliatamente le varie componenti e relative dipendenze.

\subsection{Scopo del prodotto}
\label{1.2}
\Prodotto{}

\subsection{Glossario}
\subsubsection{Glossario2}
\subsubsubsection{Glossario3}
\subsubsubsubsection{Glossario4}
\label{1.3}
\Glossario{}

\subsection{Riferimenti}
\label{1.4}

\subsubsection{Normativi}
\label{1.4.1}
\begin{itemize}
\item \grassetto{Analisi dei requisiti}: Analisi\_{}dei\_{}Requisiti\_{}v\versioneNormeDiProgetto{}.pdf
\item Norme di Progetto: Norme\_{}di\_{}Progetto\_{}v\versioneNormeDiProgetto{}.pdf  (allegato alla presente documentazione)\\
\end{itemize}

\subsubsection{Informativi}
\label{1.4.2}
\begin{itemize}
\item Learning Node: O'Reilly Shelley Powers
\item AngularJS: O'Reilly Brad Green e Shyam Seshadri
\item Software Engineering (8th edition), Ian Sommerville, Pearson Education | Addison-Wesley
\item Design Patterns, E. Gamma, R. Helm, R. Johnson, J. Vlissides, Pearson Education | Addison-Wesley
\item Dall'idea al codice con UML 2       L. Baresi, L. Lavazza, M. Pianciamore, Pearson Education
\end{itemize}

\newpage
\section{Tecnologie utilizzate}
In questa sezione verrano elencate e descritte le tecnologie che si utilizzeranno durante lo sviluppo del progetto. In particolare la colonna portante del progetto sar\`{a} lo stack MEAN, ovvero MongoDB, Express, Angular e Node.js.

\subsection{MongoDB}
Il database con il quale la nostra applicazione dovrà interagire è realizzato con MongoDB, come specificato nel capitolato. Questa tecnologia offre i seguenti vantaggi:
\begin{itemize}
\item Facile indicizzazione: Ogni campo in MongoDB puo' diventare un indice;
\item Bilanciamento di carico: MongoDB scala orizzontalmente molto facilmente grazie all'utilizzo di shard;
\item Integrazione con Javascript: Query o altre funzioni scritte in Javascript possono essere eseguite direttamente dal dabatase;
\end{itemize}

\subsection{Javascript}
Si è deciso di utilizzare Javascript in quanto è il linguaggio su cui si basano tutte le altre tecnologie che andremo ad utilizzare, e offre quindi una facile integrazione, oltre ad essere un ottimo linguaggio per applicazioni web e client side.


\subsection{NodeJs}
Si è deciso di utilizzare il linguaggio Node.js in quanto consigliato dal capitolato e adatto al progetto. Le sue caratteristiche piu' vantaggiose sono:
\begin{itemize}
\item Modello event-driven: ovvero "programmazione ad eventi", che si basa su un concetto semplice: il flusso del programma non segue un corso specifico ma è guidato dalle azioni dell'utilizzatore;
\item Modello asincrono: grazie a questa caratteristica è possibile ridurre al minimo i tempi di morti in quanto, nell’attesa dell completamento di una operazione, si procede con altri flussi logici. 
\item Grande scalabilità: Grazie al modo in cui è implementato, Node.js riesce ad essere largamente scalabile con minimo sforzo.
\end{itemize}

\subsection{jQuery}
Per migliorare e semplificare la scrittura di funzioni nel linguaggio JavaScript, si è deciso di adottare il framework jQuery. I vantaggi sono:
\begin{itemize}
\item Semplicità: l'utilizzo di jQuery semplifica e facilita la scrittura di codice JavaScript, inoltre offre plug-in on-line per fornire nuove funzionalità.
\end{itemize}
Svantaggi:
\begin{itemize}
\item Pericolosità: jQuery offre funzionalità e plug-in molto utili, ma non tutto è compatibile con i vari browser, inoltre alcune funzionalità possono essere vecchie, non aggiornate o scritte male.
\end{itemize}

\subsection{JSON}
Rappresenta il tipo di messaggi con cui client e server si scambiano informazioni. I vantaggi offerti sono:
\begin{itemize}
\item Semplicità: i messaggi JSON sono più corti rispetto ad altri formati di interscambio, e vengono eseguiti più velocemente dal parser. JSON inoltre risulta più semplice e immediato rispetto ad esempio a XML.
\end{itemize}
Svantaggi:
\begin{itemize}
\item Restrittività: JSON è meno restrittivo rispetto ad XML, e questo può permettere di inserire errori nello scambio di messaggi.
\end{itemize}

\subsection{AngularJs}

\begin{itemize}
\item Two Way Data-Binding: Una delle caratteristiche principali di angular. Le modifiche apportate al model si rifletto direttamente sugli elementi del DOM, e le modifiche al DOM si ripercuotono automaticamente sul model. Questo alleggerisce tremendamente il codice necessario a controllare ad ascoltare e gestire il DOM, automatizzando il processo. E noi sappiamo che automatico \`{e} bene.
\item Templates: I template HTML sono parsati dal browswe nel DOM,il quale costituisce poi l’input per il compilatore Angular. Quest’ultimo poi crea il data binding tra il DOM e lo scope dei dati. Uno dei piu’ grandi vantaggi di questa tecnica e’ che separa presentazione da implementazione, in quanto i template html possono modificati senza alterare il modo in cui sono inseriti i dati.
\item Dependecy Injection: Angular possiede nativamente una “dependecy injection”, che aiuta gli sviluppatori facilitando la creazione, la comprensione e il testing dell’applicazione.
\item Directives: Le directives possono essere usate per definire tag HTML personalizzati che fungono da widget. Possono inoltre essere usate per “decorare” elementi con comportamenti personalizzati o per manipolare attributi del DOM. 

\end{itemize}

\subsection{HTML5}

\newpage
\section{Descrizione architettura}
\subsection{Metodo e formalismo di specifica}
Si è deciso di procedere utilizzando un approccio top-down per l’esposizione dell’architettura dell’applicazione, ovvero descrivendo inizialmente le componenti in generale per poi arrivare a trattarle al particolare.
Si descriveranno i package e i componenti per poi dettagliare le singole classi, specificando per ciascuna di esse il tipo, l’obiettivo e la funzionalità. Poi si passerà ad illustrare degli esempi d’uso di Design Pattern (descritti approfonditamente nell’ Appendice A) e le tecnologie utilizzate.


\subsection{Architettura generale}
-	L'architettura del framework segue un modello di architettura in stile three-tier che prevede la suddivisione dell'applicazione  in tre diversi strati dedicati rispettivamente all'interfaccia utente (Client), alla business logic (Controller) e alla gestione dei dati persistenti (Model). La parte Client segue il design pattern MVVM utilizzato da AngularJS ed è quindi suddivisa in Model, View, ViewModel.\\


Il seguente diagramma rappresenta l'architettura ad alto livello del framework, indicando i package e le relazioni che intercorrono tra questi.

...TODO aggiungere diagramma dell'installer CLI comprensivo di descrizione...

\immagine{./Diagrammi/GeneralePackage}{Architettura generale del software - vista package}
Nel precedente diagramma sono presenti le relazioni tra i package Client, Controller e ModelServer.\\
Vengono inoltre presentati tutti i sotto-package così da facilitare la comprensione dell'intero sistema.
\immagine{./Diagrammi/GeneraleClassi}{Architettura generale del software}
Nel precedente diagramma è presente l'architettura ad alto livello del software e vengono indicate le classi fondamentali per rappresentare le relazioni del modello  three-tier. TO DO DESCRIVERE 3TIER IN APPENDICE. I diagrammi di sequenza relativi allo scambio di segnali, lo scopo ed il contesto di utilizzo sono presenti nella sezione ????.

\newpage
\subsubsection{ModelServer}
\immagine{./Diagrammi/ModelClassi}{Diagramma delle classi del ModelServer}
Nel ModelServer sono presenti oggetti che rappresentano:
\begin{itemize}
\item Il database di analisi e quello degli utenti;
\item La gestione del file DSL e il suo parsing;
\item La gestione dei dati richiesti dal controller.
\end{itemize}
Tutte le operazioni di gestione, modifica e recupero dei dati vengono messe a disposizione dal model. In tal modo il controller è responsabile solamente di gestire la logica dell’applicazione.

\subsubsection{Controller}
\immagine{./Diagrammi/ControllerClassi}{Diagramma delle classi del Controller}
Il controller è responsabile dell’autenticazione delle richieste e del loro routing da Client a ModelServer e viceversa.

\subsubsection{Client}
\immagine{./Diagrammi/ClientClassi}{Diagramma delle classi del Client}
Nel Client sono presenti oggetti che rappresentano:
\begin{itemize}
\item I template per le pagine web;
\item I Controller per la gestione dei template;
\item Lo Scope per l'aggiornamento dei dati dei template;
\item I Servizi utilizzati dai Controller.
\end{itemize}

\newpage
\section{Componenti e Classi}

\subsection{MaaP}
\subsubsection{Informazioni sul package}
\immagine{./Diagrammi/ComponentiMaaP}{Componenti MaaP}
\subsubsubsection{Descrizione}
Namespace globale per il progetto. Le relazioni tra i package ModelServer, Controller e Client identificano il modello di architettura three-tier.

\subsubsubsection{Sotto-componenti}
\begin{itemize}
\item MaaP::ModelServer
\item MaaP::Controller
\item MaaP::Client
\end{itemize}

\subsection{MaaP::ModelServer}
\subsubsection{Informazioni sul package}
\immagine{./Diagrammi/ComponentiModelServer}{Componente MaaP::ModelServer}
\subsubsubsection{Descrizione}
Package per il componente ModelServer del modello di architettura three-tier.

\subsubsubsection{Sottocomponenti}
\begin{itemize}
\item MaaP::ModelServer::DataManager;
\item MaaP::ModelServer::Database;
\item MaaP::ModelServer::DSL.
\end{itemize}

\subsubsection{MaaP::ModelServer::DataManager}
\subsubsubsection{Informazioni sul package}
\immagine{./Diagrammi/ComponentiDataManager}{Componente MaaP::ModelServer::DataManager}
\subsubsubsection{Descrizione}
Componente parte del ServerModel per la gestione dei dati.
\subsubsubsection{Sotto-componenti}
\begin{itemize}
\item MaaP::ModelServer::DataManager::DataBaseAnalysisManager;
\item MaaP::ModelServer::DataManager::DatabaseUserManager;
\item MaaP::ModelServer::DataManager::IndexManager.
\end{itemize}
\subsubsubsection{Classi}

	\subsubsubsubsection{JSonComposer}
	\grassetto{Nome}\\
	MaaP::ModelServer::DataManager::JSonComposer\\
	\grassetto{Descrizione}\\
	Classe che costruisce un file JSON a partire dalla struttura di una Collection, o di un Document, e dai suoi dati.\\
	\grassetto{Utilizzo}\\
	Viene utilizzata dai DatabaseManager per costruire il file JSON da inviare al Controller.

	\subsubsubsubsection{IDatabaseManager}
	\grassetto{Nome}\\
	MaaP::ModelServer::DataManager::IDatabaseManager\\
	\grassetto{Descrizione}\\
	Interfaccia che rappresenta il gestore dei database. Contiene tutte le operazioni che si possono effettuare sul database e l'elaborazione dei dati recuperati da essi.\\
	\grassetto{Utilizzo}\\
	Viene utilizzata per la gestione delle richieste inoltrate dal Controller.\\
	\grassetto{Classi che ereditano}
	\begin{itemize}
	\item MaaP::ModelServer::DataManager::DatabaseAnalysisManager::DatabaseAnalysisManager;
	\item MaaP::ModelServer::DataManager::DatabaseUserManager::DatabaseUserManager.	
	\end{itemize}

	\subsubsubsubsection{IDataRetriever}
	\grassetto{Nome}\\
	MaaP::ModelServer::DataManager::IDataRetriever\\
	\grassetto{Descrizione}\\
	Interfaccia attraverso cui i DatabaseManager dialogano con i batabase. Contiene le operazioni di lettura e scrittura nei database.\\
	\grassetto{Utilizzo}\\
	Viene utilizzata per recuperare e inserire dati, sui database, su richiesta dei DataManager.\\
	\grassetto{Classi che ereditano}
	\begin{itemize}
	\item MaaP::ModelServer::DataManager::DatabaseAnalysisManager::DataRetrieverAnalysis;
	\item MaaP::ModelServer::DataManager::DatabaseUserManager::DataRetrieverUsers.	
	\end{itemize}

\subsubsubsection{MaaP::ModelServer::DataManager::DatabaseAnalysisManager}
\subsubsubsubsection{Informazioni sul package}
\immagine{./Diagrammi/ComponentiDatabaseAnalysisManager}{Componente MaaP::ModelServer::DataManager::DatabaseAnalysisManager}
\subsubsubsubsection{Descrizione}
Componente parte del DataManager per la gestione dei dati del database di analisi.
\subsubsubsubsection{Classi}

\subsubsubsubsubsection{DatabaseAnalysisManager}
\grassetto{Nome}\\
MaaP::ModelServer::DataManager::DatabaseAnalysisManager::DatabaseAnalysisManager\\
\grassetto{Descrizione}\\
Classe che rappresenta il gestore dei database di analisi. Contiene tutte le operazioni che si possono effettuare sul database di analisi e l'elaborazione dei dati recuperati da essi.\\
\grassetto{Utilizzo}\\
Viene utilizzata per la gestione delle richieste, relative al database di analisi, inoltrate dal Controller.
\grassetto{Classi da cui eredita}
\begin{itemize}
\item MaaP::ModelServer::DataManager::IDatabaseManager;
\end{itemize}
\grassetto{Relazioni con altre classi}
\begin{itemize}
\item\grassetto{MaaP::ModelServer::DataManager::JSonComposer}\\
Relazione uscente, utilizza un riferimento a un oggetto di tipo JsonComposer per ottenere il JSON da spedire;
\item\grassetto{MaaP::ModelServer::DSL::CollectionData}\\
Relazione uscente, utilizza un riferimento a un oggetto CollectionData che contiene la struttura di un file di descrizione;
\item\grassetto{MaaP::ModelServer::DataManager::DataAnalysisManager::DataRetrieverAnalysis}\\
Relazione uscente, utilizza un riferimento a un oggetto DataRetrieverAnalysis per relazionarsi con il database di analisi;
\item\grassetto{MaaP::ModelServer::DataManager::IndexManager::IndexManager}\\
Relazione uscente, utilizza un riferimento a un oggetto IndexManager per la creazione degli indici;
\item\grassetto{MaaP::Controller::Dispatcher}\\
Relazione entrante, interazioni con le funzionalità del gestore del database di analisi.
\end{itemize}

\subsubsubsubsubsection{DatabaseRetrieverAnalysis}
\grassetto{Nome}\\
MaaP::ModelServer::DataManager::DatabaseAnalysisManager::DatabaseRetrieverAnalysis\\
\grassetto{Descrizione}\\
Classe che rappresenta l'oggetto per interagire con i database.\\
\grassetto{Utilizzo}\\
Viene utilizzata per inserire e leggere dati sui database di analisi e framework.\\
\grassetto{Classi da cui eredita}
\begin{itemize}
\item MaaP::ModelServer::DataManager::IDataRetriever;
\end{itemize}
\grassetto{Relazioni con altre classi}
\begin{itemize}
\item\grassetto{MaaP::ModelServer::DataManager::DatabaseAnalysisManager::DatabaseAnalysisManager}\\
Relazione entrante, interazione con il database;
\item\grassetto{MaaP::ModelServer::Database::MongooseDBAnalysis}\\
Relazione uscente, utilizza un riferimento a un oggetto di tipo MongooseDBAnalysis per creare lo schema dei dati del database di analisi e per interagire con essi;
\item\grassetto{MaaP::ModelServer::Database::MongooseDBFramework}\\
Relazione uscente, utilizza un riferimento a un oggetto di tipo MongooseDBFramework per creare lo schema dei dati del database del framework e per interagire con essi;
\end{itemize}


\subsubsubsection{MaaP::ModelServer::DataManager::DatabaseUserManager}
\subsubsubsubsection{Informazioni sul package}
\immagine{./Diagrammi/ComponentiDatabaseUserManager}{Componente MaaP::ModelServer::DataManager::DatabaseUserManager}
\subsubsubsubsection{Descrizione}
Componente parte del DataManager per la gestione dei dati del database del framwork che comprende sia dati utente che impostazioni del sistema.
\subsubsubsubsection{Classi}

\subsubsubsubsubsection{DatabaseUserManager}
\grassetto{Nome}\\
MaaP::ModelServer::DataManager::DatabaseUserManager::DatabaseUserManager\\
\grassetto{Descrizione}\\
Classe che rappresenta il gestore del database del framework. Contiene tutte le operazioni che si possono effettuare sul database del framework e l'elaborazione dei dati recuperati da esso.\\
\grassetto{Utilizzo}\\
Viene utilizzata per la gestione delle richieste relative al database del framework inoltrate dal Controller.\\
\grassetto{Classi da cui eredita}
\begin{itemize}
\item MaaP::ModelServer::DataManager::IDatabaseManager;
\end{itemize}
\grassetto{Relazioni con altre classi}
\begin{itemize}
\item\grassetto{MaaP::ModelServer::DataManager::JSonComposer}\\
Relazione uscente, utilizza un riferimento a un oggetto di tipo JsonComposer per ottenere il JSON da spedire;
\item\grassetto{MaaP::ModelServer::DataManager::DataUserManager::DataRetrieverUsers}\\
Relazione uscente, utilizza un riferimento a un oggetto DataRetrieverUsers per relazionarsi con il database del framework;
\item\grassetto{MaaP::Controller::Dispatcher}\\
Relazione entrante, interazioni con le funzionalità del gestore del database di analisi.
\end{itemize}

\subsubsubsubsubsection{DataRetrieverUsers}
\grassetto{Nome}\\
MaaP::ModelServer::DataManager::DatabaseUserManager::DataRetrieverUsers\\
Classe che rappresenta l'oggetto per interagire con il database del framework.\\
\grassetto{Utilizzo}\\
Viene utilizzata per inserire e leggere dati sul database del framework.\\
\grassetto{Classi da cui eredita}
\begin{itemize}
\item MaaP::ModelServer::DataManager::IDataRetriever;
\end{itemize}
\grassetto{Relazioni con altre classi}
\begin{itemize}
\item\grassetto{MaaP::ModelServer::DataManager::DatabaseUserManager::DatabaseUserManager}\\
Relazione entrante, interazione con il database;
\item\grassetto{MaaP::ModelServer::Database::MongooseDBFramework}\\
Relazione uscente, utilizza un riferimento a un oggetto di tipo MongooseDBFramework per creare lo schema dei dati del database del framework e per interagire con essi.
\end{itemize}



\subsubsubsection{MaaP::ModelServer::DataManager::IndexManager}
\subsubsubsubsection{Informazioni sul package}
\immagine{./Diagrammi/ComponentiIndexManager}{Componente MaaP::ModelServer::DataManager::IndexManager}
\subsubsubsubsection{Descrizione}
Componente parte del DataManager per la creazione e gestione degli indici.
\subsubsubsubsection{Classi}
	\subsubsubsubsubsection{IndexManager}
	\grassetto{Nome}\\
	MaaP::ModelServer::DataManager::IndexManager::IndexManager\\
	\grassetto{Descrizione}\\
	Classe che rappresenta il gestore degli indici. Contiene tutte le operazioni per la creazione degli indici.\\
	\grassetto{Utilizzo}\\
	Viene utilizzata per la creazione di indici personalizzati su richiesta del DatavaseAnalysisManager.\\
	\grassetto{Relazioni con altre classi}
	\begin{itemize}
	\item\grassetto{MaaP::ModelServer::DataManager::DatabaseAnalysisManager::DatabaseAnalysisManager}\\
	Relazione entrante, interazione con il database;
	\item\grassetto{MaaP::ModelServer::DataManager::Database::MongooseDBAnalysis}\\
	Relazione uscente, utilizza un riferimento ad un oggetto di tipo MongooseDBAnalysis per creare lo schema dei dati del database di analisi e per interagire con essi;
	\item\grassetto{MaaP::ModelServer::DataManager::Database::Query}\\
	Relazione uscente debole, utilizza un riferimento ad un oggetto Query per il recupero delle query più utilizzate.
	\end{itemize}

\subsubsection{MaaP::ModelServer::Database}
\subsubsubsection{Informazioni sul package}
\immagine{./Diagrammi/ComponentiDatabase}{Componente MaaP::ModelServer::Database}
\subsubsubsection{Descrizione}
Componente parte del ModelServer per la gestione dei dati.
\subsubsubsection{Classi}

	\subsubsubsubsection{MongooseDBAnalysis}
	\grassetto{Nome}\\
	MaaP::ModelServer::Database::MongooseDBAnalysis\\
	\grassetto{Descrizione}\\
	Classe che rappresenta l'interfaccia di connessione con il database di analisi.\\
	\grassetto{Utilizzo}\\
	Viene utilizzata per interfacciarsi con il database di analisi fornendo uno schema adeguato.\\
	\grassetto{Classi da cui eredita}
	\begin{itemize}
	\item MaaP::ModelServer::Database::Mongoose;
	\end{itemize}
	\grassetto{Relazioni con altre classi}
	\begin{itemize}
	\item\grassetto{MaaP::ModelServer::DataManager::DatabaseAnalysisManager::DataRetrieverAnalysis}\\
	Relazione entrante, interazione con il database di analisi;
	\item\grassetto{MaaP::ModelServer::DataManager::IndexManager::IndexManager}\\
	Relazione entrante, interazione con il database di analisi;
	\item\grassetto{MaaP::ModelServer::Database::DBAnalysis}\\
	Relazione uscente debole, utilizza un riferimento al database di analisi a cui connettersi.
	\end{itemize}
		
	\subsubsubsubsection{DBAnalysis}
	\grassetto{Nome}\\
	MaaP::ModelServer::Database::DBAnalysis\\
	\grassetto{Descrizione}\\
	Classe che rappresenta il database di analisi.\\
	\grassetto{Utilizzo}\\
	Viene utilizzata per contenere i dati di analisi.\\
	\grassetto{Relazioni con altre classi}
	\begin{itemize}
	\item\grassetto{MaaP::ModelServer::Database::MongooseDBAnalysis}\\
	Relazione entrante debole, interazione con il database di analisi.
	\end{itemize}
		
	\subsubsubsubsection{Mongoose}
	\grassetto{Nome}\\
	MaaP::ModelServer::Database::Mongoose\\
	\grassetto{Descrizione}\\
	Interfaccia che permette di dialogare con i database utilizzando Mongoose.\\
	\grassetto{Utilizzo}\\
	Viene utilizzata per interfacciarsi con i vari database.\\
	\grassetto{Classi che ereditano}
	\begin{itemize}
	\item MaaP::ModelServer::Database::MongooseDBAnalysis;
	\item MaaP::ModelServer::Database::MongooseDBFramework.	
	\end{itemize}
		
	\subsubsubsubsection{MongooseDBFramework}
	\grassetto{Nome}\\
	MaaP::ModelServer::Database::MongooseDBMongooseDBFramework\\
	\grassetto{Descrizione}\\
	Classe che rappresenta l'interfaccia di connessione con il database del framework.\\
	\grassetto{Utilizzo}\\
	Viene utilizzata per interfacciarsi con il database del framework fornendo uno schema adeguato.\\
	\grassetto{Classi da cui eredita}
	\begin{itemize}
	\item MaaP::ModelServer::Database::Mongoose;
	\end{itemize}
	\grassetto{Relazioni con altre classi}
	\begin{itemize}
	\item\grassetto{MaaP::ModelServer::DataManager::DatabaseAnalysisManager::DataRetrieverAnalysis}\\
	Relazione entrante, interazione con il database del framework;
	\item\grassetto{MaaP::ModelServer::DataManager::DatabaseUserManager::DataRetrieverUsers}\\
	Relazione entrante, interazione con il database del framework;
	\item\grassetto{MaaP::ModelServer::Database::DBFramework}\\
	Relazione uscente debole, utilizza un riferimento al database del framework a cui connettersi.
	\end{itemize}
	
		
	\subsubsubsubsection{DBFramework}
	\grassetto{Nome}\\
	MaaP::ModelServer::Database::DBFramework\\
	\grassetto{Descrizione}\\
	Classe che rappresenta il database del framework.\\
	\grassetto{Utilizzo}\\
	Viene utilizzata per contenere i dati utente ed impostazioni varie del sistema.\\
	\grassetto{Relazioni con altre classi}
	\begin{itemize}
	\item\grassetto{MaaP::ModelServer::Database::User}\\
	Relazione uscente, utilizza un riferimento ad un oggetto User per gestire i dati utente.
	\item\grassetto{MaaP::ModelServer::Database::Query}\\
	Relazione uscente, utilizza un riferimento ad un oggetto Query per gestire la lista di query fin'ora effettuate dal sistema;
	\item\grassetto{MaaP::Controller::FrontController}\\
	Relazione entrante, interazione con il database del framework;
	\item\grassetto{MaaP::Controller::Passport}\\
	Relazione entrante debole, interazione con il database del framework.
	\end{itemize}
	
	\subsubsubsubsection{User}
	\grassetto{Nome}\\
	MaaP::ModelServer::Database::User\\
	\grassetto{Descrizione}\\
	Classe che rappresenta la parte contenuta nel database del framework relativa ai dati utenti.\\
	\grassetto{Utilizzo}\\
	Viene utilizzata per contenere i dati utente.\\
	\grassetto{Relazioni con altre classi}
	\begin{itemize}
	\item\grassetto{MaaP::ModelServer::Database::DBFramework}\\
	Relazione entrante, interazione con i dati utente.
	\end{itemize}
	
	\subsubsubsubsection{Query}
	\grassetto{Nome}\\
	MaaP::ModelServer::Database::Query\\
	\grassetto{Descrizione}\\
	Classe che rappresenta la parte contenuta nel database del framework relativa alle query effettuate del sistema.\\
	\grassetto{Utilizzo}\\
	Viene utilizzata per contenere le query effettuate del sistema.\\
	\grassetto{Relazioni con altre classi}
	\begin{itemize}
	\item\grassetto{MaaP::ModelServer::Database::DBFramework}\\
	Relazione entrante, interazione con le query effettuate del sistema.
	\item\grassetto{MaaP::ModelServer::DataManager::IndexManager::IndexManager}\\
	Relazione entrante debole, interazione con le query effettuate del sistema.
	\end{itemize}
	

\subsubsection{MaaP::ModelServer::DSL}
\subsubsubsection{Informazioni sul package}
\immagine{./Diagrammi/ComponentiDSL}{Componente MaaP::ModelServer::DSL}
\subsubsubsection{Descrizione}
Componente parte del ServerModel per la gestione dei file di descrizione.
\subsubsubsection{Sotto-componenti}
\subsubsubsection{Classi}

	\subsubsubsubsection{ParserInterface}
	\grassetto{Nome}\\
	MaaP::ModelServer::DSL::ParserInterface\\
	\grassetto{Descrizione}\\
	Interfaccia che rappresenta la componente interfaccia del design pattern strategy per il parser di un linguaggio DSL.\\
	\grassetto{Utilizzo}\\
	Viene utilizzata per la effettuare il parsing di un file di descrizione.\\
	\grassetto{Classi che ereditano}
	\begin{itemize}
	\item MaaP::ModelServer::DSL::DSLParser.
	\end{itemize}
		
	\subsubsubsubsection{DSLParser}
	\grassetto{Nome}\\
	MaaP::ModelServer::DSL::DSLParser\\
	\grassetto{Descrizione}\\
	Classe che rappresenta l'algoritmo per il parser DSL del design pattern strategy.\\
	\grassetto{Utilizzo}\\
	Viene utilizzata all'avvio del sistema per eseguire il parsing dei file di descrizione.
	\grassetto{Classi da cui eredita}
	\begin{itemize}
	\item MaaP::ModelServer::DSL::ParserInterface;
	\end{itemize}
	\grassetto{Relazioni con altre classi}
	\begin{itemize}
	\item\grassetto{MaaP::ModelServer::DSL::DSLDescriptionFile}\\
	Relazione uscente debole, utilizza un riferimento ad un oggetto DSLDescriptionFile per leggere il file di descrizione;
	\end{itemize}
		
	\subsubsubsubsection{DSLManager}
	\grassetto{Nome}\\
	MaaP::ModelServer::DSL::DSLManager\\
	\grassetto{Descrizione}\\
	Classe che rappresenta il gestore dei file di descrizione. Contiene tutte le operazioni per eseguire il parsing dei file di descrizione e per salvare il risultato su appositi file di tipo CollectionData\\
	\grassetto{Utilizzo}\\
	Viene utilizzata all'avvio del sistema per eseguire il parsing dei file di descrizione e salvare il risultato su file.
	\grassetto{Classi da cui eredita}
	\grassetto{Relazioni con altre classi}
	\begin{itemize}
	\item\grassetto{MaaP::ModelServer::DSL::ParserInterface}\\
	Relazione uscente, utilizza un riferimento ad un oggetto ParserInterface per eseguire il parsing del file di descrizione;
	\item\grassetto{MaaP::ModelServer::DSL::DSLDescriptionFile}\\
	Relazione uscente, utilizza un riferimento ad un oggetto DSLDescriptionFile per leggere il file di descrizione;
	\item\grassetto{MaaP::ModelServer::DSL::CollectionData}\\
	Relazione uscente, utilizza un riferimento ad un oggetto CollectionData per salvare i risultati dell'operazione di parsing.
	\end{itemize}
	
	\subsubsubsubsection{CollectionData}
	\grassetto{Nome}\\
	MaaP::ModelServer::DSL::CollectionData\\
	\grassetto{Descrizione}\\
	Classe che rappresenta il file contenente il risultato dell'operazione di parsing.\\
	\grassetto{Utilizzo}\\
	Viene utilizzata all'avvio del sistema per salvare il risultato dell'operazione di parsing del file di descrizione.
	\grassetto{Relazioni con altre classi}
	\begin{itemize}
	\item\grassetto{MaaP::ModelServer::DSL::DSLManager}\\
	Relazione entrante, interazione con il file;
	\item\grassetto{MaaP::ModelServer::DataManager::DatabaseAnalysisManager::DatabaseAnalysisManager}\\
	Relazione entrante, interazione con il file.
	\end{itemize}
	

\subsection{MaaP::Controller}
\subsubsection{Informazioni sul package}
\immagine{./Diagrammi/ComponentiController}{Componente MaaP::Controller}
\subsubsubsection{Descrizione}
Package per il componente Controller del modello di architettura three-tier.
\subsubsubsection{Classi}

	\subsubsubsubsection{IPassport}
	\grassetto{Nome}\\
	MaaP::Controller::IPassport\\
	\grassetto{Descrizione}\\
	Interfaccia che rappresenta il componente target del design pattern object adapter.\\
	\grassetto{Utilizzo}\\
	Viene utilizzata per gestire l'autenticazione utente.
	\grassetto{Relazioni con altre classi}
	\begin{itemize}
	\item\grassetto{MaaP::Controller::FrontController}\\
	Relazione entrante, interazione con il gestore dell'autenticazione.
	\end{itemize}
	\grassetto{Classi che ereditano}
	\begin{itemize}
	\item MaaP::Controller::PassportAdapter.
	\end{itemize}
	
	\subsubsubsubsection{PassportAdapter}
	\grassetto{Nome}\\
	MaaP::Controller::PassportAdapter\\
	\grassetto{Descrizione}\\
	Classe che rappresenta il componente adapter del design pattern object adapter.\\
	\grassetto{Utilizzo}\\
	Viene utilizzata per gestire l'autenticazione utente.
	\grassetto{Classi da cui eredita}
	\begin{itemize}
	\item MaaP::Controller::IPassport.
	\end{itemize}
	\grassetto{Relazioni con altre classi}
	\begin{itemize}
	\item\grassetto{MaaP::Controller::Passport}\\
	Relazione uscente, utilizza un riferimento ad un oggetto di tipo Passport per gestire l'autenticazione utente.
	\end{itemize}
	
	\subsubsubsubsection{Passport}
	\grassetto{Nome}\\
	MaaP::Controller::Passport\\
	\grassetto{Descrizione}\\
	Classe che rappresenta il componente adaptee del design patter object adapter.\\
	\grassetto{Utilizzo}\\
	Viene utilizzata per gestire l'autenticazione utente.
	\grassetto{Relazioni con altre classi}
	\begin{itemize}
	\item\grassetto{MaaP::ModelServer::Database::MongooseDBFramework}\\
	Relazione uscente debole, utilizza un riferimento ad un oggetto MongooseDBFramework per accedere ai dati utente.
	\end{itemize}
	
	\subsubsubsubsection{FrontController}
	\grassetto{Nome}\\
	MaaP::Controller::FrontController\\
	\grassetto{Descrizione}\\
	Classe che rappresenta il componente controller del design patter Front Controller.\\
	\grassetto{Utilizzo}\\
	Viene utilizzata per gestire le richieste del client ed inoltrarle al dispatcher.
	\grassetto{Relazioni con altre classi}
	\begin{itemize}
	\item\grassetto{MaaP::Controller::IPassport}\\
	Relazione uscente, contiene un riferimento ad un oggetto IPassport per gestire l'autenticazione utente;
	\item\grassetto{MaaP::Controller::Dispatcher}\\
	Relazione uscente, contiene un riferimento ad un oggetto Dispatcher per smistare le richieste del client ai vari manager;
	\item\grassetto{MaaP::ModelServer::Database::MongooseDBFramework}\\
	Relazione uscente, contiene un riferimento ad un oggetto MongooseDBFramework per inserire nuovi dati nel database del framework relativi a nuovi utenti;
	\item\grassetto{MaaP::Client::ModelClient::Services::HTTP}\\
	Relazione entrante debole, interazione con il servizio HTTP.
	\end{itemize}
	
	\subsubsubsubsection{Dispatcher}
	\grassetto{Nome}\\
	MaaP::Controller::Dispatcher\\
	\grassetto{Descrizione}\\
	Classe che rappresenta il componente dispatcher del design patter Front Controller.\\
	\grassetto{Utilizzo}\\
	Viene utilizzata per smistare le richieste del client ai vari gestori dei dati.
	\grassetto{Relazioni con altre classi}
	\begin{itemize}
	\item\grassetto{MaaP::Controller::FrontController}\\
	Relazione entrante, interazione con il FrontController;
	\item\grassetto{MaaP::ModelServer::DataManager::DatabaseAnalysisManager::DatabaseAnalysisManager}\\
	Relazione uscente, contiene un riferimento ad un oggetto DatabaseAnalysisManager per richiedere azioni relative ai dati di analisi;
	\item\grassetto{MaaP::ModelServer::DataManager::DatabaseUserManager::DatabaseUserManager}\\
	Relazione uscente, contiene un riferimento ad un oggetto DatabaseUserManager per richiedere azioni relative ai dati utenti ed impostazioni di sistema.
	\end{itemize}


\subsection{MaaP::Client}
\subsubsection{Informazioni sul package}
\immagine{./Diagrammi/ComponentiClient}{Componente MaaP::Client}
\subsubsubsection{Descrizione}
Package per il componente Client del modello di architettura three-tier.
\subsubsubsection{Sottocomponenti}
\begin{itemize}
\item MaaP::Client::View;
\item MaaP::Client::ControllerModelView;
\item MaaP::Client::ModelClient.
\end{itemize}

\subsubsection{MaaP::Client::View}
\subsubsubsection{Informazioni sul package}
\immagine{./Diagrammi/ComponentiView}{Componente MaaP::Client::View}
\subsubsubsection{Descrizione}
Componente view del design pattern MVVM.
\subsubsubsection{Sotto-componenti}
\begin{itemize}
\item MaaP::Client::Template.
\end{itemize}

\subsubsubsection{MaaP::Client::View::Template}
\subsubsubsection{Informazioni sul package}
\immagine{./Diagrammi/ComponentiTemplate}{Componente MaaP::Client::View::Template}
\subsubsubsection{Descrizione}
Componente che contiene i template per la visualizzazione delle pagine web.
\subsubsubsection{Classi}

	\subsubsubsubsection{SignIn}
	\grassetto{Nome}\\
	MaaP::Client::View::Template::SignIn\\
	\grassetto{Descrizione}\\
	Classe che rappresenta il template per la pagina di login.\\
	\grassetto{Utilizzo}\\
	Viene utilizzata per renderizzare la pagina web di login.\\
	\grassetto{Relazioni con altre classi}
	\begin{itemize}
	\item\grassetto{MaaP::Client::ControllerModelView::ControllerClient::ControllerAutenticazione}\\
	Relazione uscente, contiene un riferimento ad un oggetto ControllerAutenticazione per gestire il login utente.
	\end{itemize}
	
	\subsubsubsubsection{SignUp}
	\grassetto{Nome}\\
	MaaP::Client::View::Template::SignUp\\
	\grassetto{Descrizione}\\
	Classe che rappresenta il template per la pagina di registrazione.\\
	\grassetto{Utilizzo}\\
	Viene utilizzata per renderizzare la pagina web di registrazione utente.
	\grassetto{Relazioni con altre classi}
	\begin{itemize}
	\item\grassetto{MaaP::Client::ControllerModelView::ControllerClient::ControllerAutenticazione}\\
	Relazione uscente, contiene un riferimento ad un oggetto ControllerAutenticazione per gestire la registrazione di un nuovo utente.
	\end{itemize}

	\subsubsubsubsection{AdminMainPageCollection}
	\grassetto{Nome}\\
	MaaP::Client::View::Template::AdminMainPageCollection\\
	\grassetto{Descrizione}\\
	Classe che rappresenta il template per la pagina di visualizzazione Collection per l'utente amministratore.\\
	\grassetto{Utilizzo}\\
	Viene utilizzata per renderizzare la pagina web di visualizzazione Collection per l'utente amministratore.\\
	\grassetto{Relazioni con altre classi}
	\begin{itemize}
	\item\grassetto{MaaP::Client::ControllerModelView::ControllerClient::ControllerCollection}\\
	Relazione uscente, contiene un riferimento ad un oggetto ControllerCollection per gestire la visualizzazione della pagina Collection;
	\item\grassetto{MaaP::Client::ControllerModelView::ControllerClient::ControllerMenu}\\
	Relazione uscente, contiene un riferimento ad un oggetto ControllerMenu per gestire la visualizzazione del menù.
	\end{itemize}
	
	\subsubsubsubsection{UserMainPageCollection}
	\grassetto{Nome}\\
	MaaP::Client::View::Template::UserMainPageCollection\\
	\grassetto{Descrizione}\\
	Classe che rappresenta il template per la pagina di visualizzazione Collection per l'utente.\\
	\grassetto{Utilizzo}\\
	Viene utilizzata per renderizzare la pagina web di visualizzazione Collection per l'utente.\\
	\grassetto{Relazioni con altre classi}
	\begin{itemize}
	\item\grassetto{MaaP::Client::ControllerModelView::ControllerClient::ControllerCollection}\\
	Relazione uscente, contiene un riferimento ad un oggetto ControllerCollection per gestire la visualizzazione della pagina Collection;
	\item\grassetto{MaaP::Client::ControllerModelView::ControllerClient::ControllerMenu}\\
	Relazione uscente, contiene un riferimento ad un oggetto ControllerMenu per gestire la visualizzazione del menù.
	\end{itemize}
	
	\subsubsubsubsection{MainPageDocument}
	\grassetto{Nome}\\
	MaaP::Client::View::Template::MainPageDocument\\
	\grassetto{Descrizione}\\
	Classe che rappresenta il template per la pagina di visualizzazione Document.\\
	\grassetto{Utilizzo}\\
	Viene utilizzata per renderizzare la pagina web di visualizzazione del Document.\\
	\grassetto{Relazioni con altre classi}
	\begin{itemize}
	\item\grassetto{MaaP::Client::ControllerModelView::ControllerClient::ControllerDocument}\\
	Relazione uscente, contiene un riferimento ad un oggetto ControllerDocument per gestire la visualizzazione della pagina Document;
	\item\grassetto{MaaP::Client::ControllerModelView::ControllerClient::ControllerMenu}\\
	Relazione uscente, contiene un riferimento ad un oggetto ControllerMenu per gestire la visualizzazione del menù.
	\end{itemize}
	
	\subsubsubsubsection{MainPageDocumentEdit}
	\grassetto{Nome}\\
	MaaP::Client::View::Template::MainPageDocumentEdit\\
	\grassetto{Descrizione}\\
	Classe che rappresenta il template per la pagina di modifica dei Document.\\
	\grassetto{Utilizzo}\\
	Viene utilizzata per renderizzare la pagina web di modifica dei Document.\\
	\grassetto{Relazioni con altre classi}
	\begin{itemize}
	\item\grassetto{MaaP::Client::ControllerModelView::ControllerClient::ControllerDocument}\\
	Relazione uscente, contiene un riferimento ad un oggetto ControllerDocument per gestire la visualizzazione della pagina di modifica dei Document;
	\item\grassetto{MaaP::Client::ControllerModelView::ControllerClient::ControllerMenu}\\
	Relazione uscente, contiene un riferimento ad un oggetto ControllerMenu per gestire la visualizzazione del menù.
	\end{itemize}

	\subsubsubsubsection{UserProfileEdit}
	\grassetto{Nome}\\
	MaaP::Client::View::Template::UserProfileEdit\\
	\grassetto{Descrizione}\\
	Classe che rappresenta il template per la pagina di modifica del profilo utente.\\
	\grassetto{Utilizzo}\\
	Viene utilizzata per renderizzare la pagina web di modifica del profilo utente.\\
	\grassetto{Relazioni con altre classi}
	\begin{itemize}
	\item\grassetto{MaaP::Client::ControllerModelView::ControllerClient::ControllerProfilo}\\
	Relazione uscente, contiene un riferimento ad un oggetto ControllerProfilo per gestire la visualizzazione della pagina di modifica del profilo utente;
	\item\grassetto{MaaP::Client::ControllerModelView::ControllerClient::ControllerMenu}\\
	Relazione uscente, contiene un riferimento ad un oggetto ControllerMenu per gestire la visualizzazione del menù.
	\end{itemize}
	
	\subsubsubsubsection{UserProfile}
	\grassetto{Nome}\\
	MaaP::Client::View::Template::UserProfile\\
	\grassetto{Descrizione}\\
	Classe che rappresenta il template per la pagina di visualizzazione del profilo utente.\\
	\grassetto{Utilizzo}\\
	Viene utilizzata per renderizzare la pagina web di visualizzazione del profilo utente.\\
	\grassetto{Relazioni con altre classi}
	\begin{itemize}
	\item\grassetto{MaaP::Client::ControllerModelView::ControllerClient::ControllerProfilo}\\
	Relazione uscente, contiene un riferimento ad un oggetto ControllerProfilo per gestire la visualizzazione della pagina del profilo utente;
	\item\grassetto{MaaP::Client::ControllerModelView::ControllerClient::ControllerMenu}\\
	Relazione uscente, contiene un riferimento ad un oggetto ControllerMenu per gestire la visualizzazione del menù.
	\end{itemize}

	\subsubsubsubsection{AdminProfile}
	\grassetto{Nome}\\
	MaaP::Client::View::Template::AdminProfile\\
	\grassetto{Descrizione}\\
	Classe che rappresenta il template per la pagina di visualizzazione del profilo utente amministratore.\\
	\grassetto{Utilizzo}\\
	Viene utilizzata per renderizzare la pagina web di visualizzazione del profilo utente amministratore.\\
	\grassetto{Relazioni con altre classi}
	\begin{itemize}
	\item\grassetto{MaaP::Client::ControllerModelView::ControllerClient::ControllerProfilo}\\
	Relazione uscente, contiene un riferimento ad un oggetto ControllerProfilo per gestire la visualizzazione della pagina del profilo utente amministratore;
	\item\grassetto{MaaP::Client::ControllerModelView::ControllerClient::ControllerMenu}\\
	Relazione uscente, contiene un riferimento ad un oggetto ControllerMenu per gestire la visualizzazione del menù.
	\end{itemize}
	
	\subsubsubsubsection{PasswordRecovery}
	\grassetto{Nome}\\
	MaaP::Client::View::Template::UserProfile\\
	\grassetto{Descrizione}\\
	Classe che rappresenta il template per la pagina di recupero password.\\
	\grassetto{Utilizzo}\\
	Viene utilizzata per renderizzare la pagina web recupero password.
	\grassetto{Relazioni con altre classi}
	\begin{itemize}
	\item\grassetto{MaaP::Client::ControllerModelView::ControllerClient::ControllerProfilo}\\
	Relazione uscente, contiene un riferimento ad un oggetto ControllerProfilo per gestire la visualizzazione della pagina di recupero password;
	\item\grassetto{MaaP::Client::ControllerModelView::ControllerClient::ControllerMenu}\\
	Relazione uscente, contiene un riferimento ad un oggetto ControllerMenu per gestire la visualizzazione del menù.
	\end{itemize}

\subsubsection{MaaP::Client::ControllerModelView}
\subsubsubsection{Informazioni sul package}
\immagine{./Diagrammi/ComponentiControllerModelView}{Componente MaaP::Client::ControllerModelView}
\subsubsubsection{Descrizione}
Componente ModelView del design pattern MVVM.
\subsubsubsection{Sotto-componenti}
\begin{itemize}
\item MaaP::Client::ControllerModelView::ControllerClient;
\item MaaP::Client::ControllerModelView::Scope.
\end{itemize}

\subsubsubsection{MaaP::Client::ControllerModelView::ControllerClient}
\subsubsubsubsection{Informazioni sul package}
\immagine{./Diagrammi/ComponentiControllerClient}{Componente MaaP::Client::ControllerModelView::ControllerClient}
\subsubsubsubsection{Descrizione}
Componente parte del ControllerModelView contenente i vari controller.
\subsubsubsubsection{Classi}

	\subsubsubsubsubsection{ControllerAutenticazione}
	\grassetto{Nome}\\
	MaaP::Client::ControllerModelView::ControllerClient::ControllerAutenticazione\\
	\grassetto{Descrizione}\\
	Classe che rappresenta il controller per indirizzare le richieste di autenticazione e registrazione.\\
	\grassetto{Utilizzo}\\
	Viene utilizzata per la indirizzare le richieste di autenticazione e registrazione.\\
	\grassetto{Relazioni con altre classi}
	\begin{itemize}
	\item\grassetto{MaaP::Client::ModelClient::Services::HTTP}\\
	Relazione uscente debole, contiene un riferimento ad un oggetto HTTP per utilizzare il relativo servizio;
	\item\grassetto{MaaP::Client::ModelClient::Model::SessionData}\\
	Relazione uscente debole, contiene un riferimento ad un oggetto SessionData per utilizzare i dati di sessione;
	\item\grassetto{MaaP::Client::View::Template::SignIn}\\
	Relazione entrante, interazione con il template;
	\item\grassetto{MaaP::Client::View::Template::SignUp}\\
	Relazione entrante, interazione con il template.
	\end{itemize}
	
	\subsubsubsubsubsection{ControllerCollection}
	\grassetto{Nome}\\
	MaaP::Client::ControllerModelView::ControllerClient::ControllerCollection\\
	\grassetto{Descrizione}\\
	Classe che rappresenta il controller per indirizzare le richieste di visualizzazione di una pagina Collection.\\
	\grassetto{Utilizzo}\\
	Viene utilizzata per indirizzare le richieste di visualizzazione di una pagina Collection.\\
	\grassetto{Relazioni con altre classi}
	\begin{itemize}
	\item\grassetto{MaaP::Client::ModelClient::Services::HTTP}\\
	Relazione uscente debole, contiene un riferimento ad un oggetto HTTP per utilizzare il relativo servizio;
	\item\grassetto{MaaP::Client::ControllerModelView::Scope::Collection}\\
	Relazione uscente debole, contiene un riferimento ad un oggetto Collection per accedere allo scope relativo ai dati di una Collection;
	\item\grassetto{MaaP::Client::ControllerModelView::Scope::Query}\\
	Relazione uscente debole, contiene un riferimento ad un oggetto Query per accedere allo scope relativo ai dati relativi alle query;
	\item\grassetto{MaaP::Client::View::Template::AdminMainPageCollection}\\
	Relazione entrante, interazione con il template;
	\item\grassetto{MaaP::Client::View::Template::UserMainPageCollection}\\
	Relazione entrante, interazione con il template.
	\end{itemize}
	
	\subsubsubsubsubsection{ControllerDocument}
	\grassetto{Nome}\\
	MaaP::Client::ControllerModelView::ControllerClient::ControllerDocument\\
	\grassetto{Descrizione}\\
	Classe che rappresenta il controller per indirizzare le richieste di visualizzazione di una pagina Document.\\
	\grassetto{Utilizzo}\\
	Viene utilizzata per indirizzare le richieste di visualizzazione di una pagina Document.\\
	\grassetto{Relazioni con altre classi}
	\begin{itemize}
	\item\grassetto{MaaP::Client::ModelClient::Services::HTTP}\\
	Relazione uscente debole, contiene un riferimento ad un oggetto HTTP per utilizzare il relativo servizio;
	\item\grassetto{MaaP::Client::ControllerModelView::Scope::Document}\\
	Relazione uscente debole, contiene un riferimento ad un oggetto Document per accedere allo scope relativo ai dati di un Document;
	\item\grassetto{MaaP::Client::View::Template::MainPageDocument}\\
	Relazione entrante, interazione con il template;
	\item\grassetto{MaaP::Client::View::Template::MainPageDocumentEdit}\\
	Relazione entrante, interazione con il template.
	\end{itemize}
	
	\subsubsubsubsubsection{ControllerProfilo}
	\grassetto{Nome}\\
	MaaP::Client::ControllerModelView::ControllerClient::ControllerProfilo\\
	\grassetto{Descrizione}\\
	Classe che rappresenta il controller per indirizzare le richieste di visualizzazione di una pagina profilo utente.\\
	\grassetto{Utilizzo}\\
	Viene utilizzata per indirizzare le richieste di visualizzazione di una pagina profilo utente.\\
	\grassetto{Relazioni con altre classi}
	\begin{itemize}
	\item\grassetto{MaaP::Client::ModelClient::Services::HTTP}\\
	Relazione uscente debole, contiene un riferimento ad un oggetto HTTP per utilizzare il relativo servizio;
	\item\grassetto{MaaP::Client::ControllerModelView::Scope::Profilo}\\
	Relazione uscente debole, contiene un riferimento ad un oggetto Profilo per accedere allo scope relativo ai dati del profilo;
	\item\grassetto{MaaP::Client::View::Template::UserProfileEdit}\\
	Relazione entrante, interazione con il template;
	\item\grassetto{MaaP::Client::View::Template::UserProfile}\\
	Relazione entrante, interazione con il template.
	\item\grassetto{MaaP::Client::View::Template::AdminProfile}\\
	Relazione entrante, interazione con il template.
	\item\grassetto{MaaP::Client::View::Template::PasswordRecovery}\\
	Relazione entrante, interazione con il template.
	\end{itemize}
	
	\subsubsubsubsubsection{ControllerMenu}
	\grassetto{Nome}\\
	MaaP::Client::ControllerModelView::ControllerClient::ControllerMenu\\
	\grassetto{Descrizione}\\
	Classe che rappresenta il controller per indirizzare le richieste di visualizzazione della parte di pagina relativa al menù.\\
	\grassetto{Utilizzo}\\
	Viene utilizzata per indirizzare le richieste di visualizzazione della parte di pagina relativa al menù.\\
	\grassetto{Relazioni con altre classi}
	\begin{itemize}
	\item\grassetto{MaaP::Client::ModelClient::Services::HTTP}\\
	Relazione uscente debole, contiene un riferimento ad un oggetto HTTP per utilizzare il relativo servizio;
	\item\grassetto{MaaP::Client::ControllerModelView::Scope::Menu}\\
	Relazione uscente debole, contiene un riferimento ad un oggetto Menu per accedere allo scope relativo ai dati del menù;
	\item\grassetto{MaaP::Client::View::Template::AdminMainPageCollection}\\
	Relazione entrante, interazione con il template.
	\item\grassetto{MaaP::Client::View::Template::UserMainPageCollection}\\
	Relazione entrante, interazione con il template.
	\item\grassetto{MaaP::Client::View::Template::MainPageDocument}\\
	Relazione entrante, interazione con il template;
	\item\grassetto{MaaP::Client::View::Template::MainPageDocumentEdit}\\
	Relazione entrante, interazione con il template.
	\item\grassetto{MaaP::Client::View::Template::UserProfileEdit}\\
	Relazione entrante, interazione con il template.
	\item\grassetto{MaaP::Client::View::Template::UserProfile}\\
	Relazione entrante, interazione con il template.
	\item\grassetto{MaaP::Client::View::Template::AdminProfile}\\
	Relazione entrante, interazione con il template.
	\item\grassetto{MaaP::Client::View::Template::PasswordRecovery}\\
	Relazione entrante, interazione con il template.
	\end{itemize}

\subsubsubsection{MaaP::Client::ControllerModelView::Scope}
\subsubsubsubsection{Informazioni sul package}
\immagine{./Diagrammi/ComponentiScope}{Componente MaaP::Client::ControllerModelView::Scope}
\subsubsubsubsection{Descrizione}
Componente parte del ControllerModelView contenente i dati per renderizzare i template.
\subsubsubsubsection{Classi}

	\subsubsubsubsubsection{Collection}
	\grassetto{Nome}\\
	MaaP::Client::ControllerModelView::Scope::Collection\\
	\grassetto{Descrizione}\\
	Classe che rappresenta i dati relativi alla Collection da visualizzare.\\
	\grassetto{Utilizzo}\\
	Viene utilizzata per memorizzare i dati relativi alla Collection da visualizzare i quali saranno sucessivamente visualizzati nella pagina web.\\
	\grassetto{Relazioni con altre classi}
	\begin{itemize}
	\item\grassetto{MaaP::Client::ControllerModelView::ControllerClient::ControllerCollection}\\
	Relazione entrante debole, interazione con il controller della Collection;
	\end{itemize}
	
	\subsubsubsubsubsection{Query}
	\grassetto{Nome}\\
	MaaP::Client::ControllerModelView::Scope::Query\\
	\grassetto{Descrizione}\\
	Classe che rappresenta i dati relativi alle query più utilizzare.\\
	\grassetto{Utilizzo}\\
	Viene utilizzata per memorizzare i dati relativi alle query più utilizzate, le quali saranno sucessivamente visualizzate nella pagina web.\\
	\grassetto{Relazioni con altre classi}
	\begin{itemize}
	\item\grassetto{MaaP::Client::ControllerModelView::ControllerClient::ControllerCollection}\\
	Relazione entrante debole, interazione con il controller della Collection;
	\end{itemize}
	
	\subsubsubsubsubsection{Document}
	\grassetto{Nome}\\
	MaaP::Client::ControllerModelView::Scope::Document\\
	\grassetto{Descrizione}\\
	Classe che rappresenta i dati relativi al Document da visualizzare.\\
	\grassetto{Utilizzo}\\
	Viene utilizzata per memorizzare i dati relativi al Document da visualizzare i quali saranno sucessivamente visualizzati nella pagina web.\\
	\grassetto{Relazioni con altre classi}
	\begin{itemize}
	\item\grassetto{MaaP::Client::ControllerModelView::ControllerClient::ControllerDocument}\\
	Relazione entrante debole, interazione con il controller del Document;
	\end{itemize}
	
	\subsubsubsubsubsection{Profilo}
	\grassetto{Nome}\\
	MaaP::Client::ControllerModelView::Scope::Profilo\\
	\grassetto{Descrizione}\\
	Classe che rappresenta i dati relativi al profilo utente da visualizzare.\\
	\grassetto{Utilizzo}\\
	Viene utilizzata per memorizzare i dati relativi al profilo utente da visualizzare i quali saranno sucessivamente visualizzati nella pagina web.\\
	\grassetto{Relazioni con altre classi}
	\begin{itemize}
	\item\grassetto{MaaP::Client::ControllerModelView::ControllerClient::ControllerProfilo}\\
	Relazione entrante debole, interazione con il controller del profilo;
	\end{itemize}
	
	\subsubsubsubsubsection{Menu}
	\grassetto{Nome}\\
	MaaP::Client::ControllerModelView::Scope::Menu\\
	\grassetto{Descrizione}\\
	Classe che rappresenta i dati relativi al menù da visualizzare.\\
	\grassetto{Utilizzo}\\
	Viene utilizzata per memorizzare i dati relativi al menù da visualizzare i quali saranno sucessivamente visualizzati nella pagina web.\\
	\grassetto{Relazioni con altre classi}
	\begin{itemize}
	\item\grassetto{MaaP::Client::ControllerModelView::ControllerClient::ControllerMenu}\\
	Relazione entrante debole, interazione con il controller del menù;
	\end{itemize}
	

\subsubsection{MaaP::Client::ModelClient}
\subsubsubsection{Informazioni sul package}
\immagine{./Diagrammi/ComponentiModelClient}{Componente MaaP::Client::ModelClient}
\subsubsubsection{Descrizione}
Componente Model del design pattern MVVM.
\subsubsubsection{Sotto-componenti}
\begin{itemize}
\item MaaP::Client::ModelClient::Services;
\item MaaP::Client::ModelClient::Model.
\end{itemize}

\subsubsubsection{MaaP::Client::ModelClient::Services}
\subsubsubsubsection{Informazioni sul package}
\immagine{./Diagrammi/ComponentiServices}{Componente MaaP::Client::ModelClient::Services}
\subsubsubsubsection{Descrizione}
Componente parte del ModelClient contenente i servizi per la comunicazione con il server.
\subsubsubsubsection{Classi}

	\subsubsubsubsubsection{HTTP}
	\grassetto{Nome}\\
	MaaP::Client::ModelClient::Services::HTTP\\
	\grassetto{Descrizione}\\
	Classe che rappresenta il servizio di comunicazione HTTP con il server.\\
	\grassetto{Utilizzo}\\
	Viene utilizzata per inviare richieste HTTP al server.\\
	\grassetto{Relazioni con altre classi}
	\begin{itemize}
	\item\grassetto{MaaP::Client::ControllerModelView::ControllerClient::ControllerAutenticazione}\\
	Relazione entrante debole, interazione con il controller dell'Autenticazione;
	\item\grassetto{MaaP::Client::ControllerModelView::ControllerClient::ControllerCollection}\\
	Relazione entrante debole, interazione con il controller della Collection;
	\item\grassetto{MaaP::Client::ControllerModelView::ControllerClient::ControllerDocument}\\
	Relazione entrante debole, interazione con il controller del Document;
	\item\grassetto{MaaP::Client::ControllerModelView::ControllerClient::ControllerProfilo}\\
	Relazione entrante debole, interazione con il controller del profilo;
	\item\grassetto{MaaP::Client::ControllerModelView::ControllerClient::ControllerMenu}\\
	Relazione entrante debole, interazione con il controller del menù;
	\item\grassetto{MaaP::Controller::FrontController}\\
	Relazione uscente debole, contiene un riferimento ad un oggetto di tipo FrontController per inviare richieste HTTP al server.
	\end{itemize}

\subsubsubsection{MaaP::Client::ModelClient::Model}
\subsubsubsubsection{Informazioni sul package}
\immagine{./Diagrammi/ComponentiModel}{Componente MaaP::Client::ModelClient::Model}
\subsubsubsubsection{Descrizione}
Componente parte del ModelClient contenente i dati di sessione.
\subsubsubsubsection{Classi}

	\subsubsubsubsubsection{SessionData}
	\grassetto{Nome}\\
	MaaP::Client::ModelClient::Model::SessionData\\
	\grassetto{Descrizione}\\
	Classe che rappresenta i dati di sessione utente.\\
	\grassetto{Utilizzo}\\
	Viene utilizzata memorizzare i dati di sessione del client.\\
	\grassetto{Relazioni con altre classi}
	\begin{itemize}
	\item\grassetto{MaaP::Client::ControllerModelView::ControllerClient::ControllerAutenticazione}\\
	Relazione entrante debole, interazione con il controller dell'Autenticazione.
	\end{itemize}



\newpage
\section{Diagrammi di attività}

\newpage
\section{Design Pattern Utilizzati}%7.0
\subsection{Design Pattern architetturali} %7.1
\begin{itemize}
\item MVC: Utilizzato per l'architettura generale dell'applicazione.
\end{itemize}
\subsection{Design Pattern creazionali} %7.2
\begin{itemize}
\item Singleton
\end{itemize}
\subsection{Design Pattern comportamentali} %7.3
\begin{itemize}
\item Factory
\end{itemize}
\subsection{Design Pattern strutturali} %7.3
\begin{itemize}
\item Facade
\item Adapter
\end{itemize}

\newpage
\section{Stime di fattibilità e di bisogno di risorse}%8.0

\newpage
\section{Tracciamento} %9.0
\subsection{Tracciamento componenti - requisiti} %9.1
\subsection{Tracciamento requisiti - componenti} %9.2

\newpage
\appendix
\section{Descrizione Design Pattern} %A.0

\subsection{Design Pattern architetturali} %A.1
\subsection{Design Pattern 1} %A.1.1
\begin{itemize}
\item \grassetto{Scopo:} 
\item \grassetto{Motivazione:} 
\item \grassetto{Applicabilità:}
\end{itemize}

\subsection{Design Pattern creazionali} %A.2
\subsection{Design Pattern 1} %A.2.1
\begin{itemize}
\item \grassetto{Scopo:} 
\item \grassetto{Motivazione:} 
\item \grassetto{Applicabilità:}
\end{itemize}

\subsection{Design Pattern strutturali} %A.3
\subsection{Design Pattern 1} %A.3.1
\begin{itemize}
\item \grassetto{Scopo:} 
\item \grassetto{Motivazione:} 
\item \grassetto{Applicabilità:}
\end{itemize}

\subsection{Design Pattern comportamentali} %A.4
\subsection{Design Pattern 1} %A.4.1
\begin{itemize}
\item \grassetto{Scopo:} 
\item \grassetto{Motivazione:} 
\item \grassetto{Applicabilità:}
\end{itemize}

%FINE DOCUMENTO NON CANCELLARE
\end{document}
