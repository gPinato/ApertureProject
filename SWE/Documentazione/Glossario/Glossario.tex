%includo il file che contiene la versione dei documenti
\newcommand{\versioneAnalisiDeiRequisiti}{2.2.0}			
\newcommand{\versioneNormeDiProgetto}{2.2.0}			
\newcommand{\versioneGlossario}{2.2.0}			
\newcommand{\versionePianoDiQualifica}{2.2.0}			
\newcommand{\versionePianoDiProgetto}{2.2.0}	
\newcommand{\versioneStudioDiFattibilita}{2.2.0}
\newcommand{\versioneSpecificaTecnica}{2.2.0}


\newcommand{\Versione}{\versioneGlossario{}}
\newcommand{\Data}{2013-11-28}
\newcommand{\DataUltimaModifica}{2013-12-18}
\newcommand{\TipoDocumento}{Glossario}
\newcommand{\Lettera}[1]{\huge #1}
\newcommand{\Riga}{\\\rule[2mm]{\textwidth}{0.25mm}}
\newcommand{\Termine}{\emph}
\newcommand{\Inizio}{}

%comando per creare una nuova pagina del glossario
%\newcommand{\paginaGlossario}[1]{\newpage\begin{flushright}\Lettera{#1}\end{flushright}}
\newcommand{\paginaGlossario}[1]{
\newpage
\setcounter{secnumdepth}{0} %elimino la numerazione della section
\section{\Lettera{#1}}
}

%comando per creare un nuovo elemento del glossario
\newcommand{\elemento}[2]{
\setcounter{secnumdepth}{0} %elimino la numerazione della subsection
\subsection{#1}
%\Riga
\begin{quote}
{#2}
\end{quote}
}

%questo file contiene impostazioni comuni per tutte i documenti

%definizione packages utilizzati
\documentclass[a4paper]{article}
\usepackage[utf8x]{inputenc}
\usepackage{enumitem}
\usepackage[italian]{babel}
\usepackage{latexsym}
\usepackage{xparse}
\usepackage{float}
\usepackage{subfloat}
\usepackage{subfig}
\usepackage{fancyhdr}
\usepackage{eurofont}
\usepackage{lastpage}
\usepackage{graphicx}
\usepackage{textcomp}
\usepackage{booktabs}
\usepackage{color}
\usepackage{lscape}
\usepackage{hyperref}
\hypersetup{colorlinks=true, linkcolor=black, anchorcolor=red, urlcolor=blue}
\usepackage{longtable}
\usepackage{tabularx}
\usepackage{abstract}
\usepackage{appendix}
\usepackage{multicol}
\usepackage{bmpsize}
\usepackage[all]{hypcap}
\usepackage{titlesec}
\usepackage{indentfirst}
\usepackage{lipsum,titletoc}

%\setcounter{secnumdepth}{4}

%****************INIZIO GESTIONE SUBSECTION MULTIPLE
\makeatletter
\newcommand\level[1]{%
  \ifcase#1\relax\expandafter\chapter\or
    \expandafter\section\or
    \expandafter\subsection\or
    \expandafter\subsubsection\else
    \def\next{\@level{#1}}\expandafter\next
  \fi}
\newcommand{\@level}[1]{%
  \@startsection{level#1}
    {#1}
    {\z@}%
    {-3.25ex\@plus -1ex \@minus -.2ex}%
    {1.5ex \@plus .2ex}%
    {\normalfont\normalsize\bfseries}}

\newdimen\@leveldim
\newdimen\@dotsdim
{\normalfont\normalsize
 \sbox\z@{0}\global\@leveldim=\wd\z@
 \sbox\z@{.}\global\@dotsdim=\wd\z@
}

\newcounter{level4}[subsubsection]
\@namedef{thelevel4}{\thesubsubsection.\arabic{level4}}
\@namedef{level4mark}#1{}
\def\l@section{\@dottedtocline{1}{0pt}{\dimexpr\@leveldim*4+\@dotsdim*1+6pt\relax}}
\def\l@subsection{\@dottedtocline{2}{0pt}{\dimexpr\@leveldim*5+\@dotsdim*2+6pt\relax}}
\def\l@subsubsection{\@dottedtocline{3}{0pt}{\dimexpr\@leveldim*6+\@dotsdim*3+6pt\relax}}
\@namedef{l@level4}{\@dottedtocline{4}{0pt}{\dimexpr\@leveldim*7+\@dotsdim*4+6pt\relax}}

\count@=4
\def\@ncp#1{\number\numexpr\count@+#1\relax}
\loop\ifnum\count@<100
  \begingroup\edef\x{\endgroup
    \noexpand\newcounter{level\@ncp{1}}[level\number\count@]
    \noexpand\@namedef{thelevel\@ncp{1}}{%
      \noexpand\@nameuse{thelevel\@ncp{0}}.\noexpand\arabic{level\@ncp{1}}}
    \noexpand\@namedef{level\@ncp{1}mark}####1{}%
    \noexpand\@namedef{l@level\@ncp{1}}%
      {\noexpand\@dottedtocline{\@ncp{1}}{0pt}{\the\dimexpr\@leveldim*\@ncp{5}+\@dotsdim*\@ncp{0}\relax}}}%
  \x
  \advance\count@\@ne
\repeat
\makeatother
\setcounter{secnumdepth}{100}
\setcounter{tocdepth}{100}
%****************FINE GESTIONE SUBSECTION MULTIPLE

%impostazioni relative alla visualizzazione delle section 
%nell'indice
\titlecontents{section}
[0pt]%left indent
{\bfseries}
{\contentslabel{2.3em}}
{\hspace*{-2.3em}}
{\hfill\contentspage}
[]%separator


\oddsidemargin=.15in
\evensidemargin=.15in
\textwidth=6in
\topmargin=-.5in
\parindent=0in
\headheight=1in
\DeclareMathSizes{10}{10}{10}{10} %per piano qualifica
\pagestyle{fancy}
\lhead{
\bfseries {\Large \TipoDocumento}\\
\bfseries Versione: \Versione\\
}
\chead{}
\lhead{
\includegraphics[scale=0.455]{../Logo&Header/apertureHead.png}
}
\lfoot{\bfseries \TipoDocumento{} v\Versione}
\cfoot{}
\rfoot{\thepage\ of \mypageref{LastPage}}
\newcommand{\mypageref}[1]{
\hypersetup{linkcolor=black}\pageref{#1}\hypersetup{linkcolor=black}}
%\userpackage{lipsum}
\renewcommand{\footrulewidth}{0.4pt}

%definizioni comandi comuni utilizzati
\newcommand{\numref}[1]{\textsl{\nameref{#1} (\ref{#1})}}
\newcommand{\NomeGruppo}{Aperture Software}
\newcommand{\Progetto}{MaaP: MongoDB as an admin Platform}
\newcommand{\Prop}{CoffeeStrap}

%definizione tecnologie
\newcommand{\Node}{Node.js}
\newcommand{\NodeJS}{Node.js}
\newcommand{\Nodejs}{Node.js}

\newcommand{\mongodb}{MongoDB}

%tanti sub quanti ne vogliamo! :)
\newcommand{\subsubsubsection}{\level{4}}
\newcommand{\subsubsubsubsection}{\level{5}}
\newcommand{\subsubsubsubsubsection}{\level{6}}
\newcommand{\subsubsubsubsubsubsection}{\level{7}}
\newcommand{\subsubsubsubsubsubsubsection}{\level{8}}


%definizione comando per parola glossario
\newcommand{\gloss}[1]{\emph{#1}\ped{\emph{\tiny{G}}}}

\newcommand{\grassetto}{\textbf}

%per inserire immagini
\newcommand{\immagine}[2]{ 
\begin{center}
\begin{figure}[H]
\includegraphics[width=\textwidth]{{{#1}}}
\caption{#2}
\label{#1}
\end{figure}
\end{center}
}

\newcommand{\Glossario}{
Al fine di evitare ogni ambiguità nella comprensione del linguaggio utilizzato nel presente documento e, in generale, nella documentazione fornita dal gruppo \NomeGruppo{}, ogni termine tecnico, di difficile comprensione o di necessario approfondimento verrà inserito nel documento \emph{Glossario\_{}v\versioneGlossario{}.pdf}.\\
Saranno in esso definiti e descritti tutti i termini in corsivo e allo stesso tempo marcati da una lettera "G" maiuscola in pedice nella documentazione fornita.
}

\newcommand{\Prodotto}{
Lo scopo del prodotto è produrre un framework per generare interfacce web di amministrazione dei dati di business basati sullo stack \Nodejs{} e \mongodb{}.\\
L'obiettivo è quello di semplificare il lavoro allo sviluppatore che dovrà rispondere in modo rapido e standard alle richieste degli esperti di business.
}

%inizio pagina del documento 
\begin{document}
\thispagestyle{empty}

\begin{center}\centerline{
%inserisco il logo grande della prima pagina
\includegraphics[scale=0.8]{../Logo&Header/logo.png}}

%metto il link dell'email sotto al logo
%{\href{mailto:ApertureSWE@gmail.com}{\color[rgb]{0.39,0.37,0.38}%ApertureSWE@gmail.com}}\\ [3pc]

\vspace{0.5in}

%titolo del progetto
{\Huge {\Progetto}}\\[.5pc]

\underline{\hspace{6in}}\\[8pc]

{\Huge {\TipoDocumento}}\\[1pc]
%{\emph{Versione \Versione}}\\
\end{center}

%\vspace{.05in}

\begin{center}

%\section{Informazioni documento}
\begin{tabular}{r|l}
%\textbf{Nome} &\TipoDocumento \\
\textbf{Versione} & \Versione\\
\textbf{Data creazione} & \Data \\
\textbf{Data ultima modifica} & \DataUltimaModifica \\
\textbf{Stato del documento} & Formale \\
\textbf{Uso del documento} & Esterno \\
\textbf{Redazione} & Alberto Garbui, Fabio Miotto,\\
                              & Mattia Sorgato, Michele Maso,\\
                              & Andrea Perin, Alessandro Benetti\\
\textbf{Verifica} & Giacomo Pinato \\
\textbf{Approvazione} & Alessandro Benetti \\
\textbf{Distribuzione} & \parbox[t]{4cm}{Prof. Tullio Vardanega \\ Prof. Riccardo Cardin \\ \Prop{} } \\
\end{tabular}
\end{center}
\vspace{.01in}

\begin{abstract}
\begin{center}
Questo documento si prefigge di chiarire le possibili ambiguità tra i vari termini utilizzati all'interno dei documenti redatti dal gruppo \NomeGruppo{}
\end{center}
\end{abstract}

%pagina diario delle modifiche
\newpage
\textbf{Diario delle modifiche}
\begin{center}
\begin{longtable}{|c|c|c|p{0.5\linewidth}|}
\toprule
\textbf{Versione} & \textbf{Data} & \textbf{Autore} & \textbf{Modifiche effettuate}\\
\midrule
2.0.2 & 2014-01-10 & Mattia Sorgato & Modifica lettere capitoli\\
\midrule
2.0.1 & 2014-01-09 & Mattia Sorgato & Correzione termini\\
\midrule
1.2.0 & 2013-12-18 & Alessandro Benetti & Approvazione finale del documento\\
\midrule
1.1.0 & 2013-12-17 & Giacomo Pinato & Verifica del documento \\
\midrule
1.0.8 & 2013-12-16 & Alessandro Benetti & Inserimento termini\\
\midrule
1.0.7 & 2013-12-13 & Fabio Miotto & Inserimento termini\\
\midrule
1.0.6 & 2013-12-12 & Michele Maso & Inserimento termini\\
\midrule
1.0.5 & 2013-12-09 & Mattia Sorgato & Inserimento termini\\
\midrule
1.0.4 & 2013-12-06 & Mattia Sorgato & Inserimento termini \\
\midrule
1.0.3 & 2013-12-06 & Andrea Perin & Inserimento termini\\
\midrule
1.0.2 & 2013-12-02 & Fabio Miotto & Inserimento termini\\
\midrule
1.0.1 & 2013-11-28 & Alberto Garbui & Inserimento termini\\
\midrule
1.0.0 & 2013-11-28 & Fabio Miotto & Creazione struttura iniziale del documento\\


\bottomrule
\caption{Registro delle modifiche}
\label{tab:changelog}
\end{longtable}
\end{center}

%terza pagina Indice (viene aggiornato in automatico con due compilazioni)
\newpage
\tableofcontents

%qui inizia la prima pagina ufficiale
\newpage
\section{Introduzione}%1.0
\label{1.0}
\subsection{Scopo del documento}%1.1
\label{1.1}
Questo documento raccoglie tutti i termini che sono sconosciuti al lettore esterno o che potrebbero generare ambiguità. Per ciascun termine viene riportato sotto il nome una breve definizione che ne chiarisce il significato.
\subsection{Riferimenti} %1.41
\label{1.4}
\subsubsection{Normativi} %1.4.1
\label{1.4.1}
\begin{itemize}
\item Norme di Progetto: \emph{Norme\_{}di\_{}Progetto\_{}v\versioneNormeDiProgetto{}.pdf}  (allegato alla presente documentazione)\\
\end{itemize}
\subsubsection{Informativi} %1.4.2
\label{1.4.2}
\begin{itemize}
\item Wikipedia:\\ \url{ http://www.wikipedia.org/}
\item OkPedia:\\ \url{ http://www.okpedia.it/}
\item W3C:\\ \url{http://www.w3.org/}
\item Techterms:\\ \url{http://www.techterms.com/}
\item ISO:\\ \url{http://www.iso.org/iso/home/about.htm}
\item Francesco Ranzato, Appunti di programmazione oggetti, Edizione II.\\
\end{itemize}

%inizio glossario
\newpage
%\Inizio{} %questo serve allo script del glossario
\section{Termini}
Di seguito saranno elencati e definiti tutti i vari termini in ordine alfabetico.

%\begin{flushright}
%\Lettera{\#}  %QUI CI SONO LE PAROLE DEL GLOSSARIO CHE INIZIANO CON UNA CIFRA
%\end{flushright}


%\newcommand {\gloss} [1]{#1}
%nuova pagina con i termini che iniziano con A
\Inizio{}
\paginaGlossario{A}


\elemento{Account} 
{L'insieme di funzionalità, strumenti e contenuti messi a disposizione da un sito web o da qualsiasi altro tipo di applicazione ad un utente, per usufruire di determinati servizi che il sistema offre.}

\elemento{Agente di provider} 
{Un agente che che ha la capacità e l'autorizzazione di eseguire azioni associate con un \gloss{servizio web} dedicato al suo possessore, ovvero l' entità di \gloss{provider}.}


\elemento{Alfanumerico}
{Carattere singolo composto solamente da lettere latine maiuscole o minuscole dalla a alla z e dai numeri dallo 0 al 9.}


\elemento{Ambiente}
{Consiste sia del sistema \gloss{hardware} che di quello \gloss{software} sui quali è stato pianificato l'utilizzo del prodotto software sviluppato.}


\elemento{Ambiguo}
{Termine usato per indicare una cosa che non è chiara, ovvero presenta un doppio senso o un fraintendimento.}

\elemento{Amministratore} 
{Classe di utenza privilegiata rispetto ad \gloss{utente} normale, ovvero ha funzionalità in più che gestiscono un sistema.}

\elemento{Applicazioni client-server}
{Applicazioni nelle quali esistono due componenti: un \gloss{client} che fa una richiesta al \gloss{server} e un \gloss{server} appunto che deve rispondere alla richiesta del \gloss{client}.}


\elemento{Applicazione web} 
{Applicazione fruibile via \gloss{web}, solitamente con architettura \gloss{client-server}.}


\elemento{Array} 
{Detto anche "vettore", è una struttura dati complessa, statica e omogenea, usata in molti \gloss{linguaggi di programmazione}. Si può immaginare un array come una sorta di casellario, le cui caselle sono dette celle dell'array stesso. Ciascuna delle celle si comporta come una \gloss{variabile} tradizionale che rappresenta un elemento dell'array; tutte le celle sono variabili di uno stesso tipo preesistente, detto \gloss{tipo base} dell'array.}

\elemento{Attributo}
{Descrizione di un campo dati di una \gloss{classe}.}

\paginaGlossario{B}


\elemento{Base di dati} 
{Detta anche banca dati, indica un archivio di dati, o un insieme di archivi, in cui le informazioni in esso contenute sono strutturate e collegate tra loro secondo un particolare modello logico e in modo tale da consentire la gestione/organizzazione efficiente dei dati stessi e l'interfacciamento con le richieste dell'\gloss{utente} attraverso le cosiddette \gloss{query} (di ricerca o interrogazione, inserimento, cancellazione, aggiornamento ecc.) grazie a particolari applicazioni \gloss{software} dedicate, basate su un'architettura di tipo \gloss{client-server}.}

\elemento{Baseline}
{Durante la pianificazione rappresenta la suddivisione iniziale delle attività nel tempo a cui si fa riferimento per l'avanzamento del processo.}

\elemento{Best practice} 
{Tradotto in "buona prassi", in genere identifica le esperienze più significative, o comunque quelle che hanno permesso di ottenere migliori risultati, relativamente a svariati contesti.}

\elemento{Branch}
{\`{E} considerato un ramo di sviluppo parallelo del filone principale.}

\elemento{Browser} 
{\gloss{Programma} che consente di usufruire dei \gloss{servizi} di connettività in \gloss{Internet}, o di una \gloss{rete} di computer, e di navigare sul \gloss{Web}.}


\elemento{Business}
{In italiano è tradotto con affari; identifica un'attività economica.}


\elemento{Business logic} 
{Si riferisce a tutta quella logica applicativa che rende operativa un'applicazione. \`{E} un termine largamente utilizzato nella progettazione del \gloss{software} per individuare un componente \gloss{software} di una architettura \gloss{software}. Nelle    \gloss{applicazioni web} viene eseguita da un \gloss{server} su richiesta di un \gloss{client} attraverso il \gloss{browser}   web e interfacciandosi con la parte dati che può essere un \gloss{database}.}



\elemento{Bytecode} 
{\gloss{Codice} di \gloss{programmazione} che, una volta effettuata la \gloss{compilazione}, è eseguibile attraverso una \gloss{macchina virtuale} invece che dal \gloss{processore} di un computer.}

\paginaGlossario{C}

\elemento{Chat}
{Insieme di \gloss{servizi} che hanno due caratteristiche in comune: la prima è che il dialogo avviene in tempo reale, mentre la seconda è che ci si può mettere in contatto con perfetti sconosciuti.}

\elemento{Chiave}
{\`{E} un identificatore di un elemento del \gloss{database} di tipo \gloss{NoSQL}.}

\elemento{Classe}
{Nella \gloss{programmazione} \gloss{orientata agli oggetti} una classe è un costrutto di un \gloss{linguaggio di programmazione} usato come modello per creare \gloss{oggetti}. Il modello comprende \gloss{attributi} e \gloss{metodi} che saranno condivisi da tutti gli oggetti creati (\gloss{istanze}) a partire dalla classe. Un "oggetto" è, di fatto, l'\gloss{istanza} di una classe.}

\elemento{Client} 
{Componente che accede ai \gloss{servizi} o alle risorse di un'altra componente detta \gloss{server}.} 

\elemento{Cloud} 
{Insieme di tecnologie che permettono di archiviare e/o elaborare dati mediante l'utilizzo di risorse \gloss{hardware} e \gloss{software} \gloss{distribuite}.}

\elemento{Codice} 
{Vedi codice sorgente.}


\elemento{Codice oggetto}
{Traduzione del \gloss{codice sorgente} in \gloss{linguaggio macchina}, comprensibile dal \gloss{processore}.}


\elemento{Codice sorgente}
{Linee di codice che compongono un \gloss{programma} scritto in un \gloss{linguaggio di programmazione}. Abbreviato anche come "sorgente" o solamente "codice".}


\elemento{Collection} 
{Raccolta di \gloss{Document} in \gloss{database} come \gloss{MongoDB}.}


\elemento{Collection-Index} 
{Pagina generata dal \gloss{framework} \gloss{MaaP} che mostra un elenco di \gloss{Document} con delle specifiche coppie chiavi-valori; inoltre è presente un menù dove è possibile spostarsi tra le varie pagine \gloss{Collection Index} o sfruttare altre funzionalità messe a disposizione dallo \gloss{sviluppatore}.}


\elemento{Compatibilità}
{In informatica indica la proprietà di due o più computer che, scambiandosi dei dati, accettano gli stessi \gloss{programmi} senza fare modifiche ad essi.}


\elemento{Compilazione} 
{Processo di traduzione che porta alla creazione di un \gloss{codice oggetto} partendo da un \gloss{codice sorgente}.}


\elemento{Criptato} 
{Questo termine viene utilizzato per riferirsi a quella metodologia che fa sì che l'invio dei dati sia reso non comprensibile ad eventuali intercettazioni.}

\paginaGlossario{D}


\elemento{Dashboard}
{Pagina principale dove l'\gloss{utente} può aver accesso alle varie funzionalità in modo chiaro e diretto.}


\elemento{Database} 
{Vedi \gloss{basi di dati}.}


\elemento{Database administration} 
{\`{E} la funzione di gestione e manutenzione dei sistemi di gestione dei \gloss{database}; molte aziende importanti necessitano continuamente di una gestione delle \gloss{basi di dati}.}


\elemento{Debugger} 
{\gloss{Software} specificatamente progettato per l'analisi e l'eliminazione dei bug, ovvero errori di programmazione interni al \gloss{codice} di altri \gloss{programmi}.}

\elemento{Default} 
{Valore o un'azione standard che caratterizza qualsiasi aspetto globale a meno di un cambiamento improvviso.}

\elemento{Deployment} 
{Traducibile letteralmente in "spiegamento". \`{E} l'insieme di attività necessarie a rendere un sistema \gloss{software} disponibile all'uso.}





\elemento{Diagramma di Gantt}
{Strumento di supporto alla gestione dei progetti. \`{E} costruito partendo da un asse orizzontale, a rappresentazione dell'arco temporale totale del progetto, suddiviso in fasi incrementali (ad esempio, giorni, settimane, mesi), e da un asse verticale, a rappresentazione delle mansioni o attività che costituiscono il progetto. Un diagramma di \gloss{Gantt} permette dunque la rappresentazione grafica di un calendario di attività, utile al fine di pianificare, coordinare e tracciare specifiche attività in un progetto dando una chiara illustrazione dello stato d'avanzamento del progetto rappresentato.} 



\elemento{Distribuito}
{Insieme di entità autonome (componenti \gloss{software} e \gloss{hardware}) fisicamente separate che comunicano e coordinano tra loro le azioni attraverso scambio di messaggi.}

\elemento{Distribuzione} 
{Collezione di \gloss{programmi} relativi ad uno o più campi di applicazione, selezionati e rilasciati come un unico pacchetto. Ad esempio il \gloss{sistema operativo} \gloss{Linux} offre più diverse distribuzioni, come ad esempio Ubuntu, Debian, Fedora, eccetera.}

\elemento{Document}
{\gloss{Istanza}, o \gloss{record}, di un \gloss{database} in \gloss{MongoDB}. Esso è costituito da un insieme di \gloss{chiavi} con il rispettivo valore.}


\elemento{Document-Show} 
{Pagina che, in seguito ad una selezione di una \gloss{chiave} di un \gloss{Document} nella pagina \gloss{Collection-Index}, mostra il Document per intero con tutte le chiavi e i relativi valori.}

\elemento{Driver}
{Insieme di procedure che permette ad un \gloss{sistema operativo} di pilotare un dispositivo \gloss{hardware} o \gloss{software}.}

\elemento{DSL} 
{Acronimo di "Domain Specific Language", "linguaggio specifico di dominio". \`{E} un \gloss{linguaggio di programmazione} altamente contestualizzato, cioè associato ad un dominio specifico.} 


\paginaGlossario{E}


\elemento{Editor}
{\gloss{Programma} di composizione di testi, il suo scopo è facilitare la scrittura di un testo. \`{E} generalmente incluso in ogni \gloss{sistema operativo}.}


\elemento{Email} 
{\gloss{Servizio} \gloss{internet} grazie al quale ogni \gloss{utente} abilitato può inviare e ricevere dei messaggi utilizzando un computer o qualsiasi altro dispositivo elettronico connesso a internet, attraverso un proprio \gloss{account} di posta registrato presso un \gloss{provider} del servizio.}


\elemento{Express} 
{\gloss{Applicazione web} scritta in \gloss{Node.js} minima e flessibile, che fornisce un robusto set di funzionalità per la costruzione di singole e multi-pagine applicazioni web ibride.}

\paginaGlossario{F}


\elemento{File di configurazione}
{\`{E} un file che permette di impostare parametri necessari al funzionamento dell'applicazione nella loro versione predefinita; ad esempio \gloss{database} per sistema di \gloss{autenticazione}, \gloss{e-mail}, \gloss{server}, ecc.}



\elemento{File di descrizione}
{File scritto con \gloss{linguaggio} \gloss{DSL} dallo \gloss{sviluppatore}, e serve per generare pagine di tipo \gloss{Collection-Index} e \gloss{Document-Show}.}





\elemento{File system}
{Meccanismo con il quale i file sono posizionati e organizzati su un dispositivo di archiviazione.}


\elemento{Filtro} 
{Componente che ha il compito di selezionare una fonte in ingresso secondo dei criteri, in modo da avere un risultato finale utile all'\gloss{utente}.}


\elemento{Firefox}
{\`{E} un browser \gloss{open-source} multipiattaforma e secondo alcune statistiche è il secondo browser più popolare.}


\elemento{Form}
{Termine usato per indicare l'\gloss{interfaccia} di una applicazione che consente ad un \gloss{utente} di inserire e inviare ad un \gloss{server} dei dati liberamente digitati dallo stesso utente.}


\elemento{Framework}
{Struttura di supporto su cui un \gloss{software} può essere progettato e realizzato. Alla base di un framework sono sempre presenti delle \gloss{librerie di codice} utilizzabili con uno o più \gloss{linguaggi di programmazione}; esse sono spesso corredate da una serie di strumenti di supporto allo sviluppo software, come ad esempio un \gloss{IDE} o un \gloss{debugger}, o altri strumenti ideati per aumentare la velocità di sviluppo del prodotto finito.
Lo scopo di un framework è quello di far risparmiare allo \gloss{sviluppatore} la riscrittura di \gloss{codice} già scritto precedentemente per fini simili. La necessità di questo strumento si è venuta a creare quando le \gloss{interfacce} utente sono diventate sempre più complesse ed è cominciata ad aumentare la quantità software con funzionalità secondarie simili.}


\elemento{Funzione-software}
{Particolare costrutto con \gloss{sintassi}, in qualche \gloss{linguaggio di programmazione}, che permette di raggruppare, all'interno di un \gloss{programma}, una sequenza di istruzioni in un unico blocco di istruzioni espletando così una determinata e in generale più complessa operazione, azione o elaborazione sui dati del \gloss{programma} stesso in modo tale che a partire da determinati \gloss{input} restituisca determinati \gloss{output}.}

\paginaGlossario{G}


\elemento{Gantt} 
{L'inventore dei \gloss{diagrammi di Gantt}, che vengono usati nella gestione e nella pianificazione delle attività.}


\elemento{Git} 
{Sistema di controllo di \gloss{versione} \gloss{distribuito} gratuito e \gloss{open source} designato alla gestione di progetti \gloss{software}}


\elemento{GitHub} 
{\gloss{Servizio} di \gloss{web} \gloss{hosting} orientato allo sviluppo \gloss{software} e basato sul sistema di controllo di \gloss{versione} di \gloss{Git}. GitHub offre \gloss{servizi} di \gloss{repository} online sia gratuiti che a pagamento.}


\elemento{Gmail} 
{\gloss{Servizio} gratuito di posta elettronica gestito e offerto da \gloss{Google Inc. }.}

\elemento{Google} 
{Motore di ricerca per \gloss{internet} che oltre alla funzione di effettuare ricerche offre molti altri \gloss{servizi}, per esempio gestione \gloss{e-mail}, gestione calendari ecc..} 


\elemento{Google Chrome} 
{\gloss{Browser} \gloss{web} gratuito creato da \gloss{Google Inc.}.}


\elemento{Google Drive} 
{\gloss{Servizio web} offerto e gestito da \gloss{Google Inc.} che permette l'archiviazione, la sincronizzazione, la condivisione e la modifica collaborativa di documenti.}

\elemento{Google Inc.}
{\`{E} un'azienda statunitense che offre \gloss{servizi} online, principalmente nota per il motore di ricerca \gloss{Google} e per servizi come \gloss{Google Drive} ecc.}

\paginaGlossario{H}


\elemento{Hardware} 
{Componenti fisiche che compongono un computer.}


\elemento{Heroku} 
{Piattaforma \gloss{cloud} per per fornire \gloss{servizi} tramite il \gloss{deployment} di applicazioni scritte in \gloss{linguaggi} quali \gloss{Node.js}, \gloss{Ruby}, \gloss{Scala}, e molti altri.}


\elemento{Host} 
{Ogni terminale collegato ad una \gloss{rete} o più in particolare ad \gloss{internet}.}


\elemento{Hosting} 
{Tradotto letteralmente in "ospitare". \gloss{Servizio} di \gloss{rete} che consiste nell'allocare su un \gloss{server web} le \gloss{pagine web} di un \gloss{sito web}, rendendolo così accessibile dalla \gloss{rete} \gloss{internet} e ai suoi \gloss{utenti}.}

\paginaGlossario{I}


\elemento{IDE} 
{Acronimo di "Integrated Developement Enviroment", tradotto in "ambiente di sviluppo integrato". \gloss{Software} che, durante la \gloss{programmazione}, aiuta i \gloss{programmatori} nello sviluppo del \gloss{codice sorgente} di un \gloss{programma}.}


\elemento{IEC} 
{Acronimo di "International Electrotechnical Commission", tradotto in "Commissione Elettrotecnica Internazionale". \`{E} un'organizzazione internazionale per la definizione di standard in materia di elettricità, elettronica e tecnologie correlate. Molti dei suoi standard sono definiti in collaborazione con l'\gloss{ISO}.}

\elemento{Infrastruttura} 
{Insieme di risorse \gloss{hardware} e risorse \gloss{software}.}


\elemento{Input}
{Tradotto letteralmente in "immettere". Sequenza di dati o informazioni, immessi per mezzo di una \gloss{periferica}, detta appunto di \gloss{input}, e successivamente elaborati.}





\elemento{Interfaccia} 
{Dispositivo fisico o virtuale che permette la comunicazione e l'interazione tra due entità. Ad esempio in un'\gloss{interfaccia} grafica di un \gloss{programma} posso inserire dei valori  e visualizzare un eventuale risultato o una risposta da parte del programma stesso.}


\elemento{Internet} 
{\`{E} una contrazione della locuzione inglese "interconnected networks", ovvero "reti interconnesse".
 Ed è una \gloss{rete} mondiale di reti di computer ad accesso pubblico, attualmente rappresentante il principale mezzo di comunicazione di massa, che offre all'\gloss{utente} una vasta serie di contenuti potenzialmente informativi e \gloss{servizi}.}


\elemento{Internet Protocol}
{\gloss{Protocollo di comunicazione} di \gloss{rete} appartenente all'insieme di protocolli \gloss{internet} TCP/IP su cui è basato il funzionamento della \gloss{rete} internet.}



\elemento{Interprete} 
{\gloss{Programma} in grado di eseguire altri programmi a partire direttamente dal relativo \gloss{codice sorgente}. Un interprete ha lo scopo di eseguire un programma in un \gloss{linguaggio di alto livello}, senza la previa \gloss{compilazione} dello stesso (\gloss{codice oggetto}) cioè di eseguire le istruzioni nel \gloss{linguaggio} usato, traducendole di volta in volta in istruzioni in \gloss{linguaggio macchina}.}

\elemento{IP} 
{Acronimo di "Internet Protocol Address".L'IP è un etichetta numerica che identifica univocamente un dispositivo (\gloss{host}) collegato a una \gloss{rete} informatica che utilizza l' Internet Protocol come \gloss{protocollo di comunicazione}.}


\elemento{Ipertesto} 
{Insieme di documenti messi in relazione tra loro per mezzo di parole chiave.} 


\elemento{ISO} 
{Acronimo di "International Organization for Standardization", tradotto in "Organizzazione Internazionale per la Normazione". Come dice il nome, è un'organizzazione internazionale atta alla realizzazione di specifiche standard per la realizzazione di prodotti, \gloss{servizi} e pratiche corrette per aiutare ogni impresa a lavorare in modo efficace ed efficiente.}


\elemento{Istanza}
{Un particolare \gloss{oggetto} di una particolare \gloss{classe}.}



\paginaGlossario{J}


\elemento{Java} 
{\gloss{Linguaggio di programmazione} \gloss{orientato agli oggetti}, creato dalla \gloss{Sun Microsystems}.}

\elemento{JavaScript} 
{\gloss{Linguaggio di scripting orientato agli oggetti} comunemente usato nella creazione di \gloss{siti web}. In applicazioni di tipo \gloss{client-server} se il codice JavaScript è sul lato \gloss{client} allora viene eseguito sul \gloss{client} e non sul \gloss{server}, così da non sovraccaricare la parte server.}


\elemento{JSON} 
{Acronimo di "JavaScript Object Notation", tradotto in "notazione oggetto \gloss{JavaScript}". \`{E} un formato di memorizzazione di dati, di piccolo peso e intuitivo.}


\elemento{JVM}
{Acronimo di "Java Virtual Machine", tradotto in "\gloss{macchina virtuale} \gloss{Java}". \`{E} il componente della \gloss{piattaforma Java} che esegue i \gloss{programmi} tradotti in \gloss{bytecode} dopo una prima \gloss{compilazione}.}

\paginaGlossario{L}


\elemento{LaTeX} 
{\gloss{Linguaggio} usato per la preparazione di testi basato sul \gloss{programma} \gloss{TeX}}


\elemento{Layout} 
{Identifica l'impaginazione e la struttura grafica di una \gloss{pagina web} o di un documento.}


\elemento{Libreria di codice} 
{Insieme di funzioni o strutture dati predisposte per essere collegate ad un \gloss{programma} \gloss{software} attraverso opportuno collegamento.}


\elemento{Linguaggio di alto livello} 
{\gloss{Linguaggio di programmazione} più astratto del \gloss{linguaggio macchina}, direttamente eseguibile da un computer, ma più vicino o familiare alla logica del nostro \gloss{linguaggio} naturale. I \gloss{programmi} ad alto livello possono essere ricondotti a programmi in \gloss{linguaggio macchina} in modo automatico, ovvero da un altro programma, detto \gloss{interprete}.}


\elemento{Linguaggio di programmazione} 
{\`{E} un \gloss{linguaggio formale}, ben definito e composto da una \gloss{sintassi} e una \gloss{semantica}.}


\elemento{Linguaggio di scripting} 
{\gloss{Linguaggio interpretato}, destinato in genere a compiti di automazione del \gloss{sistema operativo}, oppure viene usato all'interno delle \gloss{pagine web} per gestire il comportamento delle \gloss{pagine web} stesse in base all' \gloss{input} dell'\gloss{utente}.}


\elemento{Linguaggio formale} 
{Notazione o formalismo con \gloss{sintassi} e \gloss{semantica} definite in modo preciso (spesso matematico/formale) e, in molti casi, tali da consentire qualche forma di elaborazione automatica del \gloss{linguaggio} stesso.}


\elemento{Link} 
{Abbreviazione di "hyperlink". \`{E} un collegamento ipertestuale in grado di rinviare a un contenuto informativo presente in un dominio fisicamente o virtualmente separato. Il link è di solito associato ad una o più parole chiave, evidenziate visivamente da una diversa colorazione o sottolineatura.}


\elemento{Linux} 
{\gloss{Sistema operativo} della \gloss{Linux Foundation}.}


\elemento{Linux Foundation} 
{Associazione senza fini di lucro, specializzata nel campo dell'informatica \gloss{open source}.}


\elemento{Linguaggio interpretato} 
{Un \gloss{linguaggio} informatico è per definizione diverso dal \gloss{linguaggio macchina}. Bisogna quindi tradurlo per renderlo leggibile dal punto di vista del \gloss{processore}. Un \gloss{programma} scritto in un \gloss{linguaggio interpretato} ha bisogno di un programma ausiliario (l'\gloss{interprete}) per tradurre man mano le istruzioni del programma in linguaggio macchina.}


\elemento{Linguaggio macchina} 
{Definito come il \gloss{linguaggio} utilizzato dal \gloss{processore}, tramite sequenze di 0 e 1. Ogni sequenza ordinata di 0 e 1, raggruppata in gruppi di una certa dimensione, identifica una precisa istruzione per il \gloss{processore}.}


\elemento{Logger}
{Componente non intrusivo di registrazione dei dati di esecuzione per analisi dei risultati.}

\paginaGlossario{M}





\elemento{MaaP}
{\gloss{Framework} che genera \gloss{interfacce} \gloss{web} di amministrazione dei dati di \gloss{business} basati sulle tecnologie\gloss{ Node.js} e \gloss{MongoDB}.}

\elemento{Macchina virtuale} 
{Implementazione \gloss{software} di un \gloss{ambiente} di elaborazione in cui un \gloss{sistema operativo} o un \gloss{programma} possono essere installati ed eseguiti.}


\elemento{MaaP's web}
{Insieme delle \gloss{pagine web} prodotte dal \gloss{framework} \gloss{MaaP}.}


\elemento{Metodo}
{Detto anche funzione-software membro, è un termine che viene usato principalmente nel contesto della \gloss{programmazione} \gloss{orientata agli oggetti} per indicare un sottoprogramma associato in modo esclusivo ad una \gloss{classe} e che rappresenta (in genere) un'operazione eseguibile sugli oggetti-software e \gloss{istanze} di quella classe.}


\elemento{Milestone}
{Viene utilizzato nella pianificazione e gestione di progetti per indicare il raggiungimento di obiettivi definiti in fase di definizione del progetto stesso. Molto spesso le milestone sono rappresentate da eventi, come ad esempio scadenze per la consegna di documenti, e indicano importanti traguardi intermedi durante lo svolgimento del progetto.}


\elemento{MongoDB}
{Sistema gestionale di \gloss{basi di dati}, non \gloss{relazionale}, \gloss{orientato ai documenti} e di tipo \gloss{NoSQL}. Il \gloss{linguaggio di programmazione} utilizzato per la gestione dei dati è \gloss{JavaScript}, in particolare la sua notazione \gloss{JSON}.}


%nuova pagina con i termini che iniziano con N
\paginaGlossario{N}

\elemento{Namespace}
{Indica una collezione di nomi di entità, definite dal programmatore, omogeneamente usate in uno o più file sorgente. Lo scopo del namespace è quello di evitare confusione ed equivoci nel caso siano necessarie molte entità con nomi simili, fornendo il modo di raggruppare i nomi per categorie.}

\elemento{Node.js}
{\gloss{Framework} basato sul \gloss{linguaggio} \gloss{JavaScript}. Pone la sua attenzione alla manipolazione di grosse quantità di dati, quali, per esempio, la consultazione di \gloss{database}.}

\elemento{NoSQL}
{Acronimo di "Not Only SQL", "non soltanto SQL". \`{E} un particolare tipo di \gloss{database} che fornisce un sistema di immagazzinamento dei dati e di un loro successivo recupero che richiede l'utilizzo di modelli meno vincolati e restrittivi rispetto ai database di tipo \gloss{relazionale}. Questo approccio permette una maggiore semplicità del processo di modellazione dei dati, una migliore propensione allo \gloss{scaling orizzontale} del database e un controllo maggiore della disponibilità dei dati.}


\paginaGlossario{O}


\elemento{Oggetto-software}
{Definito come un \gloss{tipo-informatica} di dato astratto, un oggetto è un insieme di valori ed operazioni che permettono di manipolare tali valori, dette operazioni proprie del tipo o metodi del tipo.}

\elemento{Open source}
{\gloss{Software} i cui autori ne permettono e favoriscono il libero studio e l'apporto di modifiche da parte di altri \gloss{programmatori} indipendenti.}

\elemento{Orientamento agli oggetti}
{\gloss{Paradigma di programmazione} che permette di definire \gloss{oggetti-software} in grado di interagire gli uni con gli altri attraverso lo scambio di messaggi.}

\elemento{Orientamento ai documenti}
{Le \gloss{basi di dati} orientate ai documenti non memorizzano i dati in tabelle con campi uniformi per ogni \gloss{record} come invece succedeva nei \gloss{database} \gloss{relazionali}, ma ogni record è memorizzato come un documento che possiede determinate caratteristiche. Al documento può essere aggiunto un numero qualsiasi di campi ed essi posso anche contenere pezzi multipli di dati.}

\elemento{Orientamento funzionale}
{\gloss{Paradigma di programmazione} in cui il flusso di esecuzione del \gloss{programma} assume la forma di una serie di valutazioni di funzioni matematiche.}

\elemento{Output}
{Tradotto letteralmente in "messo fuori". Indica in senso stretto il risultato di una elaborazione ed in senso più ampio il risultato o l'insieme dei risultati prodotti a partire da un \gloss{input}.}

\elemento{Overflow}
{Errore che occorre quando si eccede la memoria allocata disponibile.}


\paginaGlossario{P}


\elemento{Package}
{Un package, definito principalmente in ambiente \gloss{Java}, è un meccanismo per organizzare \gloss{classi} all'interno di sottogruppi ordinati. I \gloss{programmatori} spesso usano i package per riunire classi logicamente correlate o che forniscono servizi simili.}

\elemento{Pagine web}
{Il modo in cui vengono rese disponibili all'\gloss{utente} finale le informazioni reperibili su \gloss{Internet}, tramite un  \gloss{web browser}. Un insieme di pagine web tra di loro correlate formano un \gloss{sito web}. Una pagina web si può suddividere in una parte relativa ai contenuti, una parte di \gloss{layout} e una parte dedicata al comportamento a seconda degli \gloss{input} dell'utente.}

\elemento{Paradigma client-server}
{Indica un'architettura di rete nella quale genericamente un computer client si connette ad un server per usufruire di un certo servizio, ad esempio la condivisione di una certa risorsa hardware o software con altri client.}

\elemento{Paradigma di programmazione}
{Stile fondamentale di \gloss{programmazione}, ovvero un insieme di strumenti concettuali forniti da un \gloss{linguaggio di programmazione} per la stesura del \gloss{codice sorgente} di un \gloss{programma}, definendo dunque il modo in cui il \gloss{programmatore} concepisce e percepisce il programma stesso.}

\elemento{Parser}
{Programma che esegue il compito del parsing.}

\elemento{Parsing}
{Processo che analizza uno stream continuo in input in modo da determinare la sua struttura grammaticale grazie ad una data grammatica formale.}

\elemento{Password}
{Parola di riconoscimento impiegata a scopo di sicurezza per garantire che l'uso di una risorsa sia concesso solo agli \gloss{utenti} autorizzati; costituita da una sequenza ordinata di caratteri \gloss{alfanumerici} e/o speciali (quali, per esempio, @, \%, \$).}

\elemento{Periferica}
{Una qualsiasi tipologia di dispositivo \gloss{hardware} del computer che si interfaccia in \gloss{input} e/o \gloss{output} con l'unità di elaborazione che sovrintende a tutte le funzioni del computer (\gloss{processore}).}

\elemento{Permesso}
{O autorizzazione, indica quali funzionalità possono competere.}

\elemento{Persistenza}
{Indica la caratteristica dei dati di sopravvivere all'esecuzione del programma che li ha creati. La persistenza si riferisce in particolare alla possibilità di far sopravvivere delle strutture dati all'esecuzione di un singolo programma. Questa possibilità è raggiunta salvando i dati in uno storage non volante come i database.}

\elemento{Pianificare}
{Termine usato per prevedere in linea di massima quando compiere un'attività e/o una serie di attività.}

\elemento{Piattaforma Java}
{Piattaforma \gloss{software} sviluppata su specifiche e implementazioni da \gloss{Sun Microsystems} che è eseguibile su piattaforme \gloss{hardware} di diversa natura.}

\elemento{Plug in}
{\gloss{Programma} non autonomo che interagisce con un altro programma per ampliarne le funzionalità.}

\elemento{Portabilità}
{Processo di adattare un software cosicchè un programma eseguibile può essere creato per un ambiente di computazione che è differente dal proprio per cui era stato originariamente designato.}

\elemento{Proattivo}
{Tipo di approccio che permette di prevenire e anticipare i problemi e i bisogni futuri; necessita di pianificazione e di esperienza; approccio molto usato per ridurre il carico di lavoro nell'attività di Verifica.}

\elemento{Procedura}
{Indica un modo di procedere, cioè di operare o di comportarsi in determinate circostanze o per ottenere un certo risultato.}

\elemento{Processo}
{Insieme di attività correlate e coese che trasformano ingressi in uscite secondo regole fissate, consumando risorse nel farlo.}

\elemento{Processore}
{Detto anche unità di elaborazione. Esso è un tipo di dispositivo \gloss{hardware} del computer che si contraddistingue per essere dedicato all'esecuzione di istruzioni. In altri termini l'unità di elaborazione è il dispositivo che nel computer esegue materialmente l'elaborazione dati.}

\elemento{Profilo}
{Insieme di dati relativi ad un \gloss{utente} in un \gloss{sistema} informatico. Può contenere informazioni differenti, a seconda del contesto e delle necessità del sistema.}

\elemento{Programma}
{Insieme di istruzioni che, una volta eseguite su un computer, produce soluzioni per una data classe di problemi automatizzati.}

\elemento{Programmatore}
{Vedi \gloss{sviluppatore}.}

\elemento{Programmazione}
{Attività di sviluppo di \gloss{software}, consistente nella stesura di \gloss{codice sorgente}.}

\elemento{Protocollo di comunicazione}
{Insieme di regole formalmente descritte, definite al fine di favorire la comunicazione tra una o più entità.}

\elemento{Protocollo di sottoscrizione}

\elemento{Provider}
{La persona o l'organizzazione che fornisce un \gloss{servizio web}. "Provider", letteralmente tradotto, significa "fornitore".}

\elemento{Python}
{Linguaggio di programmazione ad alto livello, orientato agli oggetti, adatto, tra gli altri usi, per sviluppare applicazioni distribuite, scripting, computazione numerica e testing.}

\paginaGlossario{Q}


\elemento{Query}
{L'interrogazione da parte di un \gloss{utente} di un \gloss{database} per compiere determinate operazioni sui dati.}


\paginaGlossario{R}


\elemento{Record}
{Un oggetto di una \gloss{base di dati} strutturata in dati compositi, che contengono un insieme di campi o elementi, ciascuno dei quali possiede nome e \gloss{tipo} di dato propri.}

\elemento{Redmine}
{Software di project management gratis e open source; mette a disposizione un calendario per la pianificazione delle attività e offre la possibilità di visualizzare i diagrammi di Gantt; inoltre supporta il sistema di ticketing.}

\elemento{Registrazione}
{Azione tramite la quale un \gloss{utente} attraverso l'inserimento di alcuni dati richiesti, entra a far parte di un sistema al quale vuole registrarsi.}

\elemento{Relazionale}
{Modello logico di rappresentazione o strutturazione dei dati di un \gloss{database}. Si basa sulla teoria degli insiemi e sulla logica del primo ordine ed è strutturato intorno al concetto matematico di relazione.}

\elemento{Repository}
{Ambiente che offre la possibilità di salvataggio di dati per la sicurezza di essi, inoltre è possibile godere di funzionalità di \gloss{versionamento} per tener traccia della storia dei dati salvati.}

\elemento{Requisito}
{Capacità che un sistema \gloss{software} deve soddisfare per rispettare un contratto.}

\elemento{Rete}
{Serie di componenti, sistemi o entità interconnessi tra di loro.}

\elemento{Risorsa}
{Componente fisico o virtuale che un sistema richiede e che grazie ad esso offre una certa funzionalità.}

\elemento{Ruby}
{\gloss{Linguaggio di scripting} completamente a oggetti.}


\paginaGlossario{S}


\elemento{Scala}
{\gloss{Linguaggio di programmazione} che integra linguaggi \gloss{orientati agli oggetti} e \gloss{linguaggi funzionali}. Un programma \gloss{Scala}, una volta \gloss{compilato}, può essere eseguito su \gloss{JVM}.}

\elemento{Scaling orizzontale}
{Per scalabilità si intende la proprietà di un sistema di crescere o decrescere in base alle necessità. Nel mondo dei \gloss{database} scalare orizzontalmente una \gloss{base di dati} significa che anche se vado ad aumentare le componenti del sistema che accedono al database, questo non provoca interferenze, ed aumentando le componenti si parallelizza il carico di lavoro, diminuendo il peso del singolo.}

\elemento{Script}
{Programma scritto in un particolare linguaggio di programmazione. Lo script ha complessità relativamente bassa e non ha una propria interfaccia grafica.}

\elemento{Scope}
{Nome, tipicamente univoco, che all'interno di un programma ne identifica una particolare parte; lo scope è utilizzato quando si vuole fare riferimento a qualche parte del programma in particolare.}

\elemento{Semantica}
{Regole per la \gloss{sintassi}.}

\elemento{Server}
{Componente o sottosistema informatico di elaborazione che fornisce un qualunque tipo di \gloss{servizio} ad altre componenti (tipicamente chiamate \gloss{client}) che ne fanno una esplicita richiesta.}

\elemento{Server web}
{Entità \gloss{hardware} o \gloss{software} che rilascia \gloss{pagine web} al \gloss{client}.}

\elemento{Servizio}
{Insieme di funzionalità \gloss{software} che possono essere riutilizzate per differenti scopi.}

\elemento{Servizio web}
{Risorsa astratta che rappresenta la capacità di effettuare compiti che compongono una funzionalità coerente dal punto di vista delle entità di \gloss{provider} e dei richiedenti del servizio. Per essere utilizzato, un \gloss{servizio} deve essere realizzato da un \gloss{agente di provider} concreto.}

\elemento{Shard}
{MONGODB ?????}

\elemento{Simil-chat}
{Vedi \gloss{chat}.}

\elemento{Singleton}
{Design patter creazionale, che assicura l'esistenza di un'unica di una classe.}

\elemento{Sintassi}
{Notazione semplice con vincoli.}

\elemento{Sistema di autenticazione}
{Sistema che gestisce un processo tramite il quale un computer, un \gloss{software} o un \gloss{utente}, verifica la corretta, o presunta, identità di un altro computer, utente, che vuole comunicare attraverso una connessione.}

\elemento{Sistema operativo}
{Insieme di componenti \gloss{software} che permettono l'utilizzo da parte di un \gloss{utente} di applicazioni installate su una data macchina.}

\elemento{Sito web}
{Insieme di \gloss{pagine web} correlate, ovvero una struttura \gloss{ipertestuale} di documenti che risiede, tramite \gloss{hosting}, su un \gloss{web} \gloss{server} e accessibile all'\gloss{utente} \gloss{client} che ne fa richiesta tramite un \gloss{browser} sul \gloss{World Wide Web} della \gloss{rete} \gloss{Internet}, digitando in esso il rispettivo \gloss{URL} o direttamente l'indirizzo \gloss{IP}.}

\elemento{Skype}
{Software proprietario gratuito di messaggistica istantanea e \gloss{VoIP}.}

\elemento{Social Network}
{Tradotto letteralmente in "rete sociale". Consiste in una struttura informatica che gestisce nel Web le reti basate su relazioni sociali. La struttura è identificata, ad esempio, per mezzo del sito web di riferimento della rete sociale.}

\elemento{Software}
{L'informazione o le informazioni utilizzate da uno o più sistemi informatici e memorizzare su uno o più supporti informatici. Queste informazioni possono essere \gloss{programmi}, dati oppure una combinazione di tutte e due.}

\elemento{Sottosistema}

\elemento{Stack}
{Indica un tipo di dato astratto che viene usato in diversi contesti per riferirsi a strutture dati; la modalità di accesso ai dati contenuti in uno stack è di tipo LIFO, ovvero Last In First Out.}

\elemento{Statement}
{Traducibile in "sequenza di istruzioni".}

\elemento{Strategy}
{Design pattern comportamentale, definisce una famiglia di algoritmi, rendendoli interscambiabili.}

\elemento{Struttura dati}
{Entità usata per organizzare un insieme di dati all'interno della memoria del computer.}

\elemento{Stub}
{Porzione di \gloss{codice} utilizzata in sostituzione di altre funzionalità \gloss{software}. Uno stub può simulare il comportamento di codice esistente, gli stub sono utili durante lo sviluppo di software e durante i test per i software.}

\elemento{Sun Microsystems}
{Azienda produttrice di \gloss{software}, nota per aver prodotto il \gloss{linguaggio di programmazione} \gloss{Java}.}

\elemento{Sviluppatore}
{\gloss{Programmatore} che si prende cura di uno o più aspetti del ciclo di vita del \gloss{software}. Questa figura può contribuire alla visione d'insieme del progetto ad un livello applicativo piuttosto che a livello di componenti o operazioni individuali di \gloss{programmazione} (la codifica).}


\paginaGlossario{T}


\elemento{Template}
{Documento nel quale è rappresentata una struttura generica o standard dove ci sono spazi temporaneamente bianchi da riempire in seguito. In italiano viene indicato come scheletro o modello di base.}

\elemento{Test}
{Indica una prova, ovvero una serie di operazioni effettuate su un prodotto per trovare malfunzionamenti o errori e correggerli, prima del rilascio finale del prodotto.}

\elemento{Testing}
{Attività di fare test.}

\elemento{Tex}
{\gloss{Programma} di tipografia digitale adatto alla stesura di testi matematici e scientifici.}

\elemento{TexMaker}
{\gloss{Editor} gratuito multi piattaforma per la scrittura di documenti in \gloss{Latex}.}

\elemento{Ticket}
{}

\elemento{Tipo-informatica}
{Nome che indica l'insieme di valori che una \gloss{variabile}, o il risultato di un'espressione, possono assumere e le operazioni che su tali valori si possono effettuare.}

\elemento{Top-down}
{Approccio che consente di formulare una visione generale del problema senza andare nel dettaglio. Si comincia a decomporre il problema iniziale in sottoproblemi, fino ad arrivare a pezzi non scomponibili.}

\elemento{Tracciamento}

\elemento{Two Way Data-Binding}

\paginaGlossario{U}

\elemento{UML}

\elemento{Upload}
{Caricamento, ovvero il processo di trasmissione di un file da un \gloss{client} ad un \gloss{server}.}

\elemento{Upper bound}
{Tradotto letteralmente con "limite superiore". In informatica, si definisce upper bound il limite superiore di memoria occupata da una struttura dati, di solito un \gloss{array}.}

\elemento{URL}
{Acronimo di "Uniform Resource Locator". Sequenza di caratteri che identifica univocamente l'indirizzo di una risorsa in \gloss{internet}, tipicamente presente su un \gloss{host} \gloss{server}, come ad esempio un documento, un'immagine, un video, rendendola accessibile ad un \gloss{client} che ne faccia richiesta attraverso l'utilizzo di un \gloss{web} \gloss{browser}.}

\elemento{User}
{Vedi \gloss{utente}.}

\elemento{Username}
{Tradotto in "nome \gloss{utente}". \`{E} il nome fornito da un utente di un \gloss{servizio} informatico (solitamente \gloss{web}) per identificarsi e accedere così al dato servizio.}

\elemento{Utente}
{Colui che può usufruire di un \gloss{servizio} che gli viene messo a disposizione.}

\elemento{Utente business}
{Persona alla quale interessano le \gloss{pagine web} create dal \gloss{framework} \gloss{Maap}; esso può consultare o operare su queste pagine in base a degli specifici scopi o interessi.}

\elemento{Utente business autenticato}
{\gloss{Utente business} che ha effettuato con successo la procedura di \gloss{autenticazione} verso il sistema che contiene le pagine generate da \gloss{MaaP}.}

\elemento{Utente business autenticato amministratore}
{\gloss{Utente business autenticato} che ha funzionalità aggiuntive rispetto a degli \gloss{utenti business autenticati}.}

\elemento{Utente sviluppatore}
{Persona che utilizza il \gloss{framework} \gloss{Maap}.}


\paginaGlossario{V}


\elemento{Variabile (informatica)}
{Insieme di dati modificabili situati in una porzione di memoria (una o più locazioni di memoria) destinata a contenere dei dati. Una variabile è caratterizzata da un nome (inteso solitamente come una sequenza di caratteri e cifre) e da un \gloss{tipo}.}

\elemento{Versionamento}
{Gestione di \gloss{versioni} multiple di un insieme di informazioni.}

\elemento{Versione}
{Identificativo univoco che rappresenta la variante di un documento o di un componente \gloss{software}.}

\elemento{View}
{Componente del design pattern MVC; gestisce la logica di presentazione verso i vari utenti e cattura gli input dell'utente delegando ad un altro componente, il controller, l'elaborazione.}

\elemento{ViewModel}

\elemento{VoIP}
{Acronimo di "Voice over IP", tradotto in "Voce tramite protocollo IP". Con VoIP si intende una tecnologia che rende possibile effettuare una conversazione telefonica sfruttando una connessione Internet o una qualsiasi altra rete dedicata a commutazione di pacchetto che utilizzi il protocollo IP senza connessione per il trasporto dati.}


\paginaGlossario{W}


\elemento{Web}
{Abbreviazione di "World Wide Web". Nel linguaggio comune è associato ad \gloss{"internet"}, ovvero è una \gloss{rete} di computer ad accesso pubblico, è il maggior sistema di comunicazione di massa che offre anche ulteriori servizi, come ad esempio condividere e cercare risorse.}

\elemento{Web-service}

\elemento{Widget}
{Componente grafico di un'interfaccia utente di un programma che ha lo scopo di facilitare all'utente l'interazione con il programma stesso.}

\elemento{World Wide Web}
{Vedi "\gloss{Web}".}


%%FINE GLOSSARIO, NON CANCELLARE
\end{document}
