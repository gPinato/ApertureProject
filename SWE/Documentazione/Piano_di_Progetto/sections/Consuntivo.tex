\section{Consuntivo e Preventivo a finire}
Questa sezione contiene il prospetto economico che riporta le spese effettivamente sostenute. Vengono riportate le ore impiegate per svolgere i compiti preventivati. In base alla differenza di ore tra il preventivo ed il consuntivo, detta conguaglio, avremmo un bilancio:
\begin{itemize}
\item \grassetto{Positivo}: se il preventivo ha superato il consuntivo;
\item \grassetto{Negativo}: se il consuntivo ha superato il preventivo;
\item \grassetto{In pari}: se preventivo e consuntivo coincidono.
\end{itemize}
\subsection{Analisi}
Di seguito è riportato il consuntivo del periodo di tempo dedicato all'Analisi dei Requisiti.\\
La tabella sottostante riporta la differenza delle ore tra preventivo e consuntivo, divise per ruolo.

\immagine{./tabelle/differenza_ruoli}{Costo per ruolo, fase di Pre RR}

Nella tabella seguente sono riportate le differenze tra le ore di lavoro previste per ogni componente con quelle realmente impiegate.

\immagine{./tabelle/differenza_preventivo_consuntivo}{Differenza preventivo consuntivo per componente, fase di Pre RR}

\subsubsection{Conclusioni}
In conclusione, come si può notare dai valori presenti nelle precedenti tabelle, è stata impiegata un'ora in più per svolgere le attività programmate con un bilancio in negativo di euro 35. Questa somma non andrà ad intaccare il costo totale del progetto, in quanto le ore spese in questo periodo non sono rendicontate e quindi non sono a carico del Proponente.

\subsection{Analisi in Dettaglio}
Di seguito è riportato il consuntivo del periodo di tempo dedicato all' Analisi in Dettaglio.\\
La tabella sottostante riporta la differenza delle ore tra preventivo e consuntivo, divise per ruolo.

\immagine{./tabelle/differenza_ruoli_analisidettaglio}{Costo per ruolo, fase di Post RR}

Nella tabella seguente sono riportate le differenze tra le ore di lavoro previste per ogni componente con quelle realmente impiegate.

\immagine{./tabelle/differenza_preventivo_consuntivo_analisidettaglio}{Differenza preventivo consuntivo per componente, fase di Post RR}

\subsubsection{Conclusioni}
L'Analisi dei Requisiti è stata incrementata dettagliandola maggiormente.\\
In conclusione, come si può notare dai valori presenti nelle precedenti tabelle, sono state impiegate due ore in più per svolgere le attività di Analisi programmate, con un bilancio in negativo di euro 50. Questa somma non andrà ad intaccare il costo totale del progetto, in quanto le ore spese in questo periodo non sono rendicontate e quindi non sono a carico del Proponente.

\subsection{Progettazione Architetturale}
Di seguito è riportato il consuntivo del periodo di tempo dedicato alla Progettazione Architetturale.\\
La tabella sottostante riporta la differenza delle ore tra preventivo e consuntivo, divise per ruolo.

\immagine{./tabelle/differenza_ruoli_progettaz}{Costo per ruolo, fase di Pre RP}

Nella tabella seguente sono riportate le differenze tra le ore di lavoro previste per ogni componente con quelle realmente impiegate.

\immagine{./tabelle/differenza_preventivo_consuntivo_progettaz}{Differenza preventivo consuntivo per componente, fase di Pre RP}

\subsubsection{Conclusioni}
Si è deciso di spostare la data per la revisione di progettazione dal 2014-03-18 al 2014-03-29, perchè, il tempo preventivato dedicato allo studio personale di tecnologie e di conoscenze specifiche per la progettazione, è stato molto sottostimato rispetto al tempo realmente occorso. Il tutto ha portato allo slittamento della consegna, ma il costo a budget speso per il tempo dedicato allo studio personale non è a carico del Proponente.
In conclusione, come si può notare dai valori presenti nelle precedenti tabelle, sono state tolte delle ore di Amministratore e di Analista, mentre sono state richieste delle ore di Verificatore in più rispetto a quanto pianificato; questa modifica ha portato ad un bilancio in positivo di 63 euro.
\subsection{Preventivo a finire}
Come si può notare da quanto scritto in precedenza, c'è stato un attivo pari a 63 euro; questa somma verrà sicuramente utilizzata per aumentare le ore di Verifica nel periodo di tempo che porta alla successiva revisione.\\
Si è deciso di investire questa somma nell'attività di Verifica nella fase di Pre RQ, così da migliorare la qualità dei prodotti in uscita. 
