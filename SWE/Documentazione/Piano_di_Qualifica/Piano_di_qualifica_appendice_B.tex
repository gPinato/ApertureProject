\newpage

\section{Metodologie di verifica e tecniche di analisi}
\subsection{Organizzazione} %2.1
\label{2.4}
Per ogni processo attuato ci sono delle attività di Verifica e per ogni processo realizzato viene verificata la qualità del processo stesso e la qualità dell'eventuale prodotto ottenuto da esso.
Ogni periodo di tempo antecedente la consegna di revisione descritto nel Piano di Progetto, necessita di attività di Verifica:
\begin{itemize}
\item \grassetto{Analisi:} si seguono i metodi di Verifica descritti nelle Norme di Progetto sui documenti prodotti e i processi attuati. La messa in opera di tali tecniche è descritta nell'appendice sezione C.3.1;
\item\grassetto{Analisi di Dettaglio:} si verificano i processi che determinano l'incremento dei documenti redatti per il precedente periodo di Analisi, e si verificano i prodotti generati dai relativi processi, seguendo le Norme di Progetto. La messa in opera di tale attività è descritta nell'appendice sezione C.3.2;
\item\grassetto{Progettazione Architetturale:}  si verificano i processi che determinano l'incremento dei documenti redatti per il precedente periodo di Analisi in Dettaglio e si verificano i prodotti generati dai relativi processi; inoltre si verificano processi e prodotti per l'attività di Progettazione Architetturale, seguendo le Norme di Progetto. La messa in opera di tale attività è descritta nell'appendice sezione C.3.3.
\item \grassetto{Progettazione di Dettaglio e Codifica}:
\end{itemize}
Questa sezione conterrà i riferimenti alle successive attività di Verifica, aggiornati dopo ogni fase di produzione e di \gloss{test} del prodotto.
La stesura dei documenti è l'attività principale e costante nello svolgimento del progetto, mentre il processo di Verifica viene diviso in due attività.
In ogni documento è presente un diario delle modifiche per mantenere una cronologia delle attività svolte e di chi le ha svolte.

\subsection{Pianificazione strategica e temporale} %2.1
\label{2.5}
Avendo scadenze prefissate nel Piano di Progetto, dobbiamo garantire che le attività di Verifica di tutti i documenti e prodotti debbano essere sistematiche, disciplinate e quantificabili. Procedendo in questa maniera si correggono gli errori il prima possibile.
La metodologia da seguire per l'individuazione e correzione degli errori è descritta nelle Norme di Progetto.
Ogni attività di redazione di documenti e di scrittura del \gloss{codice} è stata preceduta da uno studio iniziale sull'impaginazione dei documenti e del contenuto degli stessi. Questo serve per minimizzare la possibilità di incorrere in errori di tipo concettuale e tecnico.

\subsection{Responsabilità} %2.1
\label{2.6}
Per garantire che il processo di Verifica sia disciplinato, sistematico e quantificabile, bisogna attribuire responsabilità a specifici ruoli di progetto. I ruoli sono Responsabile di Progetto e Verificatore. I compiti di ciascun ruolo sono descritti nelle \emph{Norme\_di\_progetto\_v\versioneNormeDiProgetto{}.pdf}, rispettivamente nelle sezioni 4.1 e 4.5.

\subsection{Risorse} %2.1
\label{2.7}
Per raggiungere gli obiettivi di qualità prefissati sono necessarie risorse umane e tecnologiche, suddivise rispettivamente in strumenti software e \gloss{hardware} utilizzati dai componenti del gruppo per effettuare Verifica su processi e prodotti. I ruoli maggiormente coinvolti nella responsabilità delle attività di Verifica e Validazione sono il Responsabile di Progetto e il Verificatore e i rispettivi compiti sono descritti dettagliatamente nelle Norme di Progetto. Per facilitare il lavoro dei Verificatori sono stati usati degli strumenti automatici che eseguono controlli sistematici sui prodotti generati. Questi strumenti sono descritti nelle Norme di Progetto.

\subsection{Tecniche di analisi} %3.0
Di seguito verranno elencate le tecniche di analisi che il gruppo adotterà durante lo svolgimento del progetto.

\subsubsection{Analisi statica}
\label{3.1}
Questa tipologia di analisi può essere applicata sia al codice che alla documentazione, dato che prevede l'utilizzo di tecniche generali per ogni tipo di prodotto del team.\\
Le tecniche di analisi statica sono:
\begin{itemize}
\item \grassetto{Walkthrough};
\item \grassetto{Inspection};
\end{itemize}

\paragraph{Walkthrough}
Questa tecnica utilizza una scansione ampia e non mirata dell'oggetto in verifica, data la mancanza di esperienza \gloss{best practice} del Verificatore.
L'attuazione di questa tecnica di Analisi è quindi molto onerosa, per questo sarà nostro obiettivo renderla più parallelizzabile possibile, così da ridurre i costi di Verifica e per essere più efficace ed efficiente.
Si comincia con una attività preliminare di lettura, seguita da una individuazione degli errori; poi si procede con la correzione degli stessi e con una successiva attività di lettura per controllare le modifiche apportate.
Dopo ogni attività di Verifica tramite Walkthrough, sperabilmente avremo trovato la maggior parte degli errori, fornendoci una visione delle erroneità commesse, di conseguenza potremo raffinare l'Analisi e avvicinarci alla metodologia di Inspection.

\paragraph{Inspection}
Questa tecnica è un'evoluzione del Walkthrough e applica una ricerca più mirata e specifica. È possibile utilizzare questa tecnica dopo aver acquisito dimestichezza con l'attività di Verifica, stilando una lista di controllo contenente i maggiori errori riscontrati applicando la tecnica di Walkthrough, quindi non sarà possibile utilizzarla fin da subito in quanto la lista inizialmente è vuota. \`{E} obiettivo di una fase di Inspection la ricerca mirata di errori, aumentando l'efficienza della Verifica e riducendo i costi in termini di tempo e risorse.
Durante l'utilizzo della tecnica di Inspection la lista verrà aggiornata; la lista risultante è in appendice A delle \emph{Norme\_di\_progetto\_v\versioneNormeDiProgetto{}.pdf}.

\subsubsection{Analisi dinamica}
\label{3.2}
Questa particolare tipologia di Analisi si applica solamente ai prodotti software sviluppati dal team, mediante l'utilizzo di test progettati e scritti appositamente per verificare la correttezza dei prodotti e la loro effettiva validazione.
Di seguito analizzeremo i 5 tipi di test attuati nelle varie parti del progetto:
\begin{itemize}
\item \grassetto{Test di unità}: attività di Verifica svolta su ogni singola unità software del sistema, mediante l'utilizzo di \gloss{stub}, \gloss{driver} e \gloss{logger}. Un'unità è la più piccola parte di lavoro che viene assegnata individualmente al \gloss{programmatore}, successivamente sarà prodotta e verificata singolarmente. Mediante tali test viene verificato il corretto funzionamento dei moduli di cui il sistema è composto, in modo da cancellare eventuali errori di implementazione commessi dai programmatori;
\item \grassetto{Test di integrazione}: attività di Verifica che controlla la corretta integrazione di più unità software aggiunte in maniera incrementale, il cui scopo è analizzare che la combinazione delle unità software funzioni come attesa. Grazie a questo test si possono rilevare errori non riscontrabili nei test di unità e comportamenti inaspettati di componenti software già esistenti rilasciati da altri fornitori che non interagiscono correttamente tra di loro. Anche in questa attività, per poter simulare le unità nell'integrazione, vengono create unità fittizie specifiche, come stub e driver, in modo da replicare componenti non ancora sviluppate in modo da non falsare i test;
\item \grassetto{Test di sistema}: questo test si propone di validare il prodotto, una volta che si è stabilita  la sua \gloss{versione} definitiva. Questo test verifica che la copertura dei requisiti obbligatori decisi nel periodo di tempo dedicato all'Analisi in Dettaglio sia completa;
\item \grassetto{Test di regressione}: attività di Verifica che consiste nel ripetere i test già effettuati su una \gloss{componente}, ogni qualvolta quella componente venga modificata o aggiornata; un aiuto lo fornisce il \gloss{tracciamento} delle componenti che permette di scovare e ripetere in modo semplificato i test di unità, di regressione e possibilmente quelli di sistema che sono stati potenzialmente alterati dalla modifica;
\item \grassetto{Test di accettazione}: test finale effettuato dal proponente del software, al cui superamento segue il rilascio del prodotto ultimato.
\end{itemize}
L'importanza di un test si attua nella sua automazione, in quanto riduce il tempo dedicato alla Verifica manuale del codice, certamente più onerosa. Per questo motivo la proprietà più importante di un test è la sua ripetibilità.
Per fare in modo che un test abbia questa qualità, è fondamentale definire a priori certe caratteristiche, ovvero:
\begin{itemize}
\item \grassetto{\gloss{Ambiente}}: deve essere specificato l'insieme di componenti hardware e software su cui verrà eseguito il prodotto software, al fine di evitare problemi di incompatibilità e malfunzionamento;
\item \grassetto{Variabili}: si deve conoscere e garantire la corretta struttura delle variabili in ingresso ai test, in modo da prevedere gli \gloss{output} attesi e verificare la loro correttezza;
\item \grassetto{Procedure}: deve essere chiara la sequenza delle operazioni e la metodologia di applicazione dei test.
\end{itemize}
\section{Pianificazione dei test}
Di seguito verranno visualizzate delle tabelle, strutturate secondo la sezione 5.4.2 delle  \emph{Norme\_di\_progetto\_v\versioneNormeDiProgetto{}.pdf}, che riportano tutti i test che si sono pianificati. \\


\subsection{Test di unit\'{a}}
Di seguito verrà mostrata la tabella che riporta tutti i test di unità pianificati, associati alla componente testata.\\

\subsubsection{Descrizione dei test di unit\'{a}}

%QUESTE TABELLE SONO STATE GENERATE MANUALMENTE DA JACK
Sono qui riportate le tabelle per i testi di unit\'{a}  riguardanti il client.
Le funzionalit\'{a} dettagliate sono descritte nella Definizione di Prodotto 4.2.0.
\begin{center}
\begin{longtable}{|p{2cm}|p{7cm}|p{2cm}|}
\toprule
\multicolumn{1}{|p{2cm}}{\textbf{Test}}
& \multicolumn{1}{|p{7cm}}{\textbf{Descrizione}}
& \multicolumn{1}{|p{2cm}|}{\textbf{Stato}}\\
\midrule
\endfirsthead
\multicolumn{2}{l}{\footnotesize\itshape\tablename~\thetable: continua dalla pagina precedente} \\
\toprule
\multicolumn{1}{|p{2cm}}{\textbf{Test}}
& \multicolumn{1}{|p{7cm}}{\textbf{Descrizione}}
& \multicolumn{1}{|p{2cm}|}{\textbf{Stato}}\\
\midrule
\endhead
\midrule
\multicolumn{2}{r}{\footnotesize\itshape\tablename~\thetable: continua nella prossima pagina} \\
\endfoot
\bottomrule
\caption{Test di unit\'{a}}
\endlastfoot

\midrule
TU1
& Vengono verificate tutte le funzionalità rese disponibili dal Collection Controller.
& D.E.\\


\midrule
TU2
& Vengono verificate tutte le funzionalità rese disponibili dal Dashoboard Controller.
& D.E.\\


\midrule
TU3
& Vengono verificate tutte le funzionalità rese disponibili dal Document Controller.
& D.E.\\


\midrule
TU4
& Vengono verificate tutte le funzionalità rese disponibili dal DocumentEdit Controller.
& D.E.\\


\midrule
TU5
& Vengono verificate tutte le funzionalità rese disponibili dal Index Controller.
& D.E.\\



\midrule
TU6
& Vengono verificate tutte le funzionalità rese disponibili dal Login Controller.
& D.E.\\


\midrule
TU7
& Vengono verificate tutte le funzionalità rese disponibili dal NavBar Controller.
& D.E.\\


\midrule
TU8
& Vengono verificate tutte le funzionalità rese disponibili dal Profile Controller.
& D.E.\\

\midrule
TU9
& Vengono verificate tutte le funzionalità rese disponibili dal ProfileEdit Controller.
& D.E.\\


\midrule
TU10
& Vengono verificate tutte le funzionalità rese disponibili dal Query Controller.
& D.E.\\


\midrule
TU11
& Vengono verificate tutte le funzionalità rese disponibili dal Register Controller.
& D.E.\\


\midrule
TU12
& Vengono verificate tutte le funzionalità rese disponibili dal UserCollection Controller.
& D.E.\\

\midrule
TU13
& Vengono verificate tutte le funzionalità rese disponibili dal UserController Controller.
& D.E.\\

\midrule
TU14
& Vengono verificate tutte le funzionalità rese disponibili dal UserEdit Controller.
& D.E.\\



\end{longtable}
\end{center}


Test di unità per il client.
\begin{center}
\begin{longtable}{|p{2cm}|p{7cm}|p{2cm}|}
\toprule
\multicolumn{1}{|p{2cm}}{\textbf{Test}}
& \multicolumn{1}{|p{7cm}}{\textbf{Descrizione}}
& \multicolumn{1}{|p{2cm}|}{\textbf{Stato}}\\
\midrule
\endfirsthead
\multicolumn{2}{l}{\footnotesize\itshape\tablename~\thetable: continua dalla pagina precedente} \\
\toprule
\multicolumn{1}{|p{2cm}}{\textbf{Test}}
& \multicolumn{1}{|p{7cm}}{\textbf{Descrizione}}
& \multicolumn{1}{|p{2cm}|}{\textbf{Stato}}\\
\midrule
\endhead
\midrule
\multicolumn{2}{r}{\footnotesize\itshape\tablename~\thetable: continua nella prossima pagina} \\
\endfoot
\bottomrule
\caption{Test di unit\'{a}}
\endlastfoot

\midrule
TU1
& Vengono verificate tutte le funzionalità rese disponibili dal servizio AuthService.
& D.E.\\


\midrule
TU2
& Vengono verificate tutte le funzionalità rese disponibili dal servizio CollectionDataService.
& D.E.\\


\midrule
TU3
& Vengono verificate tutte le funzionalità rese disponibili dal servizio CollectionListService.
& D.E.\\


\midrule
TU4
& Vengono verificate tutte le funzionalità rese disponibili dal servizio DocumentDataService.
& D.E.\\


\midrule
TU5
& Vengono verificate tutte le funzionalità rese disponibili dal servizio DocumentEditService.
& D.E.\\



\midrule
TU6
& Vengono verificate tutte le funzionalità rese disponibili dal dal servizio IndexService.
& D.E.\\


\midrule
TU7
& Vengono verificate tutte le funzionalità rese disponibili dal servizio LogoutService.
& D.E.\\


\midrule
TU8
& Vengono verificate tutte le funzionalità rese disponibili dal servizio ProfileDataService.
& D.E.\\

\midrule
TU9
& Vengono verificate tutte le funzionalità rese disponibili dal servizio ProfileEditService.
& D.E.\\


\midrule
TU10
& Vengono verificate tutte le funzionalità rese disponibili dal servizio QueryService.
& D.E.\\


\midrule
TU11
& Vengono verificate tutte le funzionalità rese disponibili dal servizio RegisterService.
& D.E.\\


\midrule
TU12
& Vengono verificate tutte le funzionalità rese disponibili dal servizio UserCollectionService.
& D.E.\\

\midrule
TU13
& Vengono verificate tutte le funzionalità rese disponibili dal servizio UserDataService.
& D.E.\\

\midrule
TU14
& Vengono verificate tutte le funzionalità rese disponibili dal servizio UserEditService.
& D.E.\\



\end{longtable}
\end{center}




\subsection{Test di sistema}
Di seguito verrà mostrata la tabella che riporta tutti i test di sistema pianificati, associati ai requisiti descritti nel documento Analisi dei Requisiti.\\
I test sono da intendere solo per requisiti ai quali è stato ragionevole associare un test.

\subsubsection{Descrizione dei test di sistema}
%la seguente tabella è generata automaticamente dallo scriptTest che
%prende i dati esportati dal database di Access e genera la tabella in latex
\begin{center}
\begin{longtable}{|p{2cm}|p{7cm}|p{2cm}|p{2cm}|}
\toprule
\textbf{Test} & \textbf{Descrizione} & \textbf{\gloss{Requisito}} & \textbf{Stato}\\
\midrule
TS1 & Viene verificato che il sistema \gloss{MaaP} generi correttamente lo scheletro necessario & ROF1 & D.E.\\
\midrule
TS1.1 & Viene verificato che il sistema MaaP installi correttamente le librerie necessarie & ROF1.1 & D.E.\\
\midrule
TS1.2 & Viene verificato che il sistema MaaP generi correttamente i file necessari & ROF1.2 & D.E.\\
\midrule
TS1.3 & Viene verificato che il sistema MaaP generi correttamente le directory necessarie & ROF1.3 & D.E.\\
\midrule
TS1.4 & Viene verificato che il \gloss{sottosistema} di autenticazione sia installato e configurato correttamente & ROF1.4 & D.E.\\
\midrule
TS1.4.1 & Viene verificato che nel \gloss{database} degli utenti sia presente un \gloss{profilo} di amministrazione di \gloss{default} & ROF1.4.1 ROF6 & D.E.\\
\midrule
TS1.5 & Viene verificato che il sistema sia in grado di eliminare un progetto esistente & ROF1.5 & D.E.\\
\midrule
TS1.6 & Viene verificato che il sistema sia in grado di clonare un progetto esistente & ROF1.6 & D.E.\\
\midrule
TS4 & Viene verificato che il sistema crei correttamente le \gloss{pagine web} partendo dal loro \gloss{file di descrizione} & ROF4 & D.E.\\
\midrule
TS5.1 & Viene verificato che la funzione di \gloss{registrazione} possa essere correttamente abilitata/disabilitata & RDF5.1 & D.E.\\
\midrule
TS5.4 & Viene verificato che il sistema possa utilizzare correttamente il database di analisi & ROF5.4 & D.E.\\
\midrule
TS5.5 & Viene verificato che la funzione di creazione indici possa essere correttamnte abilitata/disabilitata & ROF5.5 & D.E.\\
\midrule
TS7 & Viene verificato che il sistema consenta all'utente registrato di potersi autenticare & ROF 7.0 & D.E.\\
\midrule
TS8 & Viene verificato che il sistema consenta all'utente di potersi registrare & RDF 8.0 & D.E.\\
\midrule
TS9 & Viene verificato che il sistema consenta all'utente di recuperare la \gloss{password}  & ROF 9.0 & D.E\\
\midrule
TS10 & Viene verificato che il sistema apra e visualizzi correttamente le \gloss{Collection} e le \gloss{Collection-Index} & ROF10 & D.E.\\
\midrule
TS10.1 & Viene verificato che il sistema visualizzi correttamente le pagine \gloss{Document-Show} & ROF10.1 & D.E.\\
\midrule
TS10.2.4 & Viene verificato che il sistema disconnetta correttamente un utente alla sua richiesta & ROF10.2.4 & D.E.\\
\midrule
TS10.3.1.1 & Viene verificato che un utente autenticato possa modificare i dati del suo profilo & ROF10.3.1.1 & D.E.\\
\midrule
TS10.3.1.2 & Viene verificato che le modifiche apportate al profilo di un \gloss{utente business} autenticato siano consistenti & ROF10.3.1.2 & D.E.\\
\midrule
TS10.3.2 & Viene verificato che la creazione di un nuovo utente da parte di un \gloss{utente business autenticato} \gloss{amministratore} avvenga correttamente & ROF10.3.2 & D.E.\\
\midrule
TS10.3.3 & Viene verificata la corretta cancellazione di un utente da parte di un \gloss{utente business autenticato amministratore} & ROF10.3.3 & D.E.\\
\midrule
TS10.4 & Viene verificato che l'utente business autenticato amministratore possa eliminare correttamente un \gloss{Document} & ROF10.4 & D.E.\\
\midrule
TS10.5 & Viene verificato che l’utente business autenticato amministratore possa modificare correttamente un Document & ROF10.5 & D.E.\\
\midrule
TS10.6 & Viene verificata la corretta visualizzazione delle \gloss{query} più utilizzate & ROF10.6 & D.E.\\
\midrule
TS10.7.1.2 & Viene verificata la corretta creazione degli indici di analisi & ROF10.7.1.2 & D.E.\\
\midrule
TS10.7.2 & Viene verificata la corretta eliminazione degli indici di analisi & ROF10.7.2 & D.E.\\
\midrule
TS10.7.3 & Viene verificata la corretta visualizzazione degli indici di analisi & ROF10.7.3 & D.E.\\
\midrule
TS17 & Viene verificato che le pagine web prodotte dal \gloss{framework} MaaP siano compatibili con la versione 30.0.x o superiore di \gloss{Google Chrome} & ROV17 & D.E.\\
\midrule
TS18 & Viene verificato che le pagine web prodotte dal framework MaaP siano compatibili con la versione 24.x o superiore di \gloss{Firefox} & ROV18 & D.E.\\
\midrule
TS19 & Viene verificato che il sistema accetti solo \gloss{file di configurazione} validi & ROV19 & D.E.\\
\midrule
TS26 & Viene verificato che il sistema di installazione del software funzioni correttamente & ROV26 & D.E.\\
\midrule
TS27 & Viene verificato che il deployment su \gloss{Heroku} avvenga con successo & ROV27 & D.E.\\
%inserire i test
\bottomrule
\caption{Test di sistema}
\label{tab:changelog}
\end{longtable}
\end{center}

\subsection{Test d'integrazione}

\subsubsection{Diagramma d'integrazione}
\immagine{./integrazione}{Diagramma d'integrazione}
Di seguito verrà mostrata la tabella che riporta tutti i test d'integrazione pianificati, associati alle componenti descritte nella progettazione ad alto livello.\\

\subsubsection{Descrizione dei test d'integrazione}
%la seguente tabella è generata automaticamente dallo scriptTest che
%prende i dati esportati dal database di Access e genera la tabella in latex
%QUESTE TABELLE SONO STATE GENERATE AUTOMATICAMENTE DA TESTSCRIPT [Mon Apr 14 23:39:48 2014
]

\begin{center}
\begin{longtable}{|p{5cm}|p{5cm}|p{5cm}|p{1cm}|}
\toprule
\textbf{Test} & \textbf{Descrizione} & \textbf{Componente} & \textbf{Stato}\\

\midrule
TI.MaaP
& Test di integrazione finale tra client e server
client e server
& MaaP
& D.E.\\


\midrule
TI.Server
& Verifica il corretto funzionamento delle componenti del server, quindi la corretta integrazione tra ModelServer e Controller del server. In particolare, viene verificato che le funzionalita
verificato che le funzionalita presenti nel Controller debbano agire consistentemente sui dati del
ModelServer.
& MaaP::Server
& D.E.\\


\midrule
TI.Server.ModelServer
& Verifica che tutte le operazioni di lettura, scrittura e interpretazione dei dati avvengano correttamente
correttamente.
& MaaP::Server::ModelServer
& D.E.\\


\midrule
TI.Server.ModelServer .DataManager-Database
& Verifica che i gestori dei dati siano correttamente integrati ai database di riferimento
siano correttamente integrati ai
database di riferimento.
& MaaP::Server::ModelServer ::DataManager
& D.E.\\


\midrule
TI.Server.ModelServer .DataManager-Database
& Verifica che i gestori dei dati siano correttamente integrati ai database di riferimento
siano correttamente integrati ai
database di riferimento.
& MaaP::Server::ModelServer ::Database
& D.E.\\


\midrule
TI.Server.ModelServer .DataManager
& Verifica che le operazioni di recupero, gestione e scrittura dati nei database avvengano correttamente.
recupero, gestione e scrittura
dati nei database avvengano
correttamente.
& MaaP::Server::ModelServer ::DataManager
& D.E.\\


\midrule
TI.Client
& Verifica che i dati ottenuti e forniti da e verso il client siano corretti e correttamente gestiti all'interno del client stesso
corretti e correttamente gestiti
all'interno del client stesso.
& MaaP::Client
& D.E.\\


\midrule
TI.Client.View-ControllerModelView
& Verifica che i dati uscenti ed entranti la View siano gestiti correttamente dal ControllerModelView
& MaaP::Client::View
& D.E.\\


\midrule
TI.Client.View-ControllerModelView
& Verifica che i dati uscenti ed entranti la View siano gestiti correttamente dal ControllerModelView
& MaaP::Client::ControllerModelView
& D.E.\\


\midrule
TI.Client.ControllerModelView-ModelClient
& Viene verificato che le funzionalita del ControllerModelView agiscano correttamente sul ModelClient
ModelClient
& MaaP::Client::ControllerModelView
& D.E.\\


\midrule
TI.Client.ControllerModelView
& Viene verificato che le funzioni del ClientController modifichino correttamente i dati presenti nello Scope
nello Scope
& MaaP::Client::ModelClient
& D.E.\\


\midrule
TI.Client.ControllerModelView
& Viene verificato che le funzioni del ClientController modifichino correttamente i dati presenti nello Scope
nello Scope
& MaaP::Client::ControllerModelView
& D.E.\\


\bottomrule
\caption{Test d'integrazione}
\end{longtable}
\end{center}
%QUESTE TABELLE SONO STATE GENERATE AUTOMATICAMENTE DA TESTSCRIPT [Mon Apr 14 23:39:48 2014
]



\subsubsection{Tracciamento}
Di seguito verranno riportati in forma tabellare i tracciamenti componente-test d'integrazione e test d'integrazione-componente.\\

\paragraph{Tracciamento componente-test d'integrazione}
%la seguente tabella è generata automaticamente dallo scriptTest che
%prende i dati esportati dal database di Access e genera la tabella in latex
\begin{center}
\begin{longtable}{|p{7cm}|p{7cm}|}
\toprule
\textbf{Componente} & \textbf{Test}\\
\midrule
MaaP & TI.MaaP\\
\midrule
Server & TI.Server\\
\midrule
Server::ModelServer & TI.Server.ModelServer\\
\midrule
Server::ModelServer::DataManager & TI.Server.ModelServer.DataManager TI.Server.ModelServer.DataManager-Database\\
\midrule
Server::ModelServer::Database & TI.Server.ModelServer.DataManager-Database\\
\midrule
Client & TI.Client\\
\midrule
Client::View & TI.Client.View-ControllerModelView\\
\midrule
Client::ControllerModelView & TI.Client.ControllerModelView TI.Client.View-ControllerModelView TI.Client.ControllerModelView-ModelClient\\
\midrule
Client::ModelClient & TI.Client.ControllerModelView-ModelClient\\
%inserire i test
\bottomrule
\caption{Tracciamento componente-test d'integrazione}
\label{tab:changelog}
\end{longtable}
\end{center}

\vspace{5cm}

\paragraph{Tracciamento test d'integrazione-componente}
%la seguente tabella è generata automaticamente dallo scriptTest che
%prende i dati esportati dal database di Access e genera la tabella in latex
\begin{center}
\begin{longtable}{|p{7cm}|p{7cm}|}
\toprule
\textbf{Test} & \textbf{Componente}\\
\midrule
TI.MaaP & MaaP\\
\midrule
TI.Server & Server\\
\midrule
TI.Server.ModelServer & Server::ModelServer\\
\midrule
TI.Server.ModelServer.DataManager-Database & Server::ModelServer::DataManager\\
& Server::ModelServer::Database\\
\midrule
TI.Server.ModelServer.DataManager & Server::ModelServer::DataManager\\
\midrule
TI.Client & Client\\
\midrule
TI.Client.View-ControllerModelView & Client::View\\
& Client::ControllerModelView\\
\midrule
TI.Client.ControllerModelView-ModelClient & Client::ControllerModelView\\
& Client::ModelClient\\
\midrule
TI.Client.ControllerModelView & Client::ControllerModelView\\
%inserire i test
\bottomrule
\caption{Tracciamento test d'integrazione-componente}
\label{tab:changelog}
\end{longtable}
\end{center}

\subsection{Test di validazione}
In questa sezione vengono descritti i test di validazione, utili ad accertarsi che il prodotto realizzato sia conforme alle attese.
Per ogni test viene descritta la serie di passi che un utente deve eseguire per validare la conformità ai requisiti ad esso associati.

\subsubsection{TV 1 - Lato Utente Business}

L'utente vuole verificare le funzionalità di visualizzazione di Collection e Document, per tanto gli è richiesto di:


\begin{enumerate}
\item Cliccare sul pulsante Collection;
\item Selezionare la Collection da visualizzare dall'apposito elenco;
\begin{enumerate}
\item Scorrere le pagine della Collection fino a trovare il Document ricercato;
\item Usare un \gloss{filtro} per ricercare un Document;
\item Annullare il filtro applicato;
\item Ordinare una \gloss{chiave}, se abilitata, in ordine crescente o decrescente;
\item Impostare un numero massimo di Document da visualizzare.
\end{enumerate}
\item Selezionare il Document ricercato cliccando sulla chiave selezionabile;
\item Verificare che il Document venga visualizzato correttamente.
\end{enumerate}

\subsubsection{TV 2 - Lato Utente Business}
L'utente vuole verificare il corretto funzionamento delle funzionalità di registrazione, autenticazione e gestione profilo.
Si dovrà quindi seguire la seguente \gloss{procedura} di validazione:

\begin{enumerate}
\item Verificare di potersi registrare;
\begin{enumerate}
\item Cliccare sul pulsante di registrazione;
\item Compilare il \gloss{form} di registrazione con dati validi;
\item Confermare la registrazione mediante il \gloss{link} contenuto nella mail inviata automaticamente dal sistema.
\end{enumerate}
\item Verificare di poter recuperare la password in caso di dimenticanza;
\begin{enumerate}
\item Cliccare sul pulsante di recupero password;
\item Inserire la \gloss{email} con cui si è registrati al sistema;
\item Verificare che il sistema invii correttamente la mail per il recupero della password;
\item Verificare che la nuova password fornita nella mail consenta l'accesso al sistema.
\end{enumerate}
\item Verificare di poter effettuare l'autenticazione;
\begin{enumerate}
\item Cliccare sul pulsante di autenticazione;
\item Inserire le credenziali precedentemente registrate;
\item Confermare l'autenticazione.
\end{enumerate}
\item Verificare che il sistema effettui correttamente l' autenticazione;
\item Verificare di poter modificare i propri dati utente;
\begin{enumerate}
\item Entrare nel proprio profilo utente;
\item Cliccare sul pulsante di modifica;
\item Verificare di poter modificare una o più chiavi del profilo;
\item Verificare di poter salvare le modifiche;
\item Verificare di poter annullare le modifiche.
\end{enumerate}
\item Verificare di potersi disconnettere correttamente dal sistema;
\begin{enumerate}
\item Cliccare sul pulsante di disconnessione;
\item Verificare di non poter più accedere a pagine riservate ad utenti autenticati.
\end{enumerate}
\end{enumerate}

\subsubsection{TV 3 - Lato Utente Business Amministratore}

L'utente amministratore vuole verificare il corretto funzionamento della gestione dei profili utenti.
Si dovrà quindi seguire la seguente procedura:

\begin{enumerate}
\item Effettuare l'autenticazione con le proprie credenziali;
\begin{enumerate}
\item Al primo accesso assoluto al sistema, utilizzare le credenziali di default fornite dal sistema stesso.
\end{enumerate}
\item Cliccare sul pulsante di gestione profili utenti;
\item Selezionare il profilo utente da modificare;
\item Verificare di poter modificare le informazioni del profilo utente;
\begin{enumerate}
\item Cliccare sul pulsante di modifica;
\item Verificare di poter modificare una o più chiavi del profilo;
\item Verificare di poter salvare le modifiche;
\item Verificare di poter annullare le modifiche.
\end{enumerate}
\item Verificare di poter modificare i permessi associati al profilo utente;
\begin{enumerate}
\item Cliccare sul pulsante di modifica;
\item Verificare di poter modificare i permessi disponibili;
\item Verificare di poter salvare le modifiche;
\item Verificare di poter annullare le modifiche.
\end{enumerate}
\item Verificare di poter eliminare correttamente un utente;
\begin{enumerate}
\item Entrare nel profilo utente da eliminare;
\item Cliccare sul pulsante di eliminazione;
\item Confermare l'eliminazione;
\item Verificare che le credenziali dell'utente cancellato non siano più presenti nel sistema.
\end{enumerate}
\item Verificare di poter creare un nuovo profilo utente;
\begin{enumerate}
\item Cliccare sul pulsante di creazione;
\item Completare il form di registrazione con informazioni valide;
\item Verificare che il profilo sia stato creato correttamente.
\end{enumerate}
\end{enumerate}


\subsubsection{TV 4 - Lato Utente Business Amministratore}

L’utente amministratore vuole verificare il corretto funzionamento della gestione dei Document.
Dovrà quindi seguire la seguente procedura:


\begin{enumerate}
\item Dopo essere entrato in una pagina Collection, selezionare un Document;
\item Verificare di poter modificare un Document;
\begin{enumerate}
\item Cliccare sul pulsante di modifica;
\item Verificare di poter modificare le chiavi disponibili;
\item Verificare di poter salvare le modifiche;
\item Verificare di poter annullare le modifiche.
\end{enumerate}
\item Verificare di poter cancellare un Document;
\begin{enumerate}
\item Selezionare il pulsante di cancellazione;
\item Verificare che il Document non sia più presente nel sistema.
\end{enumerate}
\end{enumerate}

\subsubsection{TV 5 - Lato Utente Business Amministratore}

L’utente amministratore vuole verificare il corretto funzionamento della gestione degli indici.
Dovrà quindi seguire la seguente procedura:

\begin{enumerate}
\item Entrare nella gestione degli indici;
\item Verificare di poter creare un indice;
\begin{enumerate}
\item Cliccare sul pulsante di creazione indici;
\item Selezionare una tra le query più utilizzate;
\item Confermare la creazione;
\item Verificar che l'indice sia stato creato correttamente.
\end{enumerate}
\item Verificare di poter eliminare un indice;
\begin{enumerate}
\item Selezionare un indice presente;
\item Cliccare sul pulsante di eliminazione;
\item Verificare che l'indice non sia più presente.
\end{enumerate}
\end{enumerate}

\subsubsection{TV 6 - Lato Utente Sviluppatore}

L'utente sviluppatore vuole verificare il corretto funzionamento della gestione ed interpretazione del \gloss{DSL} per la creazione di una pagina Collection-Index.
Dovrà quindi seguire la seguente procedura:

\begin{enumerate}
\item Scrivere un file DSL valido;
\begin{enumerate}
\item Specificare le chiavi da visualizzare;
\item Specificare la provenienza dei dati;
\begin{enumerate}
\item Provenienti da query standard;
\item Provenienti da query personalizzate;
\item Provenienti da funzioni personalizzate;
\item Provenienti da riferimenti esterni.
\end{enumerate}
\item Specificare eventuali etichette;
\item Specificare eventuali chiavi ordinabili;
\item Specificare il numero massimo di Document per pagina;
\item Specificare il nome della Collection;
\item Specificare la posizione della Collection nel menù.
\end{enumerate}
\item Caricare il file DSL nell'apposita cartella;
\item Avviare il server MaaP;
\item Effettuare l'autenticazione;
\item Entrare nella Collection la cui pagina è descritta dal file DSL inserito e verificare la bontà della sua interpretazione.
\end{enumerate}


\subsubsection{TV 7 - Lato Utente Sviluppatore}

L’utente sviluppatore vuole verificare il corretto funzionamento della gestione ed interpretazione del DSL per la creazione di una pagina Document-Show.
Dovrà quindi seguire la seguente procedura:

\begin{enumerate}
\item Scrivere un file DSL valido;
\begin{enumerate}
\item Specificare le chiavi da visualizzare;
\item Specificare la provenienza dei dati;
\begin{enumerate}
\item Provenienti da query standard;
\item Provenienti da query personalizzate;
\item Provenienti da funzioni personalizzate;
\item Provenienti da riferimenti esterni.
\end{enumerate}
\item Specificare eventuali etichette per le chiavi;
\end{enumerate}
\item Caricare il file DSL nell'apposita cartella;
\item Avviare il server MaaP;
\item Effettuare l'autenticazione;
\item Entrare nel document la cui pagina è descritta dal file DSL inserito e verificare la bontà della sua interpretazione.
\end{enumerate}


\subsubsection{TV 8 - Lato Utente Sviluppatore}

L'utente sviluppatore vuole verificare il corretto funzionamento delle funzioni di gestione del progetto.
Dovrà quindi seguire la seguente procedura:

\begin{enumerate}
\item Creare un nuovo progetto;
\item Verificare che il nuovo progetto sia disponibile;
\item Clonare un progetto;
\item Verificare che il progetto clonato sia identico all'originale;
\item Verificare di poter gestire un progetto;
\begin{enumerate}
\item Verificare di poter avviare, fermare o riavviare il server MaaP;
\item Verificare di poter modificare i file di configurazione del progetto;
\begin{enumerate}
\item Abilitare/disabilitare la registrazione;
\item Abilitare/disabilitare la creazione di indici;
\item Impostare la connessione al database di analisi.
\end{enumerate}
\item Verificare di poter modificare o caricare nuovi file DSL;
\item Verificare di poter modificare i \gloss{template} disponibili.
\end{enumerate}
\item Eliminare un progetto;
\item Verificare che il progetto eliminato non sia più disponibile.
\end{enumerate}

\subsubsection{Tracciamento}
Di seguito verranno riportati in forma tabellare i tracciamenti test di validazione - requisiti e l'inverso, ovvero requisiti - test di validazione.\\

\paragraph{Tracciamento Test di Validazione - Requisiti}
%la seguente tabella è generata automaticamente dallo scriptTest che
%prende i dati esportati dal database di Access e genera la tabella in latex
\begin{center}
\begin{longtable}{|p{7cm}|p{7cm}|}
\toprule
\textbf{Test} & \textbf{Requisiti}\\
\midrule
TV 1 & ROF10\\ & ROF10.1\\ & RDF10.2\\ & RDF10.2.1\\ &  RDF10.2.1.1\\ &  RDF10.2.1.2\\ &  RDF10.2.2 \\ & RDF10.2.3\\ &   ROF10.2.5\\
\midrule
TV 2 & ROF7\\ &  ROF7.1\\ &  ROF7.2\\ &  ROF7.2.1\\ &  RDF8\\ &  RDF8.1\\ &  RDF8.2\\ &  RDF8.2.1\\ &  ROF9\\ &  ROF10.2.4\\ &  ROF10.3\\ &  ROF10.3.1\\ &  ROF10.3.1.1\\ &  ROF10.3.1.2\\ &  ROF10.3.1.3\\
\midrule
TV 3 & ROF6\\ &  ROF10.3.1.4\\ &  ROF10.3.2\\ &  ROF10.3.3\\
\midrule
TV 4 &  ROF10.4\\ &  ROF10.5\\ &  ROF10.5.1\\ &  ROF10.5.2\\ &  ROF10.5.3\\
\midrule
TV 5 &  ROF10.6\\ &  ROF10.7\\ &  ROF10.7.1\\ &  ROF10.7.1.1\\ &  ROF10.7.1.2\\ &  ROF10.7.2\\ &  ROF10.7.2.1\\ &  ROF10.7.2.2\\ &  ROF10.7.3\\
\midrule
TV 6 & ROF3\\ &  ROF4\\ &  ROF4.1\\ &  ROF4.1.2\\ &  ROF4.1.2.1\\ &  ROF4.1.2.1.1\\ &  ROF4.1.2.1.2\\ &  ROF4.1.2.1.3\\ &  ROF4.1.2.1.4\\ &  ROF4.1.2.1.5\\ &  ROF4.1.2.2\\ &  ROF4.1.2.3\\ &  ROF4.2\\ &  ROF4.2.1\\ &  ROF4.2.1.1\\ &  ROF4.2.1.2\\ &  ROF4.2.2\\ &  ROF4.2.2.1\\ &  ROF4.2.2.2\\ &  ROF4.2.2.3\\ &  ROF4.2.2.4\\ &  ROF4.2.2.5\\ &  ROF4.2.2.6\\ &  RFF4.2.2.7\\ &  ROF4.3\\ &  ROF4.4\\
\midrule
TV 7 & ROF4.1.3\\ &  ROF4.1.3.1\\ &  ROF4.1.3.1.1\\ &  ROF4.1.3.1.2\\ &  ROF4.1.3.1.3\\ &  ROF4.1.3.1.4\\ &  ROF4.1.3.1.5\\ &  ROF4.2.3\\ &  ROF4.2.3.1\\ &  ROF4.2.3.2\\ &  ROF4.3\\ &  ROF4.4\\
\midrule
TV 8 & ROF5\\ &  RDF5.1\\ &  RDF5.3\\ &  ROF5.4\\ &  ROF5.5\\

%inserire i test
\bottomrule
\caption{Tracciamento Test di Validazione - Requisiti}
\label{tab:changelog}
\end{longtable}
\end{center}

\paragraph{Tracciamento Requisiti - Test di Validazione}
%la seguente tabella è generata automaticamente dallo scriptTest che
%prende i dati esportati dal database di Access e genera la tabella in latex
%QUESTE TABELLE SONO STATE GENERATE AUTOMATICAMENTE DA TESTSCRIPT [Apr 15 12:17:32 2014]

\begin{center}
\begin{longtable}{|p{7cm}|p{7cm}|}
\toprule
\multicolumn{1}{|p{7cm}}{\textbf{Requisiti}}
& \multicolumn{1}{|p{7cm}|}{\textbf{Test}} \\
\midrule
\endfirsthead
\multicolumn{2}{l}{\footnotesize\itshape\tablename~\thetable: continua dalla pagina precedente} \\
\toprule
\multicolumn{1}{|p{7cm}}{\textbf{Requisiti}}
& \multicolumn{1}{|p{7cm}|}{\textbf{Test}} \\
\midrule
\endhead
\midrule
\multicolumn{2}{r}{\footnotesize\itshape\tablename~\thetable: continua nella prossima pagina} \\
\endfoot
\bottomrule
\caption{Tracciamento Requisiti - Test di validazione}
\endlastfoot


\midrule
RDF10.2.3
& TV1\\

\midrule
ROF10
& TV1\\

\midrule
ROF10.2.5
& TV1\\

\midrule
RDF10.2.2
& TV1\\

\midrule
RDF10.2.1.2
& TV1\\

\midrule
RDF10.2.1.1
& TV1\\

\midrule
RDF10.2.1
& TV1\\

\midrule
RDF10.2
& TV1\\

\midrule
ROF10.1
& TV1\\

\midrule
ROF10.3.1
& TV2\\

\midrule
RDF8.2.1
& TV2\\

\midrule
ROF10.3.1.1
& TV2\\

\midrule
ROF10.3
& TV2\\

\midrule
ROF10.2.4
& TV2\\

\midrule
ROF9
& TV2\\

\midrule
ROF10.3.1.3
& TV2\\

\midrule
RDF8.1
& TV2\\

\midrule
RDF8
& TV2\\

\midrule
ROF7.2.1
& TV2\\

\midrule
ROF7.2
& TV2\\

\midrule
ROF7.1
& TV2\\

\midrule
ROF7
& TV2\\

\midrule
RDF8.2
& TV2\\

\midrule
ROF10.3.1.2
& TV2\\

\midrule
ROF6
& TV3\\

\midrule
ROF10.3.1.4
& TV3\\

\midrule
ROF10.3.2
& TV3\\

\midrule
ROF10.3.3
& TV3\\

\midrule
ROF10.4
& TV4\\

\midrule
ROF10.5.3
& TV4\\

\midrule
ROF10.5.2
& TV4\\

\midrule
ROF10.5
& TV4\\

\midrule
ROF10.5.1
& TV4\\

\midrule
ROF10.7.3
& TV5\\

\midrule
ROF10.6
& TV5\\

\midrule
ROF10.7
& TV5\\

\midrule
ROF10.7.1
& TV5\\

\midrule
ROF10.7.1.1
& TV5\\

\midrule
ROF10.7.1.2
& TV5\\

\midrule
ROF10.7.2
& TV5\\

\midrule
ROF10.7.2.1
& TV5\\

\midrule
ROF10.7.2.2
& TV5\\

\midrule
ROF4.2.2.4
& TV6\\

\midrule
ROF4.4
& TV6\\

\midrule
ROF4.3
& TV6\\

\midrule
RFF4.2.2.7
& TV6\\

\midrule
ROF3
& TV6\\

\midrule
ROF4.2.2.5
& TV6\\

\midrule
ROF4
& TV6\\

\midrule
ROF4.2.2.3
& TV6\\

\midrule
ROF4.2.2.2
& TV6\\

\midrule
ROF4.2.2.1
& TV6\\

\midrule
ROF4.2.2
& TV6\\

\midrule
ROF4.2.1.2
& TV6\\

\midrule
ROF4.2.1.1
& TV6\\

\midrule
ROF4.1.2.1.1
& TV6\\

\midrule
ROF4.2.2.6
& TV6\\

\midrule
ROF4.2.1
& TV6\\

\midrule
ROF4.1
& TV6\\

\midrule
ROF4.1.2.1
& TV6\\

\midrule
ROF4.1.2.1.2
& TV6\\

\midrule
ROF4.1.2.1.3
& TV6\\

\midrule
ROF4.1.2.1.4
& TV6\\

\midrule
ROF4.1.2.1.5
& TV6\\

\midrule
ROF4.1.2.2
& TV6\\

\midrule
ROF4.1.2.3
& TV6\\

\midrule
ROF4.2
& TV6\\

\midrule
ROF4.1.2
& TV6\\

\midrule
ROF4.1.3.1
& TV7\\

\midrule
ROF4.2.3
& TV7\\

\midrule
ROF4.4
& TV7\\

\midrule
ROF4.3
& TV7\\

\midrule
ROF4.2.3.2
& TV7\\

\midrule
ROF4.2.3.1
& TV7\\

\midrule
ROF4.1.3.1.5
& TV7\\

\midrule
ROF4.1.3.1.4
& TV7\\

\midrule
ROF4.1.3.1.3
& TV7\\

\midrule
ROF4.1.3.1.1
& TV7\\

\midrule
ROF4.1.3
& TV7\\

\midrule
ROF4.1.3.1.2
& TV7\\

\midrule
ROF5.5
& TV8\\

\midrule
ROF5
& TV8\\

\midrule
RDF5.1
& TV8\\

\midrule
RDF5.3
& TV8\\

\midrule
ROF5.4
& TV8\\

\end{longtable}
\end{center}
%QUESTE TABELLE SONO STATE GENERATE AUTOMATICAMENTE DA TESTSCRIPT [Apr 15 12:17:32 2014]



