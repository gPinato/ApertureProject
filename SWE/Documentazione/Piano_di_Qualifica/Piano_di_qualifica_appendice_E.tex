\newpage

\section{Resoconto delle attività di verifica}

\subsection{Resoconto PDCA}

Durante lo sviluppo del progetto si è sempre applicato il ciclo di Deming per cercare di migliorare la qualità dei processi. Questa attività ha portato a molteplici miglioramenti.
Tra i più significativi si riportano:
\begin{itemize}


\item Migliore politica di assegnazione delle attività:

All'inizio della progettazione, le attività principali e maggiormente propedeutiche erano state assegnate a gruppi di quattro persone, con l'obiettivo di colmare velocemente il monte ore/persona e ridurre i tempi di completamento. Tuttavia si è notato come il lavoro procedesse più lentamente di quanto atteso e fossero presenti dei ritardi rispetto alle consegne. Nelle seguenti pianificazioni si è ridotto il numero di persone assegnate ad una singola attività cercando nel contempo di massimizzare il parallelismo. Nonostante questo abbia leggermente aumentato il tempo di completamento delle singole attività, i benefici portati dal lavoro concorrente di team più piccoli hanno largamente sorpassato gli svantaggi e comportato un maggior avanzamento complessivo del progetto.

\item Migliore sviluppo dei documenti:

All'inizio i documenti erano sviluppati utilizzando \gloss{editor} di testo come OpenOffice e in seguito, una volta completati, trasformati in \LaTeX\ con una formattazione appropriata.
In seguito si è visto come lo sviluppo dei documenti direttamente in \LaTeX, nonostante sia più oneroso in termini di tempo durante la stesura, sia comunque più rapido rispetto al primo metodo e, soprattutto, privo degli errori di trasposizione, riducendo quindi le ore/persona necessarie e gli errori contenuti nei documenti.

\end{itemize}


\subsection{Riassunto delle attività di verifica}
In questa sezione sono descritti i resoconti delle attività di verifica effettuate sui documenti prima di ciascuna revisione.

\subsubsection{Revisione dei Requisiti}
Nel periodo precedente a questa revisione i documenti sono stati controllati dai Verificatori seguendo le \emph{Norme\_di\_progetto\_v\versioneNormeDiProgetto{}.pdf}; è stata applicata l'analisi statica descritta nella sezione 2.8.1 di questo documento.
Inizialmente è stata applicata la tecnica di Walkthrough, dove sono scovati e successivamente corretti gli errori; ogni volta che si trovava un errore, esso veniva messo nell'apposta lista che serve per l'Inspection.
Dopo il Walkthrough è stata applicata la tecnica di Inspection, utilizzando l'apposita lista, disponibile in appendice delle Norme di Progetto. Inoltre per questo documento sono state calcolate le metriche descritte nella sezione 2.9.2 del documento corrente.
Per quanto riguarda i processi, essi sono stati controllati e verificati secondo le metodologie descritte nelle  \emph{Norme\_di\_progetto\_v\versioneNormeDiProgetto{}.pdf}. Sono state calcolate le metriche per i processi descritti in sezione 2.9.1 di questo documento, e riportati i corrispondenti valori di BV e SV in forma tabellare.

\subsubsection{Revisione di Progettazione}
Nel periodo precedente a questa revisione i documenti sono stati controllati dai verificatori seguendo le \emph{Norme\_di\_progetto\_v\versioneNormeDiProgetto{}.pdf}; è stata applicata l'analisi statica descritta nella sezione 2.8.1 di questo documento.
Inizialmente è stata applicata la tecnica di Walkthrough, dove sono scovati e successivamente corretti gli errori; ogni volta che si trovava un errore, esso veniva messo nell'apposta lista che serve per l'Inspection.
Dopo il Walkthrough è stata applicata la tecnica di Inspection, utilizzando l'apposita lista, disponibile in appendice delle Norme di Progetto; è stata posta particolare attenzione al documento Specifica Tecnica. Inoltre per questo documento sono state calcolate le metriche descritte nella sezione 2.9.2 del documento corrente.
Per quanto riguarda i processi, essi sono stati controllati e verificati secondo le metodologie descritte nelle \emph{Norme\_di\_progetto\_v\versioneNormeDiProgetto{}.pdf}. Sono state calcolate le metriche per i processi descritti in sezione 2.9.1 di questo documento, e riportati i corrispondenti valori di BV e SV in forma tabellare.
\subsubsection{Revisione di Qualifica}
Nel periodo precedente a questa revisione i documenti sono stati controllati dai verificatori seguendo le \emph{Norme\_di\_progetto\_v\versioneNormeDiProgetto{}.pdf}; è stata applicata l'analisi statica descritta nella sezione 2.8.1 di questo documento. Inizialmente è stata applicata la tecnica di Walkthrough, dove sono scovati e successivamente corretti gli errori; ogni volta che si trovava un errore, esso veniva messo nell'apposta lista che serve per l'Inspection. Dopo il Walkthrough è stata applicata la tecnica di Inspection, utilizzando l'apposita lista, disponibile in appendice delle Norme di Progetto. E' stata posta particolare attenzione al documento Specifica Tecnica. Inoltre per questo documento sono state calcolate le metriche descritte nella sezione 2.9.2 del documento corrente. Per quanto riguarda i processi, essi sono stati controllati e verificati secondo le metodologie descritte nelle\emph{Norme\_di\_progetto\_v\versioneNormeDiProgetto{}.pdf}. Sono state calcolate le metriche per i processi descritti in
sezione 2.9.1 di questo documento, e riportati i corrispondenti valori di BV e SV in forma tabellare.

\subsection{Dettaglio delle verifiche tramite analisi}
\subsubsection{Analisi dei Requisiti}
\paragraph{Processi}
Di seguito vengono riportati i valori degli indici SV e BV calcolati durante il periodo di tempo dedicato all'Analisi dei Requisiti.
\begin{longtable}{|c|p{3cm}|p{3cm}|}
\toprule
\textbf{Attività} & \textbf{SV} & \textbf{BV} \\

%aggiungere qui una midrule per aggiungere una nuova riga alla tabella

\midrule
\emph{Studio Fattibilità} & 0 & 0 \\
\midrule
\emph{Analisi dei Requisiti} & +50 & +50\\
\midrule
\emph{Glossario} & 0  & 0\\
\midrule
\emph{Norme di Progetto} & 0 & 0\\
\midrule
\emph{Piano di Progetto} & 0 & 0\\
\midrule
\emph{Piano di Qualifica} & -15 & -15\\
\bottomrule
\caption{BV e SV calcolati sui documenti durante l'Analisi}
\label{tab:changelog}
\end{longtable}

\paragraph{Conclusioni}
In questa tabella, i valori positivi indicano un costo risparmiato, viceversa i valori negativi mostrano un costo eccedente.
I valori indicati in tabella sono espressi in euro.
Non avendo previsto degli intervalli di tempo libero tra un'attività e la successiva, abbiamo ottenuto degli SV positivi in Analisi dei Requisiti e negativi in Piano di Qualifica.
Questa è stata una mancanza da parte del team, che vedrà di migliorarsi nelle prossime fasi e di adottare una tattica di pianificazione più flessibile.
I costi aggiuntivi sono comunque in linea con i nostri obiettivi.

\paragraph{Documenti}
Di seguito vengono riportati, per ogni documento, i valori dell'indice di Gulpease calcolati durante il periodo di tempo dedicato all'Analisi dei Requisiti. Un documento è valido solo se rispecchia i range in sezione 2.9.2.1.
\begin{longtable}{|c|p{3cm}|p{3cm}|}
\toprule
\textbf{Documento} & \textbf{Valore indice} & \textbf{Esito} \\

%aggiungere qui una midrule per aggiungere una nuova riga alla tabella

\midrule
\emph{Studio Fattibilità v1.2.0} & 46 & Sufficiente\\
\midrule
\emph{Analisi dei Requisiti v1.2.0} & 52 & Superato\\
\midrule
\emph{Glossario v1.2.0} & 46 & Sufficiente\\
\midrule
\emph{Norme di Progetto v1.2.0} & 52 & Superato\\
\midrule
\emph{Piano di Progetto v1.2.0} & 50 & Superato\\
\midrule
\emph{Piano di Qualifica v1.2.0} & 47 & Sufficiente\\
\bottomrule
\caption{Esiti dell'indice di Gulpease calcolato sui documenti durante l'Analisi}
\label{tab:changelog}
\end{longtable}

\subsubsection{Analisi in Dettaglio}
\paragraph{Processi}
Di seguito vengono riportati i valori degli indici SV e BV calcolati durante il periodo di tempo dedicato all'Analisi in Dettaglio.
\begin{longtable}{|c|p{3cm}|p{3cm}|}
\toprule
\textbf{Attività} & \textbf{SV} & \textbf{BV} \\

%aggiungere qui una midrule per aggiungere una nuova riga alla tabella

\midrule
\emph{Studio Fattibilità} & 0 & 0 \\
\midrule
\emph{Analisi dei Requisiti} & 0 & -50\\
\midrule
\emph{Glossario} & 0  & 0\\
\midrule
\emph{Norme di Progetto} & 0 & 0\\
\midrule
\emph{Piano di Progetto} & 0 & 0\\
\midrule
\emph{Piano di Qualifica} & 0 & 0\\
\bottomrule
\caption{BV e SV calcolati sui documenti durante l'Analisi in Dettaglio}
\label{tab:changelog}
\end{longtable}

\paragraph{Conclusioni}
Come si può notare dalla tabella, il BV è negativo, in quanto non sono state pianificate alcune attività correttive, ed è stato messo a budget il costo necessario per effettuare queste attività non previste.\\
Lo SV invece è pari a zero, in quanto l'ampio slack di tempo pianificato è servito a coprire le correzioni non previste e di conseguenza non è stato prodotto niente di più rispetto a quanto pianificato.
\paragraph{Documenti}
Di seguito vengono riportati, per ogni documento, i valori dell'indice di Gulpease calcolati durante il periodo di tempo dedicato all'Analisi in Dettaglio.

\begin{longtable}{|c|p{3cm}|p{3cm}|}
\toprule
\textbf{Documento} & \textbf{Valore indice} & \textbf{Esito} \\

%aggiungere qui una midrule per aggiungere una nuova riga alla tabella

\midrule
\emph{Analisi dei Requisiti v2.2.0} & 55 & Superato \\
\midrule
\emph{Glossario v2.2.0} & 56 & Sufficiente\\
\midrule
\emph{Norme di Progetto v2.2.0} & 52 & Superato\\
\midrule
\emph{Piano di Progetto v2.2.0} & 48 & Sufficiente \\
\midrule
\emph{Piano di Qualifica v2.2.0} & 47 & Sufficiente \\
\bottomrule
\caption{Esiti dell'indice di Gulpease calcolato sui documenti durante l'Analisi in Dettaglio}
\label{tab:changelog}
\end{longtable}

\subsubsection{Progettazione Architetturale}
\paragraph{Processi}
Di seguito vengono riportati i valori degli indici SV e BV calcolati durante il periodo di tempo dedicato alla Progettazione Architetturale.
\begin{longtable}{|c|p{3cm}|p{3cm}|}
\toprule
\textbf{Attività} & \textbf{SV} & \textbf{BV} \\

%aggiungere qui una midrule per aggiungere una nuova riga alla tabella

\midrule
\emph{Analisi dei Requisiti} & 0 & +35\\
\midrule
\emph{Glossario} & 0  & +45\\
\midrule
\emph{Norme di Progetto} & 20 & +45\\
\midrule
\emph{Piano di Progetto} & 0 & +45 \\
\midrule
\emph{Piano di Qualifica} & 0 & -20\\
\midrule
\emph{Specifica Tecnica} & 0 & -87\\
\bottomrule
\caption{BV e SV calcolati sui documenti durante la Progettazione Architetturale}
\label{tab:changelog}
\end{longtable}

\paragraph{Conclusioni}
Lo SV è positivo, in quanto lo slack dedicato al documento Norme di Progetto ha \gloss{permesso} l'aggiunta di valore non pianificato, come l'aggiunta di sezioni.\\
Il BV è positivo, e nonostante il fatto che si è dedicato più tempo alla progettazione, e quindi dedicandoci più budget; a causa di questo si è riuscito a risparmiare budget per le attività dedicate agli altri documenti, dedicando maggior budget per la Verifica della progettazione, che nella pianificazione non era adeguato, e togliendone da altre attività.

\paragraph{Documenti}
Di seguito vengono riportati, per ogni documento, i valori dell'indice di Gulpease calcolati durante il periodo di tempo dedicato alla Progettazione Architetturale.

\begin{longtable}{|c|p{3cm}|p{3cm}|}
\toprule
\textbf{Documento} & \textbf{Valore indice} & \textbf{Esito} \\

%aggiungere qui una midrule per aggiungere una nuova riga alla tabella

\midrule
\emph{Analisi dei Requisiti v3.2.0} & 58 & Superato \\
\midrule
\emph{Glossario v3.2.0} & 56 & Superato \\
\midrule
\emph{Norme di Progetto v3.2.0} & 53  & Superato\\
\midrule
\emph{Piano di Progetto v3.2.0} & 49  & Sufficiente\\
\midrule
\emph{Piano di Qualifica v3.2.0} & 48  & Sufficiente\\
\midrule
\emph{Specifica Tecnica v3.2.0} & 44 & Sufficiente\\
\bottomrule
\caption{Esiti dell'indice di Gulpease calcolato sui documenti durante la Progettazione}
\label{tab:changelog}
\end{longtable}

\paragraph{Progettazione}
Viene qui riportata una tabella riassuntiva che riporta il calcolo dei parametri di accoppiamento afferente ed efferente per i componenti individuati nella progettazione architetturale.

\begin{longtable}{|p{11cm}|c|c|}
\toprule
\textbf{Componente} & \textbf{Afferente} & \textbf{Efferente} \\

\midrule
MaaP::Server
& 1 & 0\\

\midrule
MaaP::Server::ModelServer
& 3 & 0\\

\midrule
MaaP::Server::ModelServer::DataManager
& 1 & 4\\

\midrule
MaaP::Server::ModelServer::DataManager::DatabaseAnalysisManager
& 1 & 7\\

\midrule
MaaP::Server::ModelServer::DataManager::DatabaseUserManager
& 1 & 4\\


\midrule
MaaP::Server::ModelServer::DataManager::IndexManager
& 1 & 2\\


\midrule
MaaP::Server::ModelServer::Database
& 4 & 0\\


\midrule
MaaP::Server::ModelServer::DSL
& 1 & 0\\

\midrule
MaaP::Server::Controller
& 1 & 4\\

\midrule
MaaP::Client
& 0 & 1\\

\midrule
MaaP::Client::View
& 0 & 14\\

\midrule
MaaP::Client::ControllerModelView
& 14 & 22\\

\midrule
MaaP::Client::ControllerModelView::ControllerClient
& 14 & 22\\

\midrule
MaaP::Client::ModelClient
& 22 & 1\\

\midrule
MaaP::Client::ModelClient::Services
& 20 & 1\\

\midrule
MaaP::Client::ModelClient::Directives
& 2 & 0\\

\midrule
MaaP::Client::ModelClient::Model
& 0 & 0\\


\bottomrule
\caption{Tabella accoppiamento componenti}
\end{longtable}

Come si può vedere dalla tabella, l'accoppiamento afferente risulta generalmente basso ad eccezione del componente ControllerModelView del package Client e relativo ControllerClient i quali hanno un valore relativamente alto. Questo delinea la criticità del componente in oggetto, che quindi andrà trattato con dovute cautele durante la generazione dei test e la loro esecuzione per ottenere un componente stabile più velocemente, prevenendo il rischio di regressione dovuto ad un alto accoppiamento.\\
Per quanto riguarda l'accoppiamento efferente, anch'esso è relativamente basso ad eccezione dei componenti interni del package MaaP::Server::ModelServer::DataManager e del componente ControllerClient che per la loro natura intrinseca hanno un alto livello di accoppiamento dovendo interagire con diverse classi di package esterni.

\subsubsection{Progettazione di dettaglio e codifica}
\paragraph{Processi}
Di seguito vengono riportati i valori degli indici SV e BV calcolati durante il periodo di tempo dedicato alla Progettazione di dettaglio e codifica.
\begin{longtable}{|c|p{3cm}|p{3cm}|}
\toprule
\textbf{Attività} & \textbf{SV} & \textbf{BV} \\

%aggiungere qui una midrule per aggiungere una nuova riga alla tabella

\midrule
\emph{Analisi dei Requisiti} & 0 & +25 \\
\midrule
\emph{Glossario} & 0 & -15  \\
\midrule
\emph{Norme di Progetto} & 0 & -30 \\
\midrule
\emph{Piano di Progetto} & 0 & +90 \\
\midrule
\emph{Piano di Qualifica} & 0 & +51\\
\midrule
\emph{Specifica Tecnica} & 0 & -1\\
\midrule
\emph{Definizione di Prodotto} & 0 & -241 \\
\midrule
\emph{Manuale utente} & -105 & +105 \\
\midrule
\emph{Codifica} & 0 & +480 \\
\midrule
\emph{Verifica della Codifica} & 0 & +150 \\
\bottomrule
\caption{BV e SV calcolati sui documenti durante la Progettazione di dettaglio e codifica}
\label{tab:changelog}
\end{longtable}

\paragraph{Conclusioni}
Lo SV è negativo in quanto non è stato completato il Manuale Utente come pianificato.
Il BV è invece positivo a causa del fatto che sono state richieste molte meno ore di Codifica per realizzare quanto pianificato, e di conseguenza sono state tolte ore di Verifica del codice per il motivo precedente.
Inoltre il fatto di non aver completato il Manuale Utente alza il BV, a discapito dello SV; il BV della Definizione di Prodotto è negativo perchè sono state aggiunte ore di Verifica non pianificate; nonostante questo, il BV complessivo è positivo.


\paragraph{Documenti}
Di seguito vengono riportati, per ogni documento, i valori dell'indice di Gulpease calcolati durante il periodo di tempo dedicato alla Progettazione di dettaglio e codifica.

\begin{longtable}{|c|p{3cm}|p{3cm}|}
\toprule
\textbf{Documento} & \textbf{Valore indice} & \textbf{Esito} \\

%aggiungere qui una midrule per aggiungere una nuova riga alla tabella

\midrule
\emph{Analisi dei Requisiti v4.2.0} &  &  \\
\midrule
\emph{Glossario v4.2.0} &  &  \\
\midrule
\emph{Norme di Progetto v4.2.0} &   & \\
\midrule
\emph{Piano di Progetto v4.2.0} &   & \\
\midrule
\emph{Piano di Qualifica v4.2.0} &   & \\
\midrule
\emph{Specifica Tecnica v4.2.0} &  & \\
\midrule
\emph{Definizione di Prodotto v4.2.0} &  & \\
\bottomrule
\caption{Esiti dell'indice di Gulpease calcolato sui documenti durante la Progettazione di dettaglio e codifica}
\label{tab:changelog}
\end{longtable}

\subsection{Dettaglio dell'esito delle revisioni}
Per ciascuna revisione alla quale si intende partecipare, il Committente avrà il compito di segnalare eventuali problematiche trovate, dando una valutazione globale dell'andamento del progetto e una descrizione per ciascun documento con correzioni e accorgimenti da apportare.
Di seguito vengono elencate le modifiche apportate ai documenti, come suggerito dal Committente, per ciascuna revisione.
\subsubsection{Revisione dei Requisiti}
\begin{itemize}
\item \grassetto{Studio di Fattibilità:} il documento ha avuto una valutazione positiva, quindi non ci sono stati accorgimenti da apportare;
\item \grassetto{Norme di Progetto:} il documento è stato riorganizzato come suggerito, ovvero per processi, attività procedure e strumenti; è stata migliorata la descrizione della rotazione dei ruoli e il documento è stato incrementato con le parti riguardanti la parte di progettazione;
\item \grassetto{Analisi dei Requisiti:} sono stati corretti degli errori grammaticali, chiariti i significati di alcune parole; i casi d'uso segnalati hanno subito modifiche e aggiustamenti alle pre e post condizioni, mentre altri sono stati descritti più approfonditamente. Sempre dei casi d'uso sono stati tolti o spostati perché in contrasto tra di loro, mentre per quanto riguarda la suddivisione dei requisiti in funzionali, desiderabili, obbligatori ecc.. sono stati rimossi e spostati perché non adatti alla categoria in cui si presentavano. Il documento ha avuto una buona valutazione sulla struttura, quindi non si è cambiata.
\item \grassetto{Piano di Progetto:} come suggerito, alcuni contenuti sono stati spostati nell'Appendice del documento; è stato corretto l'utilizzo della parola fase, e usata solo se strettamente necessario e in contesti che la richiedono. La sezione "Preventivo a finire" è stata corretta in "Consuntivo", in quanto si è capito la differenza tra i significati dei due termini. Si è deciso, per i prossimi intervalli di tempo antecedenti le revisioni, di dedicare più tempo all'attività di Verifica, cercando di raggiungere la soglia del 30\% del tempo totale, come suggerito. Il documento inoltre è stato incrementato con le parti relative alla progettazione;
\item \grassetto{Piano di Qualifica:} il documento ha subito profonde modifiche, è stato ristrutturato e riorganizzato. Per fare ciò, è stata seguita la best practice per la struttura dei documenti presente nel sito del Professor Vardanega; il documento ha subito profonde modifiche anche nei contenuti, inoltre è stato incrementato con le parti relative alla progettazione;
\item \grassetto{Glossario:} il documento ha subito una lieve ristrutturazione, è stato tolto l'indice come suggerito; il documento è stato incrementato con l'inserimento di altri termini.
\end{itemize}

\subsubsection{Revisione di Progettazione}
\begin{itemize}
\item \grassetto{Norme di Progetto:} il documento è stato riorganizzato come suggerito ed è stato incrementato con le parti riguardanti la parte di progettazione in dettaglio e codifica;
\item \grassetto{Analisi dei Requisiti:} sono stati corretti alcuni scenari alternativi e inseriti come estensioni, come da suggerimento, e apportate piccole modifiche. Nel complesso il documento ha raggiunto un buon livello di maturità.
\item \grassetto{Piano di Progetto:} come suggerito, il documento è stato riorganizzato e nella forma e nei contenuti. Inoltre è stato incrementato con le parti relative alla progettazione;
\item \grassetto{Piano di Qualifica:} il documento ha subito profonde modifiche, è stato ristrutturato e riorganizzato.Sono state aggiunte le sezioni riguardanti le strategie e gli obiettivi di qualità.
Il documento inoltre ha subito un incremento con l'aggiunta delle parti relative alla progettazione di dettaglio e codifica;
\item \grassetto{Glossario:} il documento ha subito piccole modifiche, ed è stato incrementato con termini nuovi.
\item \grassetto{Specifica tecnica:} il documento ha subito profonde modifiche; come suggerito, sono stati rivisti e rifatti alcuni diagrammi ed è stato inserito il tracciamento mancante.
\end{itemize}





