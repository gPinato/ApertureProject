%comandi per generare i casi d'uso:

%comando per inserire un item nella lista
\newcommand{\inserisciItem}[1]{\item #1 }

%comando per generare una lista completa, necessita del package xparse
\NewDocumentCommand{\scenario}{>{\SplitList{|}}m}
{
	\textbf{Scenario Principale:}
	\begin{enumerate}
		\ProcessList{#1}{\inserisciItem}
	\end{enumerate}
}

\NewDocumentCommand{\scenarioAlt}{>{\SplitList{|}}m}
{
	\textbf{Scenario Alternativo:}
	\begin{enumerate}
		\ProcessList{#1}{\inserisciItem}
	\end{enumerate}
}

\NewDocumentCommand{\lista}{>{\SplitList{;}}m}
{
	\begin{itemize}
		\ProcessList{#1}{\inserisciItem}
	\end{itemize}
}

\NewDocumentCommand{\inclusioni}{>{\SplitList{|}}m}
{
	\textbf{Inclusioni:}
	\begin{itemize}
		\ProcessList{#1}{\inserisciItem}
	\end{itemize}
}

\NewDocumentCommand{\estensioni}{>{\SplitList{|}}m}
{
	\textbf{Estensioni:}
	\begin{itemize}
		\ProcessList{#1}{\inserisciItem}
	\end{itemize}
}

%\UC{nome}{descrizione}{attori}{scopo e descrizione}{pre}{scenario1;scenario2;...}{post}
\newcommand{\UCtitle}[2]{
\subsection{{#1}: {#2}}
}

\newcommand{\UC}[4]{
\textbf{Diagramma associato:} \ref{#1}\\[\baselineskip]
\textbf{Attori:} {#2};\\[\baselineskip]
\textbf{Scopo e Descrizione:} {#3};\\[\baselineskip]
\textbf{Precondizione:} {#4};\\[\baselineskip]
}

\newcommand{\post}[1]{
\textbf{Postcondizione:} {#1}.
}

%comando per inserire un diagramma
% \UCimmagine{nomeFileSenzaEstensione}{descrizioneDidascalia}
\newcommand{\UCimmagine}[2]{ 
\begin{center}
\begin{figure}[H]
\includegraphics[width=\textwidth]{{{#1}}}
\caption{#2}
\label{#1}
\end{figure}
\end{center}
}