\newpage
\section{Resoconto delle attività di verifica}
\subsection{Tracciamento componenti requisiti}
\subsection{Riassunto delle attività di verifica}
In questa sezione sono descritti i resoconti delle attività di verifica effettuate sui documenti prima di ciascuna revisione.
\subsubsection{Revisione dei Requisiti}
Nel periodo precedente a questa revisione i documenti sono stati controllati dai verificatori seguendo le Norme di Progetto nelle sezione 6.4.1 e 6.4.2; è stata applicata l'analisi statica descritta nella sezione 2.8.1 di questo documento.
Inizialmente è stata applicata la tecnica di Walkthrough, dove sono scovati e successivamente corretti gli errori; ogni volta che si trovava un errore, esso veniva messo nell'apposta lista che serve per l'inspection.
Dopo il walkthrough è stata applicata la tecnica di Inspection, utilizzando l'apposita lista, disponible in appendice delle Norme di Progetto. Inoltre per questo documento sono state calcolate le metriche descritte nella sezione 2.9.2 del documento corrente.
Per quanto riguarda i processi, essi sono stati controllati e verificati secondo le metodologie descritte nelle Norme di Progetto in sezione 5.4.4. Sono state calcolate le metriche per i processi descritti in sezione 2.9.1 di questo documento, e riportati i corrispondenti valori di BV e SV in forma tabellare.
\subsubsection{Revisione di Progettazione}
Nel periodo precedente a questa revisione i documenti sono stati controllati dai verificatori seguendo le Norme di Progetto nelle sezione 6.4.1 e 6.4.2; è stata applicata l'analisi statica descritta nella sezione 2.8.1 di questo documento.
Inizialmente è stata applicata la tecnica di Walkthrough, dove sono scovati e successivamente corretti gli errori; ogni volta che si trovava un errore, esso veniva messo nell'apposta lista che serve per l'inspection.
Dopo il walkthrough è stata applicata la tecnica di Inspection, utilizzando l'apposita lista, disponible in appendice delle Norme di Progetto; è stata posta particolare attenzione al documento Specifica Tecnica. Inoltre per questo documento sono state calcolate le metriche descritte nella sezione 2.9.2 del documento corrente.
Per quanto riguarda i processi, essi sono stati controllati e verificati secondo le metodologie descritte nelle Norme di Progetto in sezione 5.4.4. Sono state calcolate le metriche per i processi descritti in sezione 2.9.1 di questo documento, e riportati i corrispondenti valori di BV e SV in forma tabellare.
\subsection{Dettaglio delle verifiche tramite analisi}
\subsubsection{Analisi dei Requisiti}
\subsubsubsection{Processi}
Di seguito vengono riportati i valori degli indici SV e BV calcolati durante il periodo di tempo dedicato all'Analisi dei Requisiti.
\begin{longtable}{|c|p{3cm}|p{3cm}|}
\toprule
\textbf{Attività} & \textbf{SV} & \textbf{BV} \\

%aggiungere qui una midrule per aggiungere una nuova riga alla tabella

\midrule
\emph{Studio Fattibilità} & 0 & 0 \\
\midrule
\emph{Analisi dei Requisiti} & +50 & +50\\
\midrule
\emph{Glossario} & 0  & 0\\
\midrule
\emph{Norme di Progetto} & 0 & 0\\
\midrule
\emph{Piano di Progetto} & 0 & \\
\midrule
\emph{Piano di Qualifica} & -15 & -15\\
\bottomrule
\caption{BV e SV calcolati sui documenti durante l'Analisi}
\label{tab:changelog}
\end{longtable}
\subsubsubsection{Conclusioni}
In questa tabella, i valori positivi indicano un costo eccedente, viceversa i valori negativi mostrano un costo risparmiato.
I valori indicati in tabella sono espressi in euro.
Non avendo previsto degli intervalli di tempo libero tra un'attività e la successiva, abbiamo ottenuto degli SV positivi in Analisi dei Requisiti e negativi in Piano di Qualifica.
Questa è stata una mancanza da parte del team, che vedrà di migliorarsi nelle prossime fasi e di adottare una tattica di pianificazione più flessibile.
I costi aggiuntivi sono comunque in linea con i nostri obiettivi.
\subsubsubsection{Documenti}
Di seguito vengono riportati, per ogni documento, i valori dell'indice di Gulpease calcolati durante il periodo di tempo dedicato all'Analisi dei Requisiti. Un documento è valido solo se rispecchia i range in sezione 2.9.2.1.
\begin{longtable}{|c|p{3cm}|p{3cm}|}
\toprule
\textbf{Documento} & \textbf{Valore indice} & \textbf{Esito} \\

%aggiungere qui una midrule per aggiungere una nuova riga alla tabella

\midrule
\emph{Studio Fattibilità v1.2.0} & 46 & Sufficiente\\
\midrule
\emph{Analisi dei Requisiti v1.2.0} & 52& Superato\\
\midrule
\emph{Glossario v1.2.0} & 46 & Sufficiente\\
\midrule
\emph{Norme di Progetto v1.2.0} & 52 & Superato\\
\midrule
\emph{Piano di Progetto v1.2.0} & 50 & Superato\\
\midrule
\emph{Piano di Qualifica v1.2.0} & 47 & Sufficiente\\
\bottomrule
\caption{Esiti dell'indice di Gulpease calcolato sui documenti durante l'Analisi}
\label{tab:changelog}
\end{longtable}
\subsubsection{Analisi in Dettaglio}
\subsubsubsection{Processi}
Di seguito vengono riportati i valori degli indici SV e BV calcolati durante il periodo di tempo dedicato all'Analisi in Dettaglio.
\begin{longtable}{|c|p{3cm}|p{3cm}|}
\toprule
\textbf{Attività} & \textbf{SV} & \textbf{BV} \\

%aggiungere qui una midrule per aggiungere una nuova riga alla tabella

\midrule
\emph{Studio Fattibilità} & 0 & 0 \\
\midrule
\emph{Analisi dei Requisiti} & 0 & -50\\
\midrule
\emph{Glossario} & 0  & 0\\
\midrule
\emph{Norme di Progetto} & 0 & 0\\
\midrule
\emph{Piano di Progetto} & 0 & \\
\midrule
\emph{Piano di Qualifica} & 0 & 0\\
\bottomrule
\caption{BV e SV calcolati sui documenti durante l'Analisi in Dettaglio}
\label{tab:changelog}
\end{longtable}
\subsubsubsection{Conclusioni}
Come si può notare dalla tabella, il BV è negativo, in quanto non sono state pianificate alcune attività correttive, ed è stato messo a budget il costo necessario per effettuare queste attività non previste.\\
Lo SV invece è pari a zero, in quanto l'ampio slack di tempo pianificato è servito a coprire le correzioni non previste e di conseguenza non è stato prodotto niente di più rispetto a quanto pianificato. 
\subsubsubsection{Documenti}
Di seguito vengono riportati, per ogni documento, i valori dell'indice di Gulpease calcolati durante il periodo di tempo dedicato all'Analisi in Dettaglio.

\begin{longtable}{|c|p{3cm}|p{3cm}|}
\toprule
\textbf{Documento} & \textbf{Valore indice} & \textbf{Esito} \\

%aggiungere qui una midrule per aggiungere una nuova riga alla tabella

\midrule
\midrule
\emph{Analisi dei Requisiti v2.2.0} & & \\
\midrule
\emph{Glossario v2.2.0} &  &\\
\midrule
\emph{Norme di Progetto v2.2.0} &  & \\
\midrule
\emph{Piano di Progetto v2.2.0} &  & \\
\midrule
\emph{Piano di Qualifica v2.2.0} &  & \\
\bottomrule
\caption{Esiti dell'indice di Gulpease calcolato sui documenti durante l'Analisi in Dettaglio}
\label{tab:changelog}
\end{longtable}
\subsubsection{Progettazione Architetturale}
\subsubsubsection{Processi}
Di seguito vengono riportati i valori degli indici SV e BV calcolati durante il periodo di tempo dedicato alla Progettazione Architetturale.
\begin{longtable}{|c|p{3cm}|p{3cm}|}
\toprule
\textbf{Attività} & \textbf{SV} & \textbf{BV} \\

%aggiungere qui una midrule per aggiungere una nuova riga alla tabella

\midrule
\emph{Analisi dei Requisiti} & 0 & +35\\
\midrule
\emph{Glossario} & 0  & +45\\
\midrule
\emph{Norme di Progetto} & 20 & +45\\
\midrule
\emph{Piano di Progetto} & 0 & +45 \\
\midrule
\emph{Piano di Qualifica} & 0 & -20\\
\midrule
\emph{Specifica Tecnica} & 0 & -87\\
\bottomrule
\caption{BV e SV calcolati sui documenti durante la Progettazione Architetturale}
\label{tab:changelog}
\end{longtable}
\subsubsubsection{Conclusioni}
Lo SV è positivo, in quanto lo slack dedicato al documento Norme di Progetto ha permesso l'aggiunta di valore non pianificato, come l'aggiunta di sezioni.\\
Il BV è positivo, e nonostante il fatto che si è dedicato più tempo alla progettazione, e quindi dedicandoci più budget; a causa di questo si è riuscito a risparmiare budget per le attività dedicate agli altri documenti, dedicando maggior budget per la Verifica della progettazione, che nella pianificazione non era adeguato, e togliendone da altre attività.
\subsubsubsection{Documenti}
Di seguito vengono riportati, per ogni documento, i valori dell'indice di Gulpease calcolati durante il periodo di tempo dedicato alla Progettazione Architetturale.

\begin{longtable}{|c|p{3cm}|p{3cm}|}
\toprule
\textbf{Documento} & \textbf{Valore indice} & \textbf{Esito} \\

%aggiungere qui una midrule per aggiungere una nuova riga alla tabella

\midrule
\midrule
\emph{Analisi dei Requisiti v3.2.0} & & \\
\midrule
\emph{Glossario v3.2.0} &  &\\
\midrule
\emph{Norme di Progetto v3.2.0} &  & \\
\midrule
\emph{Piano di Progetto v3.2.0} &  & \\
\midrule
\emph{Piano di Qualifica v3.2.0} &  & \\
\midrule
\emph{Specifica Tecnica v3.2.0} &  & \\
\bottomrule
\caption{Esiti dell'indice di Gulpease calcolato sui documenti durante la Progettazione}
\label{tab:changelog}
\end{longtable}

\subsection{Dettaglio dell'esito delle revisioni}
Per ciascuna revisione alla quale si intende partecipare, il Committente avrà il compito di segnalare eventuali problematiche trovate, dando una valutazione globale dell'andamento del progetto e una descrizione per ciascun documento con correzioni e accorgimenti da apportare.
Di seguito vengono elencate le modifiche apportate ai documenti, come suggerito dal Committente, per ciascuna revisione.
\subsubsection{Revisione dei Requisiti}
\begin{itemize}
\item \grassetto{Studio di Fattibilità:} il documento ha avuto una valutazione positiva, quindi non ci sono stati accorgimenti da apportare;
\item \grassetto{Norme di Progetto:} il documento è stato riorganizzato come suggerito, ovvero per processi, attività procedure e strumenti; è stata migliorata la descrizione della rotazione dei ruoli e il documento è stato incrementato con le parti riguardanti la parte di progettazione;
\item \grassetto{Analisi dei Requisiti:} sono stati corretti degli errori grammaticali, chiariti i significati di alcune parole; i casi d'uso segnalati hanno subito modifiche e aggiustamenti alle pre e post condizioni, mentre altri sono stati descritti più approfonditamente. Sempre dei casi d'uso sono stati tolti o spostati perchè in contrasto tra di loro, mentre per quanto riguarda la suddivisione dei requisiti in funzionali, desiderabili, obbligatori ecc.. sono stati rimossi e spostati perchè non adatti alla categoria in cui si presentavano. Il documento ha avuto una buona valutazione sulla struttura, quindi non si è cambiata.
\item \grassetto{Piano di Progetto:} come suggerito, alcuni contenuti sono stati spostati nell'Appendice del documento; è stato corretto l'utilizzo della parola fase, e usata solo se strettamente necessario e in contesti che la richiedono. La sezione "Preventivo a finire" è stata corretta in "Consuntivo", in quanto si è capito la differenza tra i significati dei due termini. Si è deciso, per i prossimi intervalli di tempo antecedenti le revisioni, di dedicare più tempo all'attività di Verifica, cercando di raggiungere la soglia del 30\% del tempo totale, come suggerito. Il documento inoltre è stato incrementato con le parti relative alla progettazione;
\item \grassetto{Piano di Qualifica:}il documento ha subito profonde modifiche, è stato ristrutturato e riorganizzato. Per fare ciò, è stata seguita la best practice per la struttura dei documenti presente nel sito del Professor Vardanega; il documento ha subito profonde modifiche anche nei contenuti, inoltre è stato incrementato con le parti relative alla progettazione;
\item \grassetto{Glossario:}il documento ha subito una lieve ristrutturazione, è stato tolto l'indice come suggerito; il documento è stato incrementato con l'inserimento di altri termini.
\end{itemize}






