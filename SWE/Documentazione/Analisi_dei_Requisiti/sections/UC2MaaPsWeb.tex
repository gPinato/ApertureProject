
\UCtitle
{Caso d'uso UC2}
{MaaP's Web}
\UCimmagine{UC2}{UC2 - Azioni utente \gloss{business}}
\UC
{UC2}
{\gloss{Utente business}, \gloss{Utente business autenticato}}
{L'utente business attraverso il \gloss{browser} visualizza la pagina di autenticazione dove può accedere, recuperare la \gloss{password} o registrarsi, mentre l'utente business autenticato può aprire una Collection, disconnettersi o gestire il \gloss{profilo}}
{Il sistema ha creato lo scheletro del progetto, la pagina di autenticazione viene proposta all'utente business mediante il browser}
\scenario{
L'utente business può registrarsi (UC2.1);|
L'utente business può autenticarsi (UC2.2);|
L'utente business può recuperare la password (UC2.3);|
L'utente business autenticato può aprire una Collection (UC2.4);|
L'utente business autenticato può disconnettersi dal sistema (UC2.5);|
L'utente business autenticato può gestire il profilo (UC2.6).
}
\inclusioni
{L'utente business inserisce la propria mail e la propria password (UC2.7).
}
\scenarioAlt{
Nel caso in cui le credenziali di accesso fornite dall'utente business non fossero corrette il sistema segnala il problema e ripresenta il \gloss{form} di autenticazione.
}
\post
{Il sistema ha memorizzato le informazioni inserite dall'utente business per l'autenticazione o per il recupero password e le operazioni che l'utente business autenticato può effettuare}



\UCtitle
{Caso d'uso UC2.1}
{Registrazione}
\UC
{UC2}
{Utente business}
{L'utente business può registrarsi}
{Il sistema si trova nello stato iniziale e propone all'utente una schermata per registrarsi}
\post
{Il sistema ha creato un nuovo utente e reputa l'utente business autenticato}


\UCtitle
{Caso d'uso UC2.2}
{Autenticazione}
\UC
{UC2}
{Utente business}
{Un utente business tenta di autenticarsi}
{Il sistema si trova nello stato iniziale e propone all'utente una schermata per inserire i dati per l'autenticazione}
\post
{Il sistema reputa l'utente business come autenticato}



\UCtitle
{Caso d'uso UC2.3}
{Recupero password}
\UC
{UC2}
{Utente business}
{Un utente business registrato recupera la password dimenticata}
{Il sistema  reputa l'utente business come registrato}
\post
{Il sistema ha fornito la password all'utente business}



\UCtitle
{Caso d'uso UC2.4}
{Apertura Collection}
\UCimmagine{UC2.4}{UC2.4 - Apertura Collection}
\UC
{UC2.4}
{Utente business autenticato, Utente business autenticato amministratore}
{L’ utente business autenticato può effettuare diverse operazioni: visualizzare una pagina Document-Show, modificare la visualizzazione dei Document, visualizzare una diversa Collection, disconnettersi dal sistema, navigare tra le pagine della pagina Collection-Index correntemente visualizzata,  gestire il profilo. L'utente business autenticato \gloss{amministratore} potrà inoltre cancellare o modificare un Document}
{Il sistema reputa l'utente business come autenticato e visualizza la Collection}
\scenario{
L'utente business autenticato può visualizzare pagine Document-Show (UC2.4.1);|
L'utente business autenticato può modificare la visualizzazione dei Document (UC2.4.2);|
L'utente business autenticato può aprire una diversa Collection (UC2.4);|
L'utente business autenticato può disconnettersi (UC2.5);|
L'utente business autenticato può navigare tra le pagine della Collection-Index attualmente visualizzata (UC2.4.3);|
L'utente business autenticato può gestire il profilo (UC2.6);|
L'\gloss{utente business autenticato amministratore} può cancellare un Document (UC2.4.4);|
L'utente business autenticato amministratore può modificare un Document (UC2.4.5);|
L'utente business autenticato amministratore può visualizzare le \gloss{query} più frequenti che il sistema gli propone (UC2.4.6);|
L'utente business autenticato amministratore può creare degli indici sulle query più frequenti visualizzare (UC2.4.7).
}
\post
{Il sistema ha ottenuto le informazioni sulle operazioni che l'utente business autenticato o l'utente business autenticato amministratore desidera eseguire}



\UCtitle
{Caso d'uso UC2.4.1}
{Visualizzazione pagina Document-Show}
\UC
{UC2.4}
{Utente business autenticato}
{L'utente business autenticato visualizza la pagina Document-Show relativa alla chiave selezionata}
{Il sistema reputa l'utente business come autenticato e visualizza la Collection scelta}
\post
{Viene visualizzata la pagina Document-Show relativa alla chiave selezionata}


\UCtitle
{Caso d'uso UC2.4.2}
{Modifica visualizzazione dei Document}
\UCimmagine{UC2.4.2}{UC2.4.2 - Modifica visualizzazione dei Document}
\UC
{UC2.4.2}
{Utente business autenticato}
{L'utente business autenticato può cambiare la modalità di visualizzazione dei Document}
{Il sistema reputa l'utente business come autenticato e visualizza la Collection scelta}
\scenario{
L'utente business autenticato può selezionare i criteri per la visualizzazione della lista di Document (UC2.4.2.1);|
L'utente business autenticato può applicare dei filtri (UC2.4.2.2).
}
\estensioni
{Solo nel caso in cui venga applicato un \gloss{filtro}, l'utente business autenticato può annullarlo (UC2.4.2.3).}
\post
{Il sistema ha ottenuto le informazioni sulle operazioni che l'utente business autenticato desidera eseguire ed aggiorna la visualizzazione della pagina}

\UCtitle
{Caso d'uso UC2.4.2.1}
{Seleziona criteri per la visualizzazione}
\UCimmagine{UC2.4.2.1}{UC2.4.2.1 - Seleziona criteri per la visualizzazione}
\UC
{UC2.4.2.1}
{Utente business autenticato}
{L'utente business autenticato può selezionare dei criteri per cambiare la visualizzazione della lista di Document: cambiare il numeri di Document da visualizzare per pagina o cambiare l'ordinamento dei Document secondo una chiave}
{Il sistema reputa l'utente business come autenticato e visualizza la Collection scelta}
\scenario{
L'utente business autenticato può cambiare il numero di Document da visualizzare per pagina (UC2.4.2.1.2);|
L'utente business autenticato può cambiare l'ordinamento dei Document secondo una chiave (UC2.4.2.1.1).
}
\post
{Il sistema ha ottenuto le informazioni sulle operazioni che l'utente business autenticato desidera eseguire ed aggiorna la visualizzazione della pagina}


\UCtitle
{Caso d'uso UC2.4.2.1.1}
{Cambio ordinamento per chiave}
\UC
{UC2.4.2.1}
{Utente business autenticato}
{L'utente business autenticato può cambiare l'ordinamento dei documenti visualizzati secondo una chiave}
{Il sistema reputa l'utente business come autenticato e visualizza la Collection scelta}
\post
{Il sistema aggiorna la visualizzazione della pagina secondo i criteri imposti dall'utente business autenticato}


\UCtitle
{Caso d'uso UC2.4.2.1.2}
{Cambio numero di Document da visualizzare per pagina}
\UC
{UC2.4.2.1}
{Utente business autenticato}
{L'utente business autenticato può cambiare il numero di Document da visualizzare per pagina}
{Il sistema reputa l'utente business come autenticato e visualizza la Collection scelta}
\post
{Il sistema aggiorna la visualizzazione del numero di Document per pagina secondo i criteri imposti dall'utente business autenticato}

\UCtitle
{Caso d'uso UC2.4.2.2}
{Applicazione filtro}
\UC
{UC2.4.2}
{Utente business autenticato}
{L'utente business autenticato può applicare un filtro disponibile nella pagina Collection-Index}
{Il sistema reputa l'utente business come autenticato e visualizza la Collection scelta}
\post
{Il sistema aggiorna la visualizzazione della pagina secondo i criteri imposti dal filtro applicato}

\UCtitle
{Caso d'uso UC2.4.2.3}
{Annullare filtro applicato}
\UC
{UC2.4.2}
{Utente business autenticato}
{L'utente business autenticato ha la possibilità di rimuovere un filtro precedentemente applicato}
{Il sistema ha applicato un filtro secondo i criteri imposti dall'utente business autenticato}
\post
{Il sistema aggiorna la visualizzazione della Collection allo stato precedente all'applicazione del filtro}


\UCtitle
{Caso d'uso UC2.4.3}
{Navigazione tra le pagine della Collection-Index}
\UC
{UC2.4}
{Utente business autenticato}
{L'utente business autenticato può navigare tra le pagine della Collection-Index attualmente visualizzata}
{Il sistema reputa l'utente business come autenticato e visualizza la Collection scelta}
\post
{Il sistema visualizza un'altra Collection scelta dall'utente business autenticato}

\UCtitle
{Caso d'uso UC2.4.4}
{Cancellazione Document}
\UC
{UC2.4}
{Utente business autenticato amministratore}
{L'utente business autenticato come amministratore può cancellare un Document presente nella Collection attualmente visualizzata}
{Il sistema reputa l'utente business autenticato come amministratore e visualizza la Collection scelta}
\post
{Il sistema ha cancellato il Document selezionato dall'utente business autenticato amministratore ed aggiorna la pagina attualmente visualizzata}



\UCtitle
{Caso d'uso UC2.4.5}
{Modifica Document}
\UCimmagine{UC2.4.5}{UC2.4.5 - Modifica Document}
\UC
{UC2.4.5}
{Utente business autenticato amministratore}
{L'utente business autenticato come amministratore può modificare un Document presente nella Collection attualmente visualizzata}
{Il sistema reputa l'utente business autenticato come amministratore e visualizza la Collection scelta}
\scenario{
L'utente business autenticato amministratore può modificare il valore di ciascuna chiave (UC2.4.5.1);|
L'utente business autenticato amministratore può salvare il Document appena modificato (UC2.4.5.2).
}
\estensioni
{L'utente business autenticato amministratore può annullare le modifiche fino a quel momento apportate (UC2.4.5.3).}
\post
{Il sistema modifica il Document selezionato secondo le specifiche imposte dall'utente business autenticato amministratore}


\UCtitle
{Caso d'uso UC2.4.5.1}
{Modifica valore della chiave}
\UC
{UC2.4.5}
{Utente business autenticato amministratore}
{L'utente business autenticato come amministratore può modificare il valore di ciascuna chiave del Document tramite form editabile}
{Il sistema reputa l'utente business autenticato come amministratore e visualizza la pagina Document-Show che l'utente business autenticato come amministratore desidera modificare}
\post
{Il sistema ha ottenuto un valore per ogni chiave del Document che l'utente business autenticato come amministratore ha modificato}

\UCtitle
{Caso d'uso UC2.4.5.2}
{Salvataggio Document}
\UC
{UC2.4.5}
{Utente business autenticato amministratore}
{L'utente business autenticato come amministratore può salvare le modifiche fino a quel momento apportate al Document}
{Il sistema ha ottenuto un valore per ogni chiave del Document che l'utente business autenticato come amministratore vuole modificare}
\post
{Il sistema salva le modifiche apportate al Document}

\UCtitle
{Caso d'uso UC2.4.5.3}
{Annulla modifiche Document}
\UC
{UC2.4.5}
{Utente business autenticato amministratore}
{L'utente business autenticato come amministratore ha la possibilità di annullare le modifiche apportate al Document (vedi UC2.4.5)}
{Il sistema ha ottenuto un valore per ogni chiave del Document che si sta modificando}
\post
{Il sistema non considera i valori inseriti per ogni chiave del Document e visualizza la pagina Collection-Index principale}

\UCtitle
{Caso d'uso UC2.4.6}
{Visualizzazione query più utilizzate}
\UC
{UC2.4}
{Utente business autenticato amministratore}
{Il sistema reputa l'utente business autenticato come amministratore}
{L'utente business autenticato amministratore può visualizzare le query più utilizzate che il sistema gli offre}
\post
{Il sistema mostra all'utente business autenticato amministratore le query più utilizzate}

\UCtitle
{Caso d'uso UC2.4.7}
{Creazione indici}
\UC
{UC2.4}
{Utente business autenticato amministratore}
{L'utente business autenticato amministratore può creare degli indici sulla base delle query più utilizzate, le quali vengono fornite dal sistema}
{Il sistema reputa l'utente business autenticato come amministratore}
\post
{Il sistema permette all'utente business autenticato amministratore di creare degli indici}

\UCtitle
{Caso d'uso UC2.5}
{Disconnessione}
\UC
{UC2}
{Utente business autenticato}
{L'utente può uscire dal proprio profilo e disconnettersi dal sistema}
{Il sistema reputa l'utente business come autenticato}
\post
{Il sistema reputa l'utente business come non autenticato}

\UCtitle
{Caso d'uso UC2.6}
{Gestione utente}
\UCimmagine{UC2.6}{UC2.6 - Gestione utente}
\UC
{UC2.6}
{Utente business autenticato}
{L'utente business autenticato può modificare i dati utente, mentre l'utente business autenticato amministratore può anche eseguire operazioni di creazione e cancellazione di un utente}
{Il sistema reputa l'utente business come autenticato}
\scenario{
L'utente business autenticato può gestire i dati (UC2.6.3);|
L'utente business autenticato amministratore può creare un nuovo utente (UC2.6.1);|
L'utente business autenticato amministratore può eliminare un utente (UC2.6.2).
}
\post
{Il sistema ha ottenuto le informazioni sulle operazioni che l'utente business desidera eseguire}

\UCtitle
{Caso d'uso UC2.6.1}
{Creazione nuovo utente}
\UC
{UC2.6}
{Utente business autenticato amministratore}
{L'utente business autenticato come amministratore può creare un nuovo utente}
{Il sistema reputa l'utente business autenticato come amministratore}
\post
{Il sistema ha creato un nuovo utente}


\UCtitle
{Caso d'uso UC2.6.2}
{Eliminazione utente}
\UC
{UC2.6}
{Utente business autenticato amministratore}
{L'utente business autenticato come amministratore può eliminare un profilo utente preesistente}
{Il sistema reputa l'utente business autenticato come amministratore}
\post
{Il sistema ha eliminato il profilo utente specificato}

\UCtitle
{Caso d'uso UC2.6.3}
{Gestione dati}
\UCimmagine{UC2.6.3}{UC2.6.3 - Gestione dati}
\UC
{UC2.6.3}
{Utente business autenticato}
{L'utente business autenticato può modificare i dati utente del proprio profilo e salvare le modifiche o annullarle. L'utente business autenticato come amministratore può modificare i dati dei vari profili utenti, modificare i permessi degli utenti, salvare le modifiche o annullarle}
{Il sistema reputa l'utente business come autenticato}
\scenario{
L'utente business autenticato può modificare i dati utente del proprio profilo (UC2.6.3.1);|
L'utente business autenticato può salvare le modifiche (UC2.6.3.2);|
L'utente business autenticato come amministratore può modificare i dati utenti (UC2.6.3.3);|
L'utente business autenticato come amministratore può modificare i permessi degli utenti (UC2.6.3.4).
}
\post
{Il sistema modifica i dati utente secondo le specifiche imposte dall'utente business autenticato}


\UCtitle
{Caso d'uso UC2.6.3.1}
{Modifica dati utente}
\UC
{UC2.6.3}
{Utente business autenticato}
{L'utente business autenticato può modificare i dati utente del proprio profilo, ovvero email e password}
{Il sistema reputa l'utente business come autenticato}
\post
{Il sistema ha ottenuto tutti i dati modificati del profilo utente}



\UCtitle
{Caso d'uso UC2.6.3.2}
{Salvataggio modifiche}
\UC
{UC2.6.3}
{Utente business autenticato}
{L'utente business autenticato può salvare le modifiche apportate durante la modifica dei dati utenti}
{Il sistema ha ottenuto i dati modificati del profilo utente}
\post
{Il sistema aggiorna il profilo utente con i nuovi dati inseriti}

\UCtitle
{Caso d'uso UC2.6.3.3}
{Modifica dati utenti}
\UC
{UC2.6.3}
{Utente business autenticato amministratore}
{L'utente business autenticato come amministratore può modificare i dati degli utenti, ovvero email e password}
{Il sistema reputa l'utente business autenticato come amministratore}
\estensioni
{L'utente business autenticato può annullare le modifiche fin'ora apportate (UC2.6.3.5).}
\post
{Il sistema ha ottenuto tutti i dati modificati del profilo utente}

\UCtitle
{Caso d'uso UC2.6.3.4}
{Modifica permessi utenti}
\UC
{UC2.6.3}
{Utente business autenticato amministratore}
{L'utente business autenticato come amministratore può modificare i permessi degli utenti, ovvero elevare un utente a livello amministratore o declassare un utente amministratore ad utente business autenticato}
{Il sistema reputa l'utente business autenticato come amministratore}
\post
{Il sistema ha ottenuto i nuovi permessi da applicare al profilo utente}


\UCtitle
{Caso d'uso UC2.6.3.5}
{Annulla modifiche}
\UC
{UC2.6.3}
{Utente business autenticato, Utente business autenticato amministratore}
{L'utente business autenticato e l'amministratore possono annullare le modifiche apportate}
{Il sistema ha ottenuto i dati modificati del profilo utente; se si tratta di amministratore il sistema ottiene i dati modificati dei profili e/o dei dati relativi ai permessi dell'utente}
\post
{Il sistema non considera le modifiche apportate e visualizza la pagina precedente}



\UCtitle
{Caso d'uso UC2.7}
{Inserimento \gloss{email} e password}
\UC
{UC2}
{Utente business}
{Un utente business inserisce email e password per autenticarsi}
{Il sistema si trova nello stato iniziale e propone all'utente business una schermata per inserire i dati di autenticazione}
\post
{Il sistema ha acquisito i dati dell'utente business}







