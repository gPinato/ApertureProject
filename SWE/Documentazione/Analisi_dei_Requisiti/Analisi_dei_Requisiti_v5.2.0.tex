%includo il file che contiene la versione dei documenti
\newcommand{\versioneAnalisiDeiRequisiti}{2.2.0}			
\newcommand{\versioneNormeDiProgetto}{2.2.0}			
\newcommand{\versioneGlossario}{2.2.0}			
\newcommand{\versionePianoDiQualifica}{2.2.0}			
\newcommand{\versionePianoDiProgetto}{2.2.0}	
\newcommand{\versioneStudioDiFattibilita}{2.2.0}
\newcommand{\versioneSpecificaTecnica}{2.2.0}


\newcommand{\Versione}{\versioneAnalisiDeiRequisiti{}}	%Versione Finale
\newcommand{\Data}{2013-12-04}							%Data di creazione
\newcommand{\DataUltimaModifica}{2014-07-02}
\newcommand{\TipoDocumento}{Analisi dei Requisiti}		%tipo documento

%includo il file header.tex (logo grande in prima pagina piu qualche altra regola)
%questo file contiene impostazioni comuni per tutte i documenti

%definizione packages utilizzati
\documentclass[a4paper]{article}
\usepackage[utf8x]{inputenc}
\usepackage{enumitem}
\usepackage[italian]{babel}
\usepackage{latexsym}
\usepackage{xparse}
\usepackage{float}
\usepackage{subfloat}
\usepackage{subfig}
\usepackage{fancyhdr}
\usepackage{eurofont}
\usepackage{lastpage}
\usepackage{graphicx}
\usepackage{textcomp}
\usepackage{booktabs}
\usepackage{color}
\usepackage{lscape}
\usepackage{hyperref}
\hypersetup{colorlinks=true, linkcolor=black, anchorcolor=red, urlcolor=blue}
\usepackage{longtable}
\usepackage{tabularx}
\usepackage{abstract}
\usepackage{appendix}
\usepackage{multicol}
\usepackage{bmpsize}
\usepackage[all]{hypcap}
\usepackage{titlesec}
\usepackage{indentfirst}
\usepackage{lipsum,titletoc}

%\setcounter{secnumdepth}{4}

%****************INIZIO GESTIONE SUBSECTION MULTIPLE
\makeatletter
\newcommand\level[1]{%
  \ifcase#1\relax\expandafter\chapter\or
    \expandafter\section\or
    \expandafter\subsection\or
    \expandafter\subsubsection\else
    \def\next{\@level{#1}}\expandafter\next
  \fi}
\newcommand{\@level}[1]{%
  \@startsection{level#1}
    {#1}
    {\z@}%
    {-3.25ex\@plus -1ex \@minus -.2ex}%
    {1.5ex \@plus .2ex}%
    {\normalfont\normalsize\bfseries}}

\newdimen\@leveldim
\newdimen\@dotsdim
{\normalfont\normalsize
 \sbox\z@{0}\global\@leveldim=\wd\z@
 \sbox\z@{.}\global\@dotsdim=\wd\z@
}

\newcounter{level4}[subsubsection]
\@namedef{thelevel4}{\thesubsubsection.\arabic{level4}}
\@namedef{level4mark}#1{}
\def\l@section{\@dottedtocline{1}{0pt}{\dimexpr\@leveldim*4+\@dotsdim*1+6pt\relax}}
\def\l@subsection{\@dottedtocline{2}{0pt}{\dimexpr\@leveldim*5+\@dotsdim*2+6pt\relax}}
\def\l@subsubsection{\@dottedtocline{3}{0pt}{\dimexpr\@leveldim*6+\@dotsdim*3+6pt\relax}}
\@namedef{l@level4}{\@dottedtocline{4}{0pt}{\dimexpr\@leveldim*7+\@dotsdim*4+6pt\relax}}

\count@=4
\def\@ncp#1{\number\numexpr\count@+#1\relax}
\loop\ifnum\count@<100
  \begingroup\edef\x{\endgroup
    \noexpand\newcounter{level\@ncp{1}}[level\number\count@]
    \noexpand\@namedef{thelevel\@ncp{1}}{%
      \noexpand\@nameuse{thelevel\@ncp{0}}.\noexpand\arabic{level\@ncp{1}}}
    \noexpand\@namedef{level\@ncp{1}mark}####1{}%
    \noexpand\@namedef{l@level\@ncp{1}}%
      {\noexpand\@dottedtocline{\@ncp{1}}{0pt}{\the\dimexpr\@leveldim*\@ncp{5}+\@dotsdim*\@ncp{0}\relax}}}%
  \x
  \advance\count@\@ne
\repeat
\makeatother
\setcounter{secnumdepth}{100}
\setcounter{tocdepth}{100}
%****************FINE GESTIONE SUBSECTION MULTIPLE

%impostazioni relative alla visualizzazione delle section 
%nell'indice
\titlecontents{section}
[0pt]%left indent
{\bfseries}
{\contentslabel{2.3em}}
{\hspace*{-2.3em}}
{\hfill\contentspage}
[]%separator


\oddsidemargin=.15in
\evensidemargin=.15in
\textwidth=6in
\topmargin=-.5in
\parindent=0in
\headheight=1in
\DeclareMathSizes{10}{10}{10}{10} %per piano qualifica
\pagestyle{fancy}
\lhead{
\bfseries {\Large \TipoDocumento}\\
\bfseries Versione: \Versione\\
}
\chead{}
\lhead{
\includegraphics[scale=0.455]{../Logo&Header/apertureHead.png}
}
\lfoot{\bfseries \TipoDocumento{} v\Versione}
\cfoot{}
\rfoot{\thepage\ of \mypageref{LastPage}}
\newcommand{\mypageref}[1]{
\hypersetup{linkcolor=black}\pageref{#1}\hypersetup{linkcolor=black}}
%\userpackage{lipsum}
\renewcommand{\footrulewidth}{0.4pt}

%definizioni comandi comuni utilizzati
\newcommand{\numref}[1]{\textsl{\nameref{#1} (\ref{#1})}}
\newcommand{\NomeGruppo}{Aperture Software}
\newcommand{\Progetto}{MaaP: MongoDB as an admin Platform}
\newcommand{\Prop}{CoffeeStrap}

%definizione tecnologie
\newcommand{\Node}{Node.js}
\newcommand{\NodeJS}{Node.js}
\newcommand{\Nodejs}{Node.js}

\newcommand{\mongodb}{MongoDB}

%tanti sub quanti ne vogliamo! :)
\newcommand{\subsubsubsection}{\level{4}}
\newcommand{\subsubsubsubsection}{\level{5}}
\newcommand{\subsubsubsubsubsection}{\level{6}}
\newcommand{\subsubsubsubsubsubsection}{\level{7}}
\newcommand{\subsubsubsubsubsubsubsection}{\level{8}}


%definizione comando per parola glossario
\newcommand{\gloss}[1]{\emph{#1}\ped{\emph{\tiny{G}}}}

\newcommand{\grassetto}{\textbf}

%per inserire immagini
\newcommand{\immagine}[2]{ 
\begin{center}
\begin{figure}[H]
\includegraphics[width=\textwidth]{{{#1}}}
\caption{#2}
\label{#1}
\end{figure}
\end{center}
}

\newcommand{\Glossario}{
Al fine di evitare ogni ambiguità nella comprensione del linguaggio utilizzato nel presente documento e, in generale, nella documentazione fornita dal gruppo \NomeGruppo{}, ogni termine tecnico, di difficile comprensione o di necessario approfondimento verrà inserito nel documento \emph{Glossario\_{}v\versioneGlossario{}.pdf}.\\
Saranno in esso definiti e descritti tutti i termini in corsivo e allo stesso tempo marcati da una lettera "G" maiuscola in pedice nella documentazione fornita.
}

\newcommand{\Prodotto}{
Lo scopo del prodotto è produrre un framework per generare interfacce web di amministrazione dei dati di business basati sullo stack \Nodejs{} e \mongodb{}.\\
L'obiettivo è quello di semplificare il lavoro allo sviluppatore che dovrà rispondere in modo rapido e standard alle richieste degli esperti di business.
}

%inizio pagina del documento 
\begin{document}
\thispagestyle{empty}

\begin{center}\centerline{
%inserisco il logo grande della prima pagina
\includegraphics[scale=0.8]{../Logo&Header/logo.png}}

%metto il link dell'email sotto al logo
%{\href{mailto:ApertureSWE@gmail.com}{\color[rgb]{0.39,0.37,0.38}%ApertureSWE@gmail.com}}\\ [3pc]

\vspace{0.5in}

%titolo del progetto
{\Huge {\Progetto}}\\[.5pc]

\underline{\hspace{6in}}\\[8pc]

{\Huge {\TipoDocumento}}\\[1pc]
%{\emph{Versione \Versione}}\\
\end{center}

%\vspace{.05in}

%comandi per generare i casi d'uso:

%comando per inserire un item nella lista
\newcommand{\inserisciItem}[1]{\item #1 }

%comando per generare una lista completa, necessita del package xparse
\NewDocumentCommand{\scenario}{>{\SplitList{|}}m}
{
	\textbf{Scenario Principale:}
	\begin{enumerate}
		\ProcessList{#1}{\inserisciItem}
	\end{enumerate}
}

\NewDocumentCommand{\scenarioAlt}{>{\SplitList{|}}m}
{
	\textbf{Scenario Alternativo:}
	\begin{enumerate}
		\ProcessList{#1}{\inserisciItem}
	\end{enumerate}
}

\NewDocumentCommand{\lista}{>{\SplitList{;}}m}
{
	\begin{itemize}
		\ProcessList{#1}{\inserisciItem}
	\end{itemize}
}

\NewDocumentCommand{\inclusioni}{>{\SplitList{|}}m}
{
	\textbf{Inclusioni:}
	\begin{itemize}
		\ProcessList{#1}{\inserisciItem}
	\end{itemize}
}

\NewDocumentCommand{\estensioni}{>{\SplitList{|}}m}
{
	\textbf{Estensioni:}
	\begin{itemize}
		\ProcessList{#1}{\inserisciItem}
	\end{itemize}
}

%\UC{nome}{descrizione}{attori}{scopo e descrizione}{pre}{scenario1;scenario2;...}{post}
\newcommand{\UCtitle}[2]{
\subsection{{#1}: {#2}}
}

\newcommand{\UC}[4]{
\textbf{Diagramma associato:} \ref{#1}\\[\baselineskip]
\textbf{Attori:} {#2};\\[\baselineskip]
\textbf{Scopo e Descrizione:} {#3};\\[\baselineskip]
\textbf{Precondizione:} {#4};\\[\baselineskip]
}

\newcommand{\post}[1]{
\textbf{Postcondizione:} {#1}.
}

%comando per inserire un diagramma
% \UCimmagine{nomeFileSenzaEstensione}{descrizioneDidascalia}
\newcommand{\UCimmagine}[2]{ 
\begin{center}
\begin{figure}[H]
\includegraphics[width=\textwidth]{{{#1}}}
\caption{#2}
\label{#1}
\end{figure}
\end{center}
}

%\vspace{.05in}

%informazioni documento
\begin{center}
%\section{Informazioni documento}
\begin{tabular}{r|l}
%\textbf{Nome} &\TipoDocumento \\
\textbf{Versione} & \Versione{} \\
\textbf{Data creazione} & \Data{} \\
\textbf{Data ultima modifica} & \DataUltimaModifica{} \\
\textbf{Stato del Documento} & Formale \\					%CAMBIARE QUI
\textbf{Uso del Documento} & Esterno \\						%CAMBIARE QUI
\textbf{Redazione} & Fabio Miotto, Alberto Garbui\\			%CAMBIARE QUI
\textbf{Verifica} & Alessandro Benetti, Mattia Sorgato\\	%ED ANCHE QUI!
\textbf{Approvazione} & Michele Maso\\					    %CAMBIARE QUI
\textbf{Distribuzione} & \parbox[t]{4cm}{Prof. Tullio Vardanega \\ Prof. Riccardo Cardin \\ \Prop{} }\\
\end{tabular}
\end{center}

\vspace{0.05in}

%inizio sommario del documento
\begin{abstract}
\begin{center}
Questo documento si propone di presentare l'analisi dei requisiti che il prodotto \Progetto{} dovrà rispettare, individuati a partire dal capitolato d'appalto del Proponente \Prop{}.
\end{center}
\end{abstract}

%\vspace{.4in}

%seconda pagina, diario delle modifiche
\newpage
\section{Diario delle modifiche}
\begin{center}
\begin{longtable}{|c|c|c|p{0.5\linewidth}|}
\toprule
\textbf{Versione} & \textbf{Data} & \textbf{Autore} & \textbf{Modifiche effettuate}\\

%aggiungere qui una midrule per aggiungere una nuova riga alla tabella
\midrule
5.2.0 & 2014-07-xx & Alberto Garbui (RE) & Approvazione documento.\\
\midrule
5.1.0 & 2014-07-xx & xxx (VR) & Verifica documento.\\

\midrule
5.0.3 & 2014-07-xx & xxx (VR) & Aggiornamento tracciamento.\\

\midrule
5.0.2 & 2014-07-xx & xxx (AN) & Effettuate correzioni segnalate dal Committente.\\
\midrule
5.0.1 & 2014-07-xx & xxx (AN) & Aggiunti casi d'uso derivanti dall'incontro con il Proponente.\\

\midrule
4.2.0 & 2014-04-14 & Mattia Sorgato (RE) & Approvazione documento.\\
\midrule
4.1.0 & 2014-04-12 & Giacomo Pinato (VR) & Verifica documento.\\
\midrule
4.0.2 & 2014-04-11 &  Fabio Miotto (AN) & Incremento documento.\\
\midrule
4.0.1 & 2014-04-09 &  Fabio Miotto (AN) & Effettuate correzioni segnalate dal Committente.\\

\midrule
3.2.0 & 2014-01-24 & Michele Maso (RE) & Approvazione documento.\\
\midrule
3.1.1 & 2014-01-23 & Alessandro Benetti (VR) & Verifica documento.\\
\midrule
3.1.0 & 2014-01-21 & Mattia Sorgato (VR) & Verifica documento.\\
\midrule
3.0.2 & 2014-01-18 & Alberto Garbui (AN) & Incremento documento.\\
\midrule
3.0.1 & 2014-01-13 & Fabio Miotto (AN) & Effettuate correzioni segnalate dal Committente.\\

\midrule
2.2.0 & 2014-01-06 & Alberto Garbui (RE) & Approvazione documento.\\
\midrule
2.1.0 & 2014-01-04 & Fabio Miotto (VR) & Verifica documento.\\
\midrule
2.0.2 & 2013-12-30 & Alessandro Benetti (AN) & Aggiornato tracciamento.\\
\midrule
2.0.1 & 2013-12-23 & Mattia Sorgato (AN) & Incremento del documento.\\

\midrule
1.2.0 & 2013-12-18 & Giacomo Pinato (RE) & Approvazione documento.\\
\midrule
1.1.1 & 2013-12-17 & Alessandro Benetti (VR) & Verifica documento.\\
\midrule
1.1.0 & 2013-12-16 & Andrea Perin (VR) & Verifica documento.\\
\midrule
1.0.6 & 2013-12-15 & Mattia Sorgato (AN) & Aggiunti tracciamenti.\\
\midrule
1.0.5 & 2013-12-12 & Mattia Sorgato (AN) & Aggiunti requisiti.\\
\midrule
1.0.4 & 2013-12-09 & Fabio Miotto (AN) & Aggiunti casi d'uso MaaP's Web.\\
\midrule
1.0.3 & 2013-12-09 & Michele Maso (AN) & Aggiunti casi d'uso MaaP.\\
\midrule
1.0.2 & 2013-12-08 & Alberto Garbui (AN) & Aggiunti casi d'uso MaaS.\\
\midrule
1.0.1 & 2013-12-04 & Alberto Garbui (AN) & Creazione documento e stesura descrizione generale.\\

\bottomrule
\caption{Registro delle modifiche}
\label{tab:changelog}
\end{longtable}
\end{center}

%terza pagina Indice (viene aggiornato in automatico con due compilazioni)
\newpage
\tableofcontents

%pagine successive hanno la lista di tabelle e lista delle figure
%(vengono aggiornate in automatico)
\newpage
\listoftables
\listoffigures

%qui inizia la prima pagina ufficiale
\newpage
\section{Introduzione}%1.0
\subsection{Scopo del documento}%1.1
Il presente documento ha lo scopo di illustrare ed analizzare in modo dettagliato i requisiti del prodotto \grassetto{MaaP} evidenziati nel capitolato d'appalto C1 dal Proponente \Prop{} presentando così le funzionalità che offrirà il prodotto.

\subsection{Scopo del prodotto} %1.2
\Prodotto{}
\subsection{Glossario}%1.3
\Glossario{}

\subsection{Riferimenti} %1.41
\subsubsection{Normativi} %1.4.1


\begin{itemize}
\item Capitolato d'appalto C1 - \Progetto{} \\
\url{http://www.math.unipd.it/~tullio/IS-1/2013/Progetto/C1.pdf};
\item \grassetto{Norme di Progetto}: \emph{Norme\_{}di\_{}Progetto\_{}v\versioneNormeDiProgetto{}.pdf};\\
\item \grassetto{Verbale}: \emph{Verbale\_esterno\_20140305\_v1.2.0.pdf}.
\end{itemize}

\subsubsection{Informativi} %1.4.2
\begin{itemize}
\item Slide dell'insegnamento Ingegneria del Software modulo A:\\
Ingegneria dei requisiti: \url{http://www.math.unipd.it/~tullio/IS-1/2013/Dispense/L06.pdf};
\item Software Engineering - Ian Sommerville - 9th Edition (2010):\\
Chapter 4: Requirements engineering;
\item IEEE 830-1998: Recommended Practice for Software Requirements Specifications;\\
\url{http://en.wikipedia.org/wiki/Software_requirements_specification}.

\end{itemize}

\newpage
\section{Descrizione generale}%2.0
\subsection{Contesto d'uso del prodotto} %2.1
La problematica che il prodotto si pone di risolvere è quella di ridurre il carico di lavoro degli sviluppatori per la creazione di pagine \gloss{web} di visualizzazione di dati provenienti da \gloss{MongoDB}.

\subsection{Funzioni del prodotto} %2.2
Il prodotto MaaP permetterà di creare uno scheletro del progetto comprensivo di:
\begin{itemize}
\item Librerie necessarie al funzionamento del progetto;
\item \gloss{File di configurazione} necessari al funzionamento dell'applicazione nella loro versione predefinita;
\item \gloss{Sistema di autenticazione};
\item Directory di descrizione \gloss{pagine web}.
\end{itemize}
Tramite l'utilizzo di un linguaggio astratto DSL(Domain Specific Language), il prodotto permetterà all'\gloss{utente} di generare due tipi di pagine:
\begin{itemize}
\item \gloss{Collection-Index};
\item \gloss{Document-Show}.
\end{itemize}

All'interno della pagina di tipo "Collection-Index" sarà possibile:
\begin{itemize}
\item Visualizzare tutti i documenti della \gloss{Collection};
\item Selezionare, per ogni documento, una \gloss{chiave} che rimanderà alla relativa pagina di tipo Document-show.
\end{itemize}

All'interno della pagina di tipo Document-Show sarà possibile:
\begin{itemize}
\item Visualizzare in forma estesa tutte le coppie chiave-valore;
\item Modificare i campi del documento nel caso l'utente abbia permessi di scrittura.
\end{itemize}

Inoltre sarà possibile, attraverso un menù, selezionare la Collection che si desidera visualizzare.

\subsection{Caratteristiche degli utenti}
Il prodotto MaaP si rivolge ad una sola tipologia di utenti che possiedono alte competenze tecniche, nello specifico sviluppatori e/o esperti in amministrazione di \gloss{database}. Le pagine generate dal \gloss{framework} verranno utilizzate da utenti esperti di \gloss{business}, i quali non dovranno obbligatoriamente possedere particolari conoscenze informatiche. Questi utenti si dividono in due categorie: utenti business, con soli permessi di lettura e visualizzazione del database di analisi ed eventuale gestione del \gloss{profilo}, e utenti business amministratori, i quali avranno ulteriori privilegi di gestione.
Il prodotto MaaP si rivolge sia agli sviluppatori sia agli esperti di business.

\subsection{Vincoli generali}
Nel computer è necessario che sia installata la versione 30.0.x di \gloss{Google} Chrome o superiori, oppure la versione 24.x o superiori di \gloss{Firefox}.

\newpage
\section{Casi d'uso} %3
Di seguito sono presentati i casi d'uso identificati a partire dal capitolato d'appalto \Progetto. \\
Ogni caso d'uso figlio eredita le precondizioni dei casi d'uso di livello superiore che fanno parte della sua gerarchia al fine di ridurre l'esposizione delle precondizioni.
Ogni caso d'uso è descritto secondo la notazione riportata in sezione 5.3.1 del documento \emph{Norme\_{}di\_{}Progetto\_{}v\versioneNormeDiProgetto{}.pdf}.


\UCtitle
{Caso d'uso UC1}
{MaaP}
\UCimmagine{UC1}{UC1 MaaP}
\UC
{UC1}
{Utente \gloss{sviluppatore}}
{L'\gloss{utente sviluppatore}  può eseguire dei comandi per generare lo scheletro, inserire il \gloss{file di descrizione}, usare un \gloss{editor} specializzato per scrivere il file di descrizione, utilizzare il file di descrizione e modificare il file di configurazione}
{Il sistema è installato e funzionante nel pc in uso}
\scenario{
L'utente sviluppatore può generare lo scheletro del progetto (UC1.1);|
L'utente sviluppatore può utilizzare il file di descrizione (UC1.2);|
L'utente sviluppatore può modificare i file di configurazione (UC1.3);|
L'utente sviluppatore può utilizzare un editor di testo specializzato (UC1.4);|
L'utente sviluppatore può inserire un file di descrizione (UC1.5).
}
\post{Il sistema ha ottenuto le informazioni sulle operazioni che l'utente sviluppatore desidera eseguire}

\UCtitle
{Caso d'uso UC1.1}
{Generazione scheletro}
\UC
{UC1}
{Utente sviluppatore}
{L'utente sviluppatore può generare lo scheletro del progetto, il quale comprende: librerie e file di configurazione necessari al funzionamento dello stesso, directory di descrizione delle pagine web generate e il sistema di autenticazione per quest'ultime}
{Il sistema è installato e funzionante nel pc in uso e permette all'utente sviluppatore di generare scheletro di un nuovo progetto}
\post{Il sistema ha generato lo scheletro del progetto}



\UCtitle
{Caso d'uso UC1.2}
{Utilizzo file di descrizione}
\UCimmagine{UC1.2}{UC1.2 Utilizzo file di descrizione}
\UC
{UC1.2}
{Utente sviluppatore}
{L'utente sviluppatore  può usare il file di descrizione per definire la visualizzazione della Collection utilizzando i \gloss{template} delle pagine web inseriti nel file di configurazione}
{Il sistema ha generato lo scheletro del progetto e il file di descrizione deve essere stato inserito}
\scenario{
L'utente sviluppatore può creare la visualizzazione di una Collection (UC1.2.1);|
L'utente sviluppatore può modificare la visualizzazione di una Collection (UC1.2.2).
}
\post{Il sistema contiene la descrizione per la visualizzazione della Collection}

\UCtitle
{Caso d'uso UC1.2.1}
{Creazione visualizzazione Collection}
\UCimmagine{UC1.2.1}{UC1.2.1 Creazione visualizzazione Collection}
\UC
{UC1.2.1}
{Utente sviluppatore}
{L'utente sviluppatore  può impostare la visualizzazione del menù delle Collection, delle pagine \gloss{Collection-Index} e \gloss{Document-Show}}
{Il sistema contiene la descrizione per la visualizzazione della Collection predefinita e l'utente sviluppatore desidera crearne una nuova}
\scenario{
L'utente sviluppatore può impostare la visualizzazione del menù per le pagine di tipo Collection-index (UC1.2.1.1);|
L'utente sviluppatore può impostare la visualizzazione della pagina Collection-Index (UC1.2.1.2);|
L'utente sviluppatore può impostare la visualizzazione della pagina Document-Show (UC1.2.1.3).
}
\post{Il sistema ha creato una nuova descrizione per la visualizzazione della Collection}

\UCtitle
{Caso d'uso UC1.2.1.1}
{Creazione visualizzazione menù}
\UCimmagine{UC1.2.1.1}{UC1.2.1.1 Creazione visualizzazione menù}
\UC
{UC1.2.1.1}
{Utente sviluppatore}
{L'utente sviluppatore può definire il nome della voce relativa alla Collection che verrà visualizzata e la posizione di questa nel menù delle Collection}
{Il sistema contiene la descrizione della visualizzazione del menù predefinita. Nella definizione è specificato il nome della voce del menù corrispondente al nome della Collection e la posizione, la prima disponibile in ordine crescente}
\scenario{
L'utente sviluppatore può definire il nome della voce relativa alla Collection all'interno del menù (UC1.2.1.1.1);|
L'utente sviluppatore può definire la posizione della Collection all'interno del menù (UC1.2.1.1.2).
}
\post{Il sistema ha creato una nuova descrizione per la visualizzazione del menù delle Collection}

\UCtitle
{Caso d'uso UC1.2.1.1.1}
{Definizione nome voce}
\UC
{UC1.2.1.1}
{Utente sviluppatore}
{L'utente sviluppatore può definire il nome della voce relativa alla Collection nel menù delle Collection}
{Il sistema contiene la descrizione per la visualizzazione del menù predefinita, nella quale il nome della voce del menù corrisponde al nome della Collection. L'utente desidera definire un nuovo nome per la voce}
\post{Il sistema contiene la descrizione per la visualizzazione del menù, nella quale il nome della voce del menù corrisponde al nome della Collection è definito dall'utente sviluppatore}

\UCtitle
{Caso d'uso UC1.2.1.1.2}
{Definizione posizione voce}
\UC
{UC1.2.1.1}
{Utente sviluppatore}
{L'utente può definire la posizione della voce relativa al nome della Collection all'interno del menù delle Collection }
{Il sistema contiene la descrizione per la visualizzazione del menù predefinita, nella quale la posizione \'e la prima disponibile in ordine crescente. L'utente sviluppatore desidera specificare la posizione della voce}
\post{Il sistema contiene la descrizione per la visualizzazione del menù, nella quale la posizione della voce nel menù è definita dall'utente sviluppatore}



\UCtitle
{Caso d'uso UC1.2.1.2}
{Creazione visualizzazione pagina Collection-Index}
\UCimmagine{UC1.2.1.2}{UC1.2.1.2 Creazione visualizzazione pagina Collection-Index}
\UC
{UC1.2.1.2}
{Utente sviluppatore}
{L'utente sviluppatore  può creare una descrizione per la visualizzazione della pagina Collection-Index e aggiungere varie specifiche al suo interno}
{Il sistema contiene la descrizione per la visualizzazione della pagina Collection-Index predefinita}
\scenario{
L'utente sviluppatore può aggiungere una chiave da visualizzare (UC1.2.1.2.1);|
L'utente sviluppatore può definire un ordinamento per chiave (UC1.2.1.2.2);|
L'utente sviluppatore può aggiungere un numero massimo di \gloss{Document} da visualizzare (UC1.2.1.2.3);|
L'utente sviluppatore può aggiungere un pulsante (UC1.2.1.2.4).
}
\post{Il sistema ha creato la descrizione per la visualizzazione della pagina Collection-Index con le specifiche aggiunte}


\UCtitle
{Caso d'uso UC1.2.1.2.1}
{Aggiunta chiave}
\UC
{UC1.2.1.2}
{Utente sviluppatore}
{L'utente sviluppatore  può aggiungere una chiave alla definizione della visualizzazione della pagina Collection-Index}
{Il sistema contiene la descrizione per la visualizzazione della pagina Collection-Index}
\post{Il sistema contiene la descrizione per la visualizzazione della pagina Collection-Index aggiornata con la chiave aggiunta dall'utente sviluppatore}


\UCtitle
{Caso d'uso UC1.2.1.2.2}
{Definizione ordinamento per chiave}
\UC
{UC1.2.1.2}
{Utente sviluppatore}
{L'utente sviluppatore  può  definire un ordinamento, alfabetico crescente o decrescente, rispetto ad una chiave nella definizione della visualizzazione della pagina Collection-Index}
{Il sistema contiene la descrizione per la visualizzazione della pagina Collection-Index predefinita, nella quale non è presente alcun ordinamento per chiave definito dall'utente}
\post{Il sistema contiene la descrizione per la visualizzazione della pagina Collection-Index aggiornata con il nuovo ordinamento per chiave definito dall'utente sviluppatore}

\UCtitle
{Caso d'uso UC1.2.1.2.3}
{Aggiunta numero massimo di Document per pagina}
\UC
{UC1.2.1.2}
{Utente sviluppatore}
{L'utente sviluppatore  può definire un numero massimo di Document da visualizzare nella definizione della visualizzazione della pagina Collection-Index}
{Il sistema contiene la descrizione per la visualizzazione della pagina Collection-Index predefinita, nella quale non è presente alcun numero massimo di Document definito dall'utente}
\post{Il sistema contiene la descrizione per la visualizzazione della pagina Collection-Index aggiornata con il nuovo numero massimo di Document per pagina definito dall'utente sviluppatore}

\UCtitle
{Caso d'uso UC1.2.1.2.4}
{Aggiunta pulsante}
\UC
{UC1.2.1.2}
{Utente sviluppatore}
{L'utente sviluppatore può aggiungere un pulsante per modificare la visualizzazione della pagina Collection-Index, specificando l'etichetta e l'azione che deve eseguire}
{Il sistema contiene la descrizione per la visualizzazione della pagina Collection-Index}
\post{Il sistema contiene la descrizione per la visualizzazione della pagina Collection-Index aggiornata con il nuovo  pulsante aggiunto dall'utente sviluppatore}


\UCtitle
{Caso d'uso UC1.2.1.3}
{Creazione visualizzazione pagina Document}
\UCimmagine{UC1.2.1.3}{UC1.2.1.3 Creazione visualizzazione pagina Document}
\UC
{UC1.2.1.3}
{Utente sviluppatore}
{L'utente sviluppatore  può creare la descrizione per la visualizzazione della pagina Document-Show e aggiungere varie specifiche al suo interno}
{Il sistema contiene la descrizione per la visualizzazione della pagina Document-Show predefinita}
\scenario{
L'utente può aggiungere una chiave da visualizzare (UC1.2.1.3.1);|
L'utente può aggiungere un pulsante da visualizzare (UC1.2.1.3.2).
}
\post{Il sistema ha creato la nuova descrizione per la visualizzazione della pagina Document-Show con le specifiche aggiunte}

\UCtitle
{Caso d'uso UC1.2.1.3.1}
{Aggiunta chiave}
\UC
{UC1.2.1.3}
{Utente sviluppatore}
{L'utente sviluppatore  può aggiungere una chiave alla descrizione della visualizzazione della pagina Document-Show}
{Il sistema contiene la descrizione per la visualizzazione della pagina Document-Show}
\post{Il sistema contiene la descrizione per la visualizzazione della pagina Document-Show aggiornata con la nuova chiave aggiunta dall'utente sviluppatore}

\UCtitle
{Caso d'uso UC1.2.1.3.2}
{Aggiunta pulsante}
\UC
{UC1.2.1.3}
{Utente sviluppatore}
{L'utente sviluppatore può aggiungere un pulsante per la visualizzazione della pagina Document-Show, specificando l'etichetta e l'azione che deve eseguire}
{Il sistema contiene la descrizione per la visualizzazione della pagina Document-Show}
\post{Il sistema contiene la descrizione per la visualizzazione della pagina Document-Show aggiornata con il nuovo pulsante aggiunto dall'utente sviluppatore}


\UCtitle
{Caso d'uso UC1.2.2}
{Modifica visualizzazione Collection}
\UCimmagine{UC1.2.2}{UC1.2.2 Modifica visualizzazione Collection}
\UC
{UC1.2.2}
{Utente sviluppatore}
{L'utente sviluppatore  può impostare la visualizzazione del menù delle Collection, delle pagine Collection-Index e Document-Show}
{Il sistema contiene la descrizione per la visualizzazione della Collection}
\scenario{
L'utente sviluppatore può impostare la visualizzazione del menù per le pagine di tipo Collection-Index (UC1.2.2.1);|
L'utente sviluppatore può impostare la visualizzazione della pagina Collection-Index (UC1.2.2.2);|
L'utente sviluppatore può impostare la visualizzazione della pagina Document-Show (UC1.2.2.3).
}
\post{Il sistema ha aggiornato la descrizione per la visualizzazione della Collection}

\UCtitle
{Caso d'uso UC1.2.2.1}
{Impostazioni visualizzazione menù}
\UCimmagine{UC1.2.2.1}{UC1.2.2.1 Impostazioni visualizzazione menù}
\UC
{UC1.2.2.1}
{Utente sviluppatore}
{L'utente sviluppatore può modificare, nella descrizione per la visualizzazione del menù della Collection, il nome della Collection e la posizione nel menù principale}
{Il sistema contiene la descrizione per la visualizzazione del menù e permette all'utente sviluppatore di modificarla}
\scenario{
L'utente può modificare il nome della Collection (UC1.2.2.1.1);|
L'utente può modificare la posizione della Collection (UC1.2.2.1.2).
}
\post{Il sistema ha aggiornato la descrizione per la visualizzazione del menù con le modifiche apportate dall'utente sviluppatore}


\UCtitle
{Caso d'uso UC1.2.2.1.1}
{Modifica nome}
\UC
{UC1.2.2.1}
{Utente sviluppatore}
{L'utente può modificare, nella descrizione per la visualizzazione del menù della Collection, il nome della voce relativa alla Collection}
{Il sistema contiene la descrizione per la visualizzazione della pagina Collection-Index}
\post{Il sistema ha aggiornato la descrizione per la visualizzazione del menù delle Collection con il nuovo nome specificato dall'utente sviluppatore}

\UCtitle
{Caso d'uso UC1.2.2.1.2}
{Modifica posizione}
\UC
{UC1.2.2.1}
{Utente sviluppatore}
{L'utente può modificare, nella descrizione per la visualizzazione del menù della Collection, la posizione della voce relativa al nome della Collection}
{L'utente desidera cambiare la posizione della voce nel menù}
\post{Il sistema ha aggiornato la descrizione per la visualizzazione del menù delle Collection con la nuova posizione della voce specificata dall'utente sviluppatore}



\UCtitle
{Caso d'uso UC1.2.2.2}
{Impostazioni visualizzazione pagina Collection-Index}
\UCimmagine{UC1.2.2.2}{UC1.2.2.2 Impostazioni visualizzazione pagina Collection-Index}
\UC
{UC1.2.2.2}
{Utente sviluppatore}
{L'utente sviluppatore  può modificare varie impostazioni della descrizione per la visualizzazione della pagina Collection-Index}
{Il sistema contiene la descrizione per la visualizzazione della pagina Collection-Index e permette all'utente sviluppatore di modificarla}
\scenario{
L'utente può aggiungere una chiave da visualizzare (UC1.2.1.2.1);|
L'utente può eliminare una chiave da visualizzare (UC1.2.2.2.1);|
L'utente può definire un ordinamento per chiave (UC1.2.1.2.2);|
L'utente può eliminare un ordinamento per chiave (UC1.2.2.2.2);|
L'utente può aggiungere il numero massimo di Document da visualizzare per pagina (UC1.2.1.2.3);|
L'utente può modificare il numero massimo di Document da visualizzare per pagina (UC1.2.2.2.3);|
L'utente può eliminare il numero massimo di Document da visualizzare per pagina (UC1.2.2.2.4);|
L'utente può aggiungere un pulsante (UC1.2.1.2.4);|
L'utente può eliminare un pulsante (UC1.2.2.2.5).
}
\post{Il sistema contiene la descrizione aggiornata per la visualizzazione della Collection}

\UCtitle
{Caso d'uso UC1.2.2.2.1}
{Elimina chiave}
\UC
{UC1.2.2.2}
{Utente sviluppatore}
{L'utente sviluppatore  può eliminare una chiave dalla descrizione per la visualizzazione della pagina Collection-Index}
{Il sistema contiene la descrizione per la visualizzazione della pagina Collection-Index, la quale contiene almeno una chiave da visualizzare}
\post{Il sistema contiene la descrizione per la visualizzazione della pagina Collection-Index aggiornata con la chiave eliminata dall'utente sviluppatore}


\UCtitle
{Caso d'uso UC1.2.2.2.2}
{Elimina ordinamento}
\UC
{UC1.2.2.2}
{Utente sviluppatore}
{L'utente sviluppatore  può eliminare un ordinamento rispetto a una chiave dalla descrizione per la visualizzazione della pagina Collection-Index}
{Il sistema contiene la descrizione per la visualizzazione della pagina Collection-Index, la quale contiene un ordinamento rispetto una chiave definito dall'utente sviluppatore}
\post{Il sistema contiene la descrizione per la visualizzazione della pagina Collection-Index aggiornata con l'ordinamento eliminato dall'utente sviluppatore}

\UCtitle
{Caso d'uso UC1.2.2.2.3}
{Modifica numero massimo di Document}
\UC
{UC1.2.2.2}
{Utente sviluppatore}
{L'utente sviluppatore  può modificare il numero massimo di Document da visualizzare per la pagina Collection-Index dalla descrizione per la visualizzazione della pagina Collection-Index}
{Il sistema contiene la descrizione per la visualizzazione della pagina Collection-Index, nella quale è presente un numero massimo di Document definito dall'utente}
\post{Il sistema contiene la descrizione per la visualizzazione della pagina Collection-Index aggiornata con il numero massimo di Document per pagina modificato dall'utente sviluppatore}

\UCtitle
{Caso d'uso UC1.2.2.2.4}
{Elimina numero massimo di Document}
\UC
{UC1.2.2.2}
{Utente sviluppatore}
{L'utente sviluppatore  può eliminare il numero massimo di Document da visualizzare dalla descrizione per la visualizzazione della pagina Collection-Index}
{Il sistema contiene la descrizione per la visualizzazione della pagina Collection-Index, nella quale è presente un numero massimo di Document definito dall'utente}
\post{Il sistema contiene la descrizione per la visualizzazione della pagina Collection-Index aggiornata con il numero massimo di Document per pagina eliminato dall'utente sviluppatore}

\UCtitle
{Caso d'uso UC1.2.2.2.5}
{Elimina pulsante}
\UC
{UC1.2.2.2}
{Utente sviluppatore}
{L'utente sviluppatore  può eliminare un pulsante da visualizzare nella pagina Collection-Index}
{Il sistema contiene la descrizione per la visualizzazione della pagina Collection-Index, nella quale è presente un pulsante definito dall'utente}
\post{Il sistema contiene la descrizione per la visualizzazione della pagina Collection-Index aggiornata con il pulsante eliminato dall'utente sviluppatore}


\UCtitle
{Caso d'uso UC1.2.2.3}
{Impostazioni visualizzazione pagina Document-Show}
\UCimmagine{UC1.2.2.3}{UC1.2.2.3 Impostazioni visualizzazione pagina Document-Show}
\UC
{UC1.2.2.3}
{Utente sviluppatore}
{L'utente sviluppatore  può modificare la descrizione per la visualizzazione della pagina Document-Show e varie specifiche al suo interno}
{Il sistema contiene la descrizione per la visualizzazione della pagina Document-Show}
\scenario{
L'utente può aggiungere una chiave da visualizzare (UC1.2.1.3.1);|
L'utente può eliminare una chiave da visualizzare (UC1.2.2.3.1);|
L'utente può aggiungere un pulsante (UC1.2.1.3.2);|
L'utente può eliminare un pulsante (UC1.2.2.3.2).
}
\post{Il sistema contiene la descrizione aggiornata per la visualizzazione della Collection}


\UCtitle
{Caso d'uso UC1.2.2.3.1}
{Elimina chiave}
\UC
{UC1.2.2.3}
{Utente sviluppatore}
{L'utente sviluppatore  può eliminare una chiave  dalla descrizione per la visualizzazione della pagina Document-Show}
{Il sistema contiene la descrizione per la visualizzazione della pagina Document-Show, la quale contiene almeno una chiave da visualizzare}
\post{Il sistema contiene la descrizione per la visualizzazione della pagina Document-Show aggiornata con la chiave eliminata dall'utente sviluppatore}

\UCtitle
{Caso d'uso UC1.2.2.3.2}
{Elimina pulsante}
\UC
{UC1.2.2.3}
{Utente sviluppatore}
{L'utente sviluppatore può eliminare un pulsante dalla descrizione per la visualizzazione della pagina Document-Show}
{Il sistema contiene la descrizione per la visualizzazione della pagina Document-Show, la quale contiene almeno una chiave da visualizzare}
\post{Il sistema contiene la descrizione per la visualizzazione della pagina Document-Show aggiornata il pulsante eliminato dall'utente sviluppatore}


\UCtitle
{Caso d'uso UC1.3}
{Modifica file di configurazione}
\UCimmagine{UC1.3}{UC1.3 Modifica file di configurazione}
\UC
{UC1.3}
{Utente sviluppatore}
{L'utente sviluppatore  può modificare varie impostazioni dei file di configurazione}
{Il sistema ha creato lo scheletro del progetto e permette all'utente sviluppatore di modificare i file di configurazione}
\scenario{
L'utente sviluppatore può abilitare la registrazione alle pagine web generate da MaaP (UC1.3.1);|
L'utente sviluppatore può abilitare la creazione di Document (UC1.3.2);|
L'utente sviluppatore può modificare i template delle pagine web (UC1.3.3);|
L'utente sviluppatore può specificare i dati per la connessione al database di analisi (UC1.3.4);|
L'utente sviluppatore può abilitare la creazione di indici (UC1.3.5).
}
\post{Il sistema ha aggiornato il file di configurazione}

\UCtitle
{Caso d'uso UC1.3.1}
{Abilitazione registrazione}
\UC
{UC1.3}
{Utente sviluppatore}
{L'utente sviluppatore può abilitare o disabilitare la registrazione alle pagine web generate da MaaP}
{Il sistema contiene il file di configurazione con l'abilitazione della registrazione predefinita}
\post{Il sistema ha aggiornato il file di configurazione con la scelta dell'utente sviluppatore}

\UCtitle
{Caso d'uso UC1.3.2}
{Abilitazione creazione Document}
\UC
{UC1.3}
{Utente sviluppatore}
{L'utente sviluppatore può abilitare o disabilitare la creazione di Document nel database di analisi}
{Il sistema contiene il file di configurazione con l'abilitazione della creazione di Document nel database di analisi}
\post{Il sistema ha aggiornato il file di configurazione con la scelta dell'utente sviluppatore}

\UCtitle
{Caso d'uso UC1.3.3}
{Modifica template pagine web}
\UC
{UC1.3}
{Utente sviluppatore}
{L'utente sviluppatore pu\'o modificare i template delle pagine web}
{Il sistema contiene le directory di descrizione delle pagine web}
\post{Il sistema ha aggiornato i template delle pagine web con le modifiche apportate dall'utente sviluppatore}

\UCtitle
{Caso d'uso UC1.3.4}
{Specifica database di analisi}
\UC
{UC1.3}
{Utente sviluppatore}
{L'utente sviluppatore può specificare i dati necessari alla connessione del database di analisi}
{Il sistema contiene il file di configurazione per la connessione a un database di analisi}
\post{Il sistema contiene le informazioni per connettersi a un database di analisi}

\UCtitle
{Caso d'uso UC1.3.5}
{Abilitazione creazione indici}
\UC
{UC1.3}
{Utente sviluppatore}
{L'utente sviluppatore può abilitare o disabilitare la creazione di indici nelle pagine web generate da MaaP}
{Il sistema contiene il file di configurazione con l'abilitazione per la creazione di indici}
\post{Il sistema ha aggiornato il file di configurazione con la scelta dell'utente sviluppatore}

\UCtitle
{Caso d'uso UC1.4}
{Utilizzo editor specializzato}
\UCimmagine{UC1.4}{UC1.4 Utilizzo editor specializzato}
\UC
{UC1.4}
{Utente sviluppatore}
{L'utente sviluppatore può utilizzare un editor specializzato per scrivere il file di descrizione}
{Il sistema ha creato lo scheletro del progetto e contiene un editor specializzato che può essere utilizzato dall'utente sviluppatore}
\scenario
{L'utente sviluppatore può creare un file di descrizione (UC1.4.1);|
L'utente sviluppatore può modificare un file di descrizione esistente (UC1.4.2);
}
\post{Il sistema consente all'utente sviluppatore di scrivere con l'editor specializzato}

\UCtitle
{Caso d'uso UC1.4.1}
{Creazione file di descrizione}
\UCimmagine{UC1.4.1}{UC1.4.1 Creazione file di descrizione}
\UC
{UC1.4.1}
{Utente sviluppatore}
{L'utente sviluppatore può utilizzare un editor specializzato per creare un file di descrizione}
{Il sistema ha creato lo scheletro del progetto}
\scenario
{L'utente sviluppatore può scrivere un file di descrizione (UC1.4.1.1);|
L'utente sviluppatore può salvare il file di descrizione (UC1.4.1.2);
}
\post{Il sistema ha ottenuto il contenuto relativo al file di descrizione che l'utente sviluppatore ha scritto ed ha salvato il file di descrizione in modo permanente}

\UCtitle
{Caso d'uso UC1.4.1.1}
{Scrittura file di descrizione}
\UC
{UC1.4.1}
{Utente sviluppatore}
{L'utente sviluppatore può utilizzare un editor specializzato per scrivere un file di descrizione}
{Il sistema propone all'utente una schermata per scrivere il file di descrizione}
\post
{Il sistema conosce il contenuto del file scritto dall'utente sviluppatore}

\UCtitle
{Caso d'uso UC1.4.1.2}
{Salvataggio file di descrizione}
\UC
{UC1.4.1}
{Utente sviluppatore}
{L'utente sviluppatore può salvare in modo permanente il file di descrizione scritto precedentemente}
{Il sistema conosce il contenuto del file scritto dall'utente sviluppatore}
\post
{Il sistema ha salvato il file di descrizione in modo permanente}

\UCtitle
{Caso d'uso UC1.4.2}
{Modifica file di descrizione}
\UCimmagine{UC1.4.2}{UC1.4.2 Modifica file di descrizione}
\UC
{UC1.4.2}
{Utente sviluppatore}
{L'utente sviluppatore può modificare un file di descrizione presente nel sistema}
{Il sistema ha creato lo scheletro del progetto e l'utente sviluppatore ha selezionato un file di descrizione da modificare}
\scenario
{L'utente sviluppatore può modificare il codice del file di descrizione (UC1.4.2.1);|
L'utente sviluppatore può salvare le modifiche apportate al file di descrizione (UC1.4.2.2);
}
\estensioni
{L'utente sviluppatore può in qualsiasi momento annullare le modifiche apportate al codice del file di descrizione selezionato. (UC1.4.2.3)
}
\post{Il sistema ha ottenuto le informazione relative alle azioni che l'utente sviluppatore desidera eseguire. Se l'utente sviluppatore ha scelto di salvare le modifiche apportate, il sistema salva tali informazioni in modo permanente sovrascrivendo il file di descrizione originario}

\UCtitle
{Caso d'uso UC1.4.2.1}
{Modifica del codice del file di descrizione}
\UC
{UC1.4.2}
{Utente sviluppatore}
{L'utente sviluppatore può modificare il codice del file di descrizione}
{Il sistema propone all'utente una schermata per modificare il codice del file di descrizione}
\post
{Il sistema conosce il contenuto del testo scritto dall'utente sviluppatore}

\UCtitle
{Caso d'uso UC1.4.2.2}
{Salvataggio delle modifiche del file di descrizione}
\UC
{UC1.4.2}
{Utente sviluppatore}
{L'utente sviluppatore può modificare in modo permanente le modifiche apportate al file di descrizione}
{Il sistema conosce il contenuto del testo scritto dall'utente sviluppatore}
\post
{Il sistema ha salvato in modo permanente le modifiche apportate al file di descrizione, sovrascrivendo il file di descrizione originario}

\UCtitle
{Caso d'uso UC1.4.2.3}
{Annullamento delle modifiche}
\UC
{UC1.4.2}
{Utente sviluppatore}
{L'utente sviluppatore può annullare le modifiche apportate al contenuto del file di descrizione}
{Il sistema conosce il contenuto del testo scritto dall'utente sviluppatore}
\post
{Il sistema ignora le modifiche apportate al contenuto del file di descrizione, e non sovrascrive il file originario}

\UCtitle
{Caso d'uso UC1.5}
{Inserimento file di descrizione}
\UC
{UC1}
{Utente sviluppatore}
{L'utente sviluppatore può inserire il file di descrizione nel progetto}
{Il sistema ha creato lo scheletro di un nuovo progetto}
\post{Il sistema contiene il file di descrizione}




\newpage

\UCtitle
{Caso d'uso UC2}
{MaaP's Web}
\UCimmagine{UC2}{UC2 - Azioni utente \gloss{business}}
\UC
{UC2}
{\gloss{Utente business}, \gloss{Utente business autenticato}}
{L'utente business attraverso il \gloss{browser} visualizza la pagina di autenticazione dove può accedere, recuperare la \gloss{password} o registrarsi, mentre l'utente business autenticato può aprire una Collection, disconnettersi o gestire il \gloss{profilo}}
{Il sistema ha creato lo scheletro del progetto, la pagina di autenticazione viene proposta all'utente business mediante il browser}
\scenario{
L'utente business può registrarsi (UC2.1);|
L'utente business può autenticarsi (UC2.2);|
L'utente business può recuperare la password (UC2.3);|
L'utente business autenticato può aprire una Collection (UC2.4);|
L'utente business autenticato può disconnettersi dal sistema (UC2.5);|
L'utente business autenticato può gestire il profilo (UC2.6).
}
\scenarioAlt{
Nel caso in cui le credenziali di accesso fornite dall'utente business non fossero corrette il sistema segnala il problema e ripresenta il \gloss{form} di autenticazione.
}
\post
{Il sistema ha memorizzato le informazioni inserite dall'utente business per l'autenticazione o per il recupero password e le operazioni che l'utente business autenticato può effettuare}



\UCtitle
{Caso d'uso UC2.1}
{Registrazione}
\UCimmagine{UC2.1}{UC2.1 Registrazione}
\UC
{UC2.1}
{Utente business}
{L'utente business può registrarsi}
{Il sistema si trova nello stato iniziale e propone all'utente una schermata per registrarsi}
\scenario
{
L'utente business può inserire l'email (UC2.1.1);|
L'utente business può inserire la password (UC2.1.2).
}
\post
{Il sistema ha creato un nuovo utente e reputa l'utente business autenticato}

\UCtitle
{Caso d'uso UC2.1.1}
{Inserimento email}
\UC
{UC2.1}
{Utente business}
{L'utente business può inserire l'email per la registrazione}
{Il sistema si trova nello stato iniziale e propone all'utente una casella per inserire l'email}
\post
{Il sistema ha ottenuto il contenuto del testo inserito dall'utente business}

\UCtitle
{Caso d'uso UC2.1.2}
{Inserimento password}
\UC
{UC2.1}
{Utente business}
{L'utente business può inserire la password per la registrazione}
{Il sistema si trova nello stato iniziale e propone all'utente una casella per inserire la password}
\post
{Il sistema ha ottenuto il contenuto del testo inserito dall'utente business}

\UCtitle
{Caso d'uso UC2.2}
{Autenticazione}
\UCimmagine{UC2.2}{UC2.2 Autenticazione}
\UC
{UC2.2}
{Utente business}
{Un utente business tenta di autenticarsi}
{Il sistema si trova nello stato iniziale e propone all'utente una schermata per inserire i dati per l'autenticazione}
\scenario
{
L'utente business può inserire l'email (UC2.2.1);|
L'utente business può inserire la password (UC2.2.2).
}
\post
{Il sistema reputa l'utente business come autenticato}

\UCtitle
{Caso d'uso UC2.2.1}
{Inserimento email}
\UC
{UC2.2}
{Utente business}
{L'utente business può inserire l'email per l'autenticazione}
{Il sistema si trova nello stato iniziale e propone all'utente una casella per inserire l'email}
\post
{Il sistema ha ottenuto il contenuto del testo inserito dall'utente business}

\UCtitle
{Caso d'uso UC2.2.2}
{Inserimento password}
\UC
{UC2.2}
{Utente business}
{L'utente business può inserire la password per l'autenticazione}
{Il sistema si trova nello stato iniziale e propone all'utente una casella per inserire la password}
\post
{Il sistema ha ottenuto il contenuto del testo inserito dall'utente business}

\UCtitle
{Caso d'uso UC2.3}
{Recupero password}
\UCimmagine{UC2.3}{UC2.3 Recupero password}
\UC
{UC2.3}
{Utente business}
{Un utente business registrato recupera la password dimenticata}
{Il sistema  reputa l'utente business come registrato}
\scenario
{L'utente business inserire l'email (UC2.3.1).
}
\post
{Il sistema ha fornito la password all'utente business}

\UCtitle
{Caso d'uso UC2.3.1}
{Inserimento email}
\UC
{UC2.3}
{Utente business}
{L'utente business può inserire l'email per il recupero password}
{Il sistema propone all'utente una casella per inserire l'email}
\post
{Il sistema ha ottenuto il contenuto del testo inserito dall'utente business}


\UCtitle
{Caso d'uso UC2.4}
{Apertura Collection}
\UCimmagine{UC2.4}{UC2.4 - Apertura Collection}
\UC
{UC2.4}
{Utente business autenticato, Utente business autenticato amministratore}
{L’ utente business autenticato può effettuare diverse operazioni: visualizzare una pagina Document-Show, modificare la visualizzazione dei Document, visualizzare una diversa Collection, disconnettersi dal sistema, navigare tra le pagine della pagina Collection-Index correntemente visualizzata,  gestire il profilo. L'utente business autenticato \gloss{amministratore} potrà inoltre cancellare o modificare un Document}
{Il sistema reputa l'utente business come autenticato e visualizza la Collection}
\scenario{
L'utente business autenticato può visualizzare pagine Document-Show (UC2.4.1);|
L'utente business autenticato può modificare la visualizzazione dei Document (UC2.4.2);|
L'utente business autenticato può aprire una diversa Collection (UC2.4);|
L'utente business autenticato può disconnettersi (UC2.5);|
L'utente business autenticato può navigare tra le pagine della Collection-Index attualmente visualizzata (UC2.4.3);|
L'utente business autenticato può gestire il profilo (UC2.6);|
L'\gloss{utente business autenticato amministratore} può cancellare un Document (UC2.4.4);|
L'utente business autenticato amministratore può modificare un Document (UC2.4.5);|
L'utente business autenticato amministratore può visualizzare le \gloss{query} più frequenti che il sistema gli propone (UC2.4.6);|
L'utente business autenticato amministratore può creare degli indici sulle query più frequenti visualizzare (UC2.4.7).
}
\post
{Il sistema ha ottenuto le informazioni sulle operazioni che l'utente business autenticato o l'utente business autenticato amministratore desidera eseguire}



\UCtitle
{Caso d'uso UC2.4.1}
{Visualizzazione pagina Document-Show}
\UC
{UC2.4}
{Utente business autenticato}
{L'utente business autenticato visualizza la pagina Document-Show relativa alla chiave selezionata}
{Il sistema reputa l'utente business come autenticato e visualizza la Collection scelta}
\post
{Viene visualizzata la pagina Document-Show relativa alla chiave selezionata}


\UCtitle
{Caso d'uso UC2.4.2}
{Modifica visualizzazione dei Document}
\UCimmagine{UC2.4.2}{UC2.4.2 - Modifica visualizzazione dei Document}
\UC
{UC2.4.2}
{Utente business autenticato}
{L'utente business autenticato può cambiare la modalità di visualizzazione dei Document}
{Il sistema reputa l'utente business come autenticato e visualizza la Collection scelta}
\scenario{
L'utente business autenticato può selezionare i criteri per la visualizzazione della lista di Document (UC2.4.2.1);|
L'utente business autenticato può applicare dei filtri (UC2.4.2.2);|
L'utente business autenticato può annullare un filtro applicato (UC2.4.2.3).
}
\post
{Il sistema ha ottenuto le informazioni sulle operazioni che l'utente business autenticato desidera eseguire ed aggiorna la visualizzazione della pagina}

\UCtitle
{Caso d'uso UC2.4.2.1}
{Seleziona criteri per la visualizzazione}
\UCimmagine{UC2.4.2.1}{UC2.4.2.1 - Seleziona criteri per la visualizzazione}
\UC
{UC2.4.2.1}
{Utente business autenticato}
{L'utente business autenticato può selezionare dei criteri per cambiare la visualizzazione della lista di Document: cambiare il numeri di Document da visualizzare per pagina o cambiare l'ordinamento dei Document secondo una chiave}
{Il sistema reputa l'utente business come autenticato e visualizza la Collection scelta}
\scenario{
L'utente business autenticato può cambiare il numero di Document da visualizzare per pagina (UC2.4.2.1.2);|
L'utente business autenticato può cambiare l'ordinamento dei Document secondo una chiave (UC2.4.2.1.1).
}
\post
{Il sistema ha ottenuto le informazioni sulle operazioni che l'utente business autenticato desidera eseguire ed aggiorna la visualizzazione della pagina}


\UCtitle
{Caso d'uso UC2.4.2.1.1}
{Cambio ordinamento per chiave}
\UC
{UC2.4.2.1}
{Utente business autenticato}
{L'utente business autenticato può cambiare l'ordinamento dei documenti visualizzati secondo una chiave}
{Il sistema reputa l'utente business come autenticato e visualizza la Collection scelta}
\post
{Il sistema aggiorna la visualizzazione della pagina secondo i criteri imposti dall'utente business autenticato}


\UCtitle
{Caso d'uso UC2.4.2.1.2}
{Cambio numero di Document da visualizzare per pagina}
\UC
{UC2.4.2.1}
{Utente business autenticato}
{L'utente business autenticato può cambiare il numero di Document da visualizzare per pagina}
{Il sistema reputa l'utente business come autenticato e visualizza la Collection scelta}
\post
{Il sistema aggiorna la visualizzazione del numero di Document per pagina secondo i criteri imposti dall'utente business autenticato}

\UCtitle
{Caso d'uso UC2.4.2.2}
{Applicazione filtro}
\UC
{UC2.4.2}
{Utente business autenticato}
{L'utente business autenticato può applicare un filtro disponibile nella pagina Collection-Index}
{Il sistema reputa l'utente business come autenticato e visualizza la Collection scelta}
\post
{Il sistema aggiorna la visualizzazione della pagina secondo i criteri imposti dal filtro applicato}

\UCtitle
{Caso d'uso UC2.4.2.3}
{Annullare filtro applicato}
\UC
{UC2.4.2}
{Utente business autenticato}
{L'utente business autenticato ha la possibilità di rimuovere un filtro precedentemente applicato}
{Il sistema ha applicato un filtro secondo i criteri imposti dall'utente business autenticato}
\post
{Il sistema aggiorna la visualizzazione della Collection allo stato precedente all'applicazione del filtro}


\UCtitle
{Caso d'uso UC2.4.3}
{Navigazione tra le pagine della Collection-Index}
\UC
{UC2.4}
{Utente business autenticato}
{L'utente business autenticato può navigare tra le pagine della Collection-Index attualmente visualizzata}
{Il sistema reputa l'utente business come autenticato e visualizza la Collection scelta}
\post
{Il sistema visualizza un'altra Collection scelta dall'utente business autenticato}

\UCtitle
{Caso d'uso UC2.4.4}
{Cancellazione Document}
\UC
{UC2.4}
{Utente business autenticato amministratore}
{L'utente business autenticato come amministratore può cancellare un Document presente nella Collection attualmente visualizzata}
{Il sistema reputa l'utente business autenticato come amministratore e visualizza la Collection scelta}
\post
{Il sistema ha cancellato il Document selezionato dall'utente business autenticato amministratore ed aggiorna la pagina attualmente visualizzata}



\UCtitle
{Caso d'uso UC2.4.5}
{Modifica Document}
\UCimmagine{UC2.4.5}{UC2.4.5 - Modifica Document}
\UC
{UC2.4.5}
{Utente business autenticato amministratore}
{L'utente business autenticato come amministratore può modificare un Document presente nella Collection attualmente visualizzata}
{Il sistema reputa l'utente business autenticato come amministratore e visualizza la Collection scelta}
\scenario{
L'utente business autenticato amministratore può modificare il valore di ciascuna chiave (UC2.4.5.1);|
L'utente business autenticato amministratore può salvare il Document appena modificato (UC2.4.5.2).
}
\estensioni
{L'utente business autenticato amministratore può annullare le modifiche fino a quel momento apportate (UC2.4.5.3).}
\post
{Il sistema modifica il Document selezionato secondo le specifiche imposte dall'utente business autenticato amministratore}


\UCtitle
{Caso d'uso UC2.4.5.1}
{Modifica valore della chiave}
\UC
{UC2.4.5}
{Utente business autenticato amministratore}
{L'utente business autenticato come amministratore può modificare il valore di ciascuna chiave del Document tramite form editabile}
{Il sistema reputa l'utente business autenticato come amministratore e visualizza la pagina Document-Show che l'utente business autenticato come amministratore desidera modificare}
\post
{Il sistema ha ottenuto un valore per ogni chiave del Document che l'utente business autenticato come amministratore ha modificato}

\UCtitle
{Caso d'uso UC2.4.5.2}
{Salvataggio Document}
\UC
{UC2.4.5}
{Utente business autenticato amministratore}
{L'utente business autenticato come amministratore può salvare le modifiche fino a quel momento apportate al Document}
{Il sistema ha ottenuto un valore per ogni chiave del Document che l'utente business autenticato come amministratore vuole modificare}
\post
{Il sistema salva le modifiche apportate al Document}

\UCtitle
{Caso d'uso UC2.4.5.3}
{Annulla modifiche Document}
\UC
{UC2.4.5}
{Utente business autenticato amministratore}
{L'utente business autenticato come amministratore ha la possibilità di annullare le modifiche apportate al Document (vedi UC2.4.5)}
{Il sistema ha ottenuto un valore per ogni chiave del Document che si sta modificando}
\post
{Il sistema non considera i valori inseriti per ogni chiave del Document e visualizza la pagina Collection-Index principale}

\UCtitle
{Caso d'uso UC2.4.6}
{Visualizzazione query più utilizzate}
\UC
{UC2.4}
{Utente business autenticato amministratore}
{Il sistema reputa l'utente business autenticato come amministratore}
{L'utente business autenticato amministratore può visualizzare le query più utilizzate che il sistema gli offre}
\post
{Il sistema mostra all'utente business autenticato amministratore le query più utilizzate}

\UCtitle
{Caso d'uso UC2.4.7}
{Creazione indici}
\UC
{UC2.4}
{Utente business autenticato amministratore}
{L'utente business autenticato amministratore può creare degli indici sulla base delle query più utilizzate, le quali vengono fornite dal sistema}
{Il sistema reputa l'utente business autenticato come amministratore}
\post
{Il sistema permette all'utente business autenticato amministratore di creare degli indici}

\UCtitle
{Caso d'uso UC2.5}
{Disconnessione}
\UC
{UC2}
{Utente business autenticato}
{L'utente può uscire dal proprio profilo e disconnettersi dal sistema}
{Il sistema reputa l'utente business come autenticato}
\post
{Il sistema reputa l'utente business come non autenticato}

\UCtitle
{Caso d'uso UC2.6}
{Gestione utente}
\UCimmagine{UC2.6}{UC2.6 - Gestione utente}
\UC
{UC2.6}
{Utente business autenticato}
{L'utente business autenticato può modificare i dati utente, mentre l'utente business autenticato amministratore può anche eseguire operazioni di creazione e cancellazione di un utente}
{Il sistema reputa l'utente business come autenticato}
\scenario{
L'utente business autenticato può gestire i dati (UC2.6.3);|
L'utente business autenticato amministratore può creare un nuovo utente (UC2.6.1);|
L'utente business autenticato amministratore può eliminare un utente (UC2.6.2).
}
\post
{Il sistema ha ottenuto le informazioni sulle operazioni che l'utente business desidera eseguire}

\UCtitle
{Caso d'uso UC2.6.1}
{Creazione nuovo utente}
\UC
{UC2.6}
{Utente business autenticato amministratore}
{L'utente business autenticato come amministratore può creare un nuovo utente}
{Il sistema reputa l'utente business autenticato come amministratore}
\post
{Il sistema ha creato un nuovo utente}


\UCtitle
{Caso d'uso UC2.6.2}
{Eliminazione utente}
\UC
{UC2.6}
{Utente business autenticato amministratore}
{L'utente business autenticato come amministratore può eliminare un profilo utente preesistente}
{Il sistema reputa l'utente business autenticato come amministratore}
\post
{Il sistema ha eliminato il profilo utente specificato}

\UCtitle
{Caso d'uso UC2.6.3}
{Gestione dati}
\UCimmagine{UC2.6.3}{UC2.6.3 - Gestione dati}
\UC
{UC2.6.3}
{Utente business autenticato}
{L'utente business autenticato può modificare i dati utente del proprio profilo e salvare le modifiche o annullarle. L'utente business autenticato come amministratore può modificare i dati dei vari profili utenti, modificare i permessi degli utenti, salvare le modifiche o annullarle}
{Il sistema reputa l'utente business come autenticato}
\scenario{
L'utente business autenticato può modificare i dati utente del proprio profilo (UC2.6.3.1);|
L'utente business autenticato può salvare le modifiche (UC2.6.3.2);|
L'utente business autenticato come amministratore può modificare i dati utenti (UC2.6.3.3);|
L'utente business autenticato come amministratore può modificare i permessi degli utenti (UC2.6.3.4).
}
\post
{Il sistema modifica i dati utente secondo le specifiche imposte dall'utente business autenticato}


\UCtitle
{Caso d'uso UC2.6.3.1}
{Modifica dati utente}
\UC
{UC2.6.3}
{Utente business autenticato}
{L'utente business autenticato può modificare i dati utente del proprio profilo, ovvero email e password}
{Il sistema reputa l'utente business come autenticato}
\post
{Il sistema ha ottenuto tutti i dati modificati del profilo utente}



\UCtitle
{Caso d'uso UC2.6.3.2}
{Salvataggio modifiche}
\UC
{UC2.6.3}
{Utente business autenticato}
{L'utente business autenticato può salvare le modifiche apportate durante la modifica dei dati utenti}
{Il sistema ha ottenuto i dati modificati del profilo utente}
\post
{Il sistema aggiorna il profilo utente con i nuovi dati inseriti}

\UCtitle
{Caso d'uso UC2.6.3.3}
{Modifica dati utenti}
\UC
{UC2.6.3}
{Utente business autenticato amministratore}
{L'utente business autenticato come amministratore può modificare i dati degli utenti, ovvero email e password}
{Il sistema reputa l'utente business autenticato come amministratore}
\estensioni
{L'utente business autenticato può annullare le modifiche fin'ora apportate (UC2.6.3.5).}
\post
{Il sistema ha ottenuto tutti i dati modificati del profilo utente}

\UCtitle
{Caso d'uso UC2.6.3.4}
{Modifica permessi utenti}
\UC
{UC2.6.3}
{Utente business autenticato amministratore}
{L'utente business autenticato come amministratore può modificare i permessi degli utenti, ovvero elevare un utente a livello amministratore o declassare un utente amministratore ad utente business autenticato}
{Il sistema reputa l'utente business autenticato come amministratore}
\post
{Il sistema ha ottenuto i nuovi permessi da applicare al profilo utente}


\UCtitle
{Caso d'uso UC2.6.3.5}
{Annulla modifiche}
\UC
{UC2.6.3}
{Utente business autenticato, Utente business autenticato amministratore}
{L'utente business autenticato e l'amministratore possono annullare le modifiche apportate}
{Il sistema ha ottenuto i dati modificati del profilo utente; se si tratta di amministratore il sistema ottiene i dati modificati dei profili e/o dei dati relativi ai permessi dell'utente}
\post
{Il sistema non considera le modifiche apportate e visualizza la pagina precedente}










\newpage

\UCtitle{Caso d'uso UC3} 
{MaaS}
\UCimmagine{UC3}{UC3 MaaS}

\UC
{UC3}
{Utente, Utente autenticato}
{L'Utente e l'Utente autenticato possono visualizzare le pagine web accedendo attraverso uno specifico URL, l'Utente può registrarsi, richiedere il recupero password, effettuare l'autenticazione. Se ritenuto autenticato dal sistema può inoltre creare lo scheletro di un nuovo progetto, gestire le pagine web, gestire il proprio profilo e disconnettersi}
{Il sistema MaaS è correttamente installato, funzionante e le pagine web sono disponibili online e raggiungibili dall'Utente e dall'Utente autenticato tramite browser. All'Utente viene proposta la pagina di login al sistema mentre all'Utente autenticato viene proposta la pagina del proprio account utente}

\scenario
{L'Utente autenticato può creare lo scheletro del progetto (UC3.1);|
L'Utente autenticato può  gestire le pagine web (UC3.2);|
L'Utente autenticato può  gestire il proprio profilo (UC3.3);|
L'Utente autenticato può  disconnettersi dal proprio account (UC3.4);|
L'Utente e l'Utente autenticato possono visualizzare le pagine web tramite URL specifico (UC3.5);|
L'Utente può autenticarsi al sistema (UC3.6);|
L'Utente può registrarsi (UC3.7);|
L'Utente può recuperare la password (UC3.8).
}

\post
{Il sistema ha ottenuto le informazioni relative alle azioni che l'Utente e l'Utente autenticato vogliono eseguire}


\UCtitle
{Caso d'uso UC3.1}  
{Creazione scheletro del progetto}	
\UCimmagine{UC3.1}{UC3.1 Creazione scheletro del progetto}
\UC	
{UC3.1}		
{Utente autenticato}
{L'Utente autenticato può creare lo scheletro di un nuovo progetto}
{Il sistema reputa l'Utente come autenticato}
\scenario
{L'Utente autenticato può inserire il nome del progetto che intende creare (UC3.1.1)}
\post
{Il sistema ha completato la creazione dello scheletro del progetto utilizzando il nome specificato dall'Utente autenticato. E' disponibile un URL specifico per accedere alla pagina principale del progetto appena creato}

\UCtitle
{Caso d'uso UC3.1.1}  
{Inserimento nome progetto}	
\UC	
{UC3.1}		
{Utente autenticato}
{L'utente può inserire il nome del progetto che intende creare}
{Il sistema ha iniziato la creazione dello scheletro del progetto}
\post
{Il sistema ha acquisito il nome del progetto che l'Utente autenticato desidera creare}

\UCtitle
{Caso d'uso UC3.2}  
{Gestione pagine web}
\UCimmagine{UC3.2}{UC3.2 Gestione pagine web}
\UC		
{UC3.2}		
{Utente autenticato}
{L'utente autenticato può gestire le pagine web presenti nel sistema}
{Il sistema reputa l'Utente come autenticato e precedentemente è stato creato con successo lo scheletro di un progetto}
\scenario
{L'Utente autenticato può creare un nuovo file di descrizione (UC3.2.1);|
L'Utente autenticato può eseguire l'upload di un file di descrizione precedentemente creato con il sistema MaaP (UC3.2.2);|
L'Utente autenticato può modificare un file di descrizione esistente (UC3.2.3);|
L'Utente autenticato può modificare il file di configurazione (UC3.2.4);|
L'Utente autenticato può visualizzare tutti i file di descrizione presenti (UC3.2.5).
}
\post
{Il sistema ha ottenuto le informazioni relative alle azioni che l'Utente autenticato vuole eseguire}


\UCtitle
{Caso d'uso UC3.2.1}  
{Creazione file di descrizione}	
\UCimmagine{UC3.2.1}{UC3.2.1 Creazione file di descrizione}
\UC	
{UC3.2.1}		
{Utente autenticato}
{L'utente autenticato può creare un file di descrizione utilizzando un apposito editor online e successivamente salvarlo nel sistema}
{Il sistema ha generato lo scheletro del progetto}
\scenario
{L'Utente autenticato può scrivere un file di descrizione (UC3.2.1.1);|
L'Utente autenticato può salvare il file di descrizione (UC3.2.1.2).
}
\post
{Il sistema ha ottenuto il contenuto relativo al file di descrizione che l'Utente autenticato ha scritto ed ha salvato il file di descrizione in modo permanente}

\UCtitle
{Caso d'uso UC3.2.1.1}  
{Scrittura file di descrizione}	
\UC	
{UC3.2.1}		
{Utente autenticato}
{L'utente autenticato può scrivere il contenuto del file di descrizione che desidera creare in un apposito form}
{Il sistema propone all'Utente autenticato un apposita casella di testo per scrivere il contenuto del file di descrizione da creare ed è in attesa dell'inserimento}
\post
{Il sistema conosce il contenuto del testo inserito dall'Utente}

\UCtitle
{Caso d'uso UC3.2.1.2}  
{Salvataggio file di descrizione}
\UC		
{UC3.2.1}		
{Utente autenticato}
{L'utente autenticato può salvare in modo permanente il contenuto del file di descrizione scritto precedentemente}
{Il sistema conosce il contenuto del testo inserito dall'Utente}
\post
{Il sistema ha ottenuto il contenuto relativo al file di descrizione che l'Utente autenticato ha scritto ed ha salvato il file di descrizione in modo permanente}

\UCtitle
{Caso d'uso UC3.2.2}  
{Upload del file di descrizione}
\UCimmagine{UC3.2.2}{UC3.2.2 Upload del file di descrizione}
\UC		
{UC3.2.2}		
{Utente autenticato}
{L'utente autenticato può eseguire l'upload di un file di descrizione precedentemente creato con il sistema MaaP}
{Il sistema ha generato lo scheletro del progetto}
\scenario
{L'Utente autenticato può navigare nel filesystem (UC3.2.2.1);|
L'Utente autenticato può selezionare un file di descrizione (UC3.2.2.2);|
L'Utente autenticato può confermare l'upload del file selezionato (UC3.2.2.3).
}
\post
{Il sistema ha ottenuto le informazioni relative al file di descrizione che l'Utente autenticato desidera caricare, ha eseguito l'upload ed ha salvato il file in modo permanente}

\UCtitle
{Caso d'uso UC3.2.2.1}  
{Navigazione nel filesystem}
\UC		
{UC3.2.2}		
{Utente autenticato}
{L'utente autenticato può navigare nel filesystem per selezionare la cartella contenente il file che vuole caricare}
{Il sistema è in attesa che l'Utente autenticato selezioni una cartella all'interno del filesystem}
\post
{Il sistema ha modificato la cartella corrente riflettendo la selezione dell'Utente autenticato}

\UCtitle
{Caso d'uso UC3.2.2.2}  
{Selezione di un file di descrizione}	
\UC	
{UC3.2.2}		
{Utente autenticato}
{L'utente autenticato può selezionare il file che desidera caricare nel sistema}
{Il sistema mostra i file presenti nella cartella selezionata dall'Utente autenticato}
\post
{Il sistema evidenzia il file indicato dall'Utente autenticato}

\UCtitle
{Caso d'uso UC3.2.2.3}  
{Conferma upload del file selezionato}	
\UC	
{UC3.2.2}		
{Utente autenticato}
{L'utente autenticato può confermare il caricamento del file selezionato}
{Il sistema ha un file selezionato pronto per essere caricato}
\post
{Il sistema ha letto il file selezionato dall'Utente autenticato, ha eseguito l'upload e lo ha salvato in modo permanente}

\UCtitle
{Caso d'uso UC3.2.3}  
{Modifica di un file di descrizione}
\UCimmagine{UC3.2.3}{UC3.2.3 Modifica di un file di descrizione}
\UC		
{UC3.2.3}		
{Utente autenticato}
{L'utente autenticato può modificare un file di descrizione presente nel sistema}
{Il sistema ha creato lo scheletro del progetto e l'Utente autenticato ha selezionato il file di descrizione che desidera modificare}
\scenario
{L'Utente autenticato può modificare il codice del file di descrizione selezionato (UC3.2.3.1);|
L'Utente autenticato può salvare le modifiche apportate al codice del file di descrizione selezionato (UC3.2.3.2).
}
\estensioni
{L'Utente autenticato può in qualsiasi momento annullare le modifiche apportate al file di descrizione selezionato (UC3.2.3.3).}
\post
{Il sistema ha ottenuto le informazioni relative alle azioni che l'Utente autenticato desidera eseguire. Nello specifico se l'Utente autenticato ha scelto di salvare le modifiche apportate, il sistema ha ottenuto le informazioni relative al contenuto del file di descrizione modificato ed ha salvato tali informazioni in modo permanente sovrascrivendo il file di descrizione originario}

\UCtitle
{Caso d'uso UC3.2.3.1}  
{Modifica del codice del file di descrizione selezionato}
\UC		
{UC3.2.3}		
{Utente autenticato}
{L'utente autenticato può modificare il codice del file di descrizione selezionato}
{Il sistema propone all'Utente autenticato un'apposita casella di testo con all'interno il codice contenuto nel file di descrizione che desidera modificare ed è in attesa dell'inserimento}
\post
{Il sistema conosce il contenuto del testo inserito dall'Utente}

\UCtitle
{Caso d'uso UC3.2.3.2}  
{Salvataggio delle modifiche al file di descrizione}	
\UC	
{UC3.2.3}		
{Utente autenticato}
{L'utente autenticato può salvare in modo permanente le modifiche apportate al file di descrizione}
{Il sistema conosce il contenuto del testo inserito dall'Utente}
\post
{Il sistema ha ottenuto le informazioni relative al contenuto del file di descrizione modificato ed ha salvato la nuova versione del file sovrascrivendo il file originario}

\UCtitle
{Caso d'uso UC3.2.3.3}  
{Annullamento delle modifiche}	
\UC	
{UC3.2.3}		
{Utente autenticato}
{L'utente autenticato può annullare le modifiche apportate al contenuto del file di descrizione}
{Il sistema ha ottenuto le informazioni sul contenuto del testo inserito dall'Utente}
\post
{Il sistema ignora le modifiche apportate dall'Utente autenticato e non sovrascrive il file originario}


\UCtitle
{Caso d'uso UC3.2.4}  
{Modifica di un file di configurazione}
\UCimmagine{UC3.2.4}{UC3.2.4 Modifica di un file di configurazione}
\UC		
{UC3.2.4}		
{Utente autenticato}
{L'utente autenticato può modificare il file di configurazione presente nel sistema}
{Il sistema ha creato lo scheletro del progetto}
\scenario
{L'Utente autenticato può modificare il codice del file di configurazione (UC3.2.4.1);|
L'Utente autenticato può salvare le modifiche apportate al codice del file di configurazione (UC3.2.4.2).
}
\estensioni
{L'Utente autenticato può in qualsiasi momento annullare le modifiche apportate al file di configurazione (UC3.2.4.3).}
\post
{Il sistema ha ottenuto le informazioni relative alle azioni che l'Utente autenticato desidera eseguire. Nello specifico se l'Utente autenticato ha scelto di salvare le modifiche apportate, il sistema ha ottenuto le informazioni relative al contenuto del file di configurazione modificato ed ha salvato tali informazioni in modo permanente sovrascrivendo il file di configurazione originario}

\UCtitle
{Caso d'uso UC3.2.4.1}  
{Modifica del codice del file di configurazione}
\UC		
{UC3.2.4}		
{Utente autenticato}
{L'utente autenticato può modificare il codice del file di configurazione}
{Il sistema propone all'Utente autenticato un'apposita casella di testo con all'interno il codice contenuto nel file di configurazione da modificare ed è in attesa dell'inserimento}
\post
{Il sistema ha ottenuto le informazioni sul contenuto del testo modificato dall'Utente}

\UCtitle
{Caso d'uso UC3.2.4.2}  
{Salvataggio delle modifiche al file di configurazione}	
\UC	
{UC3.2.4}		
{Utente autenticato}
{L'utente autenticato può salvare in modo permanente le modifiche apportate al file di configurazione}
{Il sistema ha ottenuto le informazioni sul contenuto del testo modificato dall'Utente}
\post
{Il sistema ha ottenuto le informazioni relative al contenuto del file di configurazione modificato ed ha salvato la nuova versione del file sovrascrivendo il file originario}

\UCtitle
{Caso d'uso UC3.2.4.3}  
{Annullamento delle modifiche}	
\UC	
{UC3.2.4}		
{Utente autenticato}
{L'utente autenticato può annullare le modifiche apportate al contenuto del file di configurazione}
{Il sistema ha ottenuto le informazioni sul contenuto del testo modificato dall'Utente}
\post
{Il sistema ignora le modifiche apportate dall'Utente autenticato}

\UCtitle
{Caso d'uso UC3.2.5}  
{Visualizza tutti i file di descrizione}	
\UC	
{UC3.2}		
{Utente autenticato}
{L'utente autenticato può visualizzare tutti i file di descrizione presenti nel sistema}
{Il sistema ha generato lo scheletro del progetto}
\post
{Il sistema propone all'Utente autenticato una lista di tutti i file di descrizione presenti relativi ai diversi progetti dell'utente stesso}


\UCtitle
{Caso d'uso UC3.3}  
{Gestione profilo}
\UCimmagine{UC3.3}{UC3.3 Gestione profilo}
\UC		
{UC3.3}		
{Utente autenticato}
{L'Utente autenticato può gestire il proprio profilo}
{Il sistema reputa l'Utente come autenticato}
\scenario
{L'Utente autenticato può generare una nuova password per accedere al sistema (UC3.3.1).}
\post
{Il sistema ha ottenuto le informazioni relative alle azioni che l'Utente autenticato desidera effettuare}

\UCtitle
{Caso d'uso UC3.3.1}  
{Genera nuova password}		
\UC
{UC3.3}		
{Utente autenticato}
{L'Utente autenticato può generare una nuova passowrd per accedere al sistema}
{Il sistema reputa l'Utente come autenticato}
\post
{Il sistema ha generato una nuova password per l'Utente autenticato ed ha inviato una email di notifica all'utente stesso}

\UCtitle
{Caso d'uso UC3.4}  
{Disconnessione}
\UC		
{UC3}		
{Utente autenticato}
{L'utente può disconnettersi dal sistema}
{Il sistema reputa l'Utente come autenticato}
\post
{Il sistema reputa l'Utente non autenticato e propone la pagina principale di login}


\UCtitle
{Caso d'uso UC3.5}  
{Visualizzazione pagine web}
\UC		
{UC3}		
{Utente, Utente autenticato}
{L'utente e l'Utente autenticato possono visualizzare le pagine web di un determinato progetto mediante un URL specifico}
{Il sistema ha creato lo scheletro del progetto}
\post
{Il sistema visualizza la pagina principale del progetto}


\UCtitle
{Caso d'uso UC3.6}  
{Autenticazione}
\UCimmagine{UC3.6}{UC3.6 Autenticazione}
\UC		
{UC3.6}		
{Utente}
{L'Utente può effettuare l'autenticazione al sistema}
{Il sistema reputa l'Utente come non autenticato}
\scenario
{L'Utente può inserire il nome utente (UC3.6.1);|
L'Utente può inserire la password (UC3.6.2).
}
\scenarioAlt
{Nel caso in cui le credenziali di accesso non fossero corrette il sistema segnala il problema e ripresenta la pagina principale di login.
}
\post
{Il sistema ha ottenuto le informazioni relative all'operazione di autenticazione inserite dall'Utente. Viene proposta la pagina relativa all'account utente}


\UCtitle
{Caso d'uso UC3.6.1}  
{Inserimento nome utente}
\UC		
{UC3.6}		
{Utente}
{L'Utente può inserire il nome utente}
{Il sistema permette di inserire il nome utente all'interno di un'apposita casella di testo ed è in attesa dell'inserimento}
\post
{Il sistema conosce il nome utente inserito}

\UCtitle
{Caso d'uso UC3.6.2}  
{Inserimento password}
\UC		
{UC3.6}		
{Utente}
{L'Utente può inserire la password}
{Il sistema permette di inserire la password all'interno di un'apposita casella di testo ed è in attesa dell'inserimento}
\post
{Il sistema conosce la password inserita}

\UCtitle
{Caso d'uso UC3.7}  
{Registrazione}
\UCimmagine{UC3.7}{UC3.7 Registrazione}
\UC		
{UC3.7}		
{Utente}
{L'utente può registrarsi al sistema}
{Il sistema reputa l'Utente come non autenticato}
\scenario
{L'Utente può inserire il nome utente (UC3.6.1);|
L'Utente può inserire la password (UC3.6.2);|
L'Utente può inserire l'email (UC3.7.1).
}
\scenarioAlt
{Nel caso in cui i dati forniti dall'Utente non siano formalmente corretti oppure già presenti nel sistema viene segnalato un messaggio di errore e viene riproposta la pagina di registrazione.
}
\post
{Il sistema ha acquisito i dati dell'Utente, li ha salvati in modo permanente e propone all'Utente la pagina relativa al suo account}

\UCtitle
{Caso d'uso UC3.7.1}  
{Inserimento email}	
\UC	
{UC3.7}		
{Utente}
{L'Utente può inserire l'email per la registrazione dell'account}
{Il sistema permette di inserire la password all'interno di un'apposita casella di testo ed è in attesa dell'inserimento}
\post
{Il sistema ha ottenuto il contenuto del testo inserito dall'Utente}

\UCtitle
{Caso d'uso UC3.8}  
{Recupero password}	
\UCimmagine{UC3.8}{UC3.8 Recupero password}
\UC	
{UC3.8}		
{Utente}
{L'utente può recuperare la password dimenticata}
{Il sistema reputa l'Utente come non autenticato}
\scenario{L'Utente può inserire il nome utente per effettuare il recupero password (UC3.6.1).}
\post{Il sistema ha ottenuto il nome dell'Utente che richiede il recupero password ed ha inviato la relativa password all'email associata all'account}

\newpage
\section{Requisiti} %4
Di seguito sono riportati tutti i requisiti individuati, in base ai casi d'uso, dal capitolato, e alla luce di quanto emerso dall'incontro con il Proponente e da necessità interne.\\
Per garantire una migliore leggibilità sono divisi in tabelle separate a seconda della categoria d'appartenenza.\\
Ogni requisito è descritto secondo la notazione riportata in sezione 5.3.2 del documento \emph{Norme\_{}di\_{}Progetto\_{}v\versioneNormeDiProgetto{}.pdf}.

%tabella requisiti codice requisito, descrizione, fonte, UC di riferimento

\subsection{Requisiti funzionali MaaP}
\begin{longtable}{|c|p{6cm}|c|c|}
\caption{Requisiti funzionali MaaP}
\label{tab:Requisiti MaaP} \\
\toprule
\multicolumn{1}{|c}{\textbf{Requisito}} & \multicolumn{1}{|p{6cm}}{\textbf{Descrizione}}   & \multicolumn{1}{|c}{\textbf{Fonte}} & \multicolumn{1}{|c|}{\textbf{Caso d'uso}}\\
\midrule
\endfirsthead
\multicolumn{2}{l}{\footnotesize\itshape\tablename~\thetable: continua dalla pagina precedente} \\
\toprule
\multicolumn{1}{|c}{\textbf{Requisito}} & \multicolumn{1}{|p{6cm}}{\textbf{Descrizione}}   & \multicolumn{1}{|c}{\textbf{Fonte}} & \multicolumn{1}{|c|}{\textbf{Caso d'uso}}\\
\midrule
\endhead
\midrule
\multicolumn{2}{r}{\footnotesize\itshape\tablename~\thetable: continua nella prossima pagina} \\
\endfoot
\bottomrule
\multicolumn{2}{r}{\footnotesize\itshape\tablename~\thetable: si conclude dalla pagina precedente} \\
\endlastfoot

% Requisiti utente sviluppatore


\midrule
ROF1
& Il sistema MaaP deve essere in grado di generare lo scheletro del progetto
& Capitolato
& UC1\\
& & & UC1.1
\\

\midrule
ROF1.1
& Il sistema MaaP deve installare le librerie necessarie al funzionamento del progetto
& Capitolato
&
\\
\midrule
ROF1.2
& Il sistema MaaP deve generare i file di configurazione necessari al funzionamento del progetto
& Capitolato
&
\\
\midrule
ROF1.3
& Il sistema MaaP deve generare le directory necessarie al funzionamento del progetto
& Capitolato
&
\\
\midrule
ROF1.4
& Il sistema MaaP deve generare il sistema di autenticazione per le pagine web
& Capitolato
&
\\
\midrule
ROF1.4.1
& Il sistema di autenticazione per le pagine web deve essere generato insieme ad un profilo amministratore di \gloss{default}
& Verbale\_2013\_12\_05
&
\\

\midrule
ROF1.5
& Il sistema MaaP deve essere in grado eliminare un progetto esistente
& Interna
& UC1\\
& & & UC1.6
\\

\midrule
ROF1.6
& Il sistema MaaP deve essere in grado di clonare un progetto esistente
& Interna
& UC1\\
& & & UC1.7
\\

\midrule
RFF2
& Il sistema MaaP deve permettere all'utente sviluppatore di utilizzare un editor interno specializzato per la scrittura/modifica dei file di descrizione
& Capitolato
& UC1\\
& & & UC1.4
\\

\midrule
RFF2.1
& Il sistema MaaP deve permettere all'utente sviluppatore di utilizzare un editor interno specializzato per la scrittura di un nuovo file di descrizione
& Interna
& UC1.4 \\
& & & UC1.4.1
\\

\midrule
RFF2.1.1
& Il sistema MaaP deve permettere all'utente sviluppatore di utilizzare un editor interno specializzato per scrivere il codice del file di descrizione che intende creare
& Interna
& UC1.4.1 \\
& & & UC1.4.1.1
\\

\midrule
RFF2.1.2
& Il sistema MaaP deve permettere all'utente sviluppatore di utilizzare un editor interno specializzato per salvare il codice scritto in modo permanente
& Interna
& UC1.4.1 \\
& & & UC1.4.1.2
\\

\midrule
RFF2.2
& Il sistema MaaP deve permettere all'utente sviluppatore di utilizzare un editor interno specializzato per la modifica di un file di descrizione esistente
& Interna
& UC1.4 \\
& & & UC1.4.2
\\

\midrule
RFF2.2.1
& Il sistema MaaP deve permettere all'utente sviluppatore di utilizzare un editor interno specializzato per modificare il codice di un file di descrizione esistente
& Interna
& UC1.4.2 \\
& & & UC1.4.2.1
\\

\midrule
RFF2.2.2
& Il sistema MaaP deve permettere all'utente sviluppatore di utilizzare un editor interno specializzato per salvare il codice modificato in modo permanente
& Interna
& UC1.4.2 \\
& & & UC1.4.2.2
\\

\midrule
RFF2.2.3
& Il sistema MaaP deve permettere all'utente sviluppatore di utilizzare un editor interno specializzato per annullare le modifiche al codice del file di descrizione modificato
& Interna
& UC1.4.2 \\
& & & UC1.4.2.3
\\

\midrule
ROF3
& Il sistema MaaP deve permette all'utente sviluppatore di inserire un file di descrizione
& Interna
& UC1\\
& & & UC1.5\\

\midrule
ROF4
& Il sistema deve permette all'utente sviluppatore di utilizzare un file di descrizione
& Capitolato
& UC1\\
& & & UC1.2
\\
\midrule
ROF4.1
& Il sistema MaaP deve permettere all'utente sviluppatore di creare la visualizzazione della Collection
& Capitolato
& UC1.2 \\
& & & UC1.2.1
\\
\midrule
ROF4.1.1
& Il sistema MaaP deve permettere all'utente sviluppatore di creare la visualizzazione del menù per le Collection
& Capitolato
& UC1.2.1\\
& & & UC1.2.1.1
\\
\midrule
ROF4.1.1.1
& Il sistema MaaP deve permettere all'utente sviluppatore di definire il nome della voce relativa alla Collection
& Capitolato
& UC1.2.1.1\\
& & & UC1.2.1.1.1
\\
\midrule
ROF4.1.1.2
& Il sistema MaaP deve permettere all'utente sviluppatore di definire la posizione di una voce all'interno del menù
& Capitolato
& UC1.2.1.1\\
& & & 1.2.1.1.2
\\
\midrule
ROF4.1.2
& Il sistema MaaP deve permettere all'utente sviluppatore di creare la visualizzazione della pagina Collection-Index
& Capitolato
& UC1.2.1\\
& & & UC1.2.1.2
\\
\midrule
ROF4.1.2.1
& Il sistema MaaP deve permettere all'utente sviluppatore di aggiungere delle chiavi da visualizzare nella pagina Collection-Index
& Capitolato
& UC1.2.1.2\\
& & & UC1.2.1.2.1
\\
\midrule
ROF4.1.2.1.1
& Il sistema MaaP deve permettere all'utente sviluppatore di aggiungere definire un'etichetta per la chiave da visualizzare
& Capitolato
& UC1.2.1.2.1\\
& & & UC1.2.1.2.1.1
\\
\midrule
ROF4.1.2.1.2
& Il sistema MaaP deve permettere all'utente sviluppatore di definire un campo associato alla chiave da visualizzare
& Capitolato
& UC1.2.1.2.1\\
& & & UC1.2.1.2.1.2
\\
\midrule
ROF4.1.2.1.3
& Il sistema MaaP deve permettere all'utente sviluppatore di definire un campo associato alla chiave da visualizzare, proveniente da un documento esterno
& Capitolato
& UC1.2.1.2.1\\
& & & UC1.2.1.2.1.3
\\
\midrule
ROF4.1.2.1.4
& Il sistema MaaP deve permettere all'utente sviluppatore di definire un campo associato alla chiave da visualizzare, proveniente dal risultato di una query
& Capitolato
& UC1.2.1.2.1\\
& & & UC1.2.1.2.1.4
\\
\midrule
ROF4.1.2.1.5
& Il sistema MaaP deve permettere all'utente sviluppatore di definire un campo associato alla chiave da visualizzare come trasformazione
& Capitolato
& UC1.2.1.2.1\\
& & & UC1.2.1.2.1.5
\\

\midrule
ROF4.1.2.2
& Il sistema MaaP deve permettere all'utente sviluppatore di definire un ordinamento rispetto a una chiave
& Capitolato
& UC1.2.1.2\\
& & & UC1.2.1.2.2
\\
\midrule
ROF4.1.2.3
& Il sistema deve permettere all'utente sviluppatore di  definire un numero massimo di Document da visualizzare per la pagina Collection-Index
& Capitolato
& UC1.2.1.2\\
& & & UC1.2.1.2.3
\\
\midrule
RFF4.1.2.4
& Il sistema MaaP deve permettere all'utente sviluppatore di aggiungere dei pulsanti all'interno della pagina Collection-Index
& Capitolato
& UC1.2.1.2\\
& & & UC1.2.1.2.4
\\
\midrule
ROF4.1.3
& Il sistema MaaP deve permettere all'utente sviluppatore di creare la visualizzazione per la pagine Document-Show
& Capitolato
& UC1.2.1\\
& & & UC1.2.1.3
\\
\midrule
ROF4.1.3.1
& Il sistema MaaP deve permettere all'utente sviluppatore di aggiungere delle chiavi da visualizzare nella pagina Document-Show
& Capitolato
& UC1.2.1.3\\
& & & UC1.2.1.3.1
\\

\midrule
ROF4.1.3.1.1
& Il sistema MaaP deve permettere all'utente sviluppatore di aggiungere definire un'etichetta per la chiave da visualizzare
& Capitolato
& UC1.2.1.3.1\\
& & & UC1.2.1.3.1.1
\\
\midrule
ROF4.1.3.1.2
& Il sistema MaaP deve permettere all'utente sviluppatore di definire un campo associato alla chiave da visualizzare
& Capitolato
& UC1.2.1.3.1\\
& & & UC1.2.1.3.1.2
\\
\midrule
ROF4.1.3.1.3
& Il sistema MaaP deve permettere all'utente sviluppatore di definire un campo associato alla chiave da visualizzare, proveniente da un documento esterno
& Capitolato
& UC1.2.1.3.1\\
& & & UC1.2.1.3.1.3
\\
\midrule
ROF4.1.3.1.4
& Il sistema MaaP deve permettere all'utente sviluppatore di definire un campo associato alla chiave da visualizzare, proveniente dal risultato di una query
& Capitolato
& UC1.2.1.3.1\\
& & & UC1.2.1.3.1.4
\\
\midrule
ROF4.1.3.1.5
& Il sistema MaaP deve permettere all'utente sviluppatore di definire un campo associato alla chiave da visualizzare come trasformazione
& Capitolato
& UC1.2.1.3.1\\
& & & UC1.2.1.3.1.5
\\

\midrule
RFF4.1.3.2
& Il sistema MaaP deve permettere all'utente sviluppatore di aggiungere un pulsante all'interno della  pagina Document-Show
& Capitolato
& UC1.2.1.3\\
& & & UC1.2.1.3.2
\\
\midrule
ROF4.2
& Il sistema MaaP deve permettere all'utente sviluppatore di modificare la visualizzazione della Collection
& Capitolato
& UC1.2\\
& & & UC1.2.2
\\


\midrule
ROF4.2.1
& Il sistema MaaP deve permettere all'utente sviluppatore di impostare la visualizzazione del menù delle le Collection
& Capitolato
& UC1.2.2\\
& & & UC1.2.2.1
\\
\midrule
ROF4.2.1.1
& Il sistema MaaP deve permettere all'utente sviluppatore di modificare il nome della voce relativa alla Collection
& Capitolato
& UC1.2.2.1\\
& & & UC1.2.2.1.1
\\
\midrule
ROF4.2.1.2
& Il sistema MaaP deve permettere all'utente sviluppatore di modificare la posizione di una voce all'interno del menù
& Capitolato
& UC1.2.2.1\\
& & & UC1.2.2.1.2
\\
\midrule
ROF4.2.2
& Il sistema MaaP deve permettere all'utente sviluppatore di impostare la visualizzazione della pagina Collection-Index
& Capitolato
& UC1.2.2\\
& & & UC1.2.2.2
\\

\midrule
ROF4.2.2.1
& Il sistema MaaP deve permettere all'utente sviluppatore di aggiungere delle chiavi da visualizzare nella pagina Collection-Index
& Capitolato
& UC1.2.1.2\\
& & & UC1.2.1.2.1
\\
\midrule
ROF4.2.2.2
& Il sistema MaaP deve permettere all'utente sviluppatore di eliminare delle chiavi da visualizzare nella pagina Collection-Index
& Interna
& UC1.2.2.2\\
& & & UC1.2.2.2.1
\\
\midrule
ROF4.2.2.3
& Il sistema MaaP deve permettere all'utente sviluppatore di definire un ordinamento, alfabetico crescente o decrescente, rispetto a una chiave
& Capitolato
& UC1.2.1.2\\
& & & UC1.2.1.2.2
\\
\midrule
ROF4.2.2.4
& Il sistema MaaP deve permettere all'utente sviluppatore di eliminare un ordinamento rispetto a una chiave
& Interna
& UC1.2.2.2\\
& & & UC1.2.2.2.2
\\
\midrule
ROF4.2.2.5
& Il sistema deve permettere all'utente sviluppatore di  definire un numero massimo di Document da visualizzare per la pagina Collection-Index
& Capitolato
& UC1.2.1.2\\
& & & UC1.2.1.2.3
\\
\midrule
ROF4.2.2.6
& Il sistema deve permettere all'utente sviluppatore di  eliminare il numero massimo di Document da visualizzare per la pagina Collection-Index
& Capitolato
& UC1.2.2.2\\
& & & UC1.2.2.2.4
\\
\midrule
RFF4.2.2.7
& Il sistema MaaP deve permettere all'utente sviluppatore di aggiungere dei pulsanti all'interno della pagina Collection-Index, specificando il nome del pulsante e l'azione che deve eseguire
& Capitolato
& UC1.2.1.2\\
& & & UC1.2.1.2.4
\\
\midrule
RFF4.2.2.8
& Il sistema MaaP deve permettere all'utente sviluppatore di eliminare dei pulsanti all'interno della pagina Collection-Index
& Capitolato
& UC1.2.2.2\\
& & & UC1.2.2.2.5
\\
\midrule
ROF4.2.3
& Il sistema MaaP deve permettere all'utente sviluppatore di impostare la visualizzazione per la pagine Document-Show
& Capitolato
& UC1.2.2\\
& & & UC1.2.2.3
\\
\midrule
ROF4.2.3.1
& Il sistema MaaP deve permettere all'utente sviluppatore di aggiungere delle chiavi da visualizzare nella pagina Document-Show
& Capitolato
& UC1.2.1.3\\
& & & UC1.2.1.3.1
\\
\midrule
ROF4.2.3.2
& Il sistema MaaP deve permettere all'utente sviluppatore di eliminare delle chiavi da visualizzare nella pagina Document-Show
& Capitolato
& UC1.2.2.3\\
& & & UC1.2.2.3.1
\\

\midrule
RFF4.2.3.3
& Il sistema MaaP deve permettere all'utente sviluppatore di aggiungere dei pulsanti all'interno della  pagina Document-Show, specificando il nome del pulsante e l'azione che deve eseguire
& Capitolato
& UC1.2.1.3\\
& & & UC1.2.1.3.2
\\
\midrule
RFF4.2.3.4
& Il sistema MaaP deve permettere all'utente sviluppatore di eliminare dei pulsanti all'interno della  pagina Document-Show
& Capitolato
& UC1.2.2.3\\
& & & UC1.2.2.3.2
\\

\midrule
ROF4.3
& Il sistema MaaP deve permettere all'utente sviluppatore di definire una query personalizzata
& Capitolato
& UC1.2.3
\\

\midrule
ROF4.4
& Il sistema MaaP deve permettere all'utente sviluppatore di eliminare una query personalizzata
& Capitolato
& UC1.2.4
\\

\midrule
ROF5
& Il sistema deve permettere all'utente sviluppatore la modifica dei file di configurazione
& Interna
& UC1.3
\\

\midrule
RDF5.1
& Il sistema deve permettere all'utente sviluppatore di abilitare la funzionalità di registrazione per l'utente finale nelle pagine web. Nel caso la registrazione sia abilitata l'utente finale può registrarsi al sistema, altrimenti no.
& Verbale\_2013\_12\_05
& UC1.3\\
& & & UC1.3.1
\\

\midrule
RFF5.2
& Il sistema MaaP deve permettere all'utente sviluppatore di abilitare la funzionalità per la creazione di nuovi Document all'interno della pagina Collection-Index
& Capitolato
& UC1.3\\
& &  & UC1.3.2
\\

\midrule
RDF5.3
& Il sistema MaaP deve permettere all'utente sviluppatore di modificare i template per le pagine web
& Interna
& UC1.3\\
& & & UC1.3.3
\\

\midrule
ROF5.4
& Il sistema deve permettere all'utente sviluppatore di specificare nome, indirizzo e password, relativi al database di analisi con il quale interagire
& Interna
& UC1.3\\
& & & UC1.3.4
\\

\midrule
ROF5.5
& Il sistema deve permettere all'utente sviluppatore di abilitare la funzionalità per la creazione di nuovi indici all'interno della pagina Collection-Index
& Capitolato
& UC1.3\\
& & Verbale\_2013\_12\_05 & UC1.3.5
\\


%FINE TABELLA REQUISITI MAAP, NON CANCELLARE
\end{longtable}

\newpage
\subsection{Requisiti funzionali MaaP's Web}
\begin{longtable}{|c|p{6cm}|c|c|}
\caption{Requisiti funzionali MaaP's Web}
\label{tab:Requisiti MaaP's Web} \\
\toprule
\multicolumn{1}{|c}{\textbf{Requisito}} & \multicolumn{1}{|p{6cm}}{\textbf{Descrizione}}   & \multicolumn{1}{|c}{\textbf{Fonte}} & \multicolumn{1}{|c|}{\textbf{Caso d'uso}}\\
\midrule
\endfirsthead
\multicolumn{2}{l}{\footnotesize\itshape\tablename~\thetable: continua dalla pagina precedente} \\
\toprule
\multicolumn{1}{|c}{\textbf{Requisito}} & \multicolumn{1}{|p{6cm}}{\textbf{Descrizione}}   & \multicolumn{1}{|c}{\textbf{Fonte}} & \multicolumn{1}{|c|}{\textbf{Caso d'uso}}\\
\midrule
\endhead
\midrule
\multicolumn{2}{r}{\footnotesize\itshape\tablename~\thetable: continua nella prossima pagina} \\
\endfoot
\bottomrule
\multicolumn{2}{r}{\footnotesize\itshape\tablename~\thetable: si conclude dalla pagina precedente} \\
\endlastfoot

%Requisiti utente business
\midrule
ROF6
& L'utente business, al primo accesso, deve poter usare il profilo amministratore di default
& Verbale\_2013\_12\_05
&
\\

\midrule
ROF7
& L'utente business deve potersi autenticare inserendo dei dati personali
& Capitolato
& UC2\\
& & & UC2.2
\\

\midrule
ROF7.1
& L'utente business deve inserire l'email per l'autenticazione
& Capitolato
& UC2.2\\
& & & UC2.2.1
\\

\midrule
ROF7.2
& L'utente business deve inserire la password per l'autenticazione
& Capitolato
& UC2.2\\
& & & UC2.2.2
\\

\midrule
ROF7.2.1
& La password per l'autenticazione deve essere alfanumerica e contenere almeno otto caratteri
& Interna
&
\\

\midrule
RDF8
& L'utente business deve potersi registrare inserendo dei dati personali
& Verbale\_2013\_12\_05
& UC2\\
& & & UC2.1
\\

\midrule
RDF8.1
& L'utente business, per registrarsi, deve inserire una email non presente nel sistema
& Capitolato
& UC2.1\\
& & & UC2.1.1
\\

\midrule
RDF8.2
& L'utente business, per registrarsi, deve inserire una password
& Capitolato
& UC2.1\\
& & & UC2.1.2
\\

\midrule
RDF8.2.1
& La password per la registrazione deve essere alfanumerica e contenere almeno otto caratteri
& Interna
&
\\

\midrule
ROF9
& L'utente business deve poter recuperare la password
& Capitolato
& UC2\\
& & & UC2.3
\\

\midrule
ROF10
& L'utente business autenticato deve poter aprire una Collection e visualizzare la sua pagina Collection-Index
& Capitolato
& UC2\\
& & & UC2.4
\\

\midrule
ROF10.1
& L'utente business autenticato deve poter visualizzare una pagina Document-Show
& Capitolato
& UC2.4\\
& & & UC2.4.1
\\

\midrule
ROF10.1.1
& L'utente business autenticato deve poter visualizzare il Document selezionato
& Capitolato
& UC2.4.1\\
& & & UC2.4.1.1
\\

\midrule
ROF10.1.2
& L'utente business autenticato deve poter eliminare il Document che sta visualizzando
& Interna
& UC2.4.1\\
& & & UC2.4.1.2
\\

\midrule
ROF10.1.3
& L'utente business autenticato deve poter modificare il Document che sta visualizzando
& Interna
& UC2.4.1\\
& & & UC2.4.1.3
\\

\midrule
RDF10.2
& L'utente business autenticato deve poter modificare la visualizzazione dei Document
& Interna
& UC2.4\\
& & & UC2.4.2
\\

\midrule
RDF10.2.1
& L'utente business autenticato deve poter selezionare dei criteri per la visualizzazione
& Interna
& UC2.4.2\\
& & & UC2.4.2.1
\\

\midrule
RDF10.2.1.1
& L'utente business autenticato deve poter effettuare un ordinamento rispetto a una chiave
& Interna
& UC2.4.2.1\\
& & & UC2.4.2.1.1
\\

\midrule
RDF10.2.1.2
& L'utente business deve poter selezionare un numero massimo di Document da visualizzare per pagina
& Interna
& UC2.4.2.1\\
& & & UC2.4.2.1.2
\\


\midrule
RDF10.2.2
& L'utente business autenticato deve poter applicare un filtro alla visualizzazione dei Document
& Interna
& UC2.4.2\\
& & Verbale\_2013\_12\_05 & UC2.4.2.2
\\

\midrule
RDF10.2.3
& L'utente business autenticato deve poter annullare il filtro
& Interna
& UC2.4.2\\
& & Verbale\_2013\_12\_05 & UC2.4.2.3
\\

\midrule
ROF10.2.4
& L'utente business autenticato deve poter disconnettersi
& Interna
& UC2.4\\
& & & UC2.5
\\

\midrule
ROF10.2.5
& L'utente business autenticato deve poter navigare tra la Collection
& Capitolato
& UC2.4\\
& & & UC2.4.3\\

\midrule
ROF10.3
& L'utente business autenticato deve poter gestire il proprio profilo
& Interna
& UC2\\
& & & UC2.6
\\

\midrule
ROF10.3.1
& L'utente business autenticato deve poter gestire i propri dati
& Capitolato
& UC2.6\\
& & & UC2.6.3
\\

\midrule
ROF10.3.1.1
& L'utente business autenticato deve poter modificare i propri dati utente
& Capitolato
& UC2.6.3\\
& & & UC2.6.3.1
\\

\midrule
ROF10.3.1.2
& L'utente business autenticato deve poter  salvare le modifiche apportate
& Interna
& UC2.6.3\\
& & & UC2.6.3.2
\\

\midrule
ROF10.3.1.3
& L'utente business autenticato deve poter annullare le modifiche apportate
& Interna
& UC2.6.3\\
& & & UC2.6.3.5
\\

\midrule
ROF10.3.1.4
& L'utente business autenticato amministratore deve poter modificare i dati degli utenti business
& Interna
& UC2.6.3\\
& & & UC2.6.3.3
\\

\midrule
ROF10.3.1.5
& L'utente business autenticato deve poter modificare i permessi degli utenti business
& Capitolato
& UC2.6.3\\
& & & UC2.6.3.4
\\

\midrule
ROF10.3.2
& L'utente business autenticato amministratore deve poter creare un nuovo utente business
& Capitolato
& UC2.6\\
& & & UC2.6.1
\\

\midrule
ROF10.3.3
& L'utente business autenticato amministratore deve poter eliminare un utente business
& Capitolato
& UC2.6\\
& & & UC2.6.2
\\

\midrule
ROF10.4
& L'utente business autenticato amministratore deve poter cancellare un Document
& Interna
& UC2.4\\
& & & UC2.4.4
\\

\midrule
ROF10.5
& L'utente business autenticato amministratore deve poter modificare un Document
& Capitolato
& UC2.4\\
& & & UC2.4.5
\\

\midrule
ROF10.5.1
& L'utente business autenticato amministratore deve poter modificare i valori associati alla chiavi
& Interna
& UC2.4.5\\
& & & UC2.4.5.1
\\

\midrule
ROF10.5.2
& L'utente business autenticato amministratore deve poter salvare le modifiche apportate al Document
& Interna
& UC2.4.5\\
& & & UC2.4.5.2
\\

\midrule
ROF10.5.3
& L'utente business autenticato amministratore deve poter annullare le modifiche apportate al Document
& Interna
& UC2.4.5\\
& & & UC2.4.5.3
\\

\midrule
ROF10.6
& L'utente business autenticato amministratore deve poter visualizzare le query più utilizzate dal sistema MaaP
& Capitolato
& UC2.4\\
& & & UC2.4.6
\\

\midrule
ROF10.7
& L'utente business autenticato amministratore deve poter creare degli indici
& Capitolato
& UC2.4\\
& & & UC2.4.7
\\

%FINE TABELLA REQUISITI MAAPSWEB, NON CANCELLARE
\end{longtable}

\newpage
\subsection{Requisiti funzionali MaaS}
\begin{longtable}{|c|p{6cm}|c|c|}
\caption{Requisiti funzionali MaaS}
\label{tab:Requisiti MaaS} \\
\toprule
\multicolumn{1}{|c}{\textbf{Requisito}} & \multicolumn{1}{|p{6cm}}{\textbf{Descrizione}}   & \multicolumn{1}{|c}{\textbf{Fonte}} & \multicolumn{1}{|c|}{\textbf{Caso d'uso}}\\
\midrule
\endfirsthead
\multicolumn{2}{l}{\footnotesize\itshape\tablename~\thetable: continua dalla pagina precedente} \\
\toprule
\multicolumn{1}{|c}{\textbf{Requisito}} & \multicolumn{1}{|p{6cm}}{\textbf{Descrizione}}   & \multicolumn{1}{|c}{\textbf{Fonte}} & \multicolumn{1}{|c|}{\textbf{Caso d'uso}}\\
\midrule
\endhead
\midrule
\multicolumn{2}{r}{\footnotesize\itshape\tablename~\thetable: continua nella prossima pagina} \\
\endfoot
\bottomrule
\multicolumn{2}{r}{\footnotesize\itshape\tablename~\thetable: si conclude dalla pagina precedente} \\
\endlastfoot

%REQUISITI MAAS
\midrule
RFF11.1
& Il sistema MaaS deve permettere all'utente di autenticarsi al sistema
& Interna
& UC3\\
& & & UC3.6\\

\midrule
RFF11.1.1
& Il sistema MaaS deve permettere all'utente di inserire il nome utente
& Interna
& UC3.6\\
& & & UC3.6.1\\

\midrule
RFF11.1.2
& Il sistema MaaS deve permettere all'utente di inserire la password
& Interna
& UC3.6\\
& & & UC3.6.2\\

\midrule
RFF11.2
& Il sistema MaaS deve permettere all'utente di registrarsi al sistema
& Interna
& UC3\\
& & & UC3.7\\

\midrule
RFF11.2.1
& Il sistema MaaS deve permettere all'utente di inserire la propria email
& Interna
& UC3.7\\
& & & UC3.7.1\\

\midrule
RFF11.3
& Il sistema MaaS deve permettere all'utente di recuperare la password
& Interna
& UC3\\
& & & UC3.8\\

\midrule
RFF11.4
& Il sistema MaaS deve permettere all'utente di visualizzare le pagine web create
& Capitolato
& UC3\\
& & & UC3.5\\

\midrule
RFF11.5
& Il sistema MaaS deve permettere all'utente autenticato di creare lo scheletro del progetto
& Interna
& UC3\\
& & & UC3.1\\

\midrule
RFF11.5.1
& Il sistema MaaS deve permettere all'utente autenticato di inserire il nome del progetto
& Interna
& UC3.1\\
& & & UC3.1.1\\

\midrule
RFF11.6
& Il sistema MaaS deve permettere all'utente autenticato di gestire le pagine web
& Capitolato
& UC3\\
& & & UC3.2\\

\midrule
RFF11.6.1
& Il sistema MaaS deve permettere all'utente autenticato di creare un file di descrizione
& Capitolato
& UC3.2\\
& & & UC3.2.1\\

\midrule
RFF11.6.1.1
& Il sistema MaaS deve permettere all'utente autenticato la scrittura di un file di descrizione tramite editor di testo
& Capitolato
& UC3.2.1\\
& & & UC3.2.1.1\\

\midrule
RFF11.6.1.2
& Il sistema MaaS deve permettere all'utente autenticato di salvare il file di descrizione
& Interna
& UC3.2.1\\
& & & UC3.2.1.2\\


\midrule
RFF11.6.2
& Il sistema MaaS deve permettere all'utente autenticato di eseguire l'upload di un file di descrizione creato precedentemente con il sistema MaaP
& Capitolato
& UC3.2\\
& & & UC3.2.2\\

\midrule
RFF11.6.2.1
& Il sistema MaaS deve permettere all'utente autenticato di navigare all'interno del file system
& Interna
& UC3.2.2\\
& & & UC3.2.2.1\\

\midrule
RFF11.6.2.2
& Il sistema MaaS deve permettere all'utente autenticato di selezionare un file di descrizione
& Interna
& UC3.2.2\\
& & & UC3.2.2.2\\

\midrule
RFF11.6.2.3
& Il sistema MaaS deve permettere all'utente autenticato di confermare l'upload del file selezionato
& Interna
& UC3.2.2\\
& & & UC3.2.2.3\\


\midrule
RFF11.6.3
& Il sistema MaaS deve permettere all'utente autenticato di modificare un file di descrizione esistente
& Interna
& UC3.2\\
& & & UC3.2.3\\

\midrule
RFF11.6.3.1
& Il sistema MaaS deve permettere all'utente autenticato di modificare il codice del file di descrizione selezionato
& Interna
& UC3.2.3\\
& & & UC3.2.3.1\\

\midrule
RFF11.6.3.2
& Il sistema MaaS deve permettere all'utente autenticato di salvare le modifiche apportate al file di descrizione
& Interna
& UC3.2.3\\
& & & UC3.2.3.2\\

\midrule
RFF11.6.3.3
& Il sistema MaaS deve permettere all'utente autenticato di annullare le modifiche apportate al file di descrizione
& Interna
& UC3.2.3\\
& & & UC3.2.3.3\\

\midrule
RFF11.6.4
& Il sistema MaaS deve permettere all'utente autenticato di modificare il file di configurazione
& Capitolato
& UC3.2\\
& & & UC3.2.4\\

\midrule
RFF11.6.4.1
& Il sistema MaaS deve permettere all'utente autenticato di modificare il codice del file di configurazione selezionato
& Interna
& UC3.2.4\\
& & & UC3.2.4.1\\

\midrule
RFF11.6.4.2
& Il sistema MaaS deve permettere all'utente autenticato di salvare le modifiche apportate al file di configurazione
& Interna
& UC3.2.4\\
& & & UC3.2.4.2\\

\midrule
RFF11.6.4.3
& Il sistema MaaS deve permettere all'utente autenticato di annullare le modifiche apportate al file di configurazione
& Interna
& UC3.2.4\\
& & & UC3.2.4.3\\

\midrule
RFF11.6.5
& Il sistema MaaS deve permettere all'utente autenticato di visualizzare tutti i file di descrizione presenti
& Capitolato
& UC3.2\\
& & & UC3.2.5\\

\midrule
RFF11.7
& Il sistema MaaS deve permettere all'utente autenticato di gestire il proprio profilo utente
& Interna
& UC3\\
& & & UC3.3\\

\midrule
RFF11.7.1
& Il sistema MaaS deve permettere all'utente autenticato di generare una nuova password
& Interna
& UC3.3\\
& & & UC3.3.1\\

\midrule
RFF11.8
& Il sistema MaaS deve permettere all'utente autenticato di disconnettersi dal sistema
& Interna
& UC3\\
& & & UC3.4\\

\midrule
RFF11.9
& Il sistema MaaS deve permettere all'utente autenticato di visualizzare le pagine web create
& Capitolato
& UC3\\
& & & UC3.5\\

\midrule
ROF12
& Definizione di un linguaggio astratto DSL per la definizione delle pagine che verranno generate
& Capitolato
&
\\

\midrule
ROF12.1
& Il linguaggio definito deve essere testuale
& Capitolato
&
\\

%FINE TABELLA REQUISITI MAAS, NON CANCELLARE
\end{longtable}
\newpage
\subsection{Requisiti di qualità}
\begin{longtable}{|c|p{6cm}|c|c|}
\caption{Requisiti di qualità}
\label{tab:Requisiti di qualita} \\
\toprule
\multicolumn{1}{|c}{\textbf{Requisito}} & \multicolumn{1}{|p{6cm}}{\textbf{Descrizione}}   & \multicolumn{1}{|c}{\textbf{Fonte}} & \multicolumn{1}{|c|}{\textbf{Caso d'uso}}\\
\midrule
\endfirsthead
\multicolumn{2}{l}{\footnotesize\itshape\tablename~\thetable: continua dalla pagina precedente} \\
\toprule
\multicolumn{1}{|c}{\textbf{Requisito}} & \multicolumn{1}{|p{6cm}}{\textbf{Descrizione}}   & \multicolumn{1}{|c}{\textbf{Fonte}} & \multicolumn{1}{|c|}{\textbf{Caso d'uso}}\\
\midrule
\endhead
\midrule
\multicolumn{2}{r}{\footnotesize\itshape\tablename~\thetable: continua nella prossima pagina} \\
\endfoot
\bottomrule
\multicolumn{2}{r}{\footnotesize\itshape\tablename~\thetable: si conclude dalla pagina precedente} \\
\endlastfoot

% Requisiti di qualita

\midrule
ROQ13
& Devono essere rispettate tutte le norme riportate nel documento Norme di Progetto
& Interna
&
\\

\midrule
ROQ14
& Il \gloss{software} deve essere dotato di un manuale utente che ne descriva l'utilizzo
& Interna
&
\\

\midrule
ROQ15
& Il progetto deve essere pubblicato su \gloss{GitHub} e utilizzare le sue distribuzioni per segnalare eventuali correzioni o errori
& Capitolato
&
\\

%FINE TABELLA REQUISITI QUALITA', NON CANCELLARE
\end{longtable}

\newpage
\subsection{Requisiti di vincolo}
\begin{longtable}{|c|p{6cm}|c|c|}
\caption{Requisiti di vincolo}
\label{tab:Requisiti di vincolo} \\
\toprule
\multicolumn{1}{|c}{\textbf{Requisito}} & \multicolumn{1}{|p{6cm}}{\textbf{Descrizione}}   & \multicolumn{1}{|c}{\textbf{Fonte}} & \multicolumn{1}{|c|}{\textbf{Caso d'uso}}\\
\midrule
\endfirsthead
\multicolumn{2}{l}{\footnotesize\itshape\tablename~\thetable: continua dalla pagina precedente} \\
\toprule
\multicolumn{1}{|c}{\textbf{Requisito}} & \multicolumn{1}{|p{6cm}}{\textbf{Descrizione}}   & \multicolumn{1}{|c}{\textbf{Fonte}} & \multicolumn{1}{|c|}{\textbf{Caso d'uso}}\\
\midrule
\endhead
\midrule
\multicolumn{2}{r}{\footnotesize\itshape\tablename~\thetable: continua nella prossima pagina} \\
\endfoot
\bottomrule
\multicolumn{2}{r}{\footnotesize\itshape\tablename~\thetable: si conclude dalla pagina precedente} \\
\endlastfoot

% Requisiti di vincolo

\midrule
ROV16
& Il database degli utenti deve essere \gloss{criptato}
& Interna
&
\\

\midrule
ROV17
& Le pagine web prodotte dal framework MaaP devono essere compatibili con la versione 30.0.x di \gloss{Google Chrome} o superiori
& Capitolato
&
\\

\midrule
ROV18
& Le pagine web prodotte dal framework MaaP devono essere compatibili con la versione 24.x o superiore di Firefox
& Capitolato
&
\\

\midrule
ROV19
& Il sistema deve accettare solo file di configurazione che hanno un determinato formato già fissato
& Interna
&
\\

\midrule
ROV20
& Il database degli utenti deve essere realizzato utilizzando MongoDB
& Capitolato
&
\\

\midrule
ROV21
& Il database degli utenti deve essere indipendente dal database di analisi
& Capitolato
&
\\

\midrule
ROV22
& Il database di analisi utilizzato deve essere stato realizzato utilizzando MongoDB
& Capitolato
&
\\

\midrule
ROV23
& L'\gloss{interfaccia}  con il database deve essere realizzata con Mongoose
& Capitolato
&
\\

\midrule
ROV24
& L'\gloss{infrastruttura} delle pagine web generate deve essere realizzata con \gloss{Express}
& Capitolato
&
\\

\midrule
ROV25
& La componente \gloss{server} deve essere realizzata con \gloss{Node.js}
& Capitolato
&
\\

\midrule
ROV26
& Il software sarà fornito di un sistema di installazione per farlo funzionare
& Interna
&
\\

\midrule
ROV27
& Deve essere possibile effettuare il \gloss{deployment} su \gloss{Heroku}
& Capitolato
&
\\

%FINE TABELLA REQUISITI DI VINCOLO, NON CANCELLARE
\end{longtable}


%AJK-FINEREQUISITI NON CANCELLARE QUESTO COMMENTO XKE SERVE ALLO SCRIPT REQUISITI :)

\subsection{Requisiti accettati}
Tutti i requisiti obbligatori e desiderabili saranno implementati. A causa di tempo e risorse limitate
i requisiti opzionali non potranno essere soddisfatti.


\newpage
\section{Tracciamento Requisiti}
%QUESTE TABELLE SONO STATE GENERATE AUTOMATICAMENTE DALLO SCRIPT TRACCIAMENTOSCRIPT

\subsection{Tracciamento requisiti-fonti}
\begin{longtable}{|c|c|}
\caption{Tracciamento requisiti-fonti}
\label{tab:Tracciamento requisiti-fonti} \\
\toprule
\multicolumn{1}{|c}{\textbf{Requisiti}}
& \multicolumn{1}{|c|}{\textbf{Fonti}} \\
\midrule
\endfirsthead
\multicolumn{2}{l}{\footnotesize\itshape\tablename~\thetable: continua dalla pagina precedente} \\
\toprule
\multicolumn{1}{|c}{\textbf{Requisiti}}
& \multicolumn{1}{|c|}{\textbf{Fonti}} \\
\midrule
\endhead
\midrule
\multicolumn{2}{r}{\footnotesize\itshape\tablename~\thetable: continua nella prossima pagina} \\
\endfoot
\bottomrule
\multicolumn{2}{r}{\footnotesize\itshape\tablename~\thetable: si conclude dalla pagina precedente} \\
\endlastfoot

\end{longtable}

\newpage
\subsection{Tracciamento fonti-requisiti}
\begin{longtable}{|c|c|}
\caption{Tracciamento fonti-requisiti}
\label{tab:Tracciamento fonti-requisiti} \\
\toprule
\multicolumn{1}{|c}{\textbf{Fonte}}
& \multicolumn{1}{|c|}{\textbf{Requisiti}} \\
\midrule
\endfirsthead
\multicolumn{2}{l}{\footnotesize\itshape\tablename~\thetable: continua dalla pagina precedente} \\
\toprule
\multicolumn{1}{|c}{\textbf{Fonte}}
& \multicolumn{1}{|c|}{\textbf{Requisiti}} \\
\midrule
\endhead
\midrule
\multicolumn{2}{r}{\footnotesize\itshape\tablename~\thetable: continua nella prossima pagina} \\
\endfoot
\bottomrule
\multicolumn{2}{r}{\footnotesize\itshape\tablename~\thetable: si conclude dalla pagina precedente} \\
\endlastfoot

\end{longtable}

%QUESTE TABELLE SONO STATE GENERATE AUTOMATICAMENTE DALLO SCRIPT TRACCIAMENTOSCRIPT



%FINE DOCUMENTO NON CANCELLARE
\end{document}
