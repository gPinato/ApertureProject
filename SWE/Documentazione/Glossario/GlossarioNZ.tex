%nuova pagina con i termini che iniziano con N
\paginaGlossario{N}

\elemento{Namespace}
{Indica una collezione di nomi di entità, definite dal programmatore, omogeneamente usate in uno o più file sorgente. Lo scopo del namespace è quello di evitare confusione ed equivoci nel caso siano necessarie molte entità con nomi simili, fornendo il modo di raggruppare i nomi per categorie.}

\elemento{Node.js}
{\gloss{Framework} basato sul \gloss{linguaggio} \gloss{JavaScript}. Pone la sua attenzione alla manipolazione di grosse quantità di dati, quali, per esempio, la consultazione di \gloss{database}.}

\elemento{NoSQL}
{Acronimo di "Not Only SQL", "non soltanto SQL". \`{E} un particolare tipo di \gloss{database} che fornisce un sistema di immagazzinamento dei dati e di un loro successivo recupero che richiede l'utilizzo di modelli meno vincolati e restrittivi rispetto ai database di tipo \gloss{relazionale}. Questo approccio permette una maggiore semplicità del processo di modellazione dei dati, una migliore propensione allo \gloss{scaling orizzontale} del database e un controllo maggiore della disponibilità dei dati.}


\paginaGlossario{O}


\elemento{Oggetto-software}
{Definito come un \gloss{tipo-informatica} di dato astratto, un oggetto è un insieme di valori ed operazioni che permettono di manipolare tali valori, dette operazioni proprie del tipo o metodi del tipo.}

\elemento{Open source}
{\gloss{Software} i cui autori ne permettono e favoriscono il libero studio e l'apporto di modifiche da parte di altri \gloss{programmatori} indipendenti.}

\elemento{Orientamento agli oggetti}
{\gloss{Paradigma di programmazione} che permette di definire \gloss{oggetti-software} in grado di interagire gli uni con gli altri attraverso lo scambio di messaggi.}

\elemento{Orientamento ai documenti}
{Le \gloss{basi di dati} orientate ai documenti non memorizzano i dati in tabelle con campi uniformi per ogni \gloss{record} come invece succedeva nei \gloss{database} \gloss{relazionali}, ma ogni record è memorizzato come un documento che possiede determinate caratteristiche. Al documento può essere aggiunto un numero qualsiasi di campi ed essi possono anche contenere pezzi multipli di dati.}

\elemento{Orientamento funzionale}
{\gloss{Paradigma di programmazione} in cui il flusso di esecuzione del \gloss{programma} assume la forma di una serie di valutazioni di funzioni matematiche.}

\elemento{Output}
{Tradotto letteralmente in "messo fuori". Indica in senso stretto il risultato di una elaborazione ed in senso più ampio il risultato o l'insieme dei risultati prodotti a partire da un \gloss{input}.}

\elemento{Overflow}
{Errore che occorre quando si eccede la memoria allocata disponibile.}


\paginaGlossario{P}


\elemento{Package}
{Un package, definito principalmente in ambiente \gloss{Java}, è un meccanismo per organizzare \gloss{classi} all'interno di sottogruppi ordinati. I \gloss{programmatori} spesso usano i package per riunire classi logicamente correlate o che forniscono servizi simili.}

\elemento{Pagine web}
{Il modo in cui vengono rese disponibili all'\gloss{utente} finale le informazioni reperibili su \gloss{Internet}, tramite un  \gloss{web browser}. Un insieme di pagine web tra di loro correlate formano un \gloss{sito web}. Una pagina web si può suddividere in una parte relativa ai contenuti, una parte di \gloss{layout} e una parte dedicata al comportamento a seconda degli \gloss{input} dell'utente.}

\elemento{Paradigma client-server}
{Indica un'architettura di rete nella quale genericamente un computer client si connette ad un server per usufruire di un certo servizio, ad esempio la condivisione di una certa risorsa hardware o software con altri client.}

\elemento{Paradigma di programmazione}
{Stile fondamentale di \gloss{programmazione}, ovvero un insieme di strumenti concettuali forniti da un \gloss{linguaggio di programmazione} per la stesura del \gloss{codice sorgente} di un \gloss{programma}, definendo dunque il modo in cui il \gloss{programmatore} concepisce e percepisce il programma stesso.}

\elemento{Parser}
{Programma che esegue il compito del parsing.}

\elemento{Parsing}
{Processo che analizza uno stream continuo in input in modo da determinare la sua struttura grammaticale grazie ad una data grammatica formale.}

\elemento{Password}
{Parola di riconoscimento impiegata a scopo di sicurezza per garantire che l'uso di una risorsa sia concesso solo agli \gloss{utenti} autorizzati; costituita da una sequenza ordinata di caratteri \gloss{alfanumerici} e/o speciali (quali, per esempio, @, \%, \$).}

\elemento{Periferica}
{Una qualsiasi tipologia di dispositivo \gloss{hardware} del computer che si interfaccia in \gloss{input} e/o \gloss{output} con l'unità di elaborazione che sovrintende a tutte le funzioni del computer (\gloss{processore}).}

\elemento{Permesso}
{O autorizzazione, indica quali funzionalità possono competere.}

\elemento{Persistenza}
{Caratteristica dei dati di sopravvivere all'esecuzione del programma che li ha creati. La persistenza si riferisce in particolare alla possibilità di far sopravvivere delle strutture dati all'esecuzione di un singolo programma. Questa possibilità è raggiunta salvando i dati in uno storage non volante come i database.}

\elemento{Pianificare}
{Termine usato per prevedere in linea di massima quando compiere un'attività e/o una serie di attività.}

\elemento{Piattaforma Java}
{Piattaforma \gloss{software} sviluppata su specifiche e implementazioni da \gloss{Sun Microsystems} che è eseguibile su piattaforme \gloss{hardware} di diversa natura.}

\elemento{Plug in}
{\gloss{Programma} non autonomo che interagisce con un altro programma per ampliarne le funzionalità.}

\elemento{Portabilità}
{Processo di adattare un software cosicchè un programma eseguibile può essere creato per un ambiente di computazione che è differente dal proprio per cui era stato originariamente designato.}

\elemento{Proattivo}
{Tipo di approccio che permette di prevenire e anticipare i problemi e i bisogni futuri; necessita di pianificazione e di esperienza; approccio molto usato per ridurre il carico di lavoro nell'attività di Verifica.}

\elemento{Procedura}
{Indica un modo di procedere, cioè di operare o di comportarsi in determinate circostanze o per ottenere un certo risultato.}

\elemento{Processo}
{Insieme di attività correlate e coese che trasformano ingressi in uscite secondo regole fissate, consumando risorse nel farlo.}

\elemento{Processore}
{Detto anche unità di elaborazione. Esso è un tipo di dispositivo \gloss{hardware} del computer che si contraddistingue per essere dedicato all'esecuzione di istruzioni. In altri termini l'unità di elaborazione è il dispositivo che nel computer esegue materialmente l'elaborazione dati.}

\elemento{Profilo}
{Insieme di dati relativi ad un \gloss{utente} in un \gloss{sistema} informatico. Può contenere informazioni differenti, a seconda del contesto e delle necessità del sistema.}

\elemento{Programma}
{Insieme di istruzioni che, una volta eseguite su un computer, produce soluzioni per una data classe di problemi automatizzati.}

\elemento{Programmatore}
{Vedi \gloss{sviluppatore}.}

\elemento{Programmazione}
{Attività di sviluppo di \gloss{software}, consistente nella stesura di \gloss{codice sorgente}.}

\elemento{Protocollo di comunicazione}
{Insieme di regole formalmente descritte, definite al fine di favorire la comunicazione tra una o più entità.}

\elemento{Protocollo di sottoscrizione}

\elemento{Provider}
{La persona o l'organizzazione che fornisce un \gloss{servizio web}. "Provider", letteralmente tradotto, significa "fornitore".}

\elemento{Python}
{Linguaggio di programmazione ad alto livello, orientato agli oggetti, adatto, tra gli altri usi, per sviluppare applicazioni distribuite, scripting, computazione numerica e testing.}

\paginaGlossario{Q}


\elemento{Query}
{L'interrogazione da parte di un \gloss{utente} di un \gloss{database} per compiere determinate operazioni sui dati.}


\paginaGlossario{R}


\elemento{Record}
{Un oggetto di una \gloss{base di dati} strutturata in dati compositi, che contengono un insieme di campi o elementi, ciascuno dei quali possiede nome e \gloss{tipo} di dato propri.}

\elemento{Redmine}
{Software di project management gratis e open source; mette a disposizione un calendario per la pianificazione delle attività e offre la possibilità di visualizzare i diagrammi di Gantt; inoltre supporta il sistema di ticketing.}

\elemento{Registrazione}
{Azione tramite la quale un \gloss{utente} attraverso l'inserimento di alcuni dati richiesti, entra a far parte di un sistema al quale vuole registrarsi.}

\elemento{Relazionale}
{Modello logico di rappresentazione o strutturazione dei dati di un \gloss{database}. Si basa sulla teoria degli insiemi e sulla logica del primo ordine ed è strutturato intorno al concetto matematico di relazione.}

\elemento{Repository}
{Ambiente che offre la possibilità di salvataggio di dati per la sicurezza di essi, inoltre è possibile godere di funzionalità di \gloss{versionamento} per tener traccia della storia dei dati salvati.}

\elemento{Requisito}
{Capacità che un sistema \gloss{software} deve soddisfare per rispettare un contratto.}

\elemento{Rete}
{Serie di componenti, sistemi o entità interconnessi tra di loro.}

\elemento{Risorsa}
{Componente fisico o virtuale che un sistema richiede e che grazie ad esso offre una certa funzionalità.}

\elemento{Ruby}
{\gloss{Linguaggio di scripting} completamente a oggetti.}


\paginaGlossario{S}


\elemento{Scala}
{\gloss{Linguaggio di programmazione} che integra linguaggi \gloss{orientati agli oggetti} e \gloss{linguaggi funzionali}. Un programma \gloss{Scala}, una volta \gloss{compilato}, può essere eseguito su \gloss{JVM}.}

\elemento{Scaling orizzontale}
{Per scalabilità si intende la proprietà di un sistema di crescere o decrescere in base alle necessità. Nel mondo dei \gloss{database} scalare orizzontalmente una \gloss{base di dati} significa che anche se vado ad aumentare le componenti del sistema che accedono al database, questo non provoca interferenze, ed aumentando le componenti si parallelizza il carico di lavoro, diminuendo il peso del singolo.}

\elemento{Script}
{Programma scritto in un particolare linguaggio di programmazione. Lo script ha complessità relativamente bassa e non ha una propria interfaccia grafica.}

\elemento{Scope}
{Nome, tipicamente univoco, che all'interno di un programma ne identifica una particolare parte; lo scope è utilizzato quando si vuole fare riferimento a qualche parte del programma in particolare.}

\elemento{Semantica}
{Regole per la \gloss{sintassi}.}

\elemento{Server}
{Componente o sottosistema informatico di elaborazione che fornisce un qualunque tipo di \gloss{servizio} ad altre componenti (tipicamente chiamate \gloss{client}) che ne fanno una esplicita richiesta.}

\elemento{Server web}
{Entità \gloss{hardware} o \gloss{software} che rilascia \gloss{pagine web} al \gloss{client}.}

\elemento{Servizio}
{Insieme di funzionalità \gloss{software} che possono essere riutilizzate per differenti scopi.}

\elemento{Servizio web}
{Risorsa astratta che rappresenta la capacità di effettuare compiti che compongono una funzionalità coerente dal punto di vista delle entità di \gloss{provider} e dei richiedenti del servizio. Per essere utilizzato, un \gloss{servizio} deve essere realizzato da un \gloss{agente di provider} concreto.}

\elemento{Shard}
{Gruppo di dati, o una replica di se stessi, di un database in MongoDB che risiede su una o più macchine. L'insieme di tutti i shard compongono tutti i dati per il cluster.}

\elemento{Simil-chat}
{Vedi \gloss{chat}.}

\elemento{Singleton}
{Design patter creazionale, che assicura l'esistenza di un'unica di una classe.}

\elemento{Sintassi}
{Notazione semplice con vincoli.}

\elemento{Sistema di autenticazione}
{Sistema che gestisce un processo tramite il quale un computer, un \gloss{software} o un \gloss{utente}, verifica la corretta, o presunta, identità di un altro computer, utente, che vuole comunicare attraverso una connessione.}

\elemento{Sistema operativo}
{Insieme di componenti \gloss{software} che permettono l'utilizzo da parte di un \gloss{utente} di applicazioni installate su una data macchina.}

\elemento{Sito web}
{Insieme di \gloss{pagine web} correlate, ovvero una struttura \gloss{ipertestuale} di documenti che risiede, tramite \gloss{hosting}, su un \gloss{web} \gloss{server} e accessibile all'\gloss{utente} \gloss{client} che ne fa richiesta tramite un \gloss{browser} sul \gloss{World Wide Web} della \gloss{rete} \gloss{Internet}, digitando in esso il rispettivo \gloss{URL} o direttamente l'indirizzo \gloss{IP}.}

\elemento{Skype}
{Software proprietario gratuito di messaggistica istantanea e \gloss{VoIP}.}

\elemento{Social Network}
{Tradotto letteralmente in "rete sociale". Consiste in una struttura informatica che gestisce nel Web le reti basate su relazioni sociali. La struttura è identificata, ad esempio, per mezzo del sito web di riferimento della rete sociale.}

\elemento{Software}
{L'informazione o le informazioni utilizzate da uno o più sistemi informatici e memorizzare su uno o più supporti informatici. Queste informazioni possono essere \gloss{programmi}, dati oppure una combinazione di tutte e due.}

\elemento{Sottosistema}
{Sistema che fa parte di un sistema più complesso.}

\elemento{Stack}
{Indica un tipo di dato astratto che viene usato in diversi contesti per riferirsi a strutture dati; la modalità di accesso ai dati contenuti in uno stack è di tipo LIFO, ovvero Last In First Out.}

\elemento{Statement}
{Traducibile in "sequenza di istruzioni".}

\elemento{Strategy}
{Design pattern comportamentale, definisce una famiglia di algoritmi, rendendoli interscambiabili.}

\elemento{Struttura dati}
{Entità usata per organizzare un insieme di dati all'interno della memoria del computer.}

\elemento{Stub}
{Porzione di \gloss{codice} utilizzata in sostituzione di altre funzionalità \gloss{software}. Uno stub può simulare il comportamento di codice esistente, gli stub sono utili durante lo sviluppo di software e durante i test per i software.}

\elemento{Sun Microsystems}
{Azienda produttrice di \gloss{software}, nota per aver prodotto il \gloss{linguaggio di programmazione} \gloss{Java}.}

\elemento{Sviluppatore}
{\gloss{Programmatore} che si prende cura di uno o più aspetti del ciclo di vita del \gloss{software}. Questa figura può contribuire alla visione d'insieme del progetto ad un livello applicativo piuttosto che a livello di componenti o operazioni individuali di \gloss{programmazione} (la codifica).}


\paginaGlossario{T}


\elemento{Template}
{Documento nel quale è rappresentata una struttura generica o standard dove ci sono spazi temporaneamente bianchi da riempire in seguito. In italiano viene indicato come scheletro o modello di base.}

\elemento{Test}
{Indica una prova, ovvero una serie di operazioni effettuate su un prodotto per trovare malfunzionamenti o errori e correggerli, prima del rilascio finale del prodotto.}

\elemento{Testing}
{Attività di fare test.}

\elemento{Tex}
{\gloss{Programma} di tipografia digitale adatto alla stesura di testi matematici e scientifici.}

\elemento{TexMaker}
{\gloss{Editor} gratuito multi piattaforma per la scrittura di documenti in \gloss{Latex}.}

\elemento{Ticket}
{Resoconto corrente di un particolare problema, con il suo stato e altri dati rilevanti.}

\elemento{Tipo-informatica}
{Nome che indica l'insieme di valori che una \gloss{variabile}, o il risultato di un'espressione, possono assumere e le operazioni che su tali valori si possono effettuare.}

\elemento{Top-down}
{Approccio che consente di formulare una visione generale del problema senza andare nel dettaglio. Si comincia a decomporre il problema iniziale in sottoproblemi, fino ad arrivare a pezzi non scomponibili.}

\elemento{Tracciamento}
{Capacità di verificare la storia e la posizione di un elemento all'interno di un documento redatto.}

\elemento{Two Way Data-Binding}
{Capacità di vincolare i cambiamenti delle proprietà di un oggetto ai cambiamenti dell'interfaccia utente, e viceversa.}

\paginaGlossario{U}

\elemento{UML}

\elemento{Upload}
{Caricamento, ovvero il processo di trasmissione di un file da un \gloss{client} ad un \gloss{server}.}

\elemento{Upper bound}
{Tradotto letteralmente con "limite superiore". In informatica, si definisce upper bound il limite superiore di memoria occupata da una struttura dati, di solito un \gloss{array}.}

\elemento{URL}
{Acronimo di "Uniform Resource Locator". Sequenza di caratteri che identifica univocamente l'indirizzo di una risorsa in \gloss{internet}, tipicamente presente su un \gloss{host} \gloss{server}, come ad esempio un documento, un'immagine, un video, rendendola accessibile ad un \gloss{client} che ne faccia richiesta attraverso l'utilizzo di un \gloss{web} \gloss{browser}.}

\elemento{User}
{Vedi \gloss{utente}.}

\elemento{Username}
{Tradotto in "nome \gloss{utente}". \`{E} il nome fornito da un utente di un \gloss{servizio} informatico (solitamente \gloss{web}) per identificarsi e accedere così al dato servizio.}

\elemento{Utente}
{Colui che può usufruire di un \gloss{servizio} che gli viene messo a disposizione.}

\elemento{Utente business}
{Persona alla quale interessano le \gloss{pagine web} create dal \gloss{framework} \gloss{Maap}; esso può consultare o operare su queste pagine in base a degli specifici scopi o interessi.}

\elemento{Utente business autenticato}
{\gloss{Utente business} che ha effettuato con successo la procedura di \gloss{autenticazione} verso il sistema che contiene le pagine generate da \gloss{MaaP}.}

\elemento{Utente business autenticato amministratore}
{\gloss{Utente business autenticato} che ha funzionalità aggiuntive rispetto a degli \gloss{utenti business autenticati}.}

\elemento{Utente sviluppatore}
{Persona che utilizza il \gloss{framework} \gloss{Maap}.}


\paginaGlossario{V}


\elemento{Variabile (informatica)}
{Insieme di dati modificabili situati in una porzione di memoria (una o più locazioni di memoria) destinata a contenere dei dati. Una variabile è caratterizzata da un nome (inteso solitamente come una sequenza di caratteri e cifre) e da un \gloss{tipo}.}

\elemento{Versionamento}
{Gestione di \gloss{versioni} multiple di un insieme di informazioni.}

\elemento{Versione}
{Identificativo univoco che rappresenta la variante di un documento o di un componente \gloss{software}.}

\elemento{View}
{Componente del design pattern MVC; gestisce la logica di presentazione verso i vari utenti e cattura gli input dell'utente delegando ad un altro componente, il controller, l'elaborazione.}

\elemento{ViewModel}
{Proiezione del modello per una vista; effettua il binding con la vista e il modello.}

\elemento{VoIP}
{Acronimo di "Voice over IP", tradotto in "Voce tramite protocollo IP". Con VoIP si intende una tecnologia che rende possibile effettuare una conversazione telefonica sfruttando una connessione Internet o una qualsiasi altra rete dedicata a commutazione di pacchetto che utilizzi il protocollo IP senza connessione per il trasporto dati.}


\paginaGlossario{W}


\elemento{Web}
{Abbreviazione di "World Wide Web". Nel linguaggio comune è associato ad \gloss{"internet"}, ovvero è una \gloss{rete} di computer ad accesso pubblico, è il maggior sistema di comunicazione di massa che offre anche ulteriori servizi, come ad esempio condividere e cercare risorse.}

\elemento{Web-service}
{Sistema software progettato per supportare l'interoperabilità tra diversi elaboratori su una medesima rete ovvero in un contesto distribuito.}

\elemento{Widget}
{Componente grafico di un'interfaccia utente di un programma che ha lo scopo di facilitare all'utente l'interazione con il programma stesso.}

\elemento{World Wide Web}
{Vedi "\gloss{Web}".}
