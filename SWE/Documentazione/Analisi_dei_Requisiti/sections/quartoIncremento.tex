\subsection{Quarto incremento}
\begin{longtable}{|c|p{6cm}|c|c|}

\label{tab:Requisiti MaaP} \\
\toprule
\multicolumn{1}{|c}{\textbf{Requisito}} & \multicolumn{1}{|p{6cm}}{\textbf{Descrizione}}   & \multicolumn{1}{|c}{\textbf{Fonte}} & \multicolumn{1}{|c|}{\textbf{Caso d'uso}}\\
\midrule
\endfirsthead
\multicolumn{2}{l}{\footnotesize\itshape\tablename~\thetable: continua dalla pagina precedente} \\
\toprule
\multicolumn{1}{|c}{\textbf{Requisito}} & \multicolumn{1}{|p{6cm}}{\textbf{Descrizione}}   & \multicolumn{1}{|c}{\textbf{Fonte}} & \multicolumn{1}{|c|}{\textbf{Caso d'uso}}\\
\midrule
\endhead
\midrule
\multicolumn{2}{r}{\footnotesize\itshape\tablename~\thetable: continua nella prossima pagina} \\
\endfoot
\bottomrule
\multicolumn{2}{r}{\footnotesize\itshape\tablename~\thetable: si conclude dalla pagina precedente} \\
\endlastfoot
\midrule
RFF2
& Il sistema MaaP deve permettere all'utente sviluppatore di utilizzare un editor interno specializzato per la scrittura/modifica dei file di descrizione
& Capitolato
& UC1 \\
& & & UC1.4
\\

\midrule
RFF2.1
& Il sistema MaaP deve permettere all'utente sviluppatore di utilizzare un editor interno specializzato per la scrittura di un nuovo file di descrizione
& Interna
& UC1.4 \\
& & & UC1.4.1
\\

\midrule
RFF2.1.1
& Il sistema MaaP deve permettere all'utente sviluppatore di utilizzare un editor interno specializzato per scrivere il codice del file di descrizione che intende creare
& Interna
& UC1.4.1 \\
& & & UC1.4.1.1
\\

\midrule
RFF2.1.2
& Il sistema MaaP deve permettere all'utente sviluppatore di utilizzare un editor interno specializzato per salvare il codice scritto in modo permanente
& Interna
& UC1.4.1 \\
& & & UC1.4.1.2
\\

\midrule
RFF2.2
& Il sistema MaaP deve permettere all'utente sviluppatore di utilizzare un editor interno specializzato per la modifica di un file di descrizione esistente
& Interna
& UC1.4 \\
& & & UC1.4.2
\\

\midrule
RFF2.2.1
& Il sistema MaaP deve permettere all'utente sviluppatore di utilizzare un editor interno specializzato per modificare il codice di un file di descrizione esistente
& Interna
& UC1.4.2 \\
& & & UC1.4.2.1
\\

\midrule
RFF2.2.2
& Il sistema MaaP deve permettere all'utente sviluppatore di utilizzare un editor interno specializzato per salvare il codice modificato in modo permanente
& Interna
& UC1.4.2 \\
& & & UC1.4.2.2
\\

\midrule
RFF2.2.3
& Il sistema MaaP deve permettere all'utente sviluppatore di utilizzare un editor interno specializzato per annullare le modifiche al codice del file di descrizione modificato
& Interna
& UC1.4.2 \\
& & & UC1.4.2.3
\\
\midrule
RFF4.1.2.4
& Il sistema MaaP deve permettere all'utente sviluppatore di aggiungere dei pulsanti all'interno della pagina Collection-Index
& Capitolato
& UC1.2.1.2\\
& & & UC1.2.1.2.4
\\

\midrule
RFF4.1.3.2
& Il sistema MaaP deve permettere all'utente sviluppatore di aggiungere un pulsante all'interno della  pagina Document-Show
& Capitolato
& UC1.2.1.3\\
& & & UC1.2.1.3.2
\\

\midrule
RFF4.2.2.7
& Il sistema MaaP deve permettere all'utente sviluppatore di aggiungere dei pulsanti all'interno della pagina Collection-Index, specificando il nome del pulsante e l'azione che deve eseguire
& Capitolato
& UC1.2.1.2\\
& & & UC1.2.1.2.4
\\
\midrule
RFF4.2.2.8
& Il sistema MaaP deve permettere all'utente sviluppatore di eliminare dei pulsanti all'interno della pagina Collection-Index
& Capitolato
& UC1.2.2.2\\
& & & UC1.2.2.2.5
\\

\midrule
RFF4.2.3.3
& Il sistema MaaP deve permettere all'utente sviluppatore di aggiungere dei pulsanti all'interno della  pagina Document-Show, specificando il nome del pulsante e l'azione che deve eseguire
& Capitolato
& UC1.2.1.3\\
& & & UC1.2.1.3.2
\\
\midrule
RFF4.2.3.4
& Il sistema MaaP deve permettere all'utente sviluppatore di eliminare dei pulsanti all'interno della  pagina Document-Show
& Capitolato
& UC1.2.2.3\\
& & & UC1.2.2.3.2
\\

\midrule
RDF8
& L'utente business deve potersi registrare inserendo dei dati personali
& Verbale\_2013\_12\_05
& UC2\\
& & & UC2.1
\\

\midrule
RDF8.1
& L'utente business, per registrarsi, deve inserire una email non presente nel sistema
& Capitolato
& UC2.1\\
& & & UC2.1.1
\\

\midrule
RDF8.2
& L'utente business, per registrarsi, deve inserire una password
& Capitolato
& UC2.1\\
& & & UC2.1.2
\\

\midrule
RDF8.2.1
& La password per la registrazione deve essere alfanumerica e contenere almeno otto caratteri
& Interna
&
\\
\midrule
RDF10.2
& L'utente business autenticato deve poter modificare la visualizzazione dei Document
& Interna
& UC2.4\\
& & & UC2.4.2
\\


\midrule
RDF10.2.1
& L'utente business autenticato deve poter selezionare dei criteri per la visualizzazione
& Interna
& UC2.4.2\\
& & & UC2.4.2.1
\\

\midrule
RDF10.2.1.1
& L'utente business autenticato deve poter effettuare un ordinamento rispetto a una chiave
& Interna
& UC2.4.2.1\\
& & & UC2.4.2.1.1
\\

\midrule
RDF10.2.1.2
& L'utente business deve poter selezionare un numero massimo di Document da visualizzare per pagina
& Interna
& UC2.4.2.1\\
& & & UC2.4.2.1.2
\\


\midrule
RDF10.2.2
& L'utente business autenticato deve poter applicare un filtro alla visualizzazione dei Document
& Interna
& UC2.4.2\\
& & Verbale\_2013\_12\_05 & UC2.4.2.2
\\

\midrule
RDF10.2.3
& L'utente business autenticato deve poter annullare il filtro
& Interna
& UC2.4.2\\
& & Verbale\_2013\_12\_05 & UC2.4.2.3
\\


%REQUISITI MAAS
\midrule
RFF11.1
& Il sistema MaaS deve permettere all'utente di autenticarsi al sistema
& Interna
& UC3\\
& & & UC3.6\\

\midrule
RFF11.1.1
& Il sistema MaaS deve permettere all'utente di inserire il nome utente
& Interna
& UC3.6\\
& & & UC3.6.1\\

\midrule
RFF11.1.2
& Il sistema MaaS deve permettere all'utente di inserire la password
& Interna
& UC3.6\\
& & & UC3.6.2\\

\midrule
RFF11.2
& Il sistema MaaS deve permettere all'utente di registrarsi al sistema
& Interna
& UC3\\
& & & UC3.7\\

\midrule
RFF11.2.1
& Il sistema MaaS deve permettere all'utente di inserire la propria email
& Interna
& UC3.7\\
& & & UC3.7.1\\

\midrule
RFF11.3
& Il sistema MaaS deve permettere all'utente di recuperare la password
& Interna
& UC3\\
& & & UC3.8\\

\midrule
RFF11.4
& Il sistema MaaS deve permettere all'utente di visualizzare le pagine web create
& Capitolato
& UC3\\
& & & UC3.5\\

\midrule
RFF11.5
& Il sistema MaaS deve permettere all'utente autenticato di creare lo scheletro del progetto
& Interna
& UC3\\
& & & UC3.1\\

\midrule
RFF11.5.1
& Il sistema MaaS deve permettere all'utente autenticato di inserire il nome del progetto
& Interna
& UC3.1\\
& & & UC3.1.1\\

\midrule
RFF11.6
& Il sistema MaaS deve permettere all'utente autenticato di gestire le pagine web
& Capitolato
& UC3\\
& & & UC3.2\\

\midrule
RFF11.6.1
& Il sistema MaaS deve permettere all'utente autenticato di creare un file di descrizione
& Capitolato
& UC3.2\\
& & & UC3.2.1\\

\midrule
RFF11.6.1.1
& Il sistema MaaS deve permettere all'utente autenticato la scrittura di un file di descrizione tramite editor di testo
& Capitolato
& UC3.2.1\\
& & & UC3.2.1.1\\

\midrule
RFF11.6.1.2
& Il sistema MaaS deve permettere all'utente autenticato di salvare il file di descrizione
& Interna
& UC3.2.1\\
& & & UC3.2.1.2\\


\midrule
RFF11.6.2
& Il sistema MaaS deve permettere all'utente autenticato di eseguire l'upload di un file di descrizione creato precedentemente con il sistema MaaP
& Capitolato
& UC3.2\\
& & & UC3.2.2\\

\midrule
RFF11.6.2.1
& Il sistema MaaS deve permettere all'utente autenticato di navigare all'interno del file system
& Interna
& UC3.2.2\\
& & & UC3.2.2.1\\

\midrule
RFF11.6.2.2
& Il sistema MaaS deve permettere all'utente autenticato di selezionare un file di descrizione
& Interna
& UC3.2.2\\
& & & UC3.2.2.2\\

\midrule
RFF11.6.2.3
& Il sistema MaaS deve permettere all'utente autenticato di confermare l'upload del file selezionato
& Interna
& UC3.2.2\\
& & & UC3.2.2.3\\


\midrule
RFF11.6.3
& Il sistema MaaS deve permettere all'utente autenticato di modificare un file di descrizione esistente
& Interna
& UC3.2\\
& & & UC3.2.3\\

\midrule
RFF11.6.3.1
& Il sistema MaaS deve permettere all'utente autenticato di modificare il codice del file di descrizione selezionato
& Interna
& UC3.2.3\\
& & & UC3.2.3.1\\

\midrule
RFF11.6.3.2
& Il sistema MaaS deve permettere all'utente autenticato di salvare le modifiche apportate al file di descrizione
& Interna
& UC3.2.3\\
& & & UC3.2.3.2\\

\midrule
RFF11.6.3.3
& Il sistema MaaS deve permettere all'utente autenticato di annullare le modifiche apportate al file di descrizione
& Interna
& UC3.2.3\\
& & & UC3.2.3.3\\

\midrule
RFF11.6.4
& Il sistema MaaS deve permettere all'utente autenticato di modificare il file di configurazione
& Capitolato
& UC3.2\\
& & & UC3.2.4\\

\midrule
RFF11.6.4.1
& Il sistema MaaS deve permettere all'utente autenticato di modificare il codice del file di configurazione selezionato
& Interna
& UC3.2.4\\
& & & UC3.2.4.1\\

\midrule
RFF11.6.4.2
& Il sistema MaaS deve permettere all'utente autenticato di salvare le modifiche apportate al file di configurazione
& Interna
& UC3.2.4\\
& & & UC3.2.4.2\\

\midrule
RFF11.6.4.3
& Il sistema MaaS deve permettere all'utente autenticato di annullare le modifiche apportate al file di configurazione
& Interna
& UC3.2.4\\
& & & UC3.2.4.3\\

\midrule
RFF11.6.5
& Il sistema MaaS deve permettere all'utente autenticato di visualizzare tutti i file di descrizione presenti
& Capitolato
& UC3.2\\
& & & UC3.2.5\\

\midrule
RFF11.7
& Il sistema MaaS deve permettere all'utente autenticato di gestire il proprio profilo utente
& Interna
& UC3\\
& & & UC3.3\\

\midrule
RFF11.7.1
& Il sistema MaaS deve permettere all'utente autenticato di generare una nuova password
& Interna
& UC3.3\\
& & & UC3.3.1\\

\midrule
RFF11.8
& Il sistema MaaS deve permettere all'utente autenticato di disconnettersi dal sistema
& Interna
& UC3\\
& & & UC3.4\\

\midrule
RFF11.9
& Il sistema MaaS deve permettere all'utente autenticato di visualizzare le pagine web create
& Capitolato
& UC3\\
& & & UC3.5\\

\end{longtable}