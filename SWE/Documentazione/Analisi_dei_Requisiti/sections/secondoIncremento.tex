
\subsection{Secondo incremento}
\begin{longtable}{|c|p{6cm}|c|c|}

\label{tab:Requisiti MaaP} \\
\toprule
\multicolumn{1}{|c}{\textbf{Requisito}} & \multicolumn{1}{|p{6cm}}{\textbf{Descrizione}}   & \multicolumn{1}{|c}{\textbf{Fonte}} & \multicolumn{1}{|c|}{\textbf{Caso d'uso}}\\
\midrule
\endfirsthead
\multicolumn{2}{l}{\footnotesize\itshape\tablename~\thetable: continua dalla pagina precedente} \\
\toprule
\multicolumn{1}{|c}{\textbf{Requisito}} & \multicolumn{1}{|p{6cm}}{\textbf{Descrizione}}   & \multicolumn{1}{|c}{\textbf{Fonte}} & \multicolumn{1}{|c|}{\textbf{Caso d'uso}}\\
\midrule
\endhead
\midrule
\multicolumn{2}{r}{\footnotesize\itshape\tablename~\thetable: continua nella prossima pagina} \\
\endfoot
\bottomrule
\multicolumn{2}{r}{\footnotesize\itshape\tablename~\thetable: si conclude dalla pagina precedente} \\
\endlastfoot

% Requisiti utente sviluppatore

\midrule
ROF4.1.1
& Il sistema MaaP deve permettere all'utente sviluppatore di creare la visualizzazione del menù per le Collection
& Capitolato
& UC1.2.1\\
& & & UC1.2.1.1
\\
\midrule
ROF4.1.1.1
& Il sistema MaaP deve permettere all'utente sviluppatore di definire il nome della voce relativa alla Collection
& Capitolato
& UC1.2.1.1\\
& & & UC1.2.1.1.1
\\
\midrule
ROF4.1.1.2
& Il sistema MaaP deve permettere all'utente sviluppatore di definire la posizione di una voce all'interno del menù
& Capitolato
& UC1.2.1.1\\
& & & 1.2.1.1.2
\\

\midrule
ROF4.1.2.2
& Il sistema MaaP deve permettere all'utente sviluppatore di definire un ordinamento rispetto a una chiave
& Capitolato
& UC1.2.1.2\\
& & & UC1.2.1.2.2
\\
\midrule
ROF4.1.2.3
& Il sistema deve permettere all'utente sviluppatore di  definire un numero massimo di Document da visualizzare per la pagina Collection-Index
& Capitolato
& UC1.2.1.2\\
& & & UC1.2.1.2.3
\\

\midrule
ROF4.2.1
& Il sistema MaaP deve permettere all'utente sviluppatore di impostare la visualizzazione del menù delle le Collection
& Capitolato
& UC1.2.2\\
& & & UC1.2.2.1
\\
\midrule
ROF4.2.1.1
& Il sistema MaaP deve permettere all'utente sviluppatore di modificare il nome della voce relativa alla Collection
& Capitolato
& UC1.2.2.1\\
& & & UC1.2.2.1.1
\\
\midrule
ROF4.2.1.2
& Il sistema MaaP deve permettere all'utente sviluppatore di modificare la posizione di una voce all'interno del menù
& Capitolato
& UC1.2.2.1\\
& & & UC1.2.2.1.2
\\

\midrule
ROF4.2.2.3
& Il sistema MaaP deve permettere all'utente sviluppatore di definire un ordinamento, alfabetico crescente o decrescente, rispetto a una chiave
& Capitolato
& UC1.2.1.2\\
& & & UC1.2.1.2.2
\\
\midrule
ROF4.2.2.4
& Il sistema MaaP deve permettere all'utente sviluppatore di eliminare un ordinamento rispetto a una chiave
& Interna
& UC1.2.2.2\\
& & & UC1.2.2.2.2
\\
\midrule
ROF4.2.2.5
& Il sistema deve permettere all'utente sviluppatore di  definire un numero massimo di Document da visualizzare per la pagina Collection-Index
& Capitolato
& UC1.2.1.2\\
& & & UC1.2.1.2.3
\\
\midrule
ROF4.2.2.6
& Il sistema deve permettere all'utente sviluppatore di  eliminare il numero massimo di Document da visualizzare per la pagina Collection-Index
& Capitolato
& UC1.2.2.2\\
& & & UC1.2.2.2.4
\\
\end{longtable}
