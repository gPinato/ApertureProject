\subsection{Consuntivo a finire}
Questa sezione contiene il prospetto economico che riporta le spese effettivamente sostenute. Vengono riportate le ore impiegate per svolgere i compiti preventivati. In base alla differenza di ore tra il preventivo ed il consuntivo, detta conguaglio, avremmo un bilancio:
\begin{itemize}
\item \grassetto{Positivo}: se il preventivo ha superato il consuntivo;
\item \grassetto{Negativo}: se il consuntivo ha superato il preventivo;
\item \grassetto{In pari}: se preventivo e consuntivo coincidono.
\end{itemize}
\subsubsection{Analisi}
Di seguito è riportato il consuntivo della fase di Analisi.\\
La tabella sottostante riporta la differenza delle ore tra preventivo e consuntivo, divise per ruolo.

\immagine{./tabelle/differenza_ruoli}{Costo per ruolo, fase di Analisi}

Nella tabella seguente sono riportate le differenze tra le ore di lavoro previste per ogni componente con quelle realmente impiegate.

\immagine{./tabelle/differenza_preventivo_consuntivo}{Differenza consuntivo preventivo per componente, fase di Analisi}

In conclusione, come si può notare dai valori presenti nelle precedenti tabelle, è stata impiegata un'ora in più per svolgere le attività programmate con un bilancio in passivo di euro 35.


