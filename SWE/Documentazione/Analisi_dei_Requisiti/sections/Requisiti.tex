%tabella requisiti codice requisito, descrizione, fonte, UC di riferimento

\subsection{Requisiti funzionali MaaP}
\begin{longtable}{|c|p{6cm}|c|c|}
\caption{Requisiti funzionali MaaP}
\label{tab:Requisiti MaaP} \\
\toprule
\multicolumn{1}{|c}{\textbf{Requisito}} & \multicolumn{1}{|p{6cm}}{\textbf{Descrizione}}   & \multicolumn{1}{|c}{\textbf{Fonte}} & \multicolumn{1}{|c|}{\textbf{Caso d'uso}}\\
\midrule
\endfirsthead
\multicolumn{2}{l}{\footnotesize\itshape\tablename~\thetable: continua dalla pagina precedente} \\
\toprule
\multicolumn{1}{|c}{\textbf{Requisito}} & \multicolumn{1}{|p{6cm}}{\textbf{Descrizione}}   & \multicolumn{1}{|c}{\textbf{Fonte}} & \multicolumn{1}{|c|}{\textbf{Caso d'uso}}\\
\midrule
\endhead
\midrule
\multicolumn{2}{r}{\footnotesize\itshape\tablename~\thetable: continua nella prossima pagina} \\
\endfoot
\bottomrule
\multicolumn{2}{r}{\footnotesize\itshape\tablename~\thetable: si conclude dalla pagina precedente} \\
\endlastfoot

% Requisiti utente sviluppatore

\midrule
ROF1
& Il framework MaaP deve essere installato nel computer in uso
& Interna
& 
\\

\midrule
ROF2
& Il sistema MaaP deve essere in grado di generare lo scheletro del progetto
& Capitolato
& UC1 \\
& & & UC1.1
\\
\midrule
ROF2.1
& Il sistema MaaP deve generare le librerie necessarie al funzionamento del progetto 
& Capitolato
&
\\
\midrule
ROF2.2
& Il sistema MaaP deve generare i file di configurazione necessari al funzionamento del progetto
& Capitolato
&
\\
\midrule
ROF2.3
& Il sistema MaaP deve generare le directory di descrizione delle pagine web
& Capitolato
&
\\
\midrule
ROF2.4
& Il sistema MaaP deve generare il sistema di autenticazione per le pagine web 
& Capitolato
&
\\
\midrule
ROF2.4.1
& Il sistema di autenticazione per le pagine web deve essere generato insieme ad un profilo amministratore di default
& Verbale\_2013\_12\_05
&
\\
\midrule
ROF3
& Il sistema MaaP deve permettere all'utente sviluppatore di utilizzare un editor specializzato per la scrittura del file di descrizione
& Capitolato
& UC1 \\
& & & UC1.4
\\
\midrule
ROF4
& Il sistema MaaP deve permette all'utente sviluppatore di inserire un file di descrizione
& Interna
& UC1\\
& & & UC1.5
\\

\midrule
ROF5
& Il sistema deve permette all'utente sviluppatore di utilizzare un file di descrizione
& Capitolato
& UC1 \\
& & & UC1.2
\\
\midrule
ROF5.1
& Il sistema MaaP deve permettere all'utente sviluppatore di creare la visualizzazione della Collection
& Capitolato
& UC1.2 \\
& & & UC1.2.1
\\
\midrule
ROF5.1.1
& Il sistema MaaP deve permettere all'utente sviluppatore di creare la visualizzazione del menu' per le Collection
& Capitolato
& UC1.2.1\\
& & & UC1.2.1.1
\\
\midrule
ROF5.1.1.1
& Il sistema MaaP deve permettere all'utente sviluppatore di definire il nome della voce relativa alla Collection
& Capitolato
& UC1.2.1.1\\
& & & UC1.2.1.1.1
\\
\midrule
ROF5.1.1.2
& Il sistema MaaP deve permettere all'utente sviluppatore di definire la posizione di una voce all'interno del menu'
& Capitolato
& UC1.2.1.1\\
& & & 1.2.1.1.2
\\
\midrule
ROF5.1.2
& Il sistema MaaP deve permettere all'utente sviluppatore di creare la visualizzazione della pagina Collection-Index
& Capitolato
& UC1.2.1\\
& & & UC1.2.1.2
\\
\midrule
ROF5.1.2.1
& Il sistema MaaP deve permettere all'utente sviluppatore di aggiungere delle chiavi da visualizzare nella pagina Collection-Index
& Capitolato
& UC1.2.1.2\\
& & & UC1.2.1.2.1
\\
\midrule
ROF5.1.2.2
& Il sistema MaaP deve permettere all'utente sviluppatore di definire un ordinamento rispetto a una chiave
& Capitolato
& UC1.2.1.2\\
& & & UC1.2.1.2.2
\\
\midrule
ROF5.1.2.3
& Il sistema deve permettere all'utente sviluppatore di  definire un numero massimo di Document da visualizzare per la pagina Collection-Index
& Capitolato
& UC1.2.1.2\\
& & & UC1.2.1.2.3
\\
\midrule
RFF5.1.2.4
& Il sistema MaaP deve permettere all'utente sviluppatore di aggiungere dei pulsanti all'interno della pagina Collection-Index 
& Capitolato
& UC1.2.1.2\\
& & & UC1.2.1.2.4
\\
\midrule
ROF5.1.3
& Il sistema MaaP deve permettere all'utente sviluppatore di creare la visualizzazione per la pagine Document-Show
& Capitolato
& UC1.2.1\\
& & & UC1.2.1.3
\\
\midrule
ROF5.1.3.1
& Il sistema MaaP deve permettere all'utente sviluppatore di aggiungere delle chiavi da visualizzare nella pagina Document-Show
& Capitolato
& UC1.2.1.3\\
& & & UC1.2.1.3.1
\\
\midrule
RFF5.1.3.2
& Il sistema MaaP deve permettere all'utente sviluppatore di aggiungere un pulsante all'interno della  pagina Document-Show
& Capitolato
& UC1.2.1.3\\
& & & UC1.2.1.3.2
\\
\midrule
ROF5.2
& Il sistema MaaP deve permettere all'utente sviluppatore di modificare la visualizzazione della Collection
& Capitolato
& UC1.2\\
& & & UC1.2.2
\\


\midrule
ROF5.2.1
& Il sistema MaaP deve permettere all'utente sviluppatore di impostare la visualizzazione del menu' delle le Collection
& Capitolato
& UC1.2.2\\
& & & UC1.2.2.1
\\
\midrule
ROF5.2.1.1
& Il sistema MaaP deve permettere all'utente sviluppatore di modificare il nome della voce relativa alla Collection
& Capitolato
& UC1.2.2.1\\
& & & UC1.2.2.1.1
\\
\midrule
ROF5.2.1.2
& Il sistema MaaP deve permettere all'utente sviluppatore di modificare la posizione di una voce all'interno del menu'
& Capitolato
& UC1.2.2.1\\
& & & UC1.2.2.1.2
\\
\midrule
ROF5.2.2
& Il sistema MaaP deve permettere all'utente sviluppatore di impostare la visualizzazione della pagina Collection-Index
& Capitolato
& UC1.2.2\\
& & & UC1.2.2.2
\\
\midrule
ROF5.2.2.1
& Il sistema MaaP deve permettere all'utente sviluppatore di aggiungere delle chiavi da visualizzare nella pagina Collection-Index
& Capitolato
& UC1.2.1.2\\
& & & UC1.2.1.2.1
\\
\midrule
ROF5.2.2.2
& Il sistema MaaP deve permettere all'utente sviluppatore di eliminare delle chiavi da visualizzare nella pagina Collection-Index
& Interna
& UC1.2.2.2\\
& & & UC1.2.2.2.1
\\
\midrule
ROF5.2.2.3
& Il sistema MaaP deve permettere all'utente sviluppatore di definire un ordinamento rispetto a una chiave
& Capitolato
& UC1.2.1.2\\
& & & UC1.2.1.2.2
\\
\midrule
ROF5.2.2.4
& Il sistema MaaP deve permettere all'utente sviluppatore di eliminare un ordinamento rispetto a una chiave
& Interna
& UC1.2.2.2\\
& & & UC1.2.2.2.2
\\
\midrule
ROF5.2.2.5
& Il sistema deve permettere all'utente sviluppatore di  definire un numero massimo di Document da visualizzare per la pagina Collection-Index
& Capitolato
& UC1.2.1.2\\
& & & UC1.2.1.2.3
\\
\midrule
ROF5.2.2.6
& Il sistema deve permettere all'utente sviluppatore di  eliminare il numero massimo di Document da visualizzare per la pagina Collection-Index
& Capitolato
& UC1.2.2.2\\
& & & UC1.2.2.2.4
\\
\midrule
RFF5.2.2.7
& Il sistema MaaP deve permettere all'utente sviluppatore di aggiungere dei pulsanti all'interno della pagina Collection-Index 
& Capitolato
& UC1.2.1.2\\
& & & UC1.2.1.2.4
\\
\midrule
RFF5.2.2.8
& Il sistema MaaP deve permettere all'utente sviluppatore di eliminare dei pulsanti all'interno della pagina Collection-Index 
& Capitolato
& UC1.2.2.2\\
& & & UC1.2.2.2.5
\\
\midrule
ROF5.2.3
& Il sistema MaaP deve permettere all'utente sviluppatore di impostare la visualizzazione per la pagine Document-Show
& Capitolato
& UC1.2.2\\
& & & UC1.2.2.3
\\
\midrule
ROF5.2.3.1
& Il sistema MaaP deve permettere all'utente sviluppatore di aggiungere delle chiavi da visualizzare nella pagina Document-Show
& Capitolato
& UC1.2.1.3\\
& & & UC1.2.1.3.1
\\
\midrule
ROF5.2.3.2
& Il sistema MaaP deve permettere all'utente sviluppatore di eliminare delle chiavi da visualizzare nella pagina Document-Show
& Capitolato
& UC1.2.2.3\\
& & & UC1.2.2.3.1
\\

\midrule
RFF5.2.3.3
& Il sistema MaaP deve permettere all'utente sviluppatore di aggiungere dei pulsanti all'interno della  pagina Document-Show
& Capitolato
& UC1.2.1.3\\
& & & UC1.2.1.3.2
\\
\midrule
RFF5.2.3.4
& Il sistema MaaP deve permettere all'utente sviluppatore di eliminare dei pulsante all'interno della  pagina Document-Show
& Capitolato
& UC1.2.2.3\\
& & & UC1.2.2.3.2
\\

\midrule
ROF6
& Il sistema deve permettere all'utente sviluppatore la modifica dei file di configurazione
& Interna
& UC1.3
\\

\midrule
RDF6.1
& Il sistema deve permettere all'utente sviluppatore di abilitare la funzionalita' di registrazione nelle pagine web 
& Verbale\_2013\_12\_05
& UC1.3\\
& & & UC1.3.1
\\

\midrule
RDF6.2
& Il sistema MaaP deve permettere all'utente sviluppatore di abilitare la funzionalita' per la creazione di nuovi Document all'interno della pagina Collection-Index
& Capitolato
& UC1.3\\
& &  & UC1.3.2
\\

\midrule
RDF6.3
& Il sistema MaaP deve permettere all'utente sviluppatore di modificare i template per le pagine web
& Interna
& UC1.3\\
& & & UC1.3.3
\\

\midrule
ROF6.4
& Il sistema deve permettere all'utente sviluppatore di specificare il database di database di analisi con il quale interagire
& Interna
& UC1.3\\
& & & UC1.3.4
\\

\midrule
ROF6.5
& Il sistema deve permettere all'utente sviluppatore di abilitare la funzionalita' per la creazione di nuovi indici all'interno della pagina Collection-Index
& Capitolato
& UC1.3\\
& & Verbale\_2013\_12\_05 & UC1.3.5 
\\


%FINE TABELLA REQUISITI MAAP, NON CANCELLARE
\end{longtable}

\newpage
\subsection{Requisiti funzionali MaaP's Web}
\begin{longtable}{|c|p{6cm}|c|c|}
\caption{Requisiti funzionali MaaP's Web}
\label{tab:Requisiti MaaP's Web} \\
\toprule
\multicolumn{1}{|c}{\textbf{Requisito}} & \multicolumn{1}{|p{6cm}}{\textbf{Descrizione}}   & \multicolumn{1}{|c}{\textbf{Fonte}} & \multicolumn{1}{|c|}{\textbf{Caso d'uso}}\\
\midrule
\endfirsthead
\multicolumn{2}{l}{\footnotesize\itshape\tablename~\thetable: continua dalla pagina precedente} \\
\toprule
\multicolumn{1}{|c}{\textbf{Requisito}} & \multicolumn{1}{|p{6cm}}{\textbf{Descrizione}}   & \multicolumn{1}{|c}{\textbf{Fonte}} & \multicolumn{1}{|c|}{\textbf{Caso d'uso}}\\
\midrule
\endhead
\midrule
\multicolumn{2}{r}{\footnotesize\itshape\tablename~\thetable: continua nella prossima pagina} \\
\endfoot
\bottomrule
\multicolumn{2}{r}{\footnotesize\itshape\tablename~\thetable: si conclude dalla pagina precedente} \\
\endlastfoot

%Requisiti utente business
\midrule
ROF7
& L'utente business, al primo accesso, deve poter usare il profilo amministratore di default
& Verbale\_2013\_12\_05
& 
\\

\midrule
ROF8
& L'utente business deve potersi autenticare
& Capitolato
& UC2\\
& & & UC2.2
\\

\midrule
ROF8.1
& L'utente business deve inserire email e password per l'autenticazione
& Capitolato
& UC2\\
& & & UC2.7
\\

\midrule
ROF8.1.1
& La password per l'autenticazione deve essere alfanumerica e contenere almeno otto caratteri
& Interna
& 
\\

\midrule
RDF9
& L'utente business deve potersi registrare
& Verbale\_2013\_12\_05
& UC2\\
& & & UC2.1
\\

\midrule
RDF9.1
& L'utente business, per registrarsi, deve inserire una mail e una password
& Capitolato
& UC2\\
& & & UC2.7
\\

\midrule
RDF9.1.1
& La password per la registrazione deve essere alfanumerica e contenere almeno otto caratteri
& Interna
& 
\\

\midrule
ROF10
& L'utente business deve poter recuperare la password
& Capitolato
& UC2\\
& & & UC2.3
\\

\midrule
ROF11
& L'utente business autenticato deve poter aprire una Collection e visualizzare la sua pagina Collection-Index
& Capitolato
& UC2\\
& & & UC2.4
\\

\midrule
ROF11.1
& L'utente business autenticato deve poter visualizzare una pagina Document-Show
& Capitolato
& UC2.4\\
& & & UC2.4.1
\\

\midrule
RDF11.2
& L'utente business autenticato deve poter modificare la visualizzazione dei Document
& Interna
& UC2.4\\
& & & UC2.4.2
\\

\midrule
RDF11.2.1
& L'utente business autenticato deve poter selezionare dei criteri per la visualizzazione
& Interna
& UC2.4.2\\
& & & UC2.4.2.1
\\

\midrule
RDF11.2.1.1
& L'utente business autenticato deve poter effettuare un ordinamento rispetto a una chiave
& Interna
& UC2.4.2.1\\
& & & UC2.4.2.1.1
\\

\midrule
RDF11.2.1.2
& L'utente business deve poter selezionare un numero massimo di Document da visualizzare per pagina
& Interna
& UC2.4.2.1\\
& & & UC2.4.2.1.2
\\


\midrule
RDF11.2.2
& L'utente business autenticato deve poter applicare un filtro alla visualizzazione dei Document
& Interna
& UC2.4.2\\
& & Verbale\_2013\_12\_05 & UC2.4.2.2
\\

\midrule
RDF11.2.3
& L'utente business autenticato deve poter annullare il filtro
& Interna
& UC2.4.2\\
& & Verbale\_2013\_12\_05 & UC2.4.2.3
\\

\midrule
ROF11.2.4
& L'utente business autenticato deve poter disconnettersi 
& Interna
& UC2.4\\
& & & UC2.5
\\

\midrule
ROF11.2.5
& L'utente business autenticato deve poter navigare tra la Collection
& Capitolato
& UC2.4\\
& & & UC2.4.3\\

\midrule
ROF11.3
& L'utente business autenticato deve poter gestire il proprio profilo
& Interna
& UC2\\
& & & UC2.6
\\

\midrule
ROF11.3.1
& L'utente business autenticato deve poter gestire i propri dati
& Capitolato
& UC2.6\\
& & & UC2.6.3
\\

\midrule
ROF11.3.1.1
& L'utente business autenticato deve poter modificare i propri dati utente
& Capitolato
& UC2.6.3\\
& & & UC2.6.3.1
\\

\midrule
ROF11.3.1.2
& L'utente business autenticato deve poter  salvare le modifiche apportate
& Interna
& UC2.6.3\\
& & & UC2.6.3.2
\\

\midrule
ROF11.3.1.3
& L'utente business autenticato deve poter annullare le modifiche apportate
& Interna
& UC2.6.3\\
& & & UC2.6.3.5
\\

\midrule
ROF11.3.1.4
& L'utente business autenticato amministratore deve poter modificare i dati degli utenti business
& Interna
& UC2.6.3\\
& & & UC2.6.3.3
\\

\midrule
ROF11.3.1.5
& L'utente business autenticato deve potee modificare i permessi degli utenti business
& Capitolato
& UC2.6.3\\
& & & UC2.6.3.4
\\

\midrule
ROF11.3.2
& L'utente business autenticato amministratore deve poter creare un nuovo utene business
& Capitolato
& UC2.6\\
& & & UC2.6.1
\\

\midrule
ROF11.3.3
& L'utente business autenticato amministratore deve poter eliminare un utene business
& Capitolato
& UC2.6\\
& & & UC2.6.2
\\

\midrule
ROF11.4
& L'utente business autenticato amministratore deve poter cancellare un Document
& Interna
& UC2.4\\
& & & UC2.4.4
\\

\midrule
ROF11.5
& L'utente business autenticato amministratore deve poter modificare un Document
& Capitolato
& UC2.4\\
& & & UC2.4.5
\\

\midrule
ROF11.5.1
& L'utente business autenticato amministratore deve poter modificare i valori associati alla chiavi
& Interna
& UC2.4.5\\
& & & UC2.4.5.1
\\

\midrule
ROF11.5.2
& L'utente business autenticato amministratore deve poter salvare le modifiche apportate al Document
& Interna
& UC2.4.5\\
& & & UC2.4.5.2
\\

\midrule
ROF11.5.3
& L'utente business autenticato amministratore deve poter annullare le modifiche apportate al Document
& Interna
& UC2.4.5\\
& & & UC2.4.5.3
\\

\midrule
RFF11.6
& L'utente business autenticato amministratore deve porter visualizzare le query piu' utilizzate dal sistema MaaP
& Capitolato
& UC2.4\\
& & & UC2.4.6
\\

\midrule
RFF11.7
& L'utente business autenticato amministratore deve poter creare degli indici
& Capitolato
& UC2.4\\
& & & UC2.4.7
\\

%FINE TABELLA REQUISITI MAAPSWEB, NON CANCELLARE
\end{longtable}

\newpage
\subsection{Requisiti funzionali MaaS}
\begin{longtable}{|c|p{6cm}|c|c|}
\caption{Requisiti funzionali MaaS}
\label{tab:Requisiti MaaS} \\
\toprule
\multicolumn{1}{|c}{\textbf{Requisito}} & \multicolumn{1}{|p{6cm}}{\textbf{Descrizione}}   & \multicolumn{1}{|c}{\textbf{Fonte}} & \multicolumn{1}{|c|}{\textbf{Caso d'uso}}\\
\midrule
\endfirsthead
\multicolumn{2}{l}{\footnotesize\itshape\tablename~\thetable: continua dalla pagina precedente} \\
\toprule
\multicolumn{1}{|c}{\textbf{Requisito}} & \multicolumn{1}{|p{6cm}}{\textbf{Descrizione}}   & \multicolumn{1}{|c}{\textbf{Fonte}} & \multicolumn{1}{|c|}{\textbf{Caso d'uso}}\\
\midrule
\endhead
\midrule
\multicolumn{2}{r}{\footnotesize\itshape\tablename~\thetable: continua nella prossima pagina} \\
\endfoot
\bottomrule
\multicolumn{2}{r}{\footnotesize\itshape\tablename~\thetable: si conclude dalla pagina precedente} \\
\endlastfoot

%REQUISITI MAAS
\midrule
RFF12.1
& Il sistema MaaS deve permettere all'utente di autenticarsi al sistema
& Interna
& UC3\\
& & & UC3.6\\

\midrule
RFF12.1.1
& Il sistema MaaS deve permettere all'utente di inserire il nome utente
& Interna
& UC3.6\\
& & & UC3.6.1\\

\midrule
RFF12.1.2
& Il sistema MaaS deve permettere all'utente di inserire la password
& Interna
& UC3.6\\
& & & UC3.6.2\\

\midrule
RFF12.2
& Il sistema MaaS deve permettere all'utente di registrarsi al sistema
& Interna
& UC3\\
& & & UC3.7\\

\midrule
RFF12.2.1
& Il sistema MaaS deve permettere all'utente di inserire l'email
& Interna
& UC3.7\\
& & & UC3.7.1\\

\midrule
RFF12.3
& Il sistema MaaS deve permettere all'utente di recuperare la password
& Interna
& UC3\\
& & & UC3.8\\

\midrule
RFF12.4
& Il sistema MaaS deve permettere all'utente di visualizzare le pagine web create
& Capitolato
& UC3\\
& & & UC3.5\\

\midrule
RFF12.5
& Il sistema MaaS deve permettere all'utente autenticato di creare lo scheletro del progetto
& Interna
& UC3\\
& & & UC3.1\\

\midrule
RFF12.5.1
& Il sistema MaaS deve permettere all'utente autenticato di inserire il nome del progetto
& Interna
& UC3.1\\
& & & UC3.1.1\\

\midrule
RFF12.6
& Il sistema MaaS deve permettere all'utente autenticato di gestire le pagine web
& Capitolato
& UC3\\
& & & UC3.2\\

\midrule
RFF12.6.1
& Il sistema MaaS deve permettere all'utente autenticato di creare un file di descrizione
& Capitolato
& UC3.2\\
& & & UC3.2.1\\

\midrule
RFF12.6.1.1
& Il sistema MaaS deve permettere all'utente autenticato la scrittura di un file di descrizione tramite editor di testo
& Capitolato
& UC3.2.1\\
& & & UC3.2.1.1\\

\midrule
RFF12.6.1.2
& Il sistema MaaS deve permettere all'utente autenticato di salvare il file di descrizione
& Interna
& UC3.2.1\\
& & & UC3.2.1.2\\


\midrule
RFF12.6.2
& Il sistema MaaS deve permettere all'utente autenticato di eseguire l'upload di un file di descrizione creato precedentemente con il sistema MaaP
& Capitolato
& UC3.2\\
& & & UC3.2.2\\

\midrule
RFF12.6.2.1
& Il sistema MaaS deve permettere all'utente autenticato di navigare all'interno del filesystem
& Interna
& UC3.2.2\\
& & & UC3.2.2.1\\

\midrule
RFF12.6.2.2
& Il sistema MaaS deve permettere all'utente autenticato di selezionare un file di descrizione
& Interna
& UC3.2.2\\
& & & UC3.2.2.2\\

\midrule
RFF12.6.2.3
& Il sistema MaaS deve permettere all'utente autenticato di confermare l'upload del file selezionato
& Interna
& UC3.2.2\\
& & & UC3.2.2.3\\


\midrule
RFF12.6.3
& Il sistema MaaS deve permettere all'utente autenticato di modificare un file di descrizione esistente
& Interna
& UC3.2\\
& & & UC3.2.3\\

\midrule
RFF12.6.3.1
& Il sistema MaaS deve permettere all'utente autenticato di modificare il codice del file di descrizione selezionato
& Interna
& UC3.2.3\\
& & & UC3.2.3.1\\

\midrule
RFF12.6.3.2
& Il sistema MaaS deve permettere all'utente autenticato di salvare le modifiche apportate al file di descrizione
& Interna
& UC3.2.3\\
& & & UC3.2.3.2\\

\midrule
RFF12.6.3.3
& Il sistema MaaS deve permettere all'utente autenticato di annullare le modifiche apportate al file di descrizione
& Interna
& UC3.2.3\\
& & & UC3.2.3.3\\

\midrule
RFF12.6.4
& Il sistema MaaS deve permettere all'utente autenticato di modificare il file di configurazione
& Capitolato
& UC3.2\\
& & & UC3.2.4\\

\midrule
RFF12.6.4.1
& Il sistema MaaS deve permettere all'utente autenticato di modificare il codice del file di configurazione selezionato
& Interna
& UC3.2.4\\
& & & UC3.2.4.1\\

\midrule
RFF12.6.4.2
& Il sistema MaaS deve permettere all'utente autenticato di salvare le modifiche apportate al file di configurazione
& Interna
& UC3.2.4\\
& & & UC3.2.4.2\\

\midrule
RFF12.6.4.3
& Il sistema MaaS deve permettere all'utente autenticato di annullare le modifiche apportate al file di configurazione
& Interna
& UC3.2.4\\
& & & UC3.2.4.3\\

\midrule
RFF12.6.5
& Il sistema MaaS deve permettere all'utente autenticato di visualizzare tutti i file di descrizione presenti
& Capitolato
& UC3.2\\
& & & UC3.2.5\\

\midrule
RFF12.7
& Il sistema MaaS deve permettere all'utente autenticato di gestire il proprio profilo utente
& Interna
& UC3\\
& & & UC3.3\\

\midrule
RFF12.7.1
& Il sistema MaaS deve permettere all'utente autenticato di generare una nuova password
& Interna
& UC3.3\\
& & & UC3.3.1\\

\midrule
RFF12.8
& Il sistema MaaS deve permettere all'utente autenticato di disconnettersi dal sistema
& Interna
& UC3\\
& & & UC3.4\\

\midrule
RFF12.9
& Il sistema MaaS deve permettere all'utente autenticato di visualizzare le pagine web create
& Capitolato
& UC3\\
& & & UC3.5\\

%FINE TABELLA REQUISITI MAAS, NON CANCELLARE
\end{longtable}

\newpage
\subsection{Requisiti di qualita'}
\begin{longtable}{|c|p{6cm}|c|c|}
\caption{Requisiti di qualita'}
\label{tab:Requisiti di qualita'} \\
\toprule
\multicolumn{1}{|c}{\textbf{Requisito}} & \multicolumn{1}{|p{6cm}}{\textbf{Descrizione}}   & \multicolumn{1}{|c}{\textbf{Fonte}} & \multicolumn{1}{|c|}{\textbf{Caso d'uso}}\\
\midrule
\endfirsthead
\multicolumn{2}{l}{\footnotesize\itshape\tablename~\thetable: continua dalla pagina precedente} \\
\toprule
\multicolumn{1}{|c}{\textbf{Requisito}} & \multicolumn{1}{|p{6cm}}{\textbf{Descrizione}}   & \multicolumn{1}{|c}{\textbf{Fonte}} & \multicolumn{1}{|c|}{\textbf{Caso d'uso}}\\
\midrule
\endhead
\midrule
\multicolumn{2}{r}{\footnotesize\itshape\tablename~\thetable: continua nella prossima pagina} \\
\endfoot
\bottomrule
\multicolumn{2}{r}{\footnotesize\itshape\tablename~\thetable: si conclude dalla pagina precedente} \\
\endlastfoot

% Requisiti di qualita

\midrule
ROQ13
& Devono essere rispettate tutte le norme riportate nei documenti Norme di Progetto e Piano di Qualifica
& Interna
& 
\\

\midrule
ROQ14
& Il software deve essere dotato di un manuale utente che ne descriva l'utilizzo
& Interna
& 
\\

\midrule
ROQ15
& Il database degli utenti deve essere criptato
& Interna
& 
\\

%FINE TABELLA REQUISITI QUALITA', NON CANCELLARE
\end{longtable}\newpage
\subsection{Requisiti di qualita'}
\begin{longtable}{|c|p{6cm}|c|c|}
\caption{Requisiti di qualita'}
\label{tab:Requisiti di qualita'} \\
\toprule
\multicolumn{1}{|c}{\textbf{Requisito}} & \multicolumn{1}{|p{6cm}}{\textbf{Descrizione}}   & \multicolumn{1}{|c}{\textbf{Fonte}} & \multicolumn{1}{|c|}{\textbf{Caso d'uso}}\\
\midrule
\endfirsthead
\multicolumn{2}{l}{\footnotesize\itshape\tablename~\thetable: continua dalla pagina precedente} \\
\toprule
\multicolumn{1}{|c}{\textbf{Requisito}} & \multicolumn{1}{|p{6cm}}{\textbf{Descrizione}}   & \multicolumn{1}{|c}{\textbf{Fonte}} & \multicolumn{1}{|c|}{\textbf{Caso d'uso}}\\
\midrule
\endhead
\midrule
\multicolumn{2}{r}{\footnotesize\itshape\tablename~\thetable: continua nella prossima pagina} \\
\endfoot
\bottomrule
\multicolumn{2}{r}{\footnotesize\itshape\tablename~\thetable: si conclude dalla pagina precedente} \\
\endlastfoot

% Requisiti di qualita

\midrule
ROQ13
& Devono essere rispettate tutte le norme riportate nei documenti Norme di Progetto e Piano di Qualifica
& Interna
& 
\\

\midrule
ROQ14
& Il software deve essere dotato di un manuale utente che ne descriva l'utilizzo
& Interna
& 
\\

\midrule
ROQ15
& Il database degli utenti deve essere criptato
& Interna
& 
\\

%FINE TABELLA REQUISITI QUALITA', NON CANCELLARE
\end{longtable}

\newpage
\subsection{Requisiti di vincolo}
\begin{longtable}{|c|p{6cm}|c|c|}
\caption{Requisiti di vincolo}
\label{tab:Requisiti di vincolo} \\
\toprule
\multicolumn{1}{|c}{\textbf{Requisito}} & \multicolumn{1}{|p{6cm}}{\textbf{Descrizione}}   & \multicolumn{1}{|c}{\textbf{Fonte}} & \multicolumn{1}{|c|}{\textbf{Caso d'uso}}\\
\midrule
\endfirsthead
\multicolumn{2}{l}{\footnotesize\itshape\tablename~\thetable: continua dalla pagina precedente} \\
\toprule
\multicolumn{1}{|c}{\textbf{Requisito}} & \multicolumn{1}{|p{6cm}}{\textbf{Descrizione}}   & \multicolumn{1}{|c}{\textbf{Fonte}} & \multicolumn{1}{|c|}{\textbf{Caso d'uso}}\\
\midrule
\endhead
\midrule
\multicolumn{2}{r}{\footnotesize\itshape\tablename~\thetable: continua nella prossima pagina} \\
\endfoot
\bottomrule
\multicolumn{2}{r}{\footnotesize\itshape\tablename~\thetable: si conclude dalla pagina precedente} \\
\endlastfoot

% Requisiti di vincolo
\midrule
ROV16
& Le pagine web prodotte dal framework MaaP devono essere compatibili con la versione 30.0.x di Google Chrome o superiori
& Capitolato
& 
\\

\midrule
ROV17
& Le pagine web prodotte dal framework MaaP devono essere compatibili con la versione 24.x o superiore di Firefox 
& Capitolato
& 
\\

\midrule
ROV18
& Il sistema deve accettare solo file di configurazione che hanno un determinato formato già fissato
& Interna
& 
\\

\midrule
ROV19
& Il database degli utenti deve essere realizzato utilizzando MongoDB
& Capitolato
& 
\\

\midrule
ROV20
& Il database degli utenti deve essere indipendente dal database di analisi
& Capitolato
& 
\\

\midrule
ROV21
& Il database di analisi utilizzato deve essere stato realizzato utilizzando MongoDB
& Capitolato
& 
\\

\midrule
ROV22
& L'interfaccia  con il database deve essere realizzata con Mongoose
& Capitolato
& 
\\

\midrule
ROV23
& L'infrastruttura delle pagine web generate deve essere realizzata con Express
& Capitolato
& 
\\

\midrule
ROV24
& La componente server deve essere realizzata con Node.js
& Capitolato
& 
\\

\midrule
ROV25
& Definizione di un linguaggio astratto DSL per la definizione delle pagine che verranno generate
& Capitolato
& 
\\

\midrule
ROV25.1
& Il linguaggio definito deve essere testuale
& Capitolato
& 
\\

\midrule
ROV26
& Il software sarà fornito di un sistema di installazione per farlo funzionare
& Interna
& 
\\

\midrule
ROV27
& Deve essere possibile effettuare il deployment su Heroku 
& Capitolato
& 
\\

\midrule
ROV28
& Il progetto deve essere pubblicato su GitHub e utilizzare le sue distribuzioni per segnalare eventuali correzioni o errori
& Capitolato
& 
\\

%FINE TABELLA REQUISITI DI VINCOLO, NON CANCELLARE
\end{longtable}