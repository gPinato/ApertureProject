\UCtitle
{Caso d'uso UC1}
{MaaP}
\UCimmagine{UC1}{UC1 MaaP}
\UC
{UC1}
{Utente \gloss{sviluppatore}}
{L'\gloss{utente sviluppatore}  può eseguire dei comandi per generare lo scheletro, inserire il \gloss{file di descrizione}, usare un \gloss{editor} specializzato per scrivere il file di descrizione, utilizzare il file di descrizione e modificare il file di configurazione}
{Il sistema è installato e funzionante nel pc in uso}
\scenario{
L'utente sviluppatore può generare lo scheletro del progetto (UC1.1);|
L'utente sviluppatore può utilizzare il file di descrizione (UC1.2);|
L'utente sviluppatore può modificare i file di configurazione (UC1.3);|
L'utente sviluppatore può utilizzare un editor di testo specializzato (UC1.4);|
L'utente sviluppatore può inserire un file di descrizione (UC1.5);|
L'utente sviluppatore può eliminare un progetto (UC1.6);|
L'utente sviluppatore può clonare un progetto (UC1.7).
}
\post{Il sistema ha ottenuto le informazioni sulle operazioni che l'utente sviluppatore desidera eseguire}

\UCtitle
{Caso d'uso UC1.1}
{Generazione scheletro}
\UC
{UC1}
{Utente sviluppatore}
{L'utente sviluppatore può generare lo scheletro del progetto, il quale comprende: librerie e file di configurazione necessari al funzionamento dello stesso, directory di descrizione delle pagine web generate e il sistema di autenticazione per quest'ultime}
{Il sistema è installato e funzionante nel pc in uso e permette all'utente sviluppatore di generare scheletro di un nuovo progetto}
\scenario
{L'utente sviluppatore può generare lo scheletro del progetto.}
\post{Il sistema ha generato lo scheletro del progetto}



\UCtitle
{Caso d'uso UC1.2}
{Utilizzo file di descrizione}
\UCimmagine{UC1.2}{UC1.2 Utilizzo file di descrizione}
\UC
{UC1.2}
{Utente sviluppatore}
{L'utente sviluppatore  può usare il file di descrizione per definire la visualizzazione della Collection utilizzando i \gloss{template} delle pagine web inseriti nel file di configurazione}
{Il sistema ha generato lo scheletro del progetto e il file di descrizione deve essere stato inserito}
\scenario{
L'utente sviluppatore può creare la visualizzazione di una Collection (UC1.2.1);|
L'utente sviluppatore può modificare la visualizzazione di una Collection (UC1.2.2);|
L'utente sviluppatore può aggiungere una \gloss{query} (UC1.2.3);|
L'utente sviluppatore può eliminare una query (UC1.2.4).
}
\post{Il sistema contiene la descrizione per la visualizzazione della Collection}

\UCtitle
{Caso d'uso UC1.2.1}
{Creazione visualizzazione Collection}
\UCimmagine{UC1.2.1}{UC1.2.1 Creazione visualizzazione Collection}
\UC
{UC1.2.1}
{Utente sviluppatore}
{L'utente sviluppatore  può impostare la visualizzazione del menù delle Collection, delle pagine Collection-Index e Document-Show}
{Il sistema contiene la descrizione per la visualizzazione della Collection predefinita e l'utente sviluppatore desidera crearne una nuova}
\scenario{
L'utente sviluppatore può impostare la visualizzazione del menù per le pagine di tipo Collection-index (UC1.2.1.1);|
L'utente sviluppatore può impostare la visualizzazione della pagina Collection-Index (UC1.2.1.2);|
L'utente sviluppatore può impostare la visualizzazione della pagina Document-Show (UC1.2.1.3).
}
\post{Il sistema ha creato una nuova descrizione per la visualizzazione della Collection}

\UCtitle
{Caso d'uso UC1.2.1.1}
{Creazione visualizzazione menù}
\UCimmagine{UC1.2.1.1}{UC1.2.1.1 Creazione visualizzazione menù}
\UC
{UC1.2.1.1}
{Utente sviluppatore}
{L'utente sviluppatore può definire il nome della voce relativa alla Collection che verrà visualizzata e la posizione di questa nel menù delle Collection}
{Il sistema contiene la descrizione della visualizzazione del menù predefinita. Nella definizione è specificato il nome della voce del menù corrispondente al nome della Collection e la posizione, la prima disponibile in ordine crescente}
\scenario{
L'utente sviluppatore può definire il nome della voce relativa alla Collection all'interno del menù (UC1.2.1.1.1);|
L'utente sviluppatore può definire la posizione della Collection all'interno del menù (UC1.2.1.1.2).
}
\post{Il sistema ha creato una nuova descrizione per la visualizzazione del menù delle Collection}

\UCtitle
{Caso d'uso UC1.2.1.1.1}
{Definizione nome voce}
\UC
{UC1.2.1.1}
{Utente sviluppatore}
{L'utente sviluppatore può definire il nome della voce relativa alla Collection nel menù delle Collection}
{Il sistema contiene la descrizione per la visualizzazione del menù predefinita, nella quale il nome della voce del menù corrisponde al nome della Collection. L'utente desidera definire un nuovo nome per la voce}
\scenario
{L'utente sviluppatore può definire il nome della Collection da visualizzare nel menù delle Collection.}
\post{Il sistema contiene la descrizione per la visualizzazione del menù, nella quale il nome della voce del menù corrisponde al nome della Collection definita dall'utente sviluppatore}

\UCtitle
{Caso d'uso UC1.2.1.1.2}
{Definizione posizione voce}
\UC
{UC1.2.1.1}
{Utente sviluppatore}
{L'utente può definire la posizione della voce relativa al nome della Collection all'interno del menù delle Collection}
{Il sistema contiene la descrizione per la visualizzazione del menù predefinita, nella quale la posizione \'e la prima disponibile in ordine crescente. L'utente sviluppatore desidera specificare la posizione della voce}
\scenario
{L'utente sviluppatore può definire la posizione della Collection nel menù delle Collection.}
\post{Il sistema contiene la descrizione per la visualizzazione del menù, nella quale la posizione della voce nel menù è definita dall'utente sviluppatore}



\UCtitle
{Caso d'uso UC1.2.1.2}
{Creazione visualizzazione pagina Collection-Index}
\UCimmagine{UC1.2.1.2}{UC1.2.1.2 Creazione visualizzazione pagina Collection-Index}
\UC
{UC1.2.1.2}
{Utente sviluppatore}
{L'utente sviluppatore  può creare una descrizione per la visualizzazione della pagina Collection-Index e aggiungere varie specifiche al suo interno}
{Il sistema contiene la descrizione per la visualizzazione della pagina Collection-Index predefinita}
\scenario{
L'utente sviluppatore può aggiungere una chiave da visualizzare (UC1.2.1.2.1);|
L'utente sviluppatore può definire un ordinamento per chiave (UC1.2.1.2.2);|
L'utente sviluppatore può aggiungere un numero massimo di \gloss{Document} da visualizzare (UC1.2.1.2.3);|
L'utente sviluppatore può aggiungere un pulsante (UC1.2.1.2.4).
}
\post{Il sistema ha creato la descrizione per la visualizzazione della pagina Collection-Index con le specifiche aggiunte}


\UCtitle
{Caso d'uso UC1.2.1.2.1}
{Aggiunta chiave}
\UCimmagine{UC1.2.1.2.1}{UC1.2.1.2.1 Aggiunta chiave}
\UC
{UC1.2.1.2.1}
{Utente sviluppatore}
{L'utente sviluppatore può aggiungere una chiave alla definizione della visualizzazione della pagina Collection-Index}
{Il sistema contiene la descrizione per la visualizzazione della pagina Collection-Index}
\scenario{
L'utente sviluppatore può definire un'etichetta per la chiave da visualizzare (UC1.2.1.2.1.1);|
L'utente sviluppatore può definire il campo del Document ed associarlo alla chiave da visualizzare (UC1.2.1.2.1.2);|
L'utente sviluppatore può definire il campo di un Document esterno e associarlo alla chiave da visualizzare (UC1.2.1.2.1.3);|
L'utente sviluppatore può definire il campo associato alla chiave da visualizzare proveniente dal risultato di una query (UC1.2.1.2.1.4);|
L'utente sviluppatore può definire il campo associato alla chiave da visualizzare come trasformazione tramite funzione Javascript (UC1.2.1.2.1.5).
}
\post{Il sistema contiene la descrizione per la visualizzazione della pagina Collection-Index aggiornata con la chiave aggiunta dall'utente sviluppatore}

\UCtitle
{Caso d'uso UC1.2.1.2.1.1}
{Definizione etichetta chiave}
\UC
{UC1.2.1.2.1}
{Utente sviluppatore}
{L'utente sviluppatore può definire un'etichetta per la chiave da visualizzare}
{Il sistema contiene la descrizione per la visualizzazione della pagina Collection-Index predefinita, nella quale non è specificata nessuna etichetta per la chiave da visualizzare}
\scenario
{L'utente sviluppatore può definire un'etichetta per la chiave da visualizzare.}
\post{Il sistema contiene la descrizione per la visualizzazione della pagina Collection-Index aggiornata con l'etichetta relativa alla chiave da visualizzare}

\UCtitle
{Caso d'uso UC1.2.1.2.1.2}
{Definizione campo associato alla chiave}
\UC
{UC1.2.1.2.1}
{Utente sviluppatore}
{L'utente sviluppatore può definire un campo associato alla chiave da visualizzare}
{Il sistema contiene la descrizione per la visualizzazione della pagina Collection-Index predefinita, nella quale non è specificato nessun campo per la chiave da visualizzare}
\scenario
{L'utente sviluppatore può definire un campo associato alla chiave da visualizzare}
\post{Il sistema contiene la descrizione per la visualizzazione della pagina Collection-Index aggiornata con il campo associato alla chiave da visualizzare}

\UCtitle
{Caso d'uso UC1.2.1.2.1.3}
{Definizione campo associato alla chiave proveniente da un documento esterno}
\UC
{UC1.2.1.2.1}
{Utente sviluppatore}
{L'utente sviluppatore può definire un campo associato alla chiave da visualizzare, proveniente da un documento esterno}
{Il sistema contiene la descrizione per la visualizzazione della pagina Collection-Index predefinita, nella quale non è specificato nessun campo per la chiave da visualizzare}
\scenario
{L'utente sviluppatore può definire un campo associato alla chiave da visualizzare, proveniente da un documento esterno.}
\post{Il sistema contiene la descrizione per la visualizzazione della pagina Collection-Index aggiornata con il campo associato alla chiave da visualizzare}

\UCtitle
{Caso d'uso UC1.2.1.2.1.4}
{Definizione campo associato alla chiave proveniente dal risultato di una query}
\UC
{UC1.2.1.2.1}
{Utente sviluppatore}
{L'utente sviluppatore può definire un campo associato alla chiave da visualizzare, proveniente dal risultato di una query}
{Il sistema contiene la descrizione per la visualizzazione della pagina Collection-Index predefinita, nella quale non è specificato nessun campo per la chiave da visualizzare, la query deve essere definita}
\scenario
{L'utente sviluppatore può definire un campo associato alla chiave da visualizzare, proveniente dal risultato di una query.}
\post{Il sistema contiene la descrizione per la visualizzazione della pagina Collection-Index aggiornata con il campo associato alla chiave da visualizzare}

\UCtitle
{Caso d'uso UC1.2.1.2.1.5}
{Definizione campo associato alla chiave come trasformazione}
\UC
{UC1.2.1.2.1}
{Utente sviluppatore}
{L'utente sviluppatore può definire un campo associato alla chiave da visualizzare come trasformazione Javascript}
{Il sistema contiene la descrizione per la visualizzazione della pagina Collection-Index predefinita, nella quale non è specificato nessun campo per la chiave da visualizzare}
\scenario
{L'utente sviluppatore può definire un campo associato alla chiave da visualizzare come trasformazione.}
\post{Il sistema contiene la descrizione per la visualizzazione della pagina Collection-Index aggiornata con il campo associato alla chiave da visualizzare}

\UCtitle
{Caso d'uso UC1.2.1.2.2}
{Definizione ordinamento per chiave}
\UC
{UC1.2.1.2}
{Utente sviluppatore}
{L'utente sviluppatore  può  definire un ordinamento, alfabetico crescente o decrescente, rispetto ad una chiave nella definizione della visualizzazione della pagina Collection-Index}
{Il sistema contiene la descrizione per la visualizzazione della pagina Collection-Index predefinita, nella quale non è presente alcun ordinamento per chiave definito dall'utente}
\scenario
{L'utente sviluppatore può definire un ordinamento rispetto ad una chiave della Collection-Index.}
\post{Il sistema contiene la descrizione per la visualizzazione della pagina Collection-Index aggiornata con il nuovo ordinamento per chiave definito dall'utente sviluppatore}

\UCtitle
{Caso d'uso UC1.2.1.2.3}
{Definizione numero di Document per pagina}
\UC
{UC1.2.1.2}
{Utente sviluppatore}
{L'utente sviluppatore  può definire un numero massimo di Document da visualizzare nella definizione della visualizzazione della pagina Collection-Index}
{Il sistema contiene la descrizione per la visualizzazione della pagina Collection-Index predefinita, nella quale non è presente alcun numero massimo di Document definito dall'utente}
\scenario
{L'utente sviluppatore può definire un numero massimo di Document da visualizzare in una Collection-Index.}
\post{Il sistema contiene la descrizione per la visualizzazione della pagina Collection-Index aggiornata con il nuovo numero massimo di Document per pagina definito dall'utente sviluppatore}

\UCtitle
{Caso d'uso UC1.2.1.2.4}
{Aggiunta pulsante}
\UC
{UC1.2.1.2}
{Utente sviluppatore}
{L'utente sviluppatore può aggiungere un pulsante per modificare la visualizzazione della pagina Collection-Index, specificando l'etichetta e l'azione che deve eseguire}
{Il sistema contiene la descrizione per la visualizzazione della pagina Collection-Index}
\scenario
{L'utente sviluppatore può aggiungere un pulsante nella Collection-Index.}
\post{Il sistema contiene la descrizione per la visualizzazione della pagina Collection-Index aggiornata con il nuovo  pulsante aggiunto dall'utente sviluppatore}


\UCtitle
{Caso d'uso UC1.2.1.3}
{Creazione visualizzazione pagina Document-Show}
\UCimmagine{UC1.2.1.3}{UC1.2.1.3 Creazione visualizzazione pagina Document-Show}
\UC
{UC1.2.1.3}
{Utente sviluppatore}
{L'utente sviluppatore  può creare la descrizione per la visualizzazione della pagina Document-Show e aggiungere varie specifiche al suo interno}
{Il sistema contiene la descrizione per la visualizzazione della pagina Document-Show predefinita}
\scenario{
L'utente può aggiungere una chiave da visualizzare (UC1.2.1.3.1);|
L'utente può aggiungere un pulsante da visualizzare (UC1.2.1.3.2).
}
\post{Il sistema ha creato la nuova descrizione per la visualizzazione della pagina Document-Show con le specifiche aggiunte}

\UCtitle
{Caso d'uso UC1.2.1.3.1}
{Aggiunta chiave}
\UCimmagine{UC1.2.1.3.1}{UC1.2.1.3.1 Aggiunta chiave}
\UC
{UC1.2.1.3.1}
{Utente sviluppatore}
{L'utente sviluppatore può aggiungere una chiave alla definizione della visualizzazione della pagina Document-Show}
{Il sistema contiene la descrizione per la visualizzazione della pagina Document-Show}
\scenario{
L'utente sviluppatore può definire un'etichetta per la chiave da visualizzare (UC1.2.1.3.1.1);|
L'utente sviluppatore può definire il campo del Document ed associarlo alla chiave da visualizzare (UC1.2.1.3.1.2);|
L'utente sviluppatore può definire il campo di un Document esterno e associarlo alla chiave da visualizzare (UC1.2.1.3.1.3);|
L'utente sviluppatore può definire il campo associato alla chiave da visualizzare come trasformazione tramite funzione Javascript (UC1.2.1.3.1.4);|
L'utente sviluppatore può definire il campo associato alla chiave da visualizzare proveniente dal risultato di una query (UC1.2.1.3.1.5).
}
\post{Il sistema contiene la descrizione per la visualizzazione della pagina Document-Show aggiornata con la chiave aggiunta dall'utente sviluppatore}

\UCtitle
{Caso d'uso UC1.2.1.3.1.1}
{Definizione etichetta chiave}
\UC
{UC1.2.1.3.1}
{Utente sviluppatore}
{L'utente sviluppatore può definire un'etichetta per la chiave da visualizzare}
{Il sistema contiene la descrizione per la visualizzazione della pagina Document-Show predefinita, nella quale non è specificata nessuna etichetta per la chiave da visualizzare}
\scenario
{L'utente sviluppatore può definire un'etichetta per la chiave da visualizzare.}
\post{Il sistema contiene la descrizione per la visualizzazione della pagina Document-Show aggiornata con l'etichetta relativa alla chiave da visualizzare}

\UCtitle
{Caso d'uso UC1.2.1.3.1.2}
{Definizione campo associato alla chiave}
\UC
{UC1.2.1.3.1}
{Utente sviluppatore}
{L'utente sviluppatore può definire un campo associato alla chiave da visualizzare}
{Il sistema contiene la descrizione per la visualizzazione della pagina Document-Show predefinita, nella quale non è specificato nessun campo per la chiave da visualizzare}
\scenario
{L'utente sviluppatore può definire un campo associato alla chiave da visualizzare}
\post{Il sistema contiene la descrizione per la visualizzazione della pagina Document-Show aggiornata con il campo associato alla chiave da visualizzare}

\UCtitle
{Caso d'uso UC1.2.1.3.1.3}
{Definizione campo associato alla chiave proveniente da un documento esterno}
\UC
{UC1.2.1.3.1}
{Utente sviluppatore}
{L'utente sviluppatore può definire un campo associato alla chiave da visualizzare, proveniente da un documento esterno}
{Il sistema contiene la descrizione per la visualizzazione della pagina Document-Show predefinita, nella quale non è specificato nessun campo per la chiave da visualizzare}
\scenario
{L'utente sviluppatore può definire un campo associato alla chiave da visualizzare, proveniente da un documento esterno.}
\post{Il sistema contiene la descrizione per la visualizzazione della pagina Document-Show aggiornata con il campo associato alla chiave da visualizzare}

\UCtitle
{Caso d'uso UC1.2.1.3.1.4}
{Definizione campo associato alla chiave proveniente dal risultato di una query}
\UC
{UC1.2.1.3.1}
{Utente sviluppatore}
{L'utente sviluppatore può definire un campo associato alla chiave da visualizzare, proveniente dal risultato di una query}
{Il sistema contiene la descrizione per la visualizzazione della pagina Document-Show predefinita, nella quale non è specificato nessun campo per la chiave da visualizzare, la query deve essere definita}
\scenario
{L'utente sviluppatore può definire un campo associato alla chiave da visualizzare, proveniente dal risultato di una query.}
\post{Il sistema contiene la descrizione per la visualizzazione della pagina Document-Show aggiornata con il campo associato alla chiave da visualizzare}

\UCtitle
{Caso d'uso UC1.2.1.3.1.5}
{Definizione campo associato alla chiave come trasformazione}
\UC
{UC1.2.1.3.1}
{Utente sviluppatore}
{L'utente sviluppatore può definire un campo associato alla chiave da visualizzare come trasformazione}
{Il sistema contiene la descrizione per la visualizzazione della pagina Document-Show predefinita, nella quale non è specificato nessun campo per la chiave da visualizzare}
\scenario
{L'utente sviluppatore può definire un campo associato alla chiave da visualizzare come trasformazione.}
\post{Il sistema contiene la descrizione per la visualizzazione della pagina Document-Show aggiornata con il campo associato alla chiave da visualizzare}

\UCtitle
{Caso d'uso UC1.2.1.3.2}
{Aggiunta pulsante}
\UC
{UC1.2.1.3}
{Utente sviluppatore}
{L'utente sviluppatore può aggiungere un pulsante per la visualizzazione della pagina Document-Show, specificando l'etichetta e l'azione che deve eseguire}
{Il sistema contiene la descrizione per la visualizzazione della pagina Document-Show}
\scenario
{L'utente sviluppatore può aggiungere un pulsante in una Document-Show.}
\post{Il sistema contiene la descrizione per la visualizzazione della pagina Document-Show aggiornata con il nuovo pulsante aggiunto dall'utente sviluppatore}

\UCtitle
{Caso d'uso UC1.2.2}
{Modifica visualizzazione Collection}
\UCimmagine{UC1.2.2}{UC1.2.2 Modifica visualizzazione Collection}
\UC
{UC1.2.2}
{Utente sviluppatore}
{L'utente sviluppatore  può impostare la visualizzazione del menù delle Collection, delle pagine Collection-Index e Document-Show}
{Il sistema contiene la descrizione per la visualizzazione della Collection}
\scenario{
L'utente sviluppatore può impostare la visualizzazione del menù per le pagine di tipo Collection-Index (UC1.2.2.1);|
L'utente sviluppatore può impostare la visualizzazione della pagina Collection-Index (UC1.2.2.2);|
L'utente sviluppatore può impostare la visualizzazione della pagina Document-Show (UC1.2.2.3).
}
\post{Il sistema ha aggiornato la descrizione per la visualizzazione della Collection}

\UCtitle
{Caso d'uso UC1.2.2.1}
{Impostazioni visualizzazione menù}
\UCimmagine{UC1.2.2.1}{UC1.2.2.1 Impostazioni visualizzazione menù}
\UC
{UC1.2.2.1}
{Utente sviluppatore}
{L'utente sviluppatore può modificare, nella descrizione per la visualizzazione del menù della Collection, il nome della Collection e la posizione nel menù principale}
{Il sistema contiene la descrizione per la visualizzazione del menù e permette all'utente sviluppatore di modificarla}
\scenario{
L'utente può modificare il nome della Collection (UC1.2.2.1.1);|
L'utente può modificare la posizione della Collection (UC1.2.2.1.2).
}
\post{Il sistema ha aggiornato la descrizione per la visualizzazione del menù con le modifiche apportate dall'utente sviluppatore}

\UCtitle
{Caso d'uso UC1.2.2.1.1}
{Modifica nome}
\UC
{UC1.2.2.1}
{Utente sviluppatore}
{L'utente può modificare, nella descrizione per la visualizzazione del menù della Collection, il nome della voce relativa alla Collection}
{Il sistema contiene la descrizione per la visualizzazione della pagina Collection-Index}
\scenario
{L'utente sviluppatore può modificare il nome della Collection nel menù delle Collection.}
\post{Il sistema ha aggiornato la descrizione per la visualizzazione del menù delle Collection con il nuovo nome specificato dall'utente sviluppatore}

\UCtitle
{Caso d'uso UC1.2.2.1.2}
{Modifica posizione}
\UC
{UC1.2.2.1}
{Utente sviluppatore}
{L'utente può modificare, nella descrizione per la visualizzazione del menù della Collection, la posizione della voce relativa al nome della Collection}
{L'utente desidera cambiare la posizione della voce nel menù}
\scenario
{L'utente sviluppatore può modificare la posizione della Collection nel menù delle Collection.}
\post{Il sistema ha aggiornato la descrizione per la visualizzazione del menù delle Collection con la nuova posizione della voce specificata dall'utente sviluppatore}

\UCtitle
{Caso d'uso UC1.2.2.2}
{Impostazioni visualizzazione pagina Collection-Index}
\UCimmagine{UC1.2.2.2}{UC1.2.2.2 Impostazioni visualizzazione pagina Collection-Index}
\UC
{UC1.2.2.2}
{Utente sviluppatore}
{L'utente sviluppatore  può modificare varie impostazioni della descrizione per la visualizzazione della pagina Collection-Index}
{Il sistema contiene la descrizione per la visualizzazione della pagina Collection-Index e permette all'utente sviluppatore di modificarla}
\scenario{
L'utente può aggiungere una chiave da visualizzare (UC1.2.1.2.1);|
L'utente può eliminare una chiave da visualizzare (UC1.2.2.2.1);|
L'utente può definire un ordinamento per chiave (UC1.2.1.2.2);|
L'utente può eliminare un ordinamento per chiave (UC1.2.2.2.2);|
L'utente può aggiungere il numero massimo di Document da visualizzare per pagina (UC1.2.1.2.3);|
L'utente può modificare il numero massimo di Document da visualizzare per pagina (UC1.2.2.2.3);|
L'utente può eliminare il numero massimo di Document da visualizzare per pagina (UC1.2.2.2.4);|
L'utente può aggiungere un pulsante (UC1.2.1.2.4);|
L'utente può eliminare un pulsante (UC1.2.2.2.5).
}
\post{Il sistema contiene la descrizione aggiornata per la visualizzazione della Collection}

\UCtitle
{Caso d'uso UC1.2.2.2.1}
{Elimina chiave}
\UC
{UC1.2.2.2}
{Utente sviluppatore}
{L'utente sviluppatore  può eliminare una chiave dalla descrizione per la visualizzazione della pagina Collection-Index}
{Il sistema contiene la descrizione per la visualizzazione della pagina Collection-Index, la quale contiene almeno una chiave da visualizzare}
\scenario
{L'utente sviluppatore può eliminare una chiave.}
\post{Il sistema contiene la descrizione per la visualizzazione della pagina Collection-Index aggiornata con la chiave eliminata dall'utente sviluppatore}


\UCtitle
{Caso d'uso UC1.2.2.2.2}
{Elimina ordinamento}
\UC
{UC1.2.2.2}
{Utente sviluppatore}
{L'utente sviluppatore  può eliminare un ordinamento rispetto a una chiave dalla descrizione per la visualizzazione della pagina Collection-Index}
{Il sistema contiene la descrizione per la visualizzazione della pagina Collection-Index, la quale contiene un ordinamento rispetto una chiave definito dall'utente sviluppatore}
\scenario
{L'utente sviluppatore può eliminare un ordinamento rispetto ad una chiave.}
\post{Il sistema contiene la descrizione per la visualizzazione della pagina Collection-Index aggiornata con l'ordinamento eliminato dall'utente sviluppatore}

\UCtitle
{Caso d'uso UC1.2.2.2.3}
{Modifica numero massimo di Document}
\UC
{UC1.2.2.2}
{Utente sviluppatore}
{L'utente sviluppatore  può modificare il numero massimo di Document da visualizzare per la pagina Collection-Index dalla descrizione per la visualizzazione della pagina Collection-Index}
{Il sistema contiene la descrizione per la visualizzazione della pagina Collection-Index, nella quale è presente un numero massimo di Document definito dall'utente}
\scenario
{L'utente sviluppatore può modificare il numero massimo di Document da visualizzare in una Collection-Index.}
\post{Il sistema contiene la descrizione per la visualizzazione della pagina Collection-Index aggiornata con il numero massimo di Document per pagina modificato dall'utente sviluppatore}

\UCtitle
{Caso d'uso UC1.2.2.2.4}
{Elimina numero massimo di Document}
\UC
{UC1.2.2.2}
{Utente sviluppatore}
{L'utente sviluppatore  può eliminare il numero massimo di Document da visualizzare dalla descrizione per la visualizzazione della pagina Collection-Index}
{Il sistema contiene la descrizione per la visualizzazione della pagina Collection-Index, nella quale è presente un numero massimo di Document definito dall'utente}
\scenario
{L'utente sviluppatore può eliminare il numero massimo di Document da visualizzare in una Collection-Index.}
\post{Il sistema contiene la descrizione per la visualizzazione della pagina Collection-Index aggiornata con il numero massimo di Document per pagina eliminato dall'utente sviluppatore}

\UCtitle
{Caso d'uso UC1.2.2.2.5}
{Elimina pulsante}
\UC
{UC1.2.2.2}
{Utente sviluppatore}
{L'utente sviluppatore  può eliminare un pulsante da visualizzare nella pagina Collection-Index}
{Il sistema contiene la descrizione per la visualizzazione della pagina Collection-Index, nella quale è presente un pulsante definito dall'utente}
\scenario
{L'utente sviluppatore può eliminare un pulsante.}
\post{Il sistema contiene la descrizione per la visualizzazione della pagina Collection-Index aggiornata con il pulsante eliminato dall'utente sviluppatore}


\UCtitle
{Caso d'uso UC1.2.2.3}
{Impostazioni visualizzazione pagina Document-Show}
\UCimmagine{UC1.2.2.3}{UC1.2.2.3 Impostazioni visualizzazione pagina Document-Show}
\UC
{UC1.2.2.3}
{Utente sviluppatore}
{L'utente sviluppatore  può modificare la descrizione per la visualizzazione della pagina Document-Show e varie specifiche al suo interno}
{Il sistema contiene la descrizione per la visualizzazione della pagina Document-Show}
\scenario{
L'utente può aggiungere una chiave da visualizzare (UC1.2.1.3.1);|
L'utente può eliminare una chiave da visualizzare (UC1.2.2.3.1);|
L'utente può aggiungere un pulsante (UC1.2.1.3.2);|
L'utente può eliminare un pulsante (UC1.2.2.3.2).
}
\post{Il sistema contiene la descrizione aggiornata per la visualizzazione della Collection}


\UCtitle
{Caso d'uso UC1.2.2.3.1}
{Elimina chiave}
\UC
{UC1.2.2.3}
{Utente sviluppatore}
{L'utente sviluppatore  può eliminare una chiave  dalla descrizione per la visualizzazione della pagina Document-Show}
{Il sistema contiene la descrizione per la visualizzazione della pagina Document-Show, la quale contiene almeno una chiave da visualizzare}
\scenario
{L'utente sviluppatore può eliminare una chiave da una Document-Show.}
\post{Il sistema contiene la descrizione per la visualizzazione della pagina Document-Show aggiornata con la chiave eliminata dall'utente sviluppatore}

\UCtitle
{Caso d'uso UC1.2.2.3.2}
{Elimina pulsante}
\UC
{UC1.2.2.3}
{Utente sviluppatore}
{L'utente sviluppatore può eliminare un pulsante dalla descrizione per la visualizzazione della pagina Document-Show}
{Il sistema contiene la descrizione per la visualizzazione della pagina Document-Show, la quale contiene almeno un pulsante}
\scenario
{L'utente sviluppatore può eliminare un pulsante da una Document-Show.}
\post{Il sistema contiene la descrizione per la visualizzazione della pagina Document-Show aggiornata il pulsante eliminato dall'utente sviluppatore}

\UCtitle
{Caso d'uso UC1.2.3}
{Aggiunta query}
\UC
{UC1.2}
{Utente sviluppatore}
{L'utente sviluppatore può aggiungere una query}
{Il sistema contiene la descrizione per la visualizzazione della pagina Collection-Index e Document-Show}
\scenario
{L'utente sviluppatore può aggiungere una query personalizzata.}
\post{Il sistema contiene la descrizione per la visualizzazione della pagina Collection-Index e Document-Show aggiornata con la definizione della query personalizzata}

\UCtitle
{Caso d'uso UC1.2.4}
{Eliminazione query}
\UC
{UC1.2}
{Utente sviluppatore}
{L'utente sviluppatore può eliminare una query precedentemente definita}
{Il sistema contiene la descrizione per la visualizzazione della pagina Collection-Index e Document-Show ed è presente almeno la definizione di una query}
\scenario
{L'utente sviluppatore può eliminare una query precedentemente definita.}
\post{Il sistema contiene la descrizione per la visualizzazione della pagina Collection-Index e Document-Show aggiornata}

\UCtitle
{Caso d'uso UC1.3}
{Modifica file di configurazione}
\UCimmagine{UC1.3}{UC1.3 Modifica file di configurazione}
\UC
{UC1.3}
{Utente sviluppatore}
{L'utente sviluppatore  può modificare varie impostazioni dei file di configurazione}
{Il sistema ha creato lo scheletro del progetto e permette all'utente sviluppatore di modificare i file di configurazione}
\scenario{
L'utente sviluppatore può abilitare la \gloss{registrazione} alle pagine web generate da MaaP (UC1.3.1);|
L'utente sviluppatore può abilitare la creazione di Document (UC1.3.2);|
L'utente sviluppatore può modificare i template delle pagine web (UC1.3.3);|
L'utente sviluppatore può specificare i dati per la connessione al database di analisi (UC1.3.4);|
L'utente sviluppatore può abilitare la creazione di indici (UC1.3.5).
}
\post{Il sistema ha aggiornato il file di configurazione}

\UCtitle
{Caso d'uso UC1.3.1}
{Abilitazione registrazione}
\UC
{UC1.3}
{Utente sviluppatore}
{L'utente sviluppatore può abilitare o disabilitare la registrazione alle pagine web generate da MaaP}
{Il sistema contiene il file di configurazione con l'abilitazione della registrazione predefinita}
\scenario
{L'utente sviluppatore può abilitare o no la registrazione.}
\post{Il sistema ha aggiornato il file di configurazione con la scelta dell'utente sviluppatore}

\UCtitle
{Caso d'uso UC1.3.2}
{Abilitazione creazione Document}
\UC
{UC1.3}
{Utente sviluppatore}
{L'utente sviluppatore può abilitare o disabilitare la creazione di Document nel database di analisi}
{Il sistema contiene il file di configurazione con l'abilitazione della creazione di Document nel database di analisi}
\scenario
{L'utente sviluppatore può abilitare o no la creazione di Document.}
\post{Il sistema ha aggiornato il file di configurazione con la scelta dell'utente sviluppatore}

\UCtitle
{Caso d'uso UC1.3.3}
{Modifica template pagine web}
\UC
{UC1.3}
{Utente sviluppatore}
{L'utente sviluppatore pu\'o modificare i template delle pagine web}
{Il sistema contiene le directory di descrizione delle pagine web}
\scenario
{L'utente sviluppatore pu\'o modificare i template delle pagine web.}
\post{Il sistema ha aggiornato i template delle pagine web con le modifiche apportate dall'utente sviluppatore}

\UCtitle
{Caso d'uso UC1.3.4}
{Specifica database di analisi}
\UC
{UC1.3}
{Utente sviluppatore}
{L'utente sviluppatore può specificare i dati necessari alla connessione del database di analisi}
{Il sistema contiene il file di configurazione per la connessione a un database di analisi}
\scenario
{L'utente sviluppatore può specificare il database di analisi.}
\post{Il sistema contiene le informazioni per connettersi a un database di analisi}

\UCtitle
{Caso d'uso UC1.3.5}
{Abilitazione creazione indici}
\UC
{UC1.3}
{Utente sviluppatore}
{L'utente sviluppatore può abilitare o disabilitare la creazione di indici nelle pagine web generate da MaaP}
{Il sistema contiene il file di configurazione con l'abilitazione per la creazione di indici}
\scenario
{L'utente sviluppatore può abilitare o no la creazione di indici.}
\post{Il sistema ha aggiornato il file di configurazione con la scelta dell'utente sviluppatore}

\UCtitle
{Caso d'uso UC1.4}
{Utilizzo editor specializzato}
\UCimmagine{UC1.4}{UC1.4 Utilizzo editor specializzato}
\UC
{UC1.4}
{Utente sviluppatore}
{L'utente sviluppatore può utilizzare un editor specializzato per scrivere il file di descrizione}
{Il sistema ha creato lo scheletro del progetto e contiene un editor specializzato che può essere utilizzato dall'utente sviluppatore}
\scenario
{L'utente sviluppatore può creare un file di descrizione (UC1.4.1);|
L'utente sviluppatore può modificare un file di descrizione esistente (UC1.4.2).
}
\post{Il sistema consente all'utente sviluppatore di scrivere con l'editor specializzato}

\UCtitle
{Caso d'uso UC1.4.1}
{Creazione file di descrizione}
\UCimmagine{UC1.4.1}{UC1.4.1 Creazione file di descrizione}
\UC
{UC1.4.1}
{Utente sviluppatore}
{L'utente sviluppatore può utilizzare un editor specializzato per creare un file di descrizione}
{Il sistema ha creato lo scheletro del progetto}
\scenario
{L'utente sviluppatore può scrivere un file di descrizione (UC1.4.1.1);|
L'utente sviluppatore può salvare il file di descrizione (UC1.4.1.2).
}
\post{Il sistema ha ottenuto il contenuto relativo al file di descrizione che l'utente sviluppatore ha scritto ed ha salvato il file di descrizione in modo permanente}

\UCtitle
{Caso d'uso UC1.4.1.1}
{Scrittura file di descrizione}
\UC
{UC1.4.1}
{Utente sviluppatore}
{L'utente sviluppatore può utilizzare un editor specializzato per scrivere un file di descrizione}
{Il sistema propone all'utente una schermata per scrivere il file di descrizione}
\scenario
{L'utente sviluppatore può utilizzare un editor per la scrittura di un file di descrizione.}
\post
{Il sistema conosce il contenuto del file scritto dall'utente sviluppatore}

\UCtitle
{Caso d'uso UC1.4.1.2}
{Salvataggio file di descrizione}
\UC
{UC1.4.1}
{Utente sviluppatore}
{L'utente sviluppatore può salvare in modo permanente il file di descrizione scritto precedentemente}
{Il sistema conosce il contenuto del file scritto dall'utente sviluppatore}
\scenario
{L'utente sviluppatore può salvare un file di configurazione scritto.}
\post
{Il sistema ha salvato il file di descrizione in modo permanente}

\UCtitle
{Caso d'uso UC1.4.2}
{Modifica file di descrizione}
\UCimmagine{UC1.4.2}{UC1.4.2 Modifica file di descrizione}
\UC
{UC1.4.2}
{Utente sviluppatore}
{L'utente sviluppatore può modificare un file di descrizione presente nel sistema}
{Il sistema ha creato lo scheletro del progetto e l'utente sviluppatore ha selezionato un file di descrizione da modificare}
\scenario
{L'utente sviluppatore può modificare il codice del file di descrizione (UC1.4.2.1);|
L'utente sviluppatore può salvare le modifiche apportate al file di descrizione (UC1.4.2.2).
}
\estensioni
{L'utente sviluppatore può in qualsiasi momento annullare le modifiche apportate al codice del file di descrizione selezionato. (UC1.4.2.3)
}
\post{Il sistema ha ottenuto le informazione relative alle azioni che l'utente sviluppatore desidera eseguire. Se l'utente sviluppatore ha scelto di salvare le modifiche apportate, il sistema salva tali informazioni in modo permanente sovrascrivendo il file di descrizione originario}

\UCtitle
{Caso d'uso UC1.4.2.1}
{Modifica del codice del file di descrizione}
\UC
{UC1.4.2}
{Utente sviluppatore}
{L'utente sviluppatore può modificare il codice del file di descrizione}
{Il sistema propone all'utente una schermata per modificare il codice del file di descrizione}
\scenario
{L'utente sviluppatore può modificare il codice del file di descrizione.}
\post
{Il sistema possiede il contenuto del testo scritto dall'utente sviluppatore}

\UCtitle
{Caso d'uso UC1.4.2.2}
{Salvataggio delle modifiche del file di descrizione}
\UC
{UC1.4.2}
{Utente sviluppatore}
{L'utente sviluppatore può modificare in modo permanente le modifiche apportate al file di descrizione}
{Il sistema conosce il contenuto del testo scritto dall'utente sviluppatore}
\scenario
{L'utente sviluppatore può salvare le modifiche apportate ad un file di descrizione.}
\post
{Il sistema ha salvato in modo permanente le modifiche apportate al file di descrizione, sovrascrivendo il file di descrizione originario}

\UCtitle
{Caso d'uso UC1.4.2.3}
{Annullamento delle modifiche}
\UC
{UC1.4.2}
{Utente sviluppatore}
{L'utente sviluppatore può annullare le modifiche apportate al contenuto del file di descrizione}
{Il sistema conosce il contenuto del testo scritto dall'utente sviluppatore}
\scenario
{L'utente sviluppatore può annullare le modifiche apportate.}
\post
{Il sistema ignora le modifiche apportate al contenuto del file di descrizione, e non sovrascrive il file originario}

\UCtitle
{Caso d'uso UC1.5}
{Inserimento file di descrizione}
\UC
{UC1}
{Utente sviluppatore}
{L'utente sviluppatore può inserire il file di descrizione nel progetto}
{Il sistema ha creato lo scheletro di un nuovo progetto}
\scenario
{L'utente sviluppatore può inserire il file di descrizione nel progetto.}
\post{Il sistema contiene il file di descrizione}

\UCtitle
{Caso d'uso UC1.6}
{Eliminazione progetto}
\UC
{UC1}
{Utente sviluppatore}
{L'utente sviluppatore può eliminare un progetto}
{Il sistema contiene almeno un progetto precedentemente creato}
\scenario
{L'utente sviluppatore può eliminare un progetto esistente.}
\post{Il sistema ha eliminato il progetto specificato dall'utente sviluppatore}

\UCtitle
{Caso d'uso UC1.7}
{Clonazione progetto}
\UC
{UC1}
{Utente sviluppatore}
{L'utente sviluppatore può clonare un progetto}
{Il sistema contiene almeno un progetto precedentemente creato}
\scenario
{L'utente sviluppatore può clonare un progetto esistente.}
\post{Il sistema ha creato un nuovo progetto clonandolo da un progetto esistente specificato dall'utente sviluppatore}


